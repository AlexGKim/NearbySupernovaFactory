\documentclass[trackchanges]{aastex62}   	% use "amsart" instead of "article" for AMSLaTeX format
\usepackage{geometry}                		% See geometry.pdf to learn the layout options. There are lots.
\geometry{letterpaper}                   		% ... or a4paper or a5paper or ... 
\usepackage{graphicx}				% Use pdf, png, jpg, or eps§ with pdflatex; use eps in DVI mode
\usepackage{amsmath}
\usepackage{amssymb}
\usepackage{natbib}
\usepackage{lineno}
\usepackage{color}
\defcitealias{1999PASP..111...63F}{F99}
\defcitealias{2017ApJ...842...93M}{M17}
\linenumbers

\begin{document}

\title{Evidence for a Third Color  Parameter Within the Type~Ia Supernovae of the Nearby Supernova Factory}
\author[0000-0001-6315-8743]{A.~G.~Kim}
\affiliation{    Physics Division, Lawrence Berkeley National Laboratory, 
    1 Cyclotron Road, Berkeley, CA, 94720}
    
\author{     G.~Aldering}
\affiliation{    Physics Division, Lawrence Berkeley National Laboratory, 
    1 Cyclotron Road, Berkeley, CA, 94720}

\author{     P.~Antilogus}
\affiliation{    Sorbonne Universit\'e, Universit\'e Paris Diderot, CNRS/IN2P3, Laboratoire de Physique Nucl\'eaire et de Hautes Energies, LPNHE, 4 Place Jussieu, F-75252, Paris, France}
    
\author{     S.~Bailey}
\affiliation{    Physics Division, Lawrence Berkeley National Laboratory, 
    1 Cyclotron Road, Berkeley, CA, 94720}

\author{     C.~Baltay}
\affiliation{    Department of Physics, Yale University, 
    New Haven, CT, 06250-8121}

\author{     K.~Barbary}
\affiliation{
    Department of Physics, University of California Berkeley,
    366 LeConte Hall MC 7300, Berkeley, CA, 94720-7300}

\author{    D.~Baugh}
\affiliation{   Tsinghua Center for Astrophysics, Tsinghua University, Beijing 100084, China }

\author{     K.~Boone}
\affiliation{    Physics Division, Lawrence Berkeley National Laboratory, 
    1 Cyclotron Road, Berkeley, CA, 94720}
\affiliation{
    Department of Physics, University of California Berkeley,
    366 LeConte Hall MC 7300, Berkeley, CA, 94720-7300}

\author{     S.~Bongard}
\affiliation{ Sorbonne Universit\'e, Universit\'e Paris Diderot, CNRS/IN2P3, Laboratoire de Physique Nucl\'eaire et de Hautes Energies, LPNHE, 4 Place Jussieu, F-75252, Paris, France}

\author{     C.~Buton}
\affiliation{    Universit\'e de Lyon, F-69622, Lyon, France ; Universit\'e de Lyon 1, Villeurbanne ; 
    CNRS/IN2P3, Institut de Physique Nucl\'eaire de Lyon}
    
\author{     J.~Chen}
\affiliation{   Tsinghua Center for Astrophysics, Tsinghua University, Beijing 100084, China }

\author{     N.~Chotard}
\affiliation{    Universit\'e de Lyon, F-69622, Lyon, France ; Universit\'e de Lyon 1, Villeurbanne ; 
    CNRS/IN2P3, Institut de Physique Nucl\'eaire de Lyon}
    
\author[0000-0002-5317-7518]{     Y.~Copin}
\affiliation{    Universit\'e de Lyon, F-69622, Lyon, France ; Universit\'e de Lyon 1, Villeurbanne ; 
    CNRS/IN2P3, Institut de Physique Nucl\'eaire de Lyon}

\author{ S.~Dixon}
\affiliation{
    Department of Physics, University of California Berkeley,
    366 LeConte Hall MC 7300, Berkeley, CA, 94720-7300}

\author{     P.~Fagrelius}
\affiliation{    Physics Division, Lawrence Berkeley National Laboratory, 
    1 Cyclotron Road, Berkeley, CA, 94720}
\affiliation{
    Department of Physics, University of California Berkeley,
    366 LeConte Hall MC 7300, Berkeley, CA, 94720-7300}

\author{     H.~K.~Fakhouri}
\affiliation{    Physics Division, Lawrence Berkeley National Laboratory, 
    1 Cyclotron Road, Berkeley, CA, 94720}
  \affiliation{
    Department of Physics, University of California Berkeley,
    366 LeConte Hall MC 7300, Berkeley, CA, 94720-7300}

\author{     U.~Feindt}
\affiliation{The Oskar Klein Centre, Department of Physics, AlbaNova, Stockholm University, SE-106 91 Stockholm, Sweden}

\author{     D.~Fouchez}
\affiliation{ Aix Marseille Univ, CNRS/IN2P3, CPPM, Marseille, France
}
    
\author[0000-0001-6728-1423]{     E.~Gangler}  
\affiliation{    Clermont Universit\'e, Universit\'e Blaise Pascal, CNRS/IN2P3, Laboratoire de Physique Corpusculaire,
    BP 10448, F-63000 Clermont-Ferrand, France}
    
\author{     B.~Hayden}
\affiliation{    Physics Division, Lawrence Berkeley National Laboratory, 
    1 Cyclotron Road, Berkeley, CA, 94720}

\author{     W.~Hillebrandt}
\affiliation{    Max-Planck-Institut f\"ur Astrophysik, Karl-Schwarzschild-Str. 1,
D-85748 Garching, Germany}

\author{     M.~Kowalski}
\affiliation{    Institut fur Physik,  Humboldt-Universitat zu Berlin,
    Newtonstr. 15, 12489 Berlin}
\affiliation{ DESY, D-15735 Zeuthen, Germany}

\author{     P.-F.~Leget}
\affiliation{    Clermont Universit\'e, Universit\'e Blaise Pascal, CNRS/IN2P3, Laboratoire de Physique Corpusculaire,
    BP 10448, F-63000 Clermont-Ferrand, France}
    
\author{     S.~Lombardo}
\affiliation{    Institut fur Physik,  Humboldt-Universitat zu Berlin,
    Newtonstr. 15, 12489 Berlin}
    
\author{     J.~Nordin}
\affiliation{    Institut fur Physik,  Humboldt-Universitat zu Berlin,
    Newtonstr. 15, 12489 Berlin}
    
\author{     R.~Pain}
\affiliation{ Sorbonne Universit\'e, Universit\'e Paris Diderot, CNRS/IN2P3, Laboratoire de Physique Nucl\'eaire et de Hautes Energies, LPNHE, 4 Place Jussieu, F-75252, Paris, France}
     
\author{     E.~Pecontal}
\affiliation{   Centre de Recherche Astronomique de Lyon, Universit\'e Lyon 1,
    9 Avenue Charles Andr\'e, 69561 Saint Genis Laval Cedex, France}
    
\author{    R.~Pereira}
 \affiliation{    Universit\'e de Lyon, F-69622, Lyon, France ; Universit\'e de Lyon 1, Villeurbanne ; 
    CNRS/IN2P3, Institut de Physique Nucl\'eaire de Lyon}
 
 \author{    S.~Perlmutter}
 \affiliation{    Physics Division, Lawrence Berkeley National Laboratory, 
    1 Cyclotron Road, Berkeley, CA, 94720} 
\affiliation{
    Department of Physics, University of California Berkeley,
    366 LeConte Hall MC 7300, Berkeley, CA, 94720-7300}
    
 \author{    D.~Rabinowitz}
 \affiliation{    Department of Physics, Yale University, 
    New Haven, CT, 06250-8121}
    
 \author[0000-0002-8121-2560]{    M.~Rigault} 
\affiliation{ Clermont Universit\'e, Universit\'e Blaise Pascal, CNRS/IN2P3, Laboratoire de Physique Corpusculaire,
    BP 10448, F-63000 Clermont-Ferrand, France}
     
 \author{    D.~Rubin}
 \affiliation{    Physics Division, Lawrence Berkeley National Laboratory, 
    1 Cyclotron Road, Berkeley, CA, 94720}
    \affiliation{   Space Telescope Science Institute, 3700 San Martin Drive, Baltimore, MD 21218}
 
 \author{    K.~Runge}
 \affiliation{    Physics Division, Lawrence Berkeley National Laboratory, 
    1 Cyclotron Road, Berkeley, CA, 94720}
 
 \author{    C.~Saunders}
 \affiliation{    Physics Division, Lawrence Berkeley National Laboratory, 
    1 Cyclotron Road, Berkeley, CA, 94720}
 \affiliation{Sorbonne Universit\'es, Institut Lagrange de Paris (ILP), 98 bis Boulevard Arago, 75014 Paris, France}

\author{    C.~Sofiatti}
\affiliation{    Physics Division, Lawrence Berkeley National Laboratory, 
    1 Cyclotron Road, Berkeley, CA, 94720} 
\affiliation{
    Department of Physics, University of California Berkeley,
    366 LeConte Hall MC 7300, Berkeley, CA, 94720-7300}

\author{    N.~Suzuki}
\affiliation{    Physics Division, Lawrence Berkeley National Laboratory, 
    1 Cyclotron Road, Berkeley, CA, 94720}

\author{     S.~Taubenberger}
\affiliation{    Max-Planck-Institut f\"ur Astrophysik, Karl-Schwarzschild-Str. 1,
D-85748 Garching, Germany}

\author{     C.~Tao}
\affiliation{   Tsinghua Center for Astrophysics, Tsinghua University, Beijing 100084, China }
\affiliation{ Aix Marseille Univ, CNRS/IN2P3, CPPM, Marseille, France
}
   
\author{     R.~C.~Thomas}
\affiliation{    Computational Cosmology Center, Computational Research Division, Lawrence Berkeley National Laboratory, 
    1 Cyclotron Road MS 50B-4206, Berkeley, CA, 94720}
    
\collaboration{(The Nearby Supernova Factory)}


\begin{abstract}
Through empirical modeling of its observed signal, the peak absolute magnitude of a Type~Ia supernova (SN~Ia) can be accurately determined,
making SNe~Ia  excellent distance indicators.  Improved modeling of SN~Ia colors and magnitudes can account for
further physical
diversity that is expected but not included
in current models, and so consequently can lead to more precise per-object distances with smaller systematic uncertainties.  In this article, we present   
an empirical model for SN~Ia peak colors with
three latent color parameters and dependence on the equivalent widths of Ca~II and Si~II, the Si~II velocity,
and the light-curve shape.
This model is applied to the supernova sample of the Nearby Supernova Factory.  The peak magnitudes in synthetic
broadband photometry and their colors are found to be 
dependent on the observable features and on the three color parameters.
The color changes allowed by two of the color parameters are similar to those allowed by the extinction model of  \citet{1999PASP..111...63F}. 
Correspondingly the sample has an effective value of $\langle R^F_{\mathit{eff}}\rangle \sim 2.43$, though with a broad distribution of 
total-to-selective extinctions. 
We detect the influence on color by a third parameter at
%--- table.py
$> 99\%$
%-----
confidence.
The third parameter has  unique color characteristics that distinguish it from 
those already  tracked by the spectral features, light-curve shape, and two-parameter dust models.
These color parameters must be accounted for when using SNe~Ia as distance indicators.
\end{abstract}

\keywords{supernovae: general; cosmology: distance scale; methods: data analysis}

\section{Introduction}
Type~Ia supernovae (SNe~Ia) form a homogenous set of exploding stars and as such were early recognized and utilized as a powerful distance indicator 
and probe of cosmology \citep[e.g.][]{1992ARA&A..30..359B, 1993ApJ...415....1S}.  After further careful consideration of supernova data, it was recognized
that SN~Ia light-curve shapes \citep{1993ApJ...413L.105P} and colors \citep{1996ApJ...473...88R, 1998A&A...331..815T} exhibit subtle signs of heterogeneity
that are correlated with absolute magnitude, and must be considered when inferring distances.  Empirical models parameterizing SNe~Ia by their light-curve shape \citep{1996ApJ...473...88R,
1997ApJ...483..565P,
1999ApJ...517..565P}
and color  \citep{1996ApJ...473...88R}  were developed that enabled absolute magnitude corrections
and accurate distance measurements of cosmological supernovae,
which 
were subsequently used in the discovery of the accelerating expansion of the Universe \citep{1998AJ....116.1009R,1999ApJ...517..565P}.

The two most commonly used supernova-cosmology light-curve fitters today are SALT2 \citep{2007A&A...466...11G} and MLCS2k2
\citep{2007ApJ...659..122J}.\footnote{Light-curve fitters with more flexible degrees of freedom
\citep[e.g.][]{2008ApJ...681..482C, 2011AJ....141...19B, 2011ApJ...731..120M} are available and have for
the most part been used to study SN~Ia heterogeneity.}
They remain two-parameter models, with one parameter
primarily
characterizing light-curve shape and the other
color.
In SALT2 the light curve shapes are described by phase-dependent flux corrections,
 whereas MLCS2k2 varies shapes through additive magnitude corrections.
The physical cause of the color diversity is interpreted differently by the two sets of authors: 
\citet{2007A&A...466...11G} pragmatically extract color variation empirically from SNe that span a wide range of colors, with no attribution
to either intrinsic or extrinsic origins;
\citet{2007ApJ...659..122J}
attribute changes in color
partially to intrinsic variations linked to light-curve shape, and partially
to the reddening of light from host-galaxy dust.  Differences between these models produce differences in the results of
analyses of both low-redshift \citep{2007ApJ...664L..13C} and high-redshift \citep{2009ApJS..185...32K} supernovae.

There is evidence that supports the expectation that a single parameter beyond light-curve shape  cannot describe the full range
of colors seen in the SN~Ia population.  One approach to look for color diversity is to find correlations between color and spectral features.
\citet{2009ApJ...699L.139W, 2011ApJ...729...55F} find two subpopulations distinguished
by Si~II velocity that exhibit differing $B_{\mathit{\mathit{max}}}-V_{\mathit{max}}$; this color correlation, in addition to one with $B-R$, is confirmed by
\citet{2014ApJ...797...75M}.
\citet{2009ApJ...699L.139W,2015MNRAS.451.1973S}
find that high-velocity Si~II~$\lambda$6355 is found in objects that have red ultraviolet/optical colors near maximum brightness.
\citet{2011MNRAS.413.3075M} show evidence that supernova asymmetry and viewing angle,
traced by wavelength shifts in nebular emission lines, is an important determinant in controlling supernova color; such correlations are also seen by \citet{2011A&A...534L..15C}.

Another approach to probe color diversity is through multiple colors (at least 3 bands)
of individual supernovae.  Color ratios are sensitive to processes of the responsible physics.   For example,
relative dust absorption varies as a function of wavelength depending on grain size, distribution, composition and shape,
independent (to first order) of the amount of dust along the line of sight.
Near and mid-UV photometry obtained by the Ultra-Violet/Optical Telescope on the Swift spacecraft cannot be entirely explained
by dust absorption and hence imply intrinsic variability of supernova UV colors
\citep{2017ApJ...836..232B}.
Measurements of color ratios are being advanced with the development of flexible empirical light curve models that accommodate flexibility in multi-band colors
\citep[e.g.][]{2011ApJ...731..120M}.
\citet{2014ApJ...789...32B, 2015MNRAS.453.3300A} find wide
ranges of total-to-selective extinction with average values significantly lower than $R_V = 3.1$,
the canonical value for diffuse Milky Way dust.
They also confirm the \citet{2006MNRAS.369.1880E, 2011ApJ...731..120M} finding that low $R_V$ is associated with high-extinction supernovae.
In contrast, \citet{2011A&A...529L...4C} argue that after accounting for the diversity of spectral features,
the $R_V=3.1$ measured for the diffuse Milky Way dust is recovered on average and \citet{2017ApJ...836..157H}
find $R_V=2.95 \pm 0.08$ for the highly-extincted SN~2012cu.

Even two parameters are probably not enough to capture supernova color diversity.  SNe~Ia 
are affected by interstellar Milky Way-like dust  and by ``intrinsic'' color variations from the supernova itself and its surrounding circumstellar dust. Physical modeling
shows that the complexity of interstellar dust shouldn't be expected to be captured by a single parameter \citep{2015ApJ...807L..26G,
2017ApJ...836...13H}. Supernova models \citep[e.g.][]{2007ApJ...662..487W,2013MNRAS.436..333S,2014MNRAS.441..532D,2015MNRAS.454.2549B,
2017ApJ...846...58H} and circumstellar dust \citep{2005ApJ...635L..33W,
2008ApJ...686L.103G} produce
color variations beyond those due to interstellar dust. 
\added{
While  \citet{2017ApJ...850...84H} have shown that ISM structure will generally be insufficient to result in detectable dust transmission variations as a SN expands, a foreground ISM dust sheet close enough to a SN can lead to light echoes, e.g.\ as reported in \citet{doi:10.1093/mnras/sty1619}, that can introduce time dependence in the SN flux and resulting colors.
 }
A third color
parameter is therefore expected.

Hierarchical modeling has recently enabled
the study of intrinsic supernova color based on SN~Ia Hubble diagrams
by allowing the addition of a supernova population model as an extra level within in the larger model.
Latent parameters that are not directly tied to observables
but  influence color can be included in such models.
\citet{2017ApJ...842...93M} \citepalias[henceforth referred to as][]{2017ApJ...842...93M}) take the approach of modeling the distribution of their parameters to find that
scatter in the Hubble diagram is better explained by a combination of 
an intrinsic color-magnitude relationship with slope $\beta_{\mathit{int}}\sim 2.3$ intrinsic
color dispersion,
and
$R_V=2.7$ dust, rather than by dust with no color dispersion.
They draw these conclusions by using only the SALT2 (v2.4)
 $c$
parameter as the summary statistic that describes color.

The Nearby Supernova Factory \citep[SNfactory;][]{2002SPIE.4836...61A} has systematically observed the
spectrophotometric time series of hundreds of Hubble-flow $0.03<z<0.08$ SNe~Ia.   The $3200$--$10000$~\AA\ spectral coverage
provides measurements of an array of supernova spectral features while also enablng synthetic broadband photometry
spanning near-UV to near-IR SN-frame wavelengths.  SNfactory specifically targeted objects
early in their temporal evolution, so that well over a hundred of these supernovae have  coverage over
peak brightness.  This dataset provides a homogenous sample with which to study SN~Ia colors and spectral features simultaneously.

In this article we use the idea that spectral indicators carry information on intrinsic supernova colors at peak magnitude.
This approach is taken by \citet{2011A&A...529L...4C}, who find that after standardization based on Ca and Si features, remaining residual color
variation is consistent with Milky Way dust models.
We accommodate up to  three independent color parameters.
The data used in this analysis are described in \S\ref{data:sec}.
The analysis itself is presented as a series of three models of increasing complexity.
\S\ref{modelI:sec}  and  \S\ref{modelII:sec} present the first two models, providing an introduction to
our methodology and yielding results to be compared with those from the third model.
That ultimate model and its results are discussed in detail in \S\ref{modelIII:sec}. 
In \S\ref{modelI:sec} we present Model~I, 
a first analysis using spectral features to standardize SN colors
and two additional latent parameters to account for color diversity. 
By latent, we mean that the parameters are not directly associated with a single observable but rather deduced
as part of a hierarchical
model.
For Model~II described in \S\ref{modelII:sec}, we add a third latent color parameter to the analysis.
In \S\ref{modelIII:sec} we present our final Model~III, which  includes light-curve shape as an additional standardization parameter.
Conclusions are presented in \S\ref{conclusions:sec}.

\section{Data}
\label{data:sec}

Our analysis uses the spectrophotometric data set obtained by
the SNfactory with the SuperNova Integral Field
Spectrograph \citep[SNIFS,][]{2002SPIE.4836...61A, 2004SPIE.5249..146L}.  SNIFS is a fully integrated
instrument optimized for automated observation of point sources on a
structured background over the full ground-based optical window at
moderate spectral resolution ($R \sim 500$).  It consists of a
high-throughput wide-band lenslet integral field spectrograph, a multi-band
imager that covers the field in the vicinity of
the IFS for atmospheric transmission monitoring simultaneous with
spectroscopy, and an acquisition/guiding channel.  The IFS possesses a
fully-filled $6\farcs 4 \times 6\farcs 4$ spectroscopic field of view
subdivided into a grid of $15 \times 15$ spatial elements, a
dual-channel spectrograph covering 3200--5200~\AA\ and 5100--10000~\AA\
simultaneously, and an internal calibration unit (continuum and arc
lamps).  SNIFS is mounted on the south bent Cassegrain port of the
University of Hawaii 2.2~m telescope on Mauna Kea, and is operated
remotely.  Observations are reduced using the SNfactory's dedicated data
reduction pipeline, similar to that presented in \S4 of \citet{2001MNRAS.326...23B}.
A discussion of the software pipeline is presented in
\citet{2006ApJ...650..510A} and is updated in \citet{2010ApJ...713.1073S}. 
The flux calibration is presented in \citet{2013A&A...549A...8B}.
A detailed
description of host-galaxy subtraction is given in \citet{2011MNRAS.418..258B}.

\added{The SNfactory sample used here consists of 198 SNe~Ia
having fully-processed spectrophotometric lightcurve data, including
observations on at least two photometric nights, final references, and a
host spectroscopic redshift.   This sample was defined for use in \citet{leget:thesis}.
These supernovae have well-sampled lightcurves as defined by having
at minimum 5 spectra while still active, with at least four within
epochs  $[-10, 35]$ days, at least one within $[-10, 7]$ days, at least one within $[7, 20]$ days,
and at least one color
%(Anyone know what this means? See slide 3 of
%\url{https://snf-doc.lbl.gov/twiki/pub/Conf/VideoConf20160317/IDR_03-17-2016.pdf}) 
within $[-8, 10]$ days of peak $B$ brightness.
Synthetic-photometric lightcurves derived from the spectra are fit to the SALT2~(v2.4)
model; we require that supernovae have no more than 20\% points that are
$>0.2$ mag from the best-fit SALT2 model, and require a Normalized Median Absolute Deviation (nMAD) of residuals
of  $< 0.12$ mag relative to this model.  \citep[It is known that more flexibility than available to SALT2 is required to capture
the full range of observed SN~Ia behavior, e.g.\ ][]{2011AJ....141...19B, 2011ApJ...731..120M}.
%These all have at least 5 spectra while the SN is
%active, and pass the quality cuts suggested by \citet{2010A&A...523A...7G}.
More than 80\% of our SNe are from searches where there was no
pre-selection based on host galaxy properties (those
whose names start with ``SNF'', ``LSQ'', or ``PTF'' in Table~\ref{data:tab}).

\citet{2013ApJ...770..108C}  present the distribution of host-galaxy parameters for a subset of the  above sample.
\citet{2015MNRAS.447.1247S} show scatter plots of the peak Si~II~$\lambda 5972$ and
Si~II~$\lambda 6355$ velocities and pseudo-equivalent
widths, absolute $B$ and $V$ magnitudes, $B-V$, SALTII $c$ and $x_1$ parameters, and $\Delta m_{15}$ for 
an preceding sample of 151 SNe.   

%Quality criteria for a 'good' lightcurve are the following: At least 5 nights of observation
%No more than 20\% of rejection (after outlier cut>0.2mag) nMAD of residuals < 0.12 mag
%Phase coverage (4 epochs in [-10 < p < +35] days, 1 epochs in [-10 < p < +7] days, 1 epochs in [+7 < p < +20] days,1 color in [-8 < p < +10] days)

From that sample, we further require at least one spectrum with $\pm 2.5$ days of B-band peak brightness.
We exclude six SNe~Ia in the
SN~1991T, SN~1991bg and SN~2002cx subclasses,
SNF20080905-005 described in \citet{leget:thesis} whose spectral features make it a $>3\sigma$ outlier in the
sample population, and
the extremely red SN2012cu  which has a SALT2 $C \approx 1$ \citep{2017ApJ...836..157H} that is several times larger than the colors
of in-sample supernovae. 
\added{Subsequent to the analysis, LSQ12fhe, PTF11bju, and PTF11mkx 
were identified as SN1991T-like \citep{2018lin} and SNBOSS38 as SN2009dc-like \citep{2018Taubenberger}, the results for these objects are discussed later. }


Considering that our empirical parameters
are not fundamental supernova parameters, their best-fit values are dependent on sample selection.
Nevertheless, this  article seeks
underlying parameters that significantly
affect SN~Ia colors; any such parameters found in a subsample are still
relevant parameters when dealing with the population as a whole.
Within this context, it is not so important whether our sample population is representative
of the underlying population.

There is a fraction of SNfactory-observed supernovae that do have the
observations to satisfy the well-sampled light-curve criteria, but are excluded due to some observations
having not completed full processing.  Incomplete processing can occur for data taken during poor
observing conditions, when there are instrumental artifacts, and late in a supernova's
light-curve when the signal-to-noise and contrast relative to the host are low.  The sample-selection exclusion of these supernovae are not related,
at least directly, to the at-maximum observables used in this analysis.  Our results are insensitive to potential sample-selection
bias due to stochasticity in the measurements.
}


A recent description of the data is presented in \citet{2015ApJ...815...58F}.
We provide a brief summary of the points important for this analysis.
The spectral time-series  are corrected for Milky Way dust
extinction \citep{1989ApJ...345..245C,1998ApJ...500..525S}.  
Each spectral time series is
blue-shifted to rest-frame
based on the systemic redshift of the host \citep[c.f.][]{2013ApJ...770..107C}, and the fluxes are converted to luminosity assuming
distances expected for the supernova redshifts given a flat
$\Lambda$CDM cosmology with $\Omega_M = 0.28$ (with an arbitrarily selected
$H_0$ since the current analysis does not depend on absolute magnitude).

This article presents the first application of a computationally intensive analysis, so although  high spectral resolution
is available we here compress the information into low-resolution broadband photometry. 
Synthetic supernova-frame photometry is generated for a top-hat filter system
comprised of five 
bands with the following wavelength ranges: ${\hat{U}}$ $[3300.00 - 3978.02]$\AA;
${\hat{B}}$ $[3978.02-4795.35]$\AA;
$\hat{V}$ $[4795.35-5780.60]$\AA;
$\hat{R}$ $[5780.60-6968.29]$\AA;
$\hat{I}$ $[6968.29-8400.00]$\AA.
(The diacritic hat serves as a reminder that these are not standard Johnson-Cousins filters.)
For each supernova, the magnitudes within 5-days of peak brightness are used to fit single-band magnitudes
at $B$-band peak brightness.
The equivalent widths of the Si~II~$\lambda 4141$ and Ca~II H\&K features are computed as
in \citet{2008A&A...477..717B} and the 
wavelength of the Si~II~$\lambda 6355$ feature
as in \citet{chotard:thesis, 2017Chotard}.
Equivalent widths and the
Si~II~$\lambda 6355$ wavelength are taken from spectra  within $\pm 2.5$ days from $B$-band maximum;
the average is used  in cases where there are multiple spectral measurements within that time window.
All the above spectral features have units of \AA ngstroms.

Our analysis sample is comprised of the
172
supernovae that have been fully processed and have the data coverage to 
give photometric and spectroscopic statistics described above.
The 
spectroscopic-feature measurements, 
\replaced{SALT2 light-curve shape parameters  $x_1$,
and their uncertainties
are presented
in \citet{2017Chotard}}{
synthetic magnitudes (plus a constant offset), SALT2 light-curve shape parameters  $x_1$,
and their uncertainties
are presented
in 
Table~\ref{data:tab}. 
For the subset supernovae for whom it is available, we use the host-galaxy mass given in \citet{2013ApJ...770..108C}.

\startlongtable
\begin{deluxetable}{crrrrrrrrr}
\tabletypesize{\tiny}
\tablecaption{Supernova Spectral-Feature and Peak-Magnitude Data
\label{data:tab}}
\tablehead{
\colhead{Name} & \colhead{$EW_{\mathrm{Ca}}$ (\AA)} & \colhead{$EW_{\mathrm{Si}}$ (\AA)} & \colhead{$\lambda_{\mathrm{Si}}$ (\AA)} & \colhead{$U$+offset} & \colhead{$B$+offset} & \colhead{$V$+offset} & \colhead{$R$+offset} & \colhead{$I$+offset} & \colhead{$x_1$}
}
\startdata
SNF20050624-000 & $121.0 \pm 5.3$ & $  9.3 \pm 3.1$& $ 6126 \pm   6$ & $-29.75 \pm   0.01$ & $-29.42 \pm   0.01$& $-28.99 \pm   0.01$& $-28.68 \pm   0.01$& $-27.97 \pm   0.01$ & $  0.69 \pm   0.62$\\
SNF20050728-006 & $127.8 \pm 2.5$ & $ 15.8 \pm 1.3$& $ 6124 \pm   6$ & $-28.80 \pm   0.02$ & $-28.68 \pm   0.02$& $-28.37 \pm   0.02$& $-28.18 \pm   0.02$& $-27.55 \pm   0.02$ & $  0.27 \pm   0.27$\\
SNF20050729-002 & $109.4 \pm 2.2$ & $ 11.5 \pm 1.7$& $ 6142 \pm   6$ & $-29.35 \pm   0.13$ & $-29.17 \pm   0.13$& $-28.68 \pm   0.13$& $-28.38 \pm   0.13$& $-27.56 \pm   0.13$ & $  0.57 \pm   0.34$\\
SNF20050821-007 & $141.7 \pm 2.6$ & $  7.7 \pm 1.0$& $ 6140 \pm   9$ & $-29.38 \pm   0.02$ & $-29.20 \pm   0.02$& $-28.77 \pm   0.02$& $-28.46 \pm   0.02$& $-27.67 \pm   0.02$ & $  0.17 \pm   0.32$\\
SNF20060511-014 & $102.6 \pm 2.8$ & $ 15.6 \pm 1.1$& $ 6141 \pm   8$ & $-29.16 \pm   0.07$ & $-29.04 \pm   0.06$& $-28.56 \pm   0.06$& $-28.30 \pm   0.06$& $-27.63 \pm   0.06$ & $ -0.68 \pm   0.18$\\
SNF20060512-001 & $ 88.4 \pm 1.2$ & $  5.4 \pm 0.4$& $ 6169 \pm   3$ & $-29.33 \pm   0.01$ & $-29.05 \pm   0.01$& $-28.68 \pm   0.01$& $-28.40 \pm   0.01$& $-27.79 \pm   0.01$ & $  0.69 \pm   0.15$\\
SNF20060512-002 & $100.2 \pm 2.8$ & $ 13.4 \pm 1.1$& $ 6107 \pm   8$ & $-29.33 \pm   0.02$ & $-29.11 \pm   0.02$& $-28.77 \pm   0.02$& $-28.52 \pm   0.02$& $-27.80 \pm   0.02$ & $ -0.92 \pm   0.19$\\
SNF20060521-001 & $ 78.9 \pm 20.2$ & $ 21.1 \pm 1.4$& $ 6123 \pm  10$ & $-29.37 \pm   0.05$ & $-29.04 \pm   0.05$& $-28.54 \pm   0.05$& $-28.30 \pm   0.05$& $-27.57 \pm   0.05$ & $ -1.22 \pm   0.26$\\
SNF20060526-003 & $112.1 \pm 2.5$ & $  9.8 \pm 1.0$& $ 6121 \pm   3$ & $-29.34 \pm   0.01$ & $-29.09 \pm   0.01$& $-28.68 \pm   0.01$& $-28.39 \pm   0.01$& $-27.70 \pm   0.01$ & $  0.29 \pm   0.17$\\
SNF20060609-002 & $ 87.7 \pm 3.6$ & $  7.3 \pm 1.3$& $ 6132 \pm   4$ & $-28.60 \pm   0.02$ & $-28.42 \pm   0.02$& $-28.19 \pm   0.02$& $-28.05 \pm   0.02$& $-27.53 \pm   0.02$ & $  0.07 \pm   0.15$\\
SNF20060618-014 & $137.2 \pm 2.5$ & $  9.3 \pm 1.1$& $ 6112 \pm   7$ & $-29.27 \pm   0.03$ & $-29.09 \pm   0.03$& $-28.73 \pm   0.03$& $-28.38 \pm   0.03$& $-27.68 \pm   0.03$ & $  0.97 \pm   0.21$\\
SNF20060618-023 & $ 74.9 \pm 4.9$ & $  5.0 \pm 1.8$& $ 6137 \pm  21$ & $-29.61 \pm   0.02$ & $-29.18 \pm   0.02$& $-28.89 \pm   0.02$& $-28.66 \pm   0.02$& $-28.08 \pm   0.02$ & $  0.12 \pm   0.18$\\
SNF20060621-015 & $111.9 \pm 1.3$ & $  9.8 \pm 0.7$& $ 6144 \pm   3$ & $-29.63 \pm   0.01$ & $-29.36 \pm   0.01$& $-28.88 \pm   0.01$& $-28.54 \pm   0.01$& $-27.81 \pm   0.01$ & $  0.43 \pm   0.20$\\
SNF20060907-000 & $106.1 \pm 10.4$ & $ 17.0 \pm 0.9$& $ 6149 \pm   4$ & $-29.54 \pm   0.02$ & $-29.28 \pm   0.01$& $-28.76 \pm   0.01$& $-28.42 \pm   0.01$& $-27.74 \pm   0.04$ & $  0.17 \pm   0.17$\\
SNF20060908-004 & $114.4 \pm 1.2$ & $ 12.6 \pm 0.6$& $ 6136 \pm   3$ & $-29.59 \pm   0.23$ & $-29.34 \pm   0.23$& $-28.91 \pm   0.23$& $-28.58 \pm   0.23$& $-27.87 \pm   0.23$ & $  0.70 \pm   0.19$\\
SNF20060912-000 & $106.5 \pm 1.8$ & $ 21.4 \pm 1.7$& $ 6163 \pm   7$ & $-28.98 \pm   0.02$ & $-28.92 \pm   0.02$& $-28.66 \pm   0.02$& $-28.42 \pm   0.02$& $-27.77 \pm   0.02$ & $  0.24 \pm   0.20$\\
SNF20061011-005 & $120.6 \pm 1.1$ & $  9.3 \pm 0.4$& $ 6132 \pm   4$ & $-29.72 \pm   0.04$ & $-29.43 \pm   0.03$& $-28.99 \pm   0.03$& $-28.64 \pm   0.03$& $-27.90 \pm   0.03$ & $  0.74 \pm   0.29$\\
SNF20061020-000 & $ 95.4 \pm 18.8$ & $ 24.1 \pm 1.0$& $ 6120 \pm   5$ & $-29.01 \pm   0.03$ & $-28.78 \pm   0.03$& $-28.35 \pm   0.03$& $-28.17 \pm   0.03$& $-27.54 \pm   0.03$ & $ -1.67 \pm   0.20$\\
SNF20061021-003 & $122.8 \pm 2.3$ & $  9.7 \pm 1.7$& $ 6131 \pm   4$ & $-29.04 \pm   0.02$ & $-28.86 \pm   0.02$& $-28.56 \pm   0.02$& $-28.30 \pm   0.02$& $-27.64 \pm   0.02$ & $  0.25 \pm   0.18$\\
SNF20061022-005 & $ 64.6 \pm 3.8$ & $  3.7 \pm 1.4$& $ 6146 \pm   7$ & $-29.49 \pm   0.02$ & $-29.06 \pm   0.02$& $-28.71 \pm   0.02$& $-28.42 \pm   0.02$& $-27.93 \pm   0.02$ & $  1.72 \pm   0.44$\\
SNF20061024-000 & $ 86.9 \pm 26.8$ & $ 30.0 \pm 1.5$& $ 6127 \pm   5$ & $-28.88 \pm   0.04$ & $-28.70 \pm   0.04$& $-28.26 \pm   0.04$& $-28.05 \pm   0.04$& $-27.40 \pm   0.04$ & $ -1.39 \pm   0.19$\\
SNF20061030-010 & $131.4 \pm 2.2$ & $ 17.4 \pm 1.1$& $ 6116 \pm   4$ & $-28.60 \pm   0.02$ & $-28.55 \pm   0.02$& $-28.25 \pm   0.02$& $-28.03 \pm   0.02$& $-27.34 \pm   0.02$ & $  0.14 \pm   0.25$\\
SNF20061108-004 & $129.5 \pm 5.6$ & $  6.3 \pm 2.5$& $ 6110 \pm   6$ & $-29.53 \pm   0.02$ & $-29.31 \pm   0.02$& $-28.95 \pm   0.02$& $-28.60 \pm   0.02$& $-27.96 \pm   0.02$ & $  2.14 \pm   0.42$\\
SNF20061111-002 & $110.8 \pm 10.7$ & $ 20.4 \pm 1.0$& $ 6145 \pm   6$ & $-29.16 \pm   0.01$ & $-28.99 \pm   0.01$& $-28.59 \pm   0.01$& $-28.29 \pm   0.01$& $-27.61 \pm   0.01$ & $  0.27 \pm   0.21$\\
SNF20070330-024 & $118.1 \pm 2.1$ & $  4.6 \pm 2.2$& $ 6101 \pm   3$ & $-29.77 \pm   0.02$ & $-29.52 \pm   0.02$& $-29.08 \pm   0.02$& $-28.74 \pm   0.01$& $-27.94 \pm   0.02$ & $  0.34 \pm   0.29$\\
SNF20070331-025 & $119.8 \pm 7.4$ & $ 14.2 \pm 2.7$& $ 6120 \pm  10$ & $-28.94 \pm   0.02$ & $-28.75 \pm   0.02$& $-28.32 \pm   0.02$& $-28.07 \pm   0.02$& $-27.31 \pm   0.02$ & $ -1.28 \pm   0.16$\\
SNF20070403-000 & $ 61.8 \pm 6.5$ & $ 27.1 \pm 1.8$& $ 6154 \pm   8$ & $-28.37 \pm   0.02$ & $-28.27 \pm   0.02$& $-27.97 \pm   0.02$& $-27.80 \pm   0.02$& $-27.24 \pm   0.02$ & $ -1.96 \pm   0.19$\\
SNF20070403-001 & $105.9 \pm 5.4$ & $ 18.3 \pm 1.8$& $ 6124 \pm   4$ & $-29.23 \pm   0.02$ & $-29.04 \pm   0.01$& $-28.63 \pm   0.01$& $-28.35 \pm   0.01$& $-27.62 \pm   0.01$ & $ -0.54 \pm   0.13$\\
SNF20070417-002 & $104.5 \pm 5.5$ & $ 24.4 \pm 2.2$& $ 6123 \pm   9$ & $-29.20 \pm   0.05$ & $-29.01 \pm   0.05$& $-28.48 \pm   0.05$& $-28.23 \pm   0.05$& $-27.54 \pm   0.05$ & $ -1.37 \pm   0.11$\\
SNF20070424-003 & $122.5 \pm 3.8$ & $ 12.7 \pm 1.6$& $ 6132 \pm   6$ & $-29.10 \pm   0.01$ & $-28.96 \pm   0.01$& $-28.51 \pm   0.01$& $-28.25 \pm   0.01$& $-27.57 \pm   0.01$ & $  0.31 \pm   0.11$\\
SNF20070427-001 & $ 81.3 \pm 2.3$ & $  6.3 \pm 0.9$& $ 6142 \pm   5$ & $-29.89 \pm   0.02$ & $-29.46 \pm   0.02$& $-28.97 \pm   0.02$& $-28.62 \pm   0.02$& $-27.97 \pm   0.02$ & $  0.62 \pm   0.21$\\
SNF20070506-006 & $ 94.1 \pm 1.3$ & $  6.7 \pm 0.6$& $ 6153 \pm   3$ & $-29.72 \pm   0.01$ & $-29.39 \pm   0.01$& $-28.97 \pm   0.01$& $-28.64 \pm   0.01$& $-27.96 \pm   0.01$ & $  1.17 \pm   0.13$\\
SNF20070531-011 & $122.4 \pm 2.7$ & $ 21.2 \pm 0.8$& $ 6114 \pm   4$ & $-29.07 \pm   0.01$ & $-28.94 \pm   0.01$& $-28.50 \pm   0.01$& $-28.26 \pm   0.01$& $-27.53 \pm   0.03$ & $ -1.82 \pm   0.25$\\
SNF20070630-006 & $125.5 \pm 3.2$ & $ 10.1 \pm 1.6$& $ 6126 \pm   4$ & $-29.34 \pm   0.01$ & $-29.12 \pm   0.01$& $-28.65 \pm   0.01$& $-28.38 \pm   0.01$& $-27.66 \pm   0.01$ & $  0.35 \pm   0.12$\\
SNF20070701-005 & $101.8 \pm 2.6$ & $ 12.4 \pm 1.0$& $ 6158 \pm   5$ & $-29.46 \pm   0.02$ & $-29.27 \pm   0.02$& $-28.87 \pm   0.02$& $-28.60 \pm   0.02$& $-27.96 \pm   0.02$ & $  0.06 \pm   0.17$\\
SNF20070712-003 & $108.8 \pm 2.7$ & $ 13.5 \pm 0.9$& $ 6155 \pm   6$ & $-29.44 \pm   0.02$ & $-29.19 \pm   0.01$& $-28.74 \pm   0.01$& $-28.42 \pm   0.01$& $-27.78 \pm   0.01$ & $  0.42 \pm   0.18$\\
SNF20070714-007 & $129.6 \pm 5.6$ & $ 31.1 \pm 23.8$& $ 6146 \pm   5$ & $-27.88 \pm   0.02$ & $-28.12 \pm   0.01$& $-28.02 \pm   0.01$& $-27.86 \pm   0.01$& $-27.24 \pm   0.03$ & $ -1.10 \pm   0.13$\\
SNF20070725-001 & $108.4 \pm 2.0$ & $ 11.1 \pm 1.5$& $ 6140 \pm   7$ & $-29.61 \pm   0.02$ & $-29.32 \pm   0.02$& $-28.84 \pm   0.02$& $-28.50 \pm   0.02$& $-27.76 \pm   0.02$ & $  0.68 \pm   0.17$\\
SNF20070727-016 & $ 77.5 \pm 2.5$ & $  5.1 \pm 0.8$& $ 6140 \pm   4$ & $-29.96 \pm   0.06$ & $-29.56 \pm   0.06$& $-29.06 \pm   0.06$& $-28.75 \pm   0.06$& $-28.01 \pm   0.06$ & $  0.27 \pm   0.12$\\
SNF20070802-000 & $158.3 \pm 3.3$ & $ 16.3 \pm 1.7$& $ 6102 \pm   5$ & $-28.90 \pm   0.01$ & $-28.81 \pm   0.01$& $-28.45 \pm   0.01$& $-28.22 \pm   0.01$& $-27.52 \pm   0.01$ & $  0.08 \pm   0.16$\\
SNF20070803-005 & $ 22.7 \pm 21.4$ & $  0.9 \pm 0.6$& $ 6157 \pm  27$ & $-29.87 \pm   0.01$ & $-29.43 \pm   0.01$& $-29.04 \pm   0.01$& $-28.74 \pm   0.01$& $-28.11 \pm   0.01$ & $  0.88 \pm   0.18$\\
SNF20070806-026 & $ 98.8 \pm 12.1$ & $ 25.9 \pm 0.7$& $ 6114 \pm   7$ & $-29.14 \pm   0.02$ & $-28.91 \pm   0.02$& $-28.44 \pm   0.02$& $-28.21 \pm   0.02$& $-27.49 \pm   0.02$ & $ -1.73 \pm   0.17$\\
SNF20070810-004 & $126.7 \pm 1.8$ & $ 21.1 \pm 1.1$& $ 6118 \pm   7$ & $-29.22 \pm   0.01$ & $-29.10 \pm   0.01$& $-28.63 \pm   0.01$& $-28.34 \pm   0.01$& $-27.62 \pm   0.01$ & $  0.10 \pm   0.12$\\
SNF20070817-003 & $ 93.9 \pm 2.4$ & $ 18.5 \pm 1.3$& $ 6116 \pm   6$ & $-29.19 \pm   0.02$ & $-29.03 \pm   0.01$& $-28.59 \pm   0.01$& $-28.30 \pm   0.01$& $-27.55 \pm   0.02$ & $ -0.57 \pm   0.13$\\
SNF20070818-001 & $157.5 \pm 7.5$ & $ 16.7 \pm 1.8$& $ 6115 \pm   5$ & $-28.97 \pm   0.02$ & $-28.96 \pm   0.01$& $-28.61 \pm   0.01$& $-28.37 \pm   0.01$& $-27.62 \pm   0.01$ & $  0.15 \pm   0.16$\\
SNF20070820-000 & $107.2 \pm 3.5$ & $ 18.6 \pm 1.3$& $ 6132 \pm  14$ & $-28.80 \pm   0.02$ & $-28.69 \pm   0.02$& $-28.34 \pm   0.02$& $-28.13 \pm   0.02$& $-27.52 \pm   0.02$ & $ -0.29 \pm   0.13$\\
SNF20070831-015 & $112.2 \pm 2.7$ & $  7.8 \pm 1.0$& $ 6145 \pm   6$ & $-29.42 \pm   0.01$ & $-29.17 \pm   0.01$& $-28.78 \pm   0.01$& $-28.46 \pm   0.01$& $-27.78 \pm   0.01$ & $  1.85 \pm   0.23$\\
SNF20070902-018 & $ 93.8 \pm 12.2$ & $ 23.8 \pm 3.0$& $ 6120 \pm   8$ & $-28.87 \pm   0.02$ & $-28.70 \pm   0.01$& $-28.26 \pm   0.01$& $-28.08 \pm   0.01$& $-27.41 \pm   0.02$ & $ -0.72 \pm   0.10$\\
SNF20070902-021 & $108.9 \pm 3.5$ & $ 17.1 \pm 1.0$& $ 6131 \pm   6$ & $-29.25 \pm   0.02$ & $-29.02 \pm   0.02$& $-28.56 \pm   0.02$& $-28.32 \pm   0.01$& $-27.65 \pm   0.02$ & $ -0.43 \pm   0.10$\\
SNF20071003-016 & $125.2 \pm 4.6$ & $ 17.1 \pm 2.0$& $ 6124 \pm  11$ & $-28.58 \pm   0.02$ & $-28.54 \pm   0.02$& $-28.19 \pm   0.02$& $-27.99 \pm   0.02$& $-27.31 \pm   0.02$ & $ -0.25 \pm   0.15$\\
SNF20071015-000 & $105.0 \pm 3.2$ & $  6.9 \pm 1.1$& $ 6124 \pm   7$ & $-27.89 \pm   0.02$ & $-27.82 \pm   0.02$& $-27.69 \pm   0.02$& $-27.63 \pm   0.02$& $-27.16 \pm   0.04$ & $  1.36 \pm   0.37$\\
SNF20071021-000 & $167.5 \pm 2.2$ & $ 20.4 \pm 0.6$& $ 6112 \pm   4$ & $-28.75 \pm   0.02$ & $-28.78 \pm   0.02$& $-28.40 \pm   0.02$& $-28.18 \pm   0.02$& $-27.41 \pm   0.02$ & $ -0.53 \pm   0.18$\\
SNF20071108-021 & $ 99.1 \pm 2.7$ & $  5.8 \pm 0.8$& $ 6164 \pm   5$ & $-29.67 \pm   0.01$ & $-29.34 \pm   0.01$& $-28.94 \pm   0.01$& $-28.60 \pm   0.01$& $-27.96 \pm   0.01$ & $  1.00 \pm   0.21$\\
SNF20080323-009 & $ 95.9 \pm 2.3$ & $ 10.6 \pm 1.1$& $ 6143 \pm   6$ & $-29.59 \pm   0.02$ & $-29.22 \pm   0.02$& $-28.68 \pm   0.02$& $-28.42 \pm   0.02$& $-27.77 \pm   0.02$ & $ -0.10 \pm   0.17$\\
SNF20080507-000 & $ 98.1 \pm 1.6$ & $ 10.6 \pm 2.1$& $ 6143 \pm   5$ & $-29.23 \pm   0.01$ & $-29.05 \pm   0.01$& $-28.71 \pm   0.01$& $-28.45 \pm   0.01$& $-27.79 \pm   0.01$ & $  0.89 \pm   0.25$\\
SNF20080510-001 & $118.8 \pm 2.1$ & $ 15.3 \pm 1.3$& $ 6115 \pm   4$ & $-29.35 \pm   0.01$ & $-29.15 \pm   0.01$& $-28.69 \pm   0.01$& $-28.38 \pm   0.01$& $-27.68 \pm   0.01$ & $  0.43 \pm   0.18$\\
SNF20080510-005 & $111.6 \pm 2.6$ & $  6.4 \pm 1.1$& $ 6115 \pm   4$ & $-29.41 \pm   0.01$ & $-29.15 \pm   0.01$& $-28.70 \pm   0.01$& $-28.38 \pm   0.01$& $-27.73 \pm   0.04$ & $  1.90 \pm   0.26$\\
SNF20080512-010 & $ 95.3 \pm 3.5$ & $ 23.3 \pm 1.5$& $ 6129 \pm   5$ & $-29.22 \pm   0.08$ & $-28.96 \pm   0.08$& $-28.50 \pm   0.08$& $-28.26 \pm   0.08$& $-27.56 \pm   0.08$ & $ -1.24 \pm   0.26$\\
SNF20080514-002 & $ 83.2 \pm 0.7$ & $ 19.4 \pm 0.6$& $ 6131 \pm   3$ & $-29.30 \pm   0.01$ & $-28.95 \pm   0.01$& $-28.44 \pm   0.01$& $-28.17 \pm   0.01$& $-27.49 \pm   0.01$ & $ -1.57 \pm   0.11$\\
SNF20080516-000 & $117.4 \pm 2.2$ & $  9.0 \pm 1.2$& $ 6135 \pm   3$ & $-29.50 \pm   0.01$ & $-29.23 \pm   0.01$& $-28.80 \pm   0.01$& $-28.47 \pm   0.01$& $-27.74 \pm   0.01$ & $  1.24 \pm   0.26$\\
SNF20080516-022 & $100.1 \pm 2.1$ & $ 13.7 \pm 1.1$& $ 6158 \pm   3$ & $-29.46 \pm   0.01$ & $-29.19 \pm   0.01$& $-28.71 \pm   0.01$& $-28.39 \pm   0.01$& $-27.77 \pm   0.01$ & $  0.20 \pm   0.14$\\
SNF20080522-000 & $ 61.8 \pm 3.5$ & $  3.3 \pm 0.9$& $ 6131 \pm   7$ & $-29.86 \pm   0.01$ & $-29.41 \pm   0.01$& $-29.03 \pm   0.01$& $-28.70 \pm   0.01$& $-28.06 \pm   0.01$ & $  1.02 \pm   0.13$\\
SNF20080522-011 & $122.1 \pm 1.7$ & $  8.3 \pm 0.5$& $ 6125 \pm   3$ & $-29.63 \pm   0.01$ & $-29.38 \pm   0.01$& $-28.92 \pm   0.01$& $-28.60 \pm   0.01$& $-27.88 \pm   0.01$ & $  0.81 \pm   0.19$\\
SNF20080531-000 & $133.0 \pm 1.5$ & $ 17.6 \pm 0.8$& $ 6114 \pm   5$ & $-29.12 \pm   0.01$ & $-28.98 \pm   0.01$& $-28.54 \pm   0.01$& $-28.28 \pm   0.01$& $-27.51 \pm   0.01$ & $ -0.49 \pm   0.18$\\
SNF20080610-000 & $119.9 \pm 10.4$ & $ 16.4 \pm 1.7$& $ 6131 \pm   6$ & $-29.05 \pm   0.07$ & $-28.92 \pm   0.07$& $-28.50 \pm   0.07$& $-28.22 \pm   0.07$& $-27.55 \pm   0.07$ & $  0.12 \pm   0.16$\\
SNF20080612-003 & $120.0 \pm 1.1$ & $  7.3 \pm 0.6$& $ 6123 \pm   3$ & $-29.64 \pm   0.02$ & $-29.41 \pm   0.02$& $-28.99 \pm   0.02$& $-28.70 \pm   0.02$& $-28.00 \pm   0.02$ & $  0.71 \pm   0.21$\\
SNF20080614-010 & $125.4 \pm 5.1$ & $ 26.9 \pm 1.6$& $ 6128 \pm   3$ & $-29.04 \pm   0.04$ & $-28.81 \pm   0.04$& $-28.38 \pm   0.04$& $-28.16 \pm   0.04$& $-27.57 \pm   0.04$ & $ -2.12 \pm   0.17$\\
SNF20080620-000 & $107.8 \pm 14.1$ & $ 20.0 \pm 0.7$& $ 6132 \pm   3$ & $-28.82 \pm   0.02$ & $-28.78 \pm   0.01$& $-28.32 \pm   0.01$& $-28.09 \pm   0.01$& $-27.39 \pm   0.01$ & $ -1.04 \pm   0.16$\\
SNF20080623-001 & $149.1 \pm 1.4$ & $ 14.9 \pm 0.7$& $ 6131 \pm   3$ & $-29.11 \pm   0.01$ & $-28.97 \pm   0.01$& $-28.50 \pm   0.01$& $-28.22 \pm   0.01$& $-27.46 \pm   0.01$ & $  0.01 \pm   0.13$\\
SNF20080626-002 & $130.0 \pm 1.0$ & $  6.1 \pm 4.2$& $ 6111 \pm   3$ & $-29.42 \pm   0.01$ & $-29.24 \pm   0.01$& $-28.84 \pm   0.01$& $-28.52 \pm   0.01$& $-27.76 \pm   0.01$ & $  1.04 \pm   0.23$\\
SNF20080714-008 & $134.8 \pm 15.7$ & $ 19.7 \pm 3.7$& $ 6100 \pm   6$ & $-28.56 \pm   0.02$ & $-28.63 \pm   0.01$& $-28.32 \pm   0.01$& $-28.13 \pm   0.01$& $-27.42 \pm   0.01$ & $  0.01 \pm   0.15$\\
SNF20080717-000 & $ 93.3 \pm 2.6$ & $  8.3 \pm 2.2$& $ 6104 \pm   3$ & $-28.58 \pm   0.01$ & $-28.47 \pm   0.01$& $-28.29 \pm   0.01$& $-28.05 \pm   0.01$& $-27.50 \pm   0.01$ & $  0.70 \pm   0.14$\\
SNF20080720-001 & $138.5 \pm 4.0$ & $ 14.0 \pm 2.0$& $ 6107 \pm   3$ & $-27.59 \pm   0.02$ & $-27.78 \pm   0.01$& $-27.73 \pm   0.01$& $-27.71 \pm   0.01$& $-27.19 \pm   0.02$ & $  0.18 \pm   0.13$\\
SNF20080725-004 & $133.6 \pm 2.1$ & $  6.9 \pm 0.9$& $ 6131 \pm   6$ & $-29.09 \pm   0.01$ & $-28.93 \pm   0.01$& $-28.59 \pm   0.01$& $-28.31 \pm   0.01$& $-27.55 \pm   0.03$ & $  0.60 \pm   0.14$\\
SNF20080802-006 & $108.2 \pm 6.0$ & $ 20.6 \pm 1.9$& $ 6122 \pm   5$ & $-29.02 \pm   0.06$ & $-28.80 \pm   0.06$& $-28.40 \pm   0.06$& $-28.20 \pm   0.06$& $-27.50 \pm   0.06$ & $ -1.56 \pm   0.13$\\
SNF20080803-000 & $117.6 \pm 2.6$ & $  8.9 \pm 2.0$& $ 6125 \pm   4$ & $-28.84 \pm   0.01$ & $-28.70 \pm   0.01$& $-28.35 \pm   0.01$& $-28.16 \pm   0.01$& $-27.50 \pm   0.01$ & $  0.37 \pm   0.14$\\
SNF20080806-002 & $135.8 \pm 1.8$ & $  7.5 \pm 0.9$& $ 6135 \pm   4$ & $-29.22 \pm   0.02$ & $-29.02 \pm   0.02$& $-28.61 \pm   0.02$& $-28.35 \pm   0.01$& $-27.71 \pm   0.02$ & $  0.64 \pm   0.16$\\
SNF20080810-001 & $ 88.4 \pm 21.6$ & $ 22.3 \pm 1.1$& $ 6145 \pm   5$ & $-29.11 \pm   0.01$ & $-28.89 \pm   0.01$& $-28.45 \pm   0.01$& $-28.23 \pm   0.01$& $-27.60 \pm   0.01$ & $ -0.87 \pm   0.11$\\
SNF20080815-017 & $ 63.8 \pm 21.5$ & $ 27.6 \pm 3.8$& $ 6132 \pm   6$ & $-29.04 \pm   0.07$ & $-28.79 \pm   0.07$& $-28.32 \pm   0.07$& $-28.12 \pm   0.07$& $-27.41 \pm   0.07$ & $ -1.78 \pm   0.18$\\
SNF20080821-000 & $105.1 \pm 2.2$ & $  8.6 \pm 1.3$& $ 6121 \pm   4$ & $-29.34 \pm   0.01$ & $-29.10 \pm   0.01$& $-28.73 \pm   0.01$& $-28.46 \pm   0.01$& $-27.82 \pm   0.01$ & $  0.70 \pm   0.19$\\
SNF20080822-005 & $ 78.5 \pm 1.8$ & $  6.3 \pm 0.9$& $ 6138 \pm   4$ & $-29.71 \pm   0.01$ & $-29.34 \pm   0.01$& $-28.93 \pm   0.01$& $-28.61 \pm   0.01$& $-27.92 \pm   0.01$ & $  0.56 \pm   0.23$\\
SNF20080825-010 & $102.4 \pm 13.4$ & $ 19.2 \pm 0.6$& $ 6116 \pm   4$ & $-29.46 \pm   0.01$ & $-29.17 \pm   0.01$& $-28.71 \pm   0.01$& $-28.47 \pm   0.01$& $-27.83 \pm   0.01$ & $ -0.77 \pm   0.18$\\
SNF20080909-030 & $ 93.7 \pm 1.0$ & $  7.8 \pm 0.4$& $ 6171 \pm   3$ & $-29.38 \pm   0.02$ & $-29.12 \pm   0.01$& $-28.74 \pm   0.01$& $-28.44 \pm   0.01$& $-27.78 \pm   0.01$ & $  1.17 \pm   0.16$\\
SNF20080913-031 & $118.2 \pm 1.5$ & $ 11.3 \pm 1.8$& $ 6158 \pm   5$ & $-29.13 \pm   0.08$ & $-29.01 \pm   0.07$& $-28.62 \pm   0.07$& $-28.32 \pm   0.07$& $-27.68 \pm   0.07$ & $  0.03 \pm   0.21$\\
SNF20080914-001 & $126.5 \pm 1.2$ & $ 15.4 \pm 1.1$& $ 6159 \pm   3$ & $-28.67 \pm   0.02$ & $-28.60 \pm   0.02$& $-28.31 \pm   0.02$& $-28.13 \pm   0.02$& $-27.58 \pm   0.02$ & $ -0.11 \pm   0.17$\\
SNF20080918-000 & $146.8 \pm 3.5$ & $  7.5 \pm 2.5$& $ 6110 \pm   5$ & $-28.79 \pm   0.02$ & $-28.65 \pm   0.02$& $-28.35 \pm   0.02$& $-28.12 \pm   0.02$& $-27.46 \pm   0.02$ & $  1.50 \pm   0.30$\\
SNF20080918-002 & $ 97.7 \pm 2.8$ & $ 12.6 \pm 1.4$& $ 6141 \pm   6$ & $-29.50 \pm   0.02$ & $-29.11 \pm   0.02$& $-28.61 \pm   0.02$& $-28.34 \pm   0.02$& $-27.71 \pm   0.02$ & $ -0.50 \pm   0.09$\\
SNF20080918-004 & $ 87.8 \pm 7.2$ & $ 21.5 \pm 0.9$& $ 6141 \pm   4$ & $-29.00 \pm   0.22$ & $-28.82 \pm   0.22$& $-28.37 \pm   0.22$& $-28.13 \pm   0.22$& $-27.43 \pm   0.22$ & $ -1.27 \pm   0.18$\\
SNF20080919-000 & $114.7 \pm 2.8$ & $  9.4 \pm 0.9$& $ 6145 \pm   5$ & $-28.53 \pm   0.02$ & $-28.41 \pm   0.01$& $-28.11 \pm   0.01$& $-27.99 \pm   0.01$& $-27.38 \pm   0.01$ & $ -0.13 \pm   0.11$\\
SNF20080919-001 & $ 85.0 \pm 1.1$ & $  6.0 \pm 0.4$& $ 6150 \pm   5$ & $-29.73 \pm   0.01$ & $-29.43 \pm   0.01$& $-29.04 \pm   0.01$& $-28.72 \pm   0.01$& $-28.07 \pm   0.01$ & $  0.84 \pm   0.18$\\
SNF20080919-002 & $103.6 \pm 7.2$ & $ 27.2 \pm 1.9$& $ 6133 \pm   8$ & $-28.74 \pm   0.02$ & $-28.46 \pm   0.01$& $-28.09 \pm   0.01$& $-27.87 \pm   0.01$& $-27.26 \pm   0.04$ & $ -1.52 \pm   0.14$\\
SNF20080920-000 & $135.2 \pm 1.4$ & $  5.6 \pm 1.6$& $ 6085 \pm   3$ & $-29.44 \pm   0.02$ & $-29.19 \pm   0.02$& $-28.79 \pm   0.02$& $-28.49 \pm   0.02$& $-27.74 \pm   0.02$ & $  1.17 \pm   0.34$\\
CSS110918\_01 & $110.6 \pm 1.0$ & $  8.0 \pm 1.3$& $ 6101 \pm   2$ & $-29.88 \pm   0.76$ & $-29.58 \pm   0.76$& $-29.09 \pm   0.76$& $-28.71 \pm   0.76$& $-27.91 \pm   0.76$ & $  0.92 \pm   0.17$\\
CSS110918\_02 & $109.1 \pm 9.4$ & $ 15.0 \pm 0.6$& $ 6137 \pm   3$ & $-29.36 \pm   0.02$ & $-29.14 \pm   0.01$& $-28.69 \pm   0.01$& $-28.41 \pm   0.01$& $-27.70 \pm   0.01$ & $ -0.27 \pm   0.17$\\
CSS120424\_01 & $138.1 \pm 2.1$ & $ 11.7 \pm 0.7$& $ 6138 \pm   3$ & $-29.40 \pm   0.02$ & $-29.23 \pm   0.02$& $-28.77 \pm   0.01$& $-28.45 \pm   0.02$& $-27.68 \pm   0.02$ & $  0.76 \pm   0.19$\\
CSS130502\_01 & $ 91.5 \pm 10.9$ & $ 15.6 \pm 0.5$& $ 6128 \pm   3$ & $-29.43 \pm   0.02$ & $-29.09 \pm   0.02$& $-28.60 \pm   0.01$& $-28.30 \pm   0.01$& $-27.62 \pm   0.04$ & $ -0.00 \pm   0.24$\\
LSQ12dbr & $106.9 \pm 0.6$ & $  7.1 \pm 0.7$& $ 6138 \pm   4$ & $-29.29 \pm   0.73$ & $-29.00 \pm   0.73$& $-28.51 \pm   0.73$& $-28.15 \pm   0.73$& $-27.38 \pm   0.73$ & $  1.12 \pm   0.14$\\
LSQ12fhe & $ 42.8 \pm 1.2$ & $  4.0 \pm 3.1$& $ 6108 \pm   4$ & $-29.76 \pm   0.02$ & $-29.40 \pm   0.02$& $-29.04 \pm   0.02$& $-28.74 \pm   0.02$& $-28.11 \pm   0.02$ & $  1.07 \pm   0.19$\\
LSQ12fxd & $122.9 \pm 1.7$ & $ 11.4 \pm 0.8$& $ 6119 \pm   4$ & $-29.62 \pm   0.07$ & $-29.39 \pm   0.07$& $-28.95 \pm   0.07$& $-28.64 \pm   0.07$& $-27.91 \pm   0.07$ & $  0.05 \pm   0.15$\\
LSQ12hjm & $ 82.6 \pm 17.5$ & $ 12.2 \pm 1.4$& $ 6144 \pm   5$ & $-29.51 \pm   0.02$ & $-29.14 \pm   0.01$& $-28.60 \pm   0.01$& $-28.30 \pm   0.01$& $-27.71 \pm   0.02$ & $ -0.50 \pm   0.21$\\
PTF09dlc & $143.5 \pm 2.2$ & $ 10.2 \pm 0.9$& $ 6143 \pm   3$ & $-29.38 \pm   0.01$ & $-29.17 \pm   0.01$& $-28.69 \pm   0.01$& $-28.40 \pm   0.01$& $-27.62 \pm   0.01$ & $  0.11 \pm   0.12$\\
PTF09dnl & $129.9 \pm 0.9$ & $  9.5 \pm 0.7$& $ 6093 \pm   3$ & $-29.23 \pm   0.01$ & $-29.07 \pm   0.01$& $-28.72 \pm   0.01$& $-28.44 \pm   0.01$& $-27.69 \pm   0.01$ & $  0.72 \pm   0.13$\\
PTF09dnp & $ 64.9 \pm 4.5$ & $ 16.5 \pm 0.7$& $ 6098 \pm   4$ & $-29.55 \pm   0.02$ & $-29.19 \pm   0.02$& $-28.68 \pm   0.02$& $-28.48 \pm   0.02$& $-27.93 \pm   0.02$ & $ -0.74 \pm   0.24$\\
PTF09fox & $117.6 \pm 2.7$ & $  9.1 \pm 1.0$& $ 6116 \pm   3$ & $-29.44 \pm   0.03$ & $-29.21 \pm   0.03$& $-28.72 \pm   0.03$& $-28.42 \pm   0.03$& $-27.68 \pm   0.03$ & $  0.68 \pm   0.27$\\
PTF09foz & $127.2 \pm 1.9$ & $ 21.7 \pm 1.2$& $ 6136 \pm   4$ & $-29.14 \pm   0.01$ & $-29.00 \pm   0.01$& $-28.59 \pm   0.01$& $-28.35 \pm   0.01$& $-27.65 \pm   0.01$ & $ -1.14 \pm   0.26$\\
PTF10hmv & $109.6 \pm 1.3$ & $  8.9 \pm 0.7$& $ 6143 \pm   3$ & $-28.54 \pm   0.01$ & $-28.40 \pm   0.01$& $-28.11 \pm   0.01$& $-27.89 \pm   0.01$& $-27.31 \pm   0.01$ & $  1.39 \pm   0.18$\\
PTF10icb & $104.8 \pm 0.9$ & $ 12.7 \pm 0.3$& $ 6138 \pm   3$ & $-28.58 \pm   0.02$ & $-28.36 \pm   0.02$& $-27.98 \pm   0.02$& $-27.77 \pm   0.02$& $-27.17 \pm   0.02$ & $  0.37 \pm   0.13$\\
PTF10mwb & $116.5 \pm 1.2$ & $ 19.8 \pm 0.9$& $ 6138 \pm   2$ & $-29.02 \pm   0.07$ & $-28.84 \pm   0.07$& $-28.40 \pm   0.07$& $-28.14 \pm   0.07$& $-27.52 \pm   0.07$ & $ -0.68 \pm   0.10$\\
PTF10ndc & $124.2 \pm 2.4$ & $  6.8 \pm 1.1$& $ 6119 \pm   3$ & $-29.52 \pm   0.01$ & $-29.25 \pm   0.01$& $-28.80 \pm   0.01$& $-28.49 \pm   0.01$& $-27.76 \pm   0.01$ & $  0.87 \pm   0.18$\\
PTF10ops & $ 38.7 \pm 9.9$ & $  7.2 \pm 8.7$& $ 6140 \pm   5$ & $-27.93 \pm   0.38$ & $-27.76 \pm   0.38$& $-27.73 \pm   0.38$& $-27.59 \pm   0.38$& $-27.21 \pm   0.38$ & $ -0.62 \pm   0.13$\\
PTF10qjq & $ 73.9 \pm 2.4$ & $ 12.8 \pm 0.8$& $ 6133 \pm   3$ & $-29.29 \pm   0.02$ & $-28.94 \pm   0.02$& $-28.53 \pm   0.01$& $-28.35 \pm   0.01$& $-27.76 \pm   0.01$ & $ -0.35 \pm   0.11$\\
PTF10qyz & $106.4 \pm 2.1$ & $ 23.0 \pm 1.0$& $ 6120 \pm   5$ & $-29.05 \pm   0.17$ & $-28.92 \pm   0.17$& $-28.41 \pm   0.17$& $-28.14 \pm   0.17$& $-27.30 \pm   0.17$ & $ -1.09 \pm   0.16$\\
PTF10tce & $135.7 \pm 1.1$ & $ 11.2 \pm 1.5$& $ 6090 \pm   4$ & $-29.13 \pm   0.02$ & $-28.99 \pm   0.01$& $-28.59 \pm   0.01$& $-28.31 \pm   0.01$& $-27.55 \pm   0.01$ & $  0.43 \pm   0.14$\\
PTF10ufj & $141.1 \pm 3.4$ & $ 11.7 \pm 1.2$& $ 6131 \pm   6$ & $-29.28 \pm   0.15$ & $-29.16 \pm   0.15$& $-28.72 \pm   0.15$& $-28.41 \pm   0.15$& $-27.65 \pm   0.15$ & $  0.53 \pm   0.16$\\
PTF10wnm & $105.8 \pm 2.3$ & $  6.5 \pm 1.0$& $ 6124 \pm   3$ & $-29.38 \pm   0.01$ & $-29.07 \pm   0.01$& $-28.68 \pm   0.01$& $-28.37 \pm   0.01$& $-27.69 \pm   0.01$ & $  0.63 \pm   0.16$\\
PTF10wof & $129.6 \pm 2.7$ & $ 17.3 \pm 1.0$& $ 6102 \pm   2$ & $-28.91 \pm   0.01$ & $-28.84 \pm   0.01$& $-28.46 \pm   0.01$& $-28.18 \pm   0.01$& $-27.43 \pm   0.01$ & $  0.16 \pm   0.13$\\
PTF10xyt & $123.7 \pm 6.6$ & $ 16.4 \pm 4.3$& $ 6101 \pm   4$ & $-28.26 \pm   0.02$ & $-28.20 \pm   0.02$& $-27.93 \pm   0.02$& $-27.74 \pm   0.02$& $-27.22 \pm   0.04$ & $  0.03 \pm   0.21$\\
PTF10zdk & $149.7 \pm 1.2$ & $ 14.3 \pm 0.6$& $ 6150 \pm   3$ & $-28.61 \pm   0.02$ & $-28.69 \pm   0.02$& $-28.32 \pm   0.02$& $-28.08 \pm   0.02$& $-27.40 \pm   0.02$ & $  0.18 \pm   0.17$\\
PTF11bgv & $ 79.4 \pm 3.2$ & $ 12.6 \pm 0.7$& $ 6146 \pm   3$ & $-28.90 \pm   0.02$ & $-28.62 \pm   0.01$& $-28.27 \pm   0.01$& $-28.08 \pm   0.01$& $-27.54 \pm   0.01$ & $ -0.33 \pm   0.12$\\
PTF11bju & $ 30.2 \pm 4.4$ & $  4.0 \pm 3.0$& $ 6139 \pm   5$ & $-29.47 \pm   0.02$ & $-29.10 \pm   0.01$& $-28.75 \pm   0.01$& $-28.45 \pm   0.01$& $-27.87 \pm   0.01$ & $  1.04 \pm   0.20$\\
PTF11bnx & $151.4 \pm 3.0$ & $ 13.9 \pm 1.1$& $ 6142 \pm   5$ & $-28.63 \pm   0.02$ & $-28.57 \pm   0.01$& $-28.20 \pm   0.01$& $-27.99 \pm   0.01$& $-27.34 \pm   0.01$ & $ -0.07 \pm   0.17$\\
PTF11cao & $143.3 \pm 1.6$ & $ 18.9 \pm 1.3$& $ 6104 \pm   5$ & $-28.78 \pm   0.02$ & $-28.79 \pm   0.02$& $-28.44 \pm   0.02$& $-28.18 \pm   0.02$& $-27.45 \pm   0.02$ & $  0.18 \pm   0.15$\\
PTF11drz & $132.6 \pm 1.4$ & $ 15.2 \pm 1.0$& $ 6116 \pm   5$ & $-29.12 \pm   0.01$ & $-28.95 \pm   0.01$& $-28.53 \pm   0.01$& $-28.27 \pm   0.01$& $-27.55 \pm   0.01$ & $  0.21 \pm   0.37$\\
PTF11mkx & $ 31.5 \pm 3.7$ & $  4.5 \pm 1.3$& $ 6169 \pm   5$ & $-29.50 \pm   0.45$ & $-29.25 \pm   0.45$& $-28.89 \pm   0.45$& $-28.61 \pm   0.45$& $-27.97 \pm   0.45$ & $  0.60 \pm   0.15$\\
PTF11mty & $111.4 \pm 2.3$ & $ 10.6 \pm 1.5$& $ 6138 \pm   5$ & $-29.54 \pm   0.01$ & $-29.23 \pm   0.01$& $-28.80 \pm   0.01$& $-28.46 \pm   0.01$& $-27.82 \pm   0.01$ & $  0.83 \pm   0.16$\\
PTF11pdk & $128.6 \pm 2.8$ & $ 15.6 \pm 1.7$& $ 6153 \pm   5$ & $-29.35 \pm   0.02$ & $-29.11 \pm   0.02$& $-28.61 \pm   0.02$& $-28.32 \pm   0.02$& $-27.67 \pm   0.02$ & $  0.06 \pm   0.19$\\
PTF11qmo & $101.7 \pm 1.1$ & $  7.7 \pm 0.7$& $ 6150 \pm   8$ & $-29.77 \pm   0.02$ & $-29.43 \pm   0.02$& $-28.97 \pm   0.02$& $-28.64 \pm   0.02$& $-27.93 \pm   0.02$ & $ -0.00 \pm   0.34$\\
PTF12dxm & $ 95.4 \pm 41.8$ & $ 35.7 \pm 2.8$& $ 6136 \pm   4$ & $-28.71 \pm   0.01$ & $-28.58 \pm   0.01$& $-28.19 \pm   0.01$& $-27.99 \pm   0.01$& $-27.34 \pm   0.01$ & $ -1.98 \pm   0.20$\\
PTF12eer & $165.6 \pm 10.7$ & $ 12.7 \pm 2.8$& $ 6150 \pm  10$ & $-28.76 \pm   0.01$ & $-28.76 \pm   0.01$& $-28.40 \pm   0.01$& $-28.17 \pm   0.01$& $-27.45 \pm   0.02$ & $  0.68 \pm   0.16$\\
PTF12efn & $144.9 \pm 3.4$ & $  7.1 \pm 1.8$& $ 6115 \pm   3$ & $-29.40 \pm   0.01$ & $-29.17 \pm   0.01$& $-28.79 \pm   0.01$& $-28.45 \pm   0.01$& $-27.64 \pm   0.01$ & $  1.21 \pm   0.20$\\
PTF12ena & $101.1 \pm 1.6$ & $  7.4 \pm 1.0$& $ 6129 \pm   4$ & $-28.01 \pm   0.01$ & $-28.00 \pm   0.01$& $-27.85 \pm   0.01$& $-27.77 \pm   0.01$& $-27.31 \pm   0.01$ & $  0.65 \pm   0.16$\\
PTF12evo & $129.2 \pm 2.8$ & $  9.1 \pm 1.3$& $ 6156 \pm   4$ & $-29.14 \pm   0.02$ & $-28.98 \pm   0.01$& $-28.56 \pm   0.01$& $-28.28 \pm   0.01$& $-27.61 \pm   0.01$ & $  0.74 \pm   0.15$\\
PTF12fuu & $105.5 \pm 3.0$ & $  6.2 \pm 1.2$& $ 6124 \pm   5$ & $-29.54 \pm   0.01$ & $-29.23 \pm   0.01$& $-28.74 \pm   0.01$& $-28.40 \pm   0.01$& $-27.64 \pm   0.01$ & $  0.34 \pm   0.12$\\
PTF12ghy & $ 99.3 \pm 3.6$ & $ 16.8 \pm 0.7$& $ 6134 \pm   3$ & $-28.29 \pm   0.02$ & $-28.27 \pm   0.01$& $-28.05 \pm   0.01$& $-27.95 \pm   0.01$& $-27.40 \pm   0.01$ & $ -0.48 \pm   0.12$\\
PTF12grk & $162.3 \pm 9.8$ & $ 19.6 \pm 1.4$& $ 6085 \pm   8$ & $-28.86 \pm   0.02$ & $-28.87 \pm   0.01$& $-28.42 \pm   0.01$& $-28.19 \pm   0.01$& $-27.50 \pm   0.03$ & $ -1.17 \pm   0.13$\\
PTF12hwb & $ 21.1 \pm 78.0$ & $ -1.8 \pm 8.9$& $ 6090 \pm  14$ & $-28.32 \pm   0.02$ & $-28.24 \pm   0.02$& $-28.03 \pm   0.02$& $-27.79 \pm   0.02$& $-27.05 \pm   0.04$ & $  1.34 \pm   0.17$\\
PTF12iiq & $150.4 \pm 2.2$ & $ 22.5 \pm 0.8$& $ 6041 \pm   6$ & $-28.60 \pm   0.01$ & $-28.77 \pm   0.01$& $-28.41 \pm   0.01$& $-28.10 \pm   0.01$& $-27.29 \pm   0.01$ & $ -1.30 \pm   0.12$\\
PTF12ikt & $110.3 \pm 1.6$ & $ 14.2 \pm 0.7$& $ 6141 \pm   4$ & $-29.34 \pm   0.01$ & $-29.04 \pm   0.01$& $-28.57 \pm   0.01$& $-28.32 \pm   0.01$& $-27.66 \pm   0.01$ & $ -0.35 \pm   0.11$\\
PTF12jqh & $151.9 \pm 1.5$ & $  7.9 \pm 0.7$& $ 6116 \pm  10$ & $-29.37 \pm   0.01$ & $-29.14 \pm   0.01$& $-28.71 \pm   0.01$& $-28.40 \pm   0.01$& $-27.64 \pm   0.01$ & $  0.69 \pm   0.19$\\
PTF13ajv & $150.5 \pm 8.9$ & $ 46.3 \pm 8.6$& $ 6110 \pm  21$ & $-28.70 \pm   0.02$ & $-28.61 \pm   0.02$& $-28.16 \pm   0.02$& $-27.91 \pm   0.02$& $-27.07 \pm   0.04$ & $ -1.59 \pm   0.22$\\
PTF13anh & $166.8 \pm 1.8$ & $ 21.8 \pm 1.2$& $ 6175 \pm   4$ & $-28.67 \pm   0.20$ & $-28.74 \pm   0.20$& $-28.30 \pm   0.20$& $-28.05 \pm   0.20$& $-27.28 \pm   0.20$ & $ -0.28 \pm   0.17$\\
PTF13asv & $ 75.6 \pm 1.1$ & $  2.2 \pm 0.4$& $ 6148 \pm   4$ & $-29.92 \pm   0.32$ & $-29.49 \pm   0.32$& $-29.02 \pm   0.32$& $-28.63 \pm   0.32$& $-27.90 \pm   0.32$ & $  0.50 \pm   0.11$\\
PTF13ayw & $104.6 \pm 2.4$ & $ 26.6 \pm 3.2$& $ 6115 \pm   6$ & $-29.16 \pm   0.02$ & $-28.82 \pm   0.02$& $-28.43 \pm   0.02$& $-28.20 \pm   0.02$& $-27.55 \pm   0.02$ & $ -1.72 \pm   0.29$\\
PTF13azs & $138.0 \pm 5.1$ & $ 16.2 \pm 1.6$& $ 6125 \pm  10$ & $-27.84 \pm   0.02$ & $-27.92 \pm   0.02$& $-27.69 \pm   0.02$& $-27.60 \pm   0.02$& $-26.99 \pm   0.02$ & $ -0.02 \pm   0.12$\\
SN2003bt & $ 92.1 \pm 16.5$ & $ 21.1 \pm 0.7$& $ 6121 \pm   2$ & $-29.28 \pm   0.02$ & $-28.95 \pm   0.02$& $-28.46 \pm   0.02$& $-28.20 \pm   0.02$& $-27.62 \pm   0.04$ & $ -2.08 \pm   0.18$\\
SN2005cf & $159.1 \pm 0.7$ & $ 15.7 \pm 0.8$& $ 6141 \pm   3$ & $-29.37 \pm   0.02$ & $-29.16 \pm   0.02$& $-28.68 \pm   0.02$& $-28.41 \pm   0.02$& $-27.69 \pm   0.02$ & $ -0.31 \pm   0.17$\\
SN2005hc & $126.9 \pm 2.5$ & $ 10.0 \pm 0.7$& $ 6123 \pm   3$ & $-29.38 \pm   0.01$ & $-29.13 \pm   0.01$& $-28.69 \pm   0.01$& $-28.37 \pm   0.01$& $-27.61 \pm   0.01$ & $  0.87 \pm   0.31$\\
SN2005hj & $ 80.8 \pm 2.4$ & $  4.3 \pm 0.8$& $ 6138 \pm   4$ & $-29.54 \pm   0.02$ & $-29.16 \pm   0.01$& $-28.87 \pm   0.01$& $-28.54 \pm   0.01$& $-28.01 \pm   0.01$ & $  1.85 \pm   0.59$\\
SN2005ir & $115.6 \pm 2.8$ & $ 13.5 \pm 6.9$& $ 6069 \pm   5$ & $-29.33 \pm   0.02$ & $-29.12 \pm   0.02$& $-28.84 \pm   0.02$& $-28.49 \pm   0.02$& $-27.77 \pm   0.02$ & $  1.71 \pm   0.69$\\
SN2006cj & $101.7 \pm 1.3$ & $  4.8 \pm 0.8$& $ 6127 \pm   3$ & $-29.43 \pm   0.01$ & $-29.14 \pm   0.01$& $-28.74 \pm   0.01$& $-28.43 \pm   0.01$& $-27.76 \pm   0.01$ & $  0.48 \pm   0.18$\\
SN2006dm & $ 99.5 \pm 1.6$ & $ 30.0 \pm 0.7$& $ 6118 \pm   3$ & $-28.81 \pm   0.01$ & $-28.65 \pm   0.01$& $-28.23 \pm   0.01$& $-28.02 \pm   0.01$& $-27.33 \pm   0.01$ & $ -1.88 \pm   0.34$\\
SN2006do & $106.4 \pm 2.1$ & $ 26.7 \pm 1.3$& $ 6101 \pm   2$ & $-29.00 \pm   0.01$ & $-28.83 \pm   0.01$& $-28.42 \pm   0.01$& $-28.20 \pm   0.01$& $-27.53 \pm   0.04$ & $ -2.32 \pm   0.54$\\
SN2006ob & $ 90.0 \pm 16.5$ & $ 26.5 \pm 1.5$& $ 6112 \pm   5$ & $-29.11 \pm   0.02$ & $-28.82 \pm   0.01$& $-28.42 \pm   0.01$& $-28.19 \pm   0.01$& $-27.54 \pm   0.01$ & $ -2.45 \pm   0.28$\\
SN2007bd & $109.7 \pm 5.9$ & $ 17.5 \pm 0.7$& $ 6101 \pm   3$ & $-29.31 \pm   0.01$ & $-29.12 \pm   0.01$& $-28.60 \pm   0.01$& $-28.35 \pm   0.01$& $-27.60 \pm   0.01$ & $ -0.85 \pm   0.34$\\
SN2007cq & $ 65.8 \pm 4.1$ & $ 10.2 \pm 0.9$& $ 6137 \pm   3$ & $-29.53 \pm   0.02$ & $-29.30 \pm   0.02$& $-28.89 \pm   0.02$& $-28.56 \pm   0.02$& $-27.90 \pm   0.02$ & $ -0.43 \pm   0.16$\\
SN2007kk & $128.5 \pm 1.4$ & $ 10.6 \pm 1.0$& $ 6098 \pm   4$ & $-29.48 \pm   0.02$ & $-29.31 \pm   0.02$& $-28.87 \pm   0.01$& $-28.54 \pm   0.01$& $-27.77 \pm   0.02$ & $  0.52 \pm   0.18$\\
SN2007nq & $ 89.8 \pm 9.9$ & $ 23.4 \pm 1.1$& $ 6109 \pm   5$ & $-29.11 \pm   0.02$ & $-28.91 \pm   0.02$& $-28.50 \pm   0.02$& $-28.27 \pm   0.02$& $-27.57 \pm   0.02$ & $ -1.89 \pm   0.26$\\
SN2008ec & $103.7 \pm 17.0$ & $ 23.1 \pm 0.4$& $ 6125 \pm   3$ & $-28.67 \pm   0.01$ & $-28.52 \pm   0.01$& $-28.18 \pm   0.01$& $-28.03 \pm   0.01$& $-27.47 \pm   0.01$ & $ -1.05 \pm   0.12$\\
SN2010dt & $116.2 \pm 14.9$ & $ 15.5 \pm 0.7$& $ 6138 \pm   6$ & $-29.30 \pm   0.01$ & $-29.15 \pm   0.01$& $-28.64 \pm   0.01$& $-28.35 \pm   0.01$& $-27.63 \pm   0.01$ & $ -0.03 \pm   0.10$\\
SN2010ex & $114.4 \pm 0.9$ & $  8.4 \pm 0.4$& $ 6129 \pm   6$ & $-29.26 \pm   0.01$ & $-28.99 \pm   0.01$& $-28.50 \pm   0.01$& $-28.20 \pm   0.01$& $-27.44 \pm   0.01$ & $  0.27 \pm   0.14$\\
SN2010kg & $ 95.1 \pm 28.5$ & $ 21.7 \pm 0.7$& $ 6077 \pm   5$ & $-28.85 \pm   0.01$ & $-28.74 \pm   0.01$& $-28.41 \pm   0.01$& $-28.20 \pm   0.01$& $-27.47 \pm   0.01$ & $ -2.26 \pm   0.40$\\
SN2011bc & $127.3 \pm 2.4$ & $ 15.5 \pm 1.2$& $ 6152 \pm   4$ & $-28.77 \pm   0.01$ & $-28.66 \pm   0.01$& $-28.37 \pm   0.01$& $-28.15 \pm   0.01$& $-27.52 \pm   0.01$ & $  0.01 \pm   0.12$\\
SN2011be & $ 94.1 \pm 2.0$ & $ 11.2 \pm 0.6$& $ 6132 \pm   4$ & $-29.50 \pm   0.02$ & $-29.11 \pm   0.02$& $-28.67 \pm   0.02$& $-28.37 \pm   0.02$& $-27.71 \pm   0.02$ & $  0.20 \pm   0.24$\\
SN2011ho & $ 88.0 \pm 2.7$ & $ 32.2 \pm 1.9$& $ 6111 \pm   7$ & $-28.37 \pm   0.03$ & $-28.26 \pm   0.03$& $-27.92 \pm   0.03$& $-27.77 \pm   0.03$& $-27.13 \pm   0.03$ & $ -1.83 \pm   0.18$\\
SN2011hr & $ 39.0 \pm 22.2$ & $  4.5 \pm 0.2$& $ 6139 \pm   8$ & $-29.46 \pm   0.02$ & $-29.06 \pm   0.02$& $-28.75 \pm   0.02$& $-28.49 \pm   0.02$& $-27.93 \pm   0.02$ & $  0.98 \pm   0.18$\\
SN2011ia & $ 85.5 \pm 0.6$ & $  5.9 \pm 0.3$& $ 6130 \pm   4$ & $-29.87 \pm   0.02$ & $-29.45 \pm   0.02$& $-28.94 \pm   0.02$& $-28.57 \pm   0.02$& $-27.85 \pm   0.02$ & $  0.22 \pm   0.26$\\
SN2012cg & $109.0 \pm 0.3$ & $  8.6 \pm 0.1$& $ 6138 \pm   2$ & $-28.35 \pm   0.01$ & $-28.15 \pm   0.01$& $-27.79 \pm   0.01$& $-27.58 \pm   0.01$& $-26.97 \pm   0.01$ & $  0.37 \pm   0.12$\\
SN2012fr & $134.2 \pm 0.5$ & $  7.4 \pm 0.2$& $ 6102 \pm   1$ & $-29.91 \pm   0.01$ & $-29.70 \pm   0.01$& $-29.31 \pm   0.01$& $-28.94 \pm   0.01$& $-28.10 \pm   0.01$ & $  0.96 \pm   0.10$\\
SN2013be & $102.7 \pm 1.8$ & $ 11.9 \pm 1.0$& $ 6142 \pm   5$ & $-29.28 \pm   0.02$ & $-29.14 \pm   0.02$& $-28.74 \pm   0.02$& $-28.46 \pm   0.01$& $-27.76 \pm   0.03$ & $  0.35 \pm   0.19$\\
SN2013bs & $ 87.0 \pm 1.4$ & $ 20.7 \pm 0.7$& $ 6136 \pm   3$ & $-28.78 \pm   0.01$ & $-28.66 \pm   0.01$& $-28.30 \pm   0.01$& $-28.08 \pm   0.01$& $-27.41 \pm   0.01$ & $ -2.09 \pm   0.25$\\
SNBOSS38 & $ 57.1 \pm 0.4$ & $ 17.9 \pm 0.3$& $ 6127 \pm   3$ & $-29.20 \pm   0.01$ & $-28.84 \pm   0.01$& $-28.47 \pm   0.01$& $-28.23 \pm   0.01$& $-27.73 \pm   0.04$ & $  1.10 \pm   0.11$\\
SNNGC0927 & $155.2 \pm 1.4$ & $ 11.0 \pm 0.7$& $ 6109 \pm   4$ & $-28.87 \pm   0.02$ & $-28.81 \pm   0.01$& $-28.46 \pm   0.01$& $-28.22 \pm   0.01$& $-27.48 \pm   0.01$ & $  0.41 \pm   0.14$\\
\enddata
\end{deluxetable}

The distributions of the central values of the inputs  $EW_{\mathrm{Ca}}$, $EW_{\mathrm{Si}}$,
$\lambda_{\mathrm{Si}}$ , $x_1$, and the SALT2 color parameter $C$ are shown in Table~\ref{input:fig}.
For reference, the $x_1$ and $C$ distributions of the $0.03<z< 0.08$ Pantheon sample,
predominantly supernovae from CFA3 \citep{2009ApJ...700..331H}, CFA4 \citep{2012ApJS..200...12H}, and CSP
\citep{2010AJ....139..519C, 2011AJ....142..156S}, that were used as part of the low-redshift anchor in the recent cosmology analysis of \citet{2017arXiv171000845S}.

\begin{figure}[htbp] %  figure placement: here, top, bottom, or page
   \centering
   \includegraphics[width=4in]{perobject_input.pdf}
   \caption{\added{Distributions of the central values of the inputs  $EW_{\mathrm{Ca}}$, $EW_{\mathrm{Si}}$,
$\lambda_{\mathrm{Si}}$ , $x_1$, and $C$ of the sample used in this article, and
the $x_1$ and $C$ distributions of the $0.03<z< 0.08$ Pantheon sample.}
   \label{input:fig}}
\end{figure}
}

\section{Model~I: Two Color Parameters}
\label{modelI:sec}
We begin by considering Model~I, in which supernova magnitudes  have linear dependence on spectral parameters
and two color parameters. The purpose of this section is to introduce some basic concepts of our models,
to show how the two  color terms can be associated with existing dust-extinction models, and to give Model~I results on dust extinction.
Deficiencies of this model are addressed in the subsequent models presented in the article.
To avoid repetitiveness,
we refrain from going into full detail on the results of Model~I in deference to Model~III, which is presented in Section~\ref{modelIII:sec}.

\added{It is important to note that the model and those in subsequent sections are applied to our supernova sample;
the ensuing results thus correspond to our SN~Ia sample's underlying population. }

\subsection{Model}
We assume 
that  peak
underlying
 ${\hat{U}}{\hat{B}}\hat{V}\hat{R}\hat{I}$ magnitudes of a supernova  with effects from  astrophysical foregrounds are linearly dependent
on its
 equivalent widths of the Ca~II H\&K and Si~II~$\lambda$4141 spectral features
$EW_{\mathrm{Ca}}$ and $EW_{\mathrm{Si}}$,
and the wavelength of the minimum of 
the Si~II~$\lambda6355$ feature $\lambda_{\mathrm{Si}}$
around $B$-band peak brightness:
these spectral features are associated with SN~Ia  spectroscopic diversity  
\citep{2005ApJ...623.1011B, 2006PASP..118..560B, 2008A&A...492..535A, 2009A&A...500L..17B, 2009PASP..121..238B, 2009ApJ...699L.139W, 2011ApJ...729...55F,2012AJ....143..126B}.
The explicit omission of light-curve shape in our model is compensated by its proxy,
$EW_{\mathrm{Si}}$, at peak brightness
\citep{2008A&A...492..535A, 2011A&A...529L...4C}. 
The underlying magnitudes are also linearly dependent on the per-supernova latent parameters $g_0$ and $g_1$.
Unlike
the spectral parameters ($EW_{\mathrm{Ca}}$, $EW_{\mathrm{Si}}$ and $\lambda_{\mathrm{Si}}$),  $g_0$ and $g_1$ are not directly associated
with observables but rather are inferred as part of the analysis.
A grey magnitude offset, $\Delta$, is included for each supernova
to capture 
potential
band-independent intrinsic dispersion, while also absorbing peculiar-velocity errors introduced when converting
fluxes to luminosities.
With this grey offset the model standardizes colors, not absolute magnitude.
The underlying magnitudes are
\replaced{\begin{equation}
\begin{pmatrix}
{\hat{U}}\\{\hat{B}}\\{\hat{V}}\\{\hat{R}}\\{\hat{I}}
\end{pmatrix}
=
\Delta \vec{1} +
\begin{pmatrix}
c_{\hat{U}}+\alpha_{\hat{U}} EW_{Ca} + \beta_{\hat{U}} EW_{Si} + \eta_{\hat{U}} \lambda_{Si}  +\gamma^0_{{\hat{U}}} g_0 +\gamma^1_{{\hat{U}}} g_1 \\
c_{\hat{B}}+\alpha_{\hat{B}} EW_{Ca} + \beta_{\hat{B}} EW_{Si} + \eta_{\hat{B}} \lambda_{Si}  +\gamma^0_{{\hat{B}}} g_0 +\gamma^1_{{\hat{B}}} g_1 \\
c_{\hat{V}}+\alpha_{\hat{V}} EW_{Ca} + \beta_{\hat{V}} EW_{Si} + \eta_{\hat{V}} \lambda_{Si} +\gamma^0_{{\hat{V}}} g_0 +\gamma^1_{{\hat{V}}} g_1 \\
c_{\hat{R}}+\alpha_{\hat{R}} EW_{Ca} + \beta_{\hat{R}} EW_{Si} + \eta_{\hat{R}} \lambda_{Si}  +\gamma^0_{{\hat{R}}} g_0 +\gamma^1_{{\hat{R}}} g_1 \\
c_{\hat{I}}+\alpha_{\hat{I}} EW_{Ca} + \beta_{\hat{I}} EW_{Si}+ \eta_{\hat{I}} \lambda_{Si}  +\gamma^0_{{\hat{I}}} g_0 +\gamma^1_{{\hat{I}}} g_1
\end{pmatrix}.
\label{ewsiv:eqn}
\end{equation}}{
\begin{equation}
\vec{m}=
 \vec{c} +  \vec{1} \, \Delta +
\vec{\alpha} \,  EW_{\mathrm{Ca}}+
\vec{\beta}  \, EW_{\mathrm{Si}} +
\vec{\eta}  \, \lambda_{\mathrm{Si}} +
\vec{\gamma}^0  \,  g_0 +
\vec{\gamma}^1  \,  g_1,
\label{ewsiv:eqn}
\end{equation}
where $\vec{m} = \{ \hat{U}, \hat{B}, \hat{V}, \hat{R}, \hat{I} \}$ and}
$\vec{1}$ is the Identity vector.
The global parameter vectors\footnote{Global parameters act on each of the five bands: a single component is written with a subscript for the corresponding band, e.g.\  $c_{\hat{U}}$; 
the combination of all components are written as a vector, e.g.\ $\vec{c}=(c_{\hat{U}}, c_{\hat{B}}, c_{\hat{V}}, c_{\hat{R}}, c_{\hat{I}}) $.  The
band subscript or arrow
notationally identify global parameters, in contrast with the per-supernova parameters.
} that describe the SN~Ia population
\replaced{
$\vec{c}$  (units of mag);
$\vec{\alpha}$, $\vec{\beta}$,
$\vec{\eta}$ (units of mag~\AA$^{-1}$), $\vec{\gamma}^0$ and $\vec{\gamma}^1$ (units of mag)
 are the intercept
and slopes
of the linear relationships that
relate
per-supernova
parameters with 
underlying magnitudes.}
{
are the intercept $\vec{c}$  (units of mag),
and slopes $\vec{\alpha}$, $\vec{\beta}$,
$\vec{\eta}$ (units of mag~\AA$^{-1}$), $\vec{\gamma}^0$ and $\vec{\gamma}^1$ (units of mag)
of the linear relationships that
relate
per-supernova
parameters with 
underlying magnitudes.
}
The notation used in this article is not to be associated with those from previous works.


The observables
$\hat{U}_o, {\hat{B}}_o, {\hat{V}}_o, {\hat{R}}_o, {\hat{I}}_o$, $EW_{Ca,o}$, $EW_{Si,o}$, $\lambda_{Si,o}$
have Gaussian measurement uncertainty with covariance $C$.
The
likelihood density for the described model
corresponds to the probability distribution function described by
\begin{equation}
\begin{pmatrix}
{\hat{U}}_o\\{\hat{B}}_o\\ {\hat{V}}_o\\{\hat{R}}_o\\{\hat{I}}_o\\EW_{Si, o}\\ EW_{Ca, o} \\ \lambda_{Si, o}
\end{pmatrix}
\sim \mathcal{N}
\left(
\begin{pmatrix}
{\hat{U}}  \\{\hat{B}}  \\
{\hat{V}}\\{\hat{R}}\\{\hat{I}}\\
EW_{\mathrm{Si}}\\ EW_{\mathrm{Ca}} \\ \lambda_{\mathrm{Si}}
\end{pmatrix}
,C
\right).
\label{dust:eqn}
\end{equation}


The model as written has degeneracies that need to be constrained in order for fits to converge.
The model does not specify the absolute magnitude nor the 
color of a $g_0=0$, $g_1=0$ supernova;
in their place the zeropoints of the per-supernova magnitude and color parameters are set to the sample means
\begin{equation}
\langle \Delta \rangle=0,\ \langle g_0 \rangle=0,\ \langle g_1 \rangle=0.
\label{zero:eqn}
\end{equation}
The model contains the product of parameters
$\vec{\gamma} g$, which leads to the degeneracy $\vec{\gamma} \rightarrow a\vec{\gamma}$, $g \rightarrow a^{-1} g$.
To aid in the convergence of $\vec{\gamma}$ and $g$ we impose a prior on the rms of $g$, though
our physical interpretations are ultimately independent of this scaling.
This prior does not specify the sign of $a$, which leaves a parity degeneracy. 
Results in this article are given in terms of $\vec{\gamma}g$, not $g$ alone.
As will be seen, the signal-to-noise in $\vec{\gamma}$ is sufficiently
high that our finite MCMC chains do not migrate between the degenerate solutions; to simplify the merging of multiple chains
we impose one of the degenerate solutions
\begin{equation}
\gamma^0_{\hat{U}} > 0,\ \gamma^1_{\hat{U}} < 0.
\end{equation}
As  in mixture models, the combinations $\vec{\gamma}^0 g_0$ and $\vec{\gamma}^1 g_1$ terms are degenerate under exchange of the 0 and 1 indices. For a cleaner presentation of our results, 
we break that degeneracy  by setting consistent initial conditions that succeed in keeping indices consistent over all analysis chains.  

Each of the multiplicative cofactors has a degeneracy with the zeropoint.  Consider
$\vec{\alpha} \rightarrow \vec{\alpha} +   \epsilon_{\alpha}$,
$\vec{\beta} \rightarrow \vec{\beta} +   \epsilon_{\beta}$,
$\vec{\eta} \rightarrow \vec{\eta} +   \epsilon_{\eta}$,
$\vec{\gamma}^0 \rightarrow \vec{\gamma}^0 +   \epsilon_{\gamma^0}$,
$\vec{\gamma}^1 \rightarrow \vec{\gamma}^1 +   \epsilon_{\gamma^1}$.
Then
each supernova's grey term can shift as
$\Delta \rightarrow \Delta  + \epsilon_\Delta$
with
$$
\epsilon_\Delta  = -  \epsilon_{\alpha} (EW_{\text{Ca}}-\langle EW_{\text{Ca}}\rangle)- 
 \epsilon_{\beta} (EW_{\mathrm{Si}}-\langle EW_{\mathrm{Si}}\rangle) -  \epsilon_{\eta} (\lambda_{\mathrm{Si}}-\langle \lambda_{\mathrm{Si}}\rangle)
- \epsilon_{\gamma^0}g_0 -  \epsilon_{\gamma^1}g_1.
$$
We break this degeneracy by fixing the values of $\epsilon$ such that $\Delta$ is
uncorrelated with  $EW_{\text{Ca}}$, $EW_{\mathrm{Si}}$, $\lambda_{\text{Ca}}$,
$g_0$, and $g_1$ reflecting
our preference for feature affects to be confined to the cofactors and not the grey magnitude offsets.

All parameters not  otherwise noted above have flat priors.

For $N$ supernovae there are $8N$ observables.  There are $3N$ spectral parameters, each of
$\Delta$, $g_0$, $g_1$ contributes $N-1$ parameters
(recall the constraint in Eq.~\ref{zero:eqn}),  and there are $5 \times 7$ global coefficients.
For $N=172$ supernovae, there are 1376 observables and 1064  parameters.


\subsection{Results for the Latent Parameter Cofactors  $\vec{\gamma}^0$ and $\vec{\gamma}^1$  }
\label{results1:sec}
\added{The MCMC chains for all parameters from this and subsequent sections are accessible from the SNfactory website\footnote{For the referee the parameters
are available at \url{https://drive.google.com/open?id=1d52nxfF7eS3HblfbmT3fxgiZod_Z-Gjg}. Concurrent with publication the chain will be
made available
at \url{https://snfactory.lbl.gov/snf/data/index.html}.}.}
In this subsection we concentrate on the results for $\vec{\gamma}^0$ and $\vec{\gamma}^1$.
In our model, $g_0$ and $g_1$ are introduced as latent parameters that are related to linear shifts in band magnitudes through
the  coefficients $\vec{\gamma}^0$ and
$\vec{\gamma}^1$.  Otherwise, no physically-motivated suppositions on the properties of $g$ and $\vec{\gamma}$ are imposed.
Since dust extinction is a dominant determinant of supernova colors that is often described by a 2-parameter linear model, we expect
for our two  color terms to capture the effects of dust.
In this section we reexpress our model $\vec{\gamma}$'s in terms of
the dust-extinction model of \citet{1999PASP..111...63F} \citepalias[henceforth referred to as][]{1999PASP..111...63F}.

The 68\% credible intervals for $\vec{\gamma}^0$ and $\vec{\gamma}^1$ (and all other global parameters) are given in \S\ref{m12results:sec}.
All elements of the two $\vec{\gamma}$ parameter vectors 
describing color
are significantly non-zero.
None of the 20000 links of 
our Monte Carlo chains for $\vec{\gamma}$ come close to the origin.   We claim 
a probability $(1-5\times 10^{-5})$-detection of
a magnitude term in the form of a 5-dimensional vector spanned by two basis vectors
\begin{equation}
\vec{A} = \vec{\gamma}^0 g_0 +  \vec{\gamma}^1 g_1.
\end{equation}
All possible magnitude changes in our five bands due to $g_0$ and $g_1$ are confined to a two-dimensional
plane defined by $\vec{\gamma}^0$ and  $\vec{\gamma}^1$. 

Dust-extinction
models are also approximately described by a 2-parameter linear model, here 
using the commonly-used parameters $A_V$ and $E(B-V)$
\begin{equation}
\vec{A} =\vec{ a}  A_{V} + \vec{b} E(B-V).
\label{f99:eqn}
\end{equation}

For example,   the wavelength-dependent model of  \citet{1989ApJ...345..245C} is linear,
though its linearity is lost when integrated over broad-band filters.
For the case of
$R^F=2.5$ and $A^F_V=0.1$ dust attenuating light from the SALT2
\citep{2007A&A...466...11G} $s=1$, $x_1=0$ SN~Ia template at ${\hat{B}}$-band peak, the 
\citetalias{1999PASP..111...63F} model
can be approximated with
$\vec{a} = (0.96,   1.00,   1.00,   0.97,   0.77)$ and $\vec{b}=(  1.77,   0.98,   0.12,  -0.50,  -0.53)$.
(The $F$ superscript is used to distinguish parameters of the \citetalias{1999PASP..111...63F} model.)
Over the ranges
 $0\le A^F_V\le 1$ and $2 \le R^F \le 3.5$ 
and the wavelengths under consideration here,  the values of the elements of $a$ and $b$ vary by $<5$\%
for the
\citetalias{1999PASP..111...63F} model with
 the largest error in extinction from the linear approximation having amplitude $<0.008$ mag.
All possible magnitude changes in our five bands due to  \citetalias{1999PASP..111...63F}  dust extinction are confined to a two-dimensional
plane defined by $\vec{a}$ and  $\vec{b}$. 


The credible interval for $\vec{\gamma}$'s and other parameters are presented in Table~\ref{global1:tab}.
Our model $\vec{\gamma}$ vectors can be written as a linear combination (described by the $2 \times 2$ matrix M) of the
dust-based $\vec{a}$ and $\vec{b}$ vectors plus residual vectors perpendicular to the plane spanned by $\vec{a}$ and $\vec{b}$,
\begin{equation}
\begin{pmatrix}
\vec{\gamma}^0 \\
\vec{\gamma}^1
\end{pmatrix}=
M
\begin{pmatrix}
\vec{a} \\
\vec{b}
\end{pmatrix}+
\begin{pmatrix}
\vec{\epsilon}_{\gamma^0} \\
\vec{\epsilon}_{\gamma^1}
\end{pmatrix}.
\label{trans_I:eqn}
\end{equation}
We find
\begin{equation}
M=
\begin{pmatrix}
\begin{array}{rr}
43.7^{+2.3}_{-2.1} & 19.4^{+0.5}_{-0.5} \\
-18.7^{+3.0}_{-3.1} & -0.8^{+0.9}_{-1.0} \\
\end{array}
\end{pmatrix} ,
\end{equation}
\begin{align}
\begin{split}
\vec{\epsilon}_{\gamma^0} &=\left(1.5^{+0.3}_{-0.3} , -1.4^{+0.3}_{-0.3} , -1.8^{+0.4}_{-0.3} , 1.4^{+0.1}_{-0.1} , 0.5^{+0.4}_{-0.5} \right) \text{mag}, \\
\vec{\epsilon}_{\gamma^1} & =\left(4.8^{+0.3}_{-0.3} , -4.3^{+0.3}_{-0.3} , -5.7^{+0.4}_{-0.4} , 0.1^{+0.2}_{-0.2} , 6.8^{+0.5}_{-0.5}\right) \text{mag},
\end{split}
\label{res_1:eqn}
\end{align}
which correspond to
a quadratic fraction of $0.9993^{+0.0002}_{-0.0003}$ of $\vec{\gamma}^0$ and
$0.9303^{+0.0210}_{-0.0341}$ of $\vec{\gamma}^1$ projecting onto the $\vec{a}$--$\vec{b}$ plane.
The  contributions of these residuals to magnitudes are relatively small compared to
the values of $\vec{\gamma}^0$ and $\vec{\gamma}^1$ in Table~\ref{global1:tab}.
For the standard deviation values of $g_0$ and $g_1$ of the supernovae in our sample, the amplitude of the magnitude residuals due to $\vec{\epsilon}$ are 
$\left(0.025, 0.023,0.029, 0.008, 0.027\right)$ mag.
The allowed color
variations in ${\hat{U}}{\hat{B}}{\hat{V}}{\hat{R}}{\hat{I}}$ allowed by the \citetalias{1999PASP..111...63F} model and our best-fit model are confined to almost identical
2-dimensional
planes within
the 5-dimensional magnitude space.
With no prior assumptions of dust extinction behavior or the distribution of $A_V$, the supernova data themselves exhibit
2-dimensional color variations that are closely aligned with the 2-dimensional color variations predicted by the \citetalias{1999PASP..111...63F} dust model.


The above result is visualized in Figure~\ref{plane:fig}, which  shows  in the
${\hat{U}}{\hat{V}}\hat{I}$-subspace
two perspectives (left and right panels)
of the unit vectors corresponding to $\vec{\gamma}^0$ and  $\vec{\gamma}^1$ of our model (solid lines),
and $\vec{a}$, $\vec{b}$ of the \citetalias{1999PASP..111...63F} model (dashed lines).  All  vectors are set to intersect the origin.
The two perspectives show that while the four vectors point in different directions for each band combination,
they are almost coplanar in ${\hat{U}}{\hat{V}}{\hat{I}}$ (they are slightly less coplanar in ${\hat{B}}{\hat{V}}{\hat{R}}$).  The $\vec{a}$ and $\vec{b}$ vectors and the $\vec{\gamma}^0$ and $\vec{\gamma}^1$
vectors span near-parallel planes in the 5-dimensional color space, and hence the latter can be almost entirely expressed in terms of
the former.

\begin{figure}[htbp] %  figure placement: here, top, bottom, or page
   \centering
   \includegraphics[width=2.95in]{fix1plane0.pdf}
   \includegraphics[width=2.95in]{fix1plane1.pdf}
   \caption{
   Visualization of how supernova magnitudes can vary in Model~I and that of \citetalias{1999PASP..111...63F}.  While the models describe
   magnitudes in 5-bands ${\hat{U}}{\hat{B}}{\hat{V}}{\hat{R}}{\hat{I}}$ this visualization shows only ${\hat{U}}{\hat{V}}{\hat{I}}$.   The left and right plots show the same information from
   two different perspectives.
   Our model is shown in solid-line unit vectors in the directions of $\vec{\gamma}^0$ and $\vec{\gamma}^1$. The only possible magnitudes
   are a linear combination of these two vectors, and hence are confined to the plane that contains both.
   The  \citetalias{1999PASP..111...63F} model is shown in dashed-line unit vectors
   in the directions of  $\vec{a}$, $\vec{b}$; the only possible magnitudes are confined to the plane that contains both.
   The perspective on the right visually shows that all four vectors are nearly coplanar.  
   The combination $\vec{a}+\vec{b}/2.25$ is shown in the dotted \replaced{red}{black}
   line: it is almost perfectly superimposed on $\vec{\gamma}^0$.
   \label{plane:fig}}
\end{figure}


We
now turn to
the specific choice of $\vec{\gamma}$'s  returned by the fit in terms of the \citetalias{1999PASP..111...63F}  dust-extinction model. 
Our linear model would be satisfied by any two independent vectors  that span
the plane defined by the $\vec{\gamma}$ vectors. 
While
Eq.~\ref{f99:eqn} expresses that model using basis vectors $\vec{a}$ and $\vec{b}$ and
their corresponding parameters 
$A_V$ and 
$E(B-V)$, it can be written more generally using arbitrary bases $\vec{a}+\kappa_1 \vec{b}$
and $\kappa_2 \vec{a} + \vec{b}$, and parameter sets
$A^F_V - \kappa_2 E^F(B-V)$ and $-\kappa_1 A^F_V + E^F(B-V)$  such that 
\begin{equation}
A_X =  (1-\kappa_1 \kappa_2)^{-1} [(A^F_V - \kappa_2 E^F(B-V))\left(a_X+\kappa_1 b_X \right) +  (-\kappa_1 A^F_V + E^F(B-V)) (\kappa_2 a_X + b_X)],
\label{newdust:eqn}
\end{equation}
where $\kappa_1$ and $\kappa_2$ are free to float.
In our analysis, the $\vec{\gamma}$ vectors converge to a specific direction, meaning that our result prefers specific values of
$\kappa_1=M[1,2]/M[1,1]$ and $\kappa_2=M[2,1]/M[2,2]$, where $M$ is the matrix in  Eq.~\ref{trans_I:eqn}.

The values of $\kappa$'s are determined by the conditions that the components $A^F_V - \kappa_2 E^F(B-V)$ and $-\kappa_1 A^F_V + E^F(B-V)$ be
uncorrelated with $\Delta$.   The corresponding basis vectors are $a_X+\kappa_1 b_X $ and $\kappa_2 a_X + b_X$.
Defining an effective total to selective extinction $\langle R^F \rangle \equiv \kappa_1^{-1}$ so that the first basis vector is
$a_X+ b_X/\langle R^F \rangle  $, we find from the 68\%-ile credible interval of $\kappa_1^{-1}$ that
$\langle R^F \rangle=2.25^{+0.11}_{-0.10}$.
The non-correlation condition does not strongly constrain the  second basis vector as
$\kappa_2 =  15.60^{+26.19}_{-32.47}$.
Figure~\ref{plane:fig} visualizes the near-perfect alignment of $\vec{\gamma}^0$ and  $\vec{a}+\vec{b}/2.25$.


\subsection{Model~I Summary}
There are two points to highlight from the analysis of
Model~I.
\begin{itemize}
\item Model~I introduces two latent supernova parameters that can influence supernova colors.
These parameters come  with ``non-informative'' priors.  Fitting
of our model gives a space of possible color variations highly consistent with those expected from two-parameter dust-extinction models. 
That we recover a physically-motivated  result lends confidence in the modeling and analysis.
\item In Model~I, any  color variation that remains after spectral corrections is allocated to two parameters.
Dust extinction is not the only effect that can contribute to $g_0$ and $g_1$ so interpreting
them as pure dust parameters is subject to systematic errors.
\item We have defined an effective parameter $\langle R^F\rangle$ whose value may be compared with dust models.
Such comparisons must be taken with a grain of salt,
as this value is not fundamentally constrained by the model but is fixed by the added condition that $\Delta$ is
uncorrelated with  $EW_{\text{Ca}}$, $EW_{\mathrm{Si}}$, $\lambda_{\text{Ca}}$,
$g_0$, and $g_1$.  The value of  $\langle R^F\rangle$ changes given different ansatzes  for breaking the $\Delta$--cofactor degeneracy 
\end{itemize}

\section{Model~II: Three Color Parameters}
\label{modelII:sec}
Model~II extends Model~I to include a third color parameter.  This increase in the number of degrees of freedom 
goes beyond the two of dust-extinction models used in supernova analysis.
 In this section we  focus on describing the consequences of the addition of the
new parameter.

\subsection{Model}
\label{modelIImodel:sec}
Model~II is identical to Model~I except for the 
addition of a third color parameter $p$.  The distribution of $p$ in our sample
is assumed to form a Normal distribution with unit standard deviation.  An unconstrained distribution,
as taken for $g_0$ and $g_1$, yields multiple local maxima in the posterior and leads to difficulty in the convergence of the MCMC.
The relative effect on the five magnitudes is specified by the global parameters 
$\sigma_p\vec{\phi}$, where $\vec{\phi}$ is a 5-dimensional unit vector.
Model~II reads as
\begin{equation}
p  \sim \mathcal{N}(0,1 ),
\end{equation}
\begin{equation}
\sigma_p  \sim \text{Cauchy}(0.1,0.1),
\end{equation}
\replaced{
\begin{equation}
\begin{pmatrix}
{\hat{U}}\\{\hat{B}}\\{\hat{V}}\\{\hat{R}}\\{\hat{I}}
\end{pmatrix}
=
\Delta  \vec{1} +
\begin{pmatrix}
c_{\hat{U}}+\alpha_{\hat{U}} EW_{\mathrm{Ca}} + \beta_{\hat{U}} EW_{\mathrm{Si}} + \eta_{\hat{U}} \lambda_{\mathrm{Si}}+\gamma^0_{{\hat{U}}} g_0 +\gamma^1_{{\hat{U}}} g_1  + \sigma_p\phi_{\hat{U}} p\\
c_{\hat{B}}+\alpha_{\hat{B}} EW_{\mathrm{Ca}} + \beta_{\hat{B}} EW_{\mathrm{Si}} + \eta_{\hat{B}} \lambda_{\mathrm{Si}} +\gamma^0_{{\hat{B}}} g_0 +\gamma^1_{{\hat{B}}} g_1  + \sigma_p\phi_{\hat{B}} p\\
c_{\hat{V}}+\alpha_{\hat{V}} EW_{\mathrm{Ca}} + \beta_{\hat{V}} EW_{\mathrm{Si}} + \eta_{\hat{V}} \lambda_{\mathrm{Si}} +\gamma^0_{{\hat{V}}} g_0 +\gamma^1_{{\hat{V}}} g_1 + \sigma_p\phi_{\hat{V}}  p\\
c_{\hat{R}}+\alpha_{\hat{R}} EW_{\mathrm{Ca}} + \beta_{\hat{R}} EW_{\mathrm{Si}} + \eta_{\hat{R}} \lambda_{\mathrm{Si}} +\gamma^0_{{\hat{R}}} g_0 +\gamma^1_{{\hat{R}}} g_1 + \sigma_p\phi_{\hat{R}} p\\
c_{\hat{I}}+\alpha_{\hat{I}} EW_{\mathrm{Ca}} + \beta_{\hat{I}} EW_{\mathrm{Si}}+ \eta_{\hat{I}} \lambda_{\mathrm{Si}}+\gamma^0_{{\hat{I}}} g_0 +\gamma^1_{{\hat{I}}} g_1 + \sigma_p\phi_{\hat{I}}  p
\end{pmatrix}.
\label{ewsiv2:eqn}
\end{equation}}
{\begin{equation}
\vec{m}=
 \vec{c} +  \vec{1} \, \Delta + 
\vec{\alpha} \,  EW_{\mathrm{Ca}}+
\vec{\beta}  \, EW_{\mathrm{Si}} +
\vec{\eta}  \, \lambda_{\mathrm{Si}} +
\vec{\gamma}^0  \,  g_0 +
\vec{\gamma}^1  \,  g_1 +
\sigma_p \vec{\phi}\,p.
\label{ewsiv2:eqn}
\end{equation}
}

The Cauchy prior is commonly used to concentrate probability around a specific value while having fat tails with infinite variance.


The model contains the product 
$\vec{\phi} p$, which has the degeneracy $\vec{\phi} \rightarrow -\vec{\phi}$, $p \rightarrow -p$.
No conditions are added to break this degeneracy to leave open the possibility that
$\sigma_p\vec{\phi}$ is consistent with zero.  The degeneracy
in $\vec{\phi}$ is apparent later in the article, but it does not affect the magnitude predictions. 
As in \S\ref{modelI:sec}, the degeneracy between $\vec{\phi}$ and $\Delta$ is fixed by enforcing zero correlation between $\Delta$ and $p$.

For $N$ supernovae there are $8N$ observables.  There are $3N$ spectral parameters, each of
$\Delta$, $g_0$, $g_1$ contributes $N-1$ parameters while $p$  contributes $N$ parameters,  and  there are $5 \times 7$ global coefficients.
For $N=172$ supernovae, there are 1376 observables and 1236  parameters.

\subsection{Results for the Latent Parameter Cofactor $\sigma_p \vec{\phi}$}
\label{results2:sec}
The credible intervals of all Model~II global parameters are given in  \S\ref{m12results:sec}.
In this subsection we focus on the effect of adding the third color term  $\sigma_p \vec{\phi}$.
There is strong evidence for a third color parameter.
The 68\% credible interval for $\sigma_p$ is $0.051^{+0.005}_{-0.004}$~mag, highly incompatible with zero. 
The posterior for $\sigma_p \vec{\phi}$ is shown in Figure~\ref{M1ev:fig}.  
Two maxima are allowed due to the sign degeneracy for $\vec{\phi}$.  Practically each of the MCMC chains converges to one of the solutions and does
not migrate between them by crossing zero.  To aid visibility of this figure we thus impose $\phi_I > 0$.  
 The new parameter does affect colors in a  manner distinct from $g_0$ and $g_1$.
A quadratic fraction of
$0.48^{+0.25}_{-0.17}$ of 
 $\vec{\phi}$ lies perpendicular to the $\vec{\gamma}_0$--$\vec{\gamma}_1$ plane.


\begin{figure}[htbp] %  figure placement: here, top, bottom, or page
   \centering
   \includegraphics[width=4in]{fix3sigev.pdf} 
            \caption{Model~II posterior  contours for $\sigma_p \vec{\phi}$. 
            The solid line shows the location of zero.          \label{M1ev:fig}}
\end{figure}

Model~II introduces a new parameter  that can absorb color variability
that in Model~I is restricted to $g_0$ and $g_1$, and whose distribution comes from a Normal distribution.
The change in the distributions of the $\gamma^0_{\hat{V}} g_0$ and $\gamma^1_{\hat{V}}g_1$ parameters from
 Model~I to Model~II
is apparent in Figure~\ref{kcomp:fig}.  (Here and in similar plots, the parameter value of one supernova,
e.g.\ $\gamma^0_{\hat{V}} g_0|_0$, is subtracted
out to null out correlated
parameter uncertainty between supernovae; recall the $\langle \Delta \rangle=0$, $\langle g_0 \rangle=0$, $\langle g_1 \rangle=0$,
and $p  \sim \mathcal{N}(0,1)$ conditions.) The  change from Model~I to II is pronounced in  $\gamma^1_{\hat{V}}g_1$,
where  a Normal distribution component apparent in Model~I disappears in Model~II, presumably having been  assigned to $p$.


\begin{figure}[htbp] %  figure placement: here, top, bottom, or page
   \centering
   \includegraphics[width=2.6in]{fix1deltagamma0_med.pdf} 
   \includegraphics[width=2.6in]{fix1deltagamma1_med.pdf} 
   \includegraphics[width=2.6in]{fix3deltagamma0_med.pdf} 
   \includegraphics[width=2.6in]{fix3deltagamma1_med.pdf} 
   \caption{
   Left:
   Normalized stack of the posteriors (includes parameter uncertainties) of all supernovae  and histogram 
      of  per-supernova 
median values of $\gamma^0_{\hat{V}} g_0-\gamma^0_{\hat{V}} g_0|_0$.  Right:  The same plots for $\gamma^1_{\hat{V}} g_1-\gamma^1_{\hat{V}} g_1|_0$,
Top: Model~I. Bottom: Model~II.
    \label{kcomp:fig}}
\end{figure}

The $\vec{\gamma}$ vectors, like those from Model~I, are almost entirely projected
onto the plane defined by
$\vec{a}$ and $\vec{b}$ of the \citetalias{1999PASP..111...63F} model.
The direction of the  $\vec{\gamma}^0$ vector corresponds to an effective
$\langle R^F_{\mathit{eff}}\rangle = 2.54^{+0.13}_{-0.12}$.
The change of $\langle R^F_{\mathit{eff}}\rangle$ from Model~I to II is anticipated by the change in the distribution of $g_1$,
whose connection with $R^F$ is shown in \S\ref{g0g1:sec}.


\subsection{Model~II Summary}
The inclusion of a third color term beyond the two of Model~I has important consequences.

\begin{itemize}
\item The data strongly supports the influence of an additional supernova parameter that affects supernova colors in a manner orthogonal to
 dust extinction.
\item The effective $\langle R^F_{\mathit{eff}}\rangle$ shifts by $>1 \sigma$ when including the new color term.  This demonstrates,
as has been noted by \citet{2009ApJ...704.1036F,2009ApJS..185...32K}, that dust analyses
that do not allow for additional color dispersion are susceptible to bias when inferring dust parameters.
\end{itemize}

\section{Model~III: Three Color Parameters and Light-Curve Shape}
\label{modelIII:sec}

Light-curve shape is an established indicator of SN~Ia diversity 
\citep{1993ApJ...413L.105P, 1996ApJ...473...88R,
1997ApJ...483..565P}
with its own color correlations \citep{2005A&A...443..781G, 2007ApJ...659..122J}.
Model~III explicitly includes light-curve shape as a feature used to standardize supernova colors, to complement the
spectral features considered in the previous two models.
We will show that light-curve shape does have a non-trivial relationship with color that is not already encoded
in the spectral features.  Model~III is thus the focus of
this article and its results are  described here in depth.


\subsection{Model}
\label{modelIImodel:sec}
Model~III is identical to Model~II except for the 
addition of a new term  that relates magnitudes with the SALT2 light-curve-shape parameter,
$x_1$, through linear coefficients $\vec{\zeta}$. 
The model reads as
\begin{equation}
p  \sim \mathcal{N}(0,1 ),
\end{equation}
\begin{equation}
\sigma_p  \sim \text{Cauchy}(0.1,0.1),
\end{equation}
\replaced{
\begin{equation}
\begin{pmatrix}
{\hat{U}}\\{\hat{B}}\\{\hat{V}}\\{\hat{R}}\\{\hat{I}}
\end{pmatrix}
=
\Delta  \vec{1} +
\begin{pmatrix}
c_{\hat{U}}+\alpha_{\hat{U}} EW_{\mathrm{Ca}} + \beta_{\hat{U}} EW_{\mathrm{Si}} + \eta_{\hat{U}} \lambda_{\mathrm{Si}} +\zeta_{\hat{U}} x_1 +\gamma^0_{{\hat{U}}} g_0 +\gamma^1_{{\hat{U}}} g_1 + \sigma_p\phi_{\hat{U}} p\\
c_{\hat{B}}+\alpha_{\hat{B}} EW_{\mathrm{Ca}} + \beta_{\hat{B}} EW_{\mathrm{Si}} + \eta_{\hat{B}} \lambda_{\mathrm{Si}} +\zeta_{\hat{B}} x_1 +\gamma^0_{{\hat{B}}} g_0 +\gamma^1_{{\hat{B}}} g_1 + \sigma_p\phi_{\hat{B}} p\\
c_{\hat{V}}+\alpha_{\hat{V}} EW_{\mathrm{Ca}} + \beta_{\hat{V}} EW_{\mathrm{Si}} + \eta_{\hat{V}} \lambda_{\mathrm{Si}} +\zeta_{\hat{V}} x_1+\gamma^0_{{\hat{V}}} g_0 +\gamma^1_{{\hat{V}}} g_1 + \sigma_p\phi_{\hat{V}}  p\\
c_{\hat{R}}+\alpha_{\hat{R}} EW_{\mathrm{Ca}} + \beta_{\hat{R}} EW_{\mathrm{Si}} + \eta_{\hat{R}} \lambda_{\mathrm{Si}} +\zeta_{\hat{R}} x_1+\gamma^0_{{\hat{R}}} g_0 +\gamma^1_{{\hat{R}}} g_1 + \sigma_p\phi_{\hat{R}} p\\
c_{\hat{I}}+\alpha_{\hat{I}} EW_{\mathrm{Ca}} + \beta_{\hat{I}} EW_{\mathrm{Si}}+ \eta_{\hat{I}} \lambda_{\mathrm{Si}}+\zeta_{\hat{I}} x_1+\gamma^0_{{\hat{I}}} g_0 +\gamma^1_{{\hat{I}}} g_1 + \sigma_p\phi_{\hat{I}}  p
\end{pmatrix},
\label{ewsiv3:eqn}
\end{equation}}
{\begin{equation}
\vec{m}=
 \vec{c} +  \vec{1} \, \Delta + 
\vec{\alpha} \,  EW_{\mathrm{Ca}}+
\vec{\beta}  \, EW_{\mathrm{Si}} +
\vec{\eta}  \, \lambda_{\mathrm{Si}} +
\vec{\zeta} x_1 +
\vec{\gamma}^0  \,  g_0 +
\vec{\gamma}^1  \,  g_1 +
\sigma_p \vec{\phi}\,p.
\label{ewsiv3:eqn}
\end{equation}
}


\begin{equation}
\begin{pmatrix}
{\hat{U}}_o\\{\hat{B}}_o\\ {\hat{V}}_o\\{\hat{R}}_o\\{\hat{I}}_o\\EW_{Si, o}\\ EW_{Ca, o} \\ \lambda_{Si, o} \\ x_{1, o}
\end{pmatrix}
\sim \mathcal{N}
\left(
\begin{pmatrix}
{\hat{U}}  \\{\hat{B}}  \\
{\hat{V}}\\{\hat{R}}\\{\hat{I}}\\
EW_{\mathrm{Si}}\\ EW_{\mathrm{Ca}} \\ \lambda_{\mathrm{Si}} \\ x_1
\end{pmatrix}
,C
\right).
\label{dust3:eqn}
\end{equation}
The changes from Model~II are the addition of the $\vec{\zeta} x_1$ terms in Eq.~\ref{ewsiv3:eqn}, and
the addition of the shape measurement and uncertainty (implicit in the covariance matrix $C$) and $x_1$   in Eq.~\ref{dust3:eqn}.
As in \S\ref{modelI:sec}, the degeneracy between $\vec{\zeta}$ and $\Delta$ is fixed by enforcing zero correlation between $\Delta$ and $x_1$.
For $N$ supernovae there are $9N$ observables.  There are $4N$ spectral and light-curve shape parameters, each of
$\Delta$, $g_0$, $g_1$ contributes $N-1$ parameters while $p$  contributes $N$ parameters,  and there are $5 \times 8$ global coefficients.
For $N=172$ supernovae, this makes 1548 observables and 1413  parameters.


\subsection{Results}
\subsubsection{Analysis Method and Validation}

In the analysis of this and previous sections of this article,
the posterior of the model parameters is evaluated using Hamiltonian Monte Carlo with a No-U-Turn
Sampler as implemented in
STAN \citep{JSSv076i01}.  We run eight chains, each with 5000 iterations of which
half are used for warmup.
STAN provides output statistics to assess
the convergence of the output Markov chains.
The 
potential scale reduction statistic, $\hat{R}$
\citep[][in this paragraph not to be confused with the synthetic $\hat{R}$-band magnitude]{Gelman92}, measures the convergence of the target distribution
in iterative simulations 
by using multiple independent sequences to estimate how much that distribution would sharpen if the simulations were run longer.
$N_{\mathit{eff}}$ is an estimate of the number of independent draws. The STAN output gives $\hat{R} \sim 1.0$ for all parameters, meaning there is no evidence for non-convergence.  The
output also gives  $N_{\mathit{eff}} \gg 100$ for all parameters, indicating that they are densely sampled.


Our analysis pipeline is run with input simulated data with known  signals and measurement uncertainties similar to those of our dataset.
Stacks of the posteriors from 100 simulated datasets  exhibit no biases that are significant relative to the statistical uncertainties of a single posterior.

As will be seen in  \S\ref{results3global:sec}, the posterior derived from the analysis has a distinct bounded  
peak showing no indication that the MCMC has failed to converge.

The model provides a fair representation of the data as seen in the residual differences between the data and the model prediction.
Figure~\ref{residual:fig} shows plots of 
residuals between measured and model-expectation colors as a function of observed $\hat{B}_o-\hat{V}_o$.
This and similar plots for the other permutations of observed colors show  no apparent trend between residuals and observed
color nor any catastrophic outliers.

\begin{figure}[htbp] %  figure placement: here, top, bottom, or page
   \centering
   \includegraphics[width=4in]{cpull.pdf} 
               \caption{
               Residuals between measured and model-expectation colors as a function of observed $\hat{B}_o-\hat{V}_o$.  The colors are
               relative to on arbitrary supernova. Plots that show
               residual
               colors containing $\hat{B}_o$ or $\hat{V}_o$ are shaded to highlight that the abscissa and ordinate have contributions
               from  a common measurement.
            \label{residual:fig}}
\end{figure}

This suite of tests lend confidence in our model and the validity of the derived posterior.

\subsubsection{Global Parameter Posterior Results}
\label{results3global:sec}
Results from our analysis are shown in Figures~\ref{global1:fig} -- \ref{global5:fig}
as contours of the posterior surface for pairs of global parameters grouped by filter.
The confidence regions are localized and unimodal, except for the combination $\sigma_p \vec{\phi}$ that
has the sign degeneracy explained in \S\ref{results2:sec} and plotted in Figure~\ref{M1ev:fig}.
Within the finite number of generated links, the MCMC chains do not migrate between the parity-degenerate $\vec{\gamma}$--$g$ solutions.
Each chain converges to one of the  $\vec{\phi}$--$p$ degenerate solutions, away from which it does not depart.  

\begin{figure}[htbp] %  figure placement: here, top, bottom, or page
   \centering
   \includegraphics[width=5.2in]{coeff0.pdf} 
            \caption{Model~III posterior contours for $\vec{c}$, $\vec{\alpha}$, $\vec{\beta}$, $\vec{\eta}$, $\vec{\gamma}^0$, $\vec{\gamma}^1$, and $\sigma_p \vec{\phi}$ in the ${\hat{U}}$ band.
            The contours shown here and in future plots represent 1-$\sigma$ in the parameter distribution (i.e.\ they should be
            projected onto the corresponding 1-d parameter axis), not to 68\%, 95\%, etc.\
            enclosed probability.  Lines for zero value for $\alpha_{\hat{U}}$, $\beta_{\hat{U}}$, $\eta_{\hat{U}}$, $\gamma_{\hat{U}}^0$, $\gamma_{\hat{U}}^1$, and $\sigma_p \phi_{\hat{U}}$ are shown for reference --
            in most cases they are outside the range of the plot.
            \label{global1:fig}}
\end{figure}

\begin{figure}[htbp] %  figure placement: here, top, bottom, or page
   \centering
   \includegraphics[width=5.2in]{coeff1.pdf} 
            \caption{Model~III posterior contours for $\vec{c}$, $\vec{\alpha}$, $\vec{\beta}$, $\vec{\eta}$,  $\vec{\gamma}^0$, $\vec{\gamma}^1$, and $\sigma_p \vec{\phi}$ in the ${\hat{B}}$ band.
 \label{global2:fig}}
\end{figure}

\begin{figure}[htbp] %  figure placement: here, top, bottom, or page
   \centering
   \includegraphics[width=5.2in]{coeff2.pdf} 
            \caption{Model~III posterior contours for $\vec{c}$, $\vec{\alpha}$, $\vec{\beta}$, $\vec{\eta}$, $\vec{\gamma}^0$, $\vec{\gamma}^1$, and $\sigma_p \vec{\phi}$ in the ${\hat{V}}$ band.
 \label{global3:fig}}
\end{figure}

\begin{figure}[htbp] %  figure placement: here, top, bottom, or page
   \centering
      \includegraphics[width=5.2in]{coeff3.pdf} 
            \caption{Model~III posterior contours for  $\vec{c}$, $\vec{\alpha}$, $\vec{\beta}$, $\vec{\eta}$,  $\vec{\gamma}^0$, $\vec{\gamma}^1$, and $\sigma_p \vec{\phi}$ in the ${\hat{R}}$ band.
 \label{global4:fig}}
\end{figure}

\begin{figure}[htbp] %  figure placement: here, top, bottom, or page
   \centering
         \includegraphics[width=5.2in]{coeff4.pdf} 
            \caption{Model~III posterior contours for  $\vec{c}$, $\vec{\alpha}$, $\vec{\beta}$, $\vec{\eta}$, $\vec{\gamma}^0$, $\vec{\gamma}^1$, and $\sigma_p \vec{\phi}$ in the ${\hat{I}}$ band.
 \label{global5:fig}}
\end{figure}


For each of the five filters, the 68\%  equal-tailed credible intervals for the global parameters $\vec{\alpha}$, $\vec{\beta}$, $\vec{\eta}$, $\vec{\zeta}$,
$\vec{\gamma}^0$, $\vec{\gamma}^1$, and $\sigma_p\vec{\phi}$
are given in Table~\ref{global2:tab}.
In constructing the credible interval of $\sigma_p\vec{\phi}$ 
we set $\vec{\phi} = -\text{sign}(\phi_{\hat{V}}) \vec{\phi}$ to break its parity degeneracy;
Table~\ref{global3:tab} shows it has the largest signal-to-noise in
in the $\hat{V}$-band.
The effect of spectral parameters on color (as opposed to magnitude)
is shown in the rows of $\alpha_X/\alpha_{\hat{V}}-1$,  $\beta_X/\beta_{\hat{V}}-1$, and  $\eta_X/\eta_{\hat{V}}-1$.
The normalization freedom of $\vec{\gamma}$ is nulled out in the statistic
 $\gamma_X/\gamma_{\hat{V}}-1$.
 These statistics give the change in $X-\hat{V}$ color normalized to the change in $\hat{V}$ magnitude, and are analogous to
$R_V^{-1}=A_B/A_V-1$ as an observational descriptor of dust properties.

\begin{table}
\centering
\begin{tabular}{|c|c|c|c|c|c|}
\hline
Parameters& $X={\hat{U}}$ &${\hat{B}}$&${\hat{V}}$&${\hat{R}}$&${\hat{I}}$\\ \hline
$\alpha_X$
& $0.0051^{+0.0007}_{-0.0008}$
& $0.0025^{+0.0006}_{-0.0007}$
& $0.0022^{+0.0005}_{-0.0005}$
& $0.0019^{+0.0004}_{-0.0004}$
& $0.0034^{+0.0003}_{-0.0004}$
\\
${\alpha_X}/\alpha_{\hat{V}}-1$
& $   1.3^{+   0.4}_{  -0.2}$
& $   0.1^{+   0.1}_{  -0.1}$
& \ldots
& $  -0.1^{+   0.0}_{  -0.0}$
& $   0.5^{+   0.3}_{  -0.2}$
\\
$\beta_X$
& $ 0.035^{+ 0.005}_{-0.005}$
& $ 0.024^{+ 0.004}_{-0.004}$
& $ 0.021^{+ 0.004}_{-0.004}$
& $ 0.019^{+ 0.003}_{-0.003}$
& $ 0.010^{+ 0.003}_{-0.003}$
\\
${\beta_X}/\beta_{\hat{V}}-1$
& $  0.68^{+  0.16}_{ -0.13}$
& $  0.14^{+  0.07}_{ -0.07}$
&\ldots
& $ -0.11^{+  0.03}_{ -0.02}$
& $ -0.50^{+  0.06}_{ -0.07}$
\\
$\eta_X$
& $0.0003^{+0.0011}_{-0.0011}$
& $0.0005^{+0.0009}_{-0.0009}$
& $0.0008^{+0.0007}_{-0.0007}$
& $0.0007^{+0.0006}_{-0.0006}$
& $-0.0000^{+0.0005}_{-0.0005}$
\\
${\eta_X}/\eta_{\hat{V}}-1$
& $ -0.39^{+  0.64}_{ -1.36}$
& $ -0.26^{+  0.35}_{ -0.77}$
&\ldots
& $ -0.19^{+  0.12}_{ -0.11}$
& $ -0.86^{+  0.33}_{ -0.71}$
\\
$\zeta_X$
& $ -0.00^{+  0.04}_{ -0.04}$
& $ -0.02^{+  0.03}_{ -0.03}$
& $ -0.05^{+  0.03}_{ -0.03}$
& $ -0.03^{+  0.02}_{ -0.02}$
& $ -0.08^{+  0.02}_{ -0.02}$
\\
${\zeta_X}/\zeta_{\hat{V}}-1$
& $ -0.94^{+  0.48}_{ -1.15}$
& $ -0.54^{+  0.28}_{ -0.65}$
& \ldots
& $ -0.42^{+  0.09}_{ -0.21}$
& $  0.47^{+  0.83}_{ -0.32}$
\\
$\gamma^0_X$
& $ 68.91^{+  2.62}_{ -2.56}$
& $ 55.44^{+  2.22}_{ -2.25}$
& $ 41.41^{+  1.98}_{ -1.98}$
& $ 31.49^{+  1.61}_{ -1.62}$
& $ 22.92^{+  1.35}_{ -1.30}$
\\
${\gamma^0_X}/\gamma^0_{\hat{V}}-1$
& $  0.67^{+  0.04}_{ -0.04}$
& $  0.34^{+  0.02}_{ -0.02}$
&\ldots
& $ -0.24^{+  0.01}_{ -0.01}$
& $ -0.45^{+  0.02}_{ -0.02}$
\\
$\gamma^1_X$
& $ -1.85^{+  4.11}_{ -4.05}$
& $  1.47^{+  3.74}_{ -3.58}$
& $ -6.75^{+  3.21}_{ -3.15}$
& $ -6.60^{+  2.71}_{ -2.65}$
& $ -7.65^{+  2.33}_{ -2.24}$
\\
${\gamma^1_X}/\gamma^1_{\hat{V}}-1$
& $ -0.70^{+  0.38}_{ -0.76}$
& $ -1.18^{+  0.43}_{ -1.08}$
& \ldots
& $ -0.03^{+  0.14}_{ -0.08}$
& $  0.12^{+  0.45}_{ -0.22}$
\\
$\sigma_p \phi_X$
& $ 0.002^{+ 0.016}_{-0.017}$
& $-0.041^{+ 0.013}_{-0.014}$
& $-0.050^{+ 0.011}_{-0.012}$
& $-0.032^{+ 0.009}_{-0.010}$
& $ 0.002^{+ 0.008}_{-0.008}$
\\
${\phi_X/\phi_{\hat{V}}-1}$
& $-1.034^{+ 0.291}_{-0.409}$
& $-0.174^{+ 0.103}_{-0.134}$
& \ldots
& $-0.347^{+ 0.042}_{-0.060}$
& $-1.032^{+ 0.141}_{-0.208}$
\\

\hline
\end{tabular}
\caption{68\% credible intervals for the global fit parameters of Model~III in \S\ref{modelIII:sec}.\label{global3:tab}}
\end{table}



\subsubsection{Results for the Observables Cofactors $\vec{\alpha}$, $\vec{\beta}$, $\vec{\eta}$,  $\vec{\zeta}$}

The cofactors  $\vec{\alpha}$, $\vec{\beta}$, and $\vec{\eta}$ have  significant non-zero values,  meaning that
$EW_{\mathrm{Ca}}$, $EW_{\mathrm{Si}}$, and $\lambda_{\mathrm{Si}}$ are indicators of broadband
colors at peak.  The significance of their influence varies between  colors; in $\hat{B}-\hat{V}$, roughly the color
where previous studies have focused, only $\vec{\beta}$ has a strong $>2\sigma$ correlation.  A positive
dependence (consistent with the positive sign
of $\beta_{\hat{B}}/\beta_{\hat{V}}-1$)
of $B-V$ on the pseudo-equivalent width of Si~II~$\lambda$4130 has previously been reported by
\citet{2011ApJ...734...42N, 2013ApJ...773...53F}.
The insignificance of  our $\alpha_{\hat{B}}/\alpha_{\hat{V}}-1$ is consistent with the
lack of correlation between the  pseudo-equivalent width of Ca~II~H\&K measurements and
intrinsic color reported by \citet{2011ApJ...742...89F}.

The story on Si~II~$\lambda$6355 is not as clear.
 \citet{2009ApJ...699L.139W,2011ApJ...742...89F, 2012ApJ...748..127F} find that
higher-velocity SNe~Ia tend to be redder than those with lower velocity. \citet{2012AJ....143..126B} find a
weaker relation than that reported by \citet{2011ApJ...742...89F},
a difference attributed to a better treatment of uncertainties and host-galaxy dust. In contrast,
\citet{2013ApJ...773...53F} find no significant relation.
Our result for $\eta_{\hat{B}}/\eta_{\hat{V}}-1$ also gives no significant relation. 

The significantly non-zero values for  $\alpha_{\hat{U}}$ and $\beta_{\hat{U}}$ from this study can be compared with the $U$-band study of
\citet{2018arXiv180101834N} carried out on a subset of the same supernovae.
Both analyses find a strong correlation between $U$ and the Si and Ca features.

The non-zero values for $\vec{\zeta}$ and their effect on color $\vec{\zeta}-\zeta_{\hat{V}}$  demonstrate that light-curve shape $x_1$ is  an indicator
of broadband colors, in the sense that SNe~Ia with broader light curves are bluer in all colors.
This is as expected based on the results of \citet{2007A&A...466...11G, 2007ApJ...659..122J}.


The signal in  $\vec{\alpha}$ and $\vec{\beta}$ cannot
entirely
be attributed to the equivalent widths themselves.
The range of Si~II~$\lambda$4130 equivalent widths is $\pm 20$~\AA\ whereas the width of the ${\hat{B}}$-band is 851~\AA, so that its direct affect on magnitude
is
$2.5 \log{(20/850)} \sim 0.03$ mag.  
The implied span in ${\hat{B}}$ magnitude based on $\beta_{\hat{B}}$ is 0.54~mag.  Therefore $\beta_{\hat{B}}$ cannot wholly be attributed to the flux deficit
from the line itself.
The Ca~II H\&K equivalent widths have range $\pm 50$~\AA, while the width of the ${\hat{U}}$ band is
701~\AA, so that its direct affect on magnitude
is
$2.5 \log{(50/701)} \sim 0.08$ mags.   The implied span in ${\hat{U}}$ magnitude is  0.21~mag, 
which would imply that  $\alpha_{\hat{U}}$ cannot be completely due to the 
presence of the line itself.  
Supernova flux has a large gradient in our $\hat{U}$ and there is a variation in the effective, flux-weighted,  $\hat{U}$ bandwidth. 
Nevertheless, the  correlation between spectral features with magnitudes in bands with no wavelength overlap is a non-trivial
signal detected in the analysis.

The posterior visualized in
Figures~\ref{global1:fig} -- \ref{global5:fig} shows
 that $\vec{\zeta}$, the magnitude vector associated with $x_1$, is highly correlated with $\vec{\beta}$, the vector
 associated with  $EW_{\mathrm{Si}}$.
The direct correlation between  $x_1$ and $EW_{\mathrm{Si}}$ has already been established
\citep{2008A&A...492..535A, 2011A&A...529L...4C}
and exists
in our sample as is shown later in \S\ref{results3per:sec}.  
Nevertheless the $\vec{\zeta}$ and $\vec{\beta}$   posteriors are bounded and inconsistent with zero, indicating that
 $x_1$ does convey color information independent of our set of spectral features.
For this reason  Model~III is highlighted in this article.

\subsubsection{Results for the Latent Parameter Cofactors $\vec{\gamma}^0$, $\vec{\gamma}^1$}
\label{results3gamma:sec}
As with Model~I in \S\ref{modelI:sec}, all elements of the two $\vec{\gamma}$ parameter vectors 
are significantly non-zero with 
none of the 20000 links of 
our Monte Carlo chains for $\vec{\gamma}$ extending to 0 (see Figures~\ref{global1:fig}--\ref{global5:fig}).
The direction of the vector in the ${\hat{U}}{\hat{V}}{\hat{I}}$-subspace is shown in Figure~\ref{plane3:fig}.

The transformation between the $\vec{\gamma}$  vectors and the   \citetalias{1999PASP..111...63F} vector given in
Eq.~\ref{trans_I:eqn}, is satisfied by the transformation matrix
\begin{equation}
M=
\begin{pmatrix}
\begin{array}{rr}
40.3^{+1.8}_{-1.8} & 16.6^{+0.6}_{-0.6} \\
-5.8^{+3.1}_{-3.0} & 3.4^{+1.0}_{-0.9}
\end{array}
\end{pmatrix} 
\end{equation}
and the two residual vectors
\begin{align}
\begin{split}
\vec{\epsilon}_{\gamma^0} &=\left(1.1^{+0.4}_{-0.4} , -1.2^{+0.4}_{-0.4} , -0.9^{+0.4}_{-0.4} , 0.7^{+0.2}_{-0.2} , 0.5^{+0.6}_{-0.5}\right) \text{mag}, \\
\vec{\epsilon}_{\gamma^1} & =\left(-2.3^{+0.6}_{-0.5} , 4.0^{+0.6}_{-0.6} , -1.3^{+0.6}_{-0.6} , 0.7^{+0.3}_{-0.3} , -1.4^{+0.8}_{-0.8}\right) \text{mag}.
\end{split}
\label{res_3:eqn}
\end{align}
This corresponds to a quadratic fraction of $0.9996^{+0.0002}_{-0.0003}$ of $\vec{\gamma}^0$ that projects onto the plane of
color changes allowed by the \citetalias{1999PASP..111...63F} dust-extinction model
as described by
$\vec{a}$ and $\vec{b}$. The quadratic fraction of $\vec{\gamma}^1$ projected
onto this plane is
$0.8426^{+0.0780}_{-0.1236}$.
For the standard deviation values of $g_0$ and $g_1$ of the supernovae in our sample, the amplitude of the magnitude residuals from $\vec{\epsilon}$ are 
$\left(0.005, 0.012,0.011, 0.007, 0.004 \right)$~mag.
The   $\vec{\gamma}^0$  vector points in the $\langle R^F_{\mathit{eff}}\rangle=2.43^{+0.10}_{-0.10}$ dust-extinction direction.
This value is consistent with the $R_V=2.5$ found in the supernova samples of \citet{2011ApJ...729...55F}, and the
center  of the range of per-supernova $R^F$ values found in the samples of \citet{2014ApJ...789...32B, 2015MNRAS.453.3300A}.
Although the model $\vec{\gamma}$'s cannot be attributed to dust, they behave like the   \citetalias{1999PASP..111...63F} model.

\begin{figure}[htbp] %  figure placement: here, top, bottom, or page
   \centering
   \includegraphics[width=2.95in]{plane0.pdf}
   \includegraphics[width=2.95in]{plane1.pdf}
   \caption{
   Visualization of how supernova magnitudes can vary in Model~III and the dust extinction of \citetalias{1999PASP..111...63F}.
   See Figure~\ref{plane:fig}.
   The combination $\vec{a}+\vec{b}/2.43$ is shown in the dotted \replaced{red}{black}
   line: it is almost perfectly superimposed on $\vec{\gamma}^0$.
   \label{plane3:fig}}
\end{figure}


\subsubsection{Results for the Latent Parameter Cofactor $\sigma_p \vec{\phi}$}

The  new  $p$ parameter has a significant influence on color.  The magnitude of its color effect is non-zero,
with $\sigma_p = 0.053^{+ 0.010}_{-0.005}$~mag.  The smallest value of $\sigma_p$ % new
over all Monte Carlo chains is 0.040~mag.
These numbers are significantly larger than the median data color uncertainty of 0.010~mag (the maximum is 0.042~mag).
The evaluation of 100 simulated data sets with no third color parameter produces no median $\sigma_p$'s  as
high as that of the data, so we assign a $>99$\% confidence in the detection of this third color term.

The direction of $\vec{\phi}$ is not aligned with the plane defined by
the $\vec{\gamma}$ vectors (nor of the  \citetalias{1999PASP..111...63F} dust-extinction model); a quadratic fraction of
$0.49^{+0.27}_{-0.19}$ of  $\vec{\phi}$ lies perpendicular to the $\vec{\gamma}^0$--$\vec{\gamma}^1$ plane.

The values of $\vec{\phi}$ does not change  monotonically with wavelength, as is apparent  in 
the combination $\phi_X/\phi_{\hat{V}}-1$ shown in Figure~\ref{phiratio:fig}.  An object that has fainter  $\hat{U}$ or $\hat{I}$-band peak magnitudes
has brighter  $\hat{B}$$\hat{V}$$\hat{R}$  peaks.  That brightening is non-uniform, in that
the reddening of $\hat{U}-\hat{B}$ and $\hat{B}-\hat{V}$ is accompanied
by a bluing in $\hat{V}-\hat{R}$ and  $\hat{R}-\hat{I}$. 

What is measured as $\sigma_p \vec{\phi}$ likely captures the effect of many physical processes.
To simplify the model it is assumed that the $p$'s are drawn from a Normal distribution, which may not be a fair description of what occurs in nature
and result in model bias.
The physical interpretation of this color should be tempered with caution.

\begin{figure}[htbp] %  figure placement: here, top, bottom, or page
   \centering
      \includegraphics[width=4in]{phiratio.pdf}
   \caption{Model~III median values  and corresponding 68\% intervals for $\phi_X/\phi_{\hat{V}}-1$ in the 5 bands.
   A dotted line at zero is shown for reference.
   \label{phiratio:fig}}
\end{figure}


\subsubsection{Results for the Per-Supernova Parameters}
\label{results3per:sec}
Each supernova is described by its parameters $\Delta$, $EW_{\mathrm{Ca}}$, $EW_{\mathrm{Si}}$, $\lambda_{\mathrm{Si}}$, $x_1$,
$E_{\gamma^0}({\hat{B}}-{\hat{V}})=(\gamma^0_{\hat{B}}-\gamma^0_{\hat{V}})g_0$, $E_{\gamma^1}({\hat{B}}-{\hat{V}})=(\gamma^1_{\hat{B}}-\gamma^1_{\hat{V}})g_1$, and $A_{p,V} =  \sigma_p\phi_{\hat{V}}p$.  (The  three latent parameters are cast as physically meaning
quantities by multiplying them by select cofactors.)
\added{The 68\% credible intervals for these parameters for all supernovae are given in Table~\ref{results:tab}.
\startlongtable
\begin{deluxetable}{crrrrrrrrr}
\tabletypesize{\tiny}
\tablecaption{Supernova Parameter Credible Intervals
\label{results:tab}}
\tablehead{
\colhead{Name} & \colhead{$EW_{\mathrm{Ca}}$ (\AA)} & \colhead{$EW_{\mathrm{Si}}$ (\AA)} & \colhead{$\lambda_{\mathrm{Si}}$ (\AA)} & \colhead{$x_1$} & \colhead{$\Delta$}
& \colhead{$E_{\gamma^0}({\hat{B}}-{\hat{V}})$} & \colhead{$E_{\gamma^1}({\hat{B}}-{\hat{V}})$} & \colhead{$A_{p,V} $}
}
\startdata
SNF20050624-000 & $119.9^{+5.1}_{-5.1}$ & $  7.9^{+2.4}_{-2.6}$ & $ 6126^{+  5}_{-  5}$ & $  0.47^{+  0.29}_{-  0.29}$ & $-0.171^{+0.026}_{-0.024}$  & $-0.051^{+0.019}_{-0.017}$ & $0.022^{+0.019}_{-0.021}$ & $0.022^{+0.019}_{-0.021}$\\
SNF20050728-006 & $127.7^{+2.5}_{-2.4}$ & $ 15.3^{+1.2}_{-1.2}$ & $ 6124^{+  6}_{-  6}$ & $  0.11^{+  0.19}_{-  0.19}$ & $-0.088^{+0.034}_{-0.032}$  & $0.064^{+0.011}_{-0.012}$ & $0.012^{+0.017}_{-0.017}$ & $0.012^{+0.017}_{-0.017}$\\
SNF20050729-002 & $110.4^{+2.3}_{-2.2}$ & $ 10.7^{+1.5}_{-1.4}$ & $ 6141^{+  6}_{-  6}$ & $ -0.22^{+  0.21}_{-  0.22}$ & $0.052^{+0.102}_{-0.075}$  & $-0.039^{+0.015}_{-0.014}$ & $-0.039^{+0.011}_{-0.008}$ & $-0.039^{+0.011}_{-0.008}$\\
SNF20050821-007 & $141.8^{+2.6}_{-2.6}$ & $  7.8^{+0.9}_{-0.9}$ & $ 6140^{+  9}_{-  8}$ & $  0.34^{+  0.20}_{-  0.19}$ & $-0.024^{+0.034}_{-0.033}$  & $-0.030^{+0.014}_{-0.013}$ & $-0.012^{+0.017}_{-0.017}$ & $-0.012^{+0.017}_{-0.017}$\\
SNF20060511-014 & $102.9^{+2.8}_{-2.8}$ & $ 17.1^{+1.0}_{-1.0}$ & $ 6144^{+  8}_{-  7}$ & $ -0.50^{+  0.15}_{-  0.15}$ & $-0.042^{+0.047}_{-0.037}$  & $-0.008^{+0.011}_{-0.010}$ & $-0.041^{+0.009}_{-0.007}$ & $-0.041^{+0.009}_{-0.007}$\\
SNF20060512-001 & $ 88.2^{+1.2}_{-1.2}$ & $  5.5^{+0.4}_{-0.4}$ & $ 6169^{+  3}_{-  3}$ & $  0.87^{+  0.11}_{-  0.11}$ & $-0.025^{+0.025}_{-0.025}$  & $0.050^{+0.011}_{-0.011}$ & $-0.010^{+0.011}_{-0.012}$ & $-0.010^{+0.011}_{-0.012}$\\
SNF20060512-002 & $100.4^{+2.7}_{-2.8}$ & $ 13.5^{+1.1}_{-1.1}$ & $ 6106^{+  7}_{-  7}$ & $ -0.87^{+  0.15}_{-  0.16}$ & $-0.186^{+0.030}_{-0.028}$  & $0.018^{+0.012}_{-0.012}$ & $0.056^{+0.017}_{-0.016}$ & $0.056^{+0.017}_{-0.016}$\\
SNF20060521-001 & $ 89.1^{+11.4}_{-12.2}$ & $ 19.1^{+1.1}_{-1.1}$ & $ 6119^{+ 10}_{-  9}$ & $ -1.70^{+  0.20}_{-  0.20}$ & $0.029^{+0.061}_{-0.059}$  & $-0.068^{+0.014}_{-0.010}$ & $0.010^{+0.019}_{-0.020}$ & $0.010^{+0.019}_{-0.020}$\\
SNF20060526-003 & $111.8^{+2.5}_{-2.5}$ & $  9.9^{+1.0}_{-1.0}$ & $ 6121^{+  3}_{-  3}$ & $  0.34^{+  0.13}_{-  0.13}$ & $-0.004^{+0.026}_{-0.027}$  & $-0.010^{+0.010}_{-0.010}$ & $-0.011^{+0.014}_{-0.014}$ & $-0.011^{+0.014}_{-0.014}$\\
SNF20060609-002 & $ 86.3^{+3.5}_{-3.5}$ & $  8.1^{+1.2}_{-1.1}$ & $ 6132^{+  4}_{-  4}$ & $  0.18^{+  0.12}_{-  0.13}$ & $-0.069^{+0.033}_{-0.032}$  & $0.192^{+0.014}_{-0.014}$ & $-0.008^{+0.018}_{-0.017}$ & $-0.008^{+0.018}_{-0.017}$\\
SNF20060618-014 & $136.8^{+2.4}_{-2.5}$ & $  9.8^{+1.1}_{-1.1}$ & $ 6113^{+  7}_{-  7}$ & $  1.20^{+  0.18}_{-  0.18}$ & $0.077^{+0.047}_{-0.047}$  & $-0.037^{+0.014}_{-0.014}$ & $0.019^{+0.025}_{-0.024}$ & $0.019^{+0.025}_{-0.024}$\\
SNF20060618-023 & $ 73.3^{+4.7}_{-4.7}$ & $  5.3^{+1.6}_{-1.5}$ & $ 6140^{+ 18}_{- 18}$ & $  0.18^{+  0.16}_{-  0.16}$ & $-0.229^{+0.039}_{-0.038}$  & $0.059^{+0.020}_{-0.020}$ & $0.092^{+0.023}_{-0.022}$ & $0.092^{+0.023}_{-0.022}$\\
SNF20060621-015 & $111.8^{+1.3}_{-1.3}$ & $  9.9^{+0.7}_{-0.7}$ & $ 6144^{+  3}_{-  3}$ & $  0.52^{+  0.14}_{-  0.14}$ & $-0.015^{+0.026}_{-0.027}$  & $-0.067^{+0.008}_{-0.008}$ & $-0.020^{+0.013}_{-0.013}$ & $-0.020^{+0.013}_{-0.013}$\\
SNF20060907-000 & $ 87.8^{+7.5}_{-7.3}$ & $ 15.9^{+0.9}_{-0.9}$ & $ 6150^{+  4}_{-  4}$ & $  0.16^{+  0.17}_{-  0.17}$ & $0.037^{+0.033}_{-0.033}$  & $-0.079^{+0.005}_{-0.005}$ & $-0.041^{+0.008}_{-0.007}$ & $-0.041^{+0.008}_{-0.007}$\\
SNF20060908-004 & $114.4^{+1.2}_{-1.2}$ & $ 12.4^{+0.6}_{-0.6}$ & $ 6136^{+  3}_{-  3}$ & $  0.63^{+  0.15}_{-  0.15}$ & $-0.099^{+0.138}_{-0.081}$  & $-0.068^{+0.009}_{-0.008}$ & $0.021^{+0.017}_{-0.017}$ & $0.021^{+0.017}_{-0.017}$\\
SNF20060912-000 & $106.9^{+1.7}_{-1.7}$ & $ 19.8^{+1.5}_{-1.5}$ & $ 6158^{+  7}_{-  7}$ & $  0.06^{+  0.17}_{-  0.17}$ & $-0.177^{+0.032}_{-0.033}$  & $0.034^{+0.018}_{-0.019}$ & $0.080^{+0.021}_{-0.020}$ & $0.080^{+0.021}_{-0.020}$\\
SNF20061011-005 & $120.5^{+1.1}_{-1.1}$ & $  9.2^{+0.3}_{-0.3}$ & $ 6132^{+  4}_{-  4}$ & $  0.66^{+  0.22}_{-  0.21}$ & $-0.091^{+0.043}_{-0.041}$  & $-0.073^{+0.008}_{-0.007}$ & $0.003^{+0.021}_{-0.021}$ & $0.003^{+0.021}_{-0.021}$\\
SNF20061020-000 & $ 90.3^{+13.4}_{-13.3}$ & $ 23.9^{+0.9}_{-0.9}$ & $ 6120^{+  4}_{-  5}$ & $ -1.75^{+  0.17}_{-  0.17}$ & $-0.096^{+0.039}_{-0.034}$  & $0.000^{+0.016}_{-0.016}$ & $0.006^{+0.022}_{-0.020}$ & $0.006^{+0.022}_{-0.020}$\\
SNF20061021-003 & $122.5^{+2.2}_{-2.2}$ & $ 10.8^{+1.5}_{-1.5}$ & $ 6131^{+  4}_{-  4}$ & $  0.39^{+  0.15}_{-  0.15}$ & $-0.012^{+0.035}_{-0.033}$  & $0.036^{+0.014}_{-0.014}$ & $0.047^{+0.019}_{-0.017}$ & $0.047^{+0.019}_{-0.017}$\\
SNF20061022-005 & $ 63.4^{+3.6}_{-3.8}$ & $  3.9^{+1.4}_{-1.4}$ & $ 6147^{+  7}_{-  7}$ & $  2.03^{+  0.25}_{-  0.25}$ & $0.056^{+0.038}_{-0.038}$  & $0.041^{+0.018}_{-0.020}$ & $-0.000^{+0.022}_{-0.021}$ & $-0.000^{+0.022}_{-0.021}$\\
SNF20061024-000 & $ 98.9^{+15.7}_{-16.7}$ & $ 29.1^{+1.4}_{-1.4}$ & $ 6127^{+  5}_{-  5}$ & $ -1.51^{+  0.18}_{-  0.18}$ & $0.045^{+0.062}_{-0.059}$  & $-0.049^{+0.021}_{-0.019}$ & $0.003^{+0.031}_{-0.028}$ & $0.003^{+0.031}_{-0.028}$\\
SNF20061030-010 & $131.6^{+2.2}_{-2.2}$ & $ 16.9^{+1.1}_{-1.1}$ & $ 6116^{+  4}_{-  4}$ & $ -0.10^{+  0.20}_{-  0.20}$ & $0.103^{+0.041}_{-0.041}$  & $0.049^{+0.012}_{-0.012}$ & $0.019^{+0.021}_{-0.021}$ & $0.019^{+0.021}_{-0.021}$\\
SNF20061108-004 & $128.2^{+5.4}_{-5.4}$ & $  6.8^{+2.1}_{-2.1}$ & $ 6110^{+  6}_{-  6}$ & $  2.36^{+  0.27}_{-  0.27}$ & $-0.089^{+0.038}_{-0.036}$  & $-0.025^{+0.021}_{-0.023}$ & $-0.005^{+0.024}_{-0.022}$ & $-0.005^{+0.024}_{-0.022}$\\
SNF20061111-002 & $117.1^{+8.6}_{-8.8}$ & $ 19.8^{+0.9}_{-0.9}$ & $ 6144^{+  6}_{-  6}$ & $  0.07^{+  0.17}_{-  0.17}$ & $0.056^{+0.032}_{-0.032}$  & $-0.062^{+0.013}_{-0.011}$ & $0.026^{+0.017}_{-0.017}$ & $0.026^{+0.017}_{-0.017}$\\
SNF20070330-024 & $118.3^{+2.1}_{-2.1}$ & $  5.5^{+1.8}_{-1.8}$ & $ 6101^{+  3}_{-  3}$ & $  0.48^{+  0.19}_{-  0.21}$ & $-0.166^{+0.025}_{-0.024}$  & $-0.040^{+0.014}_{-0.014}$ & $-0.018^{+0.017}_{-0.016}$ & $-0.018^{+0.017}_{-0.016}$\\
SNF20070331-025 & $118.7^{+6.9}_{-7.0}$ & $ 15.4^{+2.0}_{-2.0}$ & $ 6121^{+ 10}_{- 10}$ & $ -1.22^{+  0.15}_{-  0.15}$ & $0.190^{+0.037}_{-0.037}$  & $-0.015^{+0.020}_{-0.020}$ & $0.001^{+0.023}_{-0.022}$ & $0.001^{+0.023}_{-0.022}$\\
SNF20070403-000 & $ 62.8^{+6.1}_{-6.2}$ & $ 27.6^{+1.6}_{-1.6}$ & $ 6154^{+  7}_{-  7}$ & $ -1.88^{+  0.17}_{-  0.17}$ & $0.147^{+0.040}_{-0.041}$  & $0.081^{+0.021}_{-0.021}$ & $0.041^{+0.025}_{-0.023}$ & $0.041^{+0.025}_{-0.023}$\\
SNF20070403-001 & $107.8^{+5.0}_{-5.1}$ & $ 17.1^{+1.4}_{-1.4}$ & $ 6123^{+  4}_{-  4}$ & $ -0.60^{+  0.12}_{-  0.12}$ & $0.007^{+0.029}_{-0.028}$  & $-0.036^{+0.014}_{-0.014}$ & $0.020^{+0.017}_{-0.017}$ & $0.020^{+0.017}_{-0.017}$\\
SNF20070417-002 & $103.8^{+5.0}_{-4.9}$ & $ 24.1^{+1.3}_{-1.3}$ & $ 6125^{+  8}_{-  9}$ & $ -1.37^{+  0.10}_{-  0.10}$ & $-0.032^{+0.050}_{-0.044}$  & $-0.072^{+0.010}_{-0.007}$ & $-0.039^{+0.010}_{-0.008}$ & $-0.039^{+0.010}_{-0.008}$\\
SNF20070424-003 & $123.6^{+3.6}_{-3.6}$ & $ 14.2^{+1.2}_{-1.1}$ & $ 6134^{+  6}_{-  6}$ & $  0.30^{+  0.10}_{-  0.10}$ & $0.008^{+0.023}_{-0.023}$  & $-0.010^{+0.010}_{-0.010}$ & $-0.041^{+0.008}_{-0.007}$ & $-0.041^{+0.008}_{-0.007}$\\
SNF20070427-001 & $ 80.7^{+2.3}_{-2.3}$ & $  6.5^{+0.8}_{-0.8}$ & $ 6143^{+  5}_{-  5}$ & $  0.82^{+  0.17}_{-  0.17}$ & $-0.009^{+0.031}_{-0.031}$  & $-0.066^{+0.011}_{-0.009}$ & $-0.018^{+0.017}_{-0.016}$ & $-0.018^{+0.017}_{-0.016}$\\
SNF20070506-006 & $ 94.3^{+1.3}_{-1.2}$ & $  6.4^{+0.6}_{-0.5}$ & $ 6152^{+  3}_{-  3}$ & $  1.02^{+  0.11}_{-  0.11}$ & $-0.075^{+0.027}_{-0.026}$  & $-0.029^{+0.009}_{-0.009}$ & $0.006^{+0.013}_{-0.013}$ & $0.006^{+0.013}_{-0.013}$\\
SNF20070531-011 & $122.2^{+2.6}_{-2.6}$ & $ 21.5^{+0.8}_{-0.8}$ & $ 6114^{+  3}_{-  4}$ & $ -1.56^{+  0.20}_{-  0.20}$ & $-0.081^{+0.027}_{-0.025}$  & $-0.032^{+0.011}_{-0.011}$ & $-0.000^{+0.015}_{-0.014}$ & $-0.000^{+0.015}_{-0.014}$\\
SNF20070630-006 & $126.3^{+3.1}_{-3.1}$ & $ 10.3^{+1.1}_{-1.2}$ & $ 6126^{+  4}_{-  4}$ & $  0.30^{+  0.11}_{-  0.11}$ & $-0.018^{+0.025}_{-0.025}$  & $-0.027^{+0.010}_{-0.010}$ & $-0.040^{+0.010}_{-0.008}$ & $-0.040^{+0.010}_{-0.008}$\\
SNF20070701-005 & $101.6^{+2.6}_{-2.6}$ & $ 13.6^{+0.9}_{-1.0}$ & $ 6155^{+  5}_{-  5}$ & $  0.39^{+  0.14}_{-  0.14}$ & $-0.187^{+0.021}_{-0.021}$  & $-0.012^{+0.011}_{-0.011}$ & $0.037^{+0.015}_{-0.015}$ & $0.037^{+0.015}_{-0.015}$\\
SNF20070712-003 & $108.1^{+2.7}_{-2.6}$ & $ 14.1^{+0.9}_{-0.9}$ & $ 6157^{+  6}_{-  6}$ & $  0.62^{+  0.14}_{-  0.14}$ & $-0.034^{+0.030}_{-0.029}$  & $-0.052^{+0.010}_{-0.011}$ & $-0.021^{+0.015}_{-0.014}$ & $-0.021^{+0.015}_{-0.014}$\\
SNF20070714-007 & $132.6^{+5.5}_{-5.5}$ & $ 33.5^{+4.3}_{-4.2}$ & $ 6145^{+  4}_{-  4}$ & $ -1.09^{+  0.13}_{-  0.13}$ & $-0.094^{+0.046}_{-0.047}$  & $0.118^{+0.035}_{-0.038}$ & $0.117^{+0.036}_{-0.032}$ & $0.117^{+0.036}_{-0.032}$\\
SNF20070725-001 & $108.3^{+1.9}_{-1.9}$ & $  9.5^{+1.0}_{-1.1}$ & $ 6139^{+  6}_{-  6}$ & $  0.61^{+  0.15}_{-  0.15}$ & $0.058^{+0.033}_{-0.033}$  & $-0.072^{+0.010}_{-0.007}$ & $-0.018^{+0.016}_{-0.015}$ & $-0.018^{+0.016}_{-0.015}$\\
SNF20070727-016 & $ 78.3^{+2.4}_{-2.5}$ & $  4.6^{+0.8}_{-0.8}$ & $ 6139^{+  4}_{-  4}$ & $  0.16^{+  0.11}_{-  0.10}$ & $-0.097^{+0.043}_{-0.036}$  & $-0.037^{+0.011}_{-0.011}$ & $-0.025^{+0.015}_{-0.014}$ & $-0.025^{+0.015}_{-0.014}$\\
SNF20070802-000 & $157.9^{+3.3}_{-3.3}$ & $ 16.0^{+1.4}_{-1.4}$ & $ 6103^{+  5}_{-  5}$ & $  0.07^{+  0.13}_{-  0.13}$ & $-0.074^{+0.028}_{-0.028}$  & $0.005^{+0.014}_{-0.014}$ & $0.005^{+0.016}_{-0.016}$ & $0.005^{+0.016}_{-0.016}$\\
SNF20070803-005 & $ 45.4^{+14.0}_{-13.8}$ & $  0.8^{+0.6}_{-0.6}$ & $ 6138^{+ 22}_{- 22}$ & $  0.81^{+  0.15}_{-  0.15}$ & $-0.137^{+0.031}_{-0.031}$  & $0.043^{+0.019}_{-0.019}$ & $-0.005^{+0.021}_{-0.021}$ & $-0.005^{+0.021}_{-0.021}$\\
SNF20070806-026 & $102.2^{+8.4}_{-9.0}$ & $ 25.3^{+0.7}_{-0.7}$ & $ 6112^{+  7}_{-  7}$ & $ -1.93^{+  0.14}_{-  0.14}$ & $0.010^{+0.032}_{-0.033}$  & $-0.071^{+0.011}_{-0.008}$ & $0.015^{+0.017}_{-0.017}$ & $0.015^{+0.017}_{-0.017}$\\
SNF20070810-004 & $127.0^{+1.8}_{-1.8}$ & $ 20.1^{+0.9}_{-0.9}$ & $ 6117^{+  7}_{-  7}$ & $ -0.01^{+  0.11}_{-  0.11}$ & $-0.044^{+0.026}_{-0.025}$  & $-0.068^{+0.011}_{-0.009}$ & $-0.033^{+0.012}_{-0.010}$ & $-0.033^{+0.012}_{-0.010}$\\
SNF20070817-003 & $ 94.5^{+2.4}_{-2.4}$ & $ 18.1^{+1.1}_{-1.1}$ & $ 6115^{+  6}_{-  6}$ & $ -0.61^{+  0.12}_{-  0.12}$ & $0.050^{+0.031}_{-0.032}$  & $-0.031^{+0.012}_{-0.012}$ & $-0.024^{+0.015}_{-0.014}$ & $-0.024^{+0.015}_{-0.014}$\\
SNF20070818-001 & $160.6^{+6.7}_{-6.8}$ & $ 16.4^{+1.5}_{-1.5}$ & $ 6113^{+  5}_{-  5}$ & $  0.14^{+  0.14}_{-  0.14}$ & $-0.178^{+0.021}_{-0.020}$  & $0.005^{+0.014}_{-0.014}$ & $0.014^{+0.016}_{-0.015}$ & $0.014^{+0.016}_{-0.015}$\\
SNF20070820-000 & $107.1^{+3.4}_{-3.4}$ & $ 18.7^{+1.2}_{-1.2}$ & $ 6133^{+ 13}_{- 13}$ & $ -0.28^{+  0.12}_{-  0.12}$ & $-0.054^{+0.031}_{-0.030}$  & $0.051^{+0.012}_{-0.013}$ & $-0.015^{+0.018}_{-0.016}$ & $-0.015^{+0.018}_{-0.016}$\\
SNF20070831-015 & $112.9^{+2.6}_{-2.6}$ & $  6.9^{+0.9}_{-0.9}$ & $ 6144^{+  6}_{-  6}$ & $  1.38^{+  0.16}_{-  0.16}$ & $0.042^{+0.027}_{-0.027}$  & $-0.016^{+0.010}_{-0.010}$ & $0.002^{+0.014}_{-0.014}$ & $0.002^{+0.014}_{-0.014}$\\
SNF20070902-018 & $108.6^{+9.5}_{-9.7}$ & $ 18.1^{+1.7}_{-1.7}$ & $ 6119^{+  7}_{-  7}$ & $ -0.81^{+  0.10}_{-  0.09}$ & $0.032^{+0.030}_{-0.030}$  & $0.023^{+0.016}_{-0.017}$ & $-0.035^{+0.014}_{-0.010}$ & $-0.035^{+0.014}_{-0.010}$\\
SNF20070902-021 & $109.0^{+3.3}_{-3.3}$ & $ 16.8^{+0.9}_{-0.9}$ & $ 6131^{+  5}_{-  5}$ & $ -0.46^{+  0.09}_{-  0.09}$ & $-0.082^{+0.027}_{-0.026}$  & $-0.023^{+0.010}_{-0.010}$ & $-0.028^{+0.014}_{-0.012}$ & $-0.028^{+0.014}_{-0.012}$\\
SNF20071003-016 & $126.5^{+4.5}_{-4.4}$ & $ 16.7^{+1.5}_{-1.5}$ & $ 6123^{+ 10}_{- 10}$ & $ -0.30^{+  0.14}_{-  0.14}$ & $0.058^{+0.032}_{-0.032}$  & $0.068^{+0.013}_{-0.014}$ & $-0.030^{+0.016}_{-0.012}$ & $-0.030^{+0.016}_{-0.012}$\\
SNF20071015-000 & $104.8^{+3.1}_{-3.1}$ & $  6.5^{+1.1}_{-1.1}$ & $ 6125^{+  7}_{-  7}$ & $  0.90^{+  0.28}_{-  0.29}$ & $0.116^{+0.040}_{-0.040}$  & $0.293^{+0.016}_{-0.017}$ & $-0.018^{+0.021}_{-0.017}$ & $-0.018^{+0.021}_{-0.017}$\\
SNF20071021-000 & $167.9^{+2.2}_{-2.2}$ & $ 20.2^{+0.6}_{-0.6}$ & $ 6111^{+  4}_{-  4}$ & $ -0.66^{+  0.13}_{-  0.13}$ & $-0.105^{+0.025}_{-0.023}$  & $-0.001^{+0.010}_{-0.010}$ & $-0.004^{+0.013}_{-0.013}$ & $-0.004^{+0.013}_{-0.013}$\\
SNF20071108-021 & $ 98.7^{+2.7}_{-2.7}$ & $  6.0^{+0.8}_{-0.8}$ & $ 6165^{+  5}_{-  5}$ & $  1.15^{+  0.16}_{-  0.15}$ & $-0.045^{+0.028}_{-0.028}$  & $-0.028^{+0.012}_{-0.012}$ & $0.024^{+0.015}_{-0.014}$ & $0.024^{+0.015}_{-0.014}$\\
SNF20080323-009 & $ 96.0^{+2.2}_{-2.2}$ & $ 11.1^{+1.0}_{-1.0}$ & $ 6145^{+  5}_{-  5}$ & $ -0.15^{+  0.14}_{-  0.14}$ & $-0.029^{+0.030}_{-0.029}$  & $-0.051^{+0.011}_{-0.011}$ & $-0.041^{+0.009}_{-0.008}$ & $-0.041^{+0.009}_{-0.008}$\\
SNF20080507-000 & $ 98.4^{+1.6}_{-1.6}$ & $  8.0^{+1.7}_{-1.6}$ & $ 6142^{+  5}_{-  5}$ & $  0.56^{+  0.18}_{-  0.18}$ & $-0.126^{+0.023}_{-0.022}$  & $0.063^{+0.014}_{-0.013}$ & $-0.003^{+0.015}_{-0.014}$ & $-0.003^{+0.015}_{-0.014}$\\
SNF20080510-001 & $118.9^{+2.0}_{-2.1}$ & $ 15.0^{+1.1}_{-1.0}$ & $ 6115^{+  4}_{-  4}$ & $  0.34^{+  0.14}_{-  0.14}$ & $0.001^{+0.026}_{-0.026}$  & $-0.058^{+0.011}_{-0.012}$ & $-0.032^{+0.012}_{-0.011}$ & $-0.032^{+0.012}_{-0.011}$\\
SNF20080510-005 & $112.2^{+2.5}_{-2.5}$ & $  7.0^{+1.0}_{-1.0}$ & $ 6117^{+  4}_{-  4}$ & $  1.68^{+  0.22}_{-  0.22}$ & $0.116^{+0.030}_{-0.030}$  & $-0.029^{+0.013}_{-0.015}$ & $-0.042^{+0.008}_{-0.007}$ & $-0.042^{+0.008}_{-0.007}$\\
SNF20080512-010 & $ 96.0^{+3.4}_{-3.4}$ & $ 21.9^{+1.3}_{-1.3}$ & $ 6128^{+  5}_{-  5}$ & $ -1.64^{+  0.18}_{-  0.18}$ & $-0.032^{+0.072}_{-0.061}$  & $-0.051^{+0.013}_{-0.012}$ & $0.018^{+0.016}_{-0.016}$ & $0.018^{+0.016}_{-0.016}$\\
SNF20080514-002 & $ 83.1^{+0.7}_{-0.7}$ & $ 19.2^{+0.5}_{-0.5}$ & $ 6131^{+  3}_{-  3}$ & $ -1.55^{+  0.09}_{-  0.09}$ & $0.174^{+0.019}_{-0.019}$  & $-0.074^{+0.007}_{-0.006}$ & $0.006^{+0.011}_{-0.012}$ & $0.006^{+0.011}_{-0.012}$\\
SNF20080516-000 & $117.6^{+2.1}_{-2.1}$ & $  8.2^{+1.2}_{-1.1}$ & $ 6135^{+  3}_{-  3}$ & $  0.96^{+  0.18}_{-  0.18}$ & $0.084^{+0.029}_{-0.028}$  & $-0.054^{+0.011}_{-0.010}$ & $0.004^{+0.015}_{-0.015}$ & $0.004^{+0.015}_{-0.015}$\\
SNF20080516-022 & $ 99.1^{+2.2}_{-2.1}$ & $ 15.3^{+0.9}_{-0.9}$ & $ 6159^{+  3}_{-  3}$ & $  0.48^{+  0.12}_{-  0.11}$ & $0.001^{+0.026}_{-0.026}$  & $-0.064^{+0.010}_{-0.009}$ & $-0.025^{+0.014}_{-0.013}$ & $-0.025^{+0.014}_{-0.013}$\\
SNF20080522-000 & $ 62.1^{+3.5}_{-3.5}$ & $  3.1^{+0.8}_{-0.8}$ & $ 6130^{+  7}_{-  7}$ & $  0.99^{+  0.11}_{-  0.11}$ & $-0.021^{+0.028}_{-0.028}$  & $-0.009^{+0.012}_{-0.013}$ & $0.047^{+0.014}_{-0.013}$ & $0.047^{+0.014}_{-0.013}$\\
SNF20080522-011 & $122.1^{+1.7}_{-1.7}$ & $  8.3^{+0.5}_{-0.5}$ & $ 6125^{+  2}_{-  2}$ & $  0.82^{+  0.13}_{-  0.13}$ & $-0.124^{+0.023}_{-0.023}$  & $-0.045^{+0.008}_{-0.008}$ & $-0.031^{+0.011}_{-0.010}$ & $-0.031^{+0.011}_{-0.010}$\\
SNF20080531-000 & $133.2^{+1.5}_{-1.5}$ & $ 17.2^{+0.8}_{-0.8}$ & $ 6113^{+  5}_{-  5}$ & $ -0.63^{+  0.13}_{-  0.13}$ & $0.002^{+0.024}_{-0.024}$  & $-0.040^{+0.009}_{-0.009}$ & $-0.009^{+0.012}_{-0.013}$ & $-0.009^{+0.012}_{-0.013}$\\
SNF20080610-000 & $119.5^{+8.6}_{-8.8}$ & $ 17.0^{+1.4}_{-1.4}$ & $ 6131^{+  6}_{-  6}$ & $  0.16^{+  0.14}_{-  0.14}$ & $0.013^{+0.068}_{-0.068}$  & $-0.019^{+0.014}_{-0.015}$ & $-0.027^{+0.017}_{-0.013}$ & $-0.027^{+0.017}_{-0.013}$\\
SNF20080612-003 & $120.0^{+1.1}_{-1.1}$ & $  7.5^{+0.6}_{-0.6}$ & $ 6122^{+  3}_{-  3}$ & $  0.96^{+  0.15}_{-  0.15}$ & $-0.221^{+0.018}_{-0.017}$  & $-0.015^{+0.007}_{-0.007}$ & $-0.001^{+0.013}_{-0.014}$ & $-0.001^{+0.013}_{-0.014}$\\
SNF20080614-010 & $118.9^{+4.9}_{-4.8}$ & $ 28.9^{+1.2}_{-1.2}$ & $ 6129^{+  3}_{-  3}$ & $ -1.81^{+  0.15}_{-  0.15}$ & $-0.114^{+0.046}_{-0.042}$  & $-0.068^{+0.012}_{-0.009}$ & $0.049^{+0.019}_{-0.019}$ & $0.049^{+0.019}_{-0.019}$\\
SNF20080620-000 & $125.3^{+9.5}_{-9.7}$ & $ 20.6^{+0.6}_{-0.6}$ & $ 6133^{+  3}_{-  3}$ & $ -0.97^{+  0.14}_{-  0.14}$ & $-0.007^{+0.029}_{-0.028}$  & $0.001^{+0.012}_{-0.012}$ & $-0.043^{+0.007}_{-0.007}$ & $-0.043^{+0.007}_{-0.007}$\\
SNF20080623-001 & $149.2^{+1.4}_{-1.4}$ & $ 14.8^{+0.6}_{-0.6}$ & $ 6131^{+  3}_{-  3}$ & $ -0.07^{+  0.11}_{-  0.10}$ & $0.072^{+0.025}_{-0.025}$  & $-0.054^{+0.010}_{-0.009}$ & $-0.034^{+0.011}_{-0.010}$ & $-0.034^{+0.011}_{-0.010}$\\
SNF20080626-002 & $130.0^{+1.0}_{-1.0}$ & $  5.8^{+2.1}_{-2.0}$ & $ 6111^{+  3}_{-  3}$ & $  1.03^{+  0.19}_{-  0.19}$ & $-0.043^{+0.026}_{-0.025}$  & $-0.009^{+0.015}_{-0.016}$ & $-0.022^{+0.017}_{-0.015}$ & $-0.022^{+0.017}_{-0.015}$\\
SNF20080714-008 & $155.6^{+11.0}_{-10.8}$ & $ 16.6^{+1.9}_{-1.8}$ & $ 6099^{+  6}_{-  6}$ & $ -0.07^{+  0.14}_{-  0.14}$ & $-0.141^{+0.024}_{-0.022}$  & $0.082^{+0.018}_{-0.020}$ & $-0.029^{+0.017}_{-0.013}$ & $-0.029^{+0.017}_{-0.013}$\\
SNF20080717-000 & $ 92.7^{+2.6}_{-2.6}$ & $ 12.4^{+1.7}_{-1.7}$ & $ 6105^{+  3}_{-  3}$ & $  0.87^{+  0.13}_{-  0.12}$ & $0.115^{+0.034}_{-0.033}$  & $0.126^{+0.018}_{-0.019}$ & $0.050^{+0.021}_{-0.019}$ & $0.050^{+0.021}_{-0.019}$\\
SNF20080720-001 & $138.3^{+3.9}_{-4.0}$ & $ 14.0^{+1.5}_{-1.5}$ & $ 6107^{+  3}_{-  3}$ & $  0.19^{+  0.12}_{-  0.12}$ & $-0.176^{+0.028}_{-0.028}$  & $0.325^{+0.018}_{-0.018}$ & $0.007^{+0.019}_{-0.018}$ & $0.007^{+0.019}_{-0.018}$\\
SNF20080725-004 & $133.8^{+2.1}_{-2.1}$ & $  6.7^{+0.9}_{-0.9}$ & $ 6131^{+  6}_{-  5}$ & $  0.59^{+  0.13}_{-  0.13}$ & $0.078^{+0.032}_{-0.032}$  & $0.025^{+0.012}_{-0.011}$ & $0.016^{+0.015}_{-0.014}$ & $0.016^{+0.015}_{-0.014}$\\
SNF20080802-006 & $108.7^{+5.6}_{-5.8}$ & $ 19.1^{+1.6}_{-1.6}$ & $ 6122^{+  5}_{-  5}$ & $ -1.61^{+  0.12}_{-  0.12}$ & $-0.042^{+0.059}_{-0.055}$  & $-0.001^{+0.016}_{-0.016}$ & $0.033^{+0.020}_{-0.020}$ & $0.033^{+0.020}_{-0.020}$\\
SNF20080803-000 & $118.3^{+2.5}_{-2.6}$ & $  8.4^{+1.3}_{-1.4}$ & $ 6125^{+  4}_{-  4}$ & $  0.25^{+  0.12}_{-  0.12}$ & $-0.030^{+0.023}_{-0.024}$  & $0.100^{+0.011}_{-0.011}$ & $-0.040^{+0.009}_{-0.008}$ & $-0.040^{+0.009}_{-0.008}$\\
SNF20080806-002 & $135.4^{+1.8}_{-1.9}$ & $  8.3^{+0.9}_{-0.9}$ & $ 6136^{+  4}_{-  4}$ & $  0.83^{+  0.14}_{-  0.14}$ & $-0.106^{+0.028}_{-0.027}$  & $0.020^{+0.010}_{-0.010}$ & $-0.036^{+0.013}_{-0.009}$ & $-0.036^{+0.013}_{-0.009}$\\
SNF20080810-001 & $ 98.8^{+12.2}_{-12.3}$ & $ 21.7^{+1.0}_{-1.0}$ & $ 6145^{+  5}_{-  5}$ & $ -0.93^{+  0.10}_{-  0.11}$ & $-0.058^{+0.024}_{-0.024}$  & $-0.023^{+0.016}_{-0.016}$ & $-0.000^{+0.017}_{-0.016}$ & $-0.000^{+0.017}_{-0.016}$\\
SNF20080815-017 & $ 90.9^{+14.3}_{-14.5}$ & $ 19.6^{+2.4}_{-2.3}$ & $ 6131^{+  6}_{-  6}$ & $ -1.96^{+  0.16}_{-  0.16}$ & $0.041^{+0.070}_{-0.064}$  & $-0.008^{+0.025}_{-0.026}$ & $-0.011^{+0.028}_{-0.022}$ & $-0.011^{+0.028}_{-0.022}$\\
SNF20080821-000 & $104.8^{+2.2}_{-2.2}$ & $  8.8^{+1.2}_{-1.2}$ & $ 6121^{+  4}_{-  4}$ & $  0.77^{+  0.15}_{-  0.14}$ & $-0.104^{+0.023}_{-0.023}$  & $0.025^{+0.011}_{-0.011}$ & $-0.006^{+0.013}_{-0.013}$ & $-0.006^{+0.013}_{-0.013}$\\
SNF20080822-005 & $ 78.6^{+1.8}_{-1.8}$ & $  6.2^{+0.8}_{-0.8}$ & $ 6138^{+  4}_{-  4}$ & $  0.51^{+  0.16}_{-  0.16}$ & $-0.027^{+0.030}_{-0.030}$  & $-0.018^{+0.011}_{-0.010}$ & $0.014^{+0.015}_{-0.015}$ & $0.014^{+0.015}_{-0.015}$\\
SNF20080825-010 & $ 96.4^{+10.2}_{-10.2}$ & $ 19.2^{+0.6}_{-0.6}$ & $ 6116^{+  4}_{-  4}$ & $ -0.76^{+  0.13}_{-  0.13}$ & $-0.192^{+0.020}_{-0.020}$  & $-0.045^{+0.013}_{-0.013}$ & $0.002^{+0.013}_{-0.013}$ & $0.002^{+0.013}_{-0.013}$\\
SNF20080909-030 & $ 93.8^{+1.0}_{-0.9}$ & $  7.7^{+0.4}_{-0.4}$ & $ 6171^{+  3}_{-  3}$ & $  0.99^{+  0.13}_{-  0.13}$ & $0.030^{+0.030}_{-0.030}$  & $0.008^{+0.013}_{-0.012}$ & $0.008^{+0.015}_{-0.014}$ & $0.008^{+0.015}_{-0.014}$\\
SNF20080913-031 & $117.9^{+1.5}_{-1.5}$ & $ 14.7^{+1.6}_{-1.6}$ & $ 6159^{+  5}_{-  5}$ & $  0.49^{+  0.17}_{-  0.17}$ & $-0.048^{+0.067}_{-0.052}$  & $-0.003^{+0.015}_{-0.015}$ & $-0.009^{+0.020}_{-0.018}$ & $-0.009^{+0.020}_{-0.018}$\\
SNF20080914-001 & $126.4^{+1.2}_{-1.2}$ & $ 16.3^{+1.0}_{-1.0}$ & $ 6159^{+  3}_{-  3}$ & $  0.10^{+  0.14}_{-  0.14}$ & $-0.144^{+0.026}_{-0.024}$  & $0.094^{+0.012}_{-0.012}$ & $0.022^{+0.016}_{-0.015}$ & $0.022^{+0.016}_{-0.015}$\\
SNF20080918-000 & $146.6^{+3.5}_{-3.5}$ & $  5.6^{+2.1}_{-2.0}$ & $ 6110^{+  5}_{-  5}$ & $  1.28^{+  0.23}_{-  0.23}$ & $0.112^{+0.038}_{-0.037}$  & $0.073^{+0.019}_{-0.019}$ & $-0.004^{+0.023}_{-0.022}$ & $-0.004^{+0.023}_{-0.022}$\\
SNF20080918-002 & $ 96.6^{+2.8}_{-2.7}$ & $ 13.3^{+1.2}_{-1.2}$ & $ 6143^{+  6}_{-  6}$ & $ -0.47^{+  0.08}_{-  0.08}$ & $0.021^{+0.034}_{-0.035}$  & $-0.056^{+0.014}_{-0.013}$ & $-0.005^{+0.020}_{-0.021}$ & $-0.005^{+0.020}_{-0.021}$\\
SNF20080918-004 & $ 91.9^{+6.6}_{-6.6}$ & $ 21.2^{+0.9}_{-0.9}$ & $ 6141^{+  4}_{-  4}$ & $ -1.37^{+  0.15}_{-  0.15}$ & $0.035^{+0.142}_{-0.087}$  & $-0.021^{+0.014}_{-0.013}$ & $-0.012^{+0.019}_{-0.018}$ & $-0.012^{+0.019}_{-0.018}$\\
SNF20080919-000 & $115.6^{+2.7}_{-2.7}$ & $  8.8^{+0.9}_{-0.8}$ & $ 6144^{+  5}_{-  5}$ & $ -0.24^{+  0.10}_{-  0.10}$ & $-0.043^{+0.027}_{-0.026}$  & $0.165^{+0.013}_{-0.012}$ & $-0.035^{+0.013}_{-0.010}$ & $-0.035^{+0.013}_{-0.010}$\\
SNF20080919-001 & $ 85.0^{+1.1}_{-1.1}$ & $  6.1^{+0.4}_{-0.4}$ & $ 6149^{+  4}_{-  4}$ & $  1.05^{+  0.12}_{-  0.12}$ & $-0.183^{+0.020}_{-0.019}$  & $0.001^{+0.008}_{-0.008}$ & $0.006^{+0.012}_{-0.012}$ & $0.006^{+0.012}_{-0.012}$\\
SNF20080919-002 & $100.4^{+6.7}_{-6.6}$ & $ 26.3^{+1.7}_{-1.7}$ & $ 6133^{+  8}_{-  8}$ & $ -1.54^{+  0.13}_{-  0.13}$ & $0.340^{+0.043}_{-0.044}$  & $-0.058^{+0.017}_{-0.014}$ & $0.114^{+0.020}_{-0.021}$ & $0.114^{+0.020}_{-0.021}$\\
SNF20080920-000 & $135.1^{+1.4}_{-1.4}$ & $  5.3^{+1.5}_{-1.5}$ & $ 6085^{+  3}_{-  3}$ & $  1.08^{+  0.23}_{-  0.23}$ & $-0.002^{+0.038}_{-0.037}$  & $-0.020^{+0.017}_{-0.017}$ & $-0.012^{+0.020}_{-0.019}$ & $-0.012^{+0.020}_{-0.019}$\\
CSS110918\_01 & $110.4^{+1.0}_{-1.0}$ & $  5.0^{+0.9}_{-0.9}$ & $ 6101^{+  2}_{-  2}$ & $  0.82^{+  0.13}_{-  0.13}$ & $-0.077^{+0.187}_{-0.094}$  & $-0.078^{+0.006}_{-0.005}$ & $-0.028^{+0.013}_{-0.012}$ & $-0.028^{+0.013}_{-0.012}$\\
CSS110918\_02 & $109.1^{+8.1}_{-8.1}$ & $ 15.0^{+0.6}_{-0.6}$ & $ 6137^{+  3}_{-  3}$ & $ -0.26^{+  0.13}_{-  0.13}$ & $-0.052^{+0.027}_{-0.027}$  & $-0.039^{+0.010}_{-0.010}$ & $-0.009^{+0.014}_{-0.014}$ & $-0.009^{+0.014}_{-0.014}$\\
CSS120424\_01 & $138.5^{+2.1}_{-2.1}$ & $ 11.5^{+0.6}_{-0.6}$ & $ 6138^{+  3}_{-  3}$ & $  0.58^{+  0.14}_{-  0.14}$ & $-0.010^{+0.028}_{-0.028}$  & $-0.061^{+0.009}_{-0.009}$ & $-0.024^{+0.013}_{-0.012}$ & $-0.024^{+0.013}_{-0.012}$\\
CSS130502\_01 & $ 87.3^{+8.0}_{-8.7}$ & $ 15.4^{+0.5}_{-0.5}$ & $ 6128^{+  3}_{-  3}$ & $ -0.15^{+  0.20}_{-  0.20}$ & $0.121^{+0.032}_{-0.032}$  & $-0.069^{+0.011}_{-0.009}$ & $-0.017^{+0.015}_{-0.015}$ & $-0.017^{+0.015}_{-0.015}$\\
LSQ12dbr & $106.9^{+0.6}_{-0.6}$ & $  6.4^{+0.6}_{-0.6}$ & $ 6137^{+  4}_{-  4}$ & $  0.86^{+  0.11}_{-  0.11}$ & $0.004^{+0.226}_{-0.109}$  & $-0.071^{+0.008}_{-0.007}$ & $-0.034^{+0.011}_{-0.010}$ & $-0.034^{+0.011}_{-0.010}$\\
LSQ12fhe & $ 42.9^{+1.2}_{-1.2}$ & $  2.3^{+1.9}_{-1.9}$ & $ 6108^{+  4}_{-  4}$ & $  1.01^{+  0.17}_{-  0.17}$ & $-0.183^{+0.032}_{-0.033}$  & $0.067^{+0.017}_{-0.018}$ & $-0.017^{+0.018}_{-0.016}$ & $-0.017^{+0.018}_{-0.016}$\\
LSQ12fxd & $122.8^{+1.6}_{-1.7}$ & $ 11.7^{+0.8}_{-0.7}$ & $ 6119^{+  4}_{-  4}$ & $  0.16^{+  0.12}_{-  0.12}$ & $-0.154^{+0.044}_{-0.032}$  & $-0.052^{+0.008}_{-0.008}$ & $0.002^{+0.013}_{-0.013}$ & $0.002^{+0.013}_{-0.013}$\\
LSQ12hjm & $ 65.3^{+10.8}_{-11.3}$ & $ 15.3^{+1.1}_{-1.1}$ & $ 6147^{+  5}_{-  5}$ & $ -0.16^{+  0.17}_{-  0.17}$ & $0.112^{+0.030}_{-0.031}$  & $-0.062^{+0.013}_{-0.011}$ & $-0.042^{+0.008}_{-0.007}$ & $-0.042^{+0.008}_{-0.007}$\\
PTF09dlc & $143.8^{+2.1}_{-2.1}$ & $  9.7^{+0.8}_{-0.8}$ & $ 6143^{+  3}_{-  3}$ & $  0.01^{+  0.10}_{-  0.10}$ & $0.008^{+0.027}_{-0.026}$  & $-0.054^{+0.012}_{-0.012}$ & $-0.017^{+0.014}_{-0.015}$ & $-0.017^{+0.014}_{-0.015}$\\
PTF09dnl & $130.0^{+0.8}_{-0.9}$ & $  9.1^{+0.7}_{-0.7}$ & $ 6092^{+  3}_{-  3}$ & $  0.59^{+  0.10}_{-  0.10}$ & $-0.061^{+0.021}_{-0.021}$  & $0.014^{+0.009}_{-0.009}$ & $0.000^{+0.011}_{-0.011}$ & $0.000^{+0.011}_{-0.011}$\\
PTF09dnp & $ 66.9^{+4.5}_{-4.6}$ & $ 17.5^{+0.7}_{-0.7}$ & $ 6098^{+  5}_{-  4}$ & $ -0.53^{+  0.17}_{-  0.17}$ & $-0.249^{+0.026}_{-0.026}$  & $-0.008^{+0.020}_{-0.021}$ & $-0.044^{+0.006}_{-0.006}$ & $-0.044^{+0.006}_{-0.006}$\\
PTF09fox & $118.5^{+2.7}_{-2.7}$ & $  9.8^{+0.9}_{-0.8}$ & $ 6117^{+  3}_{-  3}$ & $  0.52^{+  0.16}_{-  0.15}$ & $0.027^{+0.039}_{-0.039}$  & $-0.047^{+0.009}_{-0.009}$ & $-0.043^{+0.007}_{-0.006}$ & $-0.043^{+0.007}_{-0.006}$\\
PTF09foz & $127.2^{+1.9}_{-1.9}$ & $ 22.1^{+1.1}_{-1.1}$ & $ 6135^{+  4}_{-  4}$ & $ -0.92^{+  0.16}_{-  0.16}$ & $-0.141^{+0.019}_{-0.018}$  & $-0.042^{+0.009}_{-0.010}$ & $0.031^{+0.013}_{-0.012}$ & $0.031^{+0.013}_{-0.012}$\\
PTF10hmv & $109.8^{+1.3}_{-1.4}$ & $  8.5^{+0.7}_{-0.6}$ & $ 6143^{+  3}_{-  3}$ & $  1.15^{+  0.13}_{-  0.12}$ & $0.284^{+0.024}_{-0.023}$  & $0.106^{+0.009}_{-0.009}$ & $-0.010^{+0.012}_{-0.012}$ & $-0.010^{+0.012}_{-0.012}$\\
PTF10icb & $104.8^{+0.9}_{-0.9}$ & $ 12.6^{+0.3}_{-0.3}$ & $ 6138^{+  2}_{-  2}$ & $  0.23^{+  0.10}_{-  0.10}$ & $0.410^{+0.024}_{-0.025}$  & $0.046^{+0.006}_{-0.006}$ & $-0.019^{+0.010}_{-0.010}$ & $-0.019^{+0.010}_{-0.010}$\\
PTF10mwb & $116.2^{+1.2}_{-1.2}$ & $ 21.1^{+0.7}_{-0.7}$ & $ 6139^{+  2}_{-  2}$ & $ -0.48^{+  0.09}_{-  0.09}$ & $-0.000^{+0.071}_{-0.070}$  & $-0.036^{+0.008}_{-0.008}$ & $-0.011^{+0.011}_{-0.011}$ & $-0.011^{+0.011}_{-0.011}$\\
PTF10ndc & $124.1^{+2.4}_{-2.4}$ & $  6.7^{+1.0}_{-1.0}$ & $ 6119^{+  3}_{-  3}$ & $  0.86^{+  0.14}_{-  0.14}$ & $0.007^{+0.026}_{-0.025}$  & $-0.040^{+0.011}_{-0.011}$ & $-0.023^{+0.014}_{-0.013}$ & $-0.023^{+0.014}_{-0.013}$\\
PTF10ops & $ 38.0^{+9.2}_{-9.2}$ & $ 21.9^{+4.6}_{-4.6}$ & $ 6140^{+  6}_{-  5}$ & $ -0.58^{+  0.12}_{-  0.12}$ & $-0.091^{+0.221}_{-0.154}$  & $0.190^{+0.040}_{-0.041}$ & $0.181^{+0.038}_{-0.036}$ & $0.181^{+0.038}_{-0.036}$\\
PTF10qjq & $ 74.4^{+2.3}_{-2.4}$ & $ 12.0^{+0.7}_{-0.7}$ & $ 6132^{+  3}_{-  3}$ & $ -0.50^{+  0.10}_{-  0.09}$ & $-0.127^{+0.026}_{-0.024}$  & $0.049^{+0.010}_{-0.011}$ & $-0.010^{+0.015}_{-0.015}$ & $-0.010^{+0.015}_{-0.015}$\\
PTF10qyz & $108.1^{+2.1}_{-2.1}$ & $ 21.0^{+0.8}_{-0.8}$ & $ 6117^{+  5}_{-  5}$ & $ -1.67^{+  0.13}_{-  0.13}$ & $0.099^{+0.133}_{-0.095}$  & $-0.067^{+0.014}_{-0.010}$ & $-0.038^{+0.012}_{-0.008}$ & $-0.038^{+0.012}_{-0.008}$\\
PTF10tce & $135.8^{+1.1}_{-1.1}$ & $ 10.6^{+1.2}_{-1.2}$ & $ 6090^{+  4}_{-  4}$ & $  0.38^{+  0.12}_{-  0.12}$ & $0.041^{+0.028}_{-0.027}$  & $-0.005^{+0.013}_{-0.014}$ & $-0.019^{+0.015}_{-0.015}$ & $-0.019^{+0.015}_{-0.015}$\\
PTF10ufj & $141.4^{+3.3}_{-3.3}$ & $ 12.0^{+1.1}_{-1.0}$ & $ 6131^{+  6}_{-  5}$ & $  0.56^{+  0.13}_{-  0.13}$ & $-0.043^{+0.106}_{-0.068}$  & $-0.040^{+0.011}_{-0.011}$ & $-0.030^{+0.014}_{-0.011}$ & $-0.030^{+0.014}_{-0.011}$\\
PTF10wnm & $105.5^{+2.3}_{-2.3}$ & $  6.8^{+0.9}_{-0.9}$ & $ 6125^{+  3}_{-  3}$ & $  0.73^{+  0.12}_{-  0.12}$ & $0.123^{+0.022}_{-0.021}$  & $-0.014^{+0.009}_{-0.009}$ & $0.015^{+0.012}_{-0.012}$ & $0.015^{+0.012}_{-0.012}$\\
PTF10wof & $130.5^{+2.6}_{-2.7}$ & $ 16.5^{+0.9}_{-0.9}$ & $ 6102^{+  2}_{-  2}$ & $  0.06^{+  0.11}_{-  0.11}$ & $0.096^{+0.023}_{-0.023}$  & $-0.009^{+0.011}_{-0.011}$ & $-0.013^{+0.013}_{-0.012}$ & $-0.013^{+0.013}_{-0.012}$\\
PTF10xyt & $120.2^{+6.3}_{-6.4}$ & $ 21.5^{+3.2}_{-3.1}$ & $ 6102^{+  4}_{-  4}$ & $  0.18^{+  0.20}_{-  0.20}$ & $0.241^{+0.049}_{-0.049}$  & $0.070^{+0.026}_{-0.028}$ & $0.014^{+0.029}_{-0.027}$ & $0.014^{+0.029}_{-0.027}$\\
PTF10zdk & $149.7^{+1.2}_{-1.2}$ & $ 14.9^{+0.6}_{-0.6}$ & $ 6150^{+  3}_{-  3}$ & $  0.37^{+  0.14}_{-  0.13}$ & $-0.029^{+0.026}_{-0.025}$  & $0.062^{+0.012}_{-0.012}$ & $-0.042^{+0.007}_{-0.007}$ & $-0.042^{+0.007}_{-0.007}$\\
PTF11bgv & $ 78.3^{+3.1}_{-3.1}$ & $ 12.7^{+0.6}_{-0.6}$ & $ 6146^{+  3}_{-  3}$ & $ -0.27^{+  0.11}_{-  0.10}$ & $0.057^{+0.028}_{-0.028}$  & $0.081^{+0.010}_{-0.010}$ & $0.006^{+0.015}_{-0.015}$ & $0.006^{+0.015}_{-0.015}$\\
PTF11bju & $ 31.3^{+4.3}_{-4.2}$ & $  3.3^{+2.0}_{-2.0}$ & $ 6138^{+  5}_{-  5}$ & $  1.02^{+  0.17}_{-  0.18}$ & $0.073^{+0.036}_{-0.036}$  & $0.077^{+0.019}_{-0.019}$ & $-0.014^{+0.020}_{-0.018}$ & $-0.014^{+0.020}_{-0.018}$\\
PTF11bnx & $150.7^{+3.0}_{-2.9}$ & $ 14.6^{+1.0}_{-1.0}$ & $ 6143^{+  5}_{-  5}$ & $  0.07^{+  0.14}_{-  0.14}$ & $0.043^{+0.029}_{-0.029}$  & $0.049^{+0.011}_{-0.011}$ & $-0.030^{+0.014}_{-0.011}$ & $-0.030^{+0.014}_{-0.011}$\\
PTF11cao & $143.4^{+1.5}_{-1.6}$ & $ 19.0^{+1.1}_{-1.1}$ & $ 6104^{+  4}_{-  4}$ & $  0.20^{+  0.13}_{-  0.13}$ & $-0.014^{+0.032}_{-0.031}$  & $0.009^{+0.013}_{-0.013}$ & $-0.017^{+0.017}_{-0.015}$ & $-0.017^{+0.017}_{-0.015}$\\
PTF11drz & $132.6^{+1.4}_{-1.4}$ & $ 14.9^{+0.9}_{-0.9}$ & $ 6116^{+  4}_{-  4}$ & $ -0.11^{+  0.17}_{-  0.18}$ & $-0.007^{+0.025}_{-0.026}$  & $-0.021^{+0.010}_{-0.009}$ & $-0.016^{+0.013}_{-0.013}$ & $-0.016^{+0.013}_{-0.013}$\\
PTF11mkx & $ 34.8^{+3.5}_{-3.5}$ & $  5.5^{+1.3}_{-1.3}$ & $ 6169^{+  5}_{-  5}$ & $  0.53^{+  0.14}_{-  0.14}$ & $0.022^{+0.135}_{-0.080}$  & $0.102^{+0.017}_{-0.016}$ & $-0.041^{+0.008}_{-0.007}$ & $-0.041^{+0.008}_{-0.007}$\\
PTF11mty & $110.5^{+2.2}_{-2.2}$ & $ 11.3^{+1.0}_{-1.1}$ & $ 6140^{+  5}_{-  5}$ & $  1.03^{+  0.13}_{-  0.13}$ & $0.062^{+0.025}_{-0.024}$  & $-0.072^{+0.009}_{-0.007}$ & $0.024^{+0.013}_{-0.014}$ & $0.024^{+0.013}_{-0.014}$\\
PTF11pdk & $127.7^{+2.6}_{-2.7}$ & $ 16.1^{+1.2}_{-1.2}$ & $ 6154^{+  5}_{-  5}$ & $  0.16^{+  0.16}_{-  0.17}$ & $-0.027^{+0.035}_{-0.035}$  & $-0.070^{+0.010}_{-0.008}$ & $-0.033^{+0.015}_{-0.011}$ & $-0.033^{+0.015}_{-0.011}$\\
PTF11qmo & $101.7^{+1.1}_{-1.1}$ & $  7.8^{+0.7}_{-0.7}$ & $ 6151^{+  7}_{-  8}$ & $  0.41^{+  0.18}_{-  0.18}$ & $-0.078^{+0.031}_{-0.028}$  & $-0.058^{+0.010}_{-0.010}$ & $0.004^{+0.014}_{-0.015}$ & $0.004^{+0.014}_{-0.015}$\\
PTF12dxm & $119.3^{+15.5}_{-16.8}$ & $ 31.3^{+1.9}_{-2.1}$ & $ 6136^{+  4}_{-  4}$ & $ -2.14^{+  0.18}_{-  0.18}$ & $0.034^{+0.034}_{-0.033}$  & $-0.047^{+0.027}_{-0.021}$ & $0.067^{+0.023}_{-0.024}$ & $0.067^{+0.023}_{-0.024}$\\
PTF12eer & $169.4^{+8.9}_{-8.8}$ & $ 12.2^{+1.8}_{-1.8}$ & $ 6148^{+ 10}_{- 10}$ & $  0.64^{+  0.14}_{-  0.15}$ & $-0.053^{+0.028}_{-0.027}$  & $0.039^{+0.018}_{-0.019}$ & $-0.029^{+0.017}_{-0.012}$ & $-0.029^{+0.017}_{-0.012}$\\
PTF12efn & $145.5^{+3.3}_{-3.2}$ & $  5.4^{+1.4}_{-1.5}$ & $ 6115^{+  3}_{-  3}$ & $  1.04^{+  0.16}_{-  0.16}$ & $0.177^{+0.030}_{-0.029}$  & $-0.058^{+0.014}_{-0.012}$ & $0.051^{+0.017}_{-0.017}$ & $0.051^{+0.017}_{-0.017}$\\
PTF12ena & $101.0^{+1.6}_{-1.6}$ & $  8.2^{+0.9}_{-0.9}$ & $ 6130^{+  3}_{-  3}$ & $  0.76^{+  0.13}_{-  0.13}$ & $-0.039^{+0.028}_{-0.028}$  & $0.289^{+0.011}_{-0.012}$ & $-0.038^{+0.010}_{-0.008}$ & $-0.038^{+0.010}_{-0.008}$\\
PTF12evo & $129.1^{+2.6}_{-2.7}$ & $  9.9^{+1.1}_{-1.1}$ & $ 6157^{+  4}_{-  4}$ & $  0.81^{+  0.13}_{-  0.12}$ & $0.024^{+0.027}_{-0.027}$  & $0.004^{+0.011}_{-0.011}$ & $-0.034^{+0.012}_{-0.010}$ & $-0.034^{+0.012}_{-0.010}$\\
PTF12fuu & $105.4^{+3.0}_{-3.0}$ & $  6.5^{+1.0}_{-1.0}$ & $ 6124^{+  5}_{-  5}$ & $  0.37^{+  0.11}_{-  0.11}$ & $0.178^{+0.026}_{-0.026}$  & $-0.057^{+0.010}_{-0.010}$ & $-0.026^{+0.014}_{-0.013}$ & $-0.026^{+0.014}_{-0.013}$\\
PTF12ghy & $100.6^{+3.5}_{-3.5}$ & $ 16.4^{+0.7}_{-0.7}$ & $ 6133^{+  3}_{-  3}$ & $ -0.58^{+  0.10}_{-  0.10}$ & $-0.138^{+0.023}_{-0.022}$  & $0.196^{+0.011}_{-0.011}$ & $0.001^{+0.013}_{-0.013}$ & $0.001^{+0.013}_{-0.013}$\\
PTF12grk & $158.1^{+8.3}_{-8.4}$ & $ 22.1^{+1.2}_{-1.2}$ & $ 6089^{+  8}_{-  8}$ & $ -1.02^{+  0.12}_{-  0.12}$ & $-0.117^{+0.027}_{-0.026}$  & $-0.025^{+0.016}_{-0.017}$ & $-0.039^{+0.010}_{-0.008}$ & $-0.039^{+0.010}_{-0.008}$\\
PTF12hwb & $139.4^{+25.9}_{-23.8}$ & $ -0.9^{+4.4}_{-4.3}$ & $ 6090^{+ 14}_{- 14}$ & $  1.34^{+  0.17}_{-  0.17}$ & $0.471^{+0.042}_{-0.044}$  & $0.151^{+0.037}_{-0.043}$ & $0.004^{+0.040}_{-0.032}$ & $0.004^{+0.040}_{-0.032}$\\
PTF12iiq & $149.1^{+2.1}_{-2.2}$ & $ 24.8^{+0.8}_{-0.7}$ & $ 6048^{+  7}_{-  7}$ & $ -0.83^{+  0.10}_{-  0.11}$ & $0.005^{+0.033}_{-0.033}$  & $-0.005^{+0.019}_{-0.020}$ & $-0.034^{+0.017}_{-0.010}$ & $-0.034^{+0.017}_{-0.010}$\\
PTF12ikt & $110.3^{+1.5}_{-1.6}$ & $ 14.0^{+0.7}_{-0.7}$ & $ 6141^{+  4}_{-  4}$ & $ -0.40^{+  0.10}_{-  0.10}$ & $-0.023^{+0.025}_{-0.025}$  & $-0.034^{+0.009}_{-0.008}$ & $-0.009^{+0.014}_{-0.014}$ & $-0.009^{+0.014}_{-0.014}$\\
PTF12jqh & $151.8^{+1.4}_{-1.5}$ & $  7.9^{+0.6}_{-0.7}$ & $ 6118^{+ 10}_{- 10}$ & $  0.73^{+  0.14}_{-  0.14}$ & $0.069^{+0.028}_{-0.028}$  & $-0.056^{+0.010}_{-0.010}$ & $0.009^{+0.014}_{-0.015}$ & $0.009^{+0.014}_{-0.015}$\\
PTF13ajv & $149.9^{+8.1}_{-8.2}$ & $ 22.0^{+2.7}_{-3.0}$ & $ 6107^{+ 20}_{- 20}$ & $ -1.64^{+  0.20}_{-  0.21}$ & $0.263^{+0.046}_{-0.045}$  & $-0.062^{+0.021}_{-0.014}$ & $-0.005^{+0.023}_{-0.022}$ & $-0.005^{+0.023}_{-0.022}$\\
PTF13anh & $167.2^{+1.8}_{-1.8}$ & $ 21.6^{+1.0}_{-1.0}$ & $ 6174^{+  3}_{-  4}$ & $ -0.38^{+  0.14}_{-  0.14}$ & $0.092^{+0.108}_{-0.067}$  & $-0.032^{+0.020}_{-0.019}$ & $-0.034^{+0.014}_{-0.010}$ & $-0.034^{+0.014}_{-0.010}$\\
PTF13asv & $ 75.4^{+1.1}_{-1.1}$ & $  2.4^{+0.4}_{-0.4}$ & $ 6149^{+  4}_{-  4}$ & $  0.66^{+  0.10}_{-  0.10}$ & $-0.018^{+0.184}_{-0.105}$  & $-0.068^{+0.012}_{-0.009}$ & $0.015^{+0.015}_{-0.015}$ & $0.015^{+0.015}_{-0.015}$\\
PTF13ayw & $103.8^{+2.4}_{-2.4}$ & $ 23.1^{+1.8}_{-1.9}$ & $ 6116^{+  6}_{-  6}$ & $ -1.67^{+  0.23}_{-  0.24}$ & $0.068^{+0.041}_{-0.041}$  & $-0.069^{+0.014}_{-0.009}$ & $0.101^{+0.019}_{-0.020}$ & $0.101^{+0.019}_{-0.020}$\\
PTF13azs & $140.6^{+4.8}_{-4.8}$ & $ 15.2^{+1.3}_{-1.3}$ & $ 6124^{+ 10}_{- 10}$ & $ -0.12^{+  0.11}_{-  0.11}$ & $0.148^{+0.029}_{-0.029}$  & $0.195^{+0.012}_{-0.013}$ & $-0.037^{+0.012}_{-0.009}$ & $-0.037^{+0.012}_{-0.009}$\\
SN2003bt & $ 64.0^{+12.0}_{-13.2}$ & $ 21.4^{+0.7}_{-0.7}$ & $ 6121^{+  2}_{-  2}$ & $ -1.85^{+  0.16}_{-  0.17}$ & $0.125^{+0.034}_{-0.034}$  & $-0.057^{+0.017}_{-0.014}$ & $0.011^{+0.018}_{-0.019}$ & $0.011^{+0.018}_{-0.019}$\\
SN2005cf & $158.9^{+0.7}_{-0.7}$ & $ 15.2^{+0.8}_{-0.7}$ & $ 6140^{+  3}_{-  3}$ & $ -0.19^{+  0.13}_{-  0.13}$ & $-0.122^{+0.028}_{-0.027}$  & $-0.077^{+0.007}_{-0.005}$ & $0.002^{+0.014}_{-0.015}$ & $0.002^{+0.014}_{-0.015}$\\
SN2005hc & $127.2^{+2.4}_{-2.5}$ & $  9.7^{+0.7}_{-0.7}$ & $ 6122^{+  3}_{-  3}$ & $  0.47^{+  0.16}_{-  0.15}$ & $0.137^{+0.025}_{-0.025}$  & $-0.062^{+0.008}_{-0.008}$ & $0.001^{+0.012}_{-0.012}$ & $0.001^{+0.012}_{-0.012}$\\
SN2005hj & $ 80.4^{+2.4}_{-2.4}$ & $  4.4^{+0.8}_{-0.8}$ & $ 6139^{+  4}_{-  4}$ & $  2.58^{+  0.24}_{-  0.24}$ & $0.076^{+0.039}_{-0.037}$  & $0.009^{+0.018}_{-0.020}$ & $0.066^{+0.022}_{-0.022}$ & $0.066^{+0.022}_{-0.022}$\\
SN2005ir & $115.7^{+2.8}_{-2.7}$ & $  8.7^{+4.6}_{-4.8}$ & $ 6069^{+  5}_{-  5}$ & $  1.64^{+  0.49}_{-  0.48}$ & $0.112^{+0.042}_{-0.043}$  & $-0.022^{+0.036}_{-0.033}$ & $0.067^{+0.037}_{-0.037}$ & $0.067^{+0.037}_{-0.037}$\\
SN2006cj & $101.4^{+1.3}_{-1.3}$ & $  5.5^{+0.8}_{-0.8}$ & $ 6128^{+  3}_{-  3}$ & $  0.86^{+  0.13}_{-  0.13}$ & $0.028^{+0.025}_{-0.025}$  & $0.006^{+0.008}_{-0.009}$ & $-0.017^{+0.013}_{-0.013}$ & $-0.017^{+0.013}_{-0.013}$\\
SN2006dm & $ 99.6^{+1.6}_{-1.6}$ & $ 29.8^{+0.7}_{-0.7}$ & $ 6118^{+  3}_{-  3}$ & $ -2.35^{+  0.18}_{-  0.18}$ & $0.084^{+0.029}_{-0.029}$  & $-0.047^{+0.011}_{-0.011}$ & $0.047^{+0.017}_{-0.016}$ & $0.047^{+0.017}_{-0.016}$\\
SN2006do & $106.3^{+2.1}_{-2.1}$ & $ 26.8^{+1.3}_{-1.2}$ & $ 6101^{+  2}_{-  2}$ & $ -1.95^{+  0.34}_{-  0.34}$ & $-0.077^{+0.031}_{-0.030}$  & $-0.038^{+0.012}_{-0.013}$ & $0.035^{+0.018}_{-0.017}$ & $0.035^{+0.018}_{-0.017}$\\
SN2006ob & $ 75.9^{+12.1}_{-12.4}$ & $ 26.8^{+1.3}_{-1.4}$ & $ 6113^{+  5}_{-  5}$ & $ -2.26^{+  0.21}_{-  0.21}$ & $0.039^{+0.035}_{-0.035}$  & $-0.051^{+0.019}_{-0.017}$ & $0.084^{+0.021}_{-0.021}$ & $0.084^{+0.021}_{-0.021}$\\
SN2007bd & $113.3^{+5.4}_{-5.4}$ & $ 17.8^{+0.7}_{-0.7}$ & $ 6101^{+  3}_{-  3}$ & $ -1.02^{+  0.18}_{-  0.18}$ & $-0.071^{+0.027}_{-0.027}$  & $-0.050^{+0.011}_{-0.012}$ & $-0.042^{+0.007}_{-0.007}$ & $-0.042^{+0.007}_{-0.007}$\\
SN2007cq & $ 63.7^{+3.9}_{-3.9}$ & $ 11.6^{+0.8}_{-0.8}$ & $ 6138^{+  3}_{-  3}$ & $  0.05^{+  0.13}_{-  0.13}$ & $-0.111^{+0.029}_{-0.027}$  & $0.008^{+0.011}_{-0.011}$ & $-0.015^{+0.015}_{-0.015}$ & $-0.015^{+0.015}_{-0.015}$\\
SN2007kk & $128.6^{+1.4}_{-1.4}$ & $ 11.0^{+0.9}_{-0.9}$ & $ 6099^{+  4}_{-  4}$ & $  0.59^{+  0.14}_{-  0.14}$ & $-0.087^{+0.029}_{-0.029}$  & $-0.048^{+0.012}_{-0.012}$ & $-0.035^{+0.012}_{-0.010}$ & $-0.035^{+0.012}_{-0.010}$\\
SN2007nq & $ 91.2^{+9.1}_{-9.0}$ & $ 23.3^{+1.0}_{-1.0}$ & $ 6109^{+  5}_{-  5}$ & $ -1.90^{+  0.20}_{-  0.20}$ & $-0.075^{+0.038}_{-0.035}$  & $-0.022^{+0.015}_{-0.015}$ & $0.031^{+0.021}_{-0.020}$ & $0.031^{+0.021}_{-0.020}$\\
SN2008ec & $ 97.7^{+10.6}_{-10.6}$ & $ 23.0^{+0.4}_{-0.4}$ & $ 6125^{+  3}_{-  3}$ & $ -1.09^{+  0.10}_{-  0.10}$ & $-0.076^{+0.025}_{-0.026}$  & $0.064^{+0.012}_{-0.012}$ & $0.017^{+0.015}_{-0.015}$ & $0.017^{+0.015}_{-0.015}$\\
SN2010dt & $133.3^{+8.4}_{-8.6}$ & $ 16.0^{+0.6}_{-0.6}$ & $ 6141^{+  5}_{-  5}$ & $ -0.02^{+  0.09}_{-  0.09}$ & $-0.046^{+0.024}_{-0.024}$  & $-0.063^{+0.010}_{-0.009}$ & $-0.043^{+0.007}_{-0.007}$ & $-0.043^{+0.007}_{-0.007}$\\
SN2010ex & $114.5^{+0.9}_{-0.9}$ & $  8.4^{+0.4}_{-0.4}$ & $ 6129^{+  6}_{-  6}$ & $  0.17^{+  0.12}_{-  0.11}$ & $0.266^{+0.026}_{-0.026}$  & $-0.042^{+0.008}_{-0.008}$ & $-0.039^{+0.010}_{-0.008}$ & $-0.039^{+0.010}_{-0.008}$\\
SN2010kg & $ 99.4^{+17.2}_{-17.5}$ & $ 21.8^{+0.7}_{-0.7}$ & $ 6077^{+  5}_{-  5}$ & $ -1.87^{+  0.24}_{-  0.25}$ & $-0.073^{+0.030}_{-0.029}$  & $0.035^{+0.021}_{-0.020}$ & $0.034^{+0.020}_{-0.021}$ & $0.034^{+0.020}_{-0.021}$\\
SN2011bc & $127.1^{+2.4}_{-2.4}$ & $ 15.7^{+1.1}_{-1.1}$ & $ 6152^{+  4}_{-  4}$ & $  0.04^{+  0.10}_{-  0.10}$ & $-0.004^{+0.030}_{-0.029}$  & $0.049^{+0.013}_{-0.013}$ & $0.045^{+0.016}_{-0.015}$ & $0.045^{+0.016}_{-0.015}$\\
SN2011be & $ 93.8^{+2.0}_{-2.0}$ & $ 11.2^{+0.6}_{-0.6}$ & $ 6133^{+  4}_{-  4}$ & $  0.22^{+  0.17}_{-  0.17}$ & $0.150^{+0.032}_{-0.032}$  & $-0.062^{+0.010}_{-0.010}$ & $0.035^{+0.016}_{-0.015}$ & $0.035^{+0.016}_{-0.015}$\\
SN2011ho & $ 88.6^{+2.7}_{-2.7}$ & $ 29.8^{+1.7}_{-1.7}$ & $ 6109^{+  7}_{-  7}$ & $ -2.02^{+  0.16}_{-  0.16}$ & $0.202^{+0.049}_{-0.047}$  & $0.030^{+0.018}_{-0.019}$ & $0.032^{+0.027}_{-0.026}$ & $0.032^{+0.027}_{-0.026}$\\
SN2011hr & $ 55.7^{+14.4}_{-14.5}$ & $  4.5^{+0.2}_{-0.2}$ & $ 6138^{+  8}_{-  8}$ & $  0.82^{+  0.16}_{-  0.16}$ & $-0.031^{+0.039}_{-0.038}$  & $0.069^{+0.018}_{-0.019}$ & $0.041^{+0.020}_{-0.020}$ & $0.041^{+0.020}_{-0.020}$\\
SN2011ia & $ 85.4^{+0.6}_{-0.6}$ & $  5.8^{+0.3}_{-0.3}$ & $ 6130^{+  4}_{-  4}$ & $  0.45^{+  0.17}_{-  0.17}$ & $0.078^{+0.032}_{-0.031}$  & $-0.077^{+0.006}_{-0.005}$ & $-0.026^{+0.015}_{-0.013}$ & $-0.026^{+0.015}_{-0.013}$\\
SN2012cg & $109.0^{+0.3}_{-0.3}$ & $  8.6^{+0.1}_{-0.1}$ & $ 6138^{+  2}_{-  2}$ & $  0.44^{+  0.09}_{-  0.09}$ & $0.600^{+0.022}_{-0.023}$  & $0.078^{+0.007}_{-0.006}$ & $-0.025^{+0.011}_{-0.011}$ & $-0.025^{+0.011}_{-0.011}$\\
SN2012fr & $134.2^{+0.5}_{-0.5}$ & $  7.4^{+0.2}_{-0.2}$ & $ 6101^{+  1}_{-  1}$ & $  0.88^{+  0.08}_{-  0.09}$ & $-0.248^{+0.021}_{-0.021}$  & $-0.077^{+0.005}_{-0.005}$ & $0.044^{+0.012}_{-0.011}$ & $0.044^{+0.012}_{-0.011}$\\
SN2013be & $103.0^{+1.8}_{-1.7}$ & $ 12.1^{+0.9}_{-0.9}$ & $ 6141^{+  5}_{-  5}$ & $  0.39^{+  0.16}_{-  0.16}$ & $-0.109^{+0.026}_{-0.024}$  & $0.015^{+0.009}_{-0.010}$ & $-0.025^{+0.014}_{-0.012}$ & $-0.025^{+0.014}_{-0.012}$\\
SN2013bs & $ 86.9^{+1.5}_{-1.4}$ & $ 21.2^{+0.7}_{-0.7}$ & $ 6136^{+  3}_{-  3}$ & $ -1.49^{+  0.16}_{-  0.16}$ & $0.055^{+0.029}_{-0.030}$  & $0.036^{+0.013}_{-0.012}$ & $0.020^{+0.016}_{-0.014}$ & $0.020^{+0.016}_{-0.014}$\\
SNBOSS38 & $ 57.1^{+0.4}_{-0.4}$ & $ 17.8^{+0.3}_{-0.3}$ & $ 6127^{+  3}_{-  3}$ & $  1.02^{+  0.11}_{-  0.11}$ & $0.018^{+0.038}_{-0.039}$  & $0.017^{+0.021}_{-0.023}$ & $-0.002^{+0.019}_{-0.020}$ & $-0.002^{+0.019}_{-0.020}$\\
SNNGC0927 & $155.2^{+1.3}_{-1.3}$ & $ 10.9^{+0.7}_{-0.7}$ & $ 6109^{+  4}_{-  4}$ & $  0.39^{+  0.12}_{-  0.12}$ & $-0.031^{+0.027}_{-0.027}$  & $0.039^{+0.010}_{-0.010}$ & $-0.018^{+0.014}_{-0.013}$ & $-0.018^{+0.014}_{-0.013}$\\
\enddata
\end{deluxetable}
}




The distributions of the Monte Carlo links for all supernovae are shown in Figure~\ref{perobject3:fig}.
Each supernova is represented by a cloud of its parameters' links.
There is a core concentration in the  parameter-space, with around eight objects that occupy its outskirts.
Many outliers appear in the red tail of $E_{\gamma^0}({\hat{B}}-{\hat{V}})$, as would be expected for the (infrequent) selection of supernovae
heavily extinguished by host-galaxy dust.

The Pearson correlation coefficients of the per-supernova parameters are given in the matrix
\begin{multline}
Cor(EW_{\mathrm{Ca}}, EW_{\mathrm{Si}}, \lambda_{\mathrm{Si}}, x_1, E_{\gamma^0}({\hat{B}}-{\hat{V}}), E_{\gamma^1}({\hat{B}}-{\hat{V}}), A_{p,V}) =\\
\begin{pmatrix}
\begin{array}{rrrrrrr}
EW_{\mathrm{Ca}} & 0.10^{+0.03}_{-0.03} & -0.25^{+0.03}_{-0.03} & 0.07^{+0.03}_{-0.03} & -0.10^{+0.05}_{-0.05} & -0.18^{+0.06}_{-0.06} & 0.00^{+0.08}_{-0.08} \\
\ldots &EW_{\mathrm{Si}} & -0.12^{+0.03}_{-0.03} & -0.84^{+0.01}_{-0.01} & -0.13^{+0.04}_{-0.04} & 0.27^{+0.08}_{-0.09} & -0.00^{+0.08}_{-0.08} \\
\ldots & \ldots& \lambda_{\mathrm{Si}} & 0.14^{+0.03}_{-0.03} & 0.01^{+0.06}_{-0.05} & 0.00^{+0.06}_{-0.06} & -0.00^{+0.08}_{-0.08} \\
\ldots& \ldots& \ldots& x_1 & 0.13^{+0.04}_{-0.04} & -0.20^{+0.09}_{-0.09} & -0.00^{+0.08}_{-0.08} \\
\ldots& \ldots& \ldots & \ldots& E_{\gamma^0}({\hat{B}}-{\hat{V}})& 0.04^{+0.06}_{-0.06} & 0.00^{+0.08}_{-0.08} \\
\ldots & \ldots & \ldots& \ldots&\ldots & E_{\gamma^1}({\hat{B}}-{\hat{V}})  & 0.00^{+0.08}_{-0.08} \\
\ldots & \ldots & \ldots&\ldots &\ldots &\ldots &  A_{p,V} \\
\end{array}
\end{pmatrix}.
\label{corr:eqn}
\end{multline}
Recall that the $\Delta$-cofactor degeneracy is broken by enforcing zero correlation between $\Delta$ and the features.
To aid readability, only the upper triangular part of the otherwise symmetric matrix is shown and the corresponding feature label is given on
the diagonal.

\begin{figure}[htbp] %  figure placement: here, top, bottom, or page
   \centering
   \includegraphics[width=5.2in]{perobject_corner.pdf} 
   \caption{
   Distributions for the supernova parameters $\Delta$, $EW_{\mathrm{Ca}}$, $EW_{\mathrm{Si}}$, $\lambda_{\mathrm{Si}}$, $x_1$, $E_{\gamma^0}({\hat{B}}-{\hat{V}})$,  $E_{\gamma^1}({\hat{B}}-{\hat{V}})$,  and  $A_{p,V}$, as well as the grey offset
$\Delta$.  All Monte Carlo links are plotted, so that each supernova contributes a cloud of points.
   \label{perobject3:fig}}
\end{figure}

\added{
The correlation matrix in Eq.~\ref{corr:eqn} is not close to being singular, meaning that the each of the features conveys
independent supernova variability.  Nevertheless, a strong correlation between two features means that the determination
of one would reduce the residual dispersion in the other.  The amount of residual dispersion is conveyed in the matrix
\begin{equation}
Z =
\begin{pmatrix}
\scriptsize
\begin{array}{rrrrrrr}
27.589  \pm 0.554 & 27.451  \pm 0.572 & 26.668  \pm 0.589 & 27.509  \pm 0.542 & 27.408  \pm 0.563 & 27.099  \pm 0.640 & 27.506  \pm 0.562 \\
7.041  \pm 0.160 & 7.076  \pm 0.157 & 7.019  \pm 0.165 & 3.860  \pm 0.186 & 7.011  \pm 0.158 & 6.790  \pm 0.206 & 7.056  \pm 0.159 \\
19.779  \pm 0.449 & 20.295  \pm 0.458 & 20.463  \pm 0.460 & 20.267  \pm 0.456 & 20.430  \pm 0.461 & 20.423  \pm 0.462 & 20.404  \pm 0.465 \\
0.992  \pm 0.018 & 0.543  \pm 0.023 & 0.986  \pm 0.019 & 0.995  \pm 0.018 & 0.986  \pm 0.019 & 0.971  \pm 0.026 & 0.992  \pm 0.019 \\
0.073  \pm 0.003 & 0.073  \pm 0.003 & 0.074  \pm 0.003 & 0.073  \pm 0.003 & 0.074  \pm 0.003 & 0.074  \pm 0.003 & 0.074  \pm 0.003 \\
0.038  \pm 0.004 & 0.037  \pm 0.004 & 0.039  \pm 0.004 & 0.038  \pm 0.004 & 0.039  \pm 0.004 & 0.039  \pm 0.004 & 0.039  \pm 0.004 \\
0.050  \pm 0.011 & 0.050  \pm 0.011 & 0.050  \pm 0.011 & 0.050  \pm 0.011 & 0.050  \pm 0.011 & 0.050  \pm 0.011 & 0.050  \pm 0.011 
\\\end{array}
\end{pmatrix},
\label{ellipse:eqn}
\end{equation}
where the associated units are [\AA, \AA, \AA, 1, mag, mag, mag]$\cdot$[\AA, \AA, \AA, 1, mag, mag, mag]$^{T}$.

For indices $i, j \in \{EW_{\mathrm{Ca}}, EW_{\mathrm{Si}}, \lambda_{\mathrm{Si}}, x_1, E_{\gamma^0}({\hat{B}}-{\hat{V}}), E_{\gamma^1}({\hat{B}}-{\hat{V}}), A_{p,V}\}$, $M_{ii}$ is the standard deviation of the parameter $i$ distribution, and for $i \ne j$ $Z_{ij}$ is the standard deviation in parameter $i$
if parameter $j$ were known perfectly.  For example, for $i=EW_{\mathrm{Si}}$ the standard deviation is 7.08\AA\ but when $j=x_1$
is fixed its
standard deviation drops to 3.86\AA.  Therefore, an analysis that considers $x_1$ but not $EW_{\mathrm{Si}}$ would not account for
3.86\AA$\vec{\beta}=\{0.14 ,  0.09,  0.08,  0.07,  0.04 \}$~mag ${\hat{U}}{\hat{B}}{\hat{V}}{\hat{R}}{\hat{I}}$ magnitude dispersions
that are not intrinsically random.
}

An extensive discussion of the correlations between the spectral features  and light-curve parameters of the SNfactory data
set can be found in \citet{chotard:thesis, leget:thesis, 2017Chotard}. 
We here confine ourselves to noting the strong anti-correlation between $EW_{\mathrm{Si}}$ and $x_1$, which was alluded
to in \S\ref{results3global:sec} when interpreting the correlation between the cofactors $\vec{\beta}$ and $\vec{\zeta}$ that connect them with magnitudes.

\added{
The
observed parameters
$EW_{\mathrm{Ca}}$, $EW_{\mathrm{Si}}$, $\lambda_{\mathrm{Si}}$, $x_1$ and their associated latent values are
statistically equivalent for the  SN1991T-like LSQ12fhe, PTF11bju, and PTF11mkx and the SN2009dc-like SNBOSS38.
In addition, their inferred  $\Delta$, $E_{\gamma^0}({\hat{B}}-{\hat{V}})$, $E_{\gamma^1}({\hat{B}}-{\hat{V}})$, $A_{p,V}$
lie well within the distributions of the full sample.  We thus conclude that these objects are not extreme outliers and
that their inclusion does not significantly bias our results.
}

The $\Delta$ parameter will be discussed in its own subsection \S\ref{Delta:sec}.

\subsubsection{Results for the Latent  Parameters $g_0$, $g_1$}
\label{g0g1:sec}
The vector
$\vec{\gamma}^0$ describes color changes for an effective $\langle R^F_{\mathit{eff}}\rangle=2.43$ dust, as shown 
in  \S\ref{results3gamma:sec}.
Hence, the per-supernova parameter $\gamma^0_{\hat{V}} g_0$ is associated with the $\hat{V}$-band
extinction for a constant $R^F=2.43$.
Figure~\ref{k0_med:fig} shows the histogram of
per-supernova
median values
from the Monte Carlo chains
and the  stack of the posteriors of all supernovae  for $\gamma^0_{\hat{V}} g_0$ 
relative to that of an arbitrary supernova  $\gamma^0_{\hat{V}} g_0|_0$.
The distributions are non-Gaussian, having a sharp rise in the blue and an extended tail in the red.  This is consistent
with
simulations based on expected dust distributions
within galaxies and the distribution of galaxy orientations with respect to the observer,
which determine column densities along the lines of sight toward supernovae that are consistent with observations
\citep{1998ApJ...502..177H, 2007ApJ...659..122J}.  
We emphasize that unlike other analyses,
this distribution is given by the data, and so is not dependent on any knowledge or prior
for the distribution of dust extinction the SN population is expected to suffer. 
There is no sign of two populations in $\gamma^1_{\hat{V}} g_1$, which would be expected from a subpopulation
experiencing extinction and scattering by circumstellar dust.

\begin{figure}[htbp] %  figure placement: here, top, bottom, or page
   \centering
   \includegraphics[width=2.8in]{deltagamma0_med.pdf}
   \includegraphics[width=2.8in]{deltagamma1_med.pdf}
      \caption{
      Normalized stack of the posteriors (includes parameter uncertainty) of all supernovae  and histogram 
      of  per-supernova 
median values of: (left) $\gamma^0_{\hat{V}} g_0-\gamma^0_{\hat{V}} g_0|_0$, which is associated with the $A^F_V$ of an $\langle R^F_{\mathit{eff}}\rangle \sim 2.43$ dust model;
(right) $\gamma^1_{\hat{V}} g_1-\gamma^1_{\hat{V}} g_1|_0$,
which  is associated with extinction corrections due to deviations away from the canonical $\langle R^F_{\mathit{eff}}\rangle$ value.
   \label{k0_med:fig}}
\end{figure}

An instructive way to consider the  $\{g_0, g_1\}$ parameters is
to transform them to $\{A_V^F, E^F(B-V)\}$ (modulo additive constants)   using the
matrix $M$ in  Eq.~\ref{trans_I:eqn}.
A plot showing the expected values and 68\% credible intervals of these parameters
for our supernova
sample is shown in Figure~\ref{kk:fig}. 
For reference, a line that represents $\langle R^F_{\mathit{eff}}\rangle=2.43$ is overplotted; points above
the line have larger $R^F$, points below have smaller $R^F$.  The majority of supernovae lie within a narrow range above the line,
while the remaining fraction fall in a broader range below the line.
Comparison with  $\gamma^1_{\hat{V}} g_1-\gamma^1_{\hat{V}} g_1|_0$ of the right plot of Figure~\ref{k0_med:fig}
shows that the 
negative tail corresponds to smaller $R^F$, and the sharper positive edge to larger $R^F$.
These findings are qualitatively consistent with previous results:
using fixed dust-extinction models,
\citet{2014ApJ...789...32B, 2015MNRAS.453.3300A} deduce a wide range of dust behavior $1.5<R^F<3$ encountered by the SN~Ia population.
Our model cannot provide the range of $R^F$ in our sample nor per-supernova determinations of $R^F$ 
without further assumptions.

\begin{figure}[htbp] %  figure placement: here, top, bottom, or page
   \centering
   \includegraphics[width=4in]{avebv_synth.pdf}
      \caption{
      Expected values and 68\% credible intervals of effective $E^F(B-V)$ and $A_V^F$ after transformation from our model $g_0$ and $g_1$ parameters, for the supernova in our sample.
      Overplotted is a line with the slope expected for $R^F=2.43$.
   \label{kk:fig}}
\end{figure}



There are $>2 \sigma$ correlations between our  color-excess parameters $g_0$, $g_1$
and the input features  $EW_{\mathrm{Ca}}$,
$EW_{\mathrm{Si}}$, and $x_1$. 
Given the association
of  $g$ with dust-extinction parameters, this implies that the host environment and the input features are connected.
The correlations between $EW_{\mathrm{Si}}$ and  $EW_{\mathrm{Ca}}$  with host have already been noted \citep{2011ApJ...734...42N, 2015MNRAS.451.1973S}.
There is a lack of correlation between  $g_0$, $g_1$ and  our velocity parameter $\lambda_{\mathrm{Si}}$
(measured at peak brightness).
This contrasts with the subset of supernovae identified by their high-velocity features (HVF) in
Si~II~$\lambda$6355
(typically detected pre-maximum), who are found to be distinguished by their host galaxy properties
\citep[e.g.][]{2014MNRAS.444.3258M, 2015MNRAS.446..354P, 2015ApJS..220...20Z}.
The population and properties of HVF SNe within the SNfactory sample will be presented in \citet{2018lin}.
The correlation of $x_1$ with  host-galaxy (including dust) properties is  well established 
\citep{2000AJ....120.1479H, 2003MNRAS.340.1057S}.


\subsubsection{Results for the Parameter $p$}
\label{p:sec}
The model assumes that $p$ has a Normal distribution, which is shown in Figure~\ref{ebv3:fig} as the 
stack of the posteriors of all supernovae of the color-excess $E_p({\hat{B}}-{\hat{V}})$.
Also shown for comparison is the distribution of
the   color-excess $E_\gamma({\hat{B}}-{\hat{V}}) \equiv (\gamma^0_{\hat{B}} -\gamma^0_{\hat{V}}) g_0
+ (\gamma^1_{\hat{B}} -\gamma^1_{\hat{V}}) g_1$.
The standard deviations of  $E_\gamma({\hat{B}}-{\hat{V}})$ and $E_p({\hat{B}}-{\hat{V}})$ are
%-----
0.085 %new
and 0.010
%-----
mag respectively.
The latter is comparable to the typical measurement uncertainty.
In $\hat{B}-\hat{V}$, 
the $g$-based color variation is much larger than that of the $p$-based and its effect per-supernova would be difficult to discern.

\begin{figure}[htbp] %  figure placement: here, top, bottom, or page
   \centering
   \includegraphics[width=4in]{ebv.pdf}
      \caption{Stack of the posteriors  (includes parameter uncertainty) of all supernovae of the  $E_{\gamma}({\hat{B}}-{\hat{V}})$ 
      and the 
 $E_p({\hat{B}}-{\hat{V}})$  contributions to color excess relative to an arbitrary supernova.
   \label{ebv3:fig}}
\end{figure}


There is no significant correlation between the parameter $p$ and the other features.
Their independence from the  $g_0$ and $g_1$ parameters
implies that two physical parameters are not being artificially attributed to three
model parameters.  One could have
worried this might
occur since our linear model does not precisely
describe the non-linearity between broad-band magnitudes and dust parameters. A
correlation 
between the parameters would have complicated any claims of the detection of a third independent supernova parameter.

Detection of intrinsic color has been previously reported by \citetalias{2017ApJ...842...93M}.
For data, they use the outputs of SALT2 fits: the $c$ color parameter;
the
supernova absolute magnitude $M$.
In the \citetalias{2017ApJ...842...93M}
model, intrinsic color contributes a linear   $\beta_{\mathit{int}} c$ to  the absolute
magnitude.
They find a significant $\beta_{\mathit{int}} = 2.2\pm 0.3$.
While we too claim the detection of a color parameter incompatible with dust,
the differences between our models make it
difficult to determine whether the results are consistent.   Our results show effects on color that are not monotonic
with wavelength.  We find that the statistic that nominally best corresponds to  $\beta_{\mathit{int}}$
to be
$\phi_{\hat{B}}/(\phi_{\hat{B}}-\phi_{\hat{V}}) =-4.4_{  -6.4}^{+   2.4}$, which appears to be inconsistent with  \citetalias{2017ApJ...842...93M}.  
However a change of the color baseline gives a significantly different
$\phi_{\hat{B}}/(\phi_{\hat{B}}-\phi_{\hat{R}}) =4.3_{  -1.4}^{+   3.8}$.  Our results indicate that  $\beta_{\mathit{int}}$
is sensitive to the color that $c$ corresponds to.  Although SALT2 $c$ is calibrated to correspond with $B-V$, it is
determined using data from all bands with the wavelength extremes  typically providing the strongest leverage.
The $c$ parameter is thus an  amalgamation of all input colors.
We therefore make no conclusions on the consistency between our $\vec{\phi}$-color properties
and those found by \citetalias{2017ApJ...842...93M}.


An independent piece of information available for a subset of our supernova sample is host-galaxy mass \citep{2013ApJ...770..108C}.  
This statistic is of interest, as there is a correlation between Hubble residual and host-galaxy mass
\citep[first noted by][]{2010ApJ...715..743K,2010MNRAS.406..782S}, a signal confirmed to exist in the SNfactory
sample \citep{2013ApJ...770..108C}.
This host-mass bias could be the result of a parameter that was not accounted for in the inference of SN~Ia absolute magnitude.
Indeed  \citetalias{2017ApJ...842...93M} find that the  introduction of intrinsic color as a latent parameter
reduces the strength of  this bias.

It is possible that the  $p$ parameter is a supernova-tracer of a population that to date has only been identified through host-galaxy tracers.
We plot in Figure~\ref{childress3:fig} our parameter
 $\sigma_p\phi_{\hat{V}}(p-p|_0) $  (subtracting out a random supernova
to suppress correlated errors) versus host mass
for the subset of supernovae whose host measurements are given in \citet{2013ApJ...770..108C}.
Low- and high-mass galaxies, divided by $\log{(M/M_\sun)}=10$, have high probability of hosting supernovae with different $p$ distributions, as
the Kolmogorov-Smirnov test gives a two-tailed $p$-value of $  0.016 _ {     0.011 } ^{     0.044 }$.
The median and median uncertainties of the two subsamples are
%---
$\langle \sigma_p\phi_{\hat{V}}(p-p|_0) \rangle=  0.041 \pm {     0.007 }$ mag,
$\langle \sigma_p\phi_{\hat{V}}(p-p|_0)  \rangle=0.065 \pm {     0.007 }$ mag
for low- and high-mass hosts respectively, with a difference of $ 0.024 \pm {     0.010 }$~mag that is significant at $>2 \sigma$.

In this article we do not associate $p$ directly with Hubble residuals. This is left for future work.

\begin{figure}[htbp] %  figure placement: here, top, bottom, or page
   \centering
   \includegraphics[width=4in]{childress.pdf} 
   \caption{Parameter $\sigma_p\phi_{\hat{V}}p$  versus host galaxy mass. Overplotted are the mean and 1$\sigma$ uncertainty on the mean for supernovae with hosts
      less than and greater than  $\log{(M/M_\sun)}=10$.
    \label{childress3:fig}}
\end{figure}


\subsubsection{Results for the Magnitude Offset $\Delta$}
\label{Delta:sec}

The color-standardization leaves magnitude residuals $\Delta$.
The histogram of the per-supernova  medians of these relative  grey offsets $\Delta-\Delta|_0$,
is shown in Figure~\ref{hist:fig}.  The distribution
has a total standard deviation of
%-----
$0.13$
%-----
mag, and a tail in the positive (fainter) direction.
Supernova peculiar velocities contribute to the standard deviation; of the four
supernovae with the highest values of $\Delta$, two have 
heliocentric
redshifts of 0.0015 and 0.0086. 
\replaced{
Eq.~\ref{corr:eqn} shows $>2 \sigma$ correlations between $\Delta$ and each of $\lambda_{\mathrm{Si}}$, $x_1$, $E_{\gamma^1}({\hat{B}}-{\hat{V}})$, and
$\sigma_p\phi_{\hat{V}}p$, showing that in addition to standardizing colors, these features can also be used to standardize
absolute magnitudes.  }{Recall that the model degeneracy between $\Delta$ and the cofactors (e.g.\ $\alpha$, $\beta$) is removed by enforcing
zero correlation between $\Delta$ and the per-supernova parameters (e.g.\ $EW_{\mathrm{Ca}}$,  $EW_{\mathrm{Si}}$).
An alternative method to break the degeneracy is to minimize the dispersion in $\Delta$.
}
The focus of this article is color standardization so we defer consideration of absolute-magnitude standardization to future work.
 
\begin{figure}[htbp] %  figure placement: here, top, bottom, or page
   \centering
   \includegraphics[width=4in]{deltaDelta_hist.pdf} 
   \caption{
   Normalized stack of the posteriors  (includes parameter uncertainty) of all supernovae and histogram of the per-supernova medians of the grey offset $\Delta$. 
   To help null correlated errors, we select
    an arbitrary supernova and subtract out its $\Delta|_0$ from those of all other objects at the level of each MCMC link.
  Of the four
supernovae with the highest values of $\Delta$, two have 
heliocentric
redshifts of 0.0015 and 0.0086. 
   \label{hist:fig}}
\end{figure}



\section{Conclusions}
\label{conclusions:sec}
To summarize, we model SN~Ia broadband optical peak magnitudes allowing for correlations with spectral features at peak,
light-curve shape, and
latent color parameters.  Analyzing SNfactory data with this model, we find significant evidence that the above parameters do
affect supernova magnitudes and colors.  Two of the latent color parameters are consistent with the
\citetalias{1999PASP..111...63F} dust-extinction model, making this
the first determination of the dust extinction curve outside the Local Group
derived entirely independently of assumptions about the shape of the extinction curve and/or assumptions about the
distribution of $A_V$.  
We find a broad distribution for total-to-selective extinction
with an effective
$\langle R^F_{\mathit{eff}}\rangle \sim 2.43$ and an asymmetric tail extending toward lower values.
We identify a new
parameter that affects supernova
colors in a manner that is distinct from the expectations for dust
and the SALT2 shape-parameter $x_1$.
This new parameter correlates with host-galaxy mass, meaning it is a candidate supernova observable that may be linked
with the Hubble diagram mass-step.
A significant bias in $R^F$ is obtained when this third source of color variation
is unaccounted for.

Just as supernova spectral and light curve features are associated with distinct color variations, the latent parameter $p$
identified through color variation may be associated with a spectral and/or light curve feature not included
in our analysis.  It
is of interest to look for tracers other than color to account for this supernova diversity.
We leave to future work the search for correlations between $p$ and other
observable features accessed by SNfactory data.

Using different arguments based on expected populations, \citetalias{2017ApJ...842...93M} infer
separate intrinsic and dust-based color-magnitude relations based on an independent sample of 248 SNe~Ia
mainly from the CfA \citep{1999AJ....117..707R, 2006AJ....131..527J, 2009ApJ...700.1097H, 2012ApJS..200...12H}
and CSP \citep{2010AJ....139..519C, 2011AJ....142..156S} surveys.
Finding slight color perturbations in the multi-band light curves 
of those samples that are  consistent with our $\sigma_p \vec{\phi}$ would provide strong evidence 
that we and  \citetalias{2017ApJ...842...93M}  are seeing the same color effect.

The current data shows a statistically significant difference in the distribution of our new parameter for supernovae
in low- and high-mass host-galaxies.
Using the SNfactory sample,
\citet{2018arXiv180603849R} find that a step in Hubble residuals is better related to local 
specific
star 
formation rate, rather than
global host mass.  The search for a correlation between our parameter and local star formation rate is planned for future work.

An interesting direction for future work is to use narrower bands for generating synthetic photometry
from the SNfactory spectrophotometry.  When the bandwidth becomes small enough
to resolve spectral features, the analysis would produce a spectroscopic model.   Higher resolution would allow us to forego 
the use of a fiducial template necessary to predict broadband fluxes generated by dust models, and also allow direct incorporation
of dust models into our framework.


The approach of our analysis is to mine for new supernova properties based on colors and spectral features.
Our model and results do carry information on absolute magnitude but are not tailored for its study.
The grey parameter $\Delta$, while containing information on absolute magnitude, 
also 
includes
contributions from peculiar velocities and measurement uncertainties.  Studies
focused on improving SNe~Ia as standard candles 
would need to
modify  our model to
distinguish between these sources of greyness and dispersion.
Our
mining exercise uses all supernovae that pass quality cuts, the approach taken by previous
exploratory work.  In contrast, the calibration of SNe~Ia as standard candles needs procedures such
as cross-validation to avoid overtraining.
An absolute magnitude calibration using the parameters identified in this analysis is left to future work.

The existence of three color parameters  has important implications when SNe~Ia are used as distance indicators.
Measurements in at least four bands are necessary to obtain three colors.  
Supernova distance determinations that do not account for three color parameters are susceptible to 
systematic errors.  
Parameter populations evolving with redshift would result in errors in
Hubble diagrams.
The use of  correlators with the SN parameters, such as host-galaxy properties e.g.\ as shown for our $p$ parameter,
may partially mitigate such errors.

\acknowledgments
We thank the STAN team for providing the statistical tool without which this analysis would not have been possible,
and Michael Betancourt specifically for his helpful guidance.  
Distribution surfaces are plotted using the ChainConsumer package \citep{Hinton2016}.
We thank Danny Goldstein and
Xiaosheng Huang for useful discussions.
We thank Dan Birchall for observing assistance, the technical and
scientific staffs of the Palomar Observatory, the High Performance
Wireless Radio Network (HPWREN), and the University of Hawaii 2.2~m
telescope.  We recognize the significant cultural role of Mauna Kea
within the indigenous Hawaiian community, and we appreciate the
opportunity to conduct observations from this revered site.  This
work was supported in part by the Director, Office of Science,
Office of High Energy Physics, of the U.S. Department of Energy
under Contract No. DE-AC02- 05CH11231.  Support in France was
provided by CNRS/IN2P3, CNRS/INSU, and PNC; LPNHE acknowledges
support from LABEX ILP, supported by French state funds managed by
the ANR within the Investissements d'Avenir programme under reference
ANR-11-IDEX-0004-02.  Support in Germany was provided by the DFG
through TRR33 ``The Dark Universe;'' and in China from Tsinghua
University 985 grant and NSFC grant No~11173017.  Some results were
obtained using resources and support from the National Energy
Research Scientific Computing Center, supported by the Director,
Office of Science, Office of Advanced Scientific Computing Research,
of the U.S. Department of Energy under Contract No. DE-AC02-05CH11231.
HPWREN is funded by National Science Foundation Grant Number
ANI-0087344, and the University of California, San Diego.
This project has received funding from the European Research Council (ERC) under the European Union's Horizon 2020 research and innovation programme (grant agreement No 759194 - USNAC 
\added{Based in part on observations obtained with the Samuel Oschin Telescope and the 60 inch Telescope at the Palomar Observatory as part of the Palomar Transient Factory project, a scientific collaboration between the California Institute of Technology, Columbia University, Las Cumbres Observatory, the Lawrence Berkeley National Laboratory, the National Energy Research Scientific Computing Center, the University of Oxford, and the Weizmann Institute of Science.}

\appendix
\section{Results from Model~I and II}
\label{m12results:sec}
For each of the five filters, the 68\%  equal-tailed credible intervals for the global parameters $\vec{\alpha}$, $\vec{\beta}$, $\vec{\eta}$,
$\vec{\gamma}^0$ and $\vec{\gamma}^1$ of Model~I are given in  Table~\ref{global1:tab}.
The results for the same parameters and $\sigma_p\vec{\phi}$
of Model~II are given in Table~\ref{global2:tab}.  In contrast to Model~III,
Figure~\ref{M1ev:fig} shows that the $\hat{I}$-band has
the largest separation between the degenerate solution of $\sigma_p \phi_X$.  To break the degeneracy we enforce 
$\vec{\phi} = \text{sign}(\phi_{\hat{I}}) \vec{\phi}$.


\begin{table}
\centering
\begin{tabular}{|c|c|c|c|c|c|}
\hline
Parameters& $X={\hat{U}}$ &${\hat{B}}$&${\hat{V}}$&${\hat{R}}$&${\hat{I}}$\\ \hline
$\alpha_X$
& $0.0037^{+0.0003}_{-0.0004}$
& $0.0011^{+0.0003}_{-0.0003}$
& $0.0012^{+0.0003}_{-0.0003}$
& $0.0013^{+0.0003}_{-0.0003}$
& $0.0026^{+0.0002}_{-0.0003}$
\\
${\alpha_X/\alpha_V-1}$
& $   2.0^{+   0.5}_{  -0.3}$
& $  -0.1^{+   0.1}_{  -0.1}$
& \ldots
& $   0.1^{+   0.1}_{  -0.0}$
& $   1.1^{+   0.4}_{  -0.2}$
\\
$\beta_X$
& $ 0.026^{+ 0.002}_{-0.002}$
& $ 0.019^{+ 0.001}_{-0.001}$
& $ 0.022^{+ 0.001}_{-0.001}$
& $ 0.018^{+ 0.001}_{-0.001}$
& $ 0.018^{+ 0.001}_{-0.001}$
\\
${\beta_X/\beta_V-1}$
& $  0.17^{+  0.02}_{ -0.02}$
& $ -0.14^{+  0.02}_{ -0.02}$
& \ldots
& $ -0.18^{+  0.01}_{ -0.01}$
& $ -0.16^{+  0.01}_{ -0.01}$
\\
$\eta_X$
& $-0.0006^{+0.0005}_{-0.0005}$
& $-0.0007^{+0.0004}_{-0.0004}$
& \ldots
& $0.0002^{+0.0003}_{-0.0004}$
& $-0.0007^{+0.0003}_{-0.0003}$
\\
${\eta_X/\eta_V-1}$
& $ -0.30^{+  4.06}_{ -3.18}$
& $ -0.73^{+  5.16}_{ -3.69}$
& \ldots
& $ -0.02^{+  0.81}_{ -0.97}$
& $ -1.03^{+  5.25}_{ -3.77}$
\\
$\gamma^0_X$
& $ 76.37^{+  3.19}_{ -3.23}$
& $ 60.22^{+  3.09}_{ -3.20}$
& $ 43.06^{+  2.95}_{ -3.11}$
& $ 32.93^{+  2.60}_{ -2.87}$
& $ 22.89^{+  2.27}_{ -2.65}$
\\
${\gamma^0_X/\gamma^0_V-1}$
& $  0.78^{+  0.06}_{ -0.06}$
& $  0.40^{+  0.03}_{ -0.03}$
& \ldots
& $ -0.24^{+  0.01}_{ -0.01}$
& $ -0.47^{+  0.02}_{ -0.03}$
\\
$\gamma^1_X$
& $-18.64^{+  4.86}_{ -4.86}$
& $-28.10^{+  4.23}_{ -4.20}$
& $-28.71^{+  3.75}_{ -3.70}$
& $-21.86^{+  3.49}_{ -3.32}$
& $-11.49^{+  3.38}_{ -3.05}$
\\
${\gamma^1_X/\gamma^1_V-1}$
& $ -0.35^{+  0.08}_{ -0.10}$
& $ -0.02^{+  0.03}_{ -0.03}$
& \ldots
& $ -0.24^{+  0.02}_{ -0.03}$
& $ -0.60^{+  0.06}_{ -0.08}$
\\
\hline
\end{tabular}
\caption{68\% credible intervals for the global fit parameters of the Two Color Parameter Model~I in \S\ref{modelI:sec}.\label{global1:tab}}
\end{table}


\begin{table}
\centering
\begin{tabular}{|c|c|c|c|c|c|}
\hline
Parameters& $X={\hat{U}}$ &${\hat{B}}$&${\hat{V}}$&${\hat{R}}$&${\hat{I}}$\\ \hline
$\alpha_X$
& $0.0048^{+0.0010}_{-0.0011}$
& $0.0019^{+0.0008}_{-0.0009}$
& $0.0018^{+0.0007}_{-0.0007}$
& $0.0017^{+0.0005}_{-0.0005}$
& $0.0027^{+0.0004}_{-0.0004}$
\\
${\alpha_X}/\alpha_{\hat{V}-1}$
& $   1.7^{+   0.8}_{  -0.4}$
& $   0.1^{+   0.1}_{  -0.1}$
& \ldots
& $  -0.0^{+   0.1}_{  -0.1}$
& $   0.5^{+   0.6}_{  -0.3}$
\\
$\beta_X$
& $ 0.035^{+ 0.002}_{-0.003}$
& $ 0.026^{+ 0.002}_{-0.002}$
& $ 0.027^{+ 0.002}_{-0.002}$
& $ 0.022^{+ 0.001}_{-0.001}$
& $ 0.021^{+ 0.001}_{-0.001}$
\\
${\beta_X}/\beta_{\hat{V}-1}$
& $  0.27^{+  0.04}_{ -0.04}$
& $ -0.05^{+  0.02}_{ -0.02}$
&  \ldots
& $ -0.19^{+  0.01}_{ -0.01}$
& $ -0.22^{+  0.03}_{ -0.02}$
\\
$\eta_X$
& $-0.0005^{+0.0008}_{-0.0008}$
& $-0.0003^{+0.0007}_{-0.0007}$
& $0.0006^{+0.0006}_{-0.0006}$
& $0.0006^{+0.0005}_{-0.0005}$
& $0.0000^{+0.0004}_{-0.0004}$
\\
${\eta_X}/\eta_{\hat{V}-1}$
& $ -1.17^{+  1.20}_{ -2.44}$
& $ -1.01^{+  0.76}_{ -1.89}$
&  \ldots
& $ -0.03^{+  0.33}_{ -0.19}$
& $ -0.78^{+  0.51}_{ -0.78}$
\\
$\gamma^0_X$
& $ 71.40^{+  3.29}_{ -3.14}$
& $ 58.39^{+  2.79}_{ -2.62}$
& $ 43.99^{+  2.27}_{ -2.12}$
& $ 33.29^{+  1.87}_{ -1.72}$
& $ 24.81^{+  1.64}_{ -1.53}$
\\
${\gamma^0_X}/\gamma^0_{\hat{V}-1}$
& $  0.62^{+  0.04}_{ -0.04}$
& $  0.33^{+  0.02}_{ -0.02}$
& \ldots
& $ -0.24^{+  0.01}_{ -0.01}$
& $ -0.44^{+  0.02}_{ -0.02}$
\\
$\gamma^1_X$
& $  9.83^{+  5.44}_{ -4.90}$
& $  2.34^{+  4.44}_{ -4.18}$
& $ -4.58^{+  3.40}_{ -3.32}$
& $ -2.41^{+  2.72}_{ -2.63}$
& $ -5.77^{+  2.22}_{ -2.15}$
\\
${\gamma^1_X}/\gamma^1_{\hat{V}-1}$
& $ -2.62^{+  1.37}_{ -4.02}$
& $ -1.25^{+  0.68}_{ -1.99}$
& \ldots
& $ -0.43^{+  0.12}_{ -0.32}$
& $  0.15^{+  0.88}_{ -0.37}$
\\
$\sigma_p \phi_X$
& $-0.007^{+ 0.012}_{-0.012}$
& $-0.039^{+ 0.010}_{-0.010}$
& $-0.039^{+ 0.009}_{-0.008}$
& $-0.026^{+ 0.007}_{-0.007}$
& $ 0.018^{+ 0.007}_{-0.007}$
\\
${\phi_X/\phi_{\hat{V}}-1}$
& $-0.804^{+ 0.245}_{-0.342}$
& $-0.009^{+ 0.095}_{-0.109}$
& \ldots
& $-0.328^{+ 0.047}_{-0.066}$
& $-1.465^{+ 0.220}_{-0.349}$
\\\hline
\end{tabular}
\caption{68\% credible intervals for the global fit parameters of the Three Color Parameter Model~II in \S\ref{modelII:sec}.\label{global2:tab}}
\end{table}
%
%\section{Covariance Matrix for Model~III Global Parameters}
%\added{The following is the  covariance matrix for the global cofactors $\{ \alpha_{\hat{U}}, \beta_{\hat{U}}, \eta_{\hat{U}}, \zeta_{\hat{U}}, \gamma^0_{\hat{U}},\gamma^1_{\hat{U}}, \sigma_p \phi_{\hat{U}}, 
%\ldots,
%\alpha_{\hat{I}}, \beta_{\hat{I}}, \eta_{\hat{I}}, \zeta_{\hat{I}}, \gamma^0_{\hat{I}},\gamma^1_{\hat{I}}, \sigma_p \phi_{\hat{I}}
% \}$.  The matrix 35x35 is big so will have to go in supplemental material.}
%\setcounter{MaxMatrixCols}{40}
%\begin{equation}
%\tiny
%\begin{bmatrix}
%5.5\times 10^{-7} & -1.8\times 10^{-6} & 3.9\times 10^{-7} & -1.6\times 10^{-5} & 2.1\times 10^{-4} & -2.6\times 10^{-4} & 3.6\times 10^{-6} & 4.5\times 10^{-7} & -1.5\times 10^{-6} & 3.1\times 10^{-7} & -1.3\times 10^{-5} & 2.6\times 10^{-4} & -1.3\times 10^{-4} & 2.8\times 10^{-6} & 3.6\times 10^{-7} & -1.2\times 10^{-6} & 2.4\times 10^{-7} & -1.0\times 10^{-5} & 1.9\times 10^{-4} & -5.5\times 10^{-5} & 1.7\times 10^{-6} & 2.9\times 10^{-7} & -9.2\times 10^{-7} & 1.9\times 10^{-7} & -8.1\times 10^{-6} & 1.1\times 10^{-4} & -1.7\times 10^{-5} & 1.1\times 10^{-6} & 2.2\times 10^{-7} & -7.2\times 10^{-7} & 1.4\times 10^{-7} & -6.3\times 10^{-6} & -3.6\times 10^{-5} & -6.7\times 10^{-6} & 4.7\times 10^{-7} \\
%-1.8\times 10^{-6} & 2.7\times 10^{-5} & -2.6\times 10^{-6} & 1.9\times 10^{-4} & -4.0\times 10^{-3} & -3.0\times 10^{-3} & -1.8\times 10^{-5} & -1.4\times 10^{-6} & 2.2\times 10^{-5} & -2.1\times 10^{-6} & 1.5\times 10^{-4} & -3.2\times 10^{-3} & -2.7\times 10^{-3} & -1.4\times 10^{-5} & -1.1\times 10^{-6} & 1.7\times 10^{-5} & -1.6\times 10^{-6} & 1.2\times 10^{-4} & -2.6\times 10^{-3} & -2.1\times 10^{-3} & -1.0\times 10^{-5} & -9.2\times 10^{-7} & 1.4\times 10^{-5} & -1.2\times 10^{-6} & 9.4\times 10^{-5} & -2.0\times 10^{-3} & -1.6\times 10^{-3} & -7.9\times 10^{-6} & -7.3\times 10^{-7} & 1.1\times 10^{-5} & -8.8\times 10^{-7} & 7.6\times 10^{-5} & -1.1\times 10^{-3} & -1.2\times 10^{-3} & -5.2\times 10^{-6} \\
%3.9\times 10^{-7} & -2.6\times 10^{-6} & 1.3\times 10^{-6} & -2.2\times 10^{-5} & 3.9\times 10^{-4} & 1.7\times 10^{-3} & 1.0\times 10^{-5} & 3.2\times 10^{-7} & -2.1\times 10^{-6} & 1.0\times 10^{-6} & -1.8\times 10^{-5} & 5.0\times 10^{-4} & 1.4\times 10^{-3} & 8.3\times 10^{-6} & 2.4\times 10^{-7} & -1.5\times 10^{-6} & 7.8\times 10^{-7} & -1.3\times 10^{-5} & 5.4\times 10^{-4} & 1.0\times 10^{-3} & 6.0\times 10^{-6} & 1.9\times 10^{-7} & -1.1\times 10^{-6} & 6.1\times 10^{-7} & -1.0\times 10^{-5} & 4.1\times 10^{-4} & 8.0\times 10^{-4} & 4.6\times 10^{-6} & 1.4\times 10^{-7} & -8.1\times 10^{-7} & 4.7\times 10^{-7} & -7.4\times 10^{-6} & 1.6\times 10^{-4} & 5.9\times 10^{-4} & 3.1\times 10^{-6} \\
%-1.6\times 10^{-5} & 1.9\times 10^{-4} & -2.2\times 10^{-5} & 1.6\times 10^{-3} & -2.9\times 10^{-2} & -1.4\times 10^{-2} & -2.7\times 10^{-4} & -1.3\times 10^{-5} & 1.5\times 10^{-4} & -1.8\times 10^{-5} & 1.3\times 10^{-3} & -2.8\times 10^{-2} & -1.3\times 10^{-2} & -2.1\times 10^{-4} & -1.0\times 10^{-5} & 1.2\times 10^{-4} & -1.4\times 10^{-5} & 9.9\times 10^{-4} & -2.3\times 10^{-2} & -1.0\times 10^{-2} & -1.5\times 10^{-4} & -8.3\times 10^{-6} & 9.5\times 10^{-5} & -1.1\times 10^{-5} & 7.9\times 10^{-4} & -1.8\times 10^{-2} & -7.7\times 10^{-3} & -1.2\times 10^{-4} & -6.4\times 10^{-6} & 7.7\times 10^{-5} & -7.7\times 10^{-6} & 6.3\times 10^{-4} & -8.1\times 10^{-3} & -4.9\times 10^{-3} & -7.9\times 10^{-5} \\
%2.1\times 10^{-4} & -4.0\times 10^{-3} & 3.9\times 10^{-4} & -2.9\times 10^{-2} & 6.9\times 10^{+0} & 1.3\times 10^{+0} & 6.7\times 10^{-3} & 1.4\times 10^{-4} & -3.0\times 10^{-3} & 3.2\times 10^{-4} & -2.2\times 10^{-2} & 5.5\times 10^{+0} & 1.1\times 10^{+0} & 5.1\times 10^{-3} & 9.6\times 10^{-5} & -2.2\times 10^{-3} & 2.3\times 10^{-4} & -1.6\times 10^{-2} & 4.5\times 10^{+0} & 9.7\times 10^{-1} & 3.7\times 10^{-3} & 6.0\times 10^{-5} & -1.7\times 10^{-3} & 1.7\times 10^{-4} & -1.3\times 10^{-2} & 3.7\times 10^{+0} & 8.5\times 10^{-1} & 3.0\times 10^{-3} & 3.9\times 10^{-5} & -1.4\times 10^{-3} & 1.3\times 10^{-4} & -1.1\times 10^{-2} & 2.9\times 10^{+0} & 6.7\times 10^{-1} & 2.1\times 10^{-3} \\
%-2.6\times 10^{-4} & -3.0\times 10^{-3} & 1.7\times 10^{-3} & -1.4\times 10^{-2} & 1.3\times 10^{+0} & 1.7\times 10^{+1} & 8.7\times 10^{-3} & -1.9\times 10^{-4} & -2.8\times 10^{-3} & 1.3\times 10^{-3} & -1.3\times 10^{-2} & 1.1\times 10^{+0} & 1.4\times 10^{+1} & 7.1\times 10^{-3} & -1.7\times 10^{-4} & -1.5\times 10^{-3} & 9.7\times 10^{-4} & -6.8\times 10^{-3} & 1.8\times 10^{+0} & 1.1\times 10^{+1} & 4.9\times 10^{-3} & -1.2\times 10^{-4} & -1.0\times 10^{-3} & 7.6\times 10^{-4} & -4.4\times 10^{-3} & 1.6\times 10^{+0} & 9.6\times 10^{+0} & 3.2\times 10^{-3} & -10.0\times 10^{-5} & -3.8\times 10^{-4} & 5.6\times 10^{-4} & -7.6\times 10^{-4} & 1.4\times 10^{+0} & 7.9\times 10^{+0} & 1.8\times 10^{-3} \\
%3.6\times 10^{-6} & -1.8\times 10^{-5} & 1.0\times 10^{-5} & -2.7\times 10^{-4} & 6.7\times 10^{-3} & 8.7\times 10^{-3} & 2.8\times 10^{-4} & 2.9\times 10^{-6} & -1.4\times 10^{-5} & 8.4\times 10^{-6} & -2.1\times 10^{-4} & 1.1\times 10^{-2} & 6.8\times 10^{-3} & 2.3\times 10^{-4} & 2.0\times 10^{-6} & -9.4\times 10^{-6} & 6.4\times 10^{-6} & -1.6\times 10^{-4} & 1.1\times 10^{-2} & 4.1\times 10^{-3} & 1.8\times 10^{-4} & 1.4\times 10^{-6} & -7.4\times 10^{-6} & 4.9\times 10^{-6} & -1.2\times 10^{-4} & 8.0\times 10^{-3} & 3.0\times 10^{-3} & 1.4\times 10^{-4} & 9.2\times 10^{-7} & -7.2\times 10^{-6} & 3.6\times 10^{-6} & -9.9\times 10^{-5} & 2.6\times 10^{-3} & 2.3\times 10^{-3} & 1.1\times 10^{-4} \\
%4.5\times 10^{-7} & -1.4\times 10^{-6} & 3.2\times 10^{-7} & -1.3\times 10^{-5} & 1.4\times 10^{-4} & -1.9\times 10^{-4} & 2.9\times 10^{-6} & 4.0\times 10^{-7} & -1.3\times 10^{-6} & 2.7\times 10^{-7} & -1.1\times 10^{-5} & 2.0\times 10^{-4} & -2.5\times 10^{-5} & 2.3\times 10^{-6} & 3.2\times 10^{-7} & -1.0\times 10^{-6} & 2.1\times 10^{-7} & -9.1\times 10^{-6} & 1.4\times 10^{-4} & 4.3\times 10^{-5} & 1.3\times 10^{-6} & 2.6\times 10^{-7} & -8.2\times 10^{-7} & 1.7\times 10^{-7} & -7.1\times 10^{-6} & 7.4\times 10^{-5} & 5.1\times 10^{-5} & 6.9\times 10^{-7} & 1.9\times 10^{-7} & -6.0\times 10^{-7} & 1.2\times 10^{-7} & -5.3\times 10^{-6} & -5.2\times 10^{-5} & 9.5\times 10^{-6} & 2.4\times 10^{-7} \\
%-1.5\times 10^{-6} & 2.2\times 10^{-5} & -2.1\times 10^{-6} & 1.5\times 10^{-4} & -3.0\times 10^{-3} & -2.8\times 10^{-3} & -1.4\times 10^{-5} & -1.3\times 10^{-6} & 1.9\times 10^{-5} & -1.8\times 10^{-6} & 1.3\times 10^{-4} & -2.4\times 10^{-3} & -3.0\times 10^{-3} & -1.2\times 10^{-5} & -1.0\times 10^{-6} & 1.6\times 10^{-5} & -1.4\times 10^{-6} & 1.0\times 10^{-4} & -2.0\times 10^{-3} & -2.6\times 10^{-3} & -7.6\times 10^{-6} & -8.4\times 10^{-7} & 1.2\times 10^{-5} & -1.0\times 10^{-6} & 8.4\times 10^{-5} & -1.5\times 10^{-3} & -2.0\times 10^{-3} & -5.6\times 10^{-6} & -6.2\times 10^{-7} & 9.3\times 10^{-6} & -7.7\times 10^{-7} & 6.3\times 10^{-5} & -7.7\times 10^{-4} & -1.1\times 10^{-3} & -4.0\times 10^{-6} \\
%3.1\times 10^{-7} & -2.1\times 10^{-6} & 1.0\times 10^{-6} & -1.8\times 10^{-5} & 3.2\times 10^{-4} & 1.3\times 10^{-3} & 8.4\times 10^{-6} & 2.7\times 10^{-7} & -1.8\times 10^{-6} & 8.5\times 10^{-7} & -1.5\times 10^{-5} & 4.1\times 10^{-4} & 1.1\times 10^{-3} & 6.7\times 10^{-6} & 2.0\times 10^{-7} & -1.2\times 10^{-6} & 6.7\times 10^{-7} & -1.1\times 10^{-5} & 4.4\times 10^{-4} & 8.2\times 10^{-4} & 4.9\times 10^{-6} & 1.6\times 10^{-7} & -9.4\times 10^{-7} & 5.2\times 10^{-7} & -8.5\times 10^{-6} & 3.4\times 10^{-4} & 6.4\times 10^{-4} & 3.7\times 10^{-6} & 1.1\times 10^{-7} & -6.9\times 10^{-7} & 3.8\times 10^{-7} & -6.2\times 10^{-6} & 1.3\times 10^{-4} & 4.8\times 10^{-4} & 2.6\times 10^{-6} \\
%-1.3\times 10^{-5} & 1.5\times 10^{-4} & -1.8\times 10^{-5} & 1.3\times 10^{-3} & -2.2\times 10^{-2} & -1.3\times 10^{-2} & -2.1\times 10^{-4} & -1.1\times 10^{-5} & 1.3\times 10^{-4} & -1.5\times 10^{-5} & 1.1\times 10^{-3} & -2.2\times 10^{-2} & -1.3\times 10^{-2} & -1.7\times 10^{-4} & -9.0\times 10^{-6} & 1.0\times 10^{-4} & -1.2\times 10^{-5} & 8.7\times 10^{-4} & -1.8\times 10^{-2} & -1.1\times 10^{-2} & -1.2\times 10^{-4} & -7.2\times 10^{-6} & 8.3\times 10^{-5} & -8.9\times 10^{-6} & 6.9\times 10^{-4} & -1.4\times 10^{-2} & -8.3\times 10^{-3} & -9.2\times 10^{-5} & -5.3\times 10^{-6} & 6.2\times 10^{-5} & -6.6\times 10^{-6} & 5.1\times 10^{-4} & -6.0\times 10^{-3} & -4.1\times 10^{-3} & -6.7\times 10^{-5} \\
%2.6\times 10^{-4} & -3.2\times 10^{-3} & 5.0\times 10^{-4} & -2.8\times 10^{-2} & 5.5\times 10^{+0} & 1.1\times 10^{+0} & 1.1\times 10^{-2} & 2.0\times 10^{-4} & -2.4\times 10^{-3} & 4.1\times 10^{-4} & -2.2\times 10^{-2} & 5.1\times 10^{+0} & 9.4\times 10^{-1} & 8.4\times 10^{-3} & 1.4\times 10^{-4} & -1.8\times 10^{-3} & 3.0\times 10^{-4} & -1.6\times 10^{-2} & 4.3\times 10^{+0} & 8.6\times 10^{-1} & 6.0\times 10^{-3} & 9.8\times 10^{-5} & -1.4\times 10^{-3} & 2.3\times 10^{-4} & -1.3\times 10^{-2} & 3.4\times 10^{+0} & 7.8\times 10^{-1} & 4.6\times 10^{-3} & 5.8\times 10^{-5} & -1.1\times 10^{-3} & 1.7\times 10^{-4} & -1.0\times 10^{-2} & 2.4\times 10^{+0} & 6.5\times 10^{-1} & 3.2\times 10^{-3} \\
%-1.3\times 10^{-4} & -2.7\times 10^{-3} & 1.4\times 10^{-3} & -1.3\times 10^{-2} & 1.1\times 10^{+0} & 1.4\times 10^{+1} & 6.8\times 10^{-3} & -2.5\times 10^{-5} & -3.0\times 10^{-3} & 1.1\times 10^{-3} & -1.3\times 10^{-2} & 9.4\times 10^{-1} & 1.4\times 10^{+1} & 6.4\times 10^{-3} & 1.8\times 10^{-5} & -2.2\times 10^{-3} & 8.6\times 10^{-4} & -1.1\times 10^{-2} & 1.5\times 10^{+0} & 1.2\times 10^{+1} & 1.6\times 10^{-3} & 2.0\times 10^{-5} & -1.6\times 10^{-3} & 6.7\times 10^{-4} & -7.7\times 10^{-3} & 1.3\times 10^{+0} & 9.5\times 10^{+0} & 3.9\times 10^{-6} & -2.3\times 10^{-5} & -6.0\times 10^{-4} & 5.1\times 10^{-4} & -3.1\times 10^{-3} & 1.1\times 10^{+0} & 7.1\times 10^{+0} & -1.3\times 10^{-3} \\
%2.8\times 10^{-6} & -1.4\times 10^{-5} & 8.3\times 10^{-6} & -2.1\times 10^{-4} & 5.1\times 10^{-3} & 7.1\times 10^{-3} & 2.3\times 10^{-4} & 2.3\times 10^{-6} & -1.2\times 10^{-5} & 6.7\times 10^{-6} & -1.7\times 10^{-4} & 8.4\times 10^{-3} & 6.4\times 10^{-3} & 1.9\times 10^{-4} & 1.6\times 10^{-6} & -8.8\times 10^{-6} & 5.2\times 10^{-6} & -1.4\times 10^{-4} & 8.6\times 10^{-3} & 4.5\times 10^{-3} & 1.5\times 10^{-4} & 1.1\times 10^{-6} & -7.0\times 10^{-6} & 4.0\times 10^{-6} & -1.1\times 10^{-4} & 6.3\times 10^{-3} & 3.2\times 10^{-3} & 1.2\times 10^{-4} & 6.7\times 10^{-7} & -6.3\times 10^{-6} & 3.0\times 10^{-6} & -8.3\times 10^{-5} & 2.1\times 10^{-3} & 2.0\times 10^{-3} & 8.9\times 10^{-5} \\
%3.6\times 10^{-7} & -1.1\times 10^{-6} & 2.4\times 10^{-7} & -1.0\times 10^{-5} & 9.6\times 10^{-5} & -1.7\times 10^{-4} & 2.0\times 10^{-6} & 3.2\times 10^{-7} & -1.0\times 10^{-6} & 2.0\times 10^{-7} & -9.0\times 10^{-6} & 1.4\times 10^{-4} & 1.8\times 10^{-5} & 1.6\times 10^{-6} & 2.8\times 10^{-7} & -9.3\times 10^{-7} & 1.7\times 10^{-7} & -7.8\times 10^{-6} & 9.4\times 10^{-5} & 1.0\times 10^{-4} & 6.8\times 10^{-7} & 2.3\times 10^{-7} & -7.4\times 10^{-7} & 1.3\times 10^{-7} & -6.2\times 10^{-6} & 3.6\times 10^{-5} & 8.4\times 10^{-5} & 2.4\times 10^{-7} & 1.7\times 10^{-7} & -5.5\times 10^{-7} & 9.8\times 10^{-8} & -4.6\times 10^{-6} & -6.3\times 10^{-5} & 3.2\times 10^{-5} & -5.5\times 10^{-8} \\
%-1.2\times 10^{-6} & 1.7\times 10^{-5} & -1.5\times 10^{-6} & 1.2\times 10^{-4} & -2.2\times 10^{-3} & -1.5\times 10^{-3} & -9.4\times 10^{-6} & -1.0\times 10^{-6} & 1.6\times 10^{-5} & -1.2\times 10^{-6} & 1.0\times 10^{-4} & -1.8\times 10^{-3} & -2.2\times 10^{-3} & -8.8\times 10^{-6} & -9.3\times 10^{-7} & 1.4\times 10^{-5} & -9.7\times 10^{-7} & 9.4\times 10^{-5} & -1.3\times 10^{-3} & -2.2\times 10^{-3} & -5.0\times 10^{-6} & -7.6\times 10^{-7} & 1.1\times 10^{-5} & -7.5\times 10^{-7} & 7.6\times 10^{-5} & -9.6\times 10^{-4} & -1.6\times 10^{-3} & -3.4\times 10^{-6} & -5.7\times 10^{-7} & 8.7\times 10^{-6} & -5.6\times 10^{-7} & 5.8\times 10^{-5} & -4.2\times 10^{-4} & -9.0\times 10^{-4} & -2.8\times 10^{-6} \\
%2.4\times 10^{-7} & -1.6\times 10^{-6} & 7.8\times 10^{-7} & -1.4\times 10^{-5} & 2.3\times 10^{-4} & 9.7\times 10^{-4} & 6.4\times 10^{-6} & 2.1\times 10^{-7} & -1.4\times 10^{-6} & 6.7\times 10^{-7} & -1.2\times 10^{-5} & 3.0\times 10^{-4} & 8.6\times 10^{-4} & 5.2\times 10^{-6} & 1.7\times 10^{-7} & -9.7\times 10^{-7} & 5.5\times 10^{-7} & -8.8\times 10^{-6} & 3.3\times 10^{-4} & 6.6\times 10^{-4} & 3.7\times 10^{-6} & 1.3\times 10^{-7} & -7.4\times 10^{-7} & 4.3\times 10^{-7} & -6.8\times 10^{-6} & 2.5\times 10^{-4} & 5.1\times 10^{-4} & 2.8\times 10^{-6} & 9.4\times 10^{-8} & -5.4\times 10^{-7} & 3.2\times 10^{-7} & -5.0\times 10^{-6} & 9.2\times 10^{-5} & 3.6\times 10^{-4} & 2.0\times 10^{-6} \\
%-1.0\times 10^{-5} & 1.2\times 10^{-4} & -1.3\times 10^{-5} & 9.9\times 10^{-4} & -1.6\times 10^{-2} & -6.8\times 10^{-3} & -1.6\times 10^{-4} & -9.1\times 10^{-6} & 1.0\times 10^{-4} & -1.1\times 10^{-5} & 8.7\times 10^{-4} & -1.6\times 10^{-2} & -1.1\times 10^{-2} & -1.4\times 10^{-4} & -7.8\times 10^{-6} & 9.4\times 10^{-5} & -8.8\times 10^{-6} & 7.8\times 10^{-4} & -1.4\times 10^{-2} & -1.0\times 10^{-2} & -9.2\times 10^{-5} & -6.3\times 10^{-6} & 7.6\times 10^{-5} & -6.8\times 10^{-6} & 6.2\times 10^{-4} & -1.0\times 10^{-2} & -7.0\times 10^{-3} & -6.7\times 10^{-5} & -4.7\times 10^{-6} & 5.8\times 10^{-5} & -5.1\times 10^{-6} & 4.7\times 10^{-4} & -4.0\times 10^{-3} & -3.9\times 10^{-3} & -5.1\times 10^{-5} \\
%1.9\times 10^{-4} & -2.6\times 10^{-3} & 5.4\times 10^{-4} & -2.3\times 10^{-2} & 4.5\times 10^{+0} & 1.8\times 10^{+0} & 1.1\times 10^{-2} & 1.4\times 10^{-4} & -2.0\times 10^{-3} & 4.4\times 10^{-4} & -1.8\times 10^{-2} & 4.3\times 10^{+0} & 1.5\times 10^{+0} & 8.6\times 10^{-3} & 9.4\times 10^{-5} & -1.3\times 10^{-3} & 3.3\times 10^{-4} & -1.4\times 10^{-2} & 4.0\times 10^{+0} & 1.2\times 10^{+0} & 6.5\times 10^{-3} & 5.8\times 10^{-5} & -9.9\times 10^{-4} & 2.5\times 10^{-4} & -1.1\times 10^{-2} & 3.2\times 10^{+0} & 1.0\times 10^{+0} & 5.0\times 10^{-3} & 2.6\times 10^{-5} & -7.8\times 10^{-4} & 1.8\times 10^{-4} & -8.2\times 10^{-3} & 2.2\times 10^{+0} & 8.9\times 10^{-1} & 3.6\times 10^{-3} \\
%-5.5\times 10^{-5} & -2.1\times 10^{-3} & 1.0\times 10^{-3} & -1.0\times 10^{-2} & 9.7\times 10^{-1} & 1.1\times 10^{+1} & 4.1\times 10^{-3} & 4.3\times 10^{-5} & -2.6\times 10^{-3} & 8.2\times 10^{-4} & -1.1\times 10^{-2} & 8.6\times 10^{-1} & 1.2\times 10^{+1} & 4.5\times 10^{-3} & 1.0\times 10^{-4} & -2.2\times 10^{-3} & 6.6\times 10^{-4} & -1.0\times 10^{-2} & 1.2\times 10^{+0} & 1.0\times 10^{+1} & -3.8\times 10^{-4} & 8.4\times 10^{-5} & -1.6\times 10^{-3} & 5.2\times 10^{-4} & -7.3\times 10^{-3} & 1.1\times 10^{+0} & 8.6\times 10^{+0} & -1.6\times 10^{-3} & 2.1\times 10^{-5} & -6.0\times 10^{-4} & 4.0\times 10^{-4} & -3.1\times 10^{-3} & 8.8\times 10^{-1} & 6.4\times 10^{+0} & -2.5\times 10^{-3} \\
%1.7\times 10^{-6} & -1.0\times 10^{-5} & 6.0\times 10^{-6} & -1.5\times 10^{-4} & 3.7\times 10^{-3} & 4.9\times 10^{-3} & 1.8\times 10^{-4} & 1.3\times 10^{-6} & -7.6\times 10^{-6} & 4.9\times 10^{-6} & -1.2\times 10^{-4} & 6.0\times 10^{-3} & 1.6\times 10^{-3} & 1.5\times 10^{-4} & 6.8\times 10^{-7} & -5.0\times 10^{-6} & 3.7\times 10^{-6} & -9.2\times 10^{-5} & 6.5\times 10^{-3} & -3.8\times 10^{-4} & 1.3\times 10^{-4} & 3.9\times 10^{-7} & -4.1\times 10^{-6} & 2.9\times 10^{-6} & -7.2\times 10^{-5} & 4.9\times 10^{-3} & -3.9\times 10^{-4} & 1.1\times 10^{-4} & 2.1\times 10^{-7} & -4.4\times 10^{-6} & 2.1\times 10^{-6} & -5.8\times 10^{-5} & 1.9\times 10^{-3} & 6.8\times 10^{-4} & 8.3\times 10^{-5} \\
%2.9\times 10^{-7} & -9.2\times 10^{-7} & 1.9\times 10^{-7} & -8.3\times 10^{-6} & 6.0\times 10^{-5} & -1.2\times 10^{-4} & 1.4\times 10^{-6} & 2.6\times 10^{-7} & -8.4\times 10^{-7} & 1.6\times 10^{-7} & -7.2\times 10^{-6} & 9.8\times 10^{-5} & 2.0\times 10^{-5} & 1.1\times 10^{-6} & 2.3\times 10^{-7} & -7.6\times 10^{-7} & 1.3\times 10^{-7} & -6.3\times 10^{-6} & 5.8\times 10^{-5} & 8.4\times 10^{-5} & 3.9\times 10^{-7} & 1.9\times 10^{-7} & -6.2\times 10^{-7} & 1.0\times 10^{-7} & -5.0\times 10^{-6} & 1.2\times 10^{-5} & 7.2\times 10^{-5} & 4.6\times 10^{-8} & 1.4\times 10^{-7} & -4.6\times 10^{-7} & 8.0\times 10^{-8} & -3.9\times 10^{-6} & -6.4\times 10^{-5} & 3.0\times 10^{-5} & -1.9\times 10^{-7} \\
%-9.2\times 10^{-7} & 1.4\times 10^{-5} & -1.1\times 10^{-6} & 9.5\times 10^{-5} & -1.7\times 10^{-3} & -1.0\times 10^{-3} & -7.4\times 10^{-6} & -8.2\times 10^{-7} & 1.2\times 10^{-5} & -9.4\times 10^{-7} & 8.3\times 10^{-5} & -1.4\times 10^{-3} & -1.6\times 10^{-3} & -7.0\times 10^{-6} & -7.4\times 10^{-7} & 1.1\times 10^{-5} & -7.4\times 10^{-7} & 7.6\times 10^{-5} & -9.9\times 10^{-4} & -1.6\times 10^{-3} & -4.1\times 10^{-6} & -6.2\times 10^{-7} & 9.5\times 10^{-6} & -5.8\times 10^{-7} & 6.3\times 10^{-5} & -7.1\times 10^{-4} & -1.1\times 10^{-3} & -3.0\times 10^{-6} & -4.8\times 10^{-7} & 7.5\times 10^{-6} & -4.4\times 10^{-7} & 5.0\times 10^{-5} & -3.0\times 10^{-4} & -6.5\times 10^{-4} & -2.4\times 10^{-6} \\
%1.9\times 10^{-7} & -1.2\times 10^{-6} & 6.1\times 10^{-7} & -1.1\times 10^{-5} & 1.7\times 10^{-4} & 7.6\times 10^{-4} & 4.9\times 10^{-6} & 1.7\times 10^{-7} & -1.0\times 10^{-6} & 5.2\times 10^{-7} & -8.9\times 10^{-6} & 2.3\times 10^{-4} & 6.7\times 10^{-4} & 4.0\times 10^{-6} & 1.3\times 10^{-7} & -7.5\times 10^{-7} & 4.3\times 10^{-7} & -6.8\times 10^{-6} & 2.5\times 10^{-4} & 5.2\times 10^{-4} & 2.9\times 10^{-6} & 1.0\times 10^{-7} & -5.8\times 10^{-7} & 3.5\times 10^{-7} & -5.3\times 10^{-6} & 1.9\times 10^{-4} & 4.1\times 10^{-4} & 2.2\times 10^{-6} & 7.7\times 10^{-8} & -4.2\times 10^{-7} & 2.6\times 10^{-7} & -4.0\times 10^{-6} & 7.3\times 10^{-5} & 2.8\times 10^{-4} & 1.6\times 10^{-6} \\
%-8.1\times 10^{-6} & 9.4\times 10^{-5} & -1.0\times 10^{-5} & 7.9\times 10^{-4} & -1.3\times 10^{-2} & -4.4\times 10^{-3} & -1.2\times 10^{-4} & -7.1\times 10^{-6} & 8.4\times 10^{-5} & -8.5\times 10^{-6} & 6.9\times 10^{-4} & -1.3\times 10^{-2} & -7.7\times 10^{-3} & -1.1\times 10^{-4} & -6.2\times 10^{-6} & 7.6\times 10^{-5} & -6.8\times 10^{-6} & 6.2\times 10^{-4} & -1.1\times 10^{-2} & -7.3\times 10^{-3} & -7.2\times 10^{-5} & -5.0\times 10^{-6} & 6.3\times 10^{-5} & -5.3\times 10^{-6} & 5.1\times 10^{-4} & -7.8\times 10^{-3} & -4.8\times 10^{-3} & -5.3\times 10^{-5} & -3.9\times 10^{-6} & 5.0\times 10^{-5} & -4.0\times 10^{-6} & 4.0\times 10^{-4} & -3.2\times 10^{-3} & -2.6\times 10^{-3} & -4.0\times 10^{-5} \\
%1.1\times 10^{-4} & -2.0\times 10^{-3} & 4.1\times 10^{-4} & -1.8\times 10^{-2} & 3.7\times 10^{+0} & 1.6\times 10^{+0} & 8.0\times 10^{-3} & 7.4\times 10^{-5} & -1.5\times 10^{-3} & 3.4\times 10^{-4} & -1.4\times 10^{-2} & 3.4\times 10^{+0} & 1.3\times 10^{+0} & 6.3\times 10^{-3} & 3.6\times 10^{-5} & -9.6\times 10^{-4} & 2.5\times 10^{-4} & -1.0\times 10^{-2} & 3.2\times 10^{+0} & 1.1\times 10^{+0} & 4.9\times 10^{-3} & 1.2\times 10^{-5} & -7.1\times 10^{-4} & 1.9\times 10^{-4} & -7.8\times 10^{-3} & 2.6\times 10^{+0} & 9.3\times 10^{-1} & 3.9\times 10^{-3} & -6.0\times 10^{-6} & -5.7\times 10^{-4} & 1.4\times 10^{-4} & -6.2\times 10^{-3} & 1.9\times 10^{+0} & 8.1\times 10^{-1} & 2.9\times 10^{-3} \\
%-1.7\times 10^{-5} & -1.6\times 10^{-3} & 8.0\times 10^{-4} & -7.7\times 10^{-3} & 8.5\times 10^{-1} & 9.6\times 10^{+0} & 3.0\times 10^{-3} & 5.1\times 10^{-5} & -2.0\times 10^{-3} & 6.4\times 10^{-4} & -8.3\times 10^{-3} & 7.8\times 10^{-1} & 9.5\times 10^{+0} & 3.2\times 10^{-3} & 8.4\times 10^{-5} & -1.6\times 10^{-3} & 5.1\times 10^{-4} & -7.0\times 10^{-3} & 1.0\times 10^{+0} & 8.6\times 10^{+0} & -3.9\times 10^{-4} & 7.2\times 10^{-5} & -1.1\times 10^{-3} & 4.1\times 10^{-4} & -4.8\times 10^{-3} & 9.3\times 10^{-1} & 7.3\times 10^{+0} & -1.4\times 10^{-3} & 2.3\times 10^{-5} & -3.9\times 10^{-4} & 3.1\times 10^{-4} & -1.6\times 10^{-3} & 7.8\times 10^{-1} & 5.7\times 10^{+0} & -2.1\times 10^{-3} \\
%1.1\times 10^{-6} & -7.9\times 10^{-6} & 4.6\times 10^{-6} & -1.2\times 10^{-4} & 3.0\times 10^{-3} & 3.2\times 10^{-3} & 1.4\times 10^{-4} & 6.9\times 10^{-7} & -5.6\times 10^{-6} & 3.7\times 10^{-6} & -9.2\times 10^{-5} & 4.6\times 10^{-3} & 3.9\times 10^{-6} & 1.2\times 10^{-4} & 2.4\times 10^{-7} & -3.4\times 10^{-6} & 2.8\times 10^{-6} & -6.7\times 10^{-5} & 5.0\times 10^{-3} & -1.6\times 10^{-3} & 1.1\times 10^{-4} & 4.6\times 10^{-8} & -3.0\times 10^{-6} & 2.2\times 10^{-6} & -5.3\times 10^{-5} & 3.9\times 10^{-3} & -1.4\times 10^{-3} & 8.9\times 10^{-5} & -3.2\times 10^{-8} & -3.6\times 10^{-6} & 1.6\times 10^{-6} & -4.5\times 10^{-5} & 1.7\times 10^{-3} & -4.8\times 10^{-5} & 7.2\times 10^{-5} \\
%2.2\times 10^{-7} & -7.3\times 10^{-7} & 1.4\times 10^{-7} & -6.4\times 10^{-6} & 3.9\times 10^{-5} & -10.0\times 10^{-5} & 9.2\times 10^{-7} & 1.9\times 10^{-7} & -6.2\times 10^{-7} & 1.1\times 10^{-7} & -5.3\times 10^{-6} & 5.8\times 10^{-5} & -2.3\times 10^{-5} & 6.7\times 10^{-7} & 1.7\times 10^{-7} & -5.7\times 10^{-7} & 9.4\times 10^{-8} & -4.7\times 10^{-6} & 2.6\times 10^{-5} & 2.1\times 10^{-5} & 2.1\times 10^{-7} & 1.4\times 10^{-7} & -4.8\times 10^{-7} & 7.7\times 10^{-8} & -3.9\times 10^{-6} & -6.0\times 10^{-6} & 2.3\times 10^{-5} & -3.2\times 10^{-8} & 1.3\times 10^{-7} & -4.2\times 10^{-7} & 6.6\times 10^{-8} & -3.3\times 10^{-6} & -5.8\times 10^{-5} & 3.4\times 10^{-5} & -2.4\times 10^{-7} \\
%-7.2\times 10^{-7} & 1.1\times 10^{-5} & -8.1\times 10^{-7} & 7.7\times 10^{-5} & -1.4\times 10^{-3} & -3.8\times 10^{-4} & -7.2\times 10^{-6} & -6.0\times 10^{-7} & 9.3\times 10^{-6} & -6.9\times 10^{-7} & 6.2\times 10^{-5} & -1.1\times 10^{-3} & -6.0\times 10^{-4} & -6.3\times 10^{-6} & -5.5\times 10^{-7} & 8.7\times 10^{-6} & -5.4\times 10^{-7} & 5.8\times 10^{-5} & -7.8\times 10^{-4} & -6.0\times 10^{-4} & -4.4\times 10^{-6} & -4.6\times 10^{-7} & 7.5\times 10^{-6} & -4.2\times 10^{-7} & 5.0\times 10^{-5} & -5.7\times 10^{-4} & -3.9\times 10^{-4} & -3.6\times 10^{-6} & -4.2\times 10^{-7} & 7.0\times 10^{-6} & -3.3\times 10^{-7} & 4.6\times 10^{-5} & -2.7\times 10^{-4} & -3.8\times 10^{-4} & -2.8\times 10^{-6} \\
%1.4\times 10^{-7} & -8.8\times 10^{-7} & 4.7\times 10^{-7} & -7.7\times 10^{-6} & 1.3\times 10^{-4} & 5.6\times 10^{-4} & 3.6\times 10^{-6} & 1.2\times 10^{-7} & -7.7\times 10^{-7} & 3.8\times 10^{-7} & -6.6\times 10^{-6} & 1.7\times 10^{-4} & 5.1\times 10^{-4} & 3.0\times 10^{-6} & 9.8\times 10^{-8} & -5.6\times 10^{-7} & 3.2\times 10^{-7} & -5.1\times 10^{-6} & 1.8\times 10^{-4} & 4.0\times 10^{-4} & 2.1\times 10^{-6} & 8.0\times 10^{-8} & -4.4\times 10^{-7} & 2.6\times 10^{-7} & -4.0\times 10^{-6} & 1.4\times 10^{-4} & 3.1\times 10^{-4} & 1.6\times 10^{-6} & 6.6\times 10^{-8} & -3.3\times 10^{-7} & 2.2\times 10^{-7} & -3.2\times 10^{-6} & 5.7\times 10^{-5} & 2.2\times 10^{-4} & 1.2\times 10^{-6} \\
%-6.3\times 10^{-6} & 7.6\times 10^{-5} & -7.4\times 10^{-6} & 6.3\times 10^{-4} & -1.1\times 10^{-2} & -7.6\times 10^{-4} & -9.9\times 10^{-5} & -5.3\times 10^{-6} & 6.3\times 10^{-5} & -6.2\times 10^{-6} & 5.1\times 10^{-4} & -1.0\times 10^{-2} & -3.1\times 10^{-3} & -8.3\times 10^{-5} & -4.6\times 10^{-6} & 5.8\times 10^{-5} & -5.0\times 10^{-6} & 4.7\times 10^{-4} & -8.2\times 10^{-3} & -3.1\times 10^{-3} & -5.8\times 10^{-5} & -3.9\times 10^{-6} & 5.0\times 10^{-5} & -4.0\times 10^{-6} & 4.0\times 10^{-4} & -6.2\times 10^{-3} & -1.6\times 10^{-3} & -4.5\times 10^{-5} & -3.3\times 10^{-6} & 4.6\times 10^{-5} & -3.2\times 10^{-6} & 3.7\times 10^{-4} & -3.0\times 10^{-3} & -1.2\times 10^{-3} & -3.3\times 10^{-5} \\
%-3.6\times 10^{-5} & -1.1\times 10^{-3} & 1.6\times 10^{-4} & -8.1\times 10^{-3} & 2.9\times 10^{+0} & 1.4\times 10^{+0} & 2.6\times 10^{-3} & -5.2\times 10^{-5} & -7.7\times 10^{-4} & 1.3\times 10^{-4} & -6.0\times 10^{-3} & 2.4\times 10^{+0} & 1.1\times 10^{+0} & 2.1\times 10^{-3} & -6.3\times 10^{-5} & -4.2\times 10^{-4} & 9.2\times 10^{-5} & -4.0\times 10^{-3} & 2.2\times 10^{+0} & 8.8\times 10^{-1} & 1.9\times 10^{-3} & -6.4\times 10^{-5} & -3.0\times 10^{-4} & 7.3\times 10^{-5} & -3.2\times 10^{-3} & 1.9\times 10^{+0} & 7.8\times 10^{-1} & 1.7\times 10^{-3} & -5.8\times 10^{-5} & -2.7\times 10^{-4} & 5.7\times 10^{-5} & -3.0\times 10^{-3} & 1.8\times 10^{+0} & 6.9\times 10^{-1} & 1.5\times 10^{-3} \\
%-6.7\times 10^{-6} & -1.2\times 10^{-3} & 5.9\times 10^{-4} & -4.9\times 10^{-3} & 6.7\times 10^{-1} & 7.9\times 10^{+0} & 2.3\times 10^{-3} & 9.5\times 10^{-6} & -1.1\times 10^{-3} & 4.8\times 10^{-4} & -4.1\times 10^{-3} & 6.5\times 10^{-1} & 7.1\times 10^{+0} & 2.0\times 10^{-3} & 3.2\times 10^{-5} & -9.0\times 10^{-4} & 3.6\times 10^{-4} & -3.9\times 10^{-3} & 8.9\times 10^{-1} & 6.4\times 10^{+0} & 6.8\times 10^{-4} & 3.0\times 10^{-5} & -6.5\times 10^{-4} & 2.8\times 10^{-4} & -2.6\times 10^{-3} & 8.1\times 10^{-1} & 5.7\times 10^{+0} & -4.8\times 10^{-5} & 3.4\times 10^{-5} & -3.8\times 10^{-4} & 2.2\times 10^{-4} & -1.2\times 10^{-3} & 6.9\times 10^{-1} & 5.4\times 10^{+0} & -6.4\times 10^{-4} \\
%4.7\times 10^{-7} & -5.2\times 10^{-6} & 3.1\times 10^{-6} & -7.9\times 10^{-5} & 2.1\times 10^{-3} & 1.8\times 10^{-3} & 1.1\times 10^{-4} & 2.4\times 10^{-7} & -4.0\times 10^{-6} & 2.6\times 10^{-6} & -6.7\times 10^{-5} & 3.2\times 10^{-3} & -1.3\times 10^{-3} & 8.9\times 10^{-5} & -5.5\times 10^{-8} & -2.8\times 10^{-6} & 2.0\times 10^{-6} & -5.1\times 10^{-5} & 3.6\times 10^{-3} & -2.5\times 10^{-3} & 8.3\times 10^{-5} & -1.9\times 10^{-7} & -2.4\times 10^{-6} & 1.6\times 10^{-6} & -4.0\times 10^{-5} & 2.9\times 10^{-3} & -2.1\times 10^{-3} & 7.2\times 10^{-5} & -2.4\times 10^{-7} & -2.8\times 10^{-6} & 1.2\times 10^{-6} & -3.3\times 10^{-5} & 1.5\times 10^{-3} & -6.4\times 10^{-4} & 6.6\times 10^{-5} \\
%\end{bmatrix}
%\end{equation}
\bibliographystyle{aasjournal}
\bibliography{alex}
%\documentclass[trackchanges]{aastex62}   	% use "amsart" instead of "article" for AMSLaTeX format
\usepackage{geometry}                		% See geometry.pdf to learn the layout options. There are lots.
\geometry{letterpaper}                   		% ... or a4paper or a5paper or ... 
\usepackage{graphicx}				% Use pdf, png, jpg, or eps§ with pdflatex; use eps in DVI mode
\usepackage{amsmath}
\usepackage{amssymb}
\usepackage{natbib}
\usepackage{lineno}
\usepackage{color}
\defcitealias{1999PASP..111...63F}{F99}
\defcitealias{2017ApJ...842...93M}{M17}
\linenumbers

\begin{document}

\title{Evidence for a Third Color  Parameter Within the Type~Ia Supernovae of the Nearby Supernova Factory}
\author[0000-0001-6315-8743]{A.~G.~Kim}
\affiliation{    Physics Division, Lawrence Berkeley National Laboratory, 
    1 Cyclotron Road, Berkeley, CA, 94720}
    
\author{     G.~Aldering}
\affiliation{    Physics Division, Lawrence Berkeley National Laboratory, 
    1 Cyclotron Road, Berkeley, CA, 94720}

\author{     P.~Antilogus}
\affiliation{    Sorbonne Universit\'e, Universit\'e Paris Diderot, CNRS/IN2P3, Laboratoire de Physique Nucl\'eaire et de Hautes Energies, LPNHE, 4 Place Jussieu, F-75252, Paris, France}
    
\author{     S.~Bailey}
\affiliation{    Physics Division, Lawrence Berkeley National Laboratory, 
    1 Cyclotron Road, Berkeley, CA, 94720}

\author{     C.~Baltay}
\affiliation{    Department of Physics, Yale University, 
    New Haven, CT, 06250-8121}

\author{     K.~Barbary}
\affiliation{
    Department of Physics, University of California Berkeley,
    366 LeConte Hall MC 7300, Berkeley, CA, 94720-7300}

\author{    D.~Baugh}
\affiliation{   Tsinghua Center for Astrophysics, Tsinghua University, Beijing 100084, China }

\author{     K.~Boone}
\affiliation{    Physics Division, Lawrence Berkeley National Laboratory, 
    1 Cyclotron Road, Berkeley, CA, 94720}
\affiliation{
    Department of Physics, University of California Berkeley,
    366 LeConte Hall MC 7300, Berkeley, CA, 94720-7300}

\author{     S.~Bongard}
\affiliation{ Sorbonne Universit\'e, Universit\'e Paris Diderot, CNRS/IN2P3, Laboratoire de Physique Nucl\'eaire et de Hautes Energies, LPNHE, 4 Place Jussieu, F-75252, Paris, France}

\author{     C.~Buton}
\affiliation{    Universit\'e de Lyon, F-69622, Lyon, France ; Universit\'e de Lyon 1, Villeurbanne ; 
    CNRS/IN2P3, Institut de Physique Nucl\'eaire de Lyon}
    
\author{     J.~Chen}
\affiliation{   Tsinghua Center for Astrophysics, Tsinghua University, Beijing 100084, China }

\author{     N.~Chotard}
\affiliation{    Universit\'e de Lyon, F-69622, Lyon, France ; Universit\'e de Lyon 1, Villeurbanne ; 
    CNRS/IN2P3, Institut de Physique Nucl\'eaire de Lyon}
    
\author[0000-0002-5317-7518]{     Y.~Copin}
\affiliation{    Universit\'e de Lyon, F-69622, Lyon, France ; Universit\'e de Lyon 1, Villeurbanne ; 
    CNRS/IN2P3, Institut de Physique Nucl\'eaire de Lyon}

\author{ S.~Dixon}
\affiliation{
    Department of Physics, University of California Berkeley,
    366 LeConte Hall MC 7300, Berkeley, CA, 94720-7300}

\author{     P.~Fagrelius}
\affiliation{    Physics Division, Lawrence Berkeley National Laboratory, 
    1 Cyclotron Road, Berkeley, CA, 94720}
\affiliation{
    Department of Physics, University of California Berkeley,
    366 LeConte Hall MC 7300, Berkeley, CA, 94720-7300}

\author{     H.~K.~Fakhouri}
\affiliation{    Physics Division, Lawrence Berkeley National Laboratory, 
    1 Cyclotron Road, Berkeley, CA, 94720}
  \affiliation{
    Department of Physics, University of California Berkeley,
    366 LeConte Hall MC 7300, Berkeley, CA, 94720-7300}

\author{     U.~Feindt}
\affiliation{The Oskar Klein Centre, Department of Physics, AlbaNova, Stockholm University, SE-106 91 Stockholm, Sweden}

\author{     D.~Fouchez}
\affiliation{ Aix Marseille Univ, CNRS/IN2P3, CPPM, Marseille, France
}
    
\author[0000-0001-6728-1423]{     E.~Gangler}  
\affiliation{    Clermont Universit\'e, Universit\'e Blaise Pascal, CNRS/IN2P3, Laboratoire de Physique Corpusculaire,
    BP 10448, F-63000 Clermont-Ferrand, France}
    
\author{     B.~Hayden}
\affiliation{    Physics Division, Lawrence Berkeley National Laboratory, 
    1 Cyclotron Road, Berkeley, CA, 94720}

\author{     W.~Hillebrandt}
\affiliation{    Max-Planck-Institut f\"ur Astrophysik, Karl-Schwarzschild-Str. 1,
D-85748 Garching, Germany}

\author{     M.~Kowalski}
\affiliation{    Institut fur Physik,  Humboldt-Universitat zu Berlin,
    Newtonstr. 15, 12489 Berlin}
\affiliation{ DESY, D-15735 Zeuthen, Germany}

\author{     P.-F.~Leget}
\affiliation{    Clermont Universit\'e, Universit\'e Blaise Pascal, CNRS/IN2P3, Laboratoire de Physique Corpusculaire,
    BP 10448, F-63000 Clermont-Ferrand, France}
    
\author{     S.~Lombardo}
\affiliation{    Institut fur Physik,  Humboldt-Universitat zu Berlin,
    Newtonstr. 15, 12489 Berlin}
    
\author{     J.~Nordin}
\affiliation{    Institut fur Physik,  Humboldt-Universitat zu Berlin,
    Newtonstr. 15, 12489 Berlin}
    
\author{     R.~Pain}
\affiliation{ Sorbonne Universit\'e, Universit\'e Paris Diderot, CNRS/IN2P3, Laboratoire de Physique Nucl\'eaire et de Hautes Energies, LPNHE, 4 Place Jussieu, F-75252, Paris, France}
     
\author{     E.~Pecontal}
\affiliation{   Centre de Recherche Astronomique de Lyon, Universit\'e Lyon 1,
    9 Avenue Charles Andr\'e, 69561 Saint Genis Laval Cedex, France}
    
\author{    R.~Pereira}
 \affiliation{    Universit\'e de Lyon, F-69622, Lyon, France ; Universit\'e de Lyon 1, Villeurbanne ; 
    CNRS/IN2P3, Institut de Physique Nucl\'eaire de Lyon}
 
 \author{    S.~Perlmutter}
 \affiliation{    Physics Division, Lawrence Berkeley National Laboratory, 
    1 Cyclotron Road, Berkeley, CA, 94720} 
\affiliation{
    Department of Physics, University of California Berkeley,
    366 LeConte Hall MC 7300, Berkeley, CA, 94720-7300}
    
 \author{    D.~Rabinowitz}
 \affiliation{    Department of Physics, Yale University, 
    New Haven, CT, 06250-8121}
    
 \author[0000-0002-8121-2560]{    M.~Rigault} 
\affiliation{ Clermont Universit\'e, Universit\'e Blaise Pascal, CNRS/IN2P3, Laboratoire de Physique Corpusculaire,
    BP 10448, F-63000 Clermont-Ferrand, France}
     
 \author{    D.~Rubin}
 \affiliation{    Physics Division, Lawrence Berkeley National Laboratory, 
    1 Cyclotron Road, Berkeley, CA, 94720}
    \affiliation{   Space Telescope Science Institute, 3700 San Martin Drive, Baltimore, MD 21218}
 
 \author{    K.~Runge}
 \affiliation{    Physics Division, Lawrence Berkeley National Laboratory, 
    1 Cyclotron Road, Berkeley, CA, 94720}
 
 \author{    C.~Saunders}
 \affiliation{    Physics Division, Lawrence Berkeley National Laboratory, 
    1 Cyclotron Road, Berkeley, CA, 94720}
 \affiliation{Sorbonne Universit\'es, Institut Lagrange de Paris (ILP), 98 bis Boulevard Arago, 75014 Paris, France}

\author{    C.~Sofiatti}
\affiliation{    Physics Division, Lawrence Berkeley National Laboratory, 
    1 Cyclotron Road, Berkeley, CA, 94720} 
\affiliation{
    Department of Physics, University of California Berkeley,
    366 LeConte Hall MC 7300, Berkeley, CA, 94720-7300}

\author{    N.~Suzuki}
\affiliation{    Physics Division, Lawrence Berkeley National Laboratory, 
    1 Cyclotron Road, Berkeley, CA, 94720}

\author{     S.~Taubenberger}
\affiliation{    Max-Planck-Institut f\"ur Astrophysik, Karl-Schwarzschild-Str. 1,
D-85748 Garching, Germany}

\author{     C.~Tao}
\affiliation{   Tsinghua Center for Astrophysics, Tsinghua University, Beijing 100084, China }
\affiliation{ Aix Marseille Univ, CNRS/IN2P3, CPPM, Marseille, France
}
   
\author{     R.~C.~Thomas}
\affiliation{    Computational Cosmology Center, Computational Research Division, Lawrence Berkeley National Laboratory, 
    1 Cyclotron Road MS 50B-4206, Berkeley, CA, 94720}
    
\collaboration{(The Nearby Supernova Factory)}


\begin{abstract}
Through empirical modeling of its observed signal, the peak absolute magnitude of a Type~Ia supernova (SN~Ia) can be accurately determined,
making SNe~Ia  excellent distance indicators.  Improved modeling of SN~Ia colors and magnitudes can account for
further physical
diversity that is expected but not included
in current models, and so consequently can lead to more precise per-object distances with smaller systematic uncertainties.  In this article, we present   
an empirical model for SN~Ia peak colors with
three latent color parameters and dependence on the equivalent widths of Ca~II and Si~II, the Si~II velocity,
and the light-curve shape.
This model is applied to the supernova sample of the Nearby Supernova Factory.  The peak magnitudes in synthetic
broadband photometry and their colors are found to be 
dependent on the observable features and on the three color parameters.
The color changes allowed by two of the color parameters are similar to those allowed by the extinction model of  \citet{1999PASP..111...63F}. 
Correspondingly the sample has an effective value of $\langle R^F_{\mathit{eff}}\rangle \sim 2.43$, though with a broad distribution of 
total-to-selective extinctions. 
We detect the influence on color by a third parameter at
%--- table.py
$> 99\%$
%-----
confidence.
The third parameter has  unique color characteristics that distinguish it from 
those already  tracked by the spectral features, light-curve shape, and two-parameter dust models.
These color parameters must be accounted for when using SNe~Ia as distance indicators.
\end{abstract}

\keywords{supernovae: general; cosmology: distance scale; methods: data analysis}

\section{Introduction}
Type~Ia supernovae (SNe~Ia) form a homogenous set of exploding stars and as such were early recognized and utilized as a powerful distance indicator 
and probe of cosmology \citep[e.g.][]{1992ARA&A..30..359B, 1993ApJ...415....1S}.  After further careful consideration of supernova data, it was recognized
that SN~Ia light-curve shapes \citep{1993ApJ...413L.105P} and colors \citep{1996ApJ...473...88R, 1998A&A...331..815T} exhibit subtle signs of heterogeneity
that are correlated with absolute magnitude, and must be considered when inferring distances.  Empirical models parameterizing SNe~Ia by their light-curve shape \citep{1996ApJ...473...88R,
1997ApJ...483..565P,
1999ApJ...517..565P}
and color  \citep{1996ApJ...473...88R}  were developed that enabled absolute magnitude corrections
and accurate distance measurements of cosmological supernovae,
which 
were subsequently used in the discovery of the accelerating expansion of the Universe \citep{1998AJ....116.1009R,1999ApJ...517..565P}.

The two most commonly used supernova-cosmology light-curve fitters today are SALT2 \citep{2007A&A...466...11G} and MLCS2k2
\citep{2007ApJ...659..122J}.\footnote{Light-curve fitters with more flexible degrees of freedom
\citep[e.g.][]{2008ApJ...681..482C, 2011AJ....141...19B, 2011ApJ...731..120M} are available and have for
the most part been used to study SN~Ia heterogeneity.}
They remain two-parameter models, with one parameter
primarily
characterizing light-curve shape and the other
color.
In SALT2 the light curve shapes are described by phase-dependent flux corrections,
 whereas MLCS2k2 varies shapes through additive magnitude corrections.
The physical cause of the color diversity is interpreted differently by the two sets of authors: 
\citet{2007A&A...466...11G} pragmatically extract color variation empirically from SNe that span a wide range of colors, with no attribution
to either intrinsic or extrinsic origins;
\citet{2007ApJ...659..122J}
attribute changes in color
partially to intrinsic variations linked to light-curve shape, and partially
to the reddening of light from host-galaxy dust.  Differences between these models produce differences in the results of
analyses of both low-redshift \citep{2007ApJ...664L..13C} and high-redshift \citep{2009ApJS..185...32K} supernovae.

There is evidence that supports the expectation that a single parameter beyond light-curve shape  cannot describe the full range
of colors seen in the SN~Ia population.  One approach to look for color diversity is to find correlations between color and spectral features.
\citet{2009ApJ...699L.139W, 2011ApJ...729...55F} find two subpopulations distinguished
by Si~II velocity that exhibit differing $B_{\mathit{\mathit{max}}}-V_{\mathit{max}}$; this color correlation, in addition to one with $B-R$, is confirmed by
\citet{2014ApJ...797...75M}.
\citet{2009ApJ...699L.139W,2015MNRAS.451.1973S}
find that high-velocity Si~II~$\lambda$6355 is found in objects that have red ultraviolet/optical colors near maximum brightness.
\citet{2011MNRAS.413.3075M} show evidence that supernova asymmetry and viewing angle,
traced by wavelength shifts in nebular emission lines, is an important determinant in controlling supernova color; such correlations are also seen by \citet{2011A&A...534L..15C}.

Another approach to probe color diversity is through multiple colors (at least 3 bands)
of individual supernovae.  Color ratios are sensitive to processes of the responsible physics.   For example,
relative dust absorption varies as a function of wavelength depending on grain size, distribution, composition and shape,
independent (to first order) of the amount of dust along the line of sight.
Near and mid-UV photometry obtained by the Ultra-Violet/Optical Telescope on the Swift spacecraft cannot be entirely explained
by dust absorption and hence imply intrinsic variability of supernova UV colors
\citep{2017ApJ...836..232B}.
Measurements of color ratios are being advanced with the development of flexible empirical light curve models that accommodate flexibility in multi-band colors
\citep[e.g.][]{2011ApJ...731..120M}.
\citet{2014ApJ...789...32B, 2015MNRAS.453.3300A} find wide
ranges of total-to-selective extinction with average values significantly lower than $R_V = 3.1$,
the canonical value for diffuse Milky Way dust.
They also confirm the \citet{2006MNRAS.369.1880E, 2011ApJ...731..120M} finding that low $R_V$ is associated with high-extinction supernovae.
In contrast, \citet{2011A&A...529L...4C} argue that after accounting for the diversity of spectral features,
the $R_V=3.1$ measured for the diffuse Milky Way dust is recovered on average and \citet{2017ApJ...836..157H}
find $R_V=2.95 \pm 0.08$ for the highly-extincted SN~2012cu.

Even two parameters are probably not enough to capture supernova color diversity.  SNe~Ia 
are affected by interstellar Milky Way-like dust  and by ``intrinsic'' color variations from the supernova itself and its surrounding circumstellar dust. Physical modeling
shows that the complexity of interstellar dust shouldn't be expected to be captured by a single parameter \citep{2015ApJ...807L..26G,
2017ApJ...836...13H}. Supernova models \citep[e.g.][]{2007ApJ...662..487W,2013MNRAS.436..333S,2014MNRAS.441..532D,2015MNRAS.454.2549B,
2017ApJ...846...58H} and circumstellar dust \citep{2005ApJ...635L..33W,
2008ApJ...686L.103G} produce color variations beyond those due to interstellar dust.  A third color
parameter is therefore expected.

Hierarchical modeling has recently enabled
the study of intrinsic supernova color based on SN~Ia Hubble diagrams
by allowing the addition of a supernova population model as an extra level within in the larger model.
Latent parameters that are not directly tied to observables
but  influence color can be included in such models.
\citet{2017ApJ...842...93M} \citepalias[henceforth referred to as][]{2017ApJ...842...93M}) take the approach of modeling the distribution of their parameters to find that
scatter in the Hubble diagram is better explained by a combination of 
an intrinsic color-magnitude relationship with slope $\beta_{\mathit{int}}\sim 2.3$ intrinsic
color dispersion,
and
$R_V=2.7$ dust, rather than by dust with no color dispersion.
They draw these conclusions by using only the SALT2 (v2.4)
 $c$
parameter as the summary statistic that describes color.

The Nearby Supernova Factory \citep[SNfactory;][]{2002SPIE.4836...61A} has systematically observed the
spectrophotometric time series of hundreds of Hubble-flow $0.03<z<0.08$ SNe~Ia.   The $3200$--$10000$~\AA\ spectral coverage
provides measurements of an array of supernova spectral features while also enablng synthetic broadband photometry
spanning near-UV to near-IR SN-frame wavelengths.  SNfactory specifically targeted objects
early in their temporal evolution, so that well over a hundred of these supernovae have  coverage over
peak brightness.  This dataset provides a homogenous sample with which to study SN~Ia colors and spectral features simultaneously.

In this article we use the idea that spectral indicators carry information on intrinsic supernova colors at peak magnitude.
This approach is taken by \citet{2011A&A...529L...4C}, who find that after standardization based on Ca and Si features, remaining residual color
variation is consistent with Milky Way dust models.
We accommodate up to  three independent color parameters.
The data used in this analysis are described in \S\ref{data:sec}.
The analysis itself is presented as a series of three models of increasing complexity.
\S\ref{modelI:sec}  and  \S\ref{modelII:sec} present the first two models, providing an introduction to
our methodology and yielding results to be compared with those from the third model.
That ultimate model and its results are discussed in detail in \S\ref{modelIII:sec}. 
In \S\ref{modelI:sec} we present Model~I, 
a first analysis using spectral features to standardize SN colors
and two additional latent parameters to account for color diversity. 
By latent, we mean that the parameters are not directly associated with a single observable but rather deduced
as part of a hierarchical
model.
For Model~II described in \S\ref{modelII:sec}, we add a third latent color parameter to the analysis.
In \S\ref{modelIII:sec} we present our final Model~III, which  includes light-curve shape as an additional standardization parameter.
Conclusions are presented in \S\ref{conclusions:sec}.

\section{Data}
\label{data:sec}

Our analysis uses the spectrophotometric data set obtained by
the SNfactory with the SuperNova Integral Field
Spectrograph \citep[SNIFS,][]{2002SPIE.4836...61A, 2004SPIE.5249..146L}.  SNIFS is a fully integrated
instrument optimized for automated observation of point sources on a
structured background over the full ground-based optical window at
moderate spectral resolution ($R \sim 500$).  It consists of a
high-throughput wide-band lenslet integral field spectrograph, a multi-band
imager that covers the field in the vicinity of
the IFS for atmospheric transmission monitoring simultaneous with
spectroscopy, and an acquisition/guiding channel.  The IFS possesses a
fully-filled $6\farcs 4 \times 6\farcs 4$ spectroscopic field of view
subdivided into a grid of $15 \times 15$ spatial elements, a
dual-channel spectrograph covering 3200--5200~\AA\ and 5100--10000~\AA\
simultaneously, and an internal calibration unit (continuum and arc
lamps).  SNIFS is mounted on the south bent Cassegrain port of the
University of Hawaii 2.2~m telescope on Mauna Kea, and is operated
remotely.  Observations are reduced using the SNfactory's dedicated data
reduction pipeline, similar to that presented in \S4 of \citet{2001MNRAS.326...23B}.
A discussion of the software pipeline is presented in
\citet{2006ApJ...650..510A} and is updated in \citet{2010ApJ...713.1073S}. 
The flux calibration is presented in \citet{2013A&A...549A...8B}.
A detailed
description of host-galaxy subtraction is given in \citet{2011MNRAS.418..258B}.

\added{The SNfactory sample used here consists of 198 SNe~Ia
having fully-processed spectrophotometric lightcurve data, including
observations on at least two photometric nights, final references, and a
host spectroscopic redshift.   This sample was defined for use in \citet{leget:thesis}.
These supernovae have well-sampled lightcurves as defined by having
at minimum 5 spectra while still active, with at least four within
epochs  $[-10, 35]$ days, at least one within $[-10, 7]$ days, at least one within $[7, 20]$ days,
and at least one color (Anyone know what this means? See slide 3 of
\url{https://snf-doc.lbl.gov/twiki/pub/Conf/VideoConf20160317/IDR_03-17-2016.pdf}) within $[-8, 10]$ days of peak $B$ brightness.
Synthetic-photometric lightcurves derived from the spectra are fit to the SALT2~(v2.4)
model; we require that supernovae have no more than 20\% points that are
$>0.2$ mag from the best-fit SALT2 model, and require a Normalized Median Absolute Deviation (nMAD) of residuals
of  $< 0.12$ mag relative to this model.  \citep[It is known that more flexibility than available to SALT2 is required to capture
the full range of observed SN~Ia behavior, e.g.\ ][]{2011AJ....141...19B, 2011ApJ...731..120M}.
%These all have at least 5 spectra while the SN is
%active, and pass the quality cuts suggested by \citet{2010A&A...523A...7G}.
More than 80\% of our SNe are from searches where there was no
pre-selection based on host galaxy properties (those
whose names start with ``SNF'', ``LSQ'', or ``PTF'' in Table~\ref{data:tab}).

\citet{2013ApJ...770..108C}  present the distribution of host-galaxy parameters for a subset of the  above sample.
\citet{2015MNRAS.447.1247S} show scatter plots of the peak Si~II~$\lambda 5972$ and
Si~II~$\lambda 6355$ velocities and pseudo-equivalent
widths, absolute $B$ and $V$ magnitudes, $B-V$, SALTII $c$ and $x_1$ parameters, and $\Delta m_{15}$ for 
an preceding sample of 151 SNe.   

%Quality criteria for a 'good' lightcurve are the following: At least 5 nights of observation
%No more than 20\% of rejection (after outlier cut>0.2mag) nMAD of residuals < 0.12 mag
%Phase coverage (4 epochs in [-10 < p < +35] days, 1 epochs in [-10 < p < +7] days, 1 epochs in [+7 < p < +20] days,1 color in [-8 < p < +10] days)

From that sample, we further require at least one spectrum with $\pm 2.5$ days of B-band peak brightness.
We exclude six SNe~Ia in the
SN~1991T, SN~1991bg and SN~2002cx subclasses,
SNF20080905-005 described in \citet{leget:thesis} whose spectral features make it a $>3\sigma$ outlier in the
sample population, and
the extremely red SN2012cu  who has a SALT2 $C \approx 1$ \citep{2017ApJ...836..157H} that is several times larger than the colors
of in-sample supernovae.

Considering that our empirical parameters
are not fundamental supernova parameters, their best-fit values are dependent on sample selection.
Nevertheless, this  article seeks
underlying parameters that significantly
affect SN~Ia colors; any such parameters found in a subsample are still
relevant parameters when dealing with the population as a whole.
Within this context, it is not so important whether our sample population is representative
of the underlying population.

There is a fraction of SNfactory-observed supernovae that do have the
observations to satisfy the well-sampled light-curve criteria, but are excluded due to some observations
having not completed full processing.  Incomplete processing can occur for data taken during poor
observing conditions, when there are instrumental artifacts, and late in a supernova's
light-curve when the signal-to-noise and contrast relative to the host are low.  The sample-selection exclusion of these supernovae are not related,
at least directly, to the at-maximum observables used in this analysis.  Our results are insensitive to potential sample-selection
bias due to stochasticity in the measurements.
}


A recent description of the data is presented in \citet{2015ApJ...815...58F}.
We provide a brief summary of the points important for this analysis.
The spectral time-series  are corrected for Milky Way dust
extinction \citep{1989ApJ...345..245C,1998ApJ...500..525S}.  
Each spectral time series is
blue-shifted to rest-frame
based on the systemic redshift of the host \citep[c.f.][]{2013ApJ...770..107C}, and the fluxes are converted to luminosity assuming
distances expected for the supernova redshifts given a flat
$\Lambda$CDM cosmology with $\Omega_M = 0.28$ (with an arbitrarily selected
$H_0$ since the current analysis does not depend on absolute magnitude).

This article presents the first application of a computationally intensive analysis, so although  high spectral resolution
is available we here compress the information into low-resolution broadband photometry. 
Synthetic supernova-frame photometry is generated for a top-hat filter system
comprised of five 
bands with the following wavelength ranges: ${\hat{U}}$ $[3300.00 - 3978.02]$\AA;
${\hat{B}}$ $[3978.02-4795.35]$\AA;
$\hat{V}$ $[4795.35-5780.60]$\AA;
$\hat{R}$ $[5780.60-6968.29]$\AA;
$\hat{I}$ $[6968.29-8400.00]$\AA.
(The diacritic hat serves as a reminder that these are not standard Johnson-Cousins filters.)
For each supernova, the magnitudes within 5-days of peak brightness are used to fit single-band magnitudes
at $B$-band peak brightness.
The equivalent widths of the Si~II~$\lambda 4141$ and Ca~II H\&K features are computed as
in \citet{2008A&A...477..717B} and the 
wavelength of the Si~II~$\lambda 6355$ feature
as in \citet{chotard:thesis, 2017Chotard}.
Equivalent widths and the
Si~II~$\lambda 6355$ wavelength are taken from spectra  within $\pm 2.5$ days from $B$-band maximum;
the average is used  in cases where there are multiple spectral measurements within that time window.
All the above spectral features have units of \AA ngstroms.

Our analysis sample is comprised of the
172
supernovae that have been fully processed and have the data coverage to 
give photometric and spectroscopic statistics described above.
The 
spectroscopic-feature measurements, 
\replaced{SALT2 light-curve shape parameters  $x_1$,
and their uncertainties
are presented
in \citet{2017Chotard}}{
synthetic magnitudes (plus a constant offset), SALT2 light-curve shape parameters  $x_1$,
and their uncertainties
are presented
in 
Table~\ref{data:tab}. 
For the subset supernovae for whom it is available, we use the host-galaxy mass given in \citet{2013ApJ...770..108C}.

\startlongtable
\begin{deluxetable}{crrrrrrrrr}
\tabletypesize{\tiny}
\tablecaption{Supernova Spectral-Feature and Peak-Magnitude Data
\label{data:tab}}
\tablehead{
\colhead{Name} & \colhead{$EW_{\mathrm{Ca}}$ (\AA)} & \colhead{$EW_{\mathrm{Si}}$ (\AA)} & \colhead{$\lambda_{\mathrm{Si}}$ (\AA)} & \colhead{$U$+offset} & \colhead{$B$+offset} & \colhead{$V$+offset} & \colhead{$R$+offset} & \colhead{$I$+offset} & \colhead{$x_1$}
}
\startdata
SN2007bd & $109.7 \pm 5.9$ & $ 17.5 \pm 0.7$& $ 6101 \pm   3$ & $-29.31 \pm   0.01$ & $-29.12 \pm   0.01$& $-28.60 \pm   0.01$& $-28.35 \pm   0.01$& $-27.60 \pm   0.01$ & $ -0.85 \pm   0.34$\\
PTF10zdk & $149.7 \pm 1.2$ & $ 14.3 \pm 0.6$& $ 6150 \pm   3$ & $-28.61 \pm   0.02$ & $-28.69 \pm   0.02$& $-28.32 \pm   0.02$& $-28.08 \pm   0.02$& $-27.40 \pm   0.02$ & $  0.18 \pm   0.17$\\
SNF20080815-017 & $ 63.8 \pm 21.5$ & $ 27.6 \pm 3.8$& $ 6132 \pm   6$ & $-29.04 \pm   0.07$ & $-28.79 \pm   0.07$& $-28.32 \pm   0.07$& $-28.12 \pm   0.07$& $-27.41 \pm   0.07$ & $ -1.78 \pm   0.18$\\
PTF09dnl & $129.9 \pm 0.9$ & $  9.5 \pm 0.7$& $ 6093 \pm   3$ & $-29.23 \pm   0.01$ & $-29.07 \pm   0.01$& $-28.72 \pm   0.01$& $-28.44 \pm   0.01$& $-27.69 \pm   0.01$ & $  0.72 \pm   0.13$\\
SN2010ex & $114.4 \pm 0.9$ & $  8.4 \pm 0.4$& $ 6129 \pm   6$ & $-29.26 \pm   0.01$ & $-28.99 \pm   0.01$& $-28.50 \pm   0.01$& $-28.20 \pm   0.01$& $-27.44 \pm   0.01$ & $  0.27 \pm   0.14$\\
PTF09dnp & $ 64.9 \pm 4.5$ & $ 16.5 \pm 0.7$& $ 6098 \pm   4$ & $-29.55 \pm   0.02$ & $-29.19 \pm   0.02$& $-28.68 \pm   0.02$& $-28.48 \pm   0.02$& $-27.93 \pm   0.02$ & $ -0.74 \pm   0.24$\\
PTF11bnx & $151.4 \pm 3.0$ & $ 13.9 \pm 1.1$& $ 6142 \pm   5$ & $-28.63 \pm   0.02$ & $-28.57 \pm   0.01$& $-28.20 \pm   0.01$& $-27.99 \pm   0.01$& $-27.34 \pm   0.01$ & $ -0.07 \pm   0.17$\\
PTF12jqh & $151.9 \pm 1.5$ & $  7.9 \pm 0.7$& $ 6116 \pm  10$ & $-29.37 \pm   0.01$ & $-29.14 \pm   0.01$& $-28.71 \pm   0.01$& $-28.40 \pm   0.01$& $-27.64 \pm   0.01$ & $  0.69 \pm   0.19$\\
SNF20080802-006 & $108.2 \pm 6.0$ & $ 20.6 \pm 1.9$& $ 6122 \pm   5$ & $-29.02 \pm   0.06$ & $-28.80 \pm   0.06$& $-28.40 \pm   0.06$& $-28.20 \pm   0.06$& $-27.50 \pm   0.06$ & $ -1.56 \pm   0.13$\\
PTF10xyt & $123.7 \pm 6.6$ & $ 16.4 \pm 4.3$& $ 6101 \pm   4$ & $-28.26 \pm   0.02$ & $-28.20 \pm   0.02$& $-27.93 \pm   0.02$& $-27.74 \pm   0.02$& $-27.22 \pm   0.04$ & $  0.03 \pm   0.21$\\
PTF11qmo & $101.7 \pm 1.1$ & $  7.7 \pm 0.7$& $ 6150 \pm   8$ & $-29.77 \pm   0.02$ & $-29.43 \pm   0.02$& $-28.97 \pm   0.02$& $-28.64 \pm   0.02$& $-27.93 \pm   0.02$ & $ -0.00 \pm   0.34$\\
SNF20070331-025 & $119.8 \pm 7.4$ & $ 14.2 \pm 2.7$& $ 6120 \pm  10$ & $-28.94 \pm   0.02$ & $-28.75 \pm   0.02$& $-28.32 \pm   0.02$& $-28.07 \pm   0.02$& $-27.31 \pm   0.02$ & $ -1.28 \pm   0.16$\\
SNF20070818-001 & $157.5 \pm 7.5$ & $ 16.7 \pm 1.8$& $ 6115 \pm   5$ & $-28.97 \pm   0.02$ & $-28.96 \pm   0.01$& $-28.61 \pm   0.01$& $-28.37 \pm   0.01$& $-27.62 \pm   0.01$ & $  0.15 \pm   0.16$\\
SNBOSS38 & $ 57.1 \pm 0.4$ & $ 17.9 \pm 0.3$& $ 6127 \pm   3$ & $-29.20 \pm   0.01$ & $-28.84 \pm   0.01$& $-28.47 \pm   0.01$& $-28.23 \pm   0.01$& $-27.73 \pm   0.04$ & $  1.10 \pm   0.11$\\
SN2006ob & $ 90.0 \pm 16.5$ & $ 26.5 \pm 1.5$& $ 6112 \pm   5$ & $-29.11 \pm   0.02$ & $-28.82 \pm   0.01$& $-28.42 \pm   0.01$& $-28.19 \pm   0.01$& $-27.54 \pm   0.01$ & $ -2.45 \pm   0.28$\\
PTF12eer & $165.6 \pm 10.7$ & $ 12.7 \pm 2.8$& $ 6150 \pm  10$ & $-28.76 \pm   0.01$ & $-28.76 \pm   0.01$& $-28.40 \pm   0.01$& $-28.17 \pm   0.01$& $-27.45 \pm   0.02$ & $  0.68 \pm   0.16$\\
PTF10ops & $ 38.7 \pm 9.9$ & $  7.2 \pm 8.7$& $ 6140 \pm   5$ & $-27.93 \pm   0.38$ & $-27.76 \pm   0.38$& $-27.73 \pm   0.38$& $-27.59 \pm   0.38$& $-27.21 \pm   0.38$ & $ -0.62 \pm   0.13$\\
SNF20080514-002 & $ 83.2 \pm 0.7$ & $ 19.4 \pm 0.6$& $ 6131 \pm   3$ & $-29.30 \pm   0.01$ & $-28.95 \pm   0.01$& $-28.44 \pm   0.01$& $-28.17 \pm   0.01$& $-27.49 \pm   0.01$ & $ -1.57 \pm   0.11$\\
PTF12evo & $129.2 \pm 2.8$ & $  9.1 \pm 1.3$& $ 6156 \pm   4$ & $-29.14 \pm   0.02$ & $-28.98 \pm   0.01$& $-28.56 \pm   0.01$& $-28.28 \pm   0.01$& $-27.61 \pm   0.01$ & $  0.74 \pm   0.15$\\
SNF20080614-010 & $125.4 \pm 5.1$ & $ 26.9 \pm 1.6$& $ 6128 \pm   3$ & $-29.04 \pm   0.04$ & $-28.81 \pm   0.04$& $-28.38 \pm   0.04$& $-28.16 \pm   0.04$& $-27.57 \pm   0.04$ & $ -2.12 \pm   0.17$\\
PTF10icb & $104.8 \pm 0.9$ & $ 12.7 \pm 0.3$& $ 6138 \pm   3$ & $-28.58 \pm   0.02$ & $-28.36 \pm   0.02$& $-27.98 \pm   0.02$& $-27.77 \pm   0.02$& $-27.17 \pm   0.02$ & $  0.37 \pm   0.13$\\
PTF12efn & $144.9 \pm 3.4$ & $  7.1 \pm 1.8$& $ 6115 \pm   3$ & $-29.40 \pm   0.01$ & $-29.17 \pm   0.01$& $-28.79 \pm   0.01$& $-28.45 \pm   0.01$& $-27.64 \pm   0.01$ & $  1.21 \pm   0.20$\\
SNNGC4424 & $109.0 \pm 0.3$ & $  8.6 \pm 0.1$& $ 6138 \pm   2$ & $-28.35 \pm   0.01$ & $-28.15 \pm   0.01$& $-27.79 \pm   0.01$& $-27.58 \pm   0.01$& $-26.97 \pm   0.01$ & $  0.37 \pm   0.12$\\
SNF20080516-022 & $100.1 \pm 2.1$ & $ 13.7 \pm 1.1$& $ 6158 \pm   3$ & $-29.46 \pm   0.01$ & $-29.19 \pm   0.01$& $-28.71 \pm   0.01$& $-28.39 \pm   0.01$& $-27.77 \pm   0.01$ & $  0.20 \pm   0.14$\\
PTF12hwb & $ 21.1 \pm 78.0$ & $ -1.8 \pm 8.9$& $ 6090 \pm  14$ & $-28.32 \pm   0.02$ & $-28.24 \pm   0.02$& $-28.03 \pm   0.02$& $-27.79 \pm   0.02$& $-27.05 \pm   0.04$ & $  1.34 \pm   0.17$\\
PTF10qyz & $106.4 \pm 2.1$ & $ 23.0 \pm 1.0$& $ 6120 \pm   5$ & $-29.05 \pm   0.17$ & $-28.92 \pm   0.17$& $-28.41 \pm   0.17$& $-28.14 \pm   0.17$& $-27.30 \pm   0.17$ & $ -1.09 \pm   0.16$\\
SNF20060907-000 & $106.1 \pm 10.4$ & $ 17.0 \pm 0.9$& $ 6149 \pm   4$ & $-29.54 \pm   0.02$ & $-29.28 \pm   0.01$& $-28.76 \pm   0.01$& $-28.42 \pm   0.01$& $-27.74 \pm   0.04$ & $  0.17 \pm   0.17$\\
LSQ12fxd & $122.9 \pm 1.7$ & $ 11.4 \pm 0.8$& $ 6119 \pm   4$ & $-29.62 \pm   0.07$ & $-29.39 \pm   0.07$& $-28.95 \pm   0.07$& $-28.64 \pm   0.07$& $-27.91 \pm   0.07$ & $  0.05 \pm   0.15$\\
SNF20080821-000 & $105.1 \pm 2.2$ & $  8.6 \pm 1.3$& $ 6121 \pm   4$ & $-29.34 \pm   0.01$ & $-29.10 \pm   0.01$& $-28.73 \pm   0.01$& $-28.46 \pm   0.01$& $-27.82 \pm   0.01$ & $  0.70 \pm   0.19$\\
SNF20070802-000 & $158.3 \pm 3.3$ & $ 16.3 \pm 1.7$& $ 6102 \pm   5$ & $-28.90 \pm   0.01$ & $-28.81 \pm   0.01$& $-28.45 \pm   0.01$& $-28.22 \pm   0.01$& $-27.52 \pm   0.01$ & $  0.08 \pm   0.16$\\
PTF10wnm & $105.8 \pm 2.3$ & $  6.5 \pm 1.0$& $ 6124 \pm   3$ & $-29.38 \pm   0.01$ & $-29.07 \pm   0.01$& $-28.68 \pm   0.01$& $-28.37 \pm   0.01$& $-27.69 \pm   0.01$ & $  0.63 \pm   0.16$\\
PTF10mwb & $116.5 \pm 1.2$ & $ 19.8 \pm 0.9$& $ 6138 \pm   2$ & $-29.02 \pm   0.07$ & $-28.84 \pm   0.07$& $-28.40 \pm   0.07$& $-28.14 \pm   0.07$& $-27.52 \pm   0.07$ & $ -0.68 \pm   0.10$\\
SN2010dt & $116.2 \pm 14.9$ & $ 15.5 \pm 0.7$& $ 6138 \pm   6$ & $-29.30 \pm   0.01$ & $-29.15 \pm   0.01$& $-28.64 \pm   0.01$& $-28.35 \pm   0.01$& $-27.63 \pm   0.01$ & $ -0.03 \pm   0.10$\\
SNF20080623-001 & $149.1 \pm 1.4$ & $ 14.9 \pm 0.7$& $ 6131 \pm   3$ & $-29.11 \pm   0.01$ & $-28.97 \pm   0.01$& $-28.50 \pm   0.01$& $-28.22 \pm   0.01$& $-27.46 \pm   0.01$ & $  0.01 \pm   0.13$\\
LSQ12fhe & $ 42.8 \pm 1.2$ & $  4.0 \pm 3.1$& $ 6108 \pm   4$ & $-29.76 \pm   0.02$ & $-29.40 \pm   0.02$& $-29.04 \pm   0.02$& $-28.74 \pm   0.02$& $-28.11 \pm   0.02$ & $  1.07 \pm   0.19$\\
PTF11bju & $ 30.2 \pm 4.4$ & $  4.0 \pm 3.0$& $ 6139 \pm   5$ & $-29.47 \pm   0.02$ & $-29.10 \pm   0.01$& $-28.75 \pm   0.01$& $-28.45 \pm   0.01$& $-27.87 \pm   0.01$ & $  1.04 \pm   0.20$\\
PTF09fox & $117.6 \pm 2.7$ & $  9.1 \pm 1.0$& $ 6116 \pm   3$ & $-29.44 \pm   0.03$ & $-29.21 \pm   0.03$& $-28.72 \pm   0.03$& $-28.42 \pm   0.03$& $-27.68 \pm   0.03$ & $  0.68 \pm   0.27$\\
PTF13ayw & $104.6 \pm 2.4$ & $ 26.6 \pm 3.2$& $ 6115 \pm   6$ & $-29.16 \pm   0.02$ & $-28.82 \pm   0.02$& $-28.43 \pm   0.02$& $-28.20 \pm   0.02$& $-27.55 \pm   0.02$ & $ -1.72 \pm   0.29$\\
SNF20070810-004 & $126.7 \pm 1.8$ & $ 21.1 \pm 1.1$& $ 6118 \pm   7$ & $-29.22 \pm   0.01$ & $-29.10 \pm   0.01$& $-28.63 \pm   0.01$& $-28.34 \pm   0.01$& $-27.62 \pm   0.01$ & $  0.10 \pm   0.12$\\
PTF11mty & $111.4 \pm 2.3$ & $ 10.6 \pm 1.5$& $ 6138 \pm   5$ & $-29.54 \pm   0.01$ & $-29.23 \pm   0.01$& $-28.80 \pm   0.01$& $-28.46 \pm   0.01$& $-27.82 \pm   0.01$ & $  0.83 \pm   0.16$\\
SNF20080512-010 & $ 95.3 \pm 3.5$ & $ 23.3 \pm 1.5$& $ 6129 \pm   5$ & $-29.22 \pm   0.08$ & $-28.96 \pm   0.08$& $-28.50 \pm   0.08$& $-28.26 \pm   0.08$& $-27.56 \pm   0.08$ & $ -1.24 \pm   0.26$\\
PTF11mkx & $ 31.5 \pm 3.7$ & $  4.5 \pm 1.3$& $ 6169 \pm   5$ & $-29.50 \pm   0.45$ & $-29.25 \pm   0.45$& $-28.89 \pm   0.45$& $-28.61 \pm   0.45$& $-27.97 \pm   0.45$ & $  0.60 \pm   0.15$\\
PTF10tce & $135.7 \pm 1.1$ & $ 11.2 \pm 1.5$& $ 6090 \pm   4$ & $-29.13 \pm   0.02$ & $-28.99 \pm   0.01$& $-28.59 \pm   0.01$& $-28.31 \pm   0.01$& $-27.55 \pm   0.01$ & $  0.43 \pm   0.14$\\
SNF20061020-000 & $ 95.4 \pm 18.8$ & $ 24.1 \pm 1.0$& $ 6120 \pm   5$ & $-29.01 \pm   0.03$ & $-28.78 \pm   0.03$& $-28.35 \pm   0.03$& $-28.17 \pm   0.03$& $-27.54 \pm   0.03$ & $ -1.67 \pm   0.20$\\
SN2005ir & $115.6 \pm 2.8$ & $ 13.5 \pm 6.9$& $ 6069 \pm   5$ & $-29.33 \pm   0.02$ & $-29.12 \pm   0.02$& $-28.84 \pm   0.02$& $-28.49 \pm   0.02$& $-27.77 \pm   0.02$ & $  1.71 \pm   0.69$\\
SNF20080717-000 & $ 93.3 \pm 2.6$ & $  8.3 \pm 2.2$& $ 6104 \pm   3$ & $-28.58 \pm   0.01$ & $-28.47 \pm   0.01$& $-28.29 \pm   0.01$& $-28.05 \pm   0.01$& $-27.50 \pm   0.01$ & $  0.70 \pm   0.14$\\
PTF12ena & $101.1 \pm 1.6$ & $  7.4 \pm 1.0$& $ 6129 \pm   4$ & $-28.01 \pm   0.01$ & $-28.00 \pm   0.01$& $-27.85 \pm   0.01$& $-27.77 \pm   0.01$& $-27.31 \pm   0.01$ & $  0.65 \pm   0.16$\\
PTF13anh & $166.8 \pm 1.8$ & $ 21.8 \pm 1.2$& $ 6175 \pm   4$ & $-28.67 \pm   0.20$ & $-28.74 \pm   0.20$& $-28.30 \pm   0.20$& $-28.05 \pm   0.20$& $-27.28 \pm   0.20$ & $ -0.28 \pm   0.17$\\
CSS110918\_01 & $110.6 \pm 1.0$ & $  8.0 \pm 1.3$& $ 6101 \pm   2$ & $-29.88 \pm   0.76$ & $-29.58 \pm   0.76$& $-29.09 \pm   0.76$& $-28.71 \pm   0.76$& $-27.91 \pm   0.76$ & $  0.92 \pm   0.17$\\
SNF20061024-000 & $ 86.9 \pm 26.8$ & $ 30.0 \pm 1.5$& $ 6127 \pm   5$ & $-28.88 \pm   0.04$ & $-28.70 \pm   0.04$& $-28.26 \pm   0.04$& $-28.05 \pm   0.04$& $-27.40 \pm   0.04$ & $ -1.39 \pm   0.19$\\
SNF20070506-006 & $ 94.1 \pm 1.3$ & $  6.7 \pm 0.6$& $ 6153 \pm   3$ & $-29.72 \pm   0.01$ & $-29.39 \pm   0.01$& $-28.97 \pm   0.01$& $-28.64 \pm   0.01$& $-27.96 \pm   0.01$ & $  1.17 \pm   0.13$\\
SNF20070403-001 & $105.9 \pm 5.4$ & $ 18.3 \pm 1.8$& $ 6124 \pm   4$ & $-29.23 \pm   0.02$ & $-29.04 \pm   0.01$& $-28.63 \pm   0.01$& $-28.35 \pm   0.01$& $-27.62 \pm   0.01$ & $ -0.54 \pm   0.13$\\
PTF10hmv & $109.6 \pm 1.3$ & $  8.9 \pm 0.7$& $ 6143 \pm   3$ & $-28.54 \pm   0.01$ & $-28.40 \pm   0.01$& $-28.11 \pm   0.01$& $-27.89 \pm   0.01$& $-27.31 \pm   0.01$ & $  1.39 \pm   0.18$\\
SNF20071015-000 & $105.0 \pm 3.2$ & $  6.9 \pm 1.1$& $ 6124 \pm   7$ & $-27.89 \pm   0.02$ & $-27.82 \pm   0.02$& $-27.69 \pm   0.02$& $-27.63 \pm   0.02$& $-27.16 \pm   0.04$ & $  1.36 \pm   0.37$\\
SNhunt89 & $ 88.0 \pm 2.7$ & $ 32.2 \pm 1.9$& $ 6111 \pm   7$ & $-28.37 \pm   0.03$ & $-28.26 \pm   0.03$& $-27.92 \pm   0.03$& $-27.77 \pm   0.03$& $-27.13 \pm   0.03$ & $ -1.83 \pm   0.18$\\
SNF20070902-021 & $108.9 \pm 3.5$ & $ 17.1 \pm 1.0$& $ 6131 \pm   6$ & $-29.25 \pm   0.02$ & $-29.02 \pm   0.02$& $-28.56 \pm   0.02$& $-28.32 \pm   0.01$& $-27.65 \pm   0.02$ & $ -0.43 \pm   0.10$\\
PTF09dlc & $143.5 \pm 2.2$ & $ 10.2 \pm 0.9$& $ 6143 \pm   3$ & $-29.38 \pm   0.01$ & $-29.17 \pm   0.01$& $-28.69 \pm   0.01$& $-28.40 \pm   0.01$& $-27.62 \pm   0.01$ & $  0.11 \pm   0.12$\\
PTF13ajv & $150.5 \pm 8.9$ & $ 46.3 \pm 8.6$& $ 6110 \pm  21$ & $-28.70 \pm   0.02$ & $-28.61 \pm   0.02$& $-28.16 \pm   0.02$& $-27.91 \pm   0.02$& $-27.07 \pm   0.04$ & $ -1.59 \pm   0.22$\\
SNF20080919-000 & $114.7 \pm 2.8$ & $  9.4 \pm 0.9$& $ 6145 \pm   5$ & $-28.53 \pm   0.02$ & $-28.41 \pm   0.01$& $-28.11 \pm   0.01$& $-27.99 \pm   0.01$& $-27.38 \pm   0.01$ & $ -0.13 \pm   0.11$\\
SNF20080919-001 & $ 85.0 \pm 1.1$ & $  6.0 \pm 0.4$& $ 6150 \pm   5$ & $-29.73 \pm   0.01$ & $-29.43 \pm   0.01$& $-29.04 \pm   0.01$& $-28.72 \pm   0.01$& $-28.07 \pm   0.01$ & $  0.84 \pm   0.18$\\
SN2010kg & $ 95.1 \pm 28.5$ & $ 21.7 \pm 0.7$& $ 6077 \pm   5$ & $-28.85 \pm   0.01$ & $-28.74 \pm   0.01$& $-28.41 \pm   0.01$& $-28.20 \pm   0.01$& $-27.47 \pm   0.01$ & $ -2.26 \pm   0.40$\\
SNF20080714-008 & $134.8 \pm 15.7$ & $ 19.7 \pm 3.7$& $ 6100 \pm   6$ & $-28.56 \pm   0.02$ & $-28.63 \pm   0.01$& $-28.32 \pm   0.01$& $-28.13 \pm   0.01$& $-27.42 \pm   0.01$ & $  0.01 \pm   0.15$\\
SNF20070714-007 & $129.6 \pm 5.6$ & $ 31.1 \pm 23.8$& $ 6146 \pm   5$ & $-27.88 \pm   0.02$ & $-28.12 \pm   0.01$& $-28.02 \pm   0.01$& $-27.86 \pm   0.01$& $-27.24 \pm   0.03$ & $ -1.10 \pm   0.13$\\
SNF20080522-011 & $122.1 \pm 1.7$ & $  8.3 \pm 0.5$& $ 6125 \pm   3$ & $-29.63 \pm   0.01$ & $-29.38 \pm   0.01$& $-28.92 \pm   0.01$& $-28.60 \pm   0.01$& $-27.88 \pm   0.01$ & $  0.81 \pm   0.19$\\
SNF20061111-002 & $110.8 \pm 10.7$ & $ 20.4 \pm 1.0$& $ 6145 \pm   6$ & $-29.16 \pm   0.01$ & $-28.99 \pm   0.01$& $-28.59 \pm   0.01$& $-28.29 \pm   0.01$& $-27.61 \pm   0.01$ & $  0.27 \pm   0.21$\\
SNNGC6343 & $ 87.0 \pm 1.4$ & $ 20.7 \pm 0.7$& $ 6136 \pm   3$ & $-28.78 \pm   0.01$ & $-28.66 \pm   0.01$& $-28.30 \pm   0.01$& $-28.08 \pm   0.01$& $-27.41 \pm   0.01$ & $ -2.09 \pm   0.25$\\
SNF20061011-005 & $120.6 \pm 1.1$ & $  9.3 \pm 0.4$& $ 6132 \pm   4$ & $-29.72 \pm   0.04$ & $-29.43 \pm   0.03$& $-28.99 \pm   0.03$& $-28.64 \pm   0.03$& $-27.90 \pm   0.03$ & $  0.74 \pm   0.29$\\
SNF20080825-010 & $102.4 \pm 13.4$ & $ 19.2 \pm 0.6$& $ 6116 \pm   4$ & $-29.46 \pm   0.01$ & $-29.17 \pm   0.01$& $-28.71 \pm   0.01$& $-28.47 \pm   0.01$& $-27.83 \pm   0.01$ & $ -0.77 \pm   0.18$\\
PTF10ufj & $141.1 \pm 3.4$ & $ 11.7 \pm 1.2$& $ 6131 \pm   6$ & $-29.28 \pm   0.15$ & $-29.16 \pm   0.15$& $-28.72 \pm   0.15$& $-28.41 \pm   0.15$& $-27.65 \pm   0.15$ & $  0.53 \pm   0.16$\\
PTF10wof & $129.6 \pm 2.7$ & $ 17.3 \pm 1.0$& $ 6102 \pm   2$ & $-28.91 \pm   0.01$ & $-28.84 \pm   0.01$& $-28.46 \pm   0.01$& $-28.18 \pm   0.01$& $-27.43 \pm   0.01$ & $  0.16 \pm   0.13$\\
SNF20080918-000 & $146.8 \pm 3.5$ & $  7.5 \pm 2.5$& $ 6110 \pm   5$ & $-28.79 \pm   0.02$ & $-28.65 \pm   0.02$& $-28.35 \pm   0.02$& $-28.12 \pm   0.02$& $-27.46 \pm   0.02$ & $  1.50 \pm   0.30$\\
SNF20080516-000 & $117.4 \pm 2.2$ & $  9.0 \pm 1.2$& $ 6135 \pm   3$ & $-29.50 \pm   0.01$ & $-29.23 \pm   0.01$& $-28.80 \pm   0.01$& $-28.47 \pm   0.01$& $-27.74 \pm   0.01$ & $  1.24 \pm   0.26$\\
SN2005cf & $159.1 \pm 0.7$ & $ 15.7 \pm 0.8$& $ 6141 \pm   3$ & $-29.37 \pm   0.02$ & $-29.16 \pm   0.02$& $-28.68 \pm   0.02$& $-28.41 \pm   0.02$& $-27.69 \pm   0.02$ & $ -0.31 \pm   0.17$\\
CSS130502\_01 & $ 91.5 \pm 10.9$ & $ 15.6 \pm 0.5$& $ 6128 \pm   3$ & $-29.43 \pm   0.02$ & $-29.09 \pm   0.02$& $-28.60 \pm   0.01$& $-28.30 \pm   0.01$& $-27.62 \pm   0.04$ & $ -0.00 \pm   0.24$\\
SNF20080620-000 & $107.8 \pm 14.1$ & $ 20.0 \pm 0.7$& $ 6132 \pm   3$ & $-28.82 \pm   0.02$ & $-28.78 \pm   0.01$& $-28.32 \pm   0.01$& $-28.09 \pm   0.01$& $-27.39 \pm   0.01$ & $ -1.04 \pm   0.16$\\
SNPGC51271 & $ 92.1 \pm 16.5$ & $ 21.1 \pm 0.7$& $ 6121 \pm   2$ & $-29.28 \pm   0.02$ & $-28.95 \pm   0.02$& $-28.46 \pm   0.02$& $-28.20 \pm   0.02$& $-27.62 \pm   0.04$ & $ -2.08 \pm   0.18$\\
PTF11pdk & $128.6 \pm 2.8$ & $ 15.6 \pm 1.7$& $ 6153 \pm   5$ & $-29.35 \pm   0.02$ & $-29.11 \pm   0.02$& $-28.61 \pm   0.02$& $-28.32 \pm   0.02$& $-27.67 \pm   0.02$ & $  0.06 \pm   0.19$\\
SNF20060511-014 & $102.6 \pm 2.8$ & $ 15.6 \pm 1.1$& $ 6141 \pm   8$ & $-29.16 \pm   0.07$ & $-29.04 \pm   0.06$& $-28.56 \pm   0.06$& $-28.30 \pm   0.06$& $-27.63 \pm   0.06$ & $ -0.68 \pm   0.18$\\
SNF20080612-003 & $120.0 \pm 1.1$ & $  7.3 \pm 0.6$& $ 6123 \pm   3$ & $-29.64 \pm   0.02$ & $-29.41 \pm   0.02$& $-28.99 \pm   0.02$& $-28.70 \pm   0.02$& $-28.00 \pm   0.02$ & $  0.71 \pm   0.21$\\
SNF20080626-002 & $130.0 \pm 1.0$ & $  6.1 \pm 4.2$& $ 6111 \pm   3$ & $-29.42 \pm   0.01$ & $-29.24 \pm   0.01$& $-28.84 \pm   0.01$& $-28.52 \pm   0.01$& $-27.76 \pm   0.01$ & $  1.04 \pm   0.23$\\
SNF20060621-015 & $111.9 \pm 1.3$ & $  9.8 \pm 0.7$& $ 6144 \pm   3$ & $-29.63 \pm   0.01$ & $-29.36 \pm   0.01$& $-28.88 \pm   0.01$& $-28.54 \pm   0.01$& $-27.81 \pm   0.01$ & $  0.43 \pm   0.20$\\
SNF20080920-000 & $135.2 \pm 1.4$ & $  5.6 \pm 1.6$& $ 6085 \pm   3$ & $-29.44 \pm   0.02$ & $-29.19 \pm   0.02$& $-28.79 \pm   0.02$& $-28.49 \pm   0.02$& $-27.74 \pm   0.02$ & $  1.17 \pm   0.34$\\
SN2007cq & $ 65.8 \pm 4.1$ & $ 10.2 \pm 0.9$& $ 6137 \pm   3$ & $-29.53 \pm   0.02$ & $-29.30 \pm   0.02$& $-28.89 \pm   0.02$& $-28.56 \pm   0.02$& $-27.90 \pm   0.02$ & $ -0.43 \pm   0.16$\\
SNF20080918-004 & $ 87.8 \pm 7.2$ & $ 21.5 \pm 0.9$& $ 6141 \pm   4$ & $-29.00 \pm   0.22$ & $-28.82 \pm   0.22$& $-28.37 \pm   0.22$& $-28.13 \pm   0.22$& $-27.43 \pm   0.22$ & $ -1.27 \pm   0.18$\\
CSS120424\_01 & $138.1 \pm 2.1$ & $ 11.7 \pm 0.7$& $ 6138 \pm   3$ & $-29.40 \pm   0.02$ & $-29.23 \pm   0.02$& $-28.77 \pm   0.01$& $-28.45 \pm   0.02$& $-27.68 \pm   0.02$ & $  0.76 \pm   0.19$\\
SNF20080610-000 & $119.9 \pm 10.4$ & $ 16.4 \pm 1.7$& $ 6131 \pm   6$ & $-29.05 \pm   0.07$ & $-28.92 \pm   0.07$& $-28.50 \pm   0.07$& $-28.22 \pm   0.07$& $-27.55 \pm   0.07$ & $  0.12 \pm   0.16$\\
SNF20070701-005 & $101.8 \pm 2.6$ & $ 12.4 \pm 1.0$& $ 6158 \pm   5$ & $-29.46 \pm   0.02$ & $-29.27 \pm   0.02$& $-28.87 \pm   0.02$& $-28.60 \pm   0.02$& $-27.96 \pm   0.02$ & $  0.06 \pm   0.17$\\
SN2007kk & $128.5 \pm 1.4$ & $ 10.6 \pm 1.0$& $ 6098 \pm   4$ & $-29.48 \pm   0.02$ & $-29.31 \pm   0.02$& $-28.87 \pm   0.01$& $-28.54 \pm   0.01$& $-27.77 \pm   0.02$ & $  0.52 \pm   0.18$\\
SNF20060908-004 & $114.4 \pm 1.2$ & $ 12.6 \pm 0.6$& $ 6136 \pm   3$ & $-29.59 \pm   0.23$ & $-29.34 \pm   0.23$& $-28.91 \pm   0.23$& $-28.58 \pm   0.23$& $-27.87 \pm   0.23$ & $  0.70 \pm   0.19$\\
SNF20080909-030 & $ 93.7 \pm 1.0$ & $  7.8 \pm 0.4$& $ 6171 \pm   3$ & $-29.38 \pm   0.02$ & $-29.12 \pm   0.01$& $-28.74 \pm   0.01$& $-28.44 \pm   0.01$& $-27.78 \pm   0.01$ & $  1.17 \pm   0.16$\\
PTF11bgv & $ 79.4 \pm 3.2$ & $ 12.6 \pm 0.7$& $ 6146 \pm   3$ & $-28.90 \pm   0.02$ & $-28.62 \pm   0.01$& $-28.27 \pm   0.01$& $-28.08 \pm   0.01$& $-27.54 \pm   0.01$ & $ -0.33 \pm   0.12$\\
SNNGC2691 & $ 39.0 \pm 22.2$ & $  4.5 \pm 0.2$& $ 6139 \pm   8$ & $-29.46 \pm   0.02$ & $-29.06 \pm   0.02$& $-28.75 \pm   0.02$& $-28.49 \pm   0.02$& $-27.93 \pm   0.02$ & $  0.98 \pm   0.18$\\
PTF13asv & $ 75.6 \pm 1.1$ & $  2.2 \pm 0.4$& $ 6148 \pm   4$ & $-29.92 \pm   0.32$ & $-29.49 \pm   0.32$& $-29.02 \pm   0.32$& $-28.63 \pm   0.32$& $-27.90 \pm   0.32$ & $  0.50 \pm   0.11$\\
SNF20070806-026 & $ 98.8 \pm 12.1$ & $ 25.9 \pm 0.7$& $ 6114 \pm   7$ & $-29.14 \pm   0.02$ & $-28.91 \pm   0.02$& $-28.44 \pm   0.02$& $-28.21 \pm   0.02$& $-27.49 \pm   0.02$ & $ -1.73 \pm   0.17$\\
SNF20070427-001 & $ 81.3 \pm 2.3$ & $  6.3 \pm 0.9$& $ 6142 \pm   5$ & $-29.89 \pm   0.02$ & $-29.46 \pm   0.02$& $-28.97 \pm   0.02$& $-28.62 \pm   0.02$& $-27.97 \pm   0.02$ & $  0.62 \pm   0.21$\\
SNF20061108-004 & $129.5 \pm 5.6$ & $  6.3 \pm 2.5$& $ 6110 \pm   6$ & $-29.53 \pm   0.02$ & $-29.31 \pm   0.02$& $-28.95 \pm   0.02$& $-28.60 \pm   0.02$& $-27.96 \pm   0.02$ & $  2.14 \pm   0.42$\\
SNF20060912-000 & $106.5 \pm 1.8$ & $ 21.4 \pm 1.7$& $ 6163 \pm   7$ & $-28.98 \pm   0.02$ & $-28.92 \pm   0.02$& $-28.66 \pm   0.02$& $-28.42 \pm   0.02$& $-27.77 \pm   0.02$ & $  0.24 \pm   0.20$\\
CSS110918\_02 & $109.1 \pm 9.4$ & $ 15.0 \pm 0.6$& $ 6137 \pm   3$ & $-29.36 \pm   0.02$ & $-29.14 \pm   0.01$& $-28.69 \pm   0.01$& $-28.41 \pm   0.01$& $-27.70 \pm   0.01$ & $ -0.27 \pm   0.17$\\
SNF20080918-002 & $ 97.7 \pm 2.8$ & $ 12.6 \pm 1.4$& $ 6141 \pm   6$ & $-29.50 \pm   0.02$ & $-29.11 \pm   0.02$& $-28.61 \pm   0.02$& $-28.34 \pm   0.02$& $-27.71 \pm   0.02$ & $ -0.50 \pm   0.09$\\
SNIC3573 & $102.7 \pm 1.8$ & $ 11.9 \pm 1.0$& $ 6142 \pm   5$ & $-29.28 \pm   0.02$ & $-29.14 \pm   0.02$& $-28.74 \pm   0.02$& $-28.46 \pm   0.01$& $-27.76 \pm   0.03$ & $  0.35 \pm   0.19$\\
SNF20080725-004 & $133.6 \pm 2.1$ & $  6.9 \pm 0.9$& $ 6131 \pm   6$ & $-29.09 \pm   0.01$ & $-28.93 \pm   0.01$& $-28.59 \pm   0.01$& $-28.31 \pm   0.01$& $-27.55 \pm   0.03$ & $  0.60 \pm   0.14$\\
SNF20050728-006 & $127.8 \pm 2.5$ & $ 15.8 \pm 1.3$& $ 6124 \pm   6$ & $-28.80 \pm   0.02$ & $-28.68 \pm   0.02$& $-28.37 \pm   0.02$& $-28.18 \pm   0.02$& $-27.55 \pm   0.02$ & $  0.27 \pm   0.27$\\
SN2012fr & $134.2 \pm 0.5$ & $  7.4 \pm 0.2$& $ 6102 \pm   1$ & $-29.91 \pm   0.01$ & $-29.70 \pm   0.01$& $-29.31 \pm   0.01$& $-28.94 \pm   0.01$& $-28.10 \pm   0.01$ & $  0.96 \pm   0.10$\\
SNF20060512-002 & $100.2 \pm 2.8$ & $ 13.4 \pm 1.1$& $ 6107 \pm   8$ & $-29.33 \pm   0.02$ & $-29.11 \pm   0.02$& $-28.77 \pm   0.02$& $-28.52 \pm   0.02$& $-27.80 \pm   0.02$ & $ -0.92 \pm   0.19$\\
SNF20060512-001 & $ 88.4 \pm 1.2$ & $  5.4 \pm 0.4$& $ 6169 \pm   3$ & $-29.33 \pm   0.01$ & $-29.05 \pm   0.01$& $-28.68 \pm   0.01$& $-28.40 \pm   0.01$& $-27.79 \pm   0.01$ & $  0.69 \pm   0.15$\\
SNF20071003-016 & $125.2 \pm 4.6$ & $ 17.1 \pm 2.0$& $ 6124 \pm  11$ & $-28.58 \pm   0.02$ & $-28.54 \pm   0.02$& $-28.19 \pm   0.02$& $-27.99 \pm   0.02$& $-27.31 \pm   0.02$ & $ -0.25 \pm   0.15$\\
SNF20050821-007 & $141.7 \pm 2.6$ & $  7.7 \pm 1.0$& $ 6140 \pm   9$ & $-29.38 \pm   0.02$ & $-29.20 \pm   0.02$& $-28.77 \pm   0.02$& $-28.46 \pm   0.02$& $-27.67 \pm   0.02$ & $  0.17 \pm   0.32$\\
SNF20070803-005 & $ 22.7 \pm 21.4$ & $  0.9 \pm 0.6$& $ 6157 \pm  27$ & $-29.87 \pm   0.01$ & $-29.43 \pm   0.01$& $-29.04 \pm   0.01$& $-28.74 \pm   0.01$& $-28.11 \pm   0.01$ & $  0.88 \pm   0.18$\\
PTF09foz & $127.2 \pm 1.9$ & $ 21.7 \pm 1.2$& $ 6136 \pm   4$ & $-29.14 \pm   0.01$ & $-29.00 \pm   0.01$& $-28.59 \pm   0.01$& $-28.35 \pm   0.01$& $-27.65 \pm   0.01$ & $ -1.14 \pm   0.26$\\
PTF12grk & $162.3 \pm 9.8$ & $ 19.6 \pm 1.4$& $ 6085 \pm   8$ & $-28.86 \pm   0.02$ & $-28.87 \pm   0.01$& $-28.42 \pm   0.01$& $-28.19 \pm   0.01$& $-27.50 \pm   0.03$ & $ -1.17 \pm   0.13$\\
SNF20080720-001 & $138.5 \pm 4.0$ & $ 14.0 \pm 2.0$& $ 6107 \pm   3$ & $-27.59 \pm   0.02$ & $-27.78 \pm   0.01$& $-27.73 \pm   0.01$& $-27.71 \pm   0.01$& $-27.19 \pm   0.02$ & $  0.18 \pm   0.13$\\
SNF20080810-001 & $ 88.4 \pm 21.6$ & $ 22.3 \pm 1.1$& $ 6145 \pm   5$ & $-29.11 \pm   0.01$ & $-28.89 \pm   0.01$& $-28.45 \pm   0.01$& $-28.23 \pm   0.01$& $-27.60 \pm   0.01$ & $ -0.87 \pm   0.11$\\
SNF20050729-002 & $109.4 \pm 2.2$ & $ 11.5 \pm 1.7$& $ 6142 \pm   6$ & $-29.35 \pm   0.13$ & $-29.17 \pm   0.13$& $-28.68 \pm   0.13$& $-28.38 \pm   0.13$& $-27.56 \pm   0.13$ & $  0.57 \pm   0.34$\\
SN2008ec & $103.7 \pm 17.0$ & $ 23.1 \pm 0.4$& $ 6125 \pm   3$ & $-28.67 \pm   0.01$ & $-28.52 \pm   0.01$& $-28.18 \pm   0.01$& $-28.03 \pm   0.01$& $-27.47 \pm   0.01$ & $ -1.05 \pm   0.12$\\
SNF20070902-018 & $ 93.8 \pm 12.2$ & $ 23.8 \pm 3.0$& $ 6120 \pm   8$ & $-28.87 \pm   0.02$ & $-28.70 \pm   0.01$& $-28.26 \pm   0.01$& $-28.08 \pm   0.01$& $-27.41 \pm   0.02$ & $ -0.72 \pm   0.10$\\
SNF20070424-003 & $122.5 \pm 3.8$ & $ 12.7 \pm 1.6$& $ 6132 \pm   6$ & $-29.10 \pm   0.01$ & $-28.96 \pm   0.01$& $-28.51 \pm   0.01$& $-28.25 \pm   0.01$& $-27.57 \pm   0.01$ & $  0.31 \pm   0.11$\\
SN2006cj & $101.7 \pm 1.3$ & $  4.8 \pm 0.8$& $ 6127 \pm   3$ & $-29.43 \pm   0.01$ & $-29.14 \pm   0.01$& $-28.74 \pm   0.01$& $-28.43 \pm   0.01$& $-27.76 \pm   0.01$ & $  0.48 \pm   0.18$\\
SN2007nq & $ 89.8 \pm 9.9$ & $ 23.4 \pm 1.1$& $ 6109 \pm   5$ & $-29.11 \pm   0.02$ & $-28.91 \pm   0.02$& $-28.50 \pm   0.02$& $-28.27 \pm   0.02$& $-27.57 \pm   0.02$ & $ -1.89 \pm   0.26$\\
SNF20070817-003 & $ 93.9 \pm 2.4$ & $ 18.5 \pm 1.3$& $ 6116 \pm   6$ & $-29.19 \pm   0.02$ & $-29.03 \pm   0.01$& $-28.59 \pm   0.01$& $-28.30 \pm   0.01$& $-27.55 \pm   0.02$ & $ -0.57 \pm   0.13$\\
SNF20070403-000 & $ 61.8 \pm 6.5$ & $ 27.1 \pm 1.8$& $ 6154 \pm   8$ & $-28.37 \pm   0.02$ & $-28.27 \pm   0.02$& $-27.97 \pm   0.02$& $-27.80 \pm   0.02$& $-27.24 \pm   0.02$ & $ -1.96 \pm   0.19$\\
SNF20061022-005 & $ 64.6 \pm 3.8$ & $  3.7 \pm 1.4$& $ 6146 \pm   7$ & $-29.49 \pm   0.02$ & $-29.06 \pm   0.02$& $-28.71 \pm   0.02$& $-28.42 \pm   0.02$& $-27.93 \pm   0.02$ & $  1.72 \pm   0.44$\\
SNNGC4076 & $127.3 \pm 2.4$ & $ 15.5 \pm 1.2$& $ 6152 \pm   4$ & $-28.77 \pm   0.01$ & $-28.66 \pm   0.01$& $-28.37 \pm   0.01$& $-28.15 \pm   0.01$& $-27.52 \pm   0.01$ & $  0.01 \pm   0.12$\\
SNF20070727-016 & $ 77.5 \pm 2.5$ & $  5.1 \pm 0.8$& $ 6140 \pm   4$ & $-29.96 \pm   0.06$ & $-29.56 \pm   0.06$& $-29.06 \pm   0.06$& $-28.75 \pm   0.06$& $-28.01 \pm   0.06$ & $  0.27 \pm   0.12$\\
PTF12fuu & $105.5 \pm 3.0$ & $  6.2 \pm 1.2$& $ 6124 \pm   5$ & $-29.54 \pm   0.01$ & $-29.23 \pm   0.01$& $-28.74 \pm   0.01$& $-28.40 \pm   0.01$& $-27.64 \pm   0.01$ & $  0.34 \pm   0.12$\\
SNF20070820-000 & $107.2 \pm 3.5$ & $ 18.6 \pm 1.3$& $ 6132 \pm  14$ & $-28.80 \pm   0.02$ & $-28.69 \pm   0.02$& $-28.34 \pm   0.02$& $-28.13 \pm   0.02$& $-27.52 \pm   0.02$ & $ -0.29 \pm   0.13$\\
SNF20070725-001 & $108.4 \pm 2.0$ & $ 11.1 \pm 1.5$& $ 6140 \pm   7$ & $-29.61 \pm   0.02$ & $-29.32 \pm   0.02$& $-28.84 \pm   0.02$& $-28.50 \pm   0.02$& $-27.76 \pm   0.02$ & $  0.68 \pm   0.17$\\
SNF20071108-021 & $ 99.1 \pm 2.7$ & $  5.8 \pm 0.8$& $ 6164 \pm   5$ & $-29.67 \pm   0.01$ & $-29.34 \pm   0.01$& $-28.94 \pm   0.01$& $-28.60 \pm   0.01$& $-27.96 \pm   0.01$ & $  1.00 \pm   0.21$\\
SNF20080914-001 & $126.5 \pm 1.2$ & $ 15.4 \pm 1.1$& $ 6159 \pm   3$ & $-28.67 \pm   0.02$ & $-28.60 \pm   0.02$& $-28.31 \pm   0.02$& $-28.13 \pm   0.02$& $-27.58 \pm   0.02$ & $ -0.11 \pm   0.17$\\
SNF20060609-002 & $ 87.7 \pm 3.6$ & $  7.3 \pm 1.3$& $ 6132 \pm   4$ & $-28.60 \pm   0.02$ & $-28.42 \pm   0.02$& $-28.19 \pm   0.02$& $-28.05 \pm   0.02$& $-27.53 \pm   0.02$ & $  0.07 \pm   0.15$\\
SNF20050624-000 & $121.0 \pm 5.3$ & $  9.3 \pm 3.1$& $ 6126 \pm   6$ & $-29.75 \pm   0.01$ & $-29.42 \pm   0.01$& $-28.99 \pm   0.01$& $-28.68 \pm   0.01$& $-27.97 \pm   0.01$ & $  0.69 \pm   0.62$\\
SNF20060618-023 & $ 74.9 \pm 4.9$ & $  5.0 \pm 1.8$& $ 6137 \pm  21$ & $-29.61 \pm   0.02$ & $-29.18 \pm   0.02$& $-28.89 \pm   0.02$& $-28.66 \pm   0.02$& $-28.08 \pm   0.02$ & $  0.12 \pm   0.18$\\
SNF20080531-000 & $133.0 \pm 1.5$ & $ 17.6 \pm 0.8$& $ 6114 \pm   5$ & $-29.12 \pm   0.01$ & $-28.98 \pm   0.01$& $-28.54 \pm   0.01$& $-28.28 \pm   0.01$& $-27.51 \pm   0.01$ & $ -0.49 \pm   0.18$\\
SN2006do & $106.4 \pm 2.1$ & $ 26.7 \pm 1.3$& $ 6101 \pm   2$ & $-29.00 \pm   0.01$ & $-28.83 \pm   0.01$& $-28.42 \pm   0.01$& $-28.20 \pm   0.01$& $-27.53 \pm   0.04$ & $ -2.32 \pm   0.54$\\
PTF12ikt & $110.3 \pm 1.6$ & $ 14.2 \pm 0.7$& $ 6141 \pm   4$ & $-29.34 \pm   0.01$ & $-29.04 \pm   0.01$& $-28.57 \pm   0.01$& $-28.32 \pm   0.01$& $-27.66 \pm   0.01$ & $ -0.35 \pm   0.11$\\
SN2006dm & $ 99.5 \pm 1.6$ & $ 30.0 \pm 0.7$& $ 6118 \pm   3$ & $-28.81 \pm   0.01$ & $-28.65 \pm   0.01$& $-28.23 \pm   0.01$& $-28.02 \pm   0.01$& $-27.33 \pm   0.01$ & $ -1.88 \pm   0.34$\\
PTF13azs & $138.0 \pm 5.1$ & $ 16.2 \pm 1.6$& $ 6125 \pm  10$ & $-27.84 \pm   0.02$ & $-27.92 \pm   0.02$& $-27.69 \pm   0.02$& $-27.60 \pm   0.02$& $-26.99 \pm   0.02$ & $ -0.02 \pm   0.12$\\
SN2005hj & $ 80.8 \pm 2.4$ & $  4.3 \pm 0.8$& $ 6138 \pm   4$ & $-29.54 \pm   0.02$ & $-29.16 \pm   0.01$& $-28.87 \pm   0.01$& $-28.54 \pm   0.01$& $-28.01 \pm   0.01$ & $  1.85 \pm   0.59$\\
PTF12iiq & $150.4 \pm 2.2$ & $ 22.5 \pm 0.8$& $ 6041 \pm   6$ & $-28.60 \pm   0.01$ & $-28.77 \pm   0.01$& $-28.41 \pm   0.01$& $-28.10 \pm   0.01$& $-27.29 \pm   0.01$ & $ -1.30 \pm   0.12$\\
PTF10ndc & $124.2 \pm 2.4$ & $  6.8 \pm 1.1$& $ 6119 \pm   3$ & $-29.52 \pm   0.01$ & $-29.25 \pm   0.01$& $-28.80 \pm   0.01$& $-28.49 \pm   0.01$& $-27.76 \pm   0.01$ & $  0.87 \pm   0.18$\\
SNF20080919-002 & $103.6 \pm 7.2$ & $ 27.2 \pm 1.9$& $ 6133 \pm   8$ & $-28.74 \pm   0.02$ & $-28.46 \pm   0.01$& $-28.09 \pm   0.01$& $-27.87 \pm   0.01$& $-27.26 \pm   0.04$ & $ -1.52 \pm   0.14$\\
SNPGC027923 & $ 85.5 \pm 0.6$ & $  5.9 \pm 0.3$& $ 6130 \pm   4$ & $-29.87 \pm   0.02$ & $-29.45 \pm   0.02$& $-28.94 \pm   0.02$& $-28.57 \pm   0.02$& $-27.85 \pm   0.02$ & $  0.22 \pm   0.26$\\
SNF20070330-024 & $118.1 \pm 2.1$ & $  4.6 \pm 2.2$& $ 6101 \pm   3$ & $-29.77 \pm   0.02$ & $-29.52 \pm   0.02$& $-29.08 \pm   0.02$& $-28.74 \pm   0.01$& $-27.94 \pm   0.02$ & $  0.34 \pm   0.29$\\
SNF20061030-010 & $131.4 \pm 2.2$ & $ 17.4 \pm 1.1$& $ 6116 \pm   4$ & $-28.60 \pm   0.02$ & $-28.55 \pm   0.02$& $-28.25 \pm   0.02$& $-28.03 \pm   0.02$& $-27.34 \pm   0.02$ & $  0.14 \pm   0.25$\\
SNhunt46 & $ 94.1 \pm 2.0$ & $ 11.2 \pm 0.6$& $ 6132 \pm   4$ & $-29.50 \pm   0.02$ & $-29.11 \pm   0.02$& $-28.67 \pm   0.02$& $-28.37 \pm   0.02$& $-27.71 \pm   0.02$ & $  0.20 \pm   0.24$\\
SN2005hc & $126.9 \pm 2.5$ & $ 10.0 \pm 0.7$& $ 6123 \pm   3$ & $-29.38 \pm   0.01$ & $-29.13 \pm   0.01$& $-28.69 \pm   0.01$& $-28.37 \pm   0.01$& $-27.61 \pm   0.01$ & $  0.87 \pm   0.31$\\
LSQ12dbr & $106.9 \pm 0.6$ & $  7.1 \pm 0.7$& $ 6138 \pm   4$ & $-29.29 \pm   0.73$ & $-29.00 \pm   0.73$& $-28.51 \pm   0.73$& $-28.15 \pm   0.73$& $-27.38 \pm   0.73$ & $  1.12 \pm   0.14$\\
LSQ12hjm & $ 82.6 \pm 17.5$ & $ 12.2 \pm 1.4$& $ 6144 \pm   5$ & $-29.51 \pm   0.02$ & $-29.14 \pm   0.01$& $-28.60 \pm   0.01$& $-28.30 \pm   0.01$& $-27.71 \pm   0.02$ & $ -0.50 \pm   0.21$\\
SNF20060521-001 & $ 78.9 \pm 20.2$ & $ 21.1 \pm 1.4$& $ 6123 \pm  10$ & $-29.37 \pm   0.05$ & $-29.04 \pm   0.05$& $-28.54 \pm   0.05$& $-28.30 \pm   0.05$& $-27.57 \pm   0.05$ & $ -1.22 \pm   0.26$\\
SNF20070630-006 & $125.5 \pm 3.2$ & $ 10.1 \pm 1.6$& $ 6126 \pm   4$ & $-29.34 \pm   0.01$ & $-29.12 \pm   0.01$& $-28.65 \pm   0.01$& $-28.38 \pm   0.01$& $-27.66 \pm   0.01$ & $  0.35 \pm   0.12$\\
PTF11drz & $132.6 \pm 1.4$ & $ 15.2 \pm 1.0$& $ 6116 \pm   5$ & $-29.12 \pm   0.01$ & $-28.95 \pm   0.01$& $-28.53 \pm   0.01$& $-28.27 \pm   0.01$& $-27.55 \pm   0.01$ & $  0.21 \pm   0.37$\\
SNF20080323-009 & $ 95.9 \pm 2.3$ & $ 10.6 \pm 1.1$& $ 6143 \pm   6$ & $-29.59 \pm   0.02$ & $-29.22 \pm   0.02$& $-28.68 \pm   0.02$& $-28.42 \pm   0.02$& $-27.77 \pm   0.02$ & $ -0.10 \pm   0.17$\\
SNF20071021-000 & $167.5 \pm 2.2$ & $ 20.4 \pm 0.6$& $ 6112 \pm   4$ & $-28.75 \pm   0.02$ & $-28.78 \pm   0.02$& $-28.40 \pm   0.02$& $-28.18 \pm   0.02$& $-27.41 \pm   0.02$ & $ -0.53 \pm   0.18$\\
SNNGC0927 & $155.2 \pm 1.4$ & $ 11.0 \pm 0.7$& $ 6109 \pm   4$ & $-28.87 \pm   0.02$ & $-28.81 \pm   0.01$& $-28.46 \pm   0.01$& $-28.22 \pm   0.01$& $-27.48 \pm   0.01$ & $  0.41 \pm   0.14$\\
SNF20060526-003 & $112.1 \pm 2.5$ & $  9.8 \pm 1.0$& $ 6121 \pm   3$ & $-29.34 \pm   0.01$ & $-29.09 \pm   0.01$& $-28.68 \pm   0.01$& $-28.39 \pm   0.01$& $-27.70 \pm   0.01$ & $  0.29 \pm   0.17$\\
SNF20080806-002 & $135.8 \pm 1.8$ & $  7.5 \pm 0.9$& $ 6135 \pm   4$ & $-29.22 \pm   0.02$ & $-29.02 \pm   0.02$& $-28.61 \pm   0.02$& $-28.35 \pm   0.01$& $-27.71 \pm   0.02$ & $  0.64 \pm   0.16$\\
SNF20080803-000 & $117.6 \pm 2.6$ & $  8.9 \pm 2.0$& $ 6125 \pm   4$ & $-28.84 \pm   0.01$ & $-28.70 \pm   0.01$& $-28.35 \pm   0.01$& $-28.16 \pm   0.01$& $-27.50 \pm   0.01$ & $  0.37 \pm   0.14$\\
SNF20080822-005 & $ 78.5 \pm 1.8$ & $  6.3 \pm 0.9$& $ 6138 \pm   4$ & $-29.71 \pm   0.01$ & $-29.34 \pm   0.01$& $-28.93 \pm   0.01$& $-28.61 \pm   0.01$& $-27.92 \pm   0.01$ & $  0.56 \pm   0.23$\\
SNF20060618-014 & $137.2 \pm 2.5$ & $  9.3 \pm 1.1$& $ 6112 \pm   7$ & $-29.27 \pm   0.03$ & $-29.09 \pm   0.03$& $-28.73 \pm   0.03$& $-28.38 \pm   0.03$& $-27.68 \pm   0.03$ & $  0.97 \pm   0.21$\\
PTF12ghy & $ 99.3 \pm 3.6$ & $ 16.8 \pm 0.7$& $ 6134 \pm   3$ & $-28.29 \pm   0.02$ & $-28.27 \pm   0.01$& $-28.05 \pm   0.01$& $-27.95 \pm   0.01$& $-27.40 \pm   0.01$ & $ -0.48 \pm   0.12$\\
SNF20070531-011 & $122.4 \pm 2.7$ & $ 21.2 \pm 0.8$& $ 6114 \pm   4$ & $-29.07 \pm   0.01$ & $-28.94 \pm   0.01$& $-28.50 \pm   0.01$& $-28.26 \pm   0.01$& $-27.53 \pm   0.03$ & $ -1.82 \pm   0.25$\\
SNF20070831-015 & $112.2 \pm 2.7$ & $  7.8 \pm 1.0$& $ 6145 \pm   6$ & $-29.42 \pm   0.01$ & $-29.17 \pm   0.01$& $-28.78 \pm   0.01$& $-28.46 \pm   0.01$& $-27.78 \pm   0.01$ & $  1.85 \pm   0.23$\\
SNF20070417-002 & $104.5 \pm 5.5$ & $ 24.4 \pm 2.2$& $ 6123 \pm   9$ & $-29.20 \pm   0.05$ & $-29.01 \pm   0.05$& $-28.48 \pm   0.05$& $-28.23 \pm   0.05$& $-27.54 \pm   0.05$ & $ -1.37 \pm   0.11$\\
PTF11cao & $143.3 \pm 1.6$ & $ 18.9 \pm 1.3$& $ 6104 \pm   5$ & $-28.78 \pm   0.02$ & $-28.79 \pm   0.02$& $-28.44 \pm   0.02$& $-28.18 \pm   0.02$& $-27.45 \pm   0.02$ & $  0.18 \pm   0.15$\\
SNF20080522-000 & $ 61.8 \pm 3.5$ & $  3.3 \pm 0.9$& $ 6131 \pm   7$ & $-29.86 \pm   0.01$ & $-29.41 \pm   0.01$& $-29.03 \pm   0.01$& $-28.70 \pm   0.01$& $-28.06 \pm   0.01$ & $  1.02 \pm   0.13$\\
PTF10qjq & $ 73.9 \pm 2.4$ & $ 12.8 \pm 0.8$& $ 6133 \pm   3$ & $-29.29 \pm   0.02$ & $-28.94 \pm   0.02$& $-28.53 \pm   0.01$& $-28.35 \pm   0.01$& $-27.76 \pm   0.01$ & $ -0.35 \pm   0.11$\\
PTF12dxm & $ 95.4 \pm 41.8$ & $ 35.7 \pm 2.8$& $ 6136 \pm   4$ & $-28.71 \pm   0.01$ & $-28.58 \pm   0.01$& $-28.19 \pm   0.01$& $-27.99 \pm   0.01$& $-27.34 \pm   0.01$ & $ -1.98 \pm   0.20$\\
SNF20061021-003 & $122.8 \pm 2.3$ & $  9.7 \pm 1.7$& $ 6131 \pm   4$ & $-29.04 \pm   0.02$ & $-28.86 \pm   0.02$& $-28.56 \pm   0.02$& $-28.30 \pm   0.02$& $-27.64 \pm   0.02$ & $  0.25 \pm   0.18$\\
SNF20080510-005 & $111.6 \pm 2.6$ & $  6.4 \pm 1.1$& $ 6115 \pm   4$ & $-29.41 \pm   0.01$ & $-29.15 \pm   0.01$& $-28.70 \pm   0.01$& $-28.38 \pm   0.01$& $-27.73 \pm   0.04$ & $  1.90 \pm   0.26$\\
SNF20080507-000 & $ 98.1 \pm 1.6$ & $ 10.6 \pm 2.1$& $ 6143 \pm   5$ & $-29.23 \pm   0.01$ & $-29.05 \pm   0.01$& $-28.71 \pm   0.01$& $-28.45 \pm   0.01$& $-27.79 \pm   0.01$ & $  0.89 \pm   0.25$\\
SNF20080913-031 & $118.2 \pm 1.5$ & $ 11.3 \pm 1.8$& $ 6158 \pm   5$ & $-29.13 \pm   0.08$ & $-29.01 \pm   0.07$& $-28.62 \pm   0.07$& $-28.32 \pm   0.07$& $-27.68 \pm   0.07$ & $  0.03 \pm   0.21$\\
SNF20080510-001 & $118.8 \pm 2.1$ & $ 15.3 \pm 1.3$& $ 6115 \pm   4$ & $-29.35 \pm   0.01$ & $-29.15 \pm   0.01$& $-28.69 \pm   0.01$& $-28.38 \pm   0.01$& $-27.68 \pm   0.01$ & $  0.43 \pm   0.18$\\
SNF20070712-003 & $108.8 \pm 2.7$ & $ 13.5 \pm 0.9$& $ 6155 \pm   6$ & $-29.44 \pm   0.02$ & $-29.19 \pm   0.01$& $-28.74 \pm   0.01$& $-28.42 \pm   0.01$& $-27.78 \pm   0.01$ & $  0.42 \pm   0.18$\\
\enddata
\end{deluxetable}

The distributions of the central values of the inputs  $EW_{\mathrm{Ca}}$, $EW_{\mathrm{Si}}$,
$\lambda_{\mathrm{Si}}$ , $x_1$, and the SALT2 color parameter $C$ are shown in Table~\ref{input:fig}.
For reference, the $x_1$ and $C$ distributions of the $0.03<z< 0.08$ Pantheon sample,
predominantly supernovae from CFA3 \citep{2009ApJ...700..331H}, CFA4 \citep{2012ApJS..200...12H}, and CSP
\citep{2010AJ....139..519C, 2011AJ....142..156S}, that were used as part of the low-redshift anchor in the recent cosmology analysis of \citet{2017arXiv171000845S}.

\begin{figure}[htbp] %  figure placement: here, top, bottom, or page
   \centering
   \includegraphics[width=4in]{perobject_input.pdf}
   \caption{\added{Distributions of the central values of the inputs  $EW_{\mathrm{Ca}}$, $EW_{\mathrm{Si}}$,
$\lambda_{\mathrm{Si}}$ , $x_1$, and $C$ of the sample used in this article, and
the $x_1$ and $C$ distributions of the $0.03<z< 0.08$ Pantheon sample.}
   \label{input:fig}}
\end{figure}
}

\section{Model~I: Two Color Parameters}
\label{modelI:sec}
We begin by considering Model~I, in which supernova magnitudes  have linear dependence on spectral parameters
and two color parameters. The purpose of this section is to introduce some basic concepts of our models,
to show how the two  color terms can be associated with existing dust-extinction models, and to give Model~I results on dust extinction.
Deficiencies of this model are addressed in the subsequent models presented in the article.
To avoid repetitiveness,
we refrain from going into full detail on the results of Model~I in deference to Model~III, which is presented in Section~\ref{modelIII:sec}.

\added{It is important to note that the model and those in subsequent sections are applied to our supernova sample;
the ensuing results thus correspond to our SN~Ia sample's underlying population. }

\subsection{Model}
We assume 
that  peak
underlying
 ${\hat{U}}{\hat{B}}\hat{V}\hat{R}\hat{I}$ magnitudes of a supernova  with effects from  astrophysical foregrounds are linearly dependent
on its
 equivalent widths of the Ca~II H\&K and Si~II~$\lambda$4141 spectral features
$EW_{\mathrm{Ca}}$ and $EW_{\mathrm{Si}}$,
and the wavelength of the minimum of 
the Si~II~$\lambda6355$ feature $\lambda_{\mathrm{Si}}$
around $B$-band peak brightness:
these spectral features are associated with SN~Ia  spectroscopic diversity  
\citep{2005ApJ...623.1011B, 2006PASP..118..560B, 2008A&A...492..535A, 2009A&A...500L..17B, 2009PASP..121..238B, 2009ApJ...699L.139W, 2011ApJ...729...55F,2012AJ....143..126B}.
The explicit omission of light-curve shape in our model is compensated by its proxy,
$EW_{\mathrm{Si}}$, at peak brightness
\citep{2008A&A...492..535A, 2011A&A...529L...4C}. 
The underlying magnitudes are also linearly dependent on the per-supernova latent parameters $g_0$ and $g_1$.
Unlike
the spectral parameters ($EW_{\mathrm{Ca}}$, $EW_{\mathrm{Si}}$ and $\lambda_{\mathrm{Si}}$),  $g_0$ and $g_1$ are not directly associated
with observables but rather are inferred as part of the analysis.
A grey magnitude offset, $\Delta$, is included for each supernova
to capture 
potential
band-independent intrinsic dispersion, while also absorbing peculiar-velocity errors introduced when converting
fluxes to luminosities.
With this grey offset the model standardizes colors, not absolute magnitude.
The underlying magnitudes are
\begin{equation}
\begin{pmatrix}
{\hat{U}}\\{\hat{B}}\\{\hat{V}}\\{\hat{R}}\\{\hat{I}}
\end{pmatrix}
=
\Delta \vec{1} +
\begin{pmatrix}
c_{\hat{U}}+\alpha_{\hat{U}} EW_{\mathrm{Ca}} + \beta_{\hat{U}} EW_{\mathrm{Si}} + \eta_{\hat{U}} \lambda_{\mathrm{Si}}  +\gamma^0_{{\hat{U}}} g_0 +\gamma^1_{{\hat{U}}} g_1 \\
c_{\hat{B}}+\alpha_{\hat{B}} EW_{\mathrm{Ca}} + \beta_{\hat{B}} EW_{\mathrm{Si}} + \eta_{\hat{B}} \lambda_{\mathrm{Si}}  +\gamma^0_{{\hat{B}}} g_0 +\gamma^1_{{\hat{B}}} g_1 \\
c_{\hat{V}}+\alpha_{\hat{V}} EW_{\mathrm{Ca}} + \beta_{\hat{V}} EW_{\mathrm{Si}} + \eta_{\hat{V}} \lambda_{\mathrm{Si}} +\gamma^0_{{\hat{V}}} g_0 +\gamma^1_{{\hat{V}}} g_1 \\
c_{\hat{R}}+\alpha_{\hat{R}} EW_{\mathrm{Ca}} + \beta_{\hat{R}} EW_{\mathrm{Si}} + \eta_{\hat{R}} \lambda_{\mathrm{Si}}  +\gamma^0_{{\hat{R}}} g_0 +\gamma^1_{{\hat{R}}} g_1 \\
c_{\hat{I}}+\alpha_{\hat{I}} EW_{\mathrm{Ca}} + \beta_{\hat{I}} EW_{\mathrm{Si}}+ \eta_{\hat{I}} \lambda_{\mathrm{Si}}  +\gamma^0_{{\hat{I}}} g_0 +\gamma^1_{{\hat{I}}} g_1
\end{pmatrix}.
\label{ewsiv:eqn}
\end{equation}
$\vec{1}$ is the Identity vector.
The global parameter vectors\footnote{Global parameters act on each of the five bands: a single component is written with a subscript for the corresponding band, e.g.\  $c_{\hat{U}}$; 
the combination of all components are written as a vector, e.g.\ $\vec{c}=(c_{\hat{U}}, c_{\hat{B}}, c_{\hat{V}}, c_{\hat{R}}, c_{\hat{I}}) $.  The
band subscript or arrow
notationally identify global parameters, in contrast with the per-supernova parameters.
} that describe the SN~Ia population
$\vec{c}$  (units of mag);
$\vec{\alpha}$, $\vec{\beta}$,
$\vec{\eta}$ (units of mag~\AA$^{-1}$), $\vec{\gamma}^0$ and $\vec{\gamma}^1$ (units of mag)
 are the intercept
and slopes
of the linear relationships that
relate
per-supernova
parameters with 
underlying magnitudes.
The notation used in this article is not to be associated with those from previous works.


The observables
$\hat{U}_o, {\hat{B}}_o, {\hat{V}}_o, {\hat{R}}_o, {\hat{I}}_o$, $EW_{Ca,o}$, $EW_{Si,o}$, $\lambda_{Si,o}$
have Gaussian measurement uncertainty with covariance $C$.
The
likelihood density for the described model
corresponds to the probability distribution function described by
\begin{equation}
\begin{pmatrix}
{\hat{U}}_o\\{\hat{B}}_o\\ {\hat{V}}_o\\{\hat{R}}_o\\{\hat{I}}_o\\EW_{Si, o}\\ EW_{Ca, o} \\ \lambda_{Si, o}
\end{pmatrix}
\sim \mathcal{N}
\left(
\begin{pmatrix}
{\hat{U}}  \\{\hat{B}}  \\
{\hat{V}}\\{\hat{R}}\\{\hat{I}}\\
EW_{\mathrm{Si}}\\ EW_{\mathrm{Ca}} \\ \lambda_{\mathrm{Si}}
\end{pmatrix}
,C
\right).
\label{dust:eqn}
\end{equation}


The model as written has degeneracies that need to be constrained in order for fits to converge.
The model does not specify the absolute magnitude nor the 
color of a $g_0=0$, $g_1=0$ supernova;
in their place the zeropoints of the per-supernova magnitude and color parameters are set to the sample means
\begin{equation}
\langle \Delta \rangle=0,\ \langle g_0 \rangle=0,\ \langle g_1 \rangle=0.
\label{zero:eqn}
\end{equation}
The model contains the product of parameters
$\vec{\gamma} g$, which leads to the degeneracy $\vec{\gamma} \rightarrow a\vec{\gamma}$, $g \rightarrow a^{-1} g$.
To aid in the convergence of $\vec{\gamma}$ and $g$ we impose a prior on the rms of $g$, though
our physical interpretations are ultimately independent of this scaling.
This prior does not specify the sign of $a$, which leaves a parity degeneracy. 
Results in this article are given in terms of $\vec{\gamma}g$, not $g$ alone.
As will be seen, the signal-to-noise in $\vec{\gamma}$ is sufficiently
high that our finite MCMC chains do not migrate between the degenerate solutions; to simplify the merging of multiple chains
we impose one of the degenerate solutions
\begin{equation}
\gamma^0_{\hat{U}} > 0,\ \gamma^1_{\hat{U}} < 0.
\end{equation}
As  in mixture models, the combinations $\vec{\gamma}^0 g_0$ and $\vec{\gamma}^1 g_1$ terms are degenerate under exchange of the 0 and 1 indices. For a cleaner presentation of our results, 
we break that degeneracy  by setting consistent initial conditions that succeed in keeping indices consistent over all analysis chains.  

Each of the multiplicative cofactors has a degeneracy with the zeropoint.  Consider
$\vec{\alpha} \rightarrow \vec{\alpha} +   \epsilon_{\alpha}$,
$\vec{\beta} \rightarrow \vec{\beta} +   \epsilon_{\beta}$,
$\vec{\eta} \rightarrow \vec{\eta} +   \epsilon_{\eta}$,
$\vec{\gamma}^0 \rightarrow \vec{\gamma}^0 +   \epsilon_{\gamma^0}$,
$\vec{\gamma}^1 \rightarrow \vec{\gamma}^1 +   \epsilon_{\gamma^1}$.
Then
each supernova's grey term can shift as
$\Delta \rightarrow \Delta  + \epsilon_\Delta$
with
$$
\epsilon_\Delta  = -  \epsilon_{\alpha} (EW_{\text{Ca}}-\langle EW_{\text{Ca}}\rangle)- 
 \epsilon_{\beta} (EW_{\mathrm{Si}}-\langle EW_{\mathrm{Si}}\rangle) -  \epsilon_{\eta} (\lambda_{\mathrm{Si}}-\langle \lambda_{\mathrm{Si}}\rangle)
- \epsilon_{\gamma^0}g_0 -  \epsilon_{\gamma^1}g_1.
$$
We break this degeneracy by fixing the values of $\epsilon$ such that $\Delta$ is
uncorrelated with  $EW_{\text{Ca}}$, $EW_{\mathrm{Si}}$, $\lambda_{\text{Ca}}$,
$g_0$, and $g_1$ reflecting
our preference for feature affects to be confined to the cofactors and not the grey magnitude offsets.

All parameters not  otherwise noted above have flat priors.

For $N$ supernovae there are $8N$ observables.  There are $3N$ spectral parameters, each of
$\Delta$, $g_0$, $g_1$ contributes $N-1$ parameters
(recall the constraint in Eq.~\ref{zero:eqn}),  and there are $5 \times 7$ global coefficients.
For $N=172$ supernovae, there are 1376 observables and 1064  parameters.


\subsection{Results for the Latent Parameter Cofactors  $\vec{\gamma}^0$ and $\vec{\gamma}^1$  }
\label{results1:sec}
\added{The MCMC chains for all parameters from this and subsequent sections are accessible from the SNfactory website\footnote{For the referee the parameters
are available at \url{https://drive.google.com/open?id=1d52nxfF7eS3HblfbmT3fxgiZod_Z-Gjg}. Concurrent with publication the chain will be
made available
at \url{https://snfactory.lbl.gov/snf/data/index.html}.}.}
In this subsection we concentrate on the results for $\vec{\gamma}^0$ and $\vec{\gamma}^1$.
In our model, $g_0$ and $g_1$ are introduced as latent parameters that are related to linear shifts in band magnitudes through
the  coefficients $\vec{\gamma}^0$ and
$\vec{\gamma}^1$.  Otherwise, no physically-motivated suppositions on the properties of $g$ and $\vec{\gamma}$ are imposed.
Since dust extinction is a dominant determinant of supernova colors that is often described by a 2-parameter linear model, we expect
for our two  color terms to capture the effects of dust.
In this section we reexpress our model $\vec{\gamma}$'s in terms of
the dust-extinction model of \citet{1999PASP..111...63F} \citepalias[henceforth referred to as][]{1999PASP..111...63F}.

The 68\% credible intervals for $\vec{\gamma}^0$ and $\vec{\gamma}^1$ (and all other global parameters) are given in \S\ref{m12results:sec}.
All elements of the two $\vec{\gamma}$ parameter vectors 
describing color
are significantly non-zero.
None of the 20000 links of 
our Monte Carlo chains for $\vec{\gamma}$ come close to the origin.   We claim 
a probability $(1-5\times 10^{-5})$-detection of
a magnitude term in the form of a 5-dimensional vector spanned by two basis vectors
\begin{equation}
\vec{A} = \vec{\gamma}^0 g_0 +  \vec{\gamma}^1 g_1.
\end{equation}
All possible magnitude changes in our five bands due to $g_0$ and $g_1$ are confined to a two-dimensional
plane defined by $\vec{\gamma}^0$ and  $\vec{\gamma}^1$. 

Dust-extinction
models are also approximately described by a 2-parameter linear model, here 
using the commonly-used parameters $A_V$ and $E(B-V)$
\begin{equation}
\vec{A} =\vec{ a}  A_{V} + \vec{b} E(B-V).
\label{f99:eqn}
\end{equation}

For example,   the wavelength-dependent model of  \citet{1989ApJ...345..245C} is linear,
though its linearity is lost when integrated over broad-band filters.
For the case of
$R^F=2.5$ and $A^F_V=0.1$ dust attenuating light from the SALT2
\citep{2007A&A...466...11G} $s=1$, $x_1=0$ SN~Ia template at ${\hat{B}}$-band peak, the 
\citetalias{1999PASP..111...63F} model
can be approximated with
$\vec{a} = (0.96,   1.00,   1.00,   0.97,   0.77)$ and $\vec{b}=(  1.77,   0.98,   0.12,  -0.50,  -0.53)$.
(The $F$ superscript is used to distinguish parameters of the \citetalias{1999PASP..111...63F} model.)
Over the ranges
 $0\le A^F_V\le 1$ and $2 \le R^F \le 3.5$ 
and the wavelengths under consideration here,  the values of the elements of $a$ and $b$ vary by $<5$\%
for the
\citetalias{1999PASP..111...63F} model with
 the largest error in extinction from the linear approximation having amplitude $<0.008$ mag.
All possible magnitude changes in our five bands due to  \citetalias{1999PASP..111...63F}  dust extinction are confined to a two-dimensional
plane defined by $\vec{a}$ and  $\vec{b}$. 


The credible interval for $\vec{\gamma}$'s and other parameters are presented in Table~\ref{global1:tab}.
Our model $\vec{\gamma}$ vectors can be written as a linear combination (described by the $2 \times 2$ matrix M) of the
dust-based $\vec{a}$ and $\vec{b}$ vectors plus residual vectors perpendicular to the plane spanned by $\vec{a}$ and $\vec{b}$,
\begin{equation}
\begin{pmatrix}
\vec{\gamma}^0 \\
\vec{\gamma}^1
\end{pmatrix}=
M
\begin{pmatrix}
\vec{a} \\
\vec{b}
\end{pmatrix}+
\begin{pmatrix}
\vec{\epsilon}_{\gamma^0} \\
\vec{\epsilon}_{\gamma^1}
\end{pmatrix}.
\label{trans_I:eqn}
\end{equation}
We find
\begin{equation}
M=
\begin{pmatrix}
\begin{array}{rr}
43.7^{+2.3}_{-2.1} & 19.4^{+0.5}_{-0.5} \\
-18.7^{+3.0}_{-3.1} & -0.8^{+0.9}_{-1.0} \\
\end{array}
\end{pmatrix} ,
\end{equation}
\begin{align}
\begin{split}
\vec{\epsilon}_{\gamma^0} &=\left(1.5^{+0.3}_{-0.3} , -1.4^{+0.3}_{-0.3} , -1.8^{+0.4}_{-0.3} , 1.4^{+0.1}_{-0.1} , 0.5^{+0.4}_{-0.5} \right) \text{mag}, \\
\vec{\epsilon}_{\gamma^1} & =\left(4.8^{+0.3}_{-0.3} , -4.3^{+0.3}_{-0.3} , -5.7^{+0.4}_{-0.4} , 0.1^{+0.2}_{-0.2} , 6.8^{+0.5}_{-0.5}\right) \text{mag},
\end{split}
\label{res_1:eqn}
\end{align}
which correspond to
a quadratic fraction of $0.9993^{+0.0002}_{-0.0003}$ of $\vec{\gamma}^0$ and
$0.9303^{+0.0210}_{-0.0341}$ of $\vec{\gamma}^1$ projecting onto the $\vec{a}$--$\vec{b}$ plane.
The  contributions of these residuals to magnitudes are relatively small compared to
the values of $\vec{\gamma}^0$ and $\vec{\gamma}^1$ in Table~\ref{global1:tab}.
For the standard deviation values of $g_0$ and $g_1$ of the supernovae in our sample, the amplitude of the magnitude residuals due to $\vec{\epsilon}$ are 
$\left(0.025, 0.023,0.029, 0.008, 0.027\right)$ mag.
The allowed color
variations in ${\hat{U}}{\hat{B}}{\hat{V}}{\hat{R}}{\hat{I}}$ allowed by the \citetalias{1999PASP..111...63F} model and our best-fit model are confined to almost identical
2-dimensional
planes within
the 5-dimensional magnitude space.
With no prior assumptions of dust extinction behavior or the distribution of $A_V$, the supernova data themselves exhibit
2-dimensional color variations that are closely aligned with the 2-dimensional color variations predicted by the \citetalias{1999PASP..111...63F} dust model.


The above result is visualized in Figure~\ref{plane:fig}, which  shows  in the
${\hat{U}}{\hat{V}}\hat{I}$-subspace
two perspectives (left and right panels)
of the unit vectors corresponding to $\vec{\gamma}^0$ and  $\vec{\gamma}^1$ of our model (solid lines),
and $\vec{a}$, $\vec{b}$ of the \citetalias{1999PASP..111...63F} model (dashed lines).  All  vectors are set to intersect the origin.
The two perspectives show that while the four vectors point in different directions for each band combination,
they are almost coplanar in ${\hat{U}}{\hat{V}}{\hat{I}}$ (they are slightly less coplanar in ${\hat{B}}{\hat{V}}{\hat{R}}$).  The $\vec{a}$ and $\vec{b}$ vectors and the $\vec{\gamma}^0$ and $\vec{\gamma}^1$
vectors span near-parallel planes in the 5-dimensional color space, and hence the latter can be almost entirely expressed in terms of
the former.

\begin{figure}[htbp] %  figure placement: here, top, bottom, or page
   \centering
   \includegraphics[width=2.95in]{fix1plane0.pdf}
   \includegraphics[width=2.95in]{fix1plane1.pdf}
   \caption{
   Visualization of how supernova magnitudes can vary in Model~I and that of \citetalias{1999PASP..111...63F}.  While the models describe
   magnitudes in 5-bands ${\hat{U}}{\hat{B}}{\hat{V}}{\hat{R}}{\hat{I}}$ this visualization shows only ${\hat{U}}{\hat{V}}{\hat{I}}$.   The left and right plots show the same information from
   two different perspectives.
   Our model is shown in solid-line unit vectors in the directions of $\vec{\gamma}^0$ and $\vec{\gamma}^1$. The only possible magnitudes
   are a linear combination of these two vectors, and hence are confined to the plane that contains both.
   The  \citetalias{1999PASP..111...63F} model is shown in dashed-line unit vectors
   in the directions of  $\vec{a}$, $\vec{b}$; the only possible magnitudes are confined to the plane that contains both.
   The perspective on the right visually shows that all four vectors are nearly coplanar.  
   The combination $\vec{a}+\vec{b}/2.25$ is shown in the dotted red
   line: it is almost perfectly superimposed on $\vec{\gamma}^0$.
   \label{plane:fig}}
\end{figure}


We
now turn to
the specific choice of $\vec{\gamma}$'s  returned by the fit in terms of the \citetalias{1999PASP..111...63F}  dust-extinction model. 
Our linear model would be satisfied by any two independent vectors  that span
the plane defined by the $\vec{\gamma}$ vectors. 
While
Eq.~\ref{f99:eqn} expresses that model using basis vectors $\vec{a}$ and $\vec{b}$ and
their corresponding parameters 
$A_V$ and 
$E(B-V)$, it can be written more generally using arbitrary bases $\vec{a}+\kappa_1 \vec{b}$
and $\kappa_2 \vec{a} + \vec{b}$, and parameter sets
$A^F_V - \kappa_2 E^F(B-V)$ and $-\kappa_1 A^F_V + E^F(B-V)$  such that 
\begin{equation}
A_X =  (1-\kappa_1 \kappa_2)^{-1} [(A^F_V - \kappa_2 E^F(B-V))\left(a_X+\kappa_1 b_X \right) +  (-\kappa_1 A^F_V + E^F(B-V)) (\kappa_2 a_X + b_X)],
\label{newdust:eqn}
\end{equation}
where $\kappa_1$ and $\kappa_2$ are free to float.
In our analysis, the $\vec{\gamma}$ vectors converge to a specific direction, meaning that our result prefers specific values of
$\kappa_1=M[1,2]/M[1,1]$ and $\kappa_2=M[2,1]/M[2,2]$, where $M$ is the matrix in  Eq.~\ref{trans_I:eqn}.

The values of $\kappa$'s are determined by the conditions that the components $A^F_V - \kappa_2 E^F(B-V)$ and $-\kappa_1 A^F_V + E^F(B-V)$ be
uncorrelated with $\Delta$.   The corresponding basis vectors are $a_X+\kappa_1 b_X $ and $\kappa_2 a_X + b_X$.
Defining an effective total to selective extinction $\langle R^F \rangle \equiv \kappa_1^{-1}$ so that the first basis vector is
$a_X+ b_X/\langle R^F \rangle  $, we find from the 68\%-ile credible interval of $\kappa_1^{-1}$ that
$\langle R^F \rangle=2.25^{+0.11}_{-0.10}$.
The non-correlation condition does not strongly constrain the  second basis vector as
$\kappa_2 =  15.60^{+26.19}_{-32.47}$.
Figure~\ref{plane:fig} visualizes the near-perfect alignment of $\vec{\gamma}^0$ and  $\vec{a}+\vec{b}/2.25$.


\subsection{Model~I Summary}
There are two points to highlight from the analysis of
Model~I.
\begin{itemize}
\item Model~I introduces two latent supernova parameters that can influence supernova colors.
These parameters come  with ``non-informative'' priors.  Fitting
of our model gives a space of possible color variations highly consistent with those expected from two-parameter dust-extinction models. 
That we recover a physically-motivated  result lends confidence in the modeling and analysis.
\item In Model~I, any  color variation that remains after spectral corrections is allocated to two parameters.
Dust extinction is not the only effect that can contribute to $g_0$ and $g_1$ so interpreting
them as pure dust parameters is subject to systematic errors.
\item We have defined an effective parameter $\langle R^F\rangle$ whose value may be compared with dust models.
Such comparisons must be taken with a grain of salt,
as this value is not fundamentally constrained by the model but is fixed by the added condition that $\Delta$ is
uncorrelated with  $EW_{\text{Ca}}$, $EW_{\mathrm{Si}}$, $\lambda_{\text{Ca}}$,
$g_0$, and $g_1$.  The value of  $\langle R^F\rangle$ changes given different ansatzes  for breaking the $\Delta$--cofactor degeneracy 
\end{itemize}

\section{Model~II: Three Color Parameters}
\label{modelII:sec}
Model~II extends Model~I to include a third color parameter.  This increase in the number of degrees of freedom 
goes beyond the two of dust-extinction models used in supernova analysis.
 In this section we  focus on describing the consequences of the addition of the
new parameter.

\subsection{Model}
\label{modelIImodel:sec}
Model~II is identical to Model~I except for the 
addition of a third color parameter $p$.  The distribution of $p$ in our sample
is assumed to form a Normal distribution with unit standard deviation.  An unconstrained distribution,
as taken for $g_0$ and $g_1$, yields multiple local maxima in the posterior and leads to difficulty in the convergence of the MCMC.
The relative effect on the five magnitudes is specified by the global parameters 
$\sigma_p\vec{\phi}$, where $\vec{\phi}$ is a 5-dimensional unit vector.
Model~II reads as
\begin{equation}
p  \sim \mathcal{N}(0,1 ),
\end{equation}
\begin{equation}
\sigma_p  \sim \text{Cauchy}(0.1,0.1),
\end{equation}
\begin{equation}
\begin{pmatrix}
{\hat{U}}\\{\hat{B}}\\{\hat{V}}\\{\hat{R}}\\{\hat{I}}
\end{pmatrix}
=
\Delta  \vec{1} +
\begin{pmatrix}
c_{\hat{U}}+\alpha_{\hat{U}} EW_{\mathrm{Ca}} + \beta_{\hat{U}} EW_{\mathrm{Si}} + \eta_{\hat{U}} \lambda_{\mathrm{Si}}+\gamma^0_{{\hat{U}}} g_0 +\gamma^1_{{\hat{U}}} g_1  + \sigma_p\phi_{\hat{U}} p\\
c_{\hat{B}}+\alpha_{\hat{B}} EW_{\mathrm{Ca}} + \beta_{\hat{B}} EW_{\mathrm{Si}} + \eta_{\hat{B}} \lambda_{\mathrm{Si}} +\gamma^0_{{\hat{B}}} g_0 +\gamma^1_{{\hat{B}}} g_1  + \sigma_p\phi_{\hat{B}} p\\
c_{\hat{V}}+\alpha_{\hat{V}} EW_{\mathrm{Ca}} + \beta_{\hat{V}} EW_{\mathrm{Si}} + \eta_{\hat{V}} \lambda_{\mathrm{Si}} +\gamma^0_{{\hat{V}}} g_0 +\gamma^1_{{\hat{V}}} g_1 + \sigma_p\phi_{\hat{V}}  p\\
c_{\hat{R}}+\alpha_{\hat{R}} EW_{\mathrm{Ca}} + \beta_{\hat{R}} EW_{\mathrm{Si}} + \eta_{\hat{R}} \lambda_{\mathrm{Si}} +\gamma^0_{{\hat{R}}} g_0 +\gamma^1_{{\hat{R}}} g_1 + \sigma_p\phi_{\hat{R}} p\\
c_{\hat{I}}+\alpha_{\hat{I}} EW_{\mathrm{Ca}} + \beta_{\hat{I}} EW_{\mathrm{Si}}+ \eta_{\hat{I}} \lambda_{\mathrm{Si}}+\gamma^0_{{\hat{I}}} g_0 +\gamma^1_{{\hat{I}}} g_1 + \sigma_p\phi_{\hat{I}}  p
\end{pmatrix}.
\label{ewsiv2:eqn}
\end{equation}

The Cauchy prior is commonly used to concentrate probability around a specific value while having fat tails with infinite variance.


The model contains the product 
$\vec{\phi} p$, which has the degeneracy $\vec{\phi} \rightarrow -\vec{\phi}$, $p \rightarrow -p$.
No conditions are added to break this degeneracy to leave open the possibility that
$\sigma_p\vec{\phi}$ is consistent with zero.  The degeneracy
in $\vec{\phi}$ is apparent later in the article, but it does not affect the magnitude predictions. 
As in \S\ref{modelI:sec}, the degeneracy between $\vec{\phi}$ and $\Delta$ is fixed by enforcing zero correlation between $\Delta$ and $p$.

For $N$ supernovae there are $8N$ observables.  There are $3N$ spectral parameters, each of
$\Delta$, $g_0$, $g_1$ contributes $N-1$ parameters while $p$  contributes $N$ parameters,  and  there are $5 \times 7$ global coefficients.
For $N=172$ supernovae, there are 1376 observables and 1236  parameters.

\subsection{Results for the Latent Parameter Cofactor $\sigma_p \vec{\phi}$}
\label{results2:sec}
The credible intervals of all Model~II global parameters are given in  \S\ref{m12results:sec}.
In this subsection we focus on the effect of adding the third color term  $\sigma_p \vec{\phi}$.
There is strong evidence for a third color parameter.
The 68\% credible interval for $\sigma_p$ is $0.051^{+0.005}_{-0.004}$~mag, highly incompatible with zero. 
The posterior for $\sigma_p \vec{\phi}$ is shown in Figure~\ref{M1ev:fig}.  
Two maxima are allowed due to the sign degeneracy for $\vec{\phi}$.  Practically each of the MCMC chains converges to one of the solutions and does
not migrate between them by crossing zero.  To aid visibility of this figure we thus impose $\phi_I > 0$.  
 The new parameter does affect colors in a  manner distinct from $g_0$ and $g_1$.
A quadratic fraction of
$0.48^{+0.25}_{-0.17}$ of 
 $\vec{\phi}$ lies perpendicular to the $\vec{\gamma}_0$--$\vec{\gamma}_1$ plane.


\begin{figure}[htbp] %  figure placement: here, top, bottom, or page
   \centering
   \includegraphics[width=4in]{fix3sigev.pdf} 
            \caption{Model~II posterior  contours for $\sigma_p \vec{\phi}$. 
            The solid line shows the location of zero.          \label{M1ev:fig}}
\end{figure}

Model~II introduces a new parameter  that can absorb color variability
that in Model~I is restricted to $g_0$ and $g_1$, and whose distribution comes from a Normal distribution.
The change in the distributions of the $\gamma^0_{\hat{V}} g_0$ and $\gamma^1_{\hat{V}}g_1$ parameters from
 Model~I to Model~II
is apparent in Figure~\ref{kcomp:fig}.  (Here and in similar plots, the parameter value of one supernova,
e.g.\ $\gamma^0_{\hat{V}} g_0|_0$, is subtracted
out to null out correlated
parameter uncertainty between supernovae; recall the $\langle \Delta \rangle=0$, $\langle g_0 \rangle=0$, $\langle g_1 \rangle=0$,
and $p  \sim \mathcal{N}(0,1)$ conditions.) The  change from Model~I to II is pronounced in  $\gamma^1_{\hat{V}}g_1$,
where  a Normal distribution component apparent in Model~I disappears in Model~II, presumably having been  assigned to $p$.


\begin{figure}[htbp] %  figure placement: here, top, bottom, or page
   \centering
   \includegraphics[width=2.6in]{fix1deltagamma0_med.pdf} 
   \includegraphics[width=2.6in]{fix1deltagamma1_med.pdf} 
   \includegraphics[width=2.6in]{fix3deltagamma0_med.pdf} 
   \includegraphics[width=2.6in]{fix3deltagamma1_med.pdf} 
   \caption{
   Left:
   Normalized stack of the posteriors (includes parameter uncertainties) of all supernovae  and histogram 
      of  per-supernova 
median values of $\gamma^0_{\hat{V}} g_0-\gamma^0_{\hat{V}} g_0|_0$.  Right:  The same plots for $\gamma^1_{\hat{V}} g_1-\gamma^1_{\hat{V}} g_1|_0$,
Top: Model~I. Bottom: Model~II.
    \label{kcomp:fig}}
\end{figure}

The $\vec{\gamma}$ vectors, like those from Model~I, are almost entirely projected
onto the plane defined by
$\vec{a}$ and $\vec{b}$ of the \citetalias{1999PASP..111...63F} model.
The direction of the  $\vec{\gamma}^0$ vector corresponds to an effective
$\langle R^F_{\mathit{eff}}\rangle = 2.54^{+0.13}_{-0.12}$.
The change of $\langle R^F_{\mathit{eff}}\rangle$ from Model~I to II is anticipated by the change in the distribution of $g_1$,
whose connection with $R^F$ is shown in \S\ref{g0g1:sec}.


\subsection{Model~II Summary}
The inclusion of a third color term beyond the two of Model~I has important consequences.

\begin{itemize}
\item The data strongly supports the influence of an additional supernova parameter that affects supernova colors in a manner orthogonal to
 dust extinction.
\item The effective $\langle R^F_{\mathit{eff}}\rangle$ shifts by $>1 \sigma$ when including the new color term.  This demonstrates,
as has been noted by \citet{2009ApJ...704.1036F,2009ApJS..185...32K}, that dust analyses
that do not allow for additional color dispersion are susceptible to bias when inferring dust parameters.
\end{itemize}

\section{Model~III: Three Color Parameters and Light-Curve Shape}
\label{modelIII:sec}

Light-curve shape is an established indicator of SN~Ia diversity 
\citep{1993ApJ...413L.105P, 1996ApJ...473...88R,
1997ApJ...483..565P}
with its own color correlations \citep{2005A&A...443..781G, 2007ApJ...659..122J}.
Model~III explicitly includes light-curve shape as a feature used to standardize supernova colors, to complement the
spectral features considered in the previous two models.
We will show that light-curve shape does have a non-trivial relationship with color that is not already encoded
in the spectral features.  Model~III is thus the focus of
this article and its results are  described here in depth.


\subsection{Model}
\label{modelIImodel:sec}
Model~III is identical to Model~II except for the 
addition of a new term  that relates magnitudes with the SALT2 light-curve-shape parameter,
$x_1$, through linear coefficients $\vec{\zeta}$. 
The model reads as
\begin{equation}
p  \sim \mathcal{N}(0,1 ),
\end{equation}
\begin{equation}
\sigma_p  \sim \text{Cauchy}(0.1,0.1),
\end{equation}
\begin{equation}
\begin{pmatrix}
{\hat{U}}\\{\hat{B}}\\{\hat{V}}\\{\hat{R}}\\{\hat{I}}
\end{pmatrix}
=
\Delta  \vec{1} +
\begin{pmatrix}
c_{\hat{U}}+\alpha_{\hat{U}} EW_{\mathrm{Ca}} + \beta_{\hat{U}} EW_{\mathrm{Si}} + \eta_{\hat{U}} \lambda_{\mathrm{Si}} +\zeta_{\hat{U}} x_1 +\gamma^0_{{\hat{U}}} g_0 +\gamma^1_{{\hat{U}}} g_1 + \sigma_p\phi_{\hat{U}} p\\
c_{\hat{B}}+\alpha_{\hat{B}} EW_{\mathrm{Ca}} + \beta_{\hat{B}} EW_{\mathrm{Si}} + \eta_{\hat{B}} \lambda_{\mathrm{Si}} +\zeta_{\hat{B}} x_1 +\gamma^0_{{\hat{B}}} g_0 +\gamma^1_{{\hat{B}}} g_1 + \sigma_p\phi_{\hat{B}} p\\
c_{\hat{V}}+\alpha_{\hat{V}} EW_{\mathrm{Ca}} + \beta_{\hat{V}} EW_{\mathrm{Si}} + \eta_{\hat{V}} \lambda_{\mathrm{Si}} +\zeta_{\hat{V}} x_1+\gamma^0_{{\hat{V}}} g_0 +\gamma^1_{{\hat{V}}} g_1 + \sigma_p\phi_{\hat{V}}  p\\
c_{\hat{R}}+\alpha_{\hat{R}} EW_{\mathrm{Ca}} + \beta_{\hat{R}} EW_{\mathrm{Si}} + \eta_{\hat{R}} \lambda_{\mathrm{Si}} +\zeta_{\hat{R}} x_1+\gamma^0_{{\hat{R}}} g_0 +\gamma^1_{{\hat{R}}} g_1 + \sigma_p\phi_{\hat{R}} p\\
c_{\hat{I}}+\alpha_{\hat{I}} EW_{\mathrm{Ca}} + \beta_{\hat{I}} EW_{\mathrm{Si}}+ \eta_{\hat{I}} \lambda_{\mathrm{Si}}+\zeta_{\hat{I}} x_1+\gamma^0_{{\hat{I}}} g_0 +\gamma^1_{{\hat{I}}} g_1 + \sigma_p\phi_{\hat{I}}  p
\end{pmatrix},
\label{ewsiv3:eqn}
\end{equation}

\begin{equation}
\begin{pmatrix}
{\hat{U}}_o\\{\hat{B}}_o\\ {\hat{V}}_o\\{\hat{R}}_o\\{\hat{I}}_o\\EW_{Si, o}\\ EW_{Ca, o} \\ \lambda_{Si, o} \\ x_{1, o}
\end{pmatrix}
\sim \mathcal{N}
\left(
\begin{pmatrix}
{\hat{U}}  \\{\hat{B}}  \\
{\hat{V}}\\{\hat{R}}\\{\hat{I}}\\
EW_{\mathrm{Si}}\\ EW_{\mathrm{Ca}} \\ \lambda_{\mathrm{Si}} \\ x_1
\end{pmatrix}
,C
\right).
\label{dust3:eqn}
\end{equation}
The changes from Model~II are the addition of the $\vec{\zeta} x_1$ terms in Eq.~\ref{ewsiv3:eqn}, and
the addition of the shape measurement and uncertainty (implicit in the covariance matrix $C$) and $x_1$   in Eq.~\ref{dust3:eqn}.
As in \S\ref{modelI:sec}, the degeneracy between $\vec{\zeta}$ and $\Delta$ is fixed by enforcing zero correlation between $\Delta$ and $x_1$.
For $N$ supernovae there are $9N$ observables.  There are $4N$ spectral and light-curve shape parameters, each of
$\Delta$, $g_0$, $g_1$ contributes $N-1$ parameters while $p$  contributes $N$ parameters,  and there are $5 \times 8$ global coefficients.
For $N=172$ supernovae, this makes 1548 observables and 1413  parameters.


\subsection{Results}
\subsubsection{Analysis Method and Validation}

In the analysis of this and previous sections of this article,
the posterior of the model parameters is evaluated using Hamiltonian Monte Carlo with a No-U-Turn
Sampler as implemented in
STAN \citep{JSSv076i01}.  We run eight chains, each with 5000 iterations of which
half are used for warmup.
STAN provides output statistics to assess
the convergence of the output Markov chains.
The 
potential scale reduction statistic, $\hat{R}$
\citep[][in this paragraph not to be confused with the synthetic $\hat{R}$-band magnitude]{Gelman92}, measures the convergence of the target distribution
in iterative simulations 
by using multiple independent sequences to estimate how much that distribution would sharpen if the simulations were run longer.
$N_{\mathit{eff}}$ is an estimate of the number of independent draws. The STAN output gives $\hat{R} \sim 1.0$ for all parameters, meaning there is no evidence for non-convergence.  The
output also gives  $N_{\mathit{eff}} \gg 100$ for all parameters, indicating that they are densely sampled.


Our analysis pipeline is run with input simulated data with known  signals and measurement uncertainties similar to those of our dataset.
Stacks of the posteriors from 100 simulated datasets  exhibit no biases that are significant relative to the statistical uncertainties of a single posterior.

As will be seen in  \S\ref{results3global:sec}, the posterior derived from the analysis has a distinct bounded  
peak showing no indication that the MCMC has failed to converge.

The model provides a fair representation of the data as seen in the residual differences between the data and the model prediction.
Figure~\ref{residual:fig} shows plots of 
residuals between measured and model-expectation colors as a function of observed $\hat{B}_o-\hat{V}_o$.
This and similar plots for the other permutations of observed colors show  no apparent trend between residuals and observed
color nor any catastrophic outliers.

\begin{figure}[htbp] %  figure placement: here, top, bottom, or page
   \centering
   \includegraphics[width=4in]{cpull.pdf} 
               \caption{
               Residuals between measured and model-expectation colors as a function of observed $\hat{B}_o-\hat{V}_o$.  The colors are
               relative to on arbitrary supernova. Plots that show
               residual
               colors containing $\hat{B}_o$ or $\hat{V}_o$ are shaded to highlight that the abscissa and ordinate have contributions
               from  a common measurement.
            \label{residual:fig}}
\end{figure}

This suite of tests lend confidence in our model and the validity of the derived posterior.

\subsubsection{Global Parameter Posterior Results}
\label{results3global:sec}
Results from our analysis are shown in Figures~\ref{global1:fig} -- \ref{global5:fig}
as contours of the posterior surface for pairs of global parameters grouped by filter.
The confidence regions are localized and unimodal, except for the combination $\sigma_p \vec{\phi}$ that
has the sign degeneracy explained in \S\ref{results2:sec} and plotted in Figure~\ref{M1ev:fig}.
Within the finite number of generated links, the MCMC chains do not migrate between the parity-degenerate $\vec{\gamma}$--$g$ solutions.
Each chain converges to one of the  $\vec{\phi}$--$p$ degenerate solutions, away from which it does not depart.  

\begin{figure}[htbp] %  figure placement: here, top, bottom, or page
   \centering
   \includegraphics[width=5.2in]{coeff0.pdf} 
            \caption{Model~III posterior contours for $\vec{c}$, $\vec{\alpha}$, $\vec{\beta}$, $\vec{\eta}$, $\vec{\gamma}^0$, $\vec{\gamma}^1$, and $\sigma_p \vec{\phi}$ in the ${\hat{U}}$ band.
            The contours shown here and in future plots represent 1-$\sigma$ in the parameter distribution (i.e.\ they should be
            projected onto the corresponding 1-d parameter axis), not to 68\%, 95\%, etc.\
            enclosed probability.  Lines for zero value for $\alpha_{\hat{U}}$, $\beta_{\hat{U}}$, $\eta_{\hat{U}}$, $\gamma_{\hat{U}}^0$, $\gamma_{\hat{U}}^1$, and $\sigma_p \phi_{\hat{U}}$ are shown for reference --
            in most cases they are outside the range of the plot.
            \label{global1:fig}}
\end{figure}

\begin{figure}[htbp] %  figure placement: here, top, bottom, or page
   \centering
   \includegraphics[width=5.2in]{coeff1.pdf} 
            \caption{Model~III posterior contours for $\vec{c}$, $\vec{\alpha}$, $\vec{\beta}$, $\vec{\eta}$,  $\vec{\gamma}^0$, $\vec{\gamma}^1$, and $\sigma_p \vec{\phi}$ in the ${\hat{B}}$ band.
 \label{global2:fig}}
\end{figure}

\begin{figure}[htbp] %  figure placement: here, top, bottom, or page
   \centering
   \includegraphics[width=5.2in]{coeff2.pdf} 
            \caption{Model~III posterior contours for $\vec{c}$, $\vec{\alpha}$, $\vec{\beta}$, $\vec{\eta}$, $\vec{\gamma}^0$, $\vec{\gamma}^1$, and $\sigma_p \vec{\phi}$ in the ${\hat{V}}$ band.
 \label{global3:fig}}
\end{figure}

\begin{figure}[htbp] %  figure placement: here, top, bottom, or page
   \centering
      \includegraphics[width=5.2in]{coeff3.pdf} 
            \caption{Model~III posterior contours for  $\vec{c}$, $\vec{\alpha}$, $\vec{\beta}$, $\vec{\eta}$,  $\vec{\gamma}^0$, $\vec{\gamma}^1$, and $\sigma_p \vec{\phi}$ in the ${\hat{R}}$ band.
 \label{global4:fig}}
\end{figure}

\begin{figure}[htbp] %  figure placement: here, top, bottom, or page
   \centering
         \includegraphics[width=5.2in]{coeff4.pdf} 
            \caption{Model~III posterior contours for  $\vec{c}$, $\vec{\alpha}$, $\vec{\beta}$, $\vec{\eta}$, $\vec{\gamma}^0$, $\vec{\gamma}^1$, and $\sigma_p \vec{\phi}$ in the ${\hat{I}}$ band.
 \label{global5:fig}}
\end{figure}


For each of the five filters, the 68\%  equal-tailed credible intervals for the global parameters $\vec{\alpha}$, $\vec{\beta}$, $\vec{\eta}$, $\vec{\zeta}$,
$\vec{\gamma}^0$, $\vec{\gamma}^1$, and $\sigma_p\vec{\phi}$
are given in Table~\ref{global2:tab}.
In constructing the credible interval of $\sigma_p\vec{\phi}$ 
we set $\vec{\phi} = -\text{sign}(\phi_{\hat{V}}) \vec{\phi}$ to break its parity degeneracy;
Table~\ref{global3:tab} shows it has the largest signal-to-noise in
in the $\hat{V}$-band.
The effect of spectral parameters on color (as opposed to magnitude)
is shown in the rows of $\alpha_X/\alpha_{\hat{V}}-1$,  $\beta_X/\beta_{\hat{V}}-1$, and  $\eta_X/\eta_{\hat{V}}-1$.
The normalization freedom of $\vec{\gamma}$ is nulled out in the statistic
 $\gamma_X/\gamma_{\hat{V}}-1$.
 These statistics give the change in $X-\hat{V}$ color normalized to the change in $\hat{V}$ magnitude, and are analogous to
$R_V^{-1}=A_B/A_V-1$ as an observational descriptor of dust properties.

\begin{table}
\centering
\begin{tabular}{|c|c|c|c|c|c|}
\hline
Parameters& $X={\hat{U}}$ &${\hat{B}}$&${\hat{V}}$&${\hat{R}}$&${\hat{I}}$\\ \hline
$\alpha_X$
& $0.0051^{+0.0007}_{-0.0008}$
& $0.0025^{+0.0006}_{-0.0007}$
& $0.0022^{+0.0005}_{-0.0005}$
& $0.0019^{+0.0004}_{-0.0004}$
& $0.0034^{+0.0003}_{-0.0004}$
\\
${\alpha_X}/\alpha_{\hat{V}}-1$
& $   1.3^{+   0.4}_{  -0.2}$
& $   0.1^{+   0.1}_{  -0.1}$
& \ldots
& $  -0.1^{+   0.0}_{  -0.0}$
& $   0.5^{+   0.3}_{  -0.2}$
\\
$\beta_X$
& $ 0.035^{+ 0.005}_{-0.005}$
& $ 0.024^{+ 0.004}_{-0.004}$
& $ 0.021^{+ 0.004}_{-0.004}$
& $ 0.019^{+ 0.003}_{-0.003}$
& $ 0.010^{+ 0.003}_{-0.003}$
\\
${\beta_X}/\beta_{\hat{V}}-1$
& $  0.68^{+  0.16}_{ -0.13}$
& $  0.14^{+  0.07}_{ -0.07}$
&\ldots
& $ -0.11^{+  0.03}_{ -0.02}$
& $ -0.50^{+  0.06}_{ -0.07}$
\\
$\eta_X$
& $0.0003^{+0.0011}_{-0.0011}$
& $0.0005^{+0.0009}_{-0.0009}$
& $0.0008^{+0.0007}_{-0.0007}$
& $0.0007^{+0.0006}_{-0.0006}$
& $-0.0000^{+0.0005}_{-0.0005}$
\\
${\eta_X}/\eta_{\hat{V}}-1$
& $ -0.39^{+  0.64}_{ -1.36}$
& $ -0.26^{+  0.35}_{ -0.77}$
&\ldots
& $ -0.19^{+  0.12}_{ -0.11}$
& $ -0.86^{+  0.33}_{ -0.71}$
\\
$\zeta_X$
& $ -0.00^{+  0.04}_{ -0.04}$
& $ -0.02^{+  0.03}_{ -0.03}$
& $ -0.05^{+  0.03}_{ -0.03}$
& $ -0.03^{+  0.02}_{ -0.02}$
& $ -0.08^{+  0.02}_{ -0.02}$
\\
${\zeta_X}/\zeta_{\hat{V}}-1$
& $ -0.94^{+  0.48}_{ -1.15}$
& $ -0.54^{+  0.28}_{ -0.65}$
& \ldots
& $ -0.42^{+  0.09}_{ -0.21}$
& $  0.47^{+  0.83}_{ -0.32}$
\\
$\gamma^0_X$
& $ 68.91^{+  2.62}_{ -2.56}$
& $ 55.44^{+  2.22}_{ -2.25}$
& $ 41.41^{+  1.98}_{ -1.98}$
& $ 31.49^{+  1.61}_{ -1.62}$
& $ 22.92^{+  1.35}_{ -1.30}$
\\
${\gamma^0_X}/\gamma^0_{\hat{V}}-1$
& $  0.67^{+  0.04}_{ -0.04}$
& $  0.34^{+  0.02}_{ -0.02}$
&\ldots
& $ -0.24^{+  0.01}_{ -0.01}$
& $ -0.45^{+  0.02}_{ -0.02}$
\\
$\gamma^1_X$
& $ -1.85^{+  4.11}_{ -4.05}$
& $  1.47^{+  3.74}_{ -3.58}$
& $ -6.75^{+  3.21}_{ -3.15}$
& $ -6.60^{+  2.71}_{ -2.65}$
& $ -7.65^{+  2.33}_{ -2.24}$
\\
${\gamma^1_X}/\gamma^1_{\hat{V}}-1$
& $ -0.70^{+  0.38}_{ -0.76}$
& $ -1.18^{+  0.43}_{ -1.08}$
& \ldots
& $ -0.03^{+  0.14}_{ -0.08}$
& $  0.12^{+  0.45}_{ -0.22}$
\\
$\sigma_p \phi_X$
& $ 0.002^{+ 0.016}_{-0.017}$
& $-0.041^{+ 0.013}_{-0.014}$
& $-0.050^{+ 0.011}_{-0.012}$
& $-0.032^{+ 0.009}_{-0.010}$
& $ 0.002^{+ 0.008}_{-0.008}$
\\
${\phi_X/\phi_{\hat{V}}-1}$
& $-1.034^{+ 0.291}_{-0.409}$
& $-0.174^{+ 0.103}_{-0.134}$
& \ldots
& $-0.347^{+ 0.042}_{-0.060}$
& $-1.032^{+ 0.141}_{-0.208}$
\\

\hline
\end{tabular}
\caption{68\% credible intervals for the global fit parameters of Model~III in \S\ref{modelIII:sec}.\label{global3:tab}}
\end{table}



\subsubsection{Results for the Observables Cofactors $\vec{\alpha}$, $\vec{\beta}$, $\vec{\eta}$,  $\vec{\zeta}$}

The cofactors  $\vec{\alpha}$, $\vec{\beta}$, and $\vec{\eta}$ have  significant non-zero values,  meaning that
$EW_{\mathrm{Ca}}$, $EW_{\mathrm{Si}}$, and $\lambda_{\mathrm{Si}}$ are indicators of broadband
colors at peak.  The significance of their influence varies between  colors; in $\hat{B}-\hat{V}$, roughly the color
where previous studies have focused, only $\vec{\beta}$ has a strong $>2\sigma$ correlation.  A positive
dependence (consistent with the positive sign
of $\beta_{\hat{B}}/\beta_{\hat{V}}-1$)
of $B-V$ on the pseudo-equivalent width of Si~II~$\lambda$4130 has previously been reported by
\citet{2011ApJ...734...42N, 2013ApJ...773...53F}.
The insignificance of  our $\alpha_{\hat{B}}/\alpha_{\hat{V}}-1$ is consistent with the
lack of correlation between the  pseudo-equivalent width of Ca~II~H\&K measurements and
intrinsic color reported by \citet{2011ApJ...742...89F}.

The story on Si~II~$\lambda$6355 is not as clear.
 \citet{2009ApJ...699L.139W,2011ApJ...742...89F, 2012ApJ...748..127F} find that
higher-velocity SNe~Ia tend to be redder than those with lower velocity. \citet{2012AJ....143..126B} find a
weaker relation than that reported by \citet{2011ApJ...742...89F},
a difference attributed to a better treatment of uncertainties and host-galaxy dust. In contrast,
\citet{2013ApJ...773...53F} find no significant relation.
Our result for $\eta_{\hat{B}}/\eta_{\hat{V}}-1$ also gives no significant relation. 

The significantly non-zero values for  $\alpha_{\hat{U}}$ and $\beta_{\hat{U}}$ from this study can be compared with the $U$-band study of
\citet{2018arXiv180101834N} carried out on a subset of the same supernovae.
Both analyses find a strong correlation between $U$ and the Si and Ca features.

The non-zero values for $\vec{\zeta}$ and their effect on color $\vec{\zeta}-\zeta_{\hat{V}}$  demonstrate that light-curve shape $x_1$ is  an indicator
of broadband colors, in the sense that SNe~Ia with broader light curves are bluer in all colors.
This is as expected based on the results of \citet{2007A&A...466...11G, 2007ApJ...659..122J}.


The signal in  $\vec{\alpha}$ and $\vec{\beta}$ cannot
entirely
be attributed to the equivalent widths themselves.
The range of Si~II~$\lambda$4130 equivalent widths is $\pm 20$~\AA\ whereas the width of the ${\hat{B}}$-band is 851~\AA, so that its direct affect on magnitude
is
$2.5 \log{(20/850)} \sim 0.03$ mag.  
The implied span in ${\hat{B}}$ magnitude based on $\beta_{\hat{B}}$ is 0.54~mag.  Therefore $\beta_{\hat{B}}$ cannot wholly be attributed to the flux deficit
from the line itself.
The Ca~II H\&K equivalent widths have range $\pm 50$~\AA, while the width of the ${\hat{U}}$ band is
701~\AA, so that its direct affect on magnitude
is
$2.5 \log{(50/701)} \sim 0.08$ mags.   The implied span in ${\hat{U}}$ magnitude is  0.21~mag, 
which would imply that  $\alpha_{\hat{U}}$ cannot be completely due to the 
presence of the line itself.  
Supernova flux has a large gradient in our $\hat{U}$ and there is a variation in the effective, flux-weighted,  $\hat{U}$ bandwidth. 
Nevertheless, the  correlation between spectral features with magnitudes in bands with no wavelength overlap is a non-trivial
signal detected in the analysis.

The posterior visualized in
Figures~\ref{global1:fig} -- \ref{global5:fig} shows
 that $\vec{\zeta}$, the magnitude vector associated with $x_1$, is highly correlated with $\vec{\beta}$, the vector
 associated with  $EW_{\mathrm{Si}}$.
The direct correlation between  $x_1$ and $EW_{\mathrm{Si}}$ has already been established
\citep{2008A&A...492..535A, 2011A&A...529L...4C}
and exists
in our sample as is shown later in \S\ref{results3per:sec}.  
Nevertheless the $\vec{\zeta}$ and $\vec{\beta}$   posteriors are bounded and inconsistent with zero, indicating that
 $x_1$ does convey color information independent of our set of spectral features.
For this reason  Model~III is highlighted in this article.

\subsubsection{Results for the Latent Parameter Cofactors $\vec{\gamma}^0$, $\vec{\gamma}^1$}
\label{results3gamma:sec}
As with Model~I in \S\ref{modelI:sec}, all elements of the two $\vec{\gamma}$ parameter vectors 
are significantly non-zero with 
none of the 20000 links of 
our Monte Carlo chains for $\vec{\gamma}$ extending to 0 (see Figures~\ref{global1:fig}--\ref{global5:fig}).
The direction of the vector in the ${\hat{U}}{\hat{V}}{\hat{I}}$-subspace is shown in Figure~\ref{plane3:fig}.

The transformation between the $\vec{\gamma}$  vectors and the   \citetalias{1999PASP..111...63F} vector given in
Eq.~\ref{trans_I:eqn}, is satisfied by the transformation matrix
\begin{equation}
M=
\begin{pmatrix}
\begin{array}{rr}
40.3^{+1.8}_{-1.8} & 16.6^{+0.6}_{-0.6} \\
-5.8^{+3.1}_{-3.0} & 3.4^{+1.0}_{-0.9}
\end{array}
\end{pmatrix} 
\end{equation}
and the two residual vectors
\begin{align}
\begin{split}
\vec{\epsilon}_{\gamma^0} &=\left(1.1^{+0.4}_{-0.4} , -1.2^{+0.4}_{-0.4} , -0.9^{+0.4}_{-0.4} , 0.7^{+0.2}_{-0.2} , 0.5^{+0.6}_{-0.5}\right) \text{mag}, \\
\vec{\epsilon}_{\gamma^1} & =\left(-2.3^{+0.6}_{-0.5} , 4.0^{+0.6}_{-0.6} , -1.3^{+0.6}_{-0.6} , 0.7^{+0.3}_{-0.3} , -1.4^{+0.8}_{-0.8}\right) \text{mag}.
\end{split}
\label{res_3:eqn}
\end{align}
This corresponds to a quadratic fraction of $0.9996^{+0.0002}_{-0.0003}$ of $\vec{\gamma}^0$ that projects onto the plane of
color changes allowed by the \citetalias{1999PASP..111...63F} dust-extinction model
as described by
$\vec{a}$ and $\vec{b}$. The quadratic fraction of $\vec{\gamma}^1$ projected
onto this plane is
$0.8426^{+0.0780}_{-0.1236}$.
For the standard deviation values of $g_0$ and $g_1$ of the supernovae in our sample, the amplitude of the magnitude residuals from $\vec{\epsilon}$ are 
$\left(0.005, 0.012,0.011, 0.007, 0.004 \right)$~mag.
The   $\vec{\gamma}^0$  vector points in the $\langle R^F_{\mathit{eff}}\rangle=2.43^{+0.10}_{-0.10}$ dust-extinction direction.
This value is consistent with the $R_V=2.5$ found in the supernova samples of \citet{2011ApJ...729...55F}, and the
center  of the range of per-supernova $R^F$ values found in the samples of \citet{2014ApJ...789...32B, 2015MNRAS.453.3300A}.
Although the model $\vec{\gamma}$'s cannot be attributed to dust, they behave like the   \citetalias{1999PASP..111...63F} model.

\begin{figure}[htbp] %  figure placement: here, top, bottom, or page
   \centering
   \includegraphics[width=2.95in]{plane0.pdf}
   \includegraphics[width=2.95in]{plane1.pdf}
   \caption{
   Visualization of how supernova magnitudes can vary in Model~III and the dust extinction of \citetalias{1999PASP..111...63F}.
   See Figure~\ref{plane:fig}.
   The combination $\vec{a}+\vec{b}/2.43$ is shown in the dotted red
   line: it is almost perfectly superimposed on $\vec{\gamma}^0$.
   \label{plane3:fig}}
\end{figure}


\subsubsection{Results for the Latent Parameter Cofactor $\sigma_p \vec{\phi}$}

The  new  $p$ parameter has a significant influence on color.  The magnitude of its color effect is non-zero,
with $\sigma_p = 0.053^{+ 0.010}_{-0.005}$~mag.  The smallest value of $\sigma_p$ % new
over all Monte Carlo chains is 0.040~mag.
These numbers are significantly larger than the median data color uncertainty of 0.010~mag (the maximum is 0.042~mag).
The evaluation of 100 simulated data sets with no third color parameter produces no median $\sigma_p$'s  as
high as that of the data, so we assign a $>99$\% confidence in the detection of this third color term.

The direction of $\vec{\phi}$ is not aligned with the plane defined by
the $\vec{\gamma}$ vectors (nor of the  \citetalias{1999PASP..111...63F} dust-extinction model); a quadratic fraction of
$0.49^{+0.27}_{-0.19}$ of  $\vec{\phi}$ lies perpendicular to the $\vec{\gamma}^0$--$\vec{\gamma}^1$ plane.

The values of $\vec{\phi}$ does not change  monotonically with wavelength, as is apparent  in 
the combination $\phi_X/\phi_{\hat{V}}-1$ shown in Figure~\ref{phiratio:fig}.  An object that has fainter  $\hat{U}$ or $\hat{I}$-band peak magnitudes
has brighter  $\hat{B}$$\hat{V}$$\hat{R}$  peaks.  That brightening is non-uniform, in that
the reddening of $\hat{U}-\hat{B}$ and $\hat{B}-\hat{V}$ is accompanied
by a bluing in $\hat{V}-\hat{R}$ and  $\hat{R}-\hat{I}$. 

What is measured as $\sigma_p \vec{\phi}$ likely captures the effect of many physical processes.
To simplify the model it is assumed that the $p$'s are drawn from a Normal distribution, which may not be a fair description of what occurs in nature
and result in model bias.
The physical interpretation of this color should be tempered with caution.

\begin{figure}[htbp] %  figure placement: here, top, bottom, or page
   \centering
      \includegraphics[width=4in]{phiratio.pdf}
   \caption{Model~III median values  and corresponding 68\% intervals for $\phi_X/\phi_{\hat{V}}-1$ in the 5 bands.
   A dotted line at zero is shown for reference.
   \label{phiratio:fig}}
\end{figure}


\subsubsection{Results for the Per-Supernova Parameters}
\label{results3per:sec}
Each supernova is described by its parameters $\Delta$, $EW_{\mathrm{Ca}}$, $EW_{\mathrm{Si}}$, $\lambda_{\mathrm{Si}}$, $x_1$,
$E_{\gamma^0}({\hat{B}}-{\hat{V}})=(\gamma^0_{\hat{B}}-\gamma^0_{\hat{V}})g_0$, $E_{\gamma^1}({\hat{B}}-{\hat{V}})=(\gamma^1_{\hat{B}}-\gamma^1_{\hat{V}})g_1$, and $A_{p,V} =  \sigma_p\phi_{\hat{V}}p$.  (The  three latent parameters are cast as physically meaning
quantities by multiplying them by select cofactors.)
\added{The 68\% credible intervals for these parameters for all supernovae are given in Table~\ref{results:tab}.
\startlongtable
\begin{deluxetable}{crrrrrrrrr}
\tabletypesize{\tiny}
\tablecaption{Supernova Parameter Credible Intervals
\label{results:tab}}
\tablehead{
\colhead{Name} & \colhead{$EW_{\mathrm{Ca}}$ (\AA)} & \colhead{$EW_{\mathrm{Si}}$ (\AA)} & \colhead{$\lambda_{\mathrm{Si}}$ (\AA)} & \colhead{$x_1$} & \colhead{$\Delta$}
& \colhead{$E_{\gamma^0}({\hat{B}}-{\hat{V}})$} & \colhead{$E_{\gamma^1}({\hat{B}}-{\hat{V}})$} & \colhead{$A_{p,V} $}
}
\startdata
SN2007bd & $113.3^{+5.4}_{-5.4}$ & $ 17.8^{+0.7}_{-0.7}$ & $ 6101^{+  3}_{-  3}$ & $ -1.02^{+  0.18}_{-  0.18}$ & $-0.071^{+0.027}_{-0.027}$  & $-0.050^{+0.011}_{-0.012}$ & $-0.042^{+0.007}_{-0.007}$ & $-0.042^{+0.007}_{-0.007}$\\
PTF10zdk & $149.7^{+1.2}_{-1.2}$ & $ 14.9^{+0.6}_{-0.6}$ & $ 6150^{+  3}_{-  3}$ & $  0.37^{+  0.14}_{-  0.13}$ & $-0.029^{+0.026}_{-0.025}$  & $0.062^{+0.012}_{-0.012}$ & $-0.042^{+0.007}_{-0.007}$ & $-0.042^{+0.007}_{-0.007}$\\
SNF20080815-017 & $ 90.9^{+14.3}_{-14.5}$ & $ 19.6^{+2.4}_{-2.3}$ & $ 6131^{+  6}_{-  6}$ & $ -1.96^{+  0.16}_{-  0.16}$ & $0.041^{+0.070}_{-0.064}$  & $-0.008^{+0.025}_{-0.026}$ & $-0.011^{+0.028}_{-0.022}$ & $-0.011^{+0.028}_{-0.022}$\\
PTF09dnl & $130.0^{+0.8}_{-0.9}$ & $  9.1^{+0.7}_{-0.7}$ & $ 6092^{+  3}_{-  3}$ & $  0.59^{+  0.10}_{-  0.10}$ & $-0.061^{+0.021}_{-0.021}$  & $0.014^{+0.009}_{-0.009}$ & $0.000^{+0.011}_{-0.011}$ & $0.000^{+0.011}_{-0.011}$\\
SN2010ex & $114.5^{+0.9}_{-0.9}$ & $  8.4^{+0.4}_{-0.4}$ & $ 6129^{+  6}_{-  6}$ & $  0.17^{+  0.12}_{-  0.11}$ & $0.266^{+0.026}_{-0.026}$  & $-0.042^{+0.008}_{-0.008}$ & $-0.039^{+0.010}_{-0.008}$ & $-0.039^{+0.010}_{-0.008}$\\
PTF09dnp & $ 66.9^{+4.5}_{-4.6}$ & $ 17.5^{+0.7}_{-0.7}$ & $ 6098^{+  5}_{-  4}$ & $ -0.53^{+  0.17}_{-  0.17}$ & $-0.249^{+0.026}_{-0.026}$  & $-0.008^{+0.020}_{-0.021}$ & $-0.044^{+0.006}_{-0.006}$ & $-0.044^{+0.006}_{-0.006}$\\
PTF11bnx & $150.7^{+3.0}_{-2.9}$ & $ 14.6^{+1.0}_{-1.0}$ & $ 6143^{+  5}_{-  5}$ & $  0.07^{+  0.14}_{-  0.14}$ & $0.043^{+0.029}_{-0.029}$  & $0.049^{+0.011}_{-0.011}$ & $-0.030^{+0.014}_{-0.011}$ & $-0.030^{+0.014}_{-0.011}$\\
PTF12jqh & $151.8^{+1.4}_{-1.5}$ & $  7.9^{+0.6}_{-0.7}$ & $ 6118^{+ 10}_{- 10}$ & $  0.73^{+  0.14}_{-  0.14}$ & $0.069^{+0.028}_{-0.028}$  & $-0.056^{+0.010}_{-0.010}$ & $0.009^{+0.014}_{-0.015}$ & $0.009^{+0.014}_{-0.015}$\\
SNF20080802-006 & $108.7^{+5.6}_{-5.8}$ & $ 19.1^{+1.6}_{-1.6}$ & $ 6122^{+  5}_{-  5}$ & $ -1.61^{+  0.12}_{-  0.12}$ & $-0.042^{+0.059}_{-0.055}$  & $-0.001^{+0.016}_{-0.016}$ & $0.033^{+0.020}_{-0.020}$ & $0.033^{+0.020}_{-0.020}$\\
PTF10xyt & $120.2^{+6.3}_{-6.4}$ & $ 21.5^{+3.2}_{-3.1}$ & $ 6102^{+  4}_{-  4}$ & $  0.18^{+  0.20}_{-  0.20}$ & $0.241^{+0.049}_{-0.049}$  & $0.070^{+0.026}_{-0.028}$ & $0.014^{+0.029}_{-0.027}$ & $0.014^{+0.029}_{-0.027}$\\
PTF11qmo & $101.7^{+1.1}_{-1.1}$ & $  7.8^{+0.7}_{-0.7}$ & $ 6151^{+  7}_{-  8}$ & $  0.41^{+  0.18}_{-  0.18}$ & $-0.078^{+0.031}_{-0.028}$  & $-0.058^{+0.010}_{-0.010}$ & $0.004^{+0.014}_{-0.015}$ & $0.004^{+0.014}_{-0.015}$\\
SNF20070331-025 & $118.7^{+6.9}_{-7.0}$ & $ 15.4^{+2.0}_{-2.0}$ & $ 6121^{+ 10}_{- 10}$ & $ -1.22^{+  0.15}_{-  0.15}$ & $0.190^{+0.037}_{-0.037}$  & $-0.015^{+0.020}_{-0.020}$ & $0.001^{+0.023}_{-0.022}$ & $0.001^{+0.023}_{-0.022}$\\
SNF20070818-001 & $160.6^{+6.7}_{-6.8}$ & $ 16.4^{+1.5}_{-1.5}$ & $ 6113^{+  5}_{-  5}$ & $  0.14^{+  0.14}_{-  0.14}$ & $-0.178^{+0.021}_{-0.020}$  & $0.005^{+0.014}_{-0.014}$ & $0.014^{+0.016}_{-0.015}$ & $0.014^{+0.016}_{-0.015}$\\
SNBOSS38 & $ 57.1^{+0.4}_{-0.4}$ & $ 17.8^{+0.3}_{-0.3}$ & $ 6127^{+  3}_{-  3}$ & $  1.02^{+  0.11}_{-  0.11}$ & $0.018^{+0.038}_{-0.039}$  & $0.017^{+0.021}_{-0.023}$ & $-0.002^{+0.019}_{-0.020}$ & $-0.002^{+0.019}_{-0.020}$\\
SN2006ob & $ 75.9^{+12.1}_{-12.4}$ & $ 26.8^{+1.3}_{-1.4}$ & $ 6113^{+  5}_{-  5}$ & $ -2.26^{+  0.21}_{-  0.21}$ & $0.039^{+0.035}_{-0.035}$  & $-0.051^{+0.019}_{-0.017}$ & $0.084^{+0.021}_{-0.021}$ & $0.084^{+0.021}_{-0.021}$\\
PTF12eer & $169.4^{+8.9}_{-8.8}$ & $ 12.2^{+1.8}_{-1.8}$ & $ 6148^{+ 10}_{- 10}$ & $  0.64^{+  0.14}_{-  0.15}$ & $-0.053^{+0.028}_{-0.027}$  & $0.039^{+0.018}_{-0.019}$ & $-0.029^{+0.017}_{-0.012}$ & $-0.029^{+0.017}_{-0.012}$\\
PTF10ops & $ 38.0^{+9.2}_{-9.2}$ & $ 21.9^{+4.6}_{-4.6}$ & $ 6140^{+  6}_{-  5}$ & $ -0.58^{+  0.12}_{-  0.12}$ & $-0.091^{+0.221}_{-0.154}$  & $0.190^{+0.040}_{-0.041}$ & $0.181^{+0.038}_{-0.036}$ & $0.181^{+0.038}_{-0.036}$\\
SNF20080514-002 & $ 83.1^{+0.7}_{-0.7}$ & $ 19.2^{+0.5}_{-0.5}$ & $ 6131^{+  3}_{-  3}$ & $ -1.55^{+  0.09}_{-  0.09}$ & $0.174^{+0.019}_{-0.019}$  & $-0.074^{+0.007}_{-0.006}$ & $0.006^{+0.011}_{-0.012}$ & $0.006^{+0.011}_{-0.012}$\\
PTF12evo & $129.1^{+2.6}_{-2.7}$ & $  9.9^{+1.1}_{-1.1}$ & $ 6157^{+  4}_{-  4}$ & $  0.81^{+  0.13}_{-  0.12}$ & $0.024^{+0.027}_{-0.027}$  & $0.004^{+0.011}_{-0.011}$ & $-0.034^{+0.012}_{-0.010}$ & $-0.034^{+0.012}_{-0.010}$\\
SNF20080614-010 & $118.9^{+4.9}_{-4.8}$ & $ 28.9^{+1.2}_{-1.2}$ & $ 6129^{+  3}_{-  3}$ & $ -1.81^{+  0.15}_{-  0.15}$ & $-0.114^{+0.046}_{-0.042}$  & $-0.068^{+0.012}_{-0.009}$ & $0.049^{+0.019}_{-0.019}$ & $0.049^{+0.019}_{-0.019}$\\
PTF10icb & $104.8^{+0.9}_{-0.9}$ & $ 12.6^{+0.3}_{-0.3}$ & $ 6138^{+  2}_{-  2}$ & $  0.23^{+  0.10}_{-  0.10}$ & $0.410^{+0.024}_{-0.025}$  & $0.046^{+0.006}_{-0.006}$ & $-0.019^{+0.010}_{-0.010}$ & $-0.019^{+0.010}_{-0.010}$\\
PTF12efn & $145.5^{+3.3}_{-3.2}$ & $  5.4^{+1.4}_{-1.5}$ & $ 6115^{+  3}_{-  3}$ & $  1.04^{+  0.16}_{-  0.16}$ & $0.177^{+0.030}_{-0.029}$  & $-0.058^{+0.014}_{-0.012}$ & $0.051^{+0.017}_{-0.017}$ & $0.051^{+0.017}_{-0.017}$\\
SNNGC4424 & $109.0^{+0.3}_{-0.3}$ & $  8.6^{+0.1}_{-0.1}$ & $ 6138^{+  2}_{-  2}$ & $  0.44^{+  0.09}_{-  0.09}$ & $0.600^{+0.022}_{-0.023}$  & $0.078^{+0.007}_{-0.006}$ & $-0.025^{+0.011}_{-0.011}$ & $-0.025^{+0.011}_{-0.011}$\\
SNF20080516-022 & $ 99.1^{+2.2}_{-2.1}$ & $ 15.3^{+0.9}_{-0.9}$ & $ 6159^{+  3}_{-  3}$ & $  0.48^{+  0.12}_{-  0.11}$ & $0.001^{+0.026}_{-0.026}$  & $-0.064^{+0.010}_{-0.009}$ & $-0.025^{+0.014}_{-0.013}$ & $-0.025^{+0.014}_{-0.013}$\\
PTF12hwb & $139.4^{+25.9}_{-23.8}$ & $ -0.9^{+4.4}_{-4.3}$ & $ 6090^{+ 14}_{- 14}$ & $  1.34^{+  0.17}_{-  0.17}$ & $0.471^{+0.042}_{-0.044}$  & $0.151^{+0.037}_{-0.043}$ & $0.004^{+0.040}_{-0.032}$ & $0.004^{+0.040}_{-0.032}$\\
PTF10qyz & $108.1^{+2.1}_{-2.1}$ & $ 21.0^{+0.8}_{-0.8}$ & $ 6117^{+  5}_{-  5}$ & $ -1.67^{+  0.13}_{-  0.13}$ & $0.099^{+0.133}_{-0.095}$  & $-0.067^{+0.014}_{-0.010}$ & $-0.038^{+0.012}_{-0.008}$ & $-0.038^{+0.012}_{-0.008}$\\
SNF20060907-000 & $ 87.8^{+7.5}_{-7.3}$ & $ 15.9^{+0.9}_{-0.9}$ & $ 6150^{+  4}_{-  4}$ & $  0.16^{+  0.17}_{-  0.17}$ & $0.037^{+0.033}_{-0.033}$  & $-0.079^{+0.005}_{-0.005}$ & $-0.041^{+0.008}_{-0.007}$ & $-0.041^{+0.008}_{-0.007}$\\
LSQ12fxd & $122.8^{+1.6}_{-1.7}$ & $ 11.7^{+0.8}_{-0.7}$ & $ 6119^{+  4}_{-  4}$ & $  0.16^{+  0.12}_{-  0.12}$ & $-0.154^{+0.044}_{-0.032}$  & $-0.052^{+0.008}_{-0.008}$ & $0.002^{+0.013}_{-0.013}$ & $0.002^{+0.013}_{-0.013}$\\
SNF20080821-000 & $104.8^{+2.2}_{-2.2}$ & $  8.8^{+1.2}_{-1.2}$ & $ 6121^{+  4}_{-  4}$ & $  0.77^{+  0.15}_{-  0.14}$ & $-0.104^{+0.023}_{-0.023}$  & $0.025^{+0.011}_{-0.011}$ & $-0.006^{+0.013}_{-0.013}$ & $-0.006^{+0.013}_{-0.013}$\\
SNF20070802-000 & $157.9^{+3.3}_{-3.3}$ & $ 16.0^{+1.4}_{-1.4}$ & $ 6103^{+  5}_{-  5}$ & $  0.07^{+  0.13}_{-  0.13}$ & $-0.074^{+0.028}_{-0.028}$  & $0.005^{+0.014}_{-0.014}$ & $0.005^{+0.016}_{-0.016}$ & $0.005^{+0.016}_{-0.016}$\\
PTF10wnm & $105.5^{+2.3}_{-2.3}$ & $  6.8^{+0.9}_{-0.9}$ & $ 6125^{+  3}_{-  3}$ & $  0.73^{+  0.12}_{-  0.12}$ & $0.123^{+0.022}_{-0.021}$  & $-0.014^{+0.009}_{-0.009}$ & $0.015^{+0.012}_{-0.012}$ & $0.015^{+0.012}_{-0.012}$\\
PTF10mwb & $116.2^{+1.2}_{-1.2}$ & $ 21.1^{+0.7}_{-0.7}$ & $ 6139^{+  2}_{-  2}$ & $ -0.48^{+  0.09}_{-  0.09}$ & $-0.000^{+0.071}_{-0.070}$  & $-0.036^{+0.008}_{-0.008}$ & $-0.011^{+0.011}_{-0.011}$ & $-0.011^{+0.011}_{-0.011}$\\
SN2010dt & $133.3^{+8.4}_{-8.6}$ & $ 16.0^{+0.6}_{-0.6}$ & $ 6141^{+  5}_{-  5}$ & $ -0.02^{+  0.09}_{-  0.09}$ & $-0.046^{+0.024}_{-0.024}$  & $-0.063^{+0.010}_{-0.009}$ & $-0.043^{+0.007}_{-0.007}$ & $-0.043^{+0.007}_{-0.007}$\\
SNF20080623-001 & $149.2^{+1.4}_{-1.4}$ & $ 14.8^{+0.6}_{-0.6}$ & $ 6131^{+  3}_{-  3}$ & $ -0.07^{+  0.11}_{-  0.10}$ & $0.072^{+0.025}_{-0.025}$  & $-0.054^{+0.010}_{-0.009}$ & $-0.034^{+0.011}_{-0.010}$ & $-0.034^{+0.011}_{-0.010}$\\
LSQ12fhe & $ 42.9^{+1.2}_{-1.2}$ & $  2.3^{+1.9}_{-1.9}$ & $ 6108^{+  4}_{-  4}$ & $  1.01^{+  0.17}_{-  0.17}$ & $-0.183^{+0.032}_{-0.033}$  & $0.067^{+0.017}_{-0.018}$ & $-0.017^{+0.018}_{-0.016}$ & $-0.017^{+0.018}_{-0.016}$\\
PTF11bju & $ 31.3^{+4.3}_{-4.2}$ & $  3.3^{+2.0}_{-2.0}$ & $ 6138^{+  5}_{-  5}$ & $  1.02^{+  0.17}_{-  0.18}$ & $0.073^{+0.036}_{-0.036}$  & $0.077^{+0.019}_{-0.019}$ & $-0.014^{+0.020}_{-0.018}$ & $-0.014^{+0.020}_{-0.018}$\\
PTF09fox & $118.5^{+2.7}_{-2.7}$ & $  9.8^{+0.9}_{-0.8}$ & $ 6117^{+  3}_{-  3}$ & $  0.52^{+  0.16}_{-  0.15}$ & $0.027^{+0.039}_{-0.039}$  & $-0.047^{+0.009}_{-0.009}$ & $-0.043^{+0.007}_{-0.006}$ & $-0.043^{+0.007}_{-0.006}$\\
PTF13ayw & $103.8^{+2.4}_{-2.4}$ & $ 23.1^{+1.8}_{-1.9}$ & $ 6116^{+  6}_{-  6}$ & $ -1.67^{+  0.23}_{-  0.24}$ & $0.068^{+0.041}_{-0.041}$  & $-0.069^{+0.014}_{-0.009}$ & $0.101^{+0.019}_{-0.020}$ & $0.101^{+0.019}_{-0.020}$\\
SNF20070810-004 & $127.0^{+1.8}_{-1.8}$ & $ 20.1^{+0.9}_{-0.9}$ & $ 6117^{+  7}_{-  7}$ & $ -0.01^{+  0.11}_{-  0.11}$ & $-0.044^{+0.026}_{-0.025}$  & $-0.068^{+0.011}_{-0.009}$ & $-0.033^{+0.012}_{-0.010}$ & $-0.033^{+0.012}_{-0.010}$\\
PTF11mty & $110.5^{+2.2}_{-2.2}$ & $ 11.3^{+1.0}_{-1.1}$ & $ 6140^{+  5}_{-  5}$ & $  1.03^{+  0.13}_{-  0.13}$ & $0.062^{+0.025}_{-0.024}$  & $-0.072^{+0.009}_{-0.007}$ & $0.024^{+0.013}_{-0.014}$ & $0.024^{+0.013}_{-0.014}$\\
SNF20080512-010 & $ 96.0^{+3.4}_{-3.4}$ & $ 21.9^{+1.3}_{-1.3}$ & $ 6128^{+  5}_{-  5}$ & $ -1.64^{+  0.18}_{-  0.18}$ & $-0.032^{+0.072}_{-0.061}$  & $-0.051^{+0.013}_{-0.012}$ & $0.018^{+0.016}_{-0.016}$ & $0.018^{+0.016}_{-0.016}$\\
PTF11mkx & $ 34.8^{+3.5}_{-3.5}$ & $  5.5^{+1.3}_{-1.3}$ & $ 6169^{+  5}_{-  5}$ & $  0.53^{+  0.14}_{-  0.14}$ & $0.022^{+0.135}_{-0.080}$  & $0.102^{+0.017}_{-0.016}$ & $-0.041^{+0.008}_{-0.007}$ & $-0.041^{+0.008}_{-0.007}$\\
PTF10tce & $135.8^{+1.1}_{-1.1}$ & $ 10.6^{+1.2}_{-1.2}$ & $ 6090^{+  4}_{-  4}$ & $  0.38^{+  0.12}_{-  0.12}$ & $0.041^{+0.028}_{-0.027}$  & $-0.005^{+0.013}_{-0.014}$ & $-0.019^{+0.015}_{-0.015}$ & $-0.019^{+0.015}_{-0.015}$\\
SNF20061020-000 & $ 90.3^{+13.4}_{-13.3}$ & $ 23.9^{+0.9}_{-0.9}$ & $ 6120^{+  4}_{-  5}$ & $ -1.75^{+  0.17}_{-  0.17}$ & $-0.096^{+0.039}_{-0.034}$  & $0.000^{+0.016}_{-0.016}$ & $0.006^{+0.022}_{-0.020}$ & $0.006^{+0.022}_{-0.020}$\\
SN2005ir & $115.7^{+2.8}_{-2.7}$ & $  8.7^{+4.6}_{-4.8}$ & $ 6069^{+  5}_{-  5}$ & $  1.64^{+  0.49}_{-  0.48}$ & $0.112^{+0.042}_{-0.043}$  & $-0.022^{+0.036}_{-0.033}$ & $0.067^{+0.037}_{-0.037}$ & $0.067^{+0.037}_{-0.037}$\\
SNF20080717-000 & $ 92.7^{+2.6}_{-2.6}$ & $ 12.4^{+1.7}_{-1.7}$ & $ 6105^{+  3}_{-  3}$ & $  0.87^{+  0.13}_{-  0.12}$ & $0.115^{+0.034}_{-0.033}$  & $0.126^{+0.018}_{-0.019}$ & $0.050^{+0.021}_{-0.019}$ & $0.050^{+0.021}_{-0.019}$\\
PTF12ena & $101.0^{+1.6}_{-1.6}$ & $  8.2^{+0.9}_{-0.9}$ & $ 6130^{+  3}_{-  3}$ & $  0.76^{+  0.13}_{-  0.13}$ & $-0.039^{+0.028}_{-0.028}$  & $0.289^{+0.011}_{-0.012}$ & $-0.038^{+0.010}_{-0.008}$ & $-0.038^{+0.010}_{-0.008}$\\
PTF13anh & $167.2^{+1.8}_{-1.8}$ & $ 21.6^{+1.0}_{-1.0}$ & $ 6174^{+  3}_{-  4}$ & $ -0.38^{+  0.14}_{-  0.14}$ & $0.092^{+0.108}_{-0.067}$  & $-0.032^{+0.020}_{-0.019}$ & $-0.034^{+0.014}_{-0.010}$ & $-0.034^{+0.014}_{-0.010}$\\
CSS110918\_01 & $110.4^{+1.0}_{-1.0}$ & $  5.0^{+0.9}_{-0.9}$ & $ 6101^{+  2}_{-  2}$ & $  0.82^{+  0.13}_{-  0.13}$ & $-0.077^{+0.187}_{-0.094}$  & $-0.078^{+0.006}_{-0.005}$ & $-0.028^{+0.013}_{-0.012}$ & $-0.028^{+0.013}_{-0.012}$\\
SNF20061024-000 & $ 98.9^{+15.7}_{-16.7}$ & $ 29.1^{+1.4}_{-1.4}$ & $ 6127^{+  5}_{-  5}$ & $ -1.51^{+  0.18}_{-  0.18}$ & $0.045^{+0.062}_{-0.059}$  & $-0.049^{+0.021}_{-0.019}$ & $0.003^{+0.031}_{-0.028}$ & $0.003^{+0.031}_{-0.028}$\\
SNF20070506-006 & $ 94.3^{+1.3}_{-1.2}$ & $  6.4^{+0.6}_{-0.5}$ & $ 6152^{+  3}_{-  3}$ & $  1.02^{+  0.11}_{-  0.11}$ & $-0.075^{+0.027}_{-0.026}$  & $-0.029^{+0.009}_{-0.009}$ & $0.006^{+0.013}_{-0.013}$ & $0.006^{+0.013}_{-0.013}$\\
SNF20070403-001 & $107.8^{+5.0}_{-5.1}$ & $ 17.1^{+1.4}_{-1.4}$ & $ 6123^{+  4}_{-  4}$ & $ -0.60^{+  0.12}_{-  0.12}$ & $0.007^{+0.029}_{-0.028}$  & $-0.036^{+0.014}_{-0.014}$ & $0.020^{+0.017}_{-0.017}$ & $0.020^{+0.017}_{-0.017}$\\
PTF10hmv & $109.8^{+1.3}_{-1.4}$ & $  8.5^{+0.7}_{-0.6}$ & $ 6143^{+  3}_{-  3}$ & $  1.15^{+  0.13}_{-  0.12}$ & $0.284^{+0.024}_{-0.023}$  & $0.106^{+0.009}_{-0.009}$ & $-0.010^{+0.012}_{-0.012}$ & $-0.010^{+0.012}_{-0.012}$\\
SNF20071015-000 & $104.8^{+3.1}_{-3.1}$ & $  6.5^{+1.1}_{-1.1}$ & $ 6125^{+  7}_{-  7}$ & $  0.90^{+  0.28}_{-  0.29}$ & $0.116^{+0.040}_{-0.040}$  & $0.293^{+0.016}_{-0.017}$ & $-0.018^{+0.021}_{-0.017}$ & $-0.018^{+0.021}_{-0.017}$\\
SNhunt89 & $ 88.6^{+2.7}_{-2.7}$ & $ 29.8^{+1.7}_{-1.7}$ & $ 6109^{+  7}_{-  7}$ & $ -2.02^{+  0.16}_{-  0.16}$ & $0.202^{+0.049}_{-0.047}$  & $0.030^{+0.018}_{-0.019}$ & $0.032^{+0.027}_{-0.026}$ & $0.032^{+0.027}_{-0.026}$\\
SNF20070902-021 & $109.0^{+3.3}_{-3.3}$ & $ 16.8^{+0.9}_{-0.9}$ & $ 6131^{+  5}_{-  5}$ & $ -0.46^{+  0.09}_{-  0.09}$ & $-0.082^{+0.027}_{-0.026}$  & $-0.023^{+0.010}_{-0.010}$ & $-0.028^{+0.014}_{-0.012}$ & $-0.028^{+0.014}_{-0.012}$\\
PTF09dlc & $143.8^{+2.1}_{-2.1}$ & $  9.7^{+0.8}_{-0.8}$ & $ 6143^{+  3}_{-  3}$ & $  0.01^{+  0.10}_{-  0.10}$ & $0.008^{+0.027}_{-0.026}$  & $-0.054^{+0.012}_{-0.012}$ & $-0.017^{+0.014}_{-0.015}$ & $-0.017^{+0.014}_{-0.015}$\\
PTF13ajv & $149.9^{+8.1}_{-8.2}$ & $ 22.0^{+2.7}_{-3.0}$ & $ 6107^{+ 20}_{- 20}$ & $ -1.64^{+  0.20}_{-  0.21}$ & $0.263^{+0.046}_{-0.045}$  & $-0.062^{+0.021}_{-0.014}$ & $-0.005^{+0.023}_{-0.022}$ & $-0.005^{+0.023}_{-0.022}$\\
SNF20080919-000 & $115.6^{+2.7}_{-2.7}$ & $  8.8^{+0.9}_{-0.8}$ & $ 6144^{+  5}_{-  5}$ & $ -0.24^{+  0.10}_{-  0.10}$ & $-0.043^{+0.027}_{-0.026}$  & $0.165^{+0.013}_{-0.012}$ & $-0.035^{+0.013}_{-0.010}$ & $-0.035^{+0.013}_{-0.010}$\\
SNF20080919-001 & $ 85.0^{+1.1}_{-1.1}$ & $  6.1^{+0.4}_{-0.4}$ & $ 6149^{+  4}_{-  4}$ & $  1.05^{+  0.12}_{-  0.12}$ & $-0.183^{+0.020}_{-0.019}$  & $0.001^{+0.008}_{-0.008}$ & $0.006^{+0.012}_{-0.012}$ & $0.006^{+0.012}_{-0.012}$\\
SN2010kg & $ 99.4^{+17.2}_{-17.5}$ & $ 21.8^{+0.7}_{-0.7}$ & $ 6077^{+  5}_{-  5}$ & $ -1.87^{+  0.24}_{-  0.25}$ & $-0.073^{+0.030}_{-0.029}$  & $0.035^{+0.021}_{-0.020}$ & $0.034^{+0.020}_{-0.021}$ & $0.034^{+0.020}_{-0.021}$\\
SNF20080714-008 & $155.6^{+11.0}_{-10.8}$ & $ 16.6^{+1.9}_{-1.8}$ & $ 6099^{+  6}_{-  6}$ & $ -0.07^{+  0.14}_{-  0.14}$ & $-0.141^{+0.024}_{-0.022}$  & $0.082^{+0.018}_{-0.020}$ & $-0.029^{+0.017}_{-0.013}$ & $-0.029^{+0.017}_{-0.013}$\\
SNF20070714-007 & $132.6^{+5.5}_{-5.5}$ & $ 33.5^{+4.3}_{-4.2}$ & $ 6145^{+  4}_{-  4}$ & $ -1.09^{+  0.13}_{-  0.13}$ & $-0.094^{+0.046}_{-0.047}$  & $0.118^{+0.035}_{-0.038}$ & $0.117^{+0.036}_{-0.032}$ & $0.117^{+0.036}_{-0.032}$\\
SNF20080522-011 & $122.1^{+1.7}_{-1.7}$ & $  8.3^{+0.5}_{-0.5}$ & $ 6125^{+  2}_{-  2}$ & $  0.82^{+  0.13}_{-  0.13}$ & $-0.124^{+0.023}_{-0.023}$  & $-0.045^{+0.008}_{-0.008}$ & $-0.031^{+0.011}_{-0.010}$ & $-0.031^{+0.011}_{-0.010}$\\
SNF20061111-002 & $117.1^{+8.6}_{-8.8}$ & $ 19.8^{+0.9}_{-0.9}$ & $ 6144^{+  6}_{-  6}$ & $  0.07^{+  0.17}_{-  0.17}$ & $0.056^{+0.032}_{-0.032}$  & $-0.062^{+0.013}_{-0.011}$ & $0.026^{+0.017}_{-0.017}$ & $0.026^{+0.017}_{-0.017}$\\
SNNGC6343 & $ 86.9^{+1.5}_{-1.4}$ & $ 21.2^{+0.7}_{-0.7}$ & $ 6136^{+  3}_{-  3}$ & $ -1.49^{+  0.16}_{-  0.16}$ & $0.055^{+0.029}_{-0.030}$  & $0.036^{+0.013}_{-0.012}$ & $0.020^{+0.016}_{-0.014}$ & $0.020^{+0.016}_{-0.014}$\\
SNF20061011-005 & $120.5^{+1.1}_{-1.1}$ & $  9.2^{+0.3}_{-0.3}$ & $ 6132^{+  4}_{-  4}$ & $  0.66^{+  0.22}_{-  0.21}$ & $-0.091^{+0.043}_{-0.041}$  & $-0.073^{+0.008}_{-0.007}$ & $0.003^{+0.021}_{-0.021}$ & $0.003^{+0.021}_{-0.021}$\\
SNF20080825-010 & $ 96.4^{+10.2}_{-10.2}$ & $ 19.2^{+0.6}_{-0.6}$ & $ 6116^{+  4}_{-  4}$ & $ -0.76^{+  0.13}_{-  0.13}$ & $-0.192^{+0.020}_{-0.020}$  & $-0.045^{+0.013}_{-0.013}$ & $0.002^{+0.013}_{-0.013}$ & $0.002^{+0.013}_{-0.013}$\\
PTF10ufj & $141.4^{+3.3}_{-3.3}$ & $ 12.0^{+1.1}_{-1.0}$ & $ 6131^{+  6}_{-  5}$ & $  0.56^{+  0.13}_{-  0.13}$ & $-0.043^{+0.106}_{-0.068}$  & $-0.040^{+0.011}_{-0.011}$ & $-0.030^{+0.014}_{-0.011}$ & $-0.030^{+0.014}_{-0.011}$\\
PTF10wof & $130.5^{+2.6}_{-2.7}$ & $ 16.5^{+0.9}_{-0.9}$ & $ 6102^{+  2}_{-  2}$ & $  0.06^{+  0.11}_{-  0.11}$ & $0.096^{+0.023}_{-0.023}$  & $-0.009^{+0.011}_{-0.011}$ & $-0.013^{+0.013}_{-0.012}$ & $-0.013^{+0.013}_{-0.012}$\\
SNF20080918-000 & $146.6^{+3.5}_{-3.5}$ & $  5.6^{+2.1}_{-2.0}$ & $ 6110^{+  5}_{-  5}$ & $  1.28^{+  0.23}_{-  0.23}$ & $0.112^{+0.038}_{-0.037}$  & $0.073^{+0.019}_{-0.019}$ & $-0.004^{+0.023}_{-0.022}$ & $-0.004^{+0.023}_{-0.022}$\\
SNF20080516-000 & $117.6^{+2.1}_{-2.1}$ & $  8.2^{+1.2}_{-1.1}$ & $ 6135^{+  3}_{-  3}$ & $  0.96^{+  0.18}_{-  0.18}$ & $0.084^{+0.029}_{-0.028}$  & $-0.054^{+0.011}_{-0.010}$ & $0.004^{+0.015}_{-0.015}$ & $0.004^{+0.015}_{-0.015}$\\
SN2005cf & $158.9^{+0.7}_{-0.7}$ & $ 15.2^{+0.8}_{-0.7}$ & $ 6140^{+  3}_{-  3}$ & $ -0.19^{+  0.13}_{-  0.13}$ & $-0.122^{+0.028}_{-0.027}$  & $-0.077^{+0.007}_{-0.005}$ & $0.002^{+0.014}_{-0.015}$ & $0.002^{+0.014}_{-0.015}$\\
CSS130502\_01 & $ 87.3^{+8.0}_{-8.7}$ & $ 15.4^{+0.5}_{-0.5}$ & $ 6128^{+  3}_{-  3}$ & $ -0.15^{+  0.20}_{-  0.20}$ & $0.121^{+0.032}_{-0.032}$  & $-0.069^{+0.011}_{-0.009}$ & $-0.017^{+0.015}_{-0.015}$ & $-0.017^{+0.015}_{-0.015}$\\
SNF20080620-000 & $125.3^{+9.5}_{-9.7}$ & $ 20.6^{+0.6}_{-0.6}$ & $ 6133^{+  3}_{-  3}$ & $ -0.97^{+  0.14}_{-  0.14}$ & $-0.007^{+0.029}_{-0.028}$  & $0.001^{+0.012}_{-0.012}$ & $-0.043^{+0.007}_{-0.007}$ & $-0.043^{+0.007}_{-0.007}$\\
SNPGC51271 & $ 64.0^{+12.0}_{-13.2}$ & $ 21.4^{+0.7}_{-0.7}$ & $ 6121^{+  2}_{-  2}$ & $ -1.85^{+  0.16}_{-  0.17}$ & $0.125^{+0.034}_{-0.034}$  & $-0.057^{+0.017}_{-0.014}$ & $0.011^{+0.018}_{-0.019}$ & $0.011^{+0.018}_{-0.019}$\\
PTF11pdk & $127.7^{+2.6}_{-2.7}$ & $ 16.1^{+1.2}_{-1.2}$ & $ 6154^{+  5}_{-  5}$ & $  0.16^{+  0.16}_{-  0.17}$ & $-0.027^{+0.035}_{-0.035}$  & $-0.070^{+0.010}_{-0.008}$ & $-0.033^{+0.015}_{-0.011}$ & $-0.033^{+0.015}_{-0.011}$\\
SNF20060511-014 & $102.9^{+2.8}_{-2.8}$ & $ 17.1^{+1.0}_{-1.0}$ & $ 6144^{+  8}_{-  7}$ & $ -0.50^{+  0.15}_{-  0.15}$ & $-0.042^{+0.047}_{-0.037}$  & $-0.008^{+0.011}_{-0.010}$ & $-0.041^{+0.009}_{-0.007}$ & $-0.041^{+0.009}_{-0.007}$\\
SNF20080612-003 & $120.0^{+1.1}_{-1.1}$ & $  7.5^{+0.6}_{-0.6}$ & $ 6122^{+  3}_{-  3}$ & $  0.96^{+  0.15}_{-  0.15}$ & $-0.221^{+0.018}_{-0.017}$  & $-0.015^{+0.007}_{-0.007}$ & $-0.001^{+0.013}_{-0.014}$ & $-0.001^{+0.013}_{-0.014}$\\
SNF20080626-002 & $130.0^{+1.0}_{-1.0}$ & $  5.8^{+2.1}_{-2.0}$ & $ 6111^{+  3}_{-  3}$ & $  1.03^{+  0.19}_{-  0.19}$ & $-0.043^{+0.026}_{-0.025}$  & $-0.009^{+0.015}_{-0.016}$ & $-0.022^{+0.017}_{-0.015}$ & $-0.022^{+0.017}_{-0.015}$\\
SNF20060621-015 & $111.8^{+1.3}_{-1.3}$ & $  9.9^{+0.7}_{-0.7}$ & $ 6144^{+  3}_{-  3}$ & $  0.52^{+  0.14}_{-  0.14}$ & $-0.015^{+0.026}_{-0.027}$  & $-0.067^{+0.008}_{-0.008}$ & $-0.020^{+0.013}_{-0.013}$ & $-0.020^{+0.013}_{-0.013}$\\
SNF20080920-000 & $135.1^{+1.4}_{-1.4}$ & $  5.3^{+1.5}_{-1.5}$ & $ 6085^{+  3}_{-  3}$ & $  1.08^{+  0.23}_{-  0.23}$ & $-0.002^{+0.038}_{-0.037}$  & $-0.020^{+0.017}_{-0.017}$ & $-0.012^{+0.020}_{-0.019}$ & $-0.012^{+0.020}_{-0.019}$\\
SN2007cq & $ 63.7^{+3.9}_{-3.9}$ & $ 11.6^{+0.8}_{-0.8}$ & $ 6138^{+  3}_{-  3}$ & $  0.05^{+  0.13}_{-  0.13}$ & $-0.111^{+0.029}_{-0.027}$  & $0.008^{+0.011}_{-0.011}$ & $-0.015^{+0.015}_{-0.015}$ & $-0.015^{+0.015}_{-0.015}$\\
SNF20080918-004 & $ 91.9^{+6.6}_{-6.6}$ & $ 21.2^{+0.9}_{-0.9}$ & $ 6141^{+  4}_{-  4}$ & $ -1.37^{+  0.15}_{-  0.15}$ & $0.035^{+0.142}_{-0.087}$  & $-0.021^{+0.014}_{-0.013}$ & $-0.012^{+0.019}_{-0.018}$ & $-0.012^{+0.019}_{-0.018}$\\
CSS120424\_01 & $138.5^{+2.1}_{-2.1}$ & $ 11.5^{+0.6}_{-0.6}$ & $ 6138^{+  3}_{-  3}$ & $  0.58^{+  0.14}_{-  0.14}$ & $-0.010^{+0.028}_{-0.028}$  & $-0.061^{+0.009}_{-0.009}$ & $-0.024^{+0.013}_{-0.012}$ & $-0.024^{+0.013}_{-0.012}$\\
SNF20080610-000 & $119.5^{+8.6}_{-8.8}$ & $ 17.0^{+1.4}_{-1.4}$ & $ 6131^{+  6}_{-  6}$ & $  0.16^{+  0.14}_{-  0.14}$ & $0.013^{+0.068}_{-0.068}$  & $-0.019^{+0.014}_{-0.015}$ & $-0.027^{+0.017}_{-0.013}$ & $-0.027^{+0.017}_{-0.013}$\\
SNF20070701-005 & $101.6^{+2.6}_{-2.6}$ & $ 13.6^{+0.9}_{-1.0}$ & $ 6155^{+  5}_{-  5}$ & $  0.39^{+  0.14}_{-  0.14}$ & $-0.187^{+0.021}_{-0.021}$  & $-0.012^{+0.011}_{-0.011}$ & $0.037^{+0.015}_{-0.015}$ & $0.037^{+0.015}_{-0.015}$\\
SN2007kk & $128.6^{+1.4}_{-1.4}$ & $ 11.0^{+0.9}_{-0.9}$ & $ 6099^{+  4}_{-  4}$ & $  0.59^{+  0.14}_{-  0.14}$ & $-0.087^{+0.029}_{-0.029}$  & $-0.048^{+0.012}_{-0.012}$ & $-0.035^{+0.012}_{-0.010}$ & $-0.035^{+0.012}_{-0.010}$\\
SNF20060908-004 & $114.4^{+1.2}_{-1.2}$ & $ 12.4^{+0.6}_{-0.6}$ & $ 6136^{+  3}_{-  3}$ & $  0.63^{+  0.15}_{-  0.15}$ & $-0.099^{+0.138}_{-0.081}$  & $-0.068^{+0.009}_{-0.008}$ & $0.021^{+0.017}_{-0.017}$ & $0.021^{+0.017}_{-0.017}$\\
SNF20080909-030 & $ 93.8^{+1.0}_{-0.9}$ & $  7.7^{+0.4}_{-0.4}$ & $ 6171^{+  3}_{-  3}$ & $  0.99^{+  0.13}_{-  0.13}$ & $0.030^{+0.030}_{-0.030}$  & $0.008^{+0.013}_{-0.012}$ & $0.008^{+0.015}_{-0.014}$ & $0.008^{+0.015}_{-0.014}$\\
PTF11bgv & $ 78.3^{+3.1}_{-3.1}$ & $ 12.7^{+0.6}_{-0.6}$ & $ 6146^{+  3}_{-  3}$ & $ -0.27^{+  0.11}_{-  0.10}$ & $0.057^{+0.028}_{-0.028}$  & $0.081^{+0.010}_{-0.010}$ & $0.006^{+0.015}_{-0.015}$ & $0.006^{+0.015}_{-0.015}$\\
SNNGC2691 & $ 55.7^{+14.4}_{-14.5}$ & $  4.5^{+0.2}_{-0.2}$ & $ 6138^{+  8}_{-  8}$ & $  0.82^{+  0.16}_{-  0.16}$ & $-0.031^{+0.039}_{-0.038}$  & $0.069^{+0.018}_{-0.019}$ & $0.041^{+0.020}_{-0.020}$ & $0.041^{+0.020}_{-0.020}$\\
PTF13asv & $ 75.4^{+1.1}_{-1.1}$ & $  2.4^{+0.4}_{-0.4}$ & $ 6149^{+  4}_{-  4}$ & $  0.66^{+  0.10}_{-  0.10}$ & $-0.018^{+0.184}_{-0.105}$  & $-0.068^{+0.012}_{-0.009}$ & $0.015^{+0.015}_{-0.015}$ & $0.015^{+0.015}_{-0.015}$\\
SNF20070806-026 & $102.2^{+8.4}_{-9.0}$ & $ 25.3^{+0.7}_{-0.7}$ & $ 6112^{+  7}_{-  7}$ & $ -1.93^{+  0.14}_{-  0.14}$ & $0.010^{+0.032}_{-0.033}$  & $-0.071^{+0.011}_{-0.008}$ & $0.015^{+0.017}_{-0.017}$ & $0.015^{+0.017}_{-0.017}$\\
SNF20070427-001 & $ 80.7^{+2.3}_{-2.3}$ & $  6.5^{+0.8}_{-0.8}$ & $ 6143^{+  5}_{-  5}$ & $  0.82^{+  0.17}_{-  0.17}$ & $-0.009^{+0.031}_{-0.031}$  & $-0.066^{+0.011}_{-0.009}$ & $-0.018^{+0.017}_{-0.016}$ & $-0.018^{+0.017}_{-0.016}$\\
SNF20061108-004 & $128.2^{+5.4}_{-5.4}$ & $  6.8^{+2.1}_{-2.1}$ & $ 6110^{+  6}_{-  6}$ & $  2.36^{+  0.27}_{-  0.27}$ & $-0.089^{+0.038}_{-0.036}$  & $-0.025^{+0.021}_{-0.023}$ & $-0.005^{+0.024}_{-0.022}$ & $-0.005^{+0.024}_{-0.022}$\\
SNF20060912-000 & $106.9^{+1.7}_{-1.7}$ & $ 19.8^{+1.5}_{-1.5}$ & $ 6158^{+  7}_{-  7}$ & $  0.06^{+  0.17}_{-  0.17}$ & $-0.177^{+0.032}_{-0.033}$  & $0.034^{+0.018}_{-0.019}$ & $0.080^{+0.021}_{-0.020}$ & $0.080^{+0.021}_{-0.020}$\\
CSS110918\_02 & $109.1^{+8.1}_{-8.1}$ & $ 15.0^{+0.6}_{-0.6}$ & $ 6137^{+  3}_{-  3}$ & $ -0.26^{+  0.13}_{-  0.13}$ & $-0.052^{+0.027}_{-0.027}$  & $-0.039^{+0.010}_{-0.010}$ & $-0.009^{+0.014}_{-0.014}$ & $-0.009^{+0.014}_{-0.014}$\\
SNF20080918-002 & $ 96.6^{+2.8}_{-2.7}$ & $ 13.3^{+1.2}_{-1.2}$ & $ 6143^{+  6}_{-  6}$ & $ -0.47^{+  0.08}_{-  0.08}$ & $0.021^{+0.034}_{-0.035}$  & $-0.056^{+0.014}_{-0.013}$ & $-0.005^{+0.020}_{-0.021}$ & $-0.005^{+0.020}_{-0.021}$\\
SNIC3573 & $103.0^{+1.8}_{-1.7}$ & $ 12.1^{+0.9}_{-0.9}$ & $ 6141^{+  5}_{-  5}$ & $  0.39^{+  0.16}_{-  0.16}$ & $-0.109^{+0.026}_{-0.024}$  & $0.015^{+0.009}_{-0.010}$ & $-0.025^{+0.014}_{-0.012}$ & $-0.025^{+0.014}_{-0.012}$\\
SNF20080725-004 & $133.8^{+2.1}_{-2.1}$ & $  6.7^{+0.9}_{-0.9}$ & $ 6131^{+  6}_{-  5}$ & $  0.59^{+  0.13}_{-  0.13}$ & $0.078^{+0.032}_{-0.032}$  & $0.025^{+0.012}_{-0.011}$ & $0.016^{+0.015}_{-0.014}$ & $0.016^{+0.015}_{-0.014}$\\
SNF20050728-006 & $127.7^{+2.5}_{-2.4}$ & $ 15.3^{+1.2}_{-1.2}$ & $ 6124^{+  6}_{-  6}$ & $  0.11^{+  0.19}_{-  0.19}$ & $-0.088^{+0.034}_{-0.032}$  & $0.064^{+0.011}_{-0.012}$ & $0.012^{+0.017}_{-0.017}$ & $0.012^{+0.017}_{-0.017}$\\
SN2012fr & $134.2^{+0.5}_{-0.5}$ & $  7.4^{+0.2}_{-0.2}$ & $ 6101^{+  1}_{-  1}$ & $  0.88^{+  0.08}_{-  0.09}$ & $-0.248^{+0.021}_{-0.021}$  & $-0.077^{+0.005}_{-0.005}$ & $0.044^{+0.012}_{-0.011}$ & $0.044^{+0.012}_{-0.011}$\\
SNF20060512-002 & $100.4^{+2.7}_{-2.8}$ & $ 13.5^{+1.1}_{-1.1}$ & $ 6106^{+  7}_{-  7}$ & $ -0.87^{+  0.15}_{-  0.16}$ & $-0.186^{+0.030}_{-0.028}$  & $0.018^{+0.012}_{-0.012}$ & $0.056^{+0.017}_{-0.016}$ & $0.056^{+0.017}_{-0.016}$\\
SNF20060512-001 & $ 88.2^{+1.2}_{-1.2}$ & $  5.5^{+0.4}_{-0.4}$ & $ 6169^{+  3}_{-  3}$ & $  0.87^{+  0.11}_{-  0.11}$ & $-0.025^{+0.025}_{-0.025}$  & $0.050^{+0.011}_{-0.011}$ & $-0.010^{+0.011}_{-0.012}$ & $-0.010^{+0.011}_{-0.012}$\\
SNF20071003-016 & $126.5^{+4.5}_{-4.4}$ & $ 16.7^{+1.5}_{-1.5}$ & $ 6123^{+ 10}_{- 10}$ & $ -0.30^{+  0.14}_{-  0.14}$ & $0.058^{+0.032}_{-0.032}$  & $0.068^{+0.013}_{-0.014}$ & $-0.030^{+0.016}_{-0.012}$ & $-0.030^{+0.016}_{-0.012}$\\
SNF20050821-007 & $141.8^{+2.6}_{-2.6}$ & $  7.8^{+0.9}_{-0.9}$ & $ 6140^{+  9}_{-  8}$ & $  0.34^{+  0.20}_{-  0.19}$ & $-0.024^{+0.034}_{-0.033}$  & $-0.030^{+0.014}_{-0.013}$ & $-0.012^{+0.017}_{-0.017}$ & $-0.012^{+0.017}_{-0.017}$\\
SNF20070803-005 & $ 45.4^{+14.0}_{-13.8}$ & $  0.8^{+0.6}_{-0.6}$ & $ 6138^{+ 22}_{- 22}$ & $  0.81^{+  0.15}_{-  0.15}$ & $-0.137^{+0.031}_{-0.031}$  & $0.043^{+0.019}_{-0.019}$ & $-0.005^{+0.021}_{-0.021}$ & $-0.005^{+0.021}_{-0.021}$\\
PTF09foz & $127.2^{+1.9}_{-1.9}$ & $ 22.1^{+1.1}_{-1.1}$ & $ 6135^{+  4}_{-  4}$ & $ -0.92^{+  0.16}_{-  0.16}$ & $-0.141^{+0.019}_{-0.018}$  & $-0.042^{+0.009}_{-0.010}$ & $0.031^{+0.013}_{-0.012}$ & $0.031^{+0.013}_{-0.012}$\\
PTF12grk & $158.1^{+8.3}_{-8.4}$ & $ 22.1^{+1.2}_{-1.2}$ & $ 6089^{+  8}_{-  8}$ & $ -1.02^{+  0.12}_{-  0.12}$ & $-0.117^{+0.027}_{-0.026}$  & $-0.025^{+0.016}_{-0.017}$ & $-0.039^{+0.010}_{-0.008}$ & $-0.039^{+0.010}_{-0.008}$\\
SNF20080720-001 & $138.3^{+3.9}_{-4.0}$ & $ 14.0^{+1.5}_{-1.5}$ & $ 6107^{+  3}_{-  3}$ & $  0.19^{+  0.12}_{-  0.12}$ & $-0.176^{+0.028}_{-0.028}$  & $0.325^{+0.018}_{-0.018}$ & $0.007^{+0.019}_{-0.018}$ & $0.007^{+0.019}_{-0.018}$\\
SNF20080810-001 & $ 98.8^{+12.2}_{-12.3}$ & $ 21.7^{+1.0}_{-1.0}$ & $ 6145^{+  5}_{-  5}$ & $ -0.93^{+  0.10}_{-  0.11}$ & $-0.058^{+0.024}_{-0.024}$  & $-0.023^{+0.016}_{-0.016}$ & $-0.000^{+0.017}_{-0.016}$ & $-0.000^{+0.017}_{-0.016}$\\
SNF20050729-002 & $110.4^{+2.3}_{-2.2}$ & $ 10.7^{+1.5}_{-1.4}$ & $ 6141^{+  6}_{-  6}$ & $ -0.22^{+  0.21}_{-  0.22}$ & $0.052^{+0.102}_{-0.075}$  & $-0.039^{+0.015}_{-0.014}$ & $-0.039^{+0.011}_{-0.008}$ & $-0.039^{+0.011}_{-0.008}$\\
SN2008ec & $ 97.7^{+10.6}_{-10.6}$ & $ 23.0^{+0.4}_{-0.4}$ & $ 6125^{+  3}_{-  3}$ & $ -1.09^{+  0.10}_{-  0.10}$ & $-0.076^{+0.025}_{-0.026}$  & $0.064^{+0.012}_{-0.012}$ & $0.017^{+0.015}_{-0.015}$ & $0.017^{+0.015}_{-0.015}$\\
SNF20070902-018 & $108.6^{+9.5}_{-9.7}$ & $ 18.1^{+1.7}_{-1.7}$ & $ 6119^{+  7}_{-  7}$ & $ -0.81^{+  0.10}_{-  0.09}$ & $0.032^{+0.030}_{-0.030}$  & $0.023^{+0.016}_{-0.017}$ & $-0.035^{+0.014}_{-0.010}$ & $-0.035^{+0.014}_{-0.010}$\\
SNF20070424-003 & $123.6^{+3.6}_{-3.6}$ & $ 14.2^{+1.2}_{-1.1}$ & $ 6134^{+  6}_{-  6}$ & $  0.30^{+  0.10}_{-  0.10}$ & $0.008^{+0.023}_{-0.023}$  & $-0.010^{+0.010}_{-0.010}$ & $-0.041^{+0.008}_{-0.007}$ & $-0.041^{+0.008}_{-0.007}$\\
SN2006cj & $101.4^{+1.3}_{-1.3}$ & $  5.5^{+0.8}_{-0.8}$ & $ 6128^{+  3}_{-  3}$ & $  0.86^{+  0.13}_{-  0.13}$ & $0.028^{+0.025}_{-0.025}$  & $0.006^{+0.008}_{-0.009}$ & $-0.017^{+0.013}_{-0.013}$ & $-0.017^{+0.013}_{-0.013}$\\
SN2007nq & $ 91.2^{+9.1}_{-9.0}$ & $ 23.3^{+1.0}_{-1.0}$ & $ 6109^{+  5}_{-  5}$ & $ -1.90^{+  0.20}_{-  0.20}$ & $-0.075^{+0.038}_{-0.035}$  & $-0.022^{+0.015}_{-0.015}$ & $0.031^{+0.021}_{-0.020}$ & $0.031^{+0.021}_{-0.020}$\\
SNF20070817-003 & $ 94.5^{+2.4}_{-2.4}$ & $ 18.1^{+1.1}_{-1.1}$ & $ 6115^{+  6}_{-  6}$ & $ -0.61^{+  0.12}_{-  0.12}$ & $0.050^{+0.031}_{-0.032}$  & $-0.031^{+0.012}_{-0.012}$ & $-0.024^{+0.015}_{-0.014}$ & $-0.024^{+0.015}_{-0.014}$\\
SNF20070403-000 & $ 62.8^{+6.1}_{-6.2}$ & $ 27.6^{+1.6}_{-1.6}$ & $ 6154^{+  7}_{-  7}$ & $ -1.88^{+  0.17}_{-  0.17}$ & $0.147^{+0.040}_{-0.041}$  & $0.081^{+0.021}_{-0.021}$ & $0.041^{+0.025}_{-0.023}$ & $0.041^{+0.025}_{-0.023}$\\
SNF20061022-005 & $ 63.4^{+3.6}_{-3.8}$ & $  3.9^{+1.4}_{-1.4}$ & $ 6147^{+  7}_{-  7}$ & $  2.03^{+  0.25}_{-  0.25}$ & $0.056^{+0.038}_{-0.038}$  & $0.041^{+0.018}_{-0.020}$ & $-0.000^{+0.022}_{-0.021}$ & $-0.000^{+0.022}_{-0.021}$\\
SNNGC4076 & $127.1^{+2.4}_{-2.4}$ & $ 15.7^{+1.1}_{-1.1}$ & $ 6152^{+  4}_{-  4}$ & $  0.04^{+  0.10}_{-  0.10}$ & $-0.004^{+0.030}_{-0.029}$  & $0.049^{+0.013}_{-0.013}$ & $0.045^{+0.016}_{-0.015}$ & $0.045^{+0.016}_{-0.015}$\\
SNF20070727-016 & $ 78.3^{+2.4}_{-2.5}$ & $  4.6^{+0.8}_{-0.8}$ & $ 6139^{+  4}_{-  4}$ & $  0.16^{+  0.11}_{-  0.10}$ & $-0.097^{+0.043}_{-0.036}$  & $-0.037^{+0.011}_{-0.011}$ & $-0.025^{+0.015}_{-0.014}$ & $-0.025^{+0.015}_{-0.014}$\\
PTF12fuu & $105.4^{+3.0}_{-3.0}$ & $  6.5^{+1.0}_{-1.0}$ & $ 6124^{+  5}_{-  5}$ & $  0.37^{+  0.11}_{-  0.11}$ & $0.178^{+0.026}_{-0.026}$  & $-0.057^{+0.010}_{-0.010}$ & $-0.026^{+0.014}_{-0.013}$ & $-0.026^{+0.014}_{-0.013}$\\
SNF20070820-000 & $107.1^{+3.4}_{-3.4}$ & $ 18.7^{+1.2}_{-1.2}$ & $ 6133^{+ 13}_{- 13}$ & $ -0.28^{+  0.12}_{-  0.12}$ & $-0.054^{+0.031}_{-0.030}$  & $0.051^{+0.012}_{-0.013}$ & $-0.015^{+0.018}_{-0.016}$ & $-0.015^{+0.018}_{-0.016}$\\
SNF20070725-001 & $108.3^{+1.9}_{-1.9}$ & $  9.5^{+1.0}_{-1.1}$ & $ 6139^{+  6}_{-  6}$ & $  0.61^{+  0.15}_{-  0.15}$ & $0.058^{+0.033}_{-0.033}$  & $-0.072^{+0.010}_{-0.007}$ & $-0.018^{+0.016}_{-0.015}$ & $-0.018^{+0.016}_{-0.015}$\\
SNF20071108-021 & $ 98.7^{+2.7}_{-2.7}$ & $  6.0^{+0.8}_{-0.8}$ & $ 6165^{+  5}_{-  5}$ & $  1.15^{+  0.16}_{-  0.15}$ & $-0.045^{+0.028}_{-0.028}$  & $-0.028^{+0.012}_{-0.012}$ & $0.024^{+0.015}_{-0.014}$ & $0.024^{+0.015}_{-0.014}$\\
SNF20080914-001 & $126.4^{+1.2}_{-1.2}$ & $ 16.3^{+1.0}_{-1.0}$ & $ 6159^{+  3}_{-  3}$ & $  0.10^{+  0.14}_{-  0.14}$ & $-0.144^{+0.026}_{-0.024}$  & $0.094^{+0.012}_{-0.012}$ & $0.022^{+0.016}_{-0.015}$ & $0.022^{+0.016}_{-0.015}$\\
SNF20060609-002 & $ 86.3^{+3.5}_{-3.5}$ & $  8.1^{+1.2}_{-1.1}$ & $ 6132^{+  4}_{-  4}$ & $  0.18^{+  0.12}_{-  0.13}$ & $-0.069^{+0.033}_{-0.032}$  & $0.192^{+0.014}_{-0.014}$ & $-0.008^{+0.018}_{-0.017}$ & $-0.008^{+0.018}_{-0.017}$\\
SNF20050624-000 & $119.9^{+5.1}_{-5.1}$ & $  7.9^{+2.4}_{-2.6}$ & $ 6126^{+  5}_{-  5}$ & $  0.47^{+  0.29}_{-  0.29}$ & $-0.171^{+0.026}_{-0.024}$  & $-0.051^{+0.019}_{-0.017}$ & $0.022^{+0.019}_{-0.021}$ & $0.022^{+0.019}_{-0.021}$\\
SNF20060618-023 & $ 73.3^{+4.7}_{-4.7}$ & $  5.3^{+1.6}_{-1.5}$ & $ 6140^{+ 18}_{- 18}$ & $  0.18^{+  0.16}_{-  0.16}$ & $-0.229^{+0.039}_{-0.038}$  & $0.059^{+0.020}_{-0.020}$ & $0.092^{+0.023}_{-0.022}$ & $0.092^{+0.023}_{-0.022}$\\
SNF20080531-000 & $133.2^{+1.5}_{-1.5}$ & $ 17.2^{+0.8}_{-0.8}$ & $ 6113^{+  5}_{-  5}$ & $ -0.63^{+  0.13}_{-  0.13}$ & $0.002^{+0.024}_{-0.024}$  & $-0.040^{+0.009}_{-0.009}$ & $-0.009^{+0.012}_{-0.013}$ & $-0.009^{+0.012}_{-0.013}$\\
SN2006do & $106.3^{+2.1}_{-2.1}$ & $ 26.8^{+1.3}_{-1.2}$ & $ 6101^{+  2}_{-  2}$ & $ -1.95^{+  0.34}_{-  0.34}$ & $-0.077^{+0.031}_{-0.030}$  & $-0.038^{+0.012}_{-0.013}$ & $0.035^{+0.018}_{-0.017}$ & $0.035^{+0.018}_{-0.017}$\\
PTF12ikt & $110.3^{+1.5}_{-1.6}$ & $ 14.0^{+0.7}_{-0.7}$ & $ 6141^{+  4}_{-  4}$ & $ -0.40^{+  0.10}_{-  0.10}$ & $-0.023^{+0.025}_{-0.025}$  & $-0.034^{+0.009}_{-0.008}$ & $-0.009^{+0.014}_{-0.014}$ & $-0.009^{+0.014}_{-0.014}$\\
SN2006dm & $ 99.6^{+1.6}_{-1.6}$ & $ 29.8^{+0.7}_{-0.7}$ & $ 6118^{+  3}_{-  3}$ & $ -2.35^{+  0.18}_{-  0.18}$ & $0.084^{+0.029}_{-0.029}$  & $-0.047^{+0.011}_{-0.011}$ & $0.047^{+0.017}_{-0.016}$ & $0.047^{+0.017}_{-0.016}$\\
PTF13azs & $140.6^{+4.8}_{-4.8}$ & $ 15.2^{+1.3}_{-1.3}$ & $ 6124^{+ 10}_{- 10}$ & $ -0.12^{+  0.11}_{-  0.11}$ & $0.148^{+0.029}_{-0.029}$  & $0.195^{+0.012}_{-0.013}$ & $-0.037^{+0.012}_{-0.009}$ & $-0.037^{+0.012}_{-0.009}$\\
SN2005hj & $ 80.4^{+2.4}_{-2.4}$ & $  4.4^{+0.8}_{-0.8}$ & $ 6139^{+  4}_{-  4}$ & $  2.58^{+  0.24}_{-  0.24}$ & $0.076^{+0.039}_{-0.037}$  & $0.009^{+0.018}_{-0.020}$ & $0.066^{+0.022}_{-0.022}$ & $0.066^{+0.022}_{-0.022}$\\
PTF12iiq & $149.1^{+2.1}_{-2.2}$ & $ 24.8^{+0.8}_{-0.7}$ & $ 6048^{+  7}_{-  7}$ & $ -0.83^{+  0.10}_{-  0.11}$ & $0.005^{+0.033}_{-0.033}$  & $-0.005^{+0.019}_{-0.020}$ & $-0.034^{+0.017}_{-0.010}$ & $-0.034^{+0.017}_{-0.010}$\\
PTF10ndc & $124.1^{+2.4}_{-2.4}$ & $  6.7^{+1.0}_{-1.0}$ & $ 6119^{+  3}_{-  3}$ & $  0.86^{+  0.14}_{-  0.14}$ & $0.007^{+0.026}_{-0.025}$  & $-0.040^{+0.011}_{-0.011}$ & $-0.023^{+0.014}_{-0.013}$ & $-0.023^{+0.014}_{-0.013}$\\
SNF20080919-002 & $100.4^{+6.7}_{-6.6}$ & $ 26.3^{+1.7}_{-1.7}$ & $ 6133^{+  8}_{-  8}$ & $ -1.54^{+  0.13}_{-  0.13}$ & $0.340^{+0.043}_{-0.044}$  & $-0.058^{+0.017}_{-0.014}$ & $0.114^{+0.020}_{-0.021}$ & $0.114^{+0.020}_{-0.021}$\\
SNPGC027923 & $ 85.4^{+0.6}_{-0.6}$ & $  5.8^{+0.3}_{-0.3}$ & $ 6130^{+  4}_{-  4}$ & $  0.45^{+  0.17}_{-  0.17}$ & $0.078^{+0.032}_{-0.031}$  & $-0.077^{+0.006}_{-0.005}$ & $-0.026^{+0.015}_{-0.013}$ & $-0.026^{+0.015}_{-0.013}$\\
SNF20070330-024 & $118.3^{+2.1}_{-2.1}$ & $  5.5^{+1.8}_{-1.8}$ & $ 6101^{+  3}_{-  3}$ & $  0.48^{+  0.19}_{-  0.21}$ & $-0.166^{+0.025}_{-0.024}$  & $-0.040^{+0.014}_{-0.014}$ & $-0.018^{+0.017}_{-0.016}$ & $-0.018^{+0.017}_{-0.016}$\\
SNF20061030-010 & $131.6^{+2.2}_{-2.2}$ & $ 16.9^{+1.1}_{-1.1}$ & $ 6116^{+  4}_{-  4}$ & $ -0.10^{+  0.20}_{-  0.20}$ & $0.103^{+0.041}_{-0.041}$  & $0.049^{+0.012}_{-0.012}$ & $0.019^{+0.021}_{-0.021}$ & $0.019^{+0.021}_{-0.021}$\\
SNhunt46 & $ 93.8^{+2.0}_{-2.0}$ & $ 11.2^{+0.6}_{-0.6}$ & $ 6133^{+  4}_{-  4}$ & $  0.22^{+  0.17}_{-  0.17}$ & $0.150^{+0.032}_{-0.032}$  & $-0.062^{+0.010}_{-0.010}$ & $0.035^{+0.016}_{-0.015}$ & $0.035^{+0.016}_{-0.015}$\\
SN2005hc & $127.2^{+2.4}_{-2.5}$ & $  9.7^{+0.7}_{-0.7}$ & $ 6122^{+  3}_{-  3}$ & $  0.47^{+  0.16}_{-  0.15}$ & $0.137^{+0.025}_{-0.025}$  & $-0.062^{+0.008}_{-0.008}$ & $0.001^{+0.012}_{-0.012}$ & $0.001^{+0.012}_{-0.012}$\\
LSQ12dbr & $106.9^{+0.6}_{-0.6}$ & $  6.4^{+0.6}_{-0.6}$ & $ 6137^{+  4}_{-  4}$ & $  0.86^{+  0.11}_{-  0.11}$ & $0.004^{+0.226}_{-0.109}$  & $-0.071^{+0.008}_{-0.007}$ & $-0.034^{+0.011}_{-0.010}$ & $-0.034^{+0.011}_{-0.010}$\\
LSQ12hjm & $ 65.3^{+10.8}_{-11.3}$ & $ 15.3^{+1.1}_{-1.1}$ & $ 6147^{+  5}_{-  5}$ & $ -0.16^{+  0.17}_{-  0.17}$ & $0.112^{+0.030}_{-0.031}$  & $-0.062^{+0.013}_{-0.011}$ & $-0.042^{+0.008}_{-0.007}$ & $-0.042^{+0.008}_{-0.007}$\\
SNF20060521-001 & $ 89.1^{+11.4}_{-12.2}$ & $ 19.1^{+1.1}_{-1.1}$ & $ 6119^{+ 10}_{-  9}$ & $ -1.70^{+  0.20}_{-  0.20}$ & $0.029^{+0.061}_{-0.059}$  & $-0.068^{+0.014}_{-0.010}$ & $0.010^{+0.019}_{-0.020}$ & $0.010^{+0.019}_{-0.020}$\\
SNF20070630-006 & $126.3^{+3.1}_{-3.1}$ & $ 10.3^{+1.1}_{-1.2}$ & $ 6126^{+  4}_{-  4}$ & $  0.30^{+  0.11}_{-  0.11}$ & $-0.018^{+0.025}_{-0.025}$  & $-0.027^{+0.010}_{-0.010}$ & $-0.040^{+0.010}_{-0.008}$ & $-0.040^{+0.010}_{-0.008}$\\
PTF11drz & $132.6^{+1.4}_{-1.4}$ & $ 14.9^{+0.9}_{-0.9}$ & $ 6116^{+  4}_{-  4}$ & $ -0.11^{+  0.17}_{-  0.18}$ & $-0.007^{+0.025}_{-0.026}$  & $-0.021^{+0.010}_{-0.009}$ & $-0.016^{+0.013}_{-0.013}$ & $-0.016^{+0.013}_{-0.013}$\\
SNF20080323-009 & $ 96.0^{+2.2}_{-2.2}$ & $ 11.1^{+1.0}_{-1.0}$ & $ 6145^{+  5}_{-  5}$ & $ -0.15^{+  0.14}_{-  0.14}$ & $-0.029^{+0.030}_{-0.029}$  & $-0.051^{+0.011}_{-0.011}$ & $-0.041^{+0.009}_{-0.008}$ & $-0.041^{+0.009}_{-0.008}$\\
SNF20071021-000 & $167.9^{+2.2}_{-2.2}$ & $ 20.2^{+0.6}_{-0.6}$ & $ 6111^{+  4}_{-  4}$ & $ -0.66^{+  0.13}_{-  0.13}$ & $-0.105^{+0.025}_{-0.023}$  & $-0.001^{+0.010}_{-0.010}$ & $-0.004^{+0.013}_{-0.013}$ & $-0.004^{+0.013}_{-0.013}$\\
SNNGC0927 & $155.2^{+1.3}_{-1.3}$ & $ 10.9^{+0.7}_{-0.7}$ & $ 6109^{+  4}_{-  4}$ & $  0.39^{+  0.12}_{-  0.12}$ & $-0.031^{+0.027}_{-0.027}$  & $0.039^{+0.010}_{-0.010}$ & $-0.018^{+0.014}_{-0.013}$ & $-0.018^{+0.014}_{-0.013}$\\
SNF20060526-003 & $111.8^{+2.5}_{-2.5}$ & $  9.9^{+1.0}_{-1.0}$ & $ 6121^{+  3}_{-  3}$ & $  0.34^{+  0.13}_{-  0.13}$ & $-0.004^{+0.026}_{-0.027}$  & $-0.010^{+0.010}_{-0.010}$ & $-0.011^{+0.014}_{-0.014}$ & $-0.011^{+0.014}_{-0.014}$\\
SNF20080806-002 & $135.4^{+1.8}_{-1.9}$ & $  8.3^{+0.9}_{-0.9}$ & $ 6136^{+  4}_{-  4}$ & $  0.83^{+  0.14}_{-  0.14}$ & $-0.106^{+0.028}_{-0.027}$  & $0.020^{+0.010}_{-0.010}$ & $-0.036^{+0.013}_{-0.009}$ & $-0.036^{+0.013}_{-0.009}$\\
SNF20080803-000 & $118.3^{+2.5}_{-2.6}$ & $  8.4^{+1.3}_{-1.4}$ & $ 6125^{+  4}_{-  4}$ & $  0.25^{+  0.12}_{-  0.12}$ & $-0.030^{+0.023}_{-0.024}$  & $0.100^{+0.011}_{-0.011}$ & $-0.040^{+0.009}_{-0.008}$ & $-0.040^{+0.009}_{-0.008}$\\
SNF20080822-005 & $ 78.6^{+1.8}_{-1.8}$ & $  6.2^{+0.8}_{-0.8}$ & $ 6138^{+  4}_{-  4}$ & $  0.51^{+  0.16}_{-  0.16}$ & $-0.027^{+0.030}_{-0.030}$  & $-0.018^{+0.011}_{-0.010}$ & $0.014^{+0.015}_{-0.015}$ & $0.014^{+0.015}_{-0.015}$\\
SNF20060618-014 & $136.8^{+2.4}_{-2.5}$ & $  9.8^{+1.1}_{-1.1}$ & $ 6113^{+  7}_{-  7}$ & $  1.20^{+  0.18}_{-  0.18}$ & $0.077^{+0.047}_{-0.047}$  & $-0.037^{+0.014}_{-0.014}$ & $0.019^{+0.025}_{-0.024}$ & $0.019^{+0.025}_{-0.024}$\\
PTF12ghy & $100.6^{+3.5}_{-3.5}$ & $ 16.4^{+0.7}_{-0.7}$ & $ 6133^{+  3}_{-  3}$ & $ -0.58^{+  0.10}_{-  0.10}$ & $-0.138^{+0.023}_{-0.022}$  & $0.196^{+0.011}_{-0.011}$ & $0.001^{+0.013}_{-0.013}$ & $0.001^{+0.013}_{-0.013}$\\
SNF20070531-011 & $122.2^{+2.6}_{-2.6}$ & $ 21.5^{+0.8}_{-0.8}$ & $ 6114^{+  3}_{-  4}$ & $ -1.56^{+  0.20}_{-  0.20}$ & $-0.081^{+0.027}_{-0.025}$  & $-0.032^{+0.011}_{-0.011}$ & $-0.000^{+0.015}_{-0.014}$ & $-0.000^{+0.015}_{-0.014}$\\
SNF20070831-015 & $112.9^{+2.6}_{-2.6}$ & $  6.9^{+0.9}_{-0.9}$ & $ 6144^{+  6}_{-  6}$ & $  1.38^{+  0.16}_{-  0.16}$ & $0.042^{+0.027}_{-0.027}$  & $-0.016^{+0.010}_{-0.010}$ & $0.002^{+0.014}_{-0.014}$ & $0.002^{+0.014}_{-0.014}$\\
SNF20070417-002 & $103.8^{+5.0}_{-4.9}$ & $ 24.1^{+1.3}_{-1.3}$ & $ 6125^{+  8}_{-  9}$ & $ -1.37^{+  0.10}_{-  0.10}$ & $-0.032^{+0.050}_{-0.044}$  & $-0.072^{+0.010}_{-0.007}$ & $-0.039^{+0.010}_{-0.008}$ & $-0.039^{+0.010}_{-0.008}$\\
PTF11cao & $143.4^{+1.5}_{-1.6}$ & $ 19.0^{+1.1}_{-1.1}$ & $ 6104^{+  4}_{-  4}$ & $  0.20^{+  0.13}_{-  0.13}$ & $-0.014^{+0.032}_{-0.031}$  & $0.009^{+0.013}_{-0.013}$ & $-0.017^{+0.017}_{-0.015}$ & $-0.017^{+0.017}_{-0.015}$\\
SNF20080522-000 & $ 62.1^{+3.5}_{-3.5}$ & $  3.1^{+0.8}_{-0.8}$ & $ 6130^{+  7}_{-  7}$ & $  0.99^{+  0.11}_{-  0.11}$ & $-0.021^{+0.028}_{-0.028}$  & $-0.009^{+0.012}_{-0.013}$ & $0.047^{+0.014}_{-0.013}$ & $0.047^{+0.014}_{-0.013}$\\
PTF10qjq & $ 74.4^{+2.3}_{-2.4}$ & $ 12.0^{+0.7}_{-0.7}$ & $ 6132^{+  3}_{-  3}$ & $ -0.50^{+  0.10}_{-  0.09}$ & $-0.127^{+0.026}_{-0.024}$  & $0.049^{+0.010}_{-0.011}$ & $-0.010^{+0.015}_{-0.015}$ & $-0.010^{+0.015}_{-0.015}$\\
PTF12dxm & $119.3^{+15.5}_{-16.8}$ & $ 31.3^{+1.9}_{-2.1}$ & $ 6136^{+  4}_{-  4}$ & $ -2.14^{+  0.18}_{-  0.18}$ & $0.034^{+0.034}_{-0.033}$  & $-0.047^{+0.027}_{-0.021}$ & $0.067^{+0.023}_{-0.024}$ & $0.067^{+0.023}_{-0.024}$\\
SNF20061021-003 & $122.5^{+2.2}_{-2.2}$ & $ 10.8^{+1.5}_{-1.5}$ & $ 6131^{+  4}_{-  4}$ & $  0.39^{+  0.15}_{-  0.15}$ & $-0.012^{+0.035}_{-0.033}$  & $0.036^{+0.014}_{-0.014}$ & $0.047^{+0.019}_{-0.017}$ & $0.047^{+0.019}_{-0.017}$\\
SNF20080510-005 & $112.2^{+2.5}_{-2.5}$ & $  7.0^{+1.0}_{-1.0}$ & $ 6117^{+  4}_{-  4}$ & $  1.68^{+  0.22}_{-  0.22}$ & $0.116^{+0.030}_{-0.030}$  & $-0.029^{+0.013}_{-0.015}$ & $-0.042^{+0.008}_{-0.007}$ & $-0.042^{+0.008}_{-0.007}$\\
SNF20080507-000 & $ 98.4^{+1.6}_{-1.6}$ & $  8.0^{+1.7}_{-1.6}$ & $ 6142^{+  5}_{-  5}$ & $  0.56^{+  0.18}_{-  0.18}$ & $-0.126^{+0.023}_{-0.022}$  & $0.063^{+0.014}_{-0.013}$ & $-0.003^{+0.015}_{-0.014}$ & $-0.003^{+0.015}_{-0.014}$\\
SNF20080913-031 & $117.9^{+1.5}_{-1.5}$ & $ 14.7^{+1.6}_{-1.6}$ & $ 6159^{+  5}_{-  5}$ & $  0.49^{+  0.17}_{-  0.17}$ & $-0.048^{+0.067}_{-0.052}$  & $-0.003^{+0.015}_{-0.015}$ & $-0.009^{+0.020}_{-0.018}$ & $-0.009^{+0.020}_{-0.018}$\\
SNF20080510-001 & $118.9^{+2.0}_{-2.1}$ & $ 15.0^{+1.1}_{-1.0}$ & $ 6115^{+  4}_{-  4}$ & $  0.34^{+  0.14}_{-  0.14}$ & $0.001^{+0.026}_{-0.026}$  & $-0.058^{+0.011}_{-0.012}$ & $-0.032^{+0.012}_{-0.011}$ & $-0.032^{+0.012}_{-0.011}$\\
SNF20070712-003 & $108.1^{+2.7}_{-2.6}$ & $ 14.1^{+0.9}_{-0.9}$ & $ 6157^{+  6}_{-  6}$ & $  0.62^{+  0.14}_{-  0.14}$ & $-0.034^{+0.030}_{-0.029}$  & $-0.052^{+0.010}_{-0.011}$ & $-0.021^{+0.015}_{-0.014}$ & $-0.021^{+0.015}_{-0.014}$\\
\enddata
\end{deluxetable}
}




The distributions of the Monte Carlo links for all supernovae are shown in Figure~\ref{perobject3:fig}.
Each supernova is represented by a cloud of its parameters' links.
There is a core concentration in the  parameter-space, with around eight objects that occupy its outskirts.
Many outliers appear in the red tail of $E_{\gamma^0}({\hat{B}}-{\hat{V}})$, as would be expected for the (infrequent) selection of supernovae
heavily extinguished by host-galaxy dust.

The Pearson correlation coefficients of the per-supernova parameters are given in the matrix
\begin{multline}
Cor(EW_{\mathrm{Ca}}, EW_{\mathrm{Si}}, \lambda_{\mathrm{Si}}, x_1, E_{\gamma^0}({\hat{B}}-{\hat{V}}), E_{\gamma^1}({\hat{B}}-{\hat{V}}), A_{p,V}) =\\
\begin{pmatrix}
\begin{array}{rrrrrrr}
EW_{\mathrm{Ca}} & 0.10^{+0.03}_{-0.03} & -0.25^{+0.03}_{-0.03} & 0.07^{+0.03}_{-0.03} & -0.10^{+0.05}_{-0.05} & -0.18^{+0.06}_{-0.06} & 0.00^{+0.08}_{-0.08} \\
\ldots &EW_{\mathrm{Si}} & -0.12^{+0.03}_{-0.03} & -0.84^{+0.01}_{-0.01} & -0.13^{+0.04}_{-0.04} & 0.27^{+0.08}_{-0.09} & -0.00^{+0.08}_{-0.08} \\
\ldots & \ldots& \lambda_{\mathrm{Si}} & 0.14^{+0.03}_{-0.03} & 0.01^{+0.06}_{-0.05} & 0.00^{+0.06}_{-0.06} & -0.00^{+0.08}_{-0.08} \\
\ldots& \ldots& \ldots& x_1 & 0.13^{+0.04}_{-0.04} & -0.20^{+0.09}_{-0.09} & -0.00^{+0.08}_{-0.08} \\
\ldots& \ldots& \ldots & \ldots& E_{\gamma^0}({\hat{B}}-{\hat{V}})& 0.04^{+0.06}_{-0.06} & 0.00^{+0.08}_{-0.08} \\
\ldots & \ldots & \ldots& \ldots&\ldots & E_{\gamma^1}({\hat{B}}-{\hat{V}})  & 0.00^{+0.08}_{-0.08} \\
\ldots & \ldots & \ldots&\ldots &\ldots &\ldots &  A_{p,V} \\
\end{array}
\end{pmatrix}.
\label{corr:eqn}
\end{multline}
Recall that the $\Delta$-cofactor degeneracy is broken by enforcing zero correlation between $\Delta$ and the features.
To aid readability, only the upper triangular part of the otherwise symmetric matrix is shown and the corresponding feature label is given on
the diagonal.

\begin{figure}[htbp] %  figure placement: here, top, bottom, or page
   \centering
   \includegraphics[width=5.2in]{perobject_corner.pdf} 
   \caption{
   Distributions for the supernova parameters $\Delta$, $EW_{\mathrm{Ca}}$, $EW_{\mathrm{Si}}$, $\lambda_{\mathrm{Si}}$, $x_1$, $E_{\gamma^0}({\hat{B}}-{\hat{V}})$,  $E_{\gamma^1}({\hat{B}}-{\hat{V}})$,  and  $A_{p,V}$, as well as the grey offset
$\Delta$.  All Monte Carlo links are plotted, so that each supernova contributes a cloud of points.
   \label{perobject3:fig}}
\end{figure}

\added{
The correlation matrix in Eq.~\ref{corr:eqn} is not close to being singular, meaning that the each of the features conveys
independent supernova variability.  Nevertheless, a strong correlation between two features means that the determination
of one would reduce the residual dispersion in the other.  The amount of residual dispersion is conveyed in the matrix
\begin{equation}
Z =
\begin{pmatrix}
\scriptsize
\begin{array}{rrrrrrr}
27.589  \pm 0.554 & 27.451  \pm 0.572 & 26.668  \pm 0.589 & 27.509  \pm 0.542 & 27.408  \pm 0.563 & 27.099  \pm 0.640 & 27.506  \pm 0.562 \\
7.041  \pm 0.160 & 7.076  \pm 0.157 & 7.019  \pm 0.165 & 3.860  \pm 0.186 & 7.011  \pm 0.158 & 6.790  \pm 0.206 & 7.056  \pm 0.159 \\
19.779  \pm 0.449 & 20.295  \pm 0.458 & 20.463  \pm 0.460 & 20.267  \pm 0.456 & 20.430  \pm 0.461 & 20.423  \pm 0.462 & 20.404  \pm 0.465 \\
0.992  \pm 0.018 & 0.543  \pm 0.023 & 0.986  \pm 0.019 & 0.995  \pm 0.018 & 0.986  \pm 0.019 & 0.971  \pm 0.026 & 0.992  \pm 0.019 \\
0.073  \pm 0.003 & 0.073  \pm 0.003 & 0.074  \pm 0.003 & 0.073  \pm 0.003 & 0.074  \pm 0.003 & 0.074  \pm 0.003 & 0.074  \pm 0.003 \\
0.038  \pm 0.004 & 0.037  \pm 0.004 & 0.039  \pm 0.004 & 0.038  \pm 0.004 & 0.039  \pm 0.004 & 0.039  \pm 0.004 & 0.039  \pm 0.004 \\
0.050  \pm 0.011 & 0.050  \pm 0.011 & 0.050  \pm 0.011 & 0.050  \pm 0.011 & 0.050  \pm 0.011 & 0.050  \pm 0.011 & 0.050  \pm 0.011 
\\\end{array}
\end{pmatrix},
\label{ellipse:eqn}
\end{equation}
where the associated units are [\AA, \AA, \AA, 1, mag, mag, mag]$\cdot$[\AA, \AA, \AA, 1, mag, mag, mag]$^{T}$.

For indices $i, j \in \{EW_{\mathrm{Ca}}, EW_{\mathrm{Si}}, \lambda_{\mathrm{Si}}, x_1, E_{\gamma^0}({\hat{B}}-{\hat{V}}), E_{\gamma^1}({\hat{B}}-{\hat{V}}), A_{p,V}\}$, $M_{ii}$ is the standard deviation of the parameter $i$ distribution, and for $i \ne j$ $Z_{ij}$ is the standard deviation in parameter $i$
if parameter $j$ were known perfectly.  For example, for $i=EW_{\mathrm{Si}}$ the standard deviation is 7.08\AA\ but when $j=x_1$
is fixed its
standard deviation drops to 3.86\AA.  Therefore, an analysis that considers $x_1$ but not $EW_{\mathrm{Si}}$ would not account for
3.86\AA$\vec{\beta}=\{0.14 ,  0.09,  0.08,  0.07,  0.04 \}$~mag ${\hat{U}}{\hat{B}}{\hat{V}}{\hat{R}}{\hat{I}}$ magnitude dispersions
that are not intrinsically random.
}

An extensive discussion of the correlations between the spectral features  and light-curve parameters of the SNfactory data
set can be found in \citet{chotard:thesis, leget:thesis, 2017Chotard}. 
We here confine ourselves to noting the strong anti-correlation between $EW_{\mathrm{Si}}$ and $x_1$, which was alluded
to in \S\ref{results3global:sec} when interpreting the correlation between the cofactors $\vec{\beta}$ and $\vec{\zeta}$ that connect them with magnitudes.

The $\Delta$ parameter will be discussed in its own subsection \S\ref{Delta:sec}.

\subsubsection{Results for the Latent  Parameters $g_0$, $g_1$}
\label{g0g1:sec}
The vector
$\vec{\gamma}^0$ describes color changes for an effective $\langle R^F_{\mathit{eff}}\rangle=2.43$ dust, as shown 
in  \S\ref{results3gamma:sec}.
Hence, the per-supernova parameter $\gamma^0_{\hat{V}} g_0$ is associated with the $\hat{V}$-band
extinction for a constant $R^F=2.43$.
Figure~\ref{k0_med:fig} shows the histogram of
per-supernova
median values
from the Monte Carlo chains
and the  stack of the posteriors of all supernovae  for $\gamma^0_{\hat{V}} g_0$ 
relative to that of an arbitrary supernova  $\gamma^0_{\hat{V}} g_0|_0$.
The distributions are non-Gaussian, having a sharp rise in the blue and an extended tail in the red.  This is consistent
with
simulations based on expected dust distributions
within galaxies and the distribution of galaxy orientations with respect to the observer,
which determine column densities along the lines of sight toward supernovae that are consistent with observations
\citep{1998ApJ...502..177H, 2007ApJ...659..122J}.  
We emphasize that unlike other analyses,
this distribution is given by the data, and so is not dependent on any knowledge or prior
for the distribution of dust extinction the SN population is expected to suffer. 
There is no sign of two populations in $\gamma^1_{\hat{V}} g_1$, which would be expected from a subpopulation
experiencing extinction and scattering by circumstellar dust.

\begin{figure}[htbp] %  figure placement: here, top, bottom, or page
   \centering
   \includegraphics[width=2.8in]{deltagamma0_med.pdf}
   \includegraphics[width=2.8in]{deltagamma1_med.pdf}
      \caption{
      Normalized stack of the posteriors (includes parameter uncertainty) of all supernovae  and histogram 
      of  per-supernova 
median values of: (left) $\gamma^0_{\hat{V}} g_0-\gamma^0_{\hat{V}} g_0|_0$, which is associated with the $A^F_V$ of an $\langle R^F_{\mathit{eff}}\rangle \sim 2.43$ dust model;
(right) $\gamma^1_{\hat{V}} g_1-\gamma^1_{\hat{V}} g_1|_0$,
which  is associated with extinction corrections due to deviations away from the canonical $\langle R^F_{\mathit{eff}}\rangle$ value.
   \label{k0_med:fig}}
\end{figure}

An instructive way to consider the  $\{g_0, g_1\}$ parameters is
to transform them to $\{A_V^F, E^F(B-V)\}$ (modulo additive constants)   using the
matrix $M$ in  Eq.~\ref{trans_I:eqn}.
A plot showing the expected values and 68\% credible intervals of these parameters
for our supernova
sample is shown in Figure~\ref{kk:fig}. 
For reference, a line that represents $\langle R^F_{\mathit{eff}}\rangle=2.43$ is overplotted; points above
the line have larger $R^F$, points below have smaller $R^F$.  The majority of supernovae lie within a narrow range above the line,
while the remaining fraction fall in a broader range below the line.
Comparison with  $\gamma^1_{\hat{V}} g_1-\gamma^1_{\hat{V}} g_1|_0$ of the right plot of Figure~\ref{k0_med:fig}
shows that the 
negative tail corresponds to smaller $R^F$, and the sharper positive edge to larger $R^F$.
These findings are qualitatively consistent with previous results:
using fixed dust-extinction models,
\citet{2014ApJ...789...32B, 2015MNRAS.453.3300A} deduce a wide range of dust behavior $1.5<R^F<3$ encountered by the SN~Ia population.
Our model cannot provide the range of $R^F$ in our sample nor per-supernova determinations of $R^F$ 
without further assumptions.

\begin{figure}[htbp] %  figure placement: here, top, bottom, or page
   \centering
   \includegraphics[width=4in]{avebv_synth.pdf}
      \caption{
      Expected values and 68\% credible intervals of effective $E^F(B-V)$ and $A_V^F$ after transformation from our model $g_0$ and $g_1$ parameters, for the supernova in our sample.
      Overplotted is a line with the slope expected for $R^F=2.43$.
   \label{kk:fig}}
\end{figure}



There are $>2 \sigma$ correlations between our  color-excess parameters $g_0$, $g_1$
and the input features  $EW_{\mathrm{Ca}}$,
$EW_{\mathrm{Si}}$, and $x_1$. 
Given the association
of  $g$ with dust-extinction parameters, this implies that the host environment and the input features are connected.
The correlations between $EW_{\mathrm{Si}}$ and  $EW_{\mathrm{Ca}}$  with host have already been noted \citep{2011ApJ...734...42N, 2015MNRAS.451.1973S}.
There is a lack of correlation between  $g_0$, $g_1$ and  our velocity parameter $\lambda_{\mathrm{Si}}$
(measured at peak brightness).
This contrasts with the subset of supernovae identified by their high-velocity features (HVF) in
Si~II~$\lambda$6355
(typically detected pre-maximum), who are found to be distinguished by their host galaxy properties
\citep[e.g.][]{2014MNRAS.444.3258M, 2015MNRAS.446..354P, 2015ApJS..220...20Z}.
The population and properties of HVF SNe within the SNfactory sample will be presented in \citet{2018lin}.
The correlation of $x_1$ with  host-galaxy (including dust) properties is  well established 
\citep{2000AJ....120.1479H, 2003MNRAS.340.1057S}.


\subsubsection{Results for the Parameter $p$}
\label{p:sec}
The model assumes that $p$ has a Normal distribution, which is shown in Figure~\ref{ebv3:fig} as the 
stack of the posteriors of all supernovae of the color-excess $E_p({\hat{B}}-{\hat{V}})$.
Also shown for comparison is the distribution of
the   color-excess $E_\gamma({\hat{B}}-{\hat{V}}) \equiv (\gamma^0_{\hat{B}} -\gamma^0_{\hat{V}}) g_0
+ (\gamma^1_{\hat{B}} -\gamma^1_{\hat{V}}) g_1$.
The standard deviations of  $E_\gamma({\hat{B}}-{\hat{V}})$ and $E_p({\hat{B}}-{\hat{V}})$ are
%-----
0.085 %new
and 0.010
%-----
mag respectively.
The latter is comparable to the typical measurement uncertainty.
In $\hat{B}-\hat{V}$, 
the $g$-based color variation is much larger than that of the $p$-based and its effect per-supernova would be difficult to discern.

\begin{figure}[htbp] %  figure placement: here, top, bottom, or page
   \centering
   \includegraphics[width=4in]{ebv.pdf}
      \caption{Stack of the posteriors  (includes parameter uncertainty) of all supernovae of the  $E_{\gamma}({\hat{B}}-{\hat{V}})$ 
      and the 
 $E_p({\hat{B}}-{\hat{V}})$  contributions to color excess relative to an arbitrary supernova.
   \label{ebv3:fig}}
\end{figure}


There is no significant correlation between the parameter $p$ and the other features.
Their independence from the  $g_0$ and $g_1$ parameters
implies that two physical parameters are not being artificially attributed to three
model parameters.  One could have
worried this might
occur since our linear model does not precisely
describe the non-linearity between broad-band magnitudes and dust parameters. A
correlation 
between the parameters would have complicated any claims of the detection of a third independent supernova parameter.

Detection of intrinsic color has been previously reported by \citetalias{2017ApJ...842...93M}.
For data, they use the outputs of SALT2 fits: the $c$ color parameter;
the
supernova absolute magnitude $M$.
In the \citetalias{2017ApJ...842...93M}
model, intrinsic color contributes a linear   $\beta_{\mathit{int}} c$ to  the absolute
magnitude.
They find a significant $\beta_{\mathit{int}} = 2.2\pm 0.3$.
While we too claim the detection of a color parameter incompatible with dust,
the differences between our models make it
difficult to determine whether the results are consistent.   Our results show effects on color that are not monotonic
with wavelength.  We find that the statistic that nominally best corresponds to  $\beta_{\mathit{int}}$
to be
$\phi_{\hat{B}}/(\phi_{\hat{B}}-\phi_{\hat{V}}) =-4.4_{  -6.4}^{+   2.4}$, which appears to be inconsistent with  \citetalias{2017ApJ...842...93M}.  
However a change of the color baseline gives a significantly different
$\phi_{\hat{B}}/(\phi_{\hat{B}}-\phi_{\hat{R}}) =4.3_{  -1.4}^{+   3.8}$.  Our results indicate that  $\beta_{\mathit{int}}$
is sensitive to the color that $c$ corresponds to.  Although SALT2 $c$ is calibrated to correspond with $B-V$, it is
determined using data from all bands with the wavelength extremes  typically providing the strongest leverage.
The $c$ parameter is thus an  amalgamation of all input colors.
We therefore make no conclusions on the consistency between our $\vec{\phi}$-color properties
and those found by \citetalias{2017ApJ...842...93M}.


An independent piece of information available for a subset of our supernova sample is host-galaxy mass \citep{2013ApJ...770..108C}.  
This statistic is of interest, as there is a correlation between Hubble residual and host-galaxy mass
\citep[first noted by][]{2010ApJ...715..743K,2010MNRAS.406..782S}, a signal confirmed to exist in the SNfactory
sample \citep{2013ApJ...770..108C}.
This host-mass bias could be the result of a parameter that was not accounted for in the inference of SN~Ia absolute magnitude.
Indeed  \citetalias{2017ApJ...842...93M} find that the  introduction of intrinsic color as a latent parameter
reduces the strength of  this bias.

It is possible that the  $p$ parameter is a supernova-tracer of a population that to date has only been identified through host-galaxy tracers.
We plot in Figure~\ref{childress3:fig} our parameter
 $\sigma_p\phi_{\hat{V}}(p-p|_0) $  (subtracting out a random supernova
to suppress correlated errors) versus host mass
for the subset of supernovae whose host measurements are given in \citet{2013ApJ...770..108C}.
Low- and high-mass galaxies, divided by $\log{(M/M_\sun)}=10$, have high probability of hosting supernovae with different $p$ distributions, as
the Kolmogorov-Smirnov test gives a two-tailed $p$-value of $  0.016 _ {     0.011 } ^{     0.044 }$.
The median and median uncertainties of the two subsamples are
%---
$\langle \sigma_p\phi_{\hat{V}}(p-p|_0) \rangle=  0.041 \pm {     0.007 }$ mag,
$\langle \sigma_p\phi_{\hat{V}}(p-p|_0)  \rangle=0.065 \pm {     0.007 }$ mag
for low- and high-mass hosts respectively, with a difference of $ 0.024 \pm {     0.010 }$~mag that is significant at $>2 \sigma$.

In this article we do not associate $p$ directly with Hubble residuals. This is left for future work.

\begin{figure}[htbp] %  figure placement: here, top, bottom, or page
   \centering
   \includegraphics[width=4in]{childress.pdf} 
   \caption{Parameter $\sigma_p\phi_{\hat{V}}p$  versus host galaxy mass. Overplotted are the mean and 1$\sigma$ uncertainty on the mean for supernovae with hosts
      less than and greater than  $\log{(M/M_\sun)}=10$.
 $p$ as a function of host-galaxy mass
    \label{childress3:fig}}
\end{figure}


\subsubsection{Results for the Magnitude Offset $\Delta$}
\label{Delta:sec}

The color-standardization leaves magnitude residuals $\Delta$.
The histogram of the per-supernova  medians of these relative  grey offsets $\Delta-\Delta|_0$,
is shown in Figure~\ref{hist:fig}.  The distribution
has a total standard deviation of
%-----
$0.13$
%-----
mag, and a tail in the positive (fainter) direction.
Supernova peculiar velocities contribute to the standard deviation; of the four
supernovae with the highest values of $\Delta$, two have 
heliocentric
redshifts of 0.0015 and 0.0086. 
\replaced{
Eq.~\ref{corr:eqn} shows $>2 \sigma$ correlations between $\Delta$ and each of $\lambda_{\mathrm{Si}}$, $x_1$, $E_{\gamma^1}({\hat{B}}-{\hat{V}})$, and
$\sigma_p\phi_{\hat{V}}p$, showing that in addition to standardizing colors, these features can also be used to standardize
absolute magnitudes.  }{Recall that the model degeneracy between $\Delta$ and the cofactors (e.g.\ $\alpha$, $\beta$) is removed by enforcing
zero correlation between $\Delta$ and the per-supernova parameters (e.g.\ $EW_{\mathrm{Ca}}$,  $EW_{\mathrm{Si}}$).
An alternative method to break the degeneracy is to minimize the dispersion in $\Delta$.
}
The focus of this article is color standardization so we defer consideration of absolute-magnitude standardization to future work.
 
\begin{figure}[htbp] %  figure placement: here, top, bottom, or page
   \centering
   \includegraphics[width=4in]{deltaDelta_hist.pdf} 
   \caption{
   Normalized stack of the posteriors  (includes parameter uncertainty) of all supernovae and histogram of the per-supernova medians of the grey offset $\Delta$. 
   To help null correlated errors, we select
    an arbitrary supernova and subtract out its $\Delta|_0$ from those of all other objects at the level of each MCMC link.
  Of the four
supernovae with the highest values of $\Delta$, two have 
heliocentric
redshifts of 0.0015 and 0.0086. 
   \label{hist:fig}}
\end{figure}



\section{Conclusions}
\label{conclusions:sec}
To summarize, we model SN~Ia broadband optical peak magnitudes allowing for correlations with spectral features at peak,
light-curve shape, and
latent color parameters.  Analyzing SNfactory data with this model, we find significant evidence that the above parameters do
affect supernova magnitudes and colors.  Two of the latent color parameters are consistent with the
\citetalias{1999PASP..111...63F} dust-extinction model, making this
the first determination of the dust extinction curve outside the Local Group
derived entirely independently of assumptions about the shape of the extinction curve and/or assumptions about the
distribution of $A_V$.  
We find a broad distribution for total-to-selective extinction
with an effective
$\langle R^F_{\mathit{eff}}\rangle \sim 2.43$ and an asymmetric tail extending toward lower values.
We identify a new
parameter that affects supernova
colors in a manner that is distinct from the expectations for dust
and the SALT2 shape-parameter $x_1$.
This new parameter correlates with host-galaxy mass, meaning it is a candidate supernova observable that may be linked
with the Hubble diagram mass-step.
A significant bias in $R^F$ is obtained when this third source of color variation
is unaccounted for.

Just as supernova spectral and light curve features are associated with distinct color variations, the latent parameter $p$
identified through color variation may be associated with a spectral and/or light curve feature not included
in our analysis.  It
is of interest to look for tracers other than color to account for this supernova diversity.
We leave to future work the search for correlations between $p$ and other
observable features accessed by SNfactory data.

Using different arguments based on expected populations, \citetalias{2017ApJ...842...93M} infer
separate intrinsic and dust-based color-magnitude relations based on an independent sample of 248 SNe~Ia
mainly from the CfA \citep{1999AJ....117..707R, 2006AJ....131..527J, 2009ApJ...700.1097H, 2012ApJS..200...12H}
and CSP \citep{2010AJ....139..519C, 2011AJ....142..156S} surveys.
Finding slight color perturbations in the multi-band light curves 
of those samples that are  consistent with our $\sigma_p \vec{\phi}$ would provide strong evidence 
that we and  \citetalias{2017ApJ...842...93M}  are seeing the same color effect.

The current data shows a statistically significant difference in the distribution of our new parameter for supernovae
in low- and high-mass host-galaxies.
Using the SNfactory sample,
\citet{2017Rigault} find that a step in Hubble residuals is better related to local 
specific
star 
formation rate, rather than
global host mass.  The search for a correlation between our parameter and local star formation rate is planned for future work.

An interesting direction for future work is to use narrower bands for generating synthetic photometry
from the SNfactory spectrophotometry.  When the bandwidth becomes small enough
to resolve spectral features, the analysis would produce a spectroscopic model.   Higher resolution would allow us to forego 
the use of a fiducial template necessary to predict broadband fluxes generated by dust models, and also allow direct incorporation
of dust models into our framework.


The approach of our analysis is to mine for new supernova properties based on colors and spectral features.
Our model and results do carry information on absolute magnitude but are not tailored for its study.
The grey parameter $\Delta$, while containing information on absolute magnitude, 
also 
includes
contributions from peculiar velocities and measurement uncertainties.  Studies
focused on improving SNe~Ia as standard candles 
would need to
modify  our model to
distinguish between these sources of greyness and dispersion.
Our
mining exercise uses all supernovae that pass quality cuts, the approach taken by previous
exploratory work.  In contrast, the calibration of SNe~Ia as standard candles needs procedures such
as cross-validation to avoid overtraining.
An absolute magnitude calibration using the parameters identified in this analysis is left to future work.

The existence of three color parameters  has important implications when SNe~Ia are used as distance indicators.
Measurements in at least four bands are necessary to obtain three colors.  
Supernova distance determinations that do not account for three color parameters are susceptible to 
systematic errors.  
Parameter populations evolving with redshift would result in errors in
Hubble diagrams.
The use of  correlators with the SN parameters, such as host-galaxy properties e.g.\ as shown for our $p$ parameter,
may partially mitigate such errors.

\acknowledgments
We thank the STAN team for providing the statistical tool without which this analysis would not have been possible,
and Michael Betancourt specifically for his helpful guidance.  
Distribution surfaces are plotted using the ChainConsumer package \citep{Hinton2016}.
We thank Danny Goldstein and
Xiaosheng Huang for useful discussions.
We thank Dan Birchall for observing assistance, the technical and
scientific staffs of the Palomar Observatory, the High Performance
Wireless Radio Network (HPWREN), and the University of Hawaii 2.2~m
telescope.  We recognize the significant cultural role of Mauna Kea
within the indigenous Hawaiian community, and we appreciate the
opportunity to conduct observations from this revered site.  This
work was supported in part by the Director, Office of Science,
Office of High Energy Physics, of the U.S. Department of Energy
under Contract No. DE-AC02- 05CH11231.  Support in France was
provided by CNRS/IN2P3, CNRS/INSU, and PNC; LPNHE acknowledges
support from LABEX ILP, supported by French state funds managed by
the ANR within the Investissements d'Avenir programme under reference
ANR-11-IDEX-0004-02.  Support in Germany was provided by the DFG
through TRR33 ``The Dark Universe;'' and in China from Tsinghua
University 985 grant and NSFC grant No~11173017.  Some results were
obtained using resources and support from the National Energy
Research Scientific Computing Center, supported by the Director,
Office of Science, Office of Advanced Scientific Computing Research,
of the U.S. Department of Energy under Contract No. DE-AC02-05CH11231.
HPWREN is funded by National Science Foundation Grant Number
ANI-0087344, and the University of California, San Diego.
This project has received funding from the European Research Council (ERC) under the European Union's Horizon 2020 research and innovation programme (grant agreement No 759194 - USNAC 

\appendix
\section{Results from Model~I and II}
\label{m12results:sec}
For each of the five filters, the 68\%  equal-tailed credible intervals for the global parameters $\vec{\alpha}$, $\vec{\beta}$, $\vec{\eta}$,
$\vec{\gamma}^0$ and $\vec{\gamma}^1$ of Model~I are given in  Table~\ref{global2:tab}.
The results for the same parameters and $\sigma_p\vec{\phi}$
of Model~II are given in Table~\ref{global2:tab}.  In contrast to Model~III,
Figure~\ref{M1ev:fig} shows that the $\hat{I}$-band has
the largest separation between the degenerate solution of $\sigma_p \phi_X$.  To break the degeneracy we enforce 
$\vec{\phi} = \text{sign}(\phi_{\hat{I}}) \vec{\phi}$.


\begin{table}
\centering
\begin{tabular}{|c|c|c|c|c|c|}
\hline
Parameters& $X={\hat{U}}$ &${\hat{B}}$&${\hat{V}}$&${\hat{R}}$&${\hat{I}}$\\ \hline
$\alpha_X$
& $0.0037^{+0.0003}_{-0.0004}$
& $0.0011^{+0.0003}_{-0.0003}$
& $0.0012^{+0.0003}_{-0.0003}$
& $0.0013^{+0.0003}_{-0.0003}$
& $0.0026^{+0.0002}_{-0.0003}$
\\
${\alpha_X/\alpha_V-1}$
& $   2.0^{+   0.5}_{  -0.3}$
& $  -0.1^{+   0.1}_{  -0.1}$
& \ldots
& $   0.1^{+   0.1}_{  -0.0}$
& $   1.1^{+   0.4}_{  -0.2}$
\\
$\beta_X$
& $ 0.026^{+ 0.002}_{-0.002}$
& $ 0.019^{+ 0.001}_{-0.001}$
& $ 0.022^{+ 0.001}_{-0.001}$
& $ 0.018^{+ 0.001}_{-0.001}$
& $ 0.018^{+ 0.001}_{-0.001}$
\\
${\beta_X/\beta_V-1}$
& $  0.17^{+  0.02}_{ -0.02}$
& $ -0.14^{+  0.02}_{ -0.02}$
& \ldots
& $ -0.18^{+  0.01}_{ -0.01}$
& $ -0.16^{+  0.01}_{ -0.01}$
\\
$\eta_X$
& $-0.0006^{+0.0005}_{-0.0005}$
& $-0.0007^{+0.0004}_{-0.0004}$
& \ldots
& $0.0002^{+0.0003}_{-0.0004}$
& $-0.0007^{+0.0003}_{-0.0003}$
\\
${\eta_X/\eta_V-1}$
& $ -0.30^{+  4.06}_{ -3.18}$
& $ -0.73^{+  5.16}_{ -3.69}$
& \ldots
& $ -0.02^{+  0.81}_{ -0.97}$
& $ -1.03^{+  5.25}_{ -3.77}$
\\
$\gamma^0_X$
& $ 76.37^{+  3.19}_{ -3.23}$
& $ 60.22^{+  3.09}_{ -3.20}$
& $ 43.06^{+  2.95}_{ -3.11}$
& $ 32.93^{+  2.60}_{ -2.87}$
& $ 22.89^{+  2.27}_{ -2.65}$
\\
${\gamma^0_X/\gamma^0_V-1}$
& $  0.78^{+  0.06}_{ -0.06}$
& $  0.40^{+  0.03}_{ -0.03}$
& \ldots
& $ -0.24^{+  0.01}_{ -0.01}$
& $ -0.47^{+  0.02}_{ -0.03}$
\\
$\gamma^1_X$
& $-18.64^{+  4.86}_{ -4.86}$
& $-28.10^{+  4.23}_{ -4.20}$
& $-28.71^{+  3.75}_{ -3.70}$
& $-21.86^{+  3.49}_{ -3.32}$
& $-11.49^{+  3.38}_{ -3.05}$
\\
${\gamma^1_X/\gamma^1_V-1}$
& $ -0.35^{+  0.08}_{ -0.10}$
& $ -0.02^{+  0.03}_{ -0.03}$
& \ldots
& $ -0.24^{+  0.02}_{ -0.03}$
& $ -0.60^{+  0.06}_{ -0.08}$
\\
\hline
\end{tabular}
\caption{68\% credible intervals for the global fit parameters of the Two Color Parameter Model~I in \S\ref{modelI:sec}.\label{global1:tab}}
\end{table}


\begin{table}
\centering
\begin{tabular}{|c|c|c|c|c|c|}
\hline
Parameters& $X={\hat{U}}$ &${\hat{B}}$&${\hat{V}}$&${\hat{R}}$&${\hat{I}}$\\ \hline
$\alpha_X$
& $0.0048^{+0.0010}_{-0.0011}$
& $0.0019^{+0.0008}_{-0.0009}$
& $0.0018^{+0.0007}_{-0.0007}$
& $0.0017^{+0.0005}_{-0.0005}$
& $0.0027^{+0.0004}_{-0.0004}$
\\
${\alpha_X}/\alpha_{\hat{V}-1}$
& $   1.7^{+   0.8}_{  -0.4}$
& $   0.1^{+   0.1}_{  -0.1}$
& \ldots
& $  -0.0^{+   0.1}_{  -0.1}$
& $   0.5^{+   0.6}_{  -0.3}$
\\
$\beta_X$
& $ 0.035^{+ 0.002}_{-0.003}$
& $ 0.026^{+ 0.002}_{-0.002}$
& $ 0.027^{+ 0.002}_{-0.002}$
& $ 0.022^{+ 0.001}_{-0.001}$
& $ 0.021^{+ 0.001}_{-0.001}$
\\
${\beta_X}/\beta_{\hat{V}-1}$
& $  0.27^{+  0.04}_{ -0.04}$
& $ -0.05^{+  0.02}_{ -0.02}$
&  \ldots
& $ -0.19^{+  0.01}_{ -0.01}$
& $ -0.22^{+  0.03}_{ -0.02}$
\\
$\eta_X$
& $-0.0005^{+0.0008}_{-0.0008}$
& $-0.0003^{+0.0007}_{-0.0007}$
& $0.0006^{+0.0006}_{-0.0006}$
& $0.0006^{+0.0005}_{-0.0005}$
& $0.0000^{+0.0004}_{-0.0004}$
\\
${\eta_X}/\eta_{\hat{V}-1}$
& $ -1.17^{+  1.20}_{ -2.44}$
& $ -1.01^{+  0.76}_{ -1.89}$
&  \ldots
& $ -0.03^{+  0.33}_{ -0.19}$
& $ -0.78^{+  0.51}_{ -0.78}$
\\
$\gamma^0_X$
& $ 71.40^{+  3.29}_{ -3.14}$
& $ 58.39^{+  2.79}_{ -2.62}$
& $ 43.99^{+  2.27}_{ -2.12}$
& $ 33.29^{+  1.87}_{ -1.72}$
& $ 24.81^{+  1.64}_{ -1.53}$
\\
${\gamma^0_X}/\gamma^0_{\hat{V}-1}$
& $  0.62^{+  0.04}_{ -0.04}$
& $  0.33^{+  0.02}_{ -0.02}$
& \ldots
& $ -0.24^{+  0.01}_{ -0.01}$
& $ -0.44^{+  0.02}_{ -0.02}$
\\
$\gamma^1_X$
& $  9.83^{+  5.44}_{ -4.90}$
& $  2.34^{+  4.44}_{ -4.18}$
& $ -4.58^{+  3.40}_{ -3.32}$
& $ -2.41^{+  2.72}_{ -2.63}$
& $ -5.77^{+  2.22}_{ -2.15}$
\\
${\gamma^1_X}/\gamma^1_{\hat{V}-1}$
& $ -2.62^{+  1.37}_{ -4.02}$
& $ -1.25^{+  0.68}_{ -1.99}$
& \ldots
& $ -0.43^{+  0.12}_{ -0.32}$
& $  0.15^{+  0.88}_{ -0.37}$
\\
$\sigma_p \phi_X$
& $-0.007^{+ 0.012}_{-0.012}$
& $-0.039^{+ 0.010}_{-0.010}$
& $-0.039^{+ 0.009}_{-0.008}$
& $-0.026^{+ 0.007}_{-0.007}$
& $ 0.018^{+ 0.007}_{-0.007}$
\\
${\phi_X/\phi_{\hat{V}}-1}$
& $-0.804^{+ 0.245}_{-0.342}$
& $-0.009^{+ 0.095}_{-0.109}$
& \ldots
& $-0.328^{+ 0.047}_{-0.066}$
& $-1.465^{+ 0.220}_{-0.349}$
\\\hline
\end{tabular}
\caption{68\% credible intervals for the global fit parameters of the Three Color Parameter Model~II in \S\ref{modelII:sec}.\label{global2:tab}}
\end{table}
%
%\section{Covariance Matrix for Model~III Global Parameters}
%\added{The following is the  covariance matrix for the global cofactors $\{ \alpha_{\hat{U}}, \beta_{\hat{U}}, \eta_{\hat{U}}, \zeta_{\hat{U}}, \gamma^0_{\hat{U}},\gamma^1_{\hat{U}}, \sigma_p \phi_{\hat{U}}, 
%\ldots,
%\alpha_{\hat{I}}, \beta_{\hat{I}}, \eta_{\hat{I}}, \zeta_{\hat{I}}, \gamma^0_{\hat{I}},\gamma^1_{\hat{I}}, \sigma_p \phi_{\hat{I}}
% \}$.  The matrix 35x35 is big so will have to go in supplemental material.}
%\setcounter{MaxMatrixCols}{40}
%\begin{equation}
%\tiny
%\begin{bmatrix}
%5.5\times 10^{-7} & -1.8\times 10^{-6} & 3.9\times 10^{-7} & -1.6\times 10^{-5} & 2.1\times 10^{-4} & -2.6\times 10^{-4} & 3.6\times 10^{-6} & 4.5\times 10^{-7} & -1.5\times 10^{-6} & 3.1\times 10^{-7} & -1.3\times 10^{-5} & 2.6\times 10^{-4} & -1.3\times 10^{-4} & 2.8\times 10^{-6} & 3.6\times 10^{-7} & -1.2\times 10^{-6} & 2.4\times 10^{-7} & -1.0\times 10^{-5} & 1.9\times 10^{-4} & -5.5\times 10^{-5} & 1.7\times 10^{-6} & 2.9\times 10^{-7} & -9.2\times 10^{-7} & 1.9\times 10^{-7} & -8.1\times 10^{-6} & 1.1\times 10^{-4} & -1.7\times 10^{-5} & 1.1\times 10^{-6} & 2.2\times 10^{-7} & -7.2\times 10^{-7} & 1.4\times 10^{-7} & -6.3\times 10^{-6} & -3.6\times 10^{-5} & -6.7\times 10^{-6} & 4.7\times 10^{-7} \\
%-1.8\times 10^{-6} & 2.7\times 10^{-5} & -2.6\times 10^{-6} & 1.9\times 10^{-4} & -4.0\times 10^{-3} & -3.0\times 10^{-3} & -1.8\times 10^{-5} & -1.4\times 10^{-6} & 2.2\times 10^{-5} & -2.1\times 10^{-6} & 1.5\times 10^{-4} & -3.2\times 10^{-3} & -2.7\times 10^{-3} & -1.4\times 10^{-5} & -1.1\times 10^{-6} & 1.7\times 10^{-5} & -1.6\times 10^{-6} & 1.2\times 10^{-4} & -2.6\times 10^{-3} & -2.1\times 10^{-3} & -1.0\times 10^{-5} & -9.2\times 10^{-7} & 1.4\times 10^{-5} & -1.2\times 10^{-6} & 9.4\times 10^{-5} & -2.0\times 10^{-3} & -1.6\times 10^{-3} & -7.9\times 10^{-6} & -7.3\times 10^{-7} & 1.1\times 10^{-5} & -8.8\times 10^{-7} & 7.6\times 10^{-5} & -1.1\times 10^{-3} & -1.2\times 10^{-3} & -5.2\times 10^{-6} \\
%3.9\times 10^{-7} & -2.6\times 10^{-6} & 1.3\times 10^{-6} & -2.2\times 10^{-5} & 3.9\times 10^{-4} & 1.7\times 10^{-3} & 1.0\times 10^{-5} & 3.2\times 10^{-7} & -2.1\times 10^{-6} & 1.0\times 10^{-6} & -1.8\times 10^{-5} & 5.0\times 10^{-4} & 1.4\times 10^{-3} & 8.3\times 10^{-6} & 2.4\times 10^{-7} & -1.5\times 10^{-6} & 7.8\times 10^{-7} & -1.3\times 10^{-5} & 5.4\times 10^{-4} & 1.0\times 10^{-3} & 6.0\times 10^{-6} & 1.9\times 10^{-7} & -1.1\times 10^{-6} & 6.1\times 10^{-7} & -1.0\times 10^{-5} & 4.1\times 10^{-4} & 8.0\times 10^{-4} & 4.6\times 10^{-6} & 1.4\times 10^{-7} & -8.1\times 10^{-7} & 4.7\times 10^{-7} & -7.4\times 10^{-6} & 1.6\times 10^{-4} & 5.9\times 10^{-4} & 3.1\times 10^{-6} \\
%-1.6\times 10^{-5} & 1.9\times 10^{-4} & -2.2\times 10^{-5} & 1.6\times 10^{-3} & -2.9\times 10^{-2} & -1.4\times 10^{-2} & -2.7\times 10^{-4} & -1.3\times 10^{-5} & 1.5\times 10^{-4} & -1.8\times 10^{-5} & 1.3\times 10^{-3} & -2.8\times 10^{-2} & -1.3\times 10^{-2} & -2.1\times 10^{-4} & -1.0\times 10^{-5} & 1.2\times 10^{-4} & -1.4\times 10^{-5} & 9.9\times 10^{-4} & -2.3\times 10^{-2} & -1.0\times 10^{-2} & -1.5\times 10^{-4} & -8.3\times 10^{-6} & 9.5\times 10^{-5} & -1.1\times 10^{-5} & 7.9\times 10^{-4} & -1.8\times 10^{-2} & -7.7\times 10^{-3} & -1.2\times 10^{-4} & -6.4\times 10^{-6} & 7.7\times 10^{-5} & -7.7\times 10^{-6} & 6.3\times 10^{-4} & -8.1\times 10^{-3} & -4.9\times 10^{-3} & -7.9\times 10^{-5} \\
%2.1\times 10^{-4} & -4.0\times 10^{-3} & 3.9\times 10^{-4} & -2.9\times 10^{-2} & 6.9\times 10^{+0} & 1.3\times 10^{+0} & 6.7\times 10^{-3} & 1.4\times 10^{-4} & -3.0\times 10^{-3} & 3.2\times 10^{-4} & -2.2\times 10^{-2} & 5.5\times 10^{+0} & 1.1\times 10^{+0} & 5.1\times 10^{-3} & 9.6\times 10^{-5} & -2.2\times 10^{-3} & 2.3\times 10^{-4} & -1.6\times 10^{-2} & 4.5\times 10^{+0} & 9.7\times 10^{-1} & 3.7\times 10^{-3} & 6.0\times 10^{-5} & -1.7\times 10^{-3} & 1.7\times 10^{-4} & -1.3\times 10^{-2} & 3.7\times 10^{+0} & 8.5\times 10^{-1} & 3.0\times 10^{-3} & 3.9\times 10^{-5} & -1.4\times 10^{-3} & 1.3\times 10^{-4} & -1.1\times 10^{-2} & 2.9\times 10^{+0} & 6.7\times 10^{-1} & 2.1\times 10^{-3} \\
%-2.6\times 10^{-4} & -3.0\times 10^{-3} & 1.7\times 10^{-3} & -1.4\times 10^{-2} & 1.3\times 10^{+0} & 1.7\times 10^{+1} & 8.7\times 10^{-3} & -1.9\times 10^{-4} & -2.8\times 10^{-3} & 1.3\times 10^{-3} & -1.3\times 10^{-2} & 1.1\times 10^{+0} & 1.4\times 10^{+1} & 7.1\times 10^{-3} & -1.7\times 10^{-4} & -1.5\times 10^{-3} & 9.7\times 10^{-4} & -6.8\times 10^{-3} & 1.8\times 10^{+0} & 1.1\times 10^{+1} & 4.9\times 10^{-3} & -1.2\times 10^{-4} & -1.0\times 10^{-3} & 7.6\times 10^{-4} & -4.4\times 10^{-3} & 1.6\times 10^{+0} & 9.6\times 10^{+0} & 3.2\times 10^{-3} & -10.0\times 10^{-5} & -3.8\times 10^{-4} & 5.6\times 10^{-4} & -7.6\times 10^{-4} & 1.4\times 10^{+0} & 7.9\times 10^{+0} & 1.8\times 10^{-3} \\
%3.6\times 10^{-6} & -1.8\times 10^{-5} & 1.0\times 10^{-5} & -2.7\times 10^{-4} & 6.7\times 10^{-3} & 8.7\times 10^{-3} & 2.8\times 10^{-4} & 2.9\times 10^{-6} & -1.4\times 10^{-5} & 8.4\times 10^{-6} & -2.1\times 10^{-4} & 1.1\times 10^{-2} & 6.8\times 10^{-3} & 2.3\times 10^{-4} & 2.0\times 10^{-6} & -9.4\times 10^{-6} & 6.4\times 10^{-6} & -1.6\times 10^{-4} & 1.1\times 10^{-2} & 4.1\times 10^{-3} & 1.8\times 10^{-4} & 1.4\times 10^{-6} & -7.4\times 10^{-6} & 4.9\times 10^{-6} & -1.2\times 10^{-4} & 8.0\times 10^{-3} & 3.0\times 10^{-3} & 1.4\times 10^{-4} & 9.2\times 10^{-7} & -7.2\times 10^{-6} & 3.6\times 10^{-6} & -9.9\times 10^{-5} & 2.6\times 10^{-3} & 2.3\times 10^{-3} & 1.1\times 10^{-4} \\
%4.5\times 10^{-7} & -1.4\times 10^{-6} & 3.2\times 10^{-7} & -1.3\times 10^{-5} & 1.4\times 10^{-4} & -1.9\times 10^{-4} & 2.9\times 10^{-6} & 4.0\times 10^{-7} & -1.3\times 10^{-6} & 2.7\times 10^{-7} & -1.1\times 10^{-5} & 2.0\times 10^{-4} & -2.5\times 10^{-5} & 2.3\times 10^{-6} & 3.2\times 10^{-7} & -1.0\times 10^{-6} & 2.1\times 10^{-7} & -9.1\times 10^{-6} & 1.4\times 10^{-4} & 4.3\times 10^{-5} & 1.3\times 10^{-6} & 2.6\times 10^{-7} & -8.2\times 10^{-7} & 1.7\times 10^{-7} & -7.1\times 10^{-6} & 7.4\times 10^{-5} & 5.1\times 10^{-5} & 6.9\times 10^{-7} & 1.9\times 10^{-7} & -6.0\times 10^{-7} & 1.2\times 10^{-7} & -5.3\times 10^{-6} & -5.2\times 10^{-5} & 9.5\times 10^{-6} & 2.4\times 10^{-7} \\
%-1.5\times 10^{-6} & 2.2\times 10^{-5} & -2.1\times 10^{-6} & 1.5\times 10^{-4} & -3.0\times 10^{-3} & -2.8\times 10^{-3} & -1.4\times 10^{-5} & -1.3\times 10^{-6} & 1.9\times 10^{-5} & -1.8\times 10^{-6} & 1.3\times 10^{-4} & -2.4\times 10^{-3} & -3.0\times 10^{-3} & -1.2\times 10^{-5} & -1.0\times 10^{-6} & 1.6\times 10^{-5} & -1.4\times 10^{-6} & 1.0\times 10^{-4} & -2.0\times 10^{-3} & -2.6\times 10^{-3} & -7.6\times 10^{-6} & -8.4\times 10^{-7} & 1.2\times 10^{-5} & -1.0\times 10^{-6} & 8.4\times 10^{-5} & -1.5\times 10^{-3} & -2.0\times 10^{-3} & -5.6\times 10^{-6} & -6.2\times 10^{-7} & 9.3\times 10^{-6} & -7.7\times 10^{-7} & 6.3\times 10^{-5} & -7.7\times 10^{-4} & -1.1\times 10^{-3} & -4.0\times 10^{-6} \\
%3.1\times 10^{-7} & -2.1\times 10^{-6} & 1.0\times 10^{-6} & -1.8\times 10^{-5} & 3.2\times 10^{-4} & 1.3\times 10^{-3} & 8.4\times 10^{-6} & 2.7\times 10^{-7} & -1.8\times 10^{-6} & 8.5\times 10^{-7} & -1.5\times 10^{-5} & 4.1\times 10^{-4} & 1.1\times 10^{-3} & 6.7\times 10^{-6} & 2.0\times 10^{-7} & -1.2\times 10^{-6} & 6.7\times 10^{-7} & -1.1\times 10^{-5} & 4.4\times 10^{-4} & 8.2\times 10^{-4} & 4.9\times 10^{-6} & 1.6\times 10^{-7} & -9.4\times 10^{-7} & 5.2\times 10^{-7} & -8.5\times 10^{-6} & 3.4\times 10^{-4} & 6.4\times 10^{-4} & 3.7\times 10^{-6} & 1.1\times 10^{-7} & -6.9\times 10^{-7} & 3.8\times 10^{-7} & -6.2\times 10^{-6} & 1.3\times 10^{-4} & 4.8\times 10^{-4} & 2.6\times 10^{-6} \\
%-1.3\times 10^{-5} & 1.5\times 10^{-4} & -1.8\times 10^{-5} & 1.3\times 10^{-3} & -2.2\times 10^{-2} & -1.3\times 10^{-2} & -2.1\times 10^{-4} & -1.1\times 10^{-5} & 1.3\times 10^{-4} & -1.5\times 10^{-5} & 1.1\times 10^{-3} & -2.2\times 10^{-2} & -1.3\times 10^{-2} & -1.7\times 10^{-4} & -9.0\times 10^{-6} & 1.0\times 10^{-4} & -1.2\times 10^{-5} & 8.7\times 10^{-4} & -1.8\times 10^{-2} & -1.1\times 10^{-2} & -1.2\times 10^{-4} & -7.2\times 10^{-6} & 8.3\times 10^{-5} & -8.9\times 10^{-6} & 6.9\times 10^{-4} & -1.4\times 10^{-2} & -8.3\times 10^{-3} & -9.2\times 10^{-5} & -5.3\times 10^{-6} & 6.2\times 10^{-5} & -6.6\times 10^{-6} & 5.1\times 10^{-4} & -6.0\times 10^{-3} & -4.1\times 10^{-3} & -6.7\times 10^{-5} \\
%2.6\times 10^{-4} & -3.2\times 10^{-3} & 5.0\times 10^{-4} & -2.8\times 10^{-2} & 5.5\times 10^{+0} & 1.1\times 10^{+0} & 1.1\times 10^{-2} & 2.0\times 10^{-4} & -2.4\times 10^{-3} & 4.1\times 10^{-4} & -2.2\times 10^{-2} & 5.1\times 10^{+0} & 9.4\times 10^{-1} & 8.4\times 10^{-3} & 1.4\times 10^{-4} & -1.8\times 10^{-3} & 3.0\times 10^{-4} & -1.6\times 10^{-2} & 4.3\times 10^{+0} & 8.6\times 10^{-1} & 6.0\times 10^{-3} & 9.8\times 10^{-5} & -1.4\times 10^{-3} & 2.3\times 10^{-4} & -1.3\times 10^{-2} & 3.4\times 10^{+0} & 7.8\times 10^{-1} & 4.6\times 10^{-3} & 5.8\times 10^{-5} & -1.1\times 10^{-3} & 1.7\times 10^{-4} & -1.0\times 10^{-2} & 2.4\times 10^{+0} & 6.5\times 10^{-1} & 3.2\times 10^{-3} \\
%-1.3\times 10^{-4} & -2.7\times 10^{-3} & 1.4\times 10^{-3} & -1.3\times 10^{-2} & 1.1\times 10^{+0} & 1.4\times 10^{+1} & 6.8\times 10^{-3} & -2.5\times 10^{-5} & -3.0\times 10^{-3} & 1.1\times 10^{-3} & -1.3\times 10^{-2} & 9.4\times 10^{-1} & 1.4\times 10^{+1} & 6.4\times 10^{-3} & 1.8\times 10^{-5} & -2.2\times 10^{-3} & 8.6\times 10^{-4} & -1.1\times 10^{-2} & 1.5\times 10^{+0} & 1.2\times 10^{+1} & 1.6\times 10^{-3} & 2.0\times 10^{-5} & -1.6\times 10^{-3} & 6.7\times 10^{-4} & -7.7\times 10^{-3} & 1.3\times 10^{+0} & 9.5\times 10^{+0} & 3.9\times 10^{-6} & -2.3\times 10^{-5} & -6.0\times 10^{-4} & 5.1\times 10^{-4} & -3.1\times 10^{-3} & 1.1\times 10^{+0} & 7.1\times 10^{+0} & -1.3\times 10^{-3} \\
%2.8\times 10^{-6} & -1.4\times 10^{-5} & 8.3\times 10^{-6} & -2.1\times 10^{-4} & 5.1\times 10^{-3} & 7.1\times 10^{-3} & 2.3\times 10^{-4} & 2.3\times 10^{-6} & -1.2\times 10^{-5} & 6.7\times 10^{-6} & -1.7\times 10^{-4} & 8.4\times 10^{-3} & 6.4\times 10^{-3} & 1.9\times 10^{-4} & 1.6\times 10^{-6} & -8.8\times 10^{-6} & 5.2\times 10^{-6} & -1.4\times 10^{-4} & 8.6\times 10^{-3} & 4.5\times 10^{-3} & 1.5\times 10^{-4} & 1.1\times 10^{-6} & -7.0\times 10^{-6} & 4.0\times 10^{-6} & -1.1\times 10^{-4} & 6.3\times 10^{-3} & 3.2\times 10^{-3} & 1.2\times 10^{-4} & 6.7\times 10^{-7} & -6.3\times 10^{-6} & 3.0\times 10^{-6} & -8.3\times 10^{-5} & 2.1\times 10^{-3} & 2.0\times 10^{-3} & 8.9\times 10^{-5} \\
%3.6\times 10^{-7} & -1.1\times 10^{-6} & 2.4\times 10^{-7} & -1.0\times 10^{-5} & 9.6\times 10^{-5} & -1.7\times 10^{-4} & 2.0\times 10^{-6} & 3.2\times 10^{-7} & -1.0\times 10^{-6} & 2.0\times 10^{-7} & -9.0\times 10^{-6} & 1.4\times 10^{-4} & 1.8\times 10^{-5} & 1.6\times 10^{-6} & 2.8\times 10^{-7} & -9.3\times 10^{-7} & 1.7\times 10^{-7} & -7.8\times 10^{-6} & 9.4\times 10^{-5} & 1.0\times 10^{-4} & 6.8\times 10^{-7} & 2.3\times 10^{-7} & -7.4\times 10^{-7} & 1.3\times 10^{-7} & -6.2\times 10^{-6} & 3.6\times 10^{-5} & 8.4\times 10^{-5} & 2.4\times 10^{-7} & 1.7\times 10^{-7} & -5.5\times 10^{-7} & 9.8\times 10^{-8} & -4.6\times 10^{-6} & -6.3\times 10^{-5} & 3.2\times 10^{-5} & -5.5\times 10^{-8} \\
%-1.2\times 10^{-6} & 1.7\times 10^{-5} & -1.5\times 10^{-6} & 1.2\times 10^{-4} & -2.2\times 10^{-3} & -1.5\times 10^{-3} & -9.4\times 10^{-6} & -1.0\times 10^{-6} & 1.6\times 10^{-5} & -1.2\times 10^{-6} & 1.0\times 10^{-4} & -1.8\times 10^{-3} & -2.2\times 10^{-3} & -8.8\times 10^{-6} & -9.3\times 10^{-7} & 1.4\times 10^{-5} & -9.7\times 10^{-7} & 9.4\times 10^{-5} & -1.3\times 10^{-3} & -2.2\times 10^{-3} & -5.0\times 10^{-6} & -7.6\times 10^{-7} & 1.1\times 10^{-5} & -7.5\times 10^{-7} & 7.6\times 10^{-5} & -9.6\times 10^{-4} & -1.6\times 10^{-3} & -3.4\times 10^{-6} & -5.7\times 10^{-7} & 8.7\times 10^{-6} & -5.6\times 10^{-7} & 5.8\times 10^{-5} & -4.2\times 10^{-4} & -9.0\times 10^{-4} & -2.8\times 10^{-6} \\
%2.4\times 10^{-7} & -1.6\times 10^{-6} & 7.8\times 10^{-7} & -1.4\times 10^{-5} & 2.3\times 10^{-4} & 9.7\times 10^{-4} & 6.4\times 10^{-6} & 2.1\times 10^{-7} & -1.4\times 10^{-6} & 6.7\times 10^{-7} & -1.2\times 10^{-5} & 3.0\times 10^{-4} & 8.6\times 10^{-4} & 5.2\times 10^{-6} & 1.7\times 10^{-7} & -9.7\times 10^{-7} & 5.5\times 10^{-7} & -8.8\times 10^{-6} & 3.3\times 10^{-4} & 6.6\times 10^{-4} & 3.7\times 10^{-6} & 1.3\times 10^{-7} & -7.4\times 10^{-7} & 4.3\times 10^{-7} & -6.8\times 10^{-6} & 2.5\times 10^{-4} & 5.1\times 10^{-4} & 2.8\times 10^{-6} & 9.4\times 10^{-8} & -5.4\times 10^{-7} & 3.2\times 10^{-7} & -5.0\times 10^{-6} & 9.2\times 10^{-5} & 3.6\times 10^{-4} & 2.0\times 10^{-6} \\
%-1.0\times 10^{-5} & 1.2\times 10^{-4} & -1.3\times 10^{-5} & 9.9\times 10^{-4} & -1.6\times 10^{-2} & -6.8\times 10^{-3} & -1.6\times 10^{-4} & -9.1\times 10^{-6} & 1.0\times 10^{-4} & -1.1\times 10^{-5} & 8.7\times 10^{-4} & -1.6\times 10^{-2} & -1.1\times 10^{-2} & -1.4\times 10^{-4} & -7.8\times 10^{-6} & 9.4\times 10^{-5} & -8.8\times 10^{-6} & 7.8\times 10^{-4} & -1.4\times 10^{-2} & -1.0\times 10^{-2} & -9.2\times 10^{-5} & -6.3\times 10^{-6} & 7.6\times 10^{-5} & -6.8\times 10^{-6} & 6.2\times 10^{-4} & -1.0\times 10^{-2} & -7.0\times 10^{-3} & -6.7\times 10^{-5} & -4.7\times 10^{-6} & 5.8\times 10^{-5} & -5.1\times 10^{-6} & 4.7\times 10^{-4} & -4.0\times 10^{-3} & -3.9\times 10^{-3} & -5.1\times 10^{-5} \\
%1.9\times 10^{-4} & -2.6\times 10^{-3} & 5.4\times 10^{-4} & -2.3\times 10^{-2} & 4.5\times 10^{+0} & 1.8\times 10^{+0} & 1.1\times 10^{-2} & 1.4\times 10^{-4} & -2.0\times 10^{-3} & 4.4\times 10^{-4} & -1.8\times 10^{-2} & 4.3\times 10^{+0} & 1.5\times 10^{+0} & 8.6\times 10^{-3} & 9.4\times 10^{-5} & -1.3\times 10^{-3} & 3.3\times 10^{-4} & -1.4\times 10^{-2} & 4.0\times 10^{+0} & 1.2\times 10^{+0} & 6.5\times 10^{-3} & 5.8\times 10^{-5} & -9.9\times 10^{-4} & 2.5\times 10^{-4} & -1.1\times 10^{-2} & 3.2\times 10^{+0} & 1.0\times 10^{+0} & 5.0\times 10^{-3} & 2.6\times 10^{-5} & -7.8\times 10^{-4} & 1.8\times 10^{-4} & -8.2\times 10^{-3} & 2.2\times 10^{+0} & 8.9\times 10^{-1} & 3.6\times 10^{-3} \\
%-5.5\times 10^{-5} & -2.1\times 10^{-3} & 1.0\times 10^{-3} & -1.0\times 10^{-2} & 9.7\times 10^{-1} & 1.1\times 10^{+1} & 4.1\times 10^{-3} & 4.3\times 10^{-5} & -2.6\times 10^{-3} & 8.2\times 10^{-4} & -1.1\times 10^{-2} & 8.6\times 10^{-1} & 1.2\times 10^{+1} & 4.5\times 10^{-3} & 1.0\times 10^{-4} & -2.2\times 10^{-3} & 6.6\times 10^{-4} & -1.0\times 10^{-2} & 1.2\times 10^{+0} & 1.0\times 10^{+1} & -3.8\times 10^{-4} & 8.4\times 10^{-5} & -1.6\times 10^{-3} & 5.2\times 10^{-4} & -7.3\times 10^{-3} & 1.1\times 10^{+0} & 8.6\times 10^{+0} & -1.6\times 10^{-3} & 2.1\times 10^{-5} & -6.0\times 10^{-4} & 4.0\times 10^{-4} & -3.1\times 10^{-3} & 8.8\times 10^{-1} & 6.4\times 10^{+0} & -2.5\times 10^{-3} \\
%1.7\times 10^{-6} & -1.0\times 10^{-5} & 6.0\times 10^{-6} & -1.5\times 10^{-4} & 3.7\times 10^{-3} & 4.9\times 10^{-3} & 1.8\times 10^{-4} & 1.3\times 10^{-6} & -7.6\times 10^{-6} & 4.9\times 10^{-6} & -1.2\times 10^{-4} & 6.0\times 10^{-3} & 1.6\times 10^{-3} & 1.5\times 10^{-4} & 6.8\times 10^{-7} & -5.0\times 10^{-6} & 3.7\times 10^{-6} & -9.2\times 10^{-5} & 6.5\times 10^{-3} & -3.8\times 10^{-4} & 1.3\times 10^{-4} & 3.9\times 10^{-7} & -4.1\times 10^{-6} & 2.9\times 10^{-6} & -7.2\times 10^{-5} & 4.9\times 10^{-3} & -3.9\times 10^{-4} & 1.1\times 10^{-4} & 2.1\times 10^{-7} & -4.4\times 10^{-6} & 2.1\times 10^{-6} & -5.8\times 10^{-5} & 1.9\times 10^{-3} & 6.8\times 10^{-4} & 8.3\times 10^{-5} \\
%2.9\times 10^{-7} & -9.2\times 10^{-7} & 1.9\times 10^{-7} & -8.3\times 10^{-6} & 6.0\times 10^{-5} & -1.2\times 10^{-4} & 1.4\times 10^{-6} & 2.6\times 10^{-7} & -8.4\times 10^{-7} & 1.6\times 10^{-7} & -7.2\times 10^{-6} & 9.8\times 10^{-5} & 2.0\times 10^{-5} & 1.1\times 10^{-6} & 2.3\times 10^{-7} & -7.6\times 10^{-7} & 1.3\times 10^{-7} & -6.3\times 10^{-6} & 5.8\times 10^{-5} & 8.4\times 10^{-5} & 3.9\times 10^{-7} & 1.9\times 10^{-7} & -6.2\times 10^{-7} & 1.0\times 10^{-7} & -5.0\times 10^{-6} & 1.2\times 10^{-5} & 7.2\times 10^{-5} & 4.6\times 10^{-8} & 1.4\times 10^{-7} & -4.6\times 10^{-7} & 8.0\times 10^{-8} & -3.9\times 10^{-6} & -6.4\times 10^{-5} & 3.0\times 10^{-5} & -1.9\times 10^{-7} \\
%-9.2\times 10^{-7} & 1.4\times 10^{-5} & -1.1\times 10^{-6} & 9.5\times 10^{-5} & -1.7\times 10^{-3} & -1.0\times 10^{-3} & -7.4\times 10^{-6} & -8.2\times 10^{-7} & 1.2\times 10^{-5} & -9.4\times 10^{-7} & 8.3\times 10^{-5} & -1.4\times 10^{-3} & -1.6\times 10^{-3} & -7.0\times 10^{-6} & -7.4\times 10^{-7} & 1.1\times 10^{-5} & -7.4\times 10^{-7} & 7.6\times 10^{-5} & -9.9\times 10^{-4} & -1.6\times 10^{-3} & -4.1\times 10^{-6} & -6.2\times 10^{-7} & 9.5\times 10^{-6} & -5.8\times 10^{-7} & 6.3\times 10^{-5} & -7.1\times 10^{-4} & -1.1\times 10^{-3} & -3.0\times 10^{-6} & -4.8\times 10^{-7} & 7.5\times 10^{-6} & -4.4\times 10^{-7} & 5.0\times 10^{-5} & -3.0\times 10^{-4} & -6.5\times 10^{-4} & -2.4\times 10^{-6} \\
%1.9\times 10^{-7} & -1.2\times 10^{-6} & 6.1\times 10^{-7} & -1.1\times 10^{-5} & 1.7\times 10^{-4} & 7.6\times 10^{-4} & 4.9\times 10^{-6} & 1.7\times 10^{-7} & -1.0\times 10^{-6} & 5.2\times 10^{-7} & -8.9\times 10^{-6} & 2.3\times 10^{-4} & 6.7\times 10^{-4} & 4.0\times 10^{-6} & 1.3\times 10^{-7} & -7.5\times 10^{-7} & 4.3\times 10^{-7} & -6.8\times 10^{-6} & 2.5\times 10^{-4} & 5.2\times 10^{-4} & 2.9\times 10^{-6} & 1.0\times 10^{-7} & -5.8\times 10^{-7} & 3.5\times 10^{-7} & -5.3\times 10^{-6} & 1.9\times 10^{-4} & 4.1\times 10^{-4} & 2.2\times 10^{-6} & 7.7\times 10^{-8} & -4.2\times 10^{-7} & 2.6\times 10^{-7} & -4.0\times 10^{-6} & 7.3\times 10^{-5} & 2.8\times 10^{-4} & 1.6\times 10^{-6} \\
%-8.1\times 10^{-6} & 9.4\times 10^{-5} & -1.0\times 10^{-5} & 7.9\times 10^{-4} & -1.3\times 10^{-2} & -4.4\times 10^{-3} & -1.2\times 10^{-4} & -7.1\times 10^{-6} & 8.4\times 10^{-5} & -8.5\times 10^{-6} & 6.9\times 10^{-4} & -1.3\times 10^{-2} & -7.7\times 10^{-3} & -1.1\times 10^{-4} & -6.2\times 10^{-6} & 7.6\times 10^{-5} & -6.8\times 10^{-6} & 6.2\times 10^{-4} & -1.1\times 10^{-2} & -7.3\times 10^{-3} & -7.2\times 10^{-5} & -5.0\times 10^{-6} & 6.3\times 10^{-5} & -5.3\times 10^{-6} & 5.1\times 10^{-4} & -7.8\times 10^{-3} & -4.8\times 10^{-3} & -5.3\times 10^{-5} & -3.9\times 10^{-6} & 5.0\times 10^{-5} & -4.0\times 10^{-6} & 4.0\times 10^{-4} & -3.2\times 10^{-3} & -2.6\times 10^{-3} & -4.0\times 10^{-5} \\
%1.1\times 10^{-4} & -2.0\times 10^{-3} & 4.1\times 10^{-4} & -1.8\times 10^{-2} & 3.7\times 10^{+0} & 1.6\times 10^{+0} & 8.0\times 10^{-3} & 7.4\times 10^{-5} & -1.5\times 10^{-3} & 3.4\times 10^{-4} & -1.4\times 10^{-2} & 3.4\times 10^{+0} & 1.3\times 10^{+0} & 6.3\times 10^{-3} & 3.6\times 10^{-5} & -9.6\times 10^{-4} & 2.5\times 10^{-4} & -1.0\times 10^{-2} & 3.2\times 10^{+0} & 1.1\times 10^{+0} & 4.9\times 10^{-3} & 1.2\times 10^{-5} & -7.1\times 10^{-4} & 1.9\times 10^{-4} & -7.8\times 10^{-3} & 2.6\times 10^{+0} & 9.3\times 10^{-1} & 3.9\times 10^{-3} & -6.0\times 10^{-6} & -5.7\times 10^{-4} & 1.4\times 10^{-4} & -6.2\times 10^{-3} & 1.9\times 10^{+0} & 8.1\times 10^{-1} & 2.9\times 10^{-3} \\
%-1.7\times 10^{-5} & -1.6\times 10^{-3} & 8.0\times 10^{-4} & -7.7\times 10^{-3} & 8.5\times 10^{-1} & 9.6\times 10^{+0} & 3.0\times 10^{-3} & 5.1\times 10^{-5} & -2.0\times 10^{-3} & 6.4\times 10^{-4} & -8.3\times 10^{-3} & 7.8\times 10^{-1} & 9.5\times 10^{+0} & 3.2\times 10^{-3} & 8.4\times 10^{-5} & -1.6\times 10^{-3} & 5.1\times 10^{-4} & -7.0\times 10^{-3} & 1.0\times 10^{+0} & 8.6\times 10^{+0} & -3.9\times 10^{-4} & 7.2\times 10^{-5} & -1.1\times 10^{-3} & 4.1\times 10^{-4} & -4.8\times 10^{-3} & 9.3\times 10^{-1} & 7.3\times 10^{+0} & -1.4\times 10^{-3} & 2.3\times 10^{-5} & -3.9\times 10^{-4} & 3.1\times 10^{-4} & -1.6\times 10^{-3} & 7.8\times 10^{-1} & 5.7\times 10^{+0} & -2.1\times 10^{-3} \\
%1.1\times 10^{-6} & -7.9\times 10^{-6} & 4.6\times 10^{-6} & -1.2\times 10^{-4} & 3.0\times 10^{-3} & 3.2\times 10^{-3} & 1.4\times 10^{-4} & 6.9\times 10^{-7} & -5.6\times 10^{-6} & 3.7\times 10^{-6} & -9.2\times 10^{-5} & 4.6\times 10^{-3} & 3.9\times 10^{-6} & 1.2\times 10^{-4} & 2.4\times 10^{-7} & -3.4\times 10^{-6} & 2.8\times 10^{-6} & -6.7\times 10^{-5} & 5.0\times 10^{-3} & -1.6\times 10^{-3} & 1.1\times 10^{-4} & 4.6\times 10^{-8} & -3.0\times 10^{-6} & 2.2\times 10^{-6} & -5.3\times 10^{-5} & 3.9\times 10^{-3} & -1.4\times 10^{-3} & 8.9\times 10^{-5} & -3.2\times 10^{-8} & -3.6\times 10^{-6} & 1.6\times 10^{-6} & -4.5\times 10^{-5} & 1.7\times 10^{-3} & -4.8\times 10^{-5} & 7.2\times 10^{-5} \\
%2.2\times 10^{-7} & -7.3\times 10^{-7} & 1.4\times 10^{-7} & -6.4\times 10^{-6} & 3.9\times 10^{-5} & -10.0\times 10^{-5} & 9.2\times 10^{-7} & 1.9\times 10^{-7} & -6.2\times 10^{-7} & 1.1\times 10^{-7} & -5.3\times 10^{-6} & 5.8\times 10^{-5} & -2.3\times 10^{-5} & 6.7\times 10^{-7} & 1.7\times 10^{-7} & -5.7\times 10^{-7} & 9.4\times 10^{-8} & -4.7\times 10^{-6} & 2.6\times 10^{-5} & 2.1\times 10^{-5} & 2.1\times 10^{-7} & 1.4\times 10^{-7} & -4.8\times 10^{-7} & 7.7\times 10^{-8} & -3.9\times 10^{-6} & -6.0\times 10^{-6} & 2.3\times 10^{-5} & -3.2\times 10^{-8} & 1.3\times 10^{-7} & -4.2\times 10^{-7} & 6.6\times 10^{-8} & -3.3\times 10^{-6} & -5.8\times 10^{-5} & 3.4\times 10^{-5} & -2.4\times 10^{-7} \\
%-7.2\times 10^{-7} & 1.1\times 10^{-5} & -8.1\times 10^{-7} & 7.7\times 10^{-5} & -1.4\times 10^{-3} & -3.8\times 10^{-4} & -7.2\times 10^{-6} & -6.0\times 10^{-7} & 9.3\times 10^{-6} & -6.9\times 10^{-7} & 6.2\times 10^{-5} & -1.1\times 10^{-3} & -6.0\times 10^{-4} & -6.3\times 10^{-6} & -5.5\times 10^{-7} & 8.7\times 10^{-6} & -5.4\times 10^{-7} & 5.8\times 10^{-5} & -7.8\times 10^{-4} & -6.0\times 10^{-4} & -4.4\times 10^{-6} & -4.6\times 10^{-7} & 7.5\times 10^{-6} & -4.2\times 10^{-7} & 5.0\times 10^{-5} & -5.7\times 10^{-4} & -3.9\times 10^{-4} & -3.6\times 10^{-6} & -4.2\times 10^{-7} & 7.0\times 10^{-6} & -3.3\times 10^{-7} & 4.6\times 10^{-5} & -2.7\times 10^{-4} & -3.8\times 10^{-4} & -2.8\times 10^{-6} \\
%1.4\times 10^{-7} & -8.8\times 10^{-7} & 4.7\times 10^{-7} & -7.7\times 10^{-6} & 1.3\times 10^{-4} & 5.6\times 10^{-4} & 3.6\times 10^{-6} & 1.2\times 10^{-7} & -7.7\times 10^{-7} & 3.8\times 10^{-7} & -6.6\times 10^{-6} & 1.7\times 10^{-4} & 5.1\times 10^{-4} & 3.0\times 10^{-6} & 9.8\times 10^{-8} & -5.6\times 10^{-7} & 3.2\times 10^{-7} & -5.1\times 10^{-6} & 1.8\times 10^{-4} & 4.0\times 10^{-4} & 2.1\times 10^{-6} & 8.0\times 10^{-8} & -4.4\times 10^{-7} & 2.6\times 10^{-7} & -4.0\times 10^{-6} & 1.4\times 10^{-4} & 3.1\times 10^{-4} & 1.6\times 10^{-6} & 6.6\times 10^{-8} & -3.3\times 10^{-7} & 2.2\times 10^{-7} & -3.2\times 10^{-6} & 5.7\times 10^{-5} & 2.2\times 10^{-4} & 1.2\times 10^{-6} \\
%-6.3\times 10^{-6} & 7.6\times 10^{-5} & -7.4\times 10^{-6} & 6.3\times 10^{-4} & -1.1\times 10^{-2} & -7.6\times 10^{-4} & -9.9\times 10^{-5} & -5.3\times 10^{-6} & 6.3\times 10^{-5} & -6.2\times 10^{-6} & 5.1\times 10^{-4} & -1.0\times 10^{-2} & -3.1\times 10^{-3} & -8.3\times 10^{-5} & -4.6\times 10^{-6} & 5.8\times 10^{-5} & -5.0\times 10^{-6} & 4.7\times 10^{-4} & -8.2\times 10^{-3} & -3.1\times 10^{-3} & -5.8\times 10^{-5} & -3.9\times 10^{-6} & 5.0\times 10^{-5} & -4.0\times 10^{-6} & 4.0\times 10^{-4} & -6.2\times 10^{-3} & -1.6\times 10^{-3} & -4.5\times 10^{-5} & -3.3\times 10^{-6} & 4.6\times 10^{-5} & -3.2\times 10^{-6} & 3.7\times 10^{-4} & -3.0\times 10^{-3} & -1.2\times 10^{-3} & -3.3\times 10^{-5} \\
%-3.6\times 10^{-5} & -1.1\times 10^{-3} & 1.6\times 10^{-4} & -8.1\times 10^{-3} & 2.9\times 10^{+0} & 1.4\times 10^{+0} & 2.6\times 10^{-3} & -5.2\times 10^{-5} & -7.7\times 10^{-4} & 1.3\times 10^{-4} & -6.0\times 10^{-3} & 2.4\times 10^{+0} & 1.1\times 10^{+0} & 2.1\times 10^{-3} & -6.3\times 10^{-5} & -4.2\times 10^{-4} & 9.2\times 10^{-5} & -4.0\times 10^{-3} & 2.2\times 10^{+0} & 8.8\times 10^{-1} & 1.9\times 10^{-3} & -6.4\times 10^{-5} & -3.0\times 10^{-4} & 7.3\times 10^{-5} & -3.2\times 10^{-3} & 1.9\times 10^{+0} & 7.8\times 10^{-1} & 1.7\times 10^{-3} & -5.8\times 10^{-5} & -2.7\times 10^{-4} & 5.7\times 10^{-5} & -3.0\times 10^{-3} & 1.8\times 10^{+0} & 6.9\times 10^{-1} & 1.5\times 10^{-3} \\
%-6.7\times 10^{-6} & -1.2\times 10^{-3} & 5.9\times 10^{-4} & -4.9\times 10^{-3} & 6.7\times 10^{-1} & 7.9\times 10^{+0} & 2.3\times 10^{-3} & 9.5\times 10^{-6} & -1.1\times 10^{-3} & 4.8\times 10^{-4} & -4.1\times 10^{-3} & 6.5\times 10^{-1} & 7.1\times 10^{+0} & 2.0\times 10^{-3} & 3.2\times 10^{-5} & -9.0\times 10^{-4} & 3.6\times 10^{-4} & -3.9\times 10^{-3} & 8.9\times 10^{-1} & 6.4\times 10^{+0} & 6.8\times 10^{-4} & 3.0\times 10^{-5} & -6.5\times 10^{-4} & 2.8\times 10^{-4} & -2.6\times 10^{-3} & 8.1\times 10^{-1} & 5.7\times 10^{+0} & -4.8\times 10^{-5} & 3.4\times 10^{-5} & -3.8\times 10^{-4} & 2.2\times 10^{-4} & -1.2\times 10^{-3} & 6.9\times 10^{-1} & 5.4\times 10^{+0} & -6.4\times 10^{-4} \\
%4.7\times 10^{-7} & -5.2\times 10^{-6} & 3.1\times 10^{-6} & -7.9\times 10^{-5} & 2.1\times 10^{-3} & 1.8\times 10^{-3} & 1.1\times 10^{-4} & 2.4\times 10^{-7} & -4.0\times 10^{-6} & 2.6\times 10^{-6} & -6.7\times 10^{-5} & 3.2\times 10^{-3} & -1.3\times 10^{-3} & 8.9\times 10^{-5} & -5.5\times 10^{-8} & -2.8\times 10^{-6} & 2.0\times 10^{-6} & -5.1\times 10^{-5} & 3.6\times 10^{-3} & -2.5\times 10^{-3} & 8.3\times 10^{-5} & -1.9\times 10^{-7} & -2.4\times 10^{-6} & 1.6\times 10^{-6} & -4.0\times 10^{-5} & 2.9\times 10^{-3} & -2.1\times 10^{-3} & 7.2\times 10^{-5} & -2.4\times 10^{-7} & -2.8\times 10^{-6} & 1.2\times 10^{-6} & -3.3\times 10^{-5} & 1.5\times 10^{-3} & -6.4\times 10^{-4} & 6.6\times 10^{-5} \\
%\end{bmatrix}
%\end{equation}
\bibliographystyle{aasjournal}
\bibliography{alex}
%\documentclass[trackchanges]{aastex62}   	% use "amsart" instead of "article" for AMSLaTeX format
\usepackage{geometry}                		% See geometry.pdf to learn the layout options. There are lots.
\geometry{letterpaper}                   		% ... or a4paper or a5paper or ... 
\usepackage{graphicx}				% Use pdf, png, jpg, or eps§ with pdflatex; use eps in DVI mode
\usepackage{amsmath}
\usepackage{amssymb}
\usepackage{natbib}
\usepackage{lineno}
\usepackage{color}
\defcitealias{1999PASP..111...63F}{F99}
\defcitealias{2017ApJ...842...93M}{M17}
\linenumbers

\begin{document}

\title{Evidence for a Third Color  Parameter Within the Type~Ia Supernovae of the Nearby Supernova Factory}
\author[0000-0001-6315-8743]{A.~G.~Kim}
\affiliation{    Physics Division, Lawrence Berkeley National Laboratory, 
    1 Cyclotron Road, Berkeley, CA, 94720}
    
\author{     G.~Aldering}
\affiliation{    Physics Division, Lawrence Berkeley National Laboratory, 
    1 Cyclotron Road, Berkeley, CA, 94720}

\author{     P.~Antilogus}
\affiliation{    Sorbonne Universit\'e, Universit\'e Paris Diderot, CNRS/IN2P3, Laboratoire de Physique Nucl\'eaire et de Hautes Energies, LPNHE, 4 Place Jussieu, F-75252, Paris, France}
    
\author{     S.~Bailey}
\affiliation{    Physics Division, Lawrence Berkeley National Laboratory, 
    1 Cyclotron Road, Berkeley, CA, 94720}

\author{     C.~Baltay}
\affiliation{    Department of Physics, Yale University, 
    New Haven, CT, 06250-8121}

\author{     K.~Barbary}
\affiliation{
    Department of Physics, University of California Berkeley,
    366 LeConte Hall MC 7300, Berkeley, CA, 94720-7300}

\author{    D.~Baugh}
\affiliation{   Tsinghua Center for Astrophysics, Tsinghua University, Beijing 100084, China }

\author{     K.~Boone}
\affiliation{    Physics Division, Lawrence Berkeley National Laboratory, 
    1 Cyclotron Road, Berkeley, CA, 94720}
\affiliation{
    Department of Physics, University of California Berkeley,
    366 LeConte Hall MC 7300, Berkeley, CA, 94720-7300}

\author{     S.~Bongard}
\affiliation{ Sorbonne Universit\'e, Universit\'e Paris Diderot, CNRS/IN2P3, Laboratoire de Physique Nucl\'eaire et de Hautes Energies, LPNHE, 4 Place Jussieu, F-75252, Paris, France}

\author{     C.~Buton}
\affiliation{    Universit\'e de Lyon, F-69622, Lyon, France ; Universit\'e de Lyon 1, Villeurbanne ; 
    CNRS/IN2P3, Institut de Physique Nucl\'eaire de Lyon}
    
\author{     J.~Chen}
\affiliation{   Tsinghua Center for Astrophysics, Tsinghua University, Beijing 100084, China }

\author{     N.~Chotard}
\affiliation{    Universit\'e de Lyon, F-69622, Lyon, France ; Universit\'e de Lyon 1, Villeurbanne ; 
    CNRS/IN2P3, Institut de Physique Nucl\'eaire de Lyon}
    
\author[0000-0002-5317-7518]{     Y.~Copin}
\affiliation{    Universit\'e de Lyon, F-69622, Lyon, France ; Universit\'e de Lyon 1, Villeurbanne ; 
    CNRS/IN2P3, Institut de Physique Nucl\'eaire de Lyon}

\author{ S.~Dixon}
\affiliation{
    Department of Physics, University of California Berkeley,
    366 LeConte Hall MC 7300, Berkeley, CA, 94720-7300}

\author{     P.~Fagrelius}
\affiliation{    Physics Division, Lawrence Berkeley National Laboratory, 
    1 Cyclotron Road, Berkeley, CA, 94720}
\affiliation{
    Department of Physics, University of California Berkeley,
    366 LeConte Hall MC 7300, Berkeley, CA, 94720-7300}

\author{     H.~K.~Fakhouri}
\affiliation{    Physics Division, Lawrence Berkeley National Laboratory, 
    1 Cyclotron Road, Berkeley, CA, 94720}
  \affiliation{
    Department of Physics, University of California Berkeley,
    366 LeConte Hall MC 7300, Berkeley, CA, 94720-7300}

\author{     U.~Feindt}
\affiliation{The Oskar Klein Centre, Department of Physics, AlbaNova, Stockholm University, SE-106 91 Stockholm, Sweden}

\author{     D.~Fouchez}
\affiliation{ Aix Marseille Univ, CNRS/IN2P3, CPPM, Marseille, France
}
    
\author[0000-0001-6728-1423]{     E.~Gangler}  
\affiliation{    Clermont Universit\'e, Universit\'e Blaise Pascal, CNRS/IN2P3, Laboratoire de Physique Corpusculaire,
    BP 10448, F-63000 Clermont-Ferrand, France}
    
\author{     B.~Hayden}
\affiliation{    Physics Division, Lawrence Berkeley National Laboratory, 
    1 Cyclotron Road, Berkeley, CA, 94720}

\author{     W.~Hillebrandt}
\affiliation{    Max-Planck-Institut f\"ur Astrophysik, Karl-Schwarzschild-Str. 1,
D-85748 Garching, Germany}

\author{     M.~Kowalski}
\affiliation{    Institut fur Physik,  Humboldt-Universitat zu Berlin,
    Newtonstr. 15, 12489 Berlin}
\affiliation{ DESY, D-15735 Zeuthen, Germany}

\author{     P.-F.~Leget}
\affiliation{    Clermont Universit\'e, Universit\'e Blaise Pascal, CNRS/IN2P3, Laboratoire de Physique Corpusculaire,
    BP 10448, F-63000 Clermont-Ferrand, France}
    
\author{     S.~Lombardo}
\affiliation{    Institut fur Physik,  Humboldt-Universitat zu Berlin,
    Newtonstr. 15, 12489 Berlin}
    
\author{     J.~Nordin}
\affiliation{    Institut fur Physik,  Humboldt-Universitat zu Berlin,
    Newtonstr. 15, 12489 Berlin}
    
\author{     R.~Pain}
\affiliation{ Sorbonne Universit\'e, Universit\'e Paris Diderot, CNRS/IN2P3, Laboratoire de Physique Nucl\'eaire et de Hautes Energies, LPNHE, 4 Place Jussieu, F-75252, Paris, France}
     
\author{     E.~Pecontal}
\affiliation{   Centre de Recherche Astronomique de Lyon, Universit\'e Lyon 1,
    9 Avenue Charles Andr\'e, 69561 Saint Genis Laval Cedex, France}
    
\author{    R.~Pereira}
 \affiliation{    Universit\'e de Lyon, F-69622, Lyon, France ; Universit\'e de Lyon 1, Villeurbanne ; 
    CNRS/IN2P3, Institut de Physique Nucl\'eaire de Lyon}
 
 \author{    S.~Perlmutter}
 \affiliation{    Physics Division, Lawrence Berkeley National Laboratory, 
    1 Cyclotron Road, Berkeley, CA, 94720} 
\affiliation{
    Department of Physics, University of California Berkeley,
    366 LeConte Hall MC 7300, Berkeley, CA, 94720-7300}
    
 \author{    D.~Rabinowitz}
 \affiliation{    Department of Physics, Yale University, 
    New Haven, CT, 06250-8121}
    
 \author[0000-0002-8121-2560]{    M.~Rigault} 
\affiliation{ Clermont Universit\'e, Universit\'e Blaise Pascal, CNRS/IN2P3, Laboratoire de Physique Corpusculaire,
    BP 10448, F-63000 Clermont-Ferrand, France}
     
 \author{    D.~Rubin}
 \affiliation{    Physics Division, Lawrence Berkeley National Laboratory, 
    1 Cyclotron Road, Berkeley, CA, 94720}
    \affiliation{   Space Telescope Science Institute, 3700 San Martin Drive, Baltimore, MD 21218}
 
 \author{    K.~Runge}
 \affiliation{    Physics Division, Lawrence Berkeley National Laboratory, 
    1 Cyclotron Road, Berkeley, CA, 94720}
 
 \author{    C.~Saunders}
 \affiliation{    Physics Division, Lawrence Berkeley National Laboratory, 
    1 Cyclotron Road, Berkeley, CA, 94720}
 \affiliation{Sorbonne Universit\'es, Institut Lagrange de Paris (ILP), 98 bis Boulevard Arago, 75014 Paris, France}

\author{    C.~Sofiatti}
\affiliation{    Physics Division, Lawrence Berkeley National Laboratory, 
    1 Cyclotron Road, Berkeley, CA, 94720} 
\affiliation{
    Department of Physics, University of California Berkeley,
    366 LeConte Hall MC 7300, Berkeley, CA, 94720-7300}

\author{    N.~Suzuki}
\affiliation{    Physics Division, Lawrence Berkeley National Laboratory, 
    1 Cyclotron Road, Berkeley, CA, 94720}

\author{     S.~Taubenberger}
\affiliation{    Max-Planck-Institut f\"ur Astrophysik, Karl-Schwarzschild-Str. 1,
D-85748 Garching, Germany}

\author{     C.~Tao}
\affiliation{   Tsinghua Center for Astrophysics, Tsinghua University, Beijing 100084, China }
\affiliation{ Aix Marseille Univ, CNRS/IN2P3, CPPM, Marseille, France
}
   
\author{     R.~C.~Thomas}
\affiliation{    Computational Cosmology Center, Computational Research Division, Lawrence Berkeley National Laboratory, 
    1 Cyclotron Road MS 50B-4206, Berkeley, CA, 94720}
    
\collaboration{(The Nearby Supernova Factory)}


\begin{abstract}
Through empirical modeling of its observed signal, the peak absolute magnitude of a Type~Ia supernova (SN~Ia) can be accurately determined,
making SNe~Ia  excellent distance indicators.  Improved modeling of SN~Ia colors and magnitudes can account for
further physical
diversity that is expected but not included
in current models, and so consequently can lead to more precise per-object distances with smaller systematic uncertainties.  In this article, we present   
an empirical model for SN~Ia peak colors with
three latent color parameters and dependence on the equivalent widths of Ca~II and Si~II, the Si~II velocity,
and the light-curve shape.
This model is applied to the supernova sample of the Nearby Supernova Factory.  The peak magnitudes in synthetic
broadband photometry and their colors are found to be 
dependent on the observable features and on the three color parameters.
The color changes allowed by two of the color parameters are similar to those allowed by the extinction model of  \citet{1999PASP..111...63F}. 
Correspondingly the sample has an effective value of $\langle R^F_{\mathit{eff}}\rangle \sim 2.43$, though with a broad distribution of 
total-to-selective extinctions. 
We detect the influence on color by a third parameter at
%--- table.py
$> 99\%$
%-----
confidence.
The third parameter has  unique color characteristics that distinguish it from 
those already  tracked by the spectral features, light-curve shape, and two-parameter dust models.
These color parameters must be accounted for when using SNe~Ia as distance indicators.
\end{abstract}

\keywords{supernovae: general; cosmology: distance scale; methods: data analysis}

\section{Introduction}
Type~Ia supernovae (SNe~Ia) form a homogenous set of exploding stars and as such were early recognized and utilized as a powerful distance indicator 
and probe of cosmology \citep[e.g.][]{1992ARA&A..30..359B, 1993ApJ...415....1S}.  After further careful consideration of supernova data, it was recognized
that SN~Ia light-curve shapes \citep{1993ApJ...413L.105P} and colors \citep{1996ApJ...473...88R, 1998A&A...331..815T} exhibit subtle signs of heterogeneity
that are correlated with absolute magnitude, and must be considered when inferring distances.  Empirical models parameterizing SNe~Ia by their light-curve shape \citep{1996ApJ...473...88R,
1997ApJ...483..565P,
1999ApJ...517..565P}
and color  \citep{1996ApJ...473...88R}  were developed that enabled absolute magnitude corrections
and accurate distance measurements of cosmological supernovae,
which 
were subsequently used in the discovery of the accelerating expansion of the Universe \citep{1998AJ....116.1009R,1999ApJ...517..565P}.

The two most commonly used supernova-cosmology light-curve fitters today are SALT2 \citep{2007A&A...466...11G} and MLCS2k2
\citep{2007ApJ...659..122J}.\footnote{Light-curve fitters with more flexible degrees of freedom
\citep[e.g.][]{2008ApJ...681..482C, 2011AJ....141...19B, 2011ApJ...731..120M} are available and have for
the most part been used to study SN~Ia heterogeneity.}
They remain two-parameter models, with one parameter
primarily
characterizing light-curve shape and the other
color.
In SALT2 the light curve shapes are described by phase-dependent flux corrections,
 whereas MLCS2k2 varies shapes through additive magnitude corrections.
The physical cause of the color diversity is interpreted differently by the two sets of authors: 
\citet{2007A&A...466...11G} pragmatically extract color variation empirically from SNe that span a wide range of colors, with no attribution
to either intrinsic or extrinsic origins;
\citet{2007ApJ...659..122J}
attribute changes in color
partially to intrinsic variations linked to light-curve shape, and partially
to the reddening of light from host-galaxy dust.  Differences between these models produce differences in the results of
analyses of both low-redshift \citep{2007ApJ...664L..13C} and high-redshift \citep{2009ApJS..185...32K} supernovae.

There is evidence that supports the expectation that a single parameter beyond light-curve shape  cannot describe the full range
of colors seen in the SN~Ia population.  One approach to look for color diversity is to find correlations between color and spectral features.
\citet{2009ApJ...699L.139W, 2011ApJ...729...55F} find two subpopulations distinguished
by Si~II velocity that exhibit differing $B_{\mathit{\mathit{max}}}-V_{\mathit{max}}$; this color correlation, in addition to one with $B-R$, is confirmed by
\citet{2014ApJ...797...75M}.
\citet{2009ApJ...699L.139W,2015MNRAS.451.1973S}
find that high-velocity Si~II~$\lambda$6355 is found in objects that have red ultraviolet/optical colors near maximum brightness.
\citet{2011MNRAS.413.3075M} show evidence that supernova asymmetry and viewing angle,
traced by wavelength shifts in nebular emission lines, is an important determinant in controlling supernova color; such correlations are also seen by \citet{2011A&A...534L..15C}.

Another approach to probe color diversity is through multiple colors (at least 3 bands)
of individual supernovae.  Color ratios are sensitive to processes of the responsible physics.   For example,
relative dust absorption varies as a function of wavelength depending on grain size, distribution, composition and shape,
independent (to first order) of the amount of dust along the line of sight.
Near and mid-UV photometry obtained by the Ultra-Violet/Optical Telescope on the Swift spacecraft cannot be entirely explained
by dust absorption and hence imply intrinsic variability of supernova UV colors
\citep{2017ApJ...836..232B}.
Measurements of color ratios are being advanced with the development of flexible empirical light curve models that accommodate flexibility in multi-band colors
\citep[e.g.][]{2011ApJ...731..120M}.
\citet{2014ApJ...789...32B, 2015MNRAS.453.3300A} find wide
ranges of total-to-selective extinction with average values significantly lower than $R_V = 3.1$,
the canonical value for diffuse Milky Way dust.
They also confirm the \citet{2006MNRAS.369.1880E, 2011ApJ...731..120M} finding that low $R_V$ is associated with high-extinction supernovae.
In contrast, \citet{2011A&A...529L...4C} argue that after accounting for the diversity of spectral features,
the $R_V=3.1$ measured for the diffuse Milky Way dust is recovered on average and \citet{2017ApJ...836..157H}
find $R_V=2.95 \pm 0.08$ for the highly-extincted SN~2012cu.

Even two parameters are probably not enough to capture supernova color diversity.  SNe~Ia 
are affected by interstellar Milky Way-like dust  and by ``intrinsic'' color variations from the supernova itself and its surrounding circumstellar dust. Physical modeling
shows that the complexity of interstellar dust shouldn't be expected to be captured by a single parameter \citep{2015ApJ...807L..26G,
2017ApJ...836...13H}. Supernova models \citep[e.g.][]{2007ApJ...662..487W,2013MNRAS.436..333S,2014MNRAS.441..532D,2015MNRAS.454.2549B,
2017ApJ...846...58H} and circumstellar dust \citep{2005ApJ...635L..33W,
2008ApJ...686L.103G} produce color variations beyond those due to interstellar dust.  A third color
parameter is therefore expected.

Hierarchical modeling has recently enabled
the study of intrinsic supernova color based on SN~Ia Hubble diagrams
by allowing the addition of a supernova population model as an extra level within in the larger model.
Latent parameters that are not directly tied to observables
but  influence color can be included in such models.
\citet{2017ApJ...842...93M} \citepalias[henceforth referred to as][]{2017ApJ...842...93M}) take the approach of modeling the distribution of their parameters to find that
scatter in the Hubble diagram is better explained by a combination of 
an intrinsic color-magnitude relationship with slope $\beta_{\mathit{int}}\sim 2.3$ intrinsic
color dispersion,
and
$R_V=2.7$ dust, rather than by dust with no color dispersion.
They draw these conclusions by using only the SALT2 (v2.4)
 $c$
parameter as the summary statistic that describes color.

The Nearby Supernova Factory \citep[SNfactory;][]{2002SPIE.4836...61A} has systematically observed the
spectrophotometric time series of hundreds of Hubble-flow $0.03<z<0.08$ SNe~Ia.   The $3200$--$10000$~\AA\ spectral coverage
provides measurements of an array of supernova spectral features while also enablng synthetic broadband photometry
spanning near-UV to near-IR SN-frame wavelengths.  SNfactory specifically targeted objects
early in their temporal evolution, so that well over a hundred of these supernovae have  coverage over
peak brightness.  This dataset provides a homogenous sample with which to study SN~Ia colors and spectral features simultaneously.

In this article we use the idea that spectral indicators carry information on intrinsic supernova colors at peak magnitude.
This approach is taken by \citet{2011A&A...529L...4C}, who find that after standardization based on Ca and Si features, remaining residual color
variation is consistent with Milky Way dust models.
We accommodate up to  three independent color parameters.
The data used in this analysis are described in \S\ref{data:sec}.
The analysis itself is presented as a series of three models of increasing complexity.
\S\ref{modelI:sec}  and  \S\ref{modelII:sec} present the first two models, providing an introduction to
our methodology and yielding results to be compared with those from the third model.
That ultimate model and its results are discussed in detail in \S\ref{modelIII:sec}. 
In \S\ref{modelI:sec} we present Model~I, 
a first analysis using spectral features to standardize SN colors
and two additional latent parameters to account for color diversity. 
By latent, we mean that the parameters are not directly associated with a single observable but rather deduced
as part of a hierarchical
model.
For Model~II described in \S\ref{modelII:sec}, we add a third latent color parameter to the analysis.
In \S\ref{modelIII:sec} we present our final Model~III, which  includes light-curve shape as an additional standardization parameter.
Conclusions are presented in \S\ref{conclusions:sec}.

\section{Data}
\label{data:sec}

Our analysis uses the spectrophotometric data set obtained by
the SNfactory with the SuperNova Integral Field
Spectrograph \citep[SNIFS,][]{2002SPIE.4836...61A, 2004SPIE.5249..146L}.  SNIFS is a fully integrated
instrument optimized for automated observation of point sources on a
structured background over the full ground-based optical window at
moderate spectral resolution ($R \sim 500$).  It consists of a
high-throughput wide-band lenslet integral field spectrograph, a multi-band
imager that covers the field in the vicinity of
the IFS for atmospheric transmission monitoring simultaneous with
spectroscopy, and an acquisition/guiding channel.  The IFS possesses a
fully-filled $6\farcs 4 \times 6\farcs 4$ spectroscopic field of view
subdivided into a grid of $15 \times 15$ spatial elements, a
dual-channel spectrograph covering 3200--5200~\AA\ and 5100--10000~\AA\
simultaneously, and an internal calibration unit (continuum and arc
lamps).  SNIFS is mounted on the south bent Cassegrain port of the
University of Hawaii 2.2~m telescope on Mauna Kea, and is operated
remotely.  Observations are reduced using the SNfactory's dedicated data
reduction pipeline, similar to that presented in \S4 of \citet{2001MNRAS.326...23B}.
A discussion of the software pipeline is presented in
\citet{2006ApJ...650..510A} and is updated in \citet{2010ApJ...713.1073S}. 
The flux calibration is presented in \citet{2013A&A...549A...8B}.
A detailed
description of host-galaxy subtraction is given in \citet{2011MNRAS.418..258B}.

\added{The SNfactory sample used here consists of 198 SNe~Ia
having fully-processed spectrophotometric lightcurve data, including
observations on at least two photometric nights, final references, and a
host spectroscopic redshift.   This sample was defined for use in \citet{leget:thesis}.
These supernovae have well-sampled lightcurves as defined by having
at minimum 5 spectra while still active, with at least four within
epochs  $[-10, 35]$ days, at least one within $[-10, 7]$ days, at least one within $[7, 20]$ days,
and at least one color (Anyone know what this means? See slide 3 of
\url{https://snf-doc.lbl.gov/twiki/pub/Conf/VideoConf20160317/IDR_03-17-2016.pdf}) within $[-8, 10]$ days of peak $B$ brightness.
Synthetic-photometric lightcurves derived from the spectra are fit to the SALT2~(v2.4)
model; we require that supernovae have no more than 20\% points that are
$>0.2$ mag from the best-fit SALT2 model, and require a Normalized Median Absolute Deviation (nMAD) of residuals
of  $< 0.12$ mag relative to this model.  \citep[It is known that more flexibility than available to SALT2 is required to capture
the full range of observed SN~Ia behavior, e.g.\ ][]{2011AJ....141...19B, 2011ApJ...731..120M}.
%These all have at least 5 spectra while the SN is
%active, and pass the quality cuts suggested by \citet{2010A&A...523A...7G}.
More than 80\% of our SNe are from searches where there was no
pre-selection based on host galaxy properties (those
whose names start with ``SNF'', ``LSQ'', or ``PTF'' in Table~\ref{data:tab}).

\citet{2013ApJ...770..108C}  present the distribution of host-galaxy parameters for a subset of the  above sample.
\citet{2015MNRAS.447.1247S} show scatter plots of the peak Si~II~$\lambda 5972$ and
Si~II~$\lambda 6355$ velocities and pseudo-equivalent
widths, absolute $B$ and $V$ magnitudes, $B-V$, SALTII $c$ and $x_1$ parameters, and $\Delta m_{15}$ for 
an preceding sample of 151 SNe.   

%Quality criteria for a 'good' lightcurve are the following: At least 5 nights of observation
%No more than 20\% of rejection (after outlier cut>0.2mag) nMAD of residuals < 0.12 mag
%Phase coverage (4 epochs in [-10 < p < +35] days, 1 epochs in [-10 < p < +7] days, 1 epochs in [+7 < p < +20] days,1 color in [-8 < p < +10] days)

From that sample, we further require at least one spectrum with $\pm 2.5$ days of B-band peak brightness.
We exclude six SNe~Ia in the
SN~1991T, SN~1991bg and SN~2002cx subclasses,
SNF20080905-005 described in \citet{leget:thesis} whose spectral features make it a $>3\sigma$ outlier in the
sample population, and
the extremely red SN2012cu  who has a SALT2 $C \approx 1$ \citep{2017ApJ...836..157H} that is several times larger than the colors
of in-sample supernovae.

Considering that our empirical parameters
are not fundamental supernova parameters, their best-fit values are dependent on sample selection.
Nevertheless, this  article seeks
underlying parameters that significantly
affect SN~Ia colors; any such parameters found in a subsample are still
relevant parameters when dealing with the population as a whole.
Within this context, it is not so important whether our sample population is representative
of the underlying population.

There is a fraction of SNfactory-observed supernovae that do have the
observations to satisfy the well-sampled light-curve criteria, but are excluded due to some observations
having not completed full processing.  Incomplete processing can occur for data taken during poor
observing conditions, when there are instrumental artifacts, and late in a supernova's
light-curve when the signal-to-noise and contrast relative to the host are low.  The sample-selection exclusion of these supernovae are not related,
at least directly, to the at-maximum observables used in this analysis.  Our results are insensitive to potential sample-selection
bias due to stochasticity in the measurements.
}


A recent description of the data is presented in \citet{2015ApJ...815...58F}.
We provide a brief summary of the points important for this analysis.
The spectral time-series  are corrected for Milky Way dust
extinction \citep{1989ApJ...345..245C,1998ApJ...500..525S}.  
Each spectral time series is
blue-shifted to rest-frame
based on the systemic redshift of the host \citep[c.f.][]{2013ApJ...770..107C}, and the fluxes are converted to luminosity assuming
distances expected for the supernova redshifts given a flat
$\Lambda$CDM cosmology with $\Omega_M = 0.28$ (with an arbitrarily selected
$H_0$ since the current analysis does not depend on absolute magnitude).

This article presents the first application of a computationally intensive analysis, so although  high spectral resolution
is available we here compress the information into low-resolution broadband photometry. 
Synthetic supernova-frame photometry is generated for a top-hat filter system
comprised of five 
bands with the following wavelength ranges: ${\hat{U}}$ $[3300.00 - 3978.02]$\AA;
${\hat{B}}$ $[3978.02-4795.35]$\AA;
$\hat{V}$ $[4795.35-5780.60]$\AA;
$\hat{R}$ $[5780.60-6968.29]$\AA;
$\hat{I}$ $[6968.29-8400.00]$\AA.
(The diacritic hat serves as a reminder that these are not standard Johnson-Cousins filters.)
For each supernova, the magnitudes within 5-days of peak brightness are used to fit single-band magnitudes
at $B$-band peak brightness.
The equivalent widths of the Si~II~$\lambda 4141$ and Ca~II H\&K features are computed as
in \citet{2008A&A...477..717B} and the 
wavelength of the Si~II~$\lambda 6355$ feature
as in \citet{chotard:thesis, 2017Chotard}.
Equivalent widths and the
Si~II~$\lambda 6355$ wavelength are taken from spectra  within $\pm 2.5$ days from $B$-band maximum;
the average is used  in cases where there are multiple spectral measurements within that time window.
All the above spectral features have units of \AA ngstroms.

Our analysis sample is comprised of the
172
supernovae that have been fully processed and have the data coverage to 
give photometric and spectroscopic statistics described above.
The 
spectroscopic-feature measurements, 
\replaced{SALT2 light-curve shape parameters  $x_1$,
and their uncertainties
are presented
in \citet{2017Chotard}}{
synthetic magnitudes (plus a constant offset), SALT2 light-curve shape parameters  $x_1$,
and their uncertainties
are presented
in 
Table~\ref{data:tab}. 
For the subset supernovae for whom it is available, we use the host-galaxy mass given in \citet{2013ApJ...770..108C}.

\startlongtable
\begin{deluxetable}{crrrrrrrrr}
\tabletypesize{\tiny}
\tablecaption{Supernova Spectral-Feature and Peak-Magnitude Data
\label{data:tab}}
\tablehead{
\colhead{Name} & \colhead{$EW_{\mathrm{Ca}}$ (\AA)} & \colhead{$EW_{\mathrm{Si}}$ (\AA)} & \colhead{$\lambda_{\mathrm{Si}}$ (\AA)} & \colhead{$U$+offset} & \colhead{$B$+offset} & \colhead{$V$+offset} & \colhead{$R$+offset} & \colhead{$I$+offset} & \colhead{$x_1$}
}
\startdata
SN2007bd & $109.7 \pm 5.9$ & $ 17.5 \pm 0.7$& $ 6101 \pm   3$ & $-29.31 \pm   0.01$ & $-29.12 \pm   0.01$& $-28.60 \pm   0.01$& $-28.35 \pm   0.01$& $-27.60 \pm   0.01$ & $ -0.85 \pm   0.34$\\
PTF10zdk & $149.7 \pm 1.2$ & $ 14.3 \pm 0.6$& $ 6150 \pm   3$ & $-28.61 \pm   0.02$ & $-28.69 \pm   0.02$& $-28.32 \pm   0.02$& $-28.08 \pm   0.02$& $-27.40 \pm   0.02$ & $  0.18 \pm   0.17$\\
SNF20080815-017 & $ 63.8 \pm 21.5$ & $ 27.6 \pm 3.8$& $ 6132 \pm   6$ & $-29.04 \pm   0.07$ & $-28.79 \pm   0.07$& $-28.32 \pm   0.07$& $-28.12 \pm   0.07$& $-27.41 \pm   0.07$ & $ -1.78 \pm   0.18$\\
PTF09dnl & $129.9 \pm 0.9$ & $  9.5 \pm 0.7$& $ 6093 \pm   3$ & $-29.23 \pm   0.01$ & $-29.07 \pm   0.01$& $-28.72 \pm   0.01$& $-28.44 \pm   0.01$& $-27.69 \pm   0.01$ & $  0.72 \pm   0.13$\\
SN2010ex & $114.4 \pm 0.9$ & $  8.4 \pm 0.4$& $ 6129 \pm   6$ & $-29.26 \pm   0.01$ & $-28.99 \pm   0.01$& $-28.50 \pm   0.01$& $-28.20 \pm   0.01$& $-27.44 \pm   0.01$ & $  0.27 \pm   0.14$\\
PTF09dnp & $ 64.9 \pm 4.5$ & $ 16.5 \pm 0.7$& $ 6098 \pm   4$ & $-29.55 \pm   0.02$ & $-29.19 \pm   0.02$& $-28.68 \pm   0.02$& $-28.48 \pm   0.02$& $-27.93 \pm   0.02$ & $ -0.74 \pm   0.24$\\
PTF11bnx & $151.4 \pm 3.0$ & $ 13.9 \pm 1.1$& $ 6142 \pm   5$ & $-28.63 \pm   0.02$ & $-28.57 \pm   0.01$& $-28.20 \pm   0.01$& $-27.99 \pm   0.01$& $-27.34 \pm   0.01$ & $ -0.07 \pm   0.17$\\
PTF12jqh & $151.9 \pm 1.5$ & $  7.9 \pm 0.7$& $ 6116 \pm  10$ & $-29.37 \pm   0.01$ & $-29.14 \pm   0.01$& $-28.71 \pm   0.01$& $-28.40 \pm   0.01$& $-27.64 \pm   0.01$ & $  0.69 \pm   0.19$\\
SNF20080802-006 & $108.2 \pm 6.0$ & $ 20.6 \pm 1.9$& $ 6122 \pm   5$ & $-29.02 \pm   0.06$ & $-28.80 \pm   0.06$& $-28.40 \pm   0.06$& $-28.20 \pm   0.06$& $-27.50 \pm   0.06$ & $ -1.56 \pm   0.13$\\
PTF10xyt & $123.7 \pm 6.6$ & $ 16.4 \pm 4.3$& $ 6101 \pm   4$ & $-28.26 \pm   0.02$ & $-28.20 \pm   0.02$& $-27.93 \pm   0.02$& $-27.74 \pm   0.02$& $-27.22 \pm   0.04$ & $  0.03 \pm   0.21$\\
PTF11qmo & $101.7 \pm 1.1$ & $  7.7 \pm 0.7$& $ 6150 \pm   8$ & $-29.77 \pm   0.02$ & $-29.43 \pm   0.02$& $-28.97 \pm   0.02$& $-28.64 \pm   0.02$& $-27.93 \pm   0.02$ & $ -0.00 \pm   0.34$\\
SNF20070331-025 & $119.8 \pm 7.4$ & $ 14.2 \pm 2.7$& $ 6120 \pm  10$ & $-28.94 \pm   0.02$ & $-28.75 \pm   0.02$& $-28.32 \pm   0.02$& $-28.07 \pm   0.02$& $-27.31 \pm   0.02$ & $ -1.28 \pm   0.16$\\
SNF20070818-001 & $157.5 \pm 7.5$ & $ 16.7 \pm 1.8$& $ 6115 \pm   5$ & $-28.97 \pm   0.02$ & $-28.96 \pm   0.01$& $-28.61 \pm   0.01$& $-28.37 \pm   0.01$& $-27.62 \pm   0.01$ & $  0.15 \pm   0.16$\\
SNBOSS38 & $ 57.1 \pm 0.4$ & $ 17.9 \pm 0.3$& $ 6127 \pm   3$ & $-29.20 \pm   0.01$ & $-28.84 \pm   0.01$& $-28.47 \pm   0.01$& $-28.23 \pm   0.01$& $-27.73 \pm   0.04$ & $  1.10 \pm   0.11$\\
SN2006ob & $ 90.0 \pm 16.5$ & $ 26.5 \pm 1.5$& $ 6112 \pm   5$ & $-29.11 \pm   0.02$ & $-28.82 \pm   0.01$& $-28.42 \pm   0.01$& $-28.19 \pm   0.01$& $-27.54 \pm   0.01$ & $ -2.45 \pm   0.28$\\
PTF12eer & $165.6 \pm 10.7$ & $ 12.7 \pm 2.8$& $ 6150 \pm  10$ & $-28.76 \pm   0.01$ & $-28.76 \pm   0.01$& $-28.40 \pm   0.01$& $-28.17 \pm   0.01$& $-27.45 \pm   0.02$ & $  0.68 \pm   0.16$\\
PTF10ops & $ 38.7 \pm 9.9$ & $  7.2 \pm 8.7$& $ 6140 \pm   5$ & $-27.93 \pm   0.38$ & $-27.76 \pm   0.38$& $-27.73 \pm   0.38$& $-27.59 \pm   0.38$& $-27.21 \pm   0.38$ & $ -0.62 \pm   0.13$\\
SNF20080514-002 & $ 83.2 \pm 0.7$ & $ 19.4 \pm 0.6$& $ 6131 \pm   3$ & $-29.30 \pm   0.01$ & $-28.95 \pm   0.01$& $-28.44 \pm   0.01$& $-28.17 \pm   0.01$& $-27.49 \pm   0.01$ & $ -1.57 \pm   0.11$\\
PTF12evo & $129.2 \pm 2.8$ & $  9.1 \pm 1.3$& $ 6156 \pm   4$ & $-29.14 \pm   0.02$ & $-28.98 \pm   0.01$& $-28.56 \pm   0.01$& $-28.28 \pm   0.01$& $-27.61 \pm   0.01$ & $  0.74 \pm   0.15$\\
SNF20080614-010 & $125.4 \pm 5.1$ & $ 26.9 \pm 1.6$& $ 6128 \pm   3$ & $-29.04 \pm   0.04$ & $-28.81 \pm   0.04$& $-28.38 \pm   0.04$& $-28.16 \pm   0.04$& $-27.57 \pm   0.04$ & $ -2.12 \pm   0.17$\\
PTF10icb & $104.8 \pm 0.9$ & $ 12.7 \pm 0.3$& $ 6138 \pm   3$ & $-28.58 \pm   0.02$ & $-28.36 \pm   0.02$& $-27.98 \pm   0.02$& $-27.77 \pm   0.02$& $-27.17 \pm   0.02$ & $  0.37 \pm   0.13$\\
PTF12efn & $144.9 \pm 3.4$ & $  7.1 \pm 1.8$& $ 6115 \pm   3$ & $-29.40 \pm   0.01$ & $-29.17 \pm   0.01$& $-28.79 \pm   0.01$& $-28.45 \pm   0.01$& $-27.64 \pm   0.01$ & $  1.21 \pm   0.20$\\
SNNGC4424 & $109.0 \pm 0.3$ & $  8.6 \pm 0.1$& $ 6138 \pm   2$ & $-28.35 \pm   0.01$ & $-28.15 \pm   0.01$& $-27.79 \pm   0.01$& $-27.58 \pm   0.01$& $-26.97 \pm   0.01$ & $  0.37 \pm   0.12$\\
SNF20080516-022 & $100.1 \pm 2.1$ & $ 13.7 \pm 1.1$& $ 6158 \pm   3$ & $-29.46 \pm   0.01$ & $-29.19 \pm   0.01$& $-28.71 \pm   0.01$& $-28.39 \pm   0.01$& $-27.77 \pm   0.01$ & $  0.20 \pm   0.14$\\
PTF12hwb & $ 21.1 \pm 78.0$ & $ -1.8 \pm 8.9$& $ 6090 \pm  14$ & $-28.32 \pm   0.02$ & $-28.24 \pm   0.02$& $-28.03 \pm   0.02$& $-27.79 \pm   0.02$& $-27.05 \pm   0.04$ & $  1.34 \pm   0.17$\\
PTF10qyz & $106.4 \pm 2.1$ & $ 23.0 \pm 1.0$& $ 6120 \pm   5$ & $-29.05 \pm   0.17$ & $-28.92 \pm   0.17$& $-28.41 \pm   0.17$& $-28.14 \pm   0.17$& $-27.30 \pm   0.17$ & $ -1.09 \pm   0.16$\\
SNF20060907-000 & $106.1 \pm 10.4$ & $ 17.0 \pm 0.9$& $ 6149 \pm   4$ & $-29.54 \pm   0.02$ & $-29.28 \pm   0.01$& $-28.76 \pm   0.01$& $-28.42 \pm   0.01$& $-27.74 \pm   0.04$ & $  0.17 \pm   0.17$\\
LSQ12fxd & $122.9 \pm 1.7$ & $ 11.4 \pm 0.8$& $ 6119 \pm   4$ & $-29.62 \pm   0.07$ & $-29.39 \pm   0.07$& $-28.95 \pm   0.07$& $-28.64 \pm   0.07$& $-27.91 \pm   0.07$ & $  0.05 \pm   0.15$\\
SNF20080821-000 & $105.1 \pm 2.2$ & $  8.6 \pm 1.3$& $ 6121 \pm   4$ & $-29.34 \pm   0.01$ & $-29.10 \pm   0.01$& $-28.73 \pm   0.01$& $-28.46 \pm   0.01$& $-27.82 \pm   0.01$ & $  0.70 \pm   0.19$\\
SNF20070802-000 & $158.3 \pm 3.3$ & $ 16.3 \pm 1.7$& $ 6102 \pm   5$ & $-28.90 \pm   0.01$ & $-28.81 \pm   0.01$& $-28.45 \pm   0.01$& $-28.22 \pm   0.01$& $-27.52 \pm   0.01$ & $  0.08 \pm   0.16$\\
PTF10wnm & $105.8 \pm 2.3$ & $  6.5 \pm 1.0$& $ 6124 \pm   3$ & $-29.38 \pm   0.01$ & $-29.07 \pm   0.01$& $-28.68 \pm   0.01$& $-28.37 \pm   0.01$& $-27.69 \pm   0.01$ & $  0.63 \pm   0.16$\\
PTF10mwb & $116.5 \pm 1.2$ & $ 19.8 \pm 0.9$& $ 6138 \pm   2$ & $-29.02 \pm   0.07$ & $-28.84 \pm   0.07$& $-28.40 \pm   0.07$& $-28.14 \pm   0.07$& $-27.52 \pm   0.07$ & $ -0.68 \pm   0.10$\\
SN2010dt & $116.2 \pm 14.9$ & $ 15.5 \pm 0.7$& $ 6138 \pm   6$ & $-29.30 \pm   0.01$ & $-29.15 \pm   0.01$& $-28.64 \pm   0.01$& $-28.35 \pm   0.01$& $-27.63 \pm   0.01$ & $ -0.03 \pm   0.10$\\
SNF20080623-001 & $149.1 \pm 1.4$ & $ 14.9 \pm 0.7$& $ 6131 \pm   3$ & $-29.11 \pm   0.01$ & $-28.97 \pm   0.01$& $-28.50 \pm   0.01$& $-28.22 \pm   0.01$& $-27.46 \pm   0.01$ & $  0.01 \pm   0.13$\\
LSQ12fhe & $ 42.8 \pm 1.2$ & $  4.0 \pm 3.1$& $ 6108 \pm   4$ & $-29.76 \pm   0.02$ & $-29.40 \pm   0.02$& $-29.04 \pm   0.02$& $-28.74 \pm   0.02$& $-28.11 \pm   0.02$ & $  1.07 \pm   0.19$\\
PTF11bju & $ 30.2 \pm 4.4$ & $  4.0 \pm 3.0$& $ 6139 \pm   5$ & $-29.47 \pm   0.02$ & $-29.10 \pm   0.01$& $-28.75 \pm   0.01$& $-28.45 \pm   0.01$& $-27.87 \pm   0.01$ & $  1.04 \pm   0.20$\\
PTF09fox & $117.6 \pm 2.7$ & $  9.1 \pm 1.0$& $ 6116 \pm   3$ & $-29.44 \pm   0.03$ & $-29.21 \pm   0.03$& $-28.72 \pm   0.03$& $-28.42 \pm   0.03$& $-27.68 \pm   0.03$ & $  0.68 \pm   0.27$\\
PTF13ayw & $104.6 \pm 2.4$ & $ 26.6 \pm 3.2$& $ 6115 \pm   6$ & $-29.16 \pm   0.02$ & $-28.82 \pm   0.02$& $-28.43 \pm   0.02$& $-28.20 \pm   0.02$& $-27.55 \pm   0.02$ & $ -1.72 \pm   0.29$\\
SNF20070810-004 & $126.7 \pm 1.8$ & $ 21.1 \pm 1.1$& $ 6118 \pm   7$ & $-29.22 \pm   0.01$ & $-29.10 \pm   0.01$& $-28.63 \pm   0.01$& $-28.34 \pm   0.01$& $-27.62 \pm   0.01$ & $  0.10 \pm   0.12$\\
PTF11mty & $111.4 \pm 2.3$ & $ 10.6 \pm 1.5$& $ 6138 \pm   5$ & $-29.54 \pm   0.01$ & $-29.23 \pm   0.01$& $-28.80 \pm   0.01$& $-28.46 \pm   0.01$& $-27.82 \pm   0.01$ & $  0.83 \pm   0.16$\\
SNF20080512-010 & $ 95.3 \pm 3.5$ & $ 23.3 \pm 1.5$& $ 6129 \pm   5$ & $-29.22 \pm   0.08$ & $-28.96 \pm   0.08$& $-28.50 \pm   0.08$& $-28.26 \pm   0.08$& $-27.56 \pm   0.08$ & $ -1.24 \pm   0.26$\\
PTF11mkx & $ 31.5 \pm 3.7$ & $  4.5 \pm 1.3$& $ 6169 \pm   5$ & $-29.50 \pm   0.45$ & $-29.25 \pm   0.45$& $-28.89 \pm   0.45$& $-28.61 \pm   0.45$& $-27.97 \pm   0.45$ & $  0.60 \pm   0.15$\\
PTF10tce & $135.7 \pm 1.1$ & $ 11.2 \pm 1.5$& $ 6090 \pm   4$ & $-29.13 \pm   0.02$ & $-28.99 \pm   0.01$& $-28.59 \pm   0.01$& $-28.31 \pm   0.01$& $-27.55 \pm   0.01$ & $  0.43 \pm   0.14$\\
SNF20061020-000 & $ 95.4 \pm 18.8$ & $ 24.1 \pm 1.0$& $ 6120 \pm   5$ & $-29.01 \pm   0.03$ & $-28.78 \pm   0.03$& $-28.35 \pm   0.03$& $-28.17 \pm   0.03$& $-27.54 \pm   0.03$ & $ -1.67 \pm   0.20$\\
SN2005ir & $115.6 \pm 2.8$ & $ 13.5 \pm 6.9$& $ 6069 \pm   5$ & $-29.33 \pm   0.02$ & $-29.12 \pm   0.02$& $-28.84 \pm   0.02$& $-28.49 \pm   0.02$& $-27.77 \pm   0.02$ & $  1.71 \pm   0.69$\\
SNF20080717-000 & $ 93.3 \pm 2.6$ & $  8.3 \pm 2.2$& $ 6104 \pm   3$ & $-28.58 \pm   0.01$ & $-28.47 \pm   0.01$& $-28.29 \pm   0.01$& $-28.05 \pm   0.01$& $-27.50 \pm   0.01$ & $  0.70 \pm   0.14$\\
PTF12ena & $101.1 \pm 1.6$ & $  7.4 \pm 1.0$& $ 6129 \pm   4$ & $-28.01 \pm   0.01$ & $-28.00 \pm   0.01$& $-27.85 \pm   0.01$& $-27.77 \pm   0.01$& $-27.31 \pm   0.01$ & $  0.65 \pm   0.16$\\
PTF13anh & $166.8 \pm 1.8$ & $ 21.8 \pm 1.2$& $ 6175 \pm   4$ & $-28.67 \pm   0.20$ & $-28.74 \pm   0.20$& $-28.30 \pm   0.20$& $-28.05 \pm   0.20$& $-27.28 \pm   0.20$ & $ -0.28 \pm   0.17$\\
CSS110918\_01 & $110.6 \pm 1.0$ & $  8.0 \pm 1.3$& $ 6101 \pm   2$ & $-29.88 \pm   0.76$ & $-29.58 \pm   0.76$& $-29.09 \pm   0.76$& $-28.71 \pm   0.76$& $-27.91 \pm   0.76$ & $  0.92 \pm   0.17$\\
SNF20061024-000 & $ 86.9 \pm 26.8$ & $ 30.0 \pm 1.5$& $ 6127 \pm   5$ & $-28.88 \pm   0.04$ & $-28.70 \pm   0.04$& $-28.26 \pm   0.04$& $-28.05 \pm   0.04$& $-27.40 \pm   0.04$ & $ -1.39 \pm   0.19$\\
SNF20070506-006 & $ 94.1 \pm 1.3$ & $  6.7 \pm 0.6$& $ 6153 \pm   3$ & $-29.72 \pm   0.01$ & $-29.39 \pm   0.01$& $-28.97 \pm   0.01$& $-28.64 \pm   0.01$& $-27.96 \pm   0.01$ & $  1.17 \pm   0.13$\\
SNF20070403-001 & $105.9 \pm 5.4$ & $ 18.3 \pm 1.8$& $ 6124 \pm   4$ & $-29.23 \pm   0.02$ & $-29.04 \pm   0.01$& $-28.63 \pm   0.01$& $-28.35 \pm   0.01$& $-27.62 \pm   0.01$ & $ -0.54 \pm   0.13$\\
PTF10hmv & $109.6 \pm 1.3$ & $  8.9 \pm 0.7$& $ 6143 \pm   3$ & $-28.54 \pm   0.01$ & $-28.40 \pm   0.01$& $-28.11 \pm   0.01$& $-27.89 \pm   0.01$& $-27.31 \pm   0.01$ & $  1.39 \pm   0.18$\\
SNF20071015-000 & $105.0 \pm 3.2$ & $  6.9 \pm 1.1$& $ 6124 \pm   7$ & $-27.89 \pm   0.02$ & $-27.82 \pm   0.02$& $-27.69 \pm   0.02$& $-27.63 \pm   0.02$& $-27.16 \pm   0.04$ & $  1.36 \pm   0.37$\\
SNhunt89 & $ 88.0 \pm 2.7$ & $ 32.2 \pm 1.9$& $ 6111 \pm   7$ & $-28.37 \pm   0.03$ & $-28.26 \pm   0.03$& $-27.92 \pm   0.03$& $-27.77 \pm   0.03$& $-27.13 \pm   0.03$ & $ -1.83 \pm   0.18$\\
SNF20070902-021 & $108.9 \pm 3.5$ & $ 17.1 \pm 1.0$& $ 6131 \pm   6$ & $-29.25 \pm   0.02$ & $-29.02 \pm   0.02$& $-28.56 \pm   0.02$& $-28.32 \pm   0.01$& $-27.65 \pm   0.02$ & $ -0.43 \pm   0.10$\\
PTF09dlc & $143.5 \pm 2.2$ & $ 10.2 \pm 0.9$& $ 6143 \pm   3$ & $-29.38 \pm   0.01$ & $-29.17 \pm   0.01$& $-28.69 \pm   0.01$& $-28.40 \pm   0.01$& $-27.62 \pm   0.01$ & $  0.11 \pm   0.12$\\
PTF13ajv & $150.5 \pm 8.9$ & $ 46.3 \pm 8.6$& $ 6110 \pm  21$ & $-28.70 \pm   0.02$ & $-28.61 \pm   0.02$& $-28.16 \pm   0.02$& $-27.91 \pm   0.02$& $-27.07 \pm   0.04$ & $ -1.59 \pm   0.22$\\
SNF20080919-000 & $114.7 \pm 2.8$ & $  9.4 \pm 0.9$& $ 6145 \pm   5$ & $-28.53 \pm   0.02$ & $-28.41 \pm   0.01$& $-28.11 \pm   0.01$& $-27.99 \pm   0.01$& $-27.38 \pm   0.01$ & $ -0.13 \pm   0.11$\\
SNF20080919-001 & $ 85.0 \pm 1.1$ & $  6.0 \pm 0.4$& $ 6150 \pm   5$ & $-29.73 \pm   0.01$ & $-29.43 \pm   0.01$& $-29.04 \pm   0.01$& $-28.72 \pm   0.01$& $-28.07 \pm   0.01$ & $  0.84 \pm   0.18$\\
SN2010kg & $ 95.1 \pm 28.5$ & $ 21.7 \pm 0.7$& $ 6077 \pm   5$ & $-28.85 \pm   0.01$ & $-28.74 \pm   0.01$& $-28.41 \pm   0.01$& $-28.20 \pm   0.01$& $-27.47 \pm   0.01$ & $ -2.26 \pm   0.40$\\
SNF20080714-008 & $134.8 \pm 15.7$ & $ 19.7 \pm 3.7$& $ 6100 \pm   6$ & $-28.56 \pm   0.02$ & $-28.63 \pm   0.01$& $-28.32 \pm   0.01$& $-28.13 \pm   0.01$& $-27.42 \pm   0.01$ & $  0.01 \pm   0.15$\\
SNF20070714-007 & $129.6 \pm 5.6$ & $ 31.1 \pm 23.8$& $ 6146 \pm   5$ & $-27.88 \pm   0.02$ & $-28.12 \pm   0.01$& $-28.02 \pm   0.01$& $-27.86 \pm   0.01$& $-27.24 \pm   0.03$ & $ -1.10 \pm   0.13$\\
SNF20080522-011 & $122.1 \pm 1.7$ & $  8.3 \pm 0.5$& $ 6125 \pm   3$ & $-29.63 \pm   0.01$ & $-29.38 \pm   0.01$& $-28.92 \pm   0.01$& $-28.60 \pm   0.01$& $-27.88 \pm   0.01$ & $  0.81 \pm   0.19$\\
SNF20061111-002 & $110.8 \pm 10.7$ & $ 20.4 \pm 1.0$& $ 6145 \pm   6$ & $-29.16 \pm   0.01$ & $-28.99 \pm   0.01$& $-28.59 \pm   0.01$& $-28.29 \pm   0.01$& $-27.61 \pm   0.01$ & $  0.27 \pm   0.21$\\
SNNGC6343 & $ 87.0 \pm 1.4$ & $ 20.7 \pm 0.7$& $ 6136 \pm   3$ & $-28.78 \pm   0.01$ & $-28.66 \pm   0.01$& $-28.30 \pm   0.01$& $-28.08 \pm   0.01$& $-27.41 \pm   0.01$ & $ -2.09 \pm   0.25$\\
SNF20061011-005 & $120.6 \pm 1.1$ & $  9.3 \pm 0.4$& $ 6132 \pm   4$ & $-29.72 \pm   0.04$ & $-29.43 \pm   0.03$& $-28.99 \pm   0.03$& $-28.64 \pm   0.03$& $-27.90 \pm   0.03$ & $  0.74 \pm   0.29$\\
SNF20080825-010 & $102.4 \pm 13.4$ & $ 19.2 \pm 0.6$& $ 6116 \pm   4$ & $-29.46 \pm   0.01$ & $-29.17 \pm   0.01$& $-28.71 \pm   0.01$& $-28.47 \pm   0.01$& $-27.83 \pm   0.01$ & $ -0.77 \pm   0.18$\\
PTF10ufj & $141.1 \pm 3.4$ & $ 11.7 \pm 1.2$& $ 6131 \pm   6$ & $-29.28 \pm   0.15$ & $-29.16 \pm   0.15$& $-28.72 \pm   0.15$& $-28.41 \pm   0.15$& $-27.65 \pm   0.15$ & $  0.53 \pm   0.16$\\
PTF10wof & $129.6 \pm 2.7$ & $ 17.3 \pm 1.0$& $ 6102 \pm   2$ & $-28.91 \pm   0.01$ & $-28.84 \pm   0.01$& $-28.46 \pm   0.01$& $-28.18 \pm   0.01$& $-27.43 \pm   0.01$ & $  0.16 \pm   0.13$\\
SNF20080918-000 & $146.8 \pm 3.5$ & $  7.5 \pm 2.5$& $ 6110 \pm   5$ & $-28.79 \pm   0.02$ & $-28.65 \pm   0.02$& $-28.35 \pm   0.02$& $-28.12 \pm   0.02$& $-27.46 \pm   0.02$ & $  1.50 \pm   0.30$\\
SNF20080516-000 & $117.4 \pm 2.2$ & $  9.0 \pm 1.2$& $ 6135 \pm   3$ & $-29.50 \pm   0.01$ & $-29.23 \pm   0.01$& $-28.80 \pm   0.01$& $-28.47 \pm   0.01$& $-27.74 \pm   0.01$ & $  1.24 \pm   0.26$\\
SN2005cf & $159.1 \pm 0.7$ & $ 15.7 \pm 0.8$& $ 6141 \pm   3$ & $-29.37 \pm   0.02$ & $-29.16 \pm   0.02$& $-28.68 \pm   0.02$& $-28.41 \pm   0.02$& $-27.69 \pm   0.02$ & $ -0.31 \pm   0.17$\\
CSS130502\_01 & $ 91.5 \pm 10.9$ & $ 15.6 \pm 0.5$& $ 6128 \pm   3$ & $-29.43 \pm   0.02$ & $-29.09 \pm   0.02$& $-28.60 \pm   0.01$& $-28.30 \pm   0.01$& $-27.62 \pm   0.04$ & $ -0.00 \pm   0.24$\\
SNF20080620-000 & $107.8 \pm 14.1$ & $ 20.0 \pm 0.7$& $ 6132 \pm   3$ & $-28.82 \pm   0.02$ & $-28.78 \pm   0.01$& $-28.32 \pm   0.01$& $-28.09 \pm   0.01$& $-27.39 \pm   0.01$ & $ -1.04 \pm   0.16$\\
SNPGC51271 & $ 92.1 \pm 16.5$ & $ 21.1 \pm 0.7$& $ 6121 \pm   2$ & $-29.28 \pm   0.02$ & $-28.95 \pm   0.02$& $-28.46 \pm   0.02$& $-28.20 \pm   0.02$& $-27.62 \pm   0.04$ & $ -2.08 \pm   0.18$\\
PTF11pdk & $128.6 \pm 2.8$ & $ 15.6 \pm 1.7$& $ 6153 \pm   5$ & $-29.35 \pm   0.02$ & $-29.11 \pm   0.02$& $-28.61 \pm   0.02$& $-28.32 \pm   0.02$& $-27.67 \pm   0.02$ & $  0.06 \pm   0.19$\\
SNF20060511-014 & $102.6 \pm 2.8$ & $ 15.6 \pm 1.1$& $ 6141 \pm   8$ & $-29.16 \pm   0.07$ & $-29.04 \pm   0.06$& $-28.56 \pm   0.06$& $-28.30 \pm   0.06$& $-27.63 \pm   0.06$ & $ -0.68 \pm   0.18$\\
SNF20080612-003 & $120.0 \pm 1.1$ & $  7.3 \pm 0.6$& $ 6123 \pm   3$ & $-29.64 \pm   0.02$ & $-29.41 \pm   0.02$& $-28.99 \pm   0.02$& $-28.70 \pm   0.02$& $-28.00 \pm   0.02$ & $  0.71 \pm   0.21$\\
SNF20080626-002 & $130.0 \pm 1.0$ & $  6.1 \pm 4.2$& $ 6111 \pm   3$ & $-29.42 \pm   0.01$ & $-29.24 \pm   0.01$& $-28.84 \pm   0.01$& $-28.52 \pm   0.01$& $-27.76 \pm   0.01$ & $  1.04 \pm   0.23$\\
SNF20060621-015 & $111.9 \pm 1.3$ & $  9.8 \pm 0.7$& $ 6144 \pm   3$ & $-29.63 \pm   0.01$ & $-29.36 \pm   0.01$& $-28.88 \pm   0.01$& $-28.54 \pm   0.01$& $-27.81 \pm   0.01$ & $  0.43 \pm   0.20$\\
SNF20080920-000 & $135.2 \pm 1.4$ & $  5.6 \pm 1.6$& $ 6085 \pm   3$ & $-29.44 \pm   0.02$ & $-29.19 \pm   0.02$& $-28.79 \pm   0.02$& $-28.49 \pm   0.02$& $-27.74 \pm   0.02$ & $  1.17 \pm   0.34$\\
SN2007cq & $ 65.8 \pm 4.1$ & $ 10.2 \pm 0.9$& $ 6137 \pm   3$ & $-29.53 \pm   0.02$ & $-29.30 \pm   0.02$& $-28.89 \pm   0.02$& $-28.56 \pm   0.02$& $-27.90 \pm   0.02$ & $ -0.43 \pm   0.16$\\
SNF20080918-004 & $ 87.8 \pm 7.2$ & $ 21.5 \pm 0.9$& $ 6141 \pm   4$ & $-29.00 \pm   0.22$ & $-28.82 \pm   0.22$& $-28.37 \pm   0.22$& $-28.13 \pm   0.22$& $-27.43 \pm   0.22$ & $ -1.27 \pm   0.18$\\
CSS120424\_01 & $138.1 \pm 2.1$ & $ 11.7 \pm 0.7$& $ 6138 \pm   3$ & $-29.40 \pm   0.02$ & $-29.23 \pm   0.02$& $-28.77 \pm   0.01$& $-28.45 \pm   0.02$& $-27.68 \pm   0.02$ & $  0.76 \pm   0.19$\\
SNF20080610-000 & $119.9 \pm 10.4$ & $ 16.4 \pm 1.7$& $ 6131 \pm   6$ & $-29.05 \pm   0.07$ & $-28.92 \pm   0.07$& $-28.50 \pm   0.07$& $-28.22 \pm   0.07$& $-27.55 \pm   0.07$ & $  0.12 \pm   0.16$\\
SNF20070701-005 & $101.8 \pm 2.6$ & $ 12.4 \pm 1.0$& $ 6158 \pm   5$ & $-29.46 \pm   0.02$ & $-29.27 \pm   0.02$& $-28.87 \pm   0.02$& $-28.60 \pm   0.02$& $-27.96 \pm   0.02$ & $  0.06 \pm   0.17$\\
SN2007kk & $128.5 \pm 1.4$ & $ 10.6 \pm 1.0$& $ 6098 \pm   4$ & $-29.48 \pm   0.02$ & $-29.31 \pm   0.02$& $-28.87 \pm   0.01$& $-28.54 \pm   0.01$& $-27.77 \pm   0.02$ & $  0.52 \pm   0.18$\\
SNF20060908-004 & $114.4 \pm 1.2$ & $ 12.6 \pm 0.6$& $ 6136 \pm   3$ & $-29.59 \pm   0.23$ & $-29.34 \pm   0.23$& $-28.91 \pm   0.23$& $-28.58 \pm   0.23$& $-27.87 \pm   0.23$ & $  0.70 \pm   0.19$\\
SNF20080909-030 & $ 93.7 \pm 1.0$ & $  7.8 \pm 0.4$& $ 6171 \pm   3$ & $-29.38 \pm   0.02$ & $-29.12 \pm   0.01$& $-28.74 \pm   0.01$& $-28.44 \pm   0.01$& $-27.78 \pm   0.01$ & $  1.17 \pm   0.16$\\
PTF11bgv & $ 79.4 \pm 3.2$ & $ 12.6 \pm 0.7$& $ 6146 \pm   3$ & $-28.90 \pm   0.02$ & $-28.62 \pm   0.01$& $-28.27 \pm   0.01$& $-28.08 \pm   0.01$& $-27.54 \pm   0.01$ & $ -0.33 \pm   0.12$\\
SNNGC2691 & $ 39.0 \pm 22.2$ & $  4.5 \pm 0.2$& $ 6139 \pm   8$ & $-29.46 \pm   0.02$ & $-29.06 \pm   0.02$& $-28.75 \pm   0.02$& $-28.49 \pm   0.02$& $-27.93 \pm   0.02$ & $  0.98 \pm   0.18$\\
PTF13asv & $ 75.6 \pm 1.1$ & $  2.2 \pm 0.4$& $ 6148 \pm   4$ & $-29.92 \pm   0.32$ & $-29.49 \pm   0.32$& $-29.02 \pm   0.32$& $-28.63 \pm   0.32$& $-27.90 \pm   0.32$ & $  0.50 \pm   0.11$\\
SNF20070806-026 & $ 98.8 \pm 12.1$ & $ 25.9 \pm 0.7$& $ 6114 \pm   7$ & $-29.14 \pm   0.02$ & $-28.91 \pm   0.02$& $-28.44 \pm   0.02$& $-28.21 \pm   0.02$& $-27.49 \pm   0.02$ & $ -1.73 \pm   0.17$\\
SNF20070427-001 & $ 81.3 \pm 2.3$ & $  6.3 \pm 0.9$& $ 6142 \pm   5$ & $-29.89 \pm   0.02$ & $-29.46 \pm   0.02$& $-28.97 \pm   0.02$& $-28.62 \pm   0.02$& $-27.97 \pm   0.02$ & $  0.62 \pm   0.21$\\
SNF20061108-004 & $129.5 \pm 5.6$ & $  6.3 \pm 2.5$& $ 6110 \pm   6$ & $-29.53 \pm   0.02$ & $-29.31 \pm   0.02$& $-28.95 \pm   0.02$& $-28.60 \pm   0.02$& $-27.96 \pm   0.02$ & $  2.14 \pm   0.42$\\
SNF20060912-000 & $106.5 \pm 1.8$ & $ 21.4 \pm 1.7$& $ 6163 \pm   7$ & $-28.98 \pm   0.02$ & $-28.92 \pm   0.02$& $-28.66 \pm   0.02$& $-28.42 \pm   0.02$& $-27.77 \pm   0.02$ & $  0.24 \pm   0.20$\\
CSS110918\_02 & $109.1 \pm 9.4$ & $ 15.0 \pm 0.6$& $ 6137 \pm   3$ & $-29.36 \pm   0.02$ & $-29.14 \pm   0.01$& $-28.69 \pm   0.01$& $-28.41 \pm   0.01$& $-27.70 \pm   0.01$ & $ -0.27 \pm   0.17$\\
SNF20080918-002 & $ 97.7 \pm 2.8$ & $ 12.6 \pm 1.4$& $ 6141 \pm   6$ & $-29.50 \pm   0.02$ & $-29.11 \pm   0.02$& $-28.61 \pm   0.02$& $-28.34 \pm   0.02$& $-27.71 \pm   0.02$ & $ -0.50 \pm   0.09$\\
SNIC3573 & $102.7 \pm 1.8$ & $ 11.9 \pm 1.0$& $ 6142 \pm   5$ & $-29.28 \pm   0.02$ & $-29.14 \pm   0.02$& $-28.74 \pm   0.02$& $-28.46 \pm   0.01$& $-27.76 \pm   0.03$ & $  0.35 \pm   0.19$\\
SNF20080725-004 & $133.6 \pm 2.1$ & $  6.9 \pm 0.9$& $ 6131 \pm   6$ & $-29.09 \pm   0.01$ & $-28.93 \pm   0.01$& $-28.59 \pm   0.01$& $-28.31 \pm   0.01$& $-27.55 \pm   0.03$ & $  0.60 \pm   0.14$\\
SNF20050728-006 & $127.8 \pm 2.5$ & $ 15.8 \pm 1.3$& $ 6124 \pm   6$ & $-28.80 \pm   0.02$ & $-28.68 \pm   0.02$& $-28.37 \pm   0.02$& $-28.18 \pm   0.02$& $-27.55 \pm   0.02$ & $  0.27 \pm   0.27$\\
SN2012fr & $134.2 \pm 0.5$ & $  7.4 \pm 0.2$& $ 6102 \pm   1$ & $-29.91 \pm   0.01$ & $-29.70 \pm   0.01$& $-29.31 \pm   0.01$& $-28.94 \pm   0.01$& $-28.10 \pm   0.01$ & $  0.96 \pm   0.10$\\
SNF20060512-002 & $100.2 \pm 2.8$ & $ 13.4 \pm 1.1$& $ 6107 \pm   8$ & $-29.33 \pm   0.02$ & $-29.11 \pm   0.02$& $-28.77 \pm   0.02$& $-28.52 \pm   0.02$& $-27.80 \pm   0.02$ & $ -0.92 \pm   0.19$\\
SNF20060512-001 & $ 88.4 \pm 1.2$ & $  5.4 \pm 0.4$& $ 6169 \pm   3$ & $-29.33 \pm   0.01$ & $-29.05 \pm   0.01$& $-28.68 \pm   0.01$& $-28.40 \pm   0.01$& $-27.79 \pm   0.01$ & $  0.69 \pm   0.15$\\
SNF20071003-016 & $125.2 \pm 4.6$ & $ 17.1 \pm 2.0$& $ 6124 \pm  11$ & $-28.58 \pm   0.02$ & $-28.54 \pm   0.02$& $-28.19 \pm   0.02$& $-27.99 \pm   0.02$& $-27.31 \pm   0.02$ & $ -0.25 \pm   0.15$\\
SNF20050821-007 & $141.7 \pm 2.6$ & $  7.7 \pm 1.0$& $ 6140 \pm   9$ & $-29.38 \pm   0.02$ & $-29.20 \pm   0.02$& $-28.77 \pm   0.02$& $-28.46 \pm   0.02$& $-27.67 \pm   0.02$ & $  0.17 \pm   0.32$\\
SNF20070803-005 & $ 22.7 \pm 21.4$ & $  0.9 \pm 0.6$& $ 6157 \pm  27$ & $-29.87 \pm   0.01$ & $-29.43 \pm   0.01$& $-29.04 \pm   0.01$& $-28.74 \pm   0.01$& $-28.11 \pm   0.01$ & $  0.88 \pm   0.18$\\
PTF09foz & $127.2 \pm 1.9$ & $ 21.7 \pm 1.2$& $ 6136 \pm   4$ & $-29.14 \pm   0.01$ & $-29.00 \pm   0.01$& $-28.59 \pm   0.01$& $-28.35 \pm   0.01$& $-27.65 \pm   0.01$ & $ -1.14 \pm   0.26$\\
PTF12grk & $162.3 \pm 9.8$ & $ 19.6 \pm 1.4$& $ 6085 \pm   8$ & $-28.86 \pm   0.02$ & $-28.87 \pm   0.01$& $-28.42 \pm   0.01$& $-28.19 \pm   0.01$& $-27.50 \pm   0.03$ & $ -1.17 \pm   0.13$\\
SNF20080720-001 & $138.5 \pm 4.0$ & $ 14.0 \pm 2.0$& $ 6107 \pm   3$ & $-27.59 \pm   0.02$ & $-27.78 \pm   0.01$& $-27.73 \pm   0.01$& $-27.71 \pm   0.01$& $-27.19 \pm   0.02$ & $  0.18 \pm   0.13$\\
SNF20080810-001 & $ 88.4 \pm 21.6$ & $ 22.3 \pm 1.1$& $ 6145 \pm   5$ & $-29.11 \pm   0.01$ & $-28.89 \pm   0.01$& $-28.45 \pm   0.01$& $-28.23 \pm   0.01$& $-27.60 \pm   0.01$ & $ -0.87 \pm   0.11$\\
SNF20050729-002 & $109.4 \pm 2.2$ & $ 11.5 \pm 1.7$& $ 6142 \pm   6$ & $-29.35 \pm   0.13$ & $-29.17 \pm   0.13$& $-28.68 \pm   0.13$& $-28.38 \pm   0.13$& $-27.56 \pm   0.13$ & $  0.57 \pm   0.34$\\
SN2008ec & $103.7 \pm 17.0$ & $ 23.1 \pm 0.4$& $ 6125 \pm   3$ & $-28.67 \pm   0.01$ & $-28.52 \pm   0.01$& $-28.18 \pm   0.01$& $-28.03 \pm   0.01$& $-27.47 \pm   0.01$ & $ -1.05 \pm   0.12$\\
SNF20070902-018 & $ 93.8 \pm 12.2$ & $ 23.8 \pm 3.0$& $ 6120 \pm   8$ & $-28.87 \pm   0.02$ & $-28.70 \pm   0.01$& $-28.26 \pm   0.01$& $-28.08 \pm   0.01$& $-27.41 \pm   0.02$ & $ -0.72 \pm   0.10$\\
SNF20070424-003 & $122.5 \pm 3.8$ & $ 12.7 \pm 1.6$& $ 6132 \pm   6$ & $-29.10 \pm   0.01$ & $-28.96 \pm   0.01$& $-28.51 \pm   0.01$& $-28.25 \pm   0.01$& $-27.57 \pm   0.01$ & $  0.31 \pm   0.11$\\
SN2006cj & $101.7 \pm 1.3$ & $  4.8 \pm 0.8$& $ 6127 \pm   3$ & $-29.43 \pm   0.01$ & $-29.14 \pm   0.01$& $-28.74 \pm   0.01$& $-28.43 \pm   0.01$& $-27.76 \pm   0.01$ & $  0.48 \pm   0.18$\\
SN2007nq & $ 89.8 \pm 9.9$ & $ 23.4 \pm 1.1$& $ 6109 \pm   5$ & $-29.11 \pm   0.02$ & $-28.91 \pm   0.02$& $-28.50 \pm   0.02$& $-28.27 \pm   0.02$& $-27.57 \pm   0.02$ & $ -1.89 \pm   0.26$\\
SNF20070817-003 & $ 93.9 \pm 2.4$ & $ 18.5 \pm 1.3$& $ 6116 \pm   6$ & $-29.19 \pm   0.02$ & $-29.03 \pm   0.01$& $-28.59 \pm   0.01$& $-28.30 \pm   0.01$& $-27.55 \pm   0.02$ & $ -0.57 \pm   0.13$\\
SNF20070403-000 & $ 61.8 \pm 6.5$ & $ 27.1 \pm 1.8$& $ 6154 \pm   8$ & $-28.37 \pm   0.02$ & $-28.27 \pm   0.02$& $-27.97 \pm   0.02$& $-27.80 \pm   0.02$& $-27.24 \pm   0.02$ & $ -1.96 \pm   0.19$\\
SNF20061022-005 & $ 64.6 \pm 3.8$ & $  3.7 \pm 1.4$& $ 6146 \pm   7$ & $-29.49 \pm   0.02$ & $-29.06 \pm   0.02$& $-28.71 \pm   0.02$& $-28.42 \pm   0.02$& $-27.93 \pm   0.02$ & $  1.72 \pm   0.44$\\
SNNGC4076 & $127.3 \pm 2.4$ & $ 15.5 \pm 1.2$& $ 6152 \pm   4$ & $-28.77 \pm   0.01$ & $-28.66 \pm   0.01$& $-28.37 \pm   0.01$& $-28.15 \pm   0.01$& $-27.52 \pm   0.01$ & $  0.01 \pm   0.12$\\
SNF20070727-016 & $ 77.5 \pm 2.5$ & $  5.1 \pm 0.8$& $ 6140 \pm   4$ & $-29.96 \pm   0.06$ & $-29.56 \pm   0.06$& $-29.06 \pm   0.06$& $-28.75 \pm   0.06$& $-28.01 \pm   0.06$ & $  0.27 \pm   0.12$\\
PTF12fuu & $105.5 \pm 3.0$ & $  6.2 \pm 1.2$& $ 6124 \pm   5$ & $-29.54 \pm   0.01$ & $-29.23 \pm   0.01$& $-28.74 \pm   0.01$& $-28.40 \pm   0.01$& $-27.64 \pm   0.01$ & $  0.34 \pm   0.12$\\
SNF20070820-000 & $107.2 \pm 3.5$ & $ 18.6 \pm 1.3$& $ 6132 \pm  14$ & $-28.80 \pm   0.02$ & $-28.69 \pm   0.02$& $-28.34 \pm   0.02$& $-28.13 \pm   0.02$& $-27.52 \pm   0.02$ & $ -0.29 \pm   0.13$\\
SNF20070725-001 & $108.4 \pm 2.0$ & $ 11.1 \pm 1.5$& $ 6140 \pm   7$ & $-29.61 \pm   0.02$ & $-29.32 \pm   0.02$& $-28.84 \pm   0.02$& $-28.50 \pm   0.02$& $-27.76 \pm   0.02$ & $  0.68 \pm   0.17$\\
SNF20071108-021 & $ 99.1 \pm 2.7$ & $  5.8 \pm 0.8$& $ 6164 \pm   5$ & $-29.67 \pm   0.01$ & $-29.34 \pm   0.01$& $-28.94 \pm   0.01$& $-28.60 \pm   0.01$& $-27.96 \pm   0.01$ & $  1.00 \pm   0.21$\\
SNF20080914-001 & $126.5 \pm 1.2$ & $ 15.4 \pm 1.1$& $ 6159 \pm   3$ & $-28.67 \pm   0.02$ & $-28.60 \pm   0.02$& $-28.31 \pm   0.02$& $-28.13 \pm   0.02$& $-27.58 \pm   0.02$ & $ -0.11 \pm   0.17$\\
SNF20060609-002 & $ 87.7 \pm 3.6$ & $  7.3 \pm 1.3$& $ 6132 \pm   4$ & $-28.60 \pm   0.02$ & $-28.42 \pm   0.02$& $-28.19 \pm   0.02$& $-28.05 \pm   0.02$& $-27.53 \pm   0.02$ & $  0.07 \pm   0.15$\\
SNF20050624-000 & $121.0 \pm 5.3$ & $  9.3 \pm 3.1$& $ 6126 \pm   6$ & $-29.75 \pm   0.01$ & $-29.42 \pm   0.01$& $-28.99 \pm   0.01$& $-28.68 \pm   0.01$& $-27.97 \pm   0.01$ & $  0.69 \pm   0.62$\\
SNF20060618-023 & $ 74.9 \pm 4.9$ & $  5.0 \pm 1.8$& $ 6137 \pm  21$ & $-29.61 \pm   0.02$ & $-29.18 \pm   0.02$& $-28.89 \pm   0.02$& $-28.66 \pm   0.02$& $-28.08 \pm   0.02$ & $  0.12 \pm   0.18$\\
SNF20080531-000 & $133.0 \pm 1.5$ & $ 17.6 \pm 0.8$& $ 6114 \pm   5$ & $-29.12 \pm   0.01$ & $-28.98 \pm   0.01$& $-28.54 \pm   0.01$& $-28.28 \pm   0.01$& $-27.51 \pm   0.01$ & $ -0.49 \pm   0.18$\\
SN2006do & $106.4 \pm 2.1$ & $ 26.7 \pm 1.3$& $ 6101 \pm   2$ & $-29.00 \pm   0.01$ & $-28.83 \pm   0.01$& $-28.42 \pm   0.01$& $-28.20 \pm   0.01$& $-27.53 \pm   0.04$ & $ -2.32 \pm   0.54$\\
PTF12ikt & $110.3 \pm 1.6$ & $ 14.2 \pm 0.7$& $ 6141 \pm   4$ & $-29.34 \pm   0.01$ & $-29.04 \pm   0.01$& $-28.57 \pm   0.01$& $-28.32 \pm   0.01$& $-27.66 \pm   0.01$ & $ -0.35 \pm   0.11$\\
SN2006dm & $ 99.5 \pm 1.6$ & $ 30.0 \pm 0.7$& $ 6118 \pm   3$ & $-28.81 \pm   0.01$ & $-28.65 \pm   0.01$& $-28.23 \pm   0.01$& $-28.02 \pm   0.01$& $-27.33 \pm   0.01$ & $ -1.88 \pm   0.34$\\
PTF13azs & $138.0 \pm 5.1$ & $ 16.2 \pm 1.6$& $ 6125 \pm  10$ & $-27.84 \pm   0.02$ & $-27.92 \pm   0.02$& $-27.69 \pm   0.02$& $-27.60 \pm   0.02$& $-26.99 \pm   0.02$ & $ -0.02 \pm   0.12$\\
SN2005hj & $ 80.8 \pm 2.4$ & $  4.3 \pm 0.8$& $ 6138 \pm   4$ & $-29.54 \pm   0.02$ & $-29.16 \pm   0.01$& $-28.87 \pm   0.01$& $-28.54 \pm   0.01$& $-28.01 \pm   0.01$ & $  1.85 \pm   0.59$\\
PTF12iiq & $150.4 \pm 2.2$ & $ 22.5 \pm 0.8$& $ 6041 \pm   6$ & $-28.60 \pm   0.01$ & $-28.77 \pm   0.01$& $-28.41 \pm   0.01$& $-28.10 \pm   0.01$& $-27.29 \pm   0.01$ & $ -1.30 \pm   0.12$\\
PTF10ndc & $124.2 \pm 2.4$ & $  6.8 \pm 1.1$& $ 6119 \pm   3$ & $-29.52 \pm   0.01$ & $-29.25 \pm   0.01$& $-28.80 \pm   0.01$& $-28.49 \pm   0.01$& $-27.76 \pm   0.01$ & $  0.87 \pm   0.18$\\
SNF20080919-002 & $103.6 \pm 7.2$ & $ 27.2 \pm 1.9$& $ 6133 \pm   8$ & $-28.74 \pm   0.02$ & $-28.46 \pm   0.01$& $-28.09 \pm   0.01$& $-27.87 \pm   0.01$& $-27.26 \pm   0.04$ & $ -1.52 \pm   0.14$\\
SNPGC027923 & $ 85.5 \pm 0.6$ & $  5.9 \pm 0.3$& $ 6130 \pm   4$ & $-29.87 \pm   0.02$ & $-29.45 \pm   0.02$& $-28.94 \pm   0.02$& $-28.57 \pm   0.02$& $-27.85 \pm   0.02$ & $  0.22 \pm   0.26$\\
SNF20070330-024 & $118.1 \pm 2.1$ & $  4.6 \pm 2.2$& $ 6101 \pm   3$ & $-29.77 \pm   0.02$ & $-29.52 \pm   0.02$& $-29.08 \pm   0.02$& $-28.74 \pm   0.01$& $-27.94 \pm   0.02$ & $  0.34 \pm   0.29$\\
SNF20061030-010 & $131.4 \pm 2.2$ & $ 17.4 \pm 1.1$& $ 6116 \pm   4$ & $-28.60 \pm   0.02$ & $-28.55 \pm   0.02$& $-28.25 \pm   0.02$& $-28.03 \pm   0.02$& $-27.34 \pm   0.02$ & $  0.14 \pm   0.25$\\
SNhunt46 & $ 94.1 \pm 2.0$ & $ 11.2 \pm 0.6$& $ 6132 \pm   4$ & $-29.50 \pm   0.02$ & $-29.11 \pm   0.02$& $-28.67 \pm   0.02$& $-28.37 \pm   0.02$& $-27.71 \pm   0.02$ & $  0.20 \pm   0.24$\\
SN2005hc & $126.9 \pm 2.5$ & $ 10.0 \pm 0.7$& $ 6123 \pm   3$ & $-29.38 \pm   0.01$ & $-29.13 \pm   0.01$& $-28.69 \pm   0.01$& $-28.37 \pm   0.01$& $-27.61 \pm   0.01$ & $  0.87 \pm   0.31$\\
LSQ12dbr & $106.9 \pm 0.6$ & $  7.1 \pm 0.7$& $ 6138 \pm   4$ & $-29.29 \pm   0.73$ & $-29.00 \pm   0.73$& $-28.51 \pm   0.73$& $-28.15 \pm   0.73$& $-27.38 \pm   0.73$ & $  1.12 \pm   0.14$\\
LSQ12hjm & $ 82.6 \pm 17.5$ & $ 12.2 \pm 1.4$& $ 6144 \pm   5$ & $-29.51 \pm   0.02$ & $-29.14 \pm   0.01$& $-28.60 \pm   0.01$& $-28.30 \pm   0.01$& $-27.71 \pm   0.02$ & $ -0.50 \pm   0.21$\\
SNF20060521-001 & $ 78.9 \pm 20.2$ & $ 21.1 \pm 1.4$& $ 6123 \pm  10$ & $-29.37 \pm   0.05$ & $-29.04 \pm   0.05$& $-28.54 \pm   0.05$& $-28.30 \pm   0.05$& $-27.57 \pm   0.05$ & $ -1.22 \pm   0.26$\\
SNF20070630-006 & $125.5 \pm 3.2$ & $ 10.1 \pm 1.6$& $ 6126 \pm   4$ & $-29.34 \pm   0.01$ & $-29.12 \pm   0.01$& $-28.65 \pm   0.01$& $-28.38 \pm   0.01$& $-27.66 \pm   0.01$ & $  0.35 \pm   0.12$\\
PTF11drz & $132.6 \pm 1.4$ & $ 15.2 \pm 1.0$& $ 6116 \pm   5$ & $-29.12 \pm   0.01$ & $-28.95 \pm   0.01$& $-28.53 \pm   0.01$& $-28.27 \pm   0.01$& $-27.55 \pm   0.01$ & $  0.21 \pm   0.37$\\
SNF20080323-009 & $ 95.9 \pm 2.3$ & $ 10.6 \pm 1.1$& $ 6143 \pm   6$ & $-29.59 \pm   0.02$ & $-29.22 \pm   0.02$& $-28.68 \pm   0.02$& $-28.42 \pm   0.02$& $-27.77 \pm   0.02$ & $ -0.10 \pm   0.17$\\
SNF20071021-000 & $167.5 \pm 2.2$ & $ 20.4 \pm 0.6$& $ 6112 \pm   4$ & $-28.75 \pm   0.02$ & $-28.78 \pm   0.02$& $-28.40 \pm   0.02$& $-28.18 \pm   0.02$& $-27.41 \pm   0.02$ & $ -0.53 \pm   0.18$\\
SNNGC0927 & $155.2 \pm 1.4$ & $ 11.0 \pm 0.7$& $ 6109 \pm   4$ & $-28.87 \pm   0.02$ & $-28.81 \pm   0.01$& $-28.46 \pm   0.01$& $-28.22 \pm   0.01$& $-27.48 \pm   0.01$ & $  0.41 \pm   0.14$\\
SNF20060526-003 & $112.1 \pm 2.5$ & $  9.8 \pm 1.0$& $ 6121 \pm   3$ & $-29.34 \pm   0.01$ & $-29.09 \pm   0.01$& $-28.68 \pm   0.01$& $-28.39 \pm   0.01$& $-27.70 \pm   0.01$ & $  0.29 \pm   0.17$\\
SNF20080806-002 & $135.8 \pm 1.8$ & $  7.5 \pm 0.9$& $ 6135 \pm   4$ & $-29.22 \pm   0.02$ & $-29.02 \pm   0.02$& $-28.61 \pm   0.02$& $-28.35 \pm   0.01$& $-27.71 \pm   0.02$ & $  0.64 \pm   0.16$\\
SNF20080803-000 & $117.6 \pm 2.6$ & $  8.9 \pm 2.0$& $ 6125 \pm   4$ & $-28.84 \pm   0.01$ & $-28.70 \pm   0.01$& $-28.35 \pm   0.01$& $-28.16 \pm   0.01$& $-27.50 \pm   0.01$ & $  0.37 \pm   0.14$\\
SNF20080822-005 & $ 78.5 \pm 1.8$ & $  6.3 \pm 0.9$& $ 6138 \pm   4$ & $-29.71 \pm   0.01$ & $-29.34 \pm   0.01$& $-28.93 \pm   0.01$& $-28.61 \pm   0.01$& $-27.92 \pm   0.01$ & $  0.56 \pm   0.23$\\
SNF20060618-014 & $137.2 \pm 2.5$ & $  9.3 \pm 1.1$& $ 6112 \pm   7$ & $-29.27 \pm   0.03$ & $-29.09 \pm   0.03$& $-28.73 \pm   0.03$& $-28.38 \pm   0.03$& $-27.68 \pm   0.03$ & $  0.97 \pm   0.21$\\
PTF12ghy & $ 99.3 \pm 3.6$ & $ 16.8 \pm 0.7$& $ 6134 \pm   3$ & $-28.29 \pm   0.02$ & $-28.27 \pm   0.01$& $-28.05 \pm   0.01$& $-27.95 \pm   0.01$& $-27.40 \pm   0.01$ & $ -0.48 \pm   0.12$\\
SNF20070531-011 & $122.4 \pm 2.7$ & $ 21.2 \pm 0.8$& $ 6114 \pm   4$ & $-29.07 \pm   0.01$ & $-28.94 \pm   0.01$& $-28.50 \pm   0.01$& $-28.26 \pm   0.01$& $-27.53 \pm   0.03$ & $ -1.82 \pm   0.25$\\
SNF20070831-015 & $112.2 \pm 2.7$ & $  7.8 \pm 1.0$& $ 6145 \pm   6$ & $-29.42 \pm   0.01$ & $-29.17 \pm   0.01$& $-28.78 \pm   0.01$& $-28.46 \pm   0.01$& $-27.78 \pm   0.01$ & $  1.85 \pm   0.23$\\
SNF20070417-002 & $104.5 \pm 5.5$ & $ 24.4 \pm 2.2$& $ 6123 \pm   9$ & $-29.20 \pm   0.05$ & $-29.01 \pm   0.05$& $-28.48 \pm   0.05$& $-28.23 \pm   0.05$& $-27.54 \pm   0.05$ & $ -1.37 \pm   0.11$\\
PTF11cao & $143.3 \pm 1.6$ & $ 18.9 \pm 1.3$& $ 6104 \pm   5$ & $-28.78 \pm   0.02$ & $-28.79 \pm   0.02$& $-28.44 \pm   0.02$& $-28.18 \pm   0.02$& $-27.45 \pm   0.02$ & $  0.18 \pm   0.15$\\
SNF20080522-000 & $ 61.8 \pm 3.5$ & $  3.3 \pm 0.9$& $ 6131 \pm   7$ & $-29.86 \pm   0.01$ & $-29.41 \pm   0.01$& $-29.03 \pm   0.01$& $-28.70 \pm   0.01$& $-28.06 \pm   0.01$ & $  1.02 \pm   0.13$\\
PTF10qjq & $ 73.9 \pm 2.4$ & $ 12.8 \pm 0.8$& $ 6133 \pm   3$ & $-29.29 \pm   0.02$ & $-28.94 \pm   0.02$& $-28.53 \pm   0.01$& $-28.35 \pm   0.01$& $-27.76 \pm   0.01$ & $ -0.35 \pm   0.11$\\
PTF12dxm & $ 95.4 \pm 41.8$ & $ 35.7 \pm 2.8$& $ 6136 \pm   4$ & $-28.71 \pm   0.01$ & $-28.58 \pm   0.01$& $-28.19 \pm   0.01$& $-27.99 \pm   0.01$& $-27.34 \pm   0.01$ & $ -1.98 \pm   0.20$\\
SNF20061021-003 & $122.8 \pm 2.3$ & $  9.7 \pm 1.7$& $ 6131 \pm   4$ & $-29.04 \pm   0.02$ & $-28.86 \pm   0.02$& $-28.56 \pm   0.02$& $-28.30 \pm   0.02$& $-27.64 \pm   0.02$ & $  0.25 \pm   0.18$\\
SNF20080510-005 & $111.6 \pm 2.6$ & $  6.4 \pm 1.1$& $ 6115 \pm   4$ & $-29.41 \pm   0.01$ & $-29.15 \pm   0.01$& $-28.70 \pm   0.01$& $-28.38 \pm   0.01$& $-27.73 \pm   0.04$ & $  1.90 \pm   0.26$\\
SNF20080507-000 & $ 98.1 \pm 1.6$ & $ 10.6 \pm 2.1$& $ 6143 \pm   5$ & $-29.23 \pm   0.01$ & $-29.05 \pm   0.01$& $-28.71 \pm   0.01$& $-28.45 \pm   0.01$& $-27.79 \pm   0.01$ & $  0.89 \pm   0.25$\\
SNF20080913-031 & $118.2 \pm 1.5$ & $ 11.3 \pm 1.8$& $ 6158 \pm   5$ & $-29.13 \pm   0.08$ & $-29.01 \pm   0.07$& $-28.62 \pm   0.07$& $-28.32 \pm   0.07$& $-27.68 \pm   0.07$ & $  0.03 \pm   0.21$\\
SNF20080510-001 & $118.8 \pm 2.1$ & $ 15.3 \pm 1.3$& $ 6115 \pm   4$ & $-29.35 \pm   0.01$ & $-29.15 \pm   0.01$& $-28.69 \pm   0.01$& $-28.38 \pm   0.01$& $-27.68 \pm   0.01$ & $  0.43 \pm   0.18$\\
SNF20070712-003 & $108.8 \pm 2.7$ & $ 13.5 \pm 0.9$& $ 6155 \pm   6$ & $-29.44 \pm   0.02$ & $-29.19 \pm   0.01$& $-28.74 \pm   0.01$& $-28.42 \pm   0.01$& $-27.78 \pm   0.01$ & $  0.42 \pm   0.18$\\
\enddata
\end{deluxetable}

The distributions of the central values of the inputs  $EW_{\mathrm{Ca}}$, $EW_{\mathrm{Si}}$,
$\lambda_{\mathrm{Si}}$ , $x_1$, and the SALT2 color parameter $C$ are shown in Table~\ref{input:fig}.
For reference, the $x_1$ and $C$ distributions of the $0.03<z< 0.08$ Pantheon sample,
predominantly supernovae from CFA3 \citep{2009ApJ...700..331H}, CFA4 \citep{2012ApJS..200...12H}, and CSP
\citep{2010AJ....139..519C, 2011AJ....142..156S}, that were used as part of the low-redshift anchor in the recent cosmology analysis of \citet{2017arXiv171000845S}.

\begin{figure}[htbp] %  figure placement: here, top, bottom, or page
   \centering
   \includegraphics[width=4in]{perobject_input.pdf}
   \caption{\added{Distributions of the central values of the inputs  $EW_{\mathrm{Ca}}$, $EW_{\mathrm{Si}}$,
$\lambda_{\mathrm{Si}}$ , $x_1$, and $C$ of the sample used in this article, and
the $x_1$ and $C$ distributions of the $0.03<z< 0.08$ Pantheon sample.}
   \label{input:fig}}
\end{figure}
}

\section{Model~I: Two Color Parameters}
\label{modelI:sec}
We begin by considering Model~I, in which supernova magnitudes  have linear dependence on spectral parameters
and two color parameters. The purpose of this section is to introduce some basic concepts of our models,
to show how the two  color terms can be associated with existing dust-extinction models, and to give Model~I results on dust extinction.
Deficiencies of this model are addressed in the subsequent models presented in the article.
To avoid repetitiveness,
we refrain from going into full detail on the results of Model~I in deference to Model~III, which is presented in Section~\ref{modelIII:sec}.

\added{It is important to note that the model and those in subsequent sections are applied to our supernova sample;
the ensuing results thus correspond to our SN~Ia sample's underlying population. }

\subsection{Model}
We assume 
that  peak
underlying
 ${\hat{U}}{\hat{B}}\hat{V}\hat{R}\hat{I}$ magnitudes of a supernova  with effects from  astrophysical foregrounds are linearly dependent
on its
 equivalent widths of the Ca~II H\&K and Si~II~$\lambda$4141 spectral features
$EW_{\mathrm{Ca}}$ and $EW_{\mathrm{Si}}$,
and the wavelength of the minimum of 
the Si~II~$\lambda6355$ feature $\lambda_{\mathrm{Si}}$
around $B$-band peak brightness:
these spectral features are associated with SN~Ia  spectroscopic diversity  
\citep{2005ApJ...623.1011B, 2006PASP..118..560B, 2008A&A...492..535A, 2009A&A...500L..17B, 2009PASP..121..238B, 2009ApJ...699L.139W, 2011ApJ...729...55F,2012AJ....143..126B}.
The explicit omission of light-curve shape in our model is compensated by its proxy,
$EW_{\mathrm{Si}}$, at peak brightness
\citep{2008A&A...492..535A, 2011A&A...529L...4C}. 
The underlying magnitudes are also linearly dependent on the per-supernova latent parameters $g_0$ and $g_1$.
Unlike
the spectral parameters ($EW_{\mathrm{Ca}}$, $EW_{\mathrm{Si}}$ and $\lambda_{\mathrm{Si}}$),  $g_0$ and $g_1$ are not directly associated
with observables but rather are inferred as part of the analysis.
A grey magnitude offset, $\Delta$, is included for each supernova
to capture 
potential
band-independent intrinsic dispersion, while also absorbing peculiar-velocity errors introduced when converting
fluxes to luminosities.
With this grey offset the model standardizes colors, not absolute magnitude.
The underlying magnitudes are
\begin{equation}
\begin{pmatrix}
{\hat{U}}\\{\hat{B}}\\{\hat{V}}\\{\hat{R}}\\{\hat{I}}
\end{pmatrix}
=
\Delta \vec{1} +
\begin{pmatrix}
c_{\hat{U}}+\alpha_{\hat{U}} EW_{\mathrm{Ca}} + \beta_{\hat{U}} EW_{\mathrm{Si}} + \eta_{\hat{U}} \lambda_{\mathrm{Si}}  +\gamma^0_{{\hat{U}}} g_0 +\gamma^1_{{\hat{U}}} g_1 \\
c_{\hat{B}}+\alpha_{\hat{B}} EW_{\mathrm{Ca}} + \beta_{\hat{B}} EW_{\mathrm{Si}} + \eta_{\hat{B}} \lambda_{\mathrm{Si}}  +\gamma^0_{{\hat{B}}} g_0 +\gamma^1_{{\hat{B}}} g_1 \\
c_{\hat{V}}+\alpha_{\hat{V}} EW_{\mathrm{Ca}} + \beta_{\hat{V}} EW_{\mathrm{Si}} + \eta_{\hat{V}} \lambda_{\mathrm{Si}} +\gamma^0_{{\hat{V}}} g_0 +\gamma^1_{{\hat{V}}} g_1 \\
c_{\hat{R}}+\alpha_{\hat{R}} EW_{\mathrm{Ca}} + \beta_{\hat{R}} EW_{\mathrm{Si}} + \eta_{\hat{R}} \lambda_{\mathrm{Si}}  +\gamma^0_{{\hat{R}}} g_0 +\gamma^1_{{\hat{R}}} g_1 \\
c_{\hat{I}}+\alpha_{\hat{I}} EW_{\mathrm{Ca}} + \beta_{\hat{I}} EW_{\mathrm{Si}}+ \eta_{\hat{I}} \lambda_{\mathrm{Si}}  +\gamma^0_{{\hat{I}}} g_0 +\gamma^1_{{\hat{I}}} g_1
\end{pmatrix}.
\label{ewsiv:eqn}
\end{equation}
$\vec{1}$ is the Identity vector.
The global parameter vectors\footnote{Global parameters act on each of the five bands: a single component is written with a subscript for the corresponding band, e.g.\  $c_{\hat{U}}$; 
the combination of all components are written as a vector, e.g.\ $\vec{c}=(c_{\hat{U}}, c_{\hat{B}}, c_{\hat{V}}, c_{\hat{R}}, c_{\hat{I}}) $.  The
band subscript or arrow
notationally identify global parameters, in contrast with the per-supernova parameters.
} that describe the SN~Ia population
$\vec{c}$  (units of mag);
$\vec{\alpha}$, $\vec{\beta}$,
$\vec{\eta}$ (units of mag~\AA$^{-1}$), $\vec{\gamma}^0$ and $\vec{\gamma}^1$ (units of mag)
 are the intercept
and slopes
of the linear relationships that
relate
per-supernova
parameters with 
underlying magnitudes.
The notation used in this article is not to be associated with those from previous works.


The observables
$\hat{U}_o, {\hat{B}}_o, {\hat{V}}_o, {\hat{R}}_o, {\hat{I}}_o$, $EW_{Ca,o}$, $EW_{Si,o}$, $\lambda_{Si,o}$
have Gaussian measurement uncertainty with covariance $C$.
The
likelihood density for the described model
corresponds to the probability distribution function described by
\begin{equation}
\begin{pmatrix}
{\hat{U}}_o\\{\hat{B}}_o\\ {\hat{V}}_o\\{\hat{R}}_o\\{\hat{I}}_o\\EW_{Si, o}\\ EW_{Ca, o} \\ \lambda_{Si, o}
\end{pmatrix}
\sim \mathcal{N}
\left(
\begin{pmatrix}
{\hat{U}}  \\{\hat{B}}  \\
{\hat{V}}\\{\hat{R}}\\{\hat{I}}\\
EW_{\mathrm{Si}}\\ EW_{\mathrm{Ca}} \\ \lambda_{\mathrm{Si}}
\end{pmatrix}
,C
\right).
\label{dust:eqn}
\end{equation}


The model as written has degeneracies that need to be constrained in order for fits to converge.
The model does not specify the absolute magnitude nor the 
color of a $g_0=0$, $g_1=0$ supernova;
in their place the zeropoints of the per-supernova magnitude and color parameters are set to the sample means
\begin{equation}
\langle \Delta \rangle=0,\ \langle g_0 \rangle=0,\ \langle g_1 \rangle=0.
\label{zero:eqn}
\end{equation}
The model contains the product of parameters
$\vec{\gamma} g$, which leads to the degeneracy $\vec{\gamma} \rightarrow a\vec{\gamma}$, $g \rightarrow a^{-1} g$.
To aid in the convergence of $\vec{\gamma}$ and $g$ we impose a prior on the rms of $g$, though
our physical interpretations are ultimately independent of this scaling.
This prior does not specify the sign of $a$, which leaves a parity degeneracy. 
Results in this article are given in terms of $\vec{\gamma}g$, not $g$ alone.
As will be seen, the signal-to-noise in $\vec{\gamma}$ is sufficiently
high that our finite MCMC chains do not migrate between the degenerate solutions; to simplify the merging of multiple chains
we impose one of the degenerate solutions
\begin{equation}
\gamma^0_{\hat{U}} > 0,\ \gamma^1_{\hat{U}} < 0.
\end{equation}
As  in mixture models, the combinations $\vec{\gamma}^0 g_0$ and $\vec{\gamma}^1 g_1$ terms are degenerate under exchange of the 0 and 1 indices. For a cleaner presentation of our results, 
we break that degeneracy  by setting consistent initial conditions that succeed in keeping indices consistent over all analysis chains.  

Each of the multiplicative cofactors has a degeneracy with the zeropoint.  Consider
$\vec{\alpha} \rightarrow \vec{\alpha} +   \epsilon_{\alpha}$,
$\vec{\beta} \rightarrow \vec{\beta} +   \epsilon_{\beta}$,
$\vec{\eta} \rightarrow \vec{\eta} +   \epsilon_{\eta}$,
$\vec{\gamma}^0 \rightarrow \vec{\gamma}^0 +   \epsilon_{\gamma^0}$,
$\vec{\gamma}^1 \rightarrow \vec{\gamma}^1 +   \epsilon_{\gamma^1}$.
Then
each supernova's grey term can shift as
$\Delta \rightarrow \Delta  + \epsilon_\Delta$
with
$$
\epsilon_\Delta  = -  \epsilon_{\alpha} (EW_{\text{Ca}}-\langle EW_{\text{Ca}}\rangle)- 
 \epsilon_{\beta} (EW_{\mathrm{Si}}-\langle EW_{\mathrm{Si}}\rangle) -  \epsilon_{\eta} (\lambda_{\mathrm{Si}}-\langle \lambda_{\mathrm{Si}}\rangle)
- \epsilon_{\gamma^0}g_0 -  \epsilon_{\gamma^1}g_1.
$$
We break this degeneracy by fixing the values of $\epsilon$ such that $\Delta$ is
uncorrelated with  $EW_{\text{Ca}}$, $EW_{\mathrm{Si}}$, $\lambda_{\text{Ca}}$,
$g_0$, and $g_1$ reflecting
our preference for feature affects to be confined to the cofactors and not the grey magnitude offsets.

All parameters not  otherwise noted above have flat priors.

For $N$ supernovae there are $8N$ observables.  There are $3N$ spectral parameters, each of
$\Delta$, $g_0$, $g_1$ contributes $N-1$ parameters
(recall the constraint in Eq.~\ref{zero:eqn}),  and there are $5 \times 7$ global coefficients.
For $N=172$ supernovae, there are 1376 observables and 1064  parameters.


\subsection{Results for the Latent Parameter Cofactors  $\vec{\gamma}^0$ and $\vec{\gamma}^1$  }
\label{results1:sec}
\added{The MCMC chains for all parameters from this and subsequent sections are accessible from the SNfactory website\footnote{For the referee the parameters
are available at \url{https://drive.google.com/open?id=1d52nxfF7eS3HblfbmT3fxgiZod_Z-Gjg}. Concurrent with publication the chain will be
made available
at \url{https://snfactory.lbl.gov/snf/data/index.html}.}.}
In this subsection we concentrate on the results for $\vec{\gamma}^0$ and $\vec{\gamma}^1$.
In our model, $g_0$ and $g_1$ are introduced as latent parameters that are related to linear shifts in band magnitudes through
the  coefficients $\vec{\gamma}^0$ and
$\vec{\gamma}^1$.  Otherwise, no physically-motivated suppositions on the properties of $g$ and $\vec{\gamma}$ are imposed.
Since dust extinction is a dominant determinant of supernova colors that is often described by a 2-parameter linear model, we expect
for our two  color terms to capture the effects of dust.
In this section we reexpress our model $\vec{\gamma}$'s in terms of
the dust-extinction model of \citet{1999PASP..111...63F} \citepalias[henceforth referred to as][]{1999PASP..111...63F}.

The 68\% credible intervals for $\vec{\gamma}^0$ and $\vec{\gamma}^1$ (and all other global parameters) are given in \S\ref{m12results:sec}.
All elements of the two $\vec{\gamma}$ parameter vectors 
describing color
are significantly non-zero.
None of the 20000 links of 
our Monte Carlo chains for $\vec{\gamma}$ come close to the origin.   We claim 
a probability $(1-5\times 10^{-5})$-detection of
a magnitude term in the form of a 5-dimensional vector spanned by two basis vectors
\begin{equation}
\vec{A} = \vec{\gamma}^0 g_0 +  \vec{\gamma}^1 g_1.
\end{equation}
All possible magnitude changes in our five bands due to $g_0$ and $g_1$ are confined to a two-dimensional
plane defined by $\vec{\gamma}^0$ and  $\vec{\gamma}^1$. 

Dust-extinction
models are also approximately described by a 2-parameter linear model, here 
using the commonly-used parameters $A_V$ and $E(B-V)$
\begin{equation}
\vec{A} =\vec{ a}  A_{V} + \vec{b} E(B-V).
\label{f99:eqn}
\end{equation}

For example,   the wavelength-dependent model of  \citet{1989ApJ...345..245C} is linear,
though its linearity is lost when integrated over broad-band filters.
For the case of
$R^F=2.5$ and $A^F_V=0.1$ dust attenuating light from the SALT2
\citep{2007A&A...466...11G} $s=1$, $x_1=0$ SN~Ia template at ${\hat{B}}$-band peak, the 
\citetalias{1999PASP..111...63F} model
can be approximated with
$\vec{a} = (0.96,   1.00,   1.00,   0.97,   0.77)$ and $\vec{b}=(  1.77,   0.98,   0.12,  -0.50,  -0.53)$.
(The $F$ superscript is used to distinguish parameters of the \citetalias{1999PASP..111...63F} model.)
Over the ranges
 $0\le A^F_V\le 1$ and $2 \le R^F \le 3.5$ 
and the wavelengths under consideration here,  the values of the elements of $a$ and $b$ vary by $<5$\%
for the
\citetalias{1999PASP..111...63F} model with
 the largest error in extinction from the linear approximation having amplitude $<0.008$ mag.
All possible magnitude changes in our five bands due to  \citetalias{1999PASP..111...63F}  dust extinction are confined to a two-dimensional
plane defined by $\vec{a}$ and  $\vec{b}$. 


The credible interval for $\vec{\gamma}$'s and other parameters are presented in Table~\ref{global1:tab}.
Our model $\vec{\gamma}$ vectors can be written as a linear combination (described by the $2 \times 2$ matrix M) of the
dust-based $\vec{a}$ and $\vec{b}$ vectors plus residual vectors perpendicular to the plane spanned by $\vec{a}$ and $\vec{b}$,
\begin{equation}
\begin{pmatrix}
\vec{\gamma}^0 \\
\vec{\gamma}^1
\end{pmatrix}=
M
\begin{pmatrix}
\vec{a} \\
\vec{b}
\end{pmatrix}+
\begin{pmatrix}
\vec{\epsilon}_{\gamma^0} \\
\vec{\epsilon}_{\gamma^1}
\end{pmatrix}.
\label{trans_I:eqn}
\end{equation}
We find
\begin{equation}
M=
\begin{pmatrix}
\begin{array}{rr}
43.7^{+2.3}_{-2.1} & 19.4^{+0.5}_{-0.5} \\
-18.7^{+3.0}_{-3.1} & -0.8^{+0.9}_{-1.0} \\
\end{array}
\end{pmatrix} ,
\end{equation}
\begin{align}
\begin{split}
\vec{\epsilon}_{\gamma^0} &=\left(1.5^{+0.3}_{-0.3} , -1.4^{+0.3}_{-0.3} , -1.8^{+0.4}_{-0.3} , 1.4^{+0.1}_{-0.1} , 0.5^{+0.4}_{-0.5} \right) \text{mag}, \\
\vec{\epsilon}_{\gamma^1} & =\left(4.8^{+0.3}_{-0.3} , -4.3^{+0.3}_{-0.3} , -5.7^{+0.4}_{-0.4} , 0.1^{+0.2}_{-0.2} , 6.8^{+0.5}_{-0.5}\right) \text{mag},
\end{split}
\label{res_1:eqn}
\end{align}
which correspond to
a quadratic fraction of $0.9993^{+0.0002}_{-0.0003}$ of $\vec{\gamma}^0$ and
$0.9303^{+0.0210}_{-0.0341}$ of $\vec{\gamma}^1$ projecting onto the $\vec{a}$--$\vec{b}$ plane.
The  contributions of these residuals to magnitudes are relatively small compared to
the values of $\vec{\gamma}^0$ and $\vec{\gamma}^1$ in Table~\ref{global1:tab}.
For the standard deviation values of $g_0$ and $g_1$ of the supernovae in our sample, the amplitude of the magnitude residuals due to $\vec{\epsilon}$ are 
$\left(0.025, 0.023,0.029, 0.008, 0.027\right)$ mag.
The allowed color
variations in ${\hat{U}}{\hat{B}}{\hat{V}}{\hat{R}}{\hat{I}}$ allowed by the \citetalias{1999PASP..111...63F} model and our best-fit model are confined to almost identical
2-dimensional
planes within
the 5-dimensional magnitude space.
With no prior assumptions of dust extinction behavior or the distribution of $A_V$, the supernova data themselves exhibit
2-dimensional color variations that are closely aligned with the 2-dimensional color variations predicted by the \citetalias{1999PASP..111...63F} dust model.


The above result is visualized in Figure~\ref{plane:fig}, which  shows  in the
${\hat{U}}{\hat{V}}\hat{I}$-subspace
two perspectives (left and right panels)
of the unit vectors corresponding to $\vec{\gamma}^0$ and  $\vec{\gamma}^1$ of our model (solid lines),
and $\vec{a}$, $\vec{b}$ of the \citetalias{1999PASP..111...63F} model (dashed lines).  All  vectors are set to intersect the origin.
The two perspectives show that while the four vectors point in different directions for each band combination,
they are almost coplanar in ${\hat{U}}{\hat{V}}{\hat{I}}$ (they are slightly less coplanar in ${\hat{B}}{\hat{V}}{\hat{R}}$).  The $\vec{a}$ and $\vec{b}$ vectors and the $\vec{\gamma}^0$ and $\vec{\gamma}^1$
vectors span near-parallel planes in the 5-dimensional color space, and hence the latter can be almost entirely expressed in terms of
the former.

\begin{figure}[htbp] %  figure placement: here, top, bottom, or page
   \centering
   \includegraphics[width=2.95in]{fix1plane0.pdf}
   \includegraphics[width=2.95in]{fix1plane1.pdf}
   \caption{
   Visualization of how supernova magnitudes can vary in Model~I and that of \citetalias{1999PASP..111...63F}.  While the models describe
   magnitudes in 5-bands ${\hat{U}}{\hat{B}}{\hat{V}}{\hat{R}}{\hat{I}}$ this visualization shows only ${\hat{U}}{\hat{V}}{\hat{I}}$.   The left and right plots show the same information from
   two different perspectives.
   Our model is shown in solid-line unit vectors in the directions of $\vec{\gamma}^0$ and $\vec{\gamma}^1$. The only possible magnitudes
   are a linear combination of these two vectors, and hence are confined to the plane that contains both.
   The  \citetalias{1999PASP..111...63F} model is shown in dashed-line unit vectors
   in the directions of  $\vec{a}$, $\vec{b}$; the only possible magnitudes are confined to the plane that contains both.
   The perspective on the right visually shows that all four vectors are nearly coplanar.  
   The combination $\vec{a}+\vec{b}/2.25$ is shown in the dotted red
   line: it is almost perfectly superimposed on $\vec{\gamma}^0$.
   \label{plane:fig}}
\end{figure}


We
now turn to
the specific choice of $\vec{\gamma}$'s  returned by the fit in terms of the \citetalias{1999PASP..111...63F}  dust-extinction model. 
Our linear model would be satisfied by any two independent vectors  that span
the plane defined by the $\vec{\gamma}$ vectors. 
While
Eq.~\ref{f99:eqn} expresses that model using basis vectors $\vec{a}$ and $\vec{b}$ and
their corresponding parameters 
$A_V$ and 
$E(B-V)$, it can be written more generally using arbitrary bases $\vec{a}+\kappa_1 \vec{b}$
and $\kappa_2 \vec{a} + \vec{b}$, and parameter sets
$A^F_V - \kappa_2 E^F(B-V)$ and $-\kappa_1 A^F_V + E^F(B-V)$  such that 
\begin{equation}
A_X =  (1-\kappa_1 \kappa_2)^{-1} [(A^F_V - \kappa_2 E^F(B-V))\left(a_X+\kappa_1 b_X \right) +  (-\kappa_1 A^F_V + E^F(B-V)) (\kappa_2 a_X + b_X)],
\label{newdust:eqn}
\end{equation}
where $\kappa_1$ and $\kappa_2$ are free to float.
In our analysis, the $\vec{\gamma}$ vectors converge to a specific direction, meaning that our result prefers specific values of
$\kappa_1=M[1,2]/M[1,1]$ and $\kappa_2=M[2,1]/M[2,2]$, where $M$ is the matrix in  Eq.~\ref{trans_I:eqn}.

The values of $\kappa$'s are determined by the conditions that the components $A^F_V - \kappa_2 E^F(B-V)$ and $-\kappa_1 A^F_V + E^F(B-V)$ be
uncorrelated with $\Delta$.   The corresponding basis vectors are $a_X+\kappa_1 b_X $ and $\kappa_2 a_X + b_X$.
Defining an effective total to selective extinction $\langle R^F \rangle \equiv \kappa_1^{-1}$ so that the first basis vector is
$a_X+ b_X/\langle R^F \rangle  $, we find from the 68\%-ile credible interval of $\kappa_1^{-1}$ that
$\langle R^F \rangle=2.25^{+0.11}_{-0.10}$.
The non-correlation condition does not strongly constrain the  second basis vector as
$\kappa_2 =  15.60^{+26.19}_{-32.47}$.
Figure~\ref{plane:fig} visualizes the near-perfect alignment of $\vec{\gamma}^0$ and  $\vec{a}+\vec{b}/2.25$.


\subsection{Model~I Summary}
There are two points to highlight from the analysis of
Model~I.
\begin{itemize}
\item Model~I introduces two latent supernova parameters that can influence supernova colors.
These parameters come  with ``non-informative'' priors.  Fitting
of our model gives a space of possible color variations highly consistent with those expected from two-parameter dust-extinction models. 
That we recover a physically-motivated  result lends confidence in the modeling and analysis.
\item In Model~I, any  color variation that remains after spectral corrections is allocated to two parameters.
Dust extinction is not the only effect that can contribute to $g_0$ and $g_1$ so interpreting
them as pure dust parameters is subject to systematic errors.
\item We have defined an effective parameter $\langle R^F\rangle$ whose value may be compared with dust models.
Such comparisons must be taken with a grain of salt,
as this value is not fundamentally constrained by the model but is fixed by the added condition that $\Delta$ is
uncorrelated with  $EW_{\text{Ca}}$, $EW_{\mathrm{Si}}$, $\lambda_{\text{Ca}}$,
$g_0$, and $g_1$.  The value of  $\langle R^F\rangle$ changes given different ansatzes  for breaking the $\Delta$--cofactor degeneracy 
\end{itemize}

\section{Model~II: Three Color Parameters}
\label{modelII:sec}
Model~II extends Model~I to include a third color parameter.  This increase in the number of degrees of freedom 
goes beyond the two of dust-extinction models used in supernova analysis.
 In this section we  focus on describing the consequences of the addition of the
new parameter.

\subsection{Model}
\label{modelIImodel:sec}
Model~II is identical to Model~I except for the 
addition of a third color parameter $p$.  The distribution of $p$ in our sample
is assumed to form a Normal distribution with unit standard deviation.  An unconstrained distribution,
as taken for $g_0$ and $g_1$, yields multiple local maxima in the posterior and leads to difficulty in the convergence of the MCMC.
The relative effect on the five magnitudes is specified by the global parameters 
$\sigma_p\vec{\phi}$, where $\vec{\phi}$ is a 5-dimensional unit vector.
Model~II reads as
\begin{equation}
p  \sim \mathcal{N}(0,1 ),
\end{equation}
\begin{equation}
\sigma_p  \sim \text{Cauchy}(0.1,0.1),
\end{equation}
\begin{equation}
\begin{pmatrix}
{\hat{U}}\\{\hat{B}}\\{\hat{V}}\\{\hat{R}}\\{\hat{I}}
\end{pmatrix}
=
\Delta  \vec{1} +
\begin{pmatrix}
c_{\hat{U}}+\alpha_{\hat{U}} EW_{\mathrm{Ca}} + \beta_{\hat{U}} EW_{\mathrm{Si}} + \eta_{\hat{U}} \lambda_{\mathrm{Si}}+\gamma^0_{{\hat{U}}} g_0 +\gamma^1_{{\hat{U}}} g_1  + \sigma_p\phi_{\hat{U}} p\\
c_{\hat{B}}+\alpha_{\hat{B}} EW_{\mathrm{Ca}} + \beta_{\hat{B}} EW_{\mathrm{Si}} + \eta_{\hat{B}} \lambda_{\mathrm{Si}} +\gamma^0_{{\hat{B}}} g_0 +\gamma^1_{{\hat{B}}} g_1  + \sigma_p\phi_{\hat{B}} p\\
c_{\hat{V}}+\alpha_{\hat{V}} EW_{\mathrm{Ca}} + \beta_{\hat{V}} EW_{\mathrm{Si}} + \eta_{\hat{V}} \lambda_{\mathrm{Si}} +\gamma^0_{{\hat{V}}} g_0 +\gamma^1_{{\hat{V}}} g_1 + \sigma_p\phi_{\hat{V}}  p\\
c_{\hat{R}}+\alpha_{\hat{R}} EW_{\mathrm{Ca}} + \beta_{\hat{R}} EW_{\mathrm{Si}} + \eta_{\hat{R}} \lambda_{\mathrm{Si}} +\gamma^0_{{\hat{R}}} g_0 +\gamma^1_{{\hat{R}}} g_1 + \sigma_p\phi_{\hat{R}} p\\
c_{\hat{I}}+\alpha_{\hat{I}} EW_{\mathrm{Ca}} + \beta_{\hat{I}} EW_{\mathrm{Si}}+ \eta_{\hat{I}} \lambda_{\mathrm{Si}}+\gamma^0_{{\hat{I}}} g_0 +\gamma^1_{{\hat{I}}} g_1 + \sigma_p\phi_{\hat{I}}  p
\end{pmatrix}.
\label{ewsiv2:eqn}
\end{equation}

The Cauchy prior is commonly used to concentrate probability around a specific value while having fat tails with infinite variance.


The model contains the product 
$\vec{\phi} p$, which has the degeneracy $\vec{\phi} \rightarrow -\vec{\phi}$, $p \rightarrow -p$.
No conditions are added to break this degeneracy to leave open the possibility that
$\sigma_p\vec{\phi}$ is consistent with zero.  The degeneracy
in $\vec{\phi}$ is apparent later in the article, but it does not affect the magnitude predictions. 
As in \S\ref{modelI:sec}, the degeneracy between $\vec{\phi}$ and $\Delta$ is fixed by enforcing zero correlation between $\Delta$ and $p$.

For $N$ supernovae there are $8N$ observables.  There are $3N$ spectral parameters, each of
$\Delta$, $g_0$, $g_1$ contributes $N-1$ parameters while $p$  contributes $N$ parameters,  and  there are $5 \times 7$ global coefficients.
For $N=172$ supernovae, there are 1376 observables and 1236  parameters.

\subsection{Results for the Latent Parameter Cofactor $\sigma_p \vec{\phi}$}
\label{results2:sec}
The credible intervals of all Model~II global parameters are given in  \S\ref{m12results:sec}.
In this subsection we focus on the effect of adding the third color term  $\sigma_p \vec{\phi}$.
There is strong evidence for a third color parameter.
The 68\% credible interval for $\sigma_p$ is $0.051^{+0.005}_{-0.004}$~mag, highly incompatible with zero. 
The posterior for $\sigma_p \vec{\phi}$ is shown in Figure~\ref{M1ev:fig}.  
Two maxima are allowed due to the sign degeneracy for $\vec{\phi}$.  Practically each of the MCMC chains converges to one of the solutions and does
not migrate between them by crossing zero.  To aid visibility of this figure we thus impose $\phi_I > 0$.  
 The new parameter does affect colors in a  manner distinct from $g_0$ and $g_1$.
A quadratic fraction of
$0.48^{+0.25}_{-0.17}$ of 
 $\vec{\phi}$ lies perpendicular to the $\vec{\gamma}_0$--$\vec{\gamma}_1$ plane.


\begin{figure}[htbp] %  figure placement: here, top, bottom, or page
   \centering
   \includegraphics[width=4in]{fix3sigev.pdf} 
            \caption{Model~II posterior  contours for $\sigma_p \vec{\phi}$. 
            The solid line shows the location of zero.          \label{M1ev:fig}}
\end{figure}

Model~II introduces a new parameter  that can absorb color variability
that in Model~I is restricted to $g_0$ and $g_1$, and whose distribution comes from a Normal distribution.
The change in the distributions of the $\gamma^0_{\hat{V}} g_0$ and $\gamma^1_{\hat{V}}g_1$ parameters from
 Model~I to Model~II
is apparent in Figure~\ref{kcomp:fig}.  (Here and in similar plots, the parameter value of one supernova,
e.g.\ $\gamma^0_{\hat{V}} g_0|_0$, is subtracted
out to null out correlated
parameter uncertainty between supernovae; recall the $\langle \Delta \rangle=0$, $\langle g_0 \rangle=0$, $\langle g_1 \rangle=0$,
and $p  \sim \mathcal{N}(0,1)$ conditions.) The  change from Model~I to II is pronounced in  $\gamma^1_{\hat{V}}g_1$,
where  a Normal distribution component apparent in Model~I disappears in Model~II, presumably having been  assigned to $p$.


\begin{figure}[htbp] %  figure placement: here, top, bottom, or page
   \centering
   \includegraphics[width=2.6in]{fix1deltagamma0_med.pdf} 
   \includegraphics[width=2.6in]{fix1deltagamma1_med.pdf} 
   \includegraphics[width=2.6in]{fix3deltagamma0_med.pdf} 
   \includegraphics[width=2.6in]{fix3deltagamma1_med.pdf} 
   \caption{
   Left:
   Normalized stack of the posteriors (includes parameter uncertainties) of all supernovae  and histogram 
      of  per-supernova 
median values of $\gamma^0_{\hat{V}} g_0-\gamma^0_{\hat{V}} g_0|_0$.  Right:  The same plots for $\gamma^1_{\hat{V}} g_1-\gamma^1_{\hat{V}} g_1|_0$,
Top: Model~I. Bottom: Model~II.
    \label{kcomp:fig}}
\end{figure}

The $\vec{\gamma}$ vectors, like those from Model~I, are almost entirely projected
onto the plane defined by
$\vec{a}$ and $\vec{b}$ of the \citetalias{1999PASP..111...63F} model.
The direction of the  $\vec{\gamma}^0$ vector corresponds to an effective
$\langle R^F_{\mathit{eff}}\rangle = 2.54^{+0.13}_{-0.12}$.
The change of $\langle R^F_{\mathit{eff}}\rangle$ from Model~I to II is anticipated by the change in the distribution of $g_1$,
whose connection with $R^F$ is shown in \S\ref{g0g1:sec}.


\subsection{Model~II Summary}
The inclusion of a third color term beyond the two of Model~I has important consequences.

\begin{itemize}
\item The data strongly supports the influence of an additional supernova parameter that affects supernova colors in a manner orthogonal to
 dust extinction.
\item The effective $\langle R^F_{\mathit{eff}}\rangle$ shifts by $>1 \sigma$ when including the new color term.  This demonstrates,
as has been noted by \citet{2009ApJ...704.1036F,2009ApJS..185...32K}, that dust analyses
that do not allow for additional color dispersion are susceptible to bias when inferring dust parameters.
\end{itemize}

\section{Model~III: Three Color Parameters and Light-Curve Shape}
\label{modelIII:sec}

Light-curve shape is an established indicator of SN~Ia diversity 
\citep{1993ApJ...413L.105P, 1996ApJ...473...88R,
1997ApJ...483..565P}
with its own color correlations \citep{2005A&A...443..781G, 2007ApJ...659..122J}.
Model~III explicitly includes light-curve shape as a feature used to standardize supernova colors, to complement the
spectral features considered in the previous two models.
We will show that light-curve shape does have a non-trivial relationship with color that is not already encoded
in the spectral features.  Model~III is thus the focus of
this article and its results are  described here in depth.


\subsection{Model}
\label{modelIImodel:sec}
Model~III is identical to Model~II except for the 
addition of a new term  that relates magnitudes with the SALT2 light-curve-shape parameter,
$x_1$, through linear coefficients $\vec{\zeta}$. 
The model reads as
\begin{equation}
p  \sim \mathcal{N}(0,1 ),
\end{equation}
\begin{equation}
\sigma_p  \sim \text{Cauchy}(0.1,0.1),
\end{equation}
\begin{equation}
\begin{pmatrix}
{\hat{U}}\\{\hat{B}}\\{\hat{V}}\\{\hat{R}}\\{\hat{I}}
\end{pmatrix}
=
\Delta  \vec{1} +
\begin{pmatrix}
c_{\hat{U}}+\alpha_{\hat{U}} EW_{\mathrm{Ca}} + \beta_{\hat{U}} EW_{\mathrm{Si}} + \eta_{\hat{U}} \lambda_{\mathrm{Si}} +\zeta_{\hat{U}} x_1 +\gamma^0_{{\hat{U}}} g_0 +\gamma^1_{{\hat{U}}} g_1 + \sigma_p\phi_{\hat{U}} p\\
c_{\hat{B}}+\alpha_{\hat{B}} EW_{\mathrm{Ca}} + \beta_{\hat{B}} EW_{\mathrm{Si}} + \eta_{\hat{B}} \lambda_{\mathrm{Si}} +\zeta_{\hat{B}} x_1 +\gamma^0_{{\hat{B}}} g_0 +\gamma^1_{{\hat{B}}} g_1 + \sigma_p\phi_{\hat{B}} p\\
c_{\hat{V}}+\alpha_{\hat{V}} EW_{\mathrm{Ca}} + \beta_{\hat{V}} EW_{\mathrm{Si}} + \eta_{\hat{V}} \lambda_{\mathrm{Si}} +\zeta_{\hat{V}} x_1+\gamma^0_{{\hat{V}}} g_0 +\gamma^1_{{\hat{V}}} g_1 + \sigma_p\phi_{\hat{V}}  p\\
c_{\hat{R}}+\alpha_{\hat{R}} EW_{\mathrm{Ca}} + \beta_{\hat{R}} EW_{\mathrm{Si}} + \eta_{\hat{R}} \lambda_{\mathrm{Si}} +\zeta_{\hat{R}} x_1+\gamma^0_{{\hat{R}}} g_0 +\gamma^1_{{\hat{R}}} g_1 + \sigma_p\phi_{\hat{R}} p\\
c_{\hat{I}}+\alpha_{\hat{I}} EW_{\mathrm{Ca}} + \beta_{\hat{I}} EW_{\mathrm{Si}}+ \eta_{\hat{I}} \lambda_{\mathrm{Si}}+\zeta_{\hat{I}} x_1+\gamma^0_{{\hat{I}}} g_0 +\gamma^1_{{\hat{I}}} g_1 + \sigma_p\phi_{\hat{I}}  p
\end{pmatrix},
\label{ewsiv3:eqn}
\end{equation}

\begin{equation}
\begin{pmatrix}
{\hat{U}}_o\\{\hat{B}}_o\\ {\hat{V}}_o\\{\hat{R}}_o\\{\hat{I}}_o\\EW_{Si, o}\\ EW_{Ca, o} \\ \lambda_{Si, o} \\ x_{1, o}
\end{pmatrix}
\sim \mathcal{N}
\left(
\begin{pmatrix}
{\hat{U}}  \\{\hat{B}}  \\
{\hat{V}}\\{\hat{R}}\\{\hat{I}}\\
EW_{\mathrm{Si}}\\ EW_{\mathrm{Ca}} \\ \lambda_{\mathrm{Si}} \\ x_1
\end{pmatrix}
,C
\right).
\label{dust3:eqn}
\end{equation}
The changes from Model~II are the addition of the $\vec{\zeta} x_1$ terms in Eq.~\ref{ewsiv3:eqn}, and
the addition of the shape measurement and uncertainty (implicit in the covariance matrix $C$) and $x_1$   in Eq.~\ref{dust3:eqn}.
As in \S\ref{modelI:sec}, the degeneracy between $\vec{\zeta}$ and $\Delta$ is fixed by enforcing zero correlation between $\Delta$ and $x_1$.
For $N$ supernovae there are $9N$ observables.  There are $4N$ spectral and light-curve shape parameters, each of
$\Delta$, $g_0$, $g_1$ contributes $N-1$ parameters while $p$  contributes $N$ parameters,  and there are $5 \times 8$ global coefficients.
For $N=172$ supernovae, this makes 1548 observables and 1413  parameters.


\subsection{Results}
\subsubsection{Analysis Method and Validation}

In the analysis of this and previous sections of this article,
the posterior of the model parameters is evaluated using Hamiltonian Monte Carlo with a No-U-Turn
Sampler as implemented in
STAN \citep{JSSv076i01}.  We run eight chains, each with 5000 iterations of which
half are used for warmup.
STAN provides output statistics to assess
the convergence of the output Markov chains.
The 
potential scale reduction statistic, $\hat{R}$
\citep[][in this paragraph not to be confused with the synthetic $\hat{R}$-band magnitude]{Gelman92}, measures the convergence of the target distribution
in iterative simulations 
by using multiple independent sequences to estimate how much that distribution would sharpen if the simulations were run longer.
$N_{\mathit{eff}}$ is an estimate of the number of independent draws. The STAN output gives $\hat{R} \sim 1.0$ for all parameters, meaning there is no evidence for non-convergence.  The
output also gives  $N_{\mathit{eff}} \gg 100$ for all parameters, indicating that they are densely sampled.


Our analysis pipeline is run with input simulated data with known  signals and measurement uncertainties similar to those of our dataset.
Stacks of the posteriors from 100 simulated datasets  exhibit no biases that are significant relative to the statistical uncertainties of a single posterior.

As will be seen in  \S\ref{results3global:sec}, the posterior derived from the analysis has a distinct bounded  
peak showing no indication that the MCMC has failed to converge.

The model provides a fair representation of the data as seen in the residual differences between the data and the model prediction.
Figure~\ref{residual:fig} shows plots of 
residuals between measured and model-expectation colors as a function of observed $\hat{B}_o-\hat{V}_o$.
This and similar plots for the other permutations of observed colors show  no apparent trend between residuals and observed
color nor any catastrophic outliers.

\begin{figure}[htbp] %  figure placement: here, top, bottom, or page
   \centering
   \includegraphics[width=4in]{cpull.pdf} 
               \caption{
               Residuals between measured and model-expectation colors as a function of observed $\hat{B}_o-\hat{V}_o$.  The colors are
               relative to on arbitrary supernova. Plots that show
               residual
               colors containing $\hat{B}_o$ or $\hat{V}_o$ are shaded to highlight that the abscissa and ordinate have contributions
               from  a common measurement.
            \label{residual:fig}}
\end{figure}

This suite of tests lend confidence in our model and the validity of the derived posterior.

\subsubsection{Global Parameter Posterior Results}
\label{results3global:sec}
Results from our analysis are shown in Figures~\ref{global1:fig} -- \ref{global5:fig}
as contours of the posterior surface for pairs of global parameters grouped by filter.
The confidence regions are localized and unimodal, except for the combination $\sigma_p \vec{\phi}$ that
has the sign degeneracy explained in \S\ref{results2:sec} and plotted in Figure~\ref{M1ev:fig}.
Within the finite number of generated links, the MCMC chains do not migrate between the parity-degenerate $\vec{\gamma}$--$g$ solutions.
Each chain converges to one of the  $\vec{\phi}$--$p$ degenerate solutions, away from which it does not depart.  

\begin{figure}[htbp] %  figure placement: here, top, bottom, or page
   \centering
   \includegraphics[width=5.2in]{coeff0.pdf} 
            \caption{Model~III posterior contours for $\vec{c}$, $\vec{\alpha}$, $\vec{\beta}$, $\vec{\eta}$, $\vec{\gamma}^0$, $\vec{\gamma}^1$, and $\sigma_p \vec{\phi}$ in the ${\hat{U}}$ band.
            The contours shown here and in future plots represent 1-$\sigma$ in the parameter distribution (i.e.\ they should be
            projected onto the corresponding 1-d parameter axis), not to 68\%, 95\%, etc.\
            enclosed probability.  Lines for zero value for $\alpha_{\hat{U}}$, $\beta_{\hat{U}}$, $\eta_{\hat{U}}$, $\gamma_{\hat{U}}^0$, $\gamma_{\hat{U}}^1$, and $\sigma_p \phi_{\hat{U}}$ are shown for reference --
            in most cases they are outside the range of the plot.
            \label{global1:fig}}
\end{figure}

\begin{figure}[htbp] %  figure placement: here, top, bottom, or page
   \centering
   \includegraphics[width=5.2in]{coeff1.pdf} 
            \caption{Model~III posterior contours for $\vec{c}$, $\vec{\alpha}$, $\vec{\beta}$, $\vec{\eta}$,  $\vec{\gamma}^0$, $\vec{\gamma}^1$, and $\sigma_p \vec{\phi}$ in the ${\hat{B}}$ band.
 \label{global2:fig}}
\end{figure}

\begin{figure}[htbp] %  figure placement: here, top, bottom, or page
   \centering
   \includegraphics[width=5.2in]{coeff2.pdf} 
            \caption{Model~III posterior contours for $\vec{c}$, $\vec{\alpha}$, $\vec{\beta}$, $\vec{\eta}$, $\vec{\gamma}^0$, $\vec{\gamma}^1$, and $\sigma_p \vec{\phi}$ in the ${\hat{V}}$ band.
 \label{global3:fig}}
\end{figure}

\begin{figure}[htbp] %  figure placement: here, top, bottom, or page
   \centering
      \includegraphics[width=5.2in]{coeff3.pdf} 
            \caption{Model~III posterior contours for  $\vec{c}$, $\vec{\alpha}$, $\vec{\beta}$, $\vec{\eta}$,  $\vec{\gamma}^0$, $\vec{\gamma}^1$, and $\sigma_p \vec{\phi}$ in the ${\hat{R}}$ band.
 \label{global4:fig}}
\end{figure}

\begin{figure}[htbp] %  figure placement: here, top, bottom, or page
   \centering
         \includegraphics[width=5.2in]{coeff4.pdf} 
            \caption{Model~III posterior contours for  $\vec{c}$, $\vec{\alpha}$, $\vec{\beta}$, $\vec{\eta}$, $\vec{\gamma}^0$, $\vec{\gamma}^1$, and $\sigma_p \vec{\phi}$ in the ${\hat{I}}$ band.
 \label{global5:fig}}
\end{figure}


For each of the five filters, the 68\%  equal-tailed credible intervals for the global parameters $\vec{\alpha}$, $\vec{\beta}$, $\vec{\eta}$, $\vec{\zeta}$,
$\vec{\gamma}^0$, $\vec{\gamma}^1$, and $\sigma_p\vec{\phi}$
are given in Table~\ref{global2:tab}.
In constructing the credible interval of $\sigma_p\vec{\phi}$ 
we set $\vec{\phi} = -\text{sign}(\phi_{\hat{V}}) \vec{\phi}$ to break its parity degeneracy;
Table~\ref{global3:tab} shows it has the largest signal-to-noise in
in the $\hat{V}$-band.
The effect of spectral parameters on color (as opposed to magnitude)
is shown in the rows of $\alpha_X/\alpha_{\hat{V}}-1$,  $\beta_X/\beta_{\hat{V}}-1$, and  $\eta_X/\eta_{\hat{V}}-1$.
The normalization freedom of $\vec{\gamma}$ is nulled out in the statistic
 $\gamma_X/\gamma_{\hat{V}}-1$.
 These statistics give the change in $X-\hat{V}$ color normalized to the change in $\hat{V}$ magnitude, and are analogous to
$R_V^{-1}=A_B/A_V-1$ as an observational descriptor of dust properties.

\begin{table}
\centering
\begin{tabular}{|c|c|c|c|c|c|}
\hline
Parameters& $X={\hat{U}}$ &${\hat{B}}$&${\hat{V}}$&${\hat{R}}$&${\hat{I}}$\\ \hline
$\alpha_X$
& $0.0051^{+0.0007}_{-0.0008}$
& $0.0025^{+0.0006}_{-0.0007}$
& $0.0022^{+0.0005}_{-0.0005}$
& $0.0019^{+0.0004}_{-0.0004}$
& $0.0034^{+0.0003}_{-0.0004}$
\\
${\alpha_X}/\alpha_{\hat{V}}-1$
& $   1.3^{+   0.4}_{  -0.2}$
& $   0.1^{+   0.1}_{  -0.1}$
& \ldots
& $  -0.1^{+   0.0}_{  -0.0}$
& $   0.5^{+   0.3}_{  -0.2}$
\\
$\beta_X$
& $ 0.035^{+ 0.005}_{-0.005}$
& $ 0.024^{+ 0.004}_{-0.004}$
& $ 0.021^{+ 0.004}_{-0.004}$
& $ 0.019^{+ 0.003}_{-0.003}$
& $ 0.010^{+ 0.003}_{-0.003}$
\\
${\beta_X}/\beta_{\hat{V}}-1$
& $  0.68^{+  0.16}_{ -0.13}$
& $  0.14^{+  0.07}_{ -0.07}$
&\ldots
& $ -0.11^{+  0.03}_{ -0.02}$
& $ -0.50^{+  0.06}_{ -0.07}$
\\
$\eta_X$
& $0.0003^{+0.0011}_{-0.0011}$
& $0.0005^{+0.0009}_{-0.0009}$
& $0.0008^{+0.0007}_{-0.0007}$
& $0.0007^{+0.0006}_{-0.0006}$
& $-0.0000^{+0.0005}_{-0.0005}$
\\
${\eta_X}/\eta_{\hat{V}}-1$
& $ -0.39^{+  0.64}_{ -1.36}$
& $ -0.26^{+  0.35}_{ -0.77}$
&\ldots
& $ -0.19^{+  0.12}_{ -0.11}$
& $ -0.86^{+  0.33}_{ -0.71}$
\\
$\zeta_X$
& $ -0.00^{+  0.04}_{ -0.04}$
& $ -0.02^{+  0.03}_{ -0.03}$
& $ -0.05^{+  0.03}_{ -0.03}$
& $ -0.03^{+  0.02}_{ -0.02}$
& $ -0.08^{+  0.02}_{ -0.02}$
\\
${\zeta_X}/\zeta_{\hat{V}}-1$
& $ -0.94^{+  0.48}_{ -1.15}$
& $ -0.54^{+  0.28}_{ -0.65}$
& \ldots
& $ -0.42^{+  0.09}_{ -0.21}$
& $  0.47^{+  0.83}_{ -0.32}$
\\
$\gamma^0_X$
& $ 68.91^{+  2.62}_{ -2.56}$
& $ 55.44^{+  2.22}_{ -2.25}$
& $ 41.41^{+  1.98}_{ -1.98}$
& $ 31.49^{+  1.61}_{ -1.62}$
& $ 22.92^{+  1.35}_{ -1.30}$
\\
${\gamma^0_X}/\gamma^0_{\hat{V}}-1$
& $  0.67^{+  0.04}_{ -0.04}$
& $  0.34^{+  0.02}_{ -0.02}$
&\ldots
& $ -0.24^{+  0.01}_{ -0.01}$
& $ -0.45^{+  0.02}_{ -0.02}$
\\
$\gamma^1_X$
& $ -1.85^{+  4.11}_{ -4.05}$
& $  1.47^{+  3.74}_{ -3.58}$
& $ -6.75^{+  3.21}_{ -3.15}$
& $ -6.60^{+  2.71}_{ -2.65}$
& $ -7.65^{+  2.33}_{ -2.24}$
\\
${\gamma^1_X}/\gamma^1_{\hat{V}}-1$
& $ -0.70^{+  0.38}_{ -0.76}$
& $ -1.18^{+  0.43}_{ -1.08}$
& \ldots
& $ -0.03^{+  0.14}_{ -0.08}$
& $  0.12^{+  0.45}_{ -0.22}$
\\
$\sigma_p \phi_X$
& $ 0.002^{+ 0.016}_{-0.017}$
& $-0.041^{+ 0.013}_{-0.014}$
& $-0.050^{+ 0.011}_{-0.012}$
& $-0.032^{+ 0.009}_{-0.010}$
& $ 0.002^{+ 0.008}_{-0.008}$
\\
${\phi_X/\phi_{\hat{V}}-1}$
& $-1.034^{+ 0.291}_{-0.409}$
& $-0.174^{+ 0.103}_{-0.134}$
& \ldots
& $-0.347^{+ 0.042}_{-0.060}$
& $-1.032^{+ 0.141}_{-0.208}$
\\

\hline
\end{tabular}
\caption{68\% credible intervals for the global fit parameters of Model~III in \S\ref{modelIII:sec}.\label{global3:tab}}
\end{table}



\subsubsection{Results for the Observables Cofactors $\vec{\alpha}$, $\vec{\beta}$, $\vec{\eta}$,  $\vec{\zeta}$}

The cofactors  $\vec{\alpha}$, $\vec{\beta}$, and $\vec{\eta}$ have  significant non-zero values,  meaning that
$EW_{\mathrm{Ca}}$, $EW_{\mathrm{Si}}$, and $\lambda_{\mathrm{Si}}$ are indicators of broadband
colors at peak.  The significance of their influence varies between  colors; in $\hat{B}-\hat{V}$, roughly the color
where previous studies have focused, only $\vec{\beta}$ has a strong $>2\sigma$ correlation.  A positive
dependence (consistent with the positive sign
of $\beta_{\hat{B}}/\beta_{\hat{V}}-1$)
of $B-V$ on the pseudo-equivalent width of Si~II~$\lambda$4130 has previously been reported by
\citet{2011ApJ...734...42N, 2013ApJ...773...53F}.
The insignificance of  our $\alpha_{\hat{B}}/\alpha_{\hat{V}}-1$ is consistent with the
lack of correlation between the  pseudo-equivalent width of Ca~II~H\&K measurements and
intrinsic color reported by \citet{2011ApJ...742...89F}.

The story on Si~II~$\lambda$6355 is not as clear.
 \citet{2009ApJ...699L.139W,2011ApJ...742...89F, 2012ApJ...748..127F} find that
higher-velocity SNe~Ia tend to be redder than those with lower velocity. \citet{2012AJ....143..126B} find a
weaker relation than that reported by \citet{2011ApJ...742...89F},
a difference attributed to a better treatment of uncertainties and host-galaxy dust. In contrast,
\citet{2013ApJ...773...53F} find no significant relation.
Our result for $\eta_{\hat{B}}/\eta_{\hat{V}}-1$ also gives no significant relation. 

The significantly non-zero values for  $\alpha_{\hat{U}}$ and $\beta_{\hat{U}}$ from this study can be compared with the $U$-band study of
\citet{2018arXiv180101834N} carried out on a subset of the same supernovae.
Both analyses find a strong correlation between $U$ and the Si and Ca features.

The non-zero values for $\vec{\zeta}$ and their effect on color $\vec{\zeta}-\zeta_{\hat{V}}$  demonstrate that light-curve shape $x_1$ is  an indicator
of broadband colors, in the sense that SNe~Ia with broader light curves are bluer in all colors.
This is as expected based on the results of \citet{2007A&A...466...11G, 2007ApJ...659..122J}.


The signal in  $\vec{\alpha}$ and $\vec{\beta}$ cannot
entirely
be attributed to the equivalent widths themselves.
The range of Si~II~$\lambda$4130 equivalent widths is $\pm 20$~\AA\ whereas the width of the ${\hat{B}}$-band is 851~\AA, so that its direct affect on magnitude
is
$2.5 \log{(20/850)} \sim 0.03$ mag.  
The implied span in ${\hat{B}}$ magnitude based on $\beta_{\hat{B}}$ is 0.54~mag.  Therefore $\beta_{\hat{B}}$ cannot wholly be attributed to the flux deficit
from the line itself.
The Ca~II H\&K equivalent widths have range $\pm 50$~\AA, while the width of the ${\hat{U}}$ band is
701~\AA, so that its direct affect on magnitude
is
$2.5 \log{(50/701)} \sim 0.08$ mags.   The implied span in ${\hat{U}}$ magnitude is  0.21~mag, 
which would imply that  $\alpha_{\hat{U}}$ cannot be completely due to the 
presence of the line itself.  
Supernova flux has a large gradient in our $\hat{U}$ and there is a variation in the effective, flux-weighted,  $\hat{U}$ bandwidth. 
Nevertheless, the  correlation between spectral features with magnitudes in bands with no wavelength overlap is a non-trivial
signal detected in the analysis.

The posterior visualized in
Figures~\ref{global1:fig} -- \ref{global5:fig} shows
 that $\vec{\zeta}$, the magnitude vector associated with $x_1$, is highly correlated with $\vec{\beta}$, the vector
 associated with  $EW_{\mathrm{Si}}$.
The direct correlation between  $x_1$ and $EW_{\mathrm{Si}}$ has already been established
\citep{2008A&A...492..535A, 2011A&A...529L...4C}
and exists
in our sample as is shown later in \S\ref{results3per:sec}.  
Nevertheless the $\vec{\zeta}$ and $\vec{\beta}$   posteriors are bounded and inconsistent with zero, indicating that
 $x_1$ does convey color information independent of our set of spectral features.
For this reason  Model~III is highlighted in this article.

\subsubsection{Results for the Latent Parameter Cofactors $\vec{\gamma}^0$, $\vec{\gamma}^1$}
\label{results3gamma:sec}
As with Model~I in \S\ref{modelI:sec}, all elements of the two $\vec{\gamma}$ parameter vectors 
are significantly non-zero with 
none of the 20000 links of 
our Monte Carlo chains for $\vec{\gamma}$ extending to 0 (see Figures~\ref{global1:fig}--\ref{global5:fig}).
The direction of the vector in the ${\hat{U}}{\hat{V}}{\hat{I}}$-subspace is shown in Figure~\ref{plane3:fig}.

The transformation between the $\vec{\gamma}$  vectors and the   \citetalias{1999PASP..111...63F} vector given in
Eq.~\ref{trans_I:eqn}, is satisfied by the transformation matrix
\begin{equation}
M=
\begin{pmatrix}
\begin{array}{rr}
40.3^{+1.8}_{-1.8} & 16.6^{+0.6}_{-0.6} \\
-5.8^{+3.1}_{-3.0} & 3.4^{+1.0}_{-0.9}
\end{array}
\end{pmatrix} 
\end{equation}
and the two residual vectors
\begin{align}
\begin{split}
\vec{\epsilon}_{\gamma^0} &=\left(1.1^{+0.4}_{-0.4} , -1.2^{+0.4}_{-0.4} , -0.9^{+0.4}_{-0.4} , 0.7^{+0.2}_{-0.2} , 0.5^{+0.6}_{-0.5}\right) \text{mag}, \\
\vec{\epsilon}_{\gamma^1} & =\left(-2.3^{+0.6}_{-0.5} , 4.0^{+0.6}_{-0.6} , -1.3^{+0.6}_{-0.6} , 0.7^{+0.3}_{-0.3} , -1.4^{+0.8}_{-0.8}\right) \text{mag}.
\end{split}
\label{res_3:eqn}
\end{align}
This corresponds to a quadratic fraction of $0.9996^{+0.0002}_{-0.0003}$ of $\vec{\gamma}^0$ that projects onto the plane of
color changes allowed by the \citetalias{1999PASP..111...63F} dust-extinction model
as described by
$\vec{a}$ and $\vec{b}$. The quadratic fraction of $\vec{\gamma}^1$ projected
onto this plane is
$0.8426^{+0.0780}_{-0.1236}$.
For the standard deviation values of $g_0$ and $g_1$ of the supernovae in our sample, the amplitude of the magnitude residuals from $\vec{\epsilon}$ are 
$\left(0.005, 0.012,0.011, 0.007, 0.004 \right)$~mag.
The   $\vec{\gamma}^0$  vector points in the $\langle R^F_{\mathit{eff}}\rangle=2.43^{+0.10}_{-0.10}$ dust-extinction direction.
This value is consistent with the $R_V=2.5$ found in the supernova samples of \citet{2011ApJ...729...55F}, and the
center  of the range of per-supernova $R^F$ values found in the samples of \citet{2014ApJ...789...32B, 2015MNRAS.453.3300A}.
Although the model $\vec{\gamma}$'s cannot be attributed to dust, they behave like the   \citetalias{1999PASP..111...63F} model.

\begin{figure}[htbp] %  figure placement: here, top, bottom, or page
   \centering
   \includegraphics[width=2.95in]{plane0.pdf}
   \includegraphics[width=2.95in]{plane1.pdf}
   \caption{
   Visualization of how supernova magnitudes can vary in Model~III and the dust extinction of \citetalias{1999PASP..111...63F}.
   See Figure~\ref{plane:fig}.
   The combination $\vec{a}+\vec{b}/2.43$ is shown in the dotted red
   line: it is almost perfectly superimposed on $\vec{\gamma}^0$.
   \label{plane3:fig}}
\end{figure}


\subsubsection{Results for the Latent Parameter Cofactor $\sigma_p \vec{\phi}$}

The  new  $p$ parameter has a significant influence on color.  The magnitude of its color effect is non-zero,
with $\sigma_p = 0.053^{+ 0.010}_{-0.005}$~mag.  The smallest value of $\sigma_p$ % new
over all Monte Carlo chains is 0.040~mag.
These numbers are significantly larger than the median data color uncertainty of 0.010~mag (the maximum is 0.042~mag).
The evaluation of 100 simulated data sets with no third color parameter produces no median $\sigma_p$'s  as
high as that of the data, so we assign a $>99$\% confidence in the detection of this third color term.

The direction of $\vec{\phi}$ is not aligned with the plane defined by
the $\vec{\gamma}$ vectors (nor of the  \citetalias{1999PASP..111...63F} dust-extinction model); a quadratic fraction of
$0.49^{+0.27}_{-0.19}$ of  $\vec{\phi}$ lies perpendicular to the $\vec{\gamma}^0$--$\vec{\gamma}^1$ plane.

The values of $\vec{\phi}$ does not change  monotonically with wavelength, as is apparent  in 
the combination $\phi_X/\phi_{\hat{V}}-1$ shown in Figure~\ref{phiratio:fig}.  An object that has fainter  $\hat{U}$ or $\hat{I}$-band peak magnitudes
has brighter  $\hat{B}$$\hat{V}$$\hat{R}$  peaks.  That brightening is non-uniform, in that
the reddening of $\hat{U}-\hat{B}$ and $\hat{B}-\hat{V}$ is accompanied
by a bluing in $\hat{V}-\hat{R}$ and  $\hat{R}-\hat{I}$. 

What is measured as $\sigma_p \vec{\phi}$ likely captures the effect of many physical processes.
To simplify the model it is assumed that the $p$'s are drawn from a Normal distribution, which may not be a fair description of what occurs in nature
and result in model bias.
The physical interpretation of this color should be tempered with caution.

\begin{figure}[htbp] %  figure placement: here, top, bottom, or page
   \centering
      \includegraphics[width=4in]{phiratio.pdf}
   \caption{Model~III median values  and corresponding 68\% intervals for $\phi_X/\phi_{\hat{V}}-1$ in the 5 bands.
   A dotted line at zero is shown for reference.
   \label{phiratio:fig}}
\end{figure}


\subsubsection{Results for the Per-Supernova Parameters}
\label{results3per:sec}
Each supernova is described by its parameters $\Delta$, $EW_{\mathrm{Ca}}$, $EW_{\mathrm{Si}}$, $\lambda_{\mathrm{Si}}$, $x_1$,
$E_{\gamma^0}({\hat{B}}-{\hat{V}})=(\gamma^0_{\hat{B}}-\gamma^0_{\hat{V}})g_0$, $E_{\gamma^1}({\hat{B}}-{\hat{V}})=(\gamma^1_{\hat{B}}-\gamma^1_{\hat{V}})g_1$, and $A_{p,V} =  \sigma_p\phi_{\hat{V}}p$.  (The  three latent parameters are cast as physically meaning
quantities by multiplying them by select cofactors.)
\added{The 68\% credible intervals for these parameters for all supernovae are given in Table~\ref{results:tab}.
\startlongtable
\begin{deluxetable}{crrrrrrrrr}
\tabletypesize{\tiny}
\tablecaption{Supernova Parameter Credible Intervals
\label{results:tab}}
\tablehead{
\colhead{Name} & \colhead{$EW_{\mathrm{Ca}}$ (\AA)} & \colhead{$EW_{\mathrm{Si}}$ (\AA)} & \colhead{$\lambda_{\mathrm{Si}}$ (\AA)} & \colhead{$x_1$} & \colhead{$\Delta$}
& \colhead{$E_{\gamma^0}({\hat{B}}-{\hat{V}})$} & \colhead{$E_{\gamma^1}({\hat{B}}-{\hat{V}})$} & \colhead{$A_{p,V} $}
}
\startdata
SN2007bd & $113.3^{+5.4}_{-5.4}$ & $ 17.8^{+0.7}_{-0.7}$ & $ 6101^{+  3}_{-  3}$ & $ -1.02^{+  0.18}_{-  0.18}$ & $-0.071^{+0.027}_{-0.027}$  & $-0.050^{+0.011}_{-0.012}$ & $-0.042^{+0.007}_{-0.007}$ & $-0.042^{+0.007}_{-0.007}$\\
PTF10zdk & $149.7^{+1.2}_{-1.2}$ & $ 14.9^{+0.6}_{-0.6}$ & $ 6150^{+  3}_{-  3}$ & $  0.37^{+  0.14}_{-  0.13}$ & $-0.029^{+0.026}_{-0.025}$  & $0.062^{+0.012}_{-0.012}$ & $-0.042^{+0.007}_{-0.007}$ & $-0.042^{+0.007}_{-0.007}$\\
SNF20080815-017 & $ 90.9^{+14.3}_{-14.5}$ & $ 19.6^{+2.4}_{-2.3}$ & $ 6131^{+  6}_{-  6}$ & $ -1.96^{+  0.16}_{-  0.16}$ & $0.041^{+0.070}_{-0.064}$  & $-0.008^{+0.025}_{-0.026}$ & $-0.011^{+0.028}_{-0.022}$ & $-0.011^{+0.028}_{-0.022}$\\
PTF09dnl & $130.0^{+0.8}_{-0.9}$ & $  9.1^{+0.7}_{-0.7}$ & $ 6092^{+  3}_{-  3}$ & $  0.59^{+  0.10}_{-  0.10}$ & $-0.061^{+0.021}_{-0.021}$  & $0.014^{+0.009}_{-0.009}$ & $0.000^{+0.011}_{-0.011}$ & $0.000^{+0.011}_{-0.011}$\\
SN2010ex & $114.5^{+0.9}_{-0.9}$ & $  8.4^{+0.4}_{-0.4}$ & $ 6129^{+  6}_{-  6}$ & $  0.17^{+  0.12}_{-  0.11}$ & $0.266^{+0.026}_{-0.026}$  & $-0.042^{+0.008}_{-0.008}$ & $-0.039^{+0.010}_{-0.008}$ & $-0.039^{+0.010}_{-0.008}$\\
PTF09dnp & $ 66.9^{+4.5}_{-4.6}$ & $ 17.5^{+0.7}_{-0.7}$ & $ 6098^{+  5}_{-  4}$ & $ -0.53^{+  0.17}_{-  0.17}$ & $-0.249^{+0.026}_{-0.026}$  & $-0.008^{+0.020}_{-0.021}$ & $-0.044^{+0.006}_{-0.006}$ & $-0.044^{+0.006}_{-0.006}$\\
PTF11bnx & $150.7^{+3.0}_{-2.9}$ & $ 14.6^{+1.0}_{-1.0}$ & $ 6143^{+  5}_{-  5}$ & $  0.07^{+  0.14}_{-  0.14}$ & $0.043^{+0.029}_{-0.029}$  & $0.049^{+0.011}_{-0.011}$ & $-0.030^{+0.014}_{-0.011}$ & $-0.030^{+0.014}_{-0.011}$\\
PTF12jqh & $151.8^{+1.4}_{-1.5}$ & $  7.9^{+0.6}_{-0.7}$ & $ 6118^{+ 10}_{- 10}$ & $  0.73^{+  0.14}_{-  0.14}$ & $0.069^{+0.028}_{-0.028}$  & $-0.056^{+0.010}_{-0.010}$ & $0.009^{+0.014}_{-0.015}$ & $0.009^{+0.014}_{-0.015}$\\
SNF20080802-006 & $108.7^{+5.6}_{-5.8}$ & $ 19.1^{+1.6}_{-1.6}$ & $ 6122^{+  5}_{-  5}$ & $ -1.61^{+  0.12}_{-  0.12}$ & $-0.042^{+0.059}_{-0.055}$  & $-0.001^{+0.016}_{-0.016}$ & $0.033^{+0.020}_{-0.020}$ & $0.033^{+0.020}_{-0.020}$\\
PTF10xyt & $120.2^{+6.3}_{-6.4}$ & $ 21.5^{+3.2}_{-3.1}$ & $ 6102^{+  4}_{-  4}$ & $  0.18^{+  0.20}_{-  0.20}$ & $0.241^{+0.049}_{-0.049}$  & $0.070^{+0.026}_{-0.028}$ & $0.014^{+0.029}_{-0.027}$ & $0.014^{+0.029}_{-0.027}$\\
PTF11qmo & $101.7^{+1.1}_{-1.1}$ & $  7.8^{+0.7}_{-0.7}$ & $ 6151^{+  7}_{-  8}$ & $  0.41^{+  0.18}_{-  0.18}$ & $-0.078^{+0.031}_{-0.028}$  & $-0.058^{+0.010}_{-0.010}$ & $0.004^{+0.014}_{-0.015}$ & $0.004^{+0.014}_{-0.015}$\\
SNF20070331-025 & $118.7^{+6.9}_{-7.0}$ & $ 15.4^{+2.0}_{-2.0}$ & $ 6121^{+ 10}_{- 10}$ & $ -1.22^{+  0.15}_{-  0.15}$ & $0.190^{+0.037}_{-0.037}$  & $-0.015^{+0.020}_{-0.020}$ & $0.001^{+0.023}_{-0.022}$ & $0.001^{+0.023}_{-0.022}$\\
SNF20070818-001 & $160.6^{+6.7}_{-6.8}$ & $ 16.4^{+1.5}_{-1.5}$ & $ 6113^{+  5}_{-  5}$ & $  0.14^{+  0.14}_{-  0.14}$ & $-0.178^{+0.021}_{-0.020}$  & $0.005^{+0.014}_{-0.014}$ & $0.014^{+0.016}_{-0.015}$ & $0.014^{+0.016}_{-0.015}$\\
SNBOSS38 & $ 57.1^{+0.4}_{-0.4}$ & $ 17.8^{+0.3}_{-0.3}$ & $ 6127^{+  3}_{-  3}$ & $  1.02^{+  0.11}_{-  0.11}$ & $0.018^{+0.038}_{-0.039}$  & $0.017^{+0.021}_{-0.023}$ & $-0.002^{+0.019}_{-0.020}$ & $-0.002^{+0.019}_{-0.020}$\\
SN2006ob & $ 75.9^{+12.1}_{-12.4}$ & $ 26.8^{+1.3}_{-1.4}$ & $ 6113^{+  5}_{-  5}$ & $ -2.26^{+  0.21}_{-  0.21}$ & $0.039^{+0.035}_{-0.035}$  & $-0.051^{+0.019}_{-0.017}$ & $0.084^{+0.021}_{-0.021}$ & $0.084^{+0.021}_{-0.021}$\\
PTF12eer & $169.4^{+8.9}_{-8.8}$ & $ 12.2^{+1.8}_{-1.8}$ & $ 6148^{+ 10}_{- 10}$ & $  0.64^{+  0.14}_{-  0.15}$ & $-0.053^{+0.028}_{-0.027}$  & $0.039^{+0.018}_{-0.019}$ & $-0.029^{+0.017}_{-0.012}$ & $-0.029^{+0.017}_{-0.012}$\\
PTF10ops & $ 38.0^{+9.2}_{-9.2}$ & $ 21.9^{+4.6}_{-4.6}$ & $ 6140^{+  6}_{-  5}$ & $ -0.58^{+  0.12}_{-  0.12}$ & $-0.091^{+0.221}_{-0.154}$  & $0.190^{+0.040}_{-0.041}$ & $0.181^{+0.038}_{-0.036}$ & $0.181^{+0.038}_{-0.036}$\\
SNF20080514-002 & $ 83.1^{+0.7}_{-0.7}$ & $ 19.2^{+0.5}_{-0.5}$ & $ 6131^{+  3}_{-  3}$ & $ -1.55^{+  0.09}_{-  0.09}$ & $0.174^{+0.019}_{-0.019}$  & $-0.074^{+0.007}_{-0.006}$ & $0.006^{+0.011}_{-0.012}$ & $0.006^{+0.011}_{-0.012}$\\
PTF12evo & $129.1^{+2.6}_{-2.7}$ & $  9.9^{+1.1}_{-1.1}$ & $ 6157^{+  4}_{-  4}$ & $  0.81^{+  0.13}_{-  0.12}$ & $0.024^{+0.027}_{-0.027}$  & $0.004^{+0.011}_{-0.011}$ & $-0.034^{+0.012}_{-0.010}$ & $-0.034^{+0.012}_{-0.010}$\\
SNF20080614-010 & $118.9^{+4.9}_{-4.8}$ & $ 28.9^{+1.2}_{-1.2}$ & $ 6129^{+  3}_{-  3}$ & $ -1.81^{+  0.15}_{-  0.15}$ & $-0.114^{+0.046}_{-0.042}$  & $-0.068^{+0.012}_{-0.009}$ & $0.049^{+0.019}_{-0.019}$ & $0.049^{+0.019}_{-0.019}$\\
PTF10icb & $104.8^{+0.9}_{-0.9}$ & $ 12.6^{+0.3}_{-0.3}$ & $ 6138^{+  2}_{-  2}$ & $  0.23^{+  0.10}_{-  0.10}$ & $0.410^{+0.024}_{-0.025}$  & $0.046^{+0.006}_{-0.006}$ & $-0.019^{+0.010}_{-0.010}$ & $-0.019^{+0.010}_{-0.010}$\\
PTF12efn & $145.5^{+3.3}_{-3.2}$ & $  5.4^{+1.4}_{-1.5}$ & $ 6115^{+  3}_{-  3}$ & $  1.04^{+  0.16}_{-  0.16}$ & $0.177^{+0.030}_{-0.029}$  & $-0.058^{+0.014}_{-0.012}$ & $0.051^{+0.017}_{-0.017}$ & $0.051^{+0.017}_{-0.017}$\\
SNNGC4424 & $109.0^{+0.3}_{-0.3}$ & $  8.6^{+0.1}_{-0.1}$ & $ 6138^{+  2}_{-  2}$ & $  0.44^{+  0.09}_{-  0.09}$ & $0.600^{+0.022}_{-0.023}$  & $0.078^{+0.007}_{-0.006}$ & $-0.025^{+0.011}_{-0.011}$ & $-0.025^{+0.011}_{-0.011}$\\
SNF20080516-022 & $ 99.1^{+2.2}_{-2.1}$ & $ 15.3^{+0.9}_{-0.9}$ & $ 6159^{+  3}_{-  3}$ & $  0.48^{+  0.12}_{-  0.11}$ & $0.001^{+0.026}_{-0.026}$  & $-0.064^{+0.010}_{-0.009}$ & $-0.025^{+0.014}_{-0.013}$ & $-0.025^{+0.014}_{-0.013}$\\
PTF12hwb & $139.4^{+25.9}_{-23.8}$ & $ -0.9^{+4.4}_{-4.3}$ & $ 6090^{+ 14}_{- 14}$ & $  1.34^{+  0.17}_{-  0.17}$ & $0.471^{+0.042}_{-0.044}$  & $0.151^{+0.037}_{-0.043}$ & $0.004^{+0.040}_{-0.032}$ & $0.004^{+0.040}_{-0.032}$\\
PTF10qyz & $108.1^{+2.1}_{-2.1}$ & $ 21.0^{+0.8}_{-0.8}$ & $ 6117^{+  5}_{-  5}$ & $ -1.67^{+  0.13}_{-  0.13}$ & $0.099^{+0.133}_{-0.095}$  & $-0.067^{+0.014}_{-0.010}$ & $-0.038^{+0.012}_{-0.008}$ & $-0.038^{+0.012}_{-0.008}$\\
SNF20060907-000 & $ 87.8^{+7.5}_{-7.3}$ & $ 15.9^{+0.9}_{-0.9}$ & $ 6150^{+  4}_{-  4}$ & $  0.16^{+  0.17}_{-  0.17}$ & $0.037^{+0.033}_{-0.033}$  & $-0.079^{+0.005}_{-0.005}$ & $-0.041^{+0.008}_{-0.007}$ & $-0.041^{+0.008}_{-0.007}$\\
LSQ12fxd & $122.8^{+1.6}_{-1.7}$ & $ 11.7^{+0.8}_{-0.7}$ & $ 6119^{+  4}_{-  4}$ & $  0.16^{+  0.12}_{-  0.12}$ & $-0.154^{+0.044}_{-0.032}$  & $-0.052^{+0.008}_{-0.008}$ & $0.002^{+0.013}_{-0.013}$ & $0.002^{+0.013}_{-0.013}$\\
SNF20080821-000 & $104.8^{+2.2}_{-2.2}$ & $  8.8^{+1.2}_{-1.2}$ & $ 6121^{+  4}_{-  4}$ & $  0.77^{+  0.15}_{-  0.14}$ & $-0.104^{+0.023}_{-0.023}$  & $0.025^{+0.011}_{-0.011}$ & $-0.006^{+0.013}_{-0.013}$ & $-0.006^{+0.013}_{-0.013}$\\
SNF20070802-000 & $157.9^{+3.3}_{-3.3}$ & $ 16.0^{+1.4}_{-1.4}$ & $ 6103^{+  5}_{-  5}$ & $  0.07^{+  0.13}_{-  0.13}$ & $-0.074^{+0.028}_{-0.028}$  & $0.005^{+0.014}_{-0.014}$ & $0.005^{+0.016}_{-0.016}$ & $0.005^{+0.016}_{-0.016}$\\
PTF10wnm & $105.5^{+2.3}_{-2.3}$ & $  6.8^{+0.9}_{-0.9}$ & $ 6125^{+  3}_{-  3}$ & $  0.73^{+  0.12}_{-  0.12}$ & $0.123^{+0.022}_{-0.021}$  & $-0.014^{+0.009}_{-0.009}$ & $0.015^{+0.012}_{-0.012}$ & $0.015^{+0.012}_{-0.012}$\\
PTF10mwb & $116.2^{+1.2}_{-1.2}$ & $ 21.1^{+0.7}_{-0.7}$ & $ 6139^{+  2}_{-  2}$ & $ -0.48^{+  0.09}_{-  0.09}$ & $-0.000^{+0.071}_{-0.070}$  & $-0.036^{+0.008}_{-0.008}$ & $-0.011^{+0.011}_{-0.011}$ & $-0.011^{+0.011}_{-0.011}$\\
SN2010dt & $133.3^{+8.4}_{-8.6}$ & $ 16.0^{+0.6}_{-0.6}$ & $ 6141^{+  5}_{-  5}$ & $ -0.02^{+  0.09}_{-  0.09}$ & $-0.046^{+0.024}_{-0.024}$  & $-0.063^{+0.010}_{-0.009}$ & $-0.043^{+0.007}_{-0.007}$ & $-0.043^{+0.007}_{-0.007}$\\
SNF20080623-001 & $149.2^{+1.4}_{-1.4}$ & $ 14.8^{+0.6}_{-0.6}$ & $ 6131^{+  3}_{-  3}$ & $ -0.07^{+  0.11}_{-  0.10}$ & $0.072^{+0.025}_{-0.025}$  & $-0.054^{+0.010}_{-0.009}$ & $-0.034^{+0.011}_{-0.010}$ & $-0.034^{+0.011}_{-0.010}$\\
LSQ12fhe & $ 42.9^{+1.2}_{-1.2}$ & $  2.3^{+1.9}_{-1.9}$ & $ 6108^{+  4}_{-  4}$ & $  1.01^{+  0.17}_{-  0.17}$ & $-0.183^{+0.032}_{-0.033}$  & $0.067^{+0.017}_{-0.018}$ & $-0.017^{+0.018}_{-0.016}$ & $-0.017^{+0.018}_{-0.016}$\\
PTF11bju & $ 31.3^{+4.3}_{-4.2}$ & $  3.3^{+2.0}_{-2.0}$ & $ 6138^{+  5}_{-  5}$ & $  1.02^{+  0.17}_{-  0.18}$ & $0.073^{+0.036}_{-0.036}$  & $0.077^{+0.019}_{-0.019}$ & $-0.014^{+0.020}_{-0.018}$ & $-0.014^{+0.020}_{-0.018}$\\
PTF09fox & $118.5^{+2.7}_{-2.7}$ & $  9.8^{+0.9}_{-0.8}$ & $ 6117^{+  3}_{-  3}$ & $  0.52^{+  0.16}_{-  0.15}$ & $0.027^{+0.039}_{-0.039}$  & $-0.047^{+0.009}_{-0.009}$ & $-0.043^{+0.007}_{-0.006}$ & $-0.043^{+0.007}_{-0.006}$\\
PTF13ayw & $103.8^{+2.4}_{-2.4}$ & $ 23.1^{+1.8}_{-1.9}$ & $ 6116^{+  6}_{-  6}$ & $ -1.67^{+  0.23}_{-  0.24}$ & $0.068^{+0.041}_{-0.041}$  & $-0.069^{+0.014}_{-0.009}$ & $0.101^{+0.019}_{-0.020}$ & $0.101^{+0.019}_{-0.020}$\\
SNF20070810-004 & $127.0^{+1.8}_{-1.8}$ & $ 20.1^{+0.9}_{-0.9}$ & $ 6117^{+  7}_{-  7}$ & $ -0.01^{+  0.11}_{-  0.11}$ & $-0.044^{+0.026}_{-0.025}$  & $-0.068^{+0.011}_{-0.009}$ & $-0.033^{+0.012}_{-0.010}$ & $-0.033^{+0.012}_{-0.010}$\\
PTF11mty & $110.5^{+2.2}_{-2.2}$ & $ 11.3^{+1.0}_{-1.1}$ & $ 6140^{+  5}_{-  5}$ & $  1.03^{+  0.13}_{-  0.13}$ & $0.062^{+0.025}_{-0.024}$  & $-0.072^{+0.009}_{-0.007}$ & $0.024^{+0.013}_{-0.014}$ & $0.024^{+0.013}_{-0.014}$\\
SNF20080512-010 & $ 96.0^{+3.4}_{-3.4}$ & $ 21.9^{+1.3}_{-1.3}$ & $ 6128^{+  5}_{-  5}$ & $ -1.64^{+  0.18}_{-  0.18}$ & $-0.032^{+0.072}_{-0.061}$  & $-0.051^{+0.013}_{-0.012}$ & $0.018^{+0.016}_{-0.016}$ & $0.018^{+0.016}_{-0.016}$\\
PTF11mkx & $ 34.8^{+3.5}_{-3.5}$ & $  5.5^{+1.3}_{-1.3}$ & $ 6169^{+  5}_{-  5}$ & $  0.53^{+  0.14}_{-  0.14}$ & $0.022^{+0.135}_{-0.080}$  & $0.102^{+0.017}_{-0.016}$ & $-0.041^{+0.008}_{-0.007}$ & $-0.041^{+0.008}_{-0.007}$\\
PTF10tce & $135.8^{+1.1}_{-1.1}$ & $ 10.6^{+1.2}_{-1.2}$ & $ 6090^{+  4}_{-  4}$ & $  0.38^{+  0.12}_{-  0.12}$ & $0.041^{+0.028}_{-0.027}$  & $-0.005^{+0.013}_{-0.014}$ & $-0.019^{+0.015}_{-0.015}$ & $-0.019^{+0.015}_{-0.015}$\\
SNF20061020-000 & $ 90.3^{+13.4}_{-13.3}$ & $ 23.9^{+0.9}_{-0.9}$ & $ 6120^{+  4}_{-  5}$ & $ -1.75^{+  0.17}_{-  0.17}$ & $-0.096^{+0.039}_{-0.034}$  & $0.000^{+0.016}_{-0.016}$ & $0.006^{+0.022}_{-0.020}$ & $0.006^{+0.022}_{-0.020}$\\
SN2005ir & $115.7^{+2.8}_{-2.7}$ & $  8.7^{+4.6}_{-4.8}$ & $ 6069^{+  5}_{-  5}$ & $  1.64^{+  0.49}_{-  0.48}$ & $0.112^{+0.042}_{-0.043}$  & $-0.022^{+0.036}_{-0.033}$ & $0.067^{+0.037}_{-0.037}$ & $0.067^{+0.037}_{-0.037}$\\
SNF20080717-000 & $ 92.7^{+2.6}_{-2.6}$ & $ 12.4^{+1.7}_{-1.7}$ & $ 6105^{+  3}_{-  3}$ & $  0.87^{+  0.13}_{-  0.12}$ & $0.115^{+0.034}_{-0.033}$  & $0.126^{+0.018}_{-0.019}$ & $0.050^{+0.021}_{-0.019}$ & $0.050^{+0.021}_{-0.019}$\\
PTF12ena & $101.0^{+1.6}_{-1.6}$ & $  8.2^{+0.9}_{-0.9}$ & $ 6130^{+  3}_{-  3}$ & $  0.76^{+  0.13}_{-  0.13}$ & $-0.039^{+0.028}_{-0.028}$  & $0.289^{+0.011}_{-0.012}$ & $-0.038^{+0.010}_{-0.008}$ & $-0.038^{+0.010}_{-0.008}$\\
PTF13anh & $167.2^{+1.8}_{-1.8}$ & $ 21.6^{+1.0}_{-1.0}$ & $ 6174^{+  3}_{-  4}$ & $ -0.38^{+  0.14}_{-  0.14}$ & $0.092^{+0.108}_{-0.067}$  & $-0.032^{+0.020}_{-0.019}$ & $-0.034^{+0.014}_{-0.010}$ & $-0.034^{+0.014}_{-0.010}$\\
CSS110918\_01 & $110.4^{+1.0}_{-1.0}$ & $  5.0^{+0.9}_{-0.9}$ & $ 6101^{+  2}_{-  2}$ & $  0.82^{+  0.13}_{-  0.13}$ & $-0.077^{+0.187}_{-0.094}$  & $-0.078^{+0.006}_{-0.005}$ & $-0.028^{+0.013}_{-0.012}$ & $-0.028^{+0.013}_{-0.012}$\\
SNF20061024-000 & $ 98.9^{+15.7}_{-16.7}$ & $ 29.1^{+1.4}_{-1.4}$ & $ 6127^{+  5}_{-  5}$ & $ -1.51^{+  0.18}_{-  0.18}$ & $0.045^{+0.062}_{-0.059}$  & $-0.049^{+0.021}_{-0.019}$ & $0.003^{+0.031}_{-0.028}$ & $0.003^{+0.031}_{-0.028}$\\
SNF20070506-006 & $ 94.3^{+1.3}_{-1.2}$ & $  6.4^{+0.6}_{-0.5}$ & $ 6152^{+  3}_{-  3}$ & $  1.02^{+  0.11}_{-  0.11}$ & $-0.075^{+0.027}_{-0.026}$  & $-0.029^{+0.009}_{-0.009}$ & $0.006^{+0.013}_{-0.013}$ & $0.006^{+0.013}_{-0.013}$\\
SNF20070403-001 & $107.8^{+5.0}_{-5.1}$ & $ 17.1^{+1.4}_{-1.4}$ & $ 6123^{+  4}_{-  4}$ & $ -0.60^{+  0.12}_{-  0.12}$ & $0.007^{+0.029}_{-0.028}$  & $-0.036^{+0.014}_{-0.014}$ & $0.020^{+0.017}_{-0.017}$ & $0.020^{+0.017}_{-0.017}$\\
PTF10hmv & $109.8^{+1.3}_{-1.4}$ & $  8.5^{+0.7}_{-0.6}$ & $ 6143^{+  3}_{-  3}$ & $  1.15^{+  0.13}_{-  0.12}$ & $0.284^{+0.024}_{-0.023}$  & $0.106^{+0.009}_{-0.009}$ & $-0.010^{+0.012}_{-0.012}$ & $-0.010^{+0.012}_{-0.012}$\\
SNF20071015-000 & $104.8^{+3.1}_{-3.1}$ & $  6.5^{+1.1}_{-1.1}$ & $ 6125^{+  7}_{-  7}$ & $  0.90^{+  0.28}_{-  0.29}$ & $0.116^{+0.040}_{-0.040}$  & $0.293^{+0.016}_{-0.017}$ & $-0.018^{+0.021}_{-0.017}$ & $-0.018^{+0.021}_{-0.017}$\\
SNhunt89 & $ 88.6^{+2.7}_{-2.7}$ & $ 29.8^{+1.7}_{-1.7}$ & $ 6109^{+  7}_{-  7}$ & $ -2.02^{+  0.16}_{-  0.16}$ & $0.202^{+0.049}_{-0.047}$  & $0.030^{+0.018}_{-0.019}$ & $0.032^{+0.027}_{-0.026}$ & $0.032^{+0.027}_{-0.026}$\\
SNF20070902-021 & $109.0^{+3.3}_{-3.3}$ & $ 16.8^{+0.9}_{-0.9}$ & $ 6131^{+  5}_{-  5}$ & $ -0.46^{+  0.09}_{-  0.09}$ & $-0.082^{+0.027}_{-0.026}$  & $-0.023^{+0.010}_{-0.010}$ & $-0.028^{+0.014}_{-0.012}$ & $-0.028^{+0.014}_{-0.012}$\\
PTF09dlc & $143.8^{+2.1}_{-2.1}$ & $  9.7^{+0.8}_{-0.8}$ & $ 6143^{+  3}_{-  3}$ & $  0.01^{+  0.10}_{-  0.10}$ & $0.008^{+0.027}_{-0.026}$  & $-0.054^{+0.012}_{-0.012}$ & $-0.017^{+0.014}_{-0.015}$ & $-0.017^{+0.014}_{-0.015}$\\
PTF13ajv & $149.9^{+8.1}_{-8.2}$ & $ 22.0^{+2.7}_{-3.0}$ & $ 6107^{+ 20}_{- 20}$ & $ -1.64^{+  0.20}_{-  0.21}$ & $0.263^{+0.046}_{-0.045}$  & $-0.062^{+0.021}_{-0.014}$ & $-0.005^{+0.023}_{-0.022}$ & $-0.005^{+0.023}_{-0.022}$\\
SNF20080919-000 & $115.6^{+2.7}_{-2.7}$ & $  8.8^{+0.9}_{-0.8}$ & $ 6144^{+  5}_{-  5}$ & $ -0.24^{+  0.10}_{-  0.10}$ & $-0.043^{+0.027}_{-0.026}$  & $0.165^{+0.013}_{-0.012}$ & $-0.035^{+0.013}_{-0.010}$ & $-0.035^{+0.013}_{-0.010}$\\
SNF20080919-001 & $ 85.0^{+1.1}_{-1.1}$ & $  6.1^{+0.4}_{-0.4}$ & $ 6149^{+  4}_{-  4}$ & $  1.05^{+  0.12}_{-  0.12}$ & $-0.183^{+0.020}_{-0.019}$  & $0.001^{+0.008}_{-0.008}$ & $0.006^{+0.012}_{-0.012}$ & $0.006^{+0.012}_{-0.012}$\\
SN2010kg & $ 99.4^{+17.2}_{-17.5}$ & $ 21.8^{+0.7}_{-0.7}$ & $ 6077^{+  5}_{-  5}$ & $ -1.87^{+  0.24}_{-  0.25}$ & $-0.073^{+0.030}_{-0.029}$  & $0.035^{+0.021}_{-0.020}$ & $0.034^{+0.020}_{-0.021}$ & $0.034^{+0.020}_{-0.021}$\\
SNF20080714-008 & $155.6^{+11.0}_{-10.8}$ & $ 16.6^{+1.9}_{-1.8}$ & $ 6099^{+  6}_{-  6}$ & $ -0.07^{+  0.14}_{-  0.14}$ & $-0.141^{+0.024}_{-0.022}$  & $0.082^{+0.018}_{-0.020}$ & $-0.029^{+0.017}_{-0.013}$ & $-0.029^{+0.017}_{-0.013}$\\
SNF20070714-007 & $132.6^{+5.5}_{-5.5}$ & $ 33.5^{+4.3}_{-4.2}$ & $ 6145^{+  4}_{-  4}$ & $ -1.09^{+  0.13}_{-  0.13}$ & $-0.094^{+0.046}_{-0.047}$  & $0.118^{+0.035}_{-0.038}$ & $0.117^{+0.036}_{-0.032}$ & $0.117^{+0.036}_{-0.032}$\\
SNF20080522-011 & $122.1^{+1.7}_{-1.7}$ & $  8.3^{+0.5}_{-0.5}$ & $ 6125^{+  2}_{-  2}$ & $  0.82^{+  0.13}_{-  0.13}$ & $-0.124^{+0.023}_{-0.023}$  & $-0.045^{+0.008}_{-0.008}$ & $-0.031^{+0.011}_{-0.010}$ & $-0.031^{+0.011}_{-0.010}$\\
SNF20061111-002 & $117.1^{+8.6}_{-8.8}$ & $ 19.8^{+0.9}_{-0.9}$ & $ 6144^{+  6}_{-  6}$ & $  0.07^{+  0.17}_{-  0.17}$ & $0.056^{+0.032}_{-0.032}$  & $-0.062^{+0.013}_{-0.011}$ & $0.026^{+0.017}_{-0.017}$ & $0.026^{+0.017}_{-0.017}$\\
SNNGC6343 & $ 86.9^{+1.5}_{-1.4}$ & $ 21.2^{+0.7}_{-0.7}$ & $ 6136^{+  3}_{-  3}$ & $ -1.49^{+  0.16}_{-  0.16}$ & $0.055^{+0.029}_{-0.030}$  & $0.036^{+0.013}_{-0.012}$ & $0.020^{+0.016}_{-0.014}$ & $0.020^{+0.016}_{-0.014}$\\
SNF20061011-005 & $120.5^{+1.1}_{-1.1}$ & $  9.2^{+0.3}_{-0.3}$ & $ 6132^{+  4}_{-  4}$ & $  0.66^{+  0.22}_{-  0.21}$ & $-0.091^{+0.043}_{-0.041}$  & $-0.073^{+0.008}_{-0.007}$ & $0.003^{+0.021}_{-0.021}$ & $0.003^{+0.021}_{-0.021}$\\
SNF20080825-010 & $ 96.4^{+10.2}_{-10.2}$ & $ 19.2^{+0.6}_{-0.6}$ & $ 6116^{+  4}_{-  4}$ & $ -0.76^{+  0.13}_{-  0.13}$ & $-0.192^{+0.020}_{-0.020}$  & $-0.045^{+0.013}_{-0.013}$ & $0.002^{+0.013}_{-0.013}$ & $0.002^{+0.013}_{-0.013}$\\
PTF10ufj & $141.4^{+3.3}_{-3.3}$ & $ 12.0^{+1.1}_{-1.0}$ & $ 6131^{+  6}_{-  5}$ & $  0.56^{+  0.13}_{-  0.13}$ & $-0.043^{+0.106}_{-0.068}$  & $-0.040^{+0.011}_{-0.011}$ & $-0.030^{+0.014}_{-0.011}$ & $-0.030^{+0.014}_{-0.011}$\\
PTF10wof & $130.5^{+2.6}_{-2.7}$ & $ 16.5^{+0.9}_{-0.9}$ & $ 6102^{+  2}_{-  2}$ & $  0.06^{+  0.11}_{-  0.11}$ & $0.096^{+0.023}_{-0.023}$  & $-0.009^{+0.011}_{-0.011}$ & $-0.013^{+0.013}_{-0.012}$ & $-0.013^{+0.013}_{-0.012}$\\
SNF20080918-000 & $146.6^{+3.5}_{-3.5}$ & $  5.6^{+2.1}_{-2.0}$ & $ 6110^{+  5}_{-  5}$ & $  1.28^{+  0.23}_{-  0.23}$ & $0.112^{+0.038}_{-0.037}$  & $0.073^{+0.019}_{-0.019}$ & $-0.004^{+0.023}_{-0.022}$ & $-0.004^{+0.023}_{-0.022}$\\
SNF20080516-000 & $117.6^{+2.1}_{-2.1}$ & $  8.2^{+1.2}_{-1.1}$ & $ 6135^{+  3}_{-  3}$ & $  0.96^{+  0.18}_{-  0.18}$ & $0.084^{+0.029}_{-0.028}$  & $-0.054^{+0.011}_{-0.010}$ & $0.004^{+0.015}_{-0.015}$ & $0.004^{+0.015}_{-0.015}$\\
SN2005cf & $158.9^{+0.7}_{-0.7}$ & $ 15.2^{+0.8}_{-0.7}$ & $ 6140^{+  3}_{-  3}$ & $ -0.19^{+  0.13}_{-  0.13}$ & $-0.122^{+0.028}_{-0.027}$  & $-0.077^{+0.007}_{-0.005}$ & $0.002^{+0.014}_{-0.015}$ & $0.002^{+0.014}_{-0.015}$\\
CSS130502\_01 & $ 87.3^{+8.0}_{-8.7}$ & $ 15.4^{+0.5}_{-0.5}$ & $ 6128^{+  3}_{-  3}$ & $ -0.15^{+  0.20}_{-  0.20}$ & $0.121^{+0.032}_{-0.032}$  & $-0.069^{+0.011}_{-0.009}$ & $-0.017^{+0.015}_{-0.015}$ & $-0.017^{+0.015}_{-0.015}$\\
SNF20080620-000 & $125.3^{+9.5}_{-9.7}$ & $ 20.6^{+0.6}_{-0.6}$ & $ 6133^{+  3}_{-  3}$ & $ -0.97^{+  0.14}_{-  0.14}$ & $-0.007^{+0.029}_{-0.028}$  & $0.001^{+0.012}_{-0.012}$ & $-0.043^{+0.007}_{-0.007}$ & $-0.043^{+0.007}_{-0.007}$\\
SNPGC51271 & $ 64.0^{+12.0}_{-13.2}$ & $ 21.4^{+0.7}_{-0.7}$ & $ 6121^{+  2}_{-  2}$ & $ -1.85^{+  0.16}_{-  0.17}$ & $0.125^{+0.034}_{-0.034}$  & $-0.057^{+0.017}_{-0.014}$ & $0.011^{+0.018}_{-0.019}$ & $0.011^{+0.018}_{-0.019}$\\
PTF11pdk & $127.7^{+2.6}_{-2.7}$ & $ 16.1^{+1.2}_{-1.2}$ & $ 6154^{+  5}_{-  5}$ & $  0.16^{+  0.16}_{-  0.17}$ & $-0.027^{+0.035}_{-0.035}$  & $-0.070^{+0.010}_{-0.008}$ & $-0.033^{+0.015}_{-0.011}$ & $-0.033^{+0.015}_{-0.011}$\\
SNF20060511-014 & $102.9^{+2.8}_{-2.8}$ & $ 17.1^{+1.0}_{-1.0}$ & $ 6144^{+  8}_{-  7}$ & $ -0.50^{+  0.15}_{-  0.15}$ & $-0.042^{+0.047}_{-0.037}$  & $-0.008^{+0.011}_{-0.010}$ & $-0.041^{+0.009}_{-0.007}$ & $-0.041^{+0.009}_{-0.007}$\\
SNF20080612-003 & $120.0^{+1.1}_{-1.1}$ & $  7.5^{+0.6}_{-0.6}$ & $ 6122^{+  3}_{-  3}$ & $  0.96^{+  0.15}_{-  0.15}$ & $-0.221^{+0.018}_{-0.017}$  & $-0.015^{+0.007}_{-0.007}$ & $-0.001^{+0.013}_{-0.014}$ & $-0.001^{+0.013}_{-0.014}$\\
SNF20080626-002 & $130.0^{+1.0}_{-1.0}$ & $  5.8^{+2.1}_{-2.0}$ & $ 6111^{+  3}_{-  3}$ & $  1.03^{+  0.19}_{-  0.19}$ & $-0.043^{+0.026}_{-0.025}$  & $-0.009^{+0.015}_{-0.016}$ & $-0.022^{+0.017}_{-0.015}$ & $-0.022^{+0.017}_{-0.015}$\\
SNF20060621-015 & $111.8^{+1.3}_{-1.3}$ & $  9.9^{+0.7}_{-0.7}$ & $ 6144^{+  3}_{-  3}$ & $  0.52^{+  0.14}_{-  0.14}$ & $-0.015^{+0.026}_{-0.027}$  & $-0.067^{+0.008}_{-0.008}$ & $-0.020^{+0.013}_{-0.013}$ & $-0.020^{+0.013}_{-0.013}$\\
SNF20080920-000 & $135.1^{+1.4}_{-1.4}$ & $  5.3^{+1.5}_{-1.5}$ & $ 6085^{+  3}_{-  3}$ & $  1.08^{+  0.23}_{-  0.23}$ & $-0.002^{+0.038}_{-0.037}$  & $-0.020^{+0.017}_{-0.017}$ & $-0.012^{+0.020}_{-0.019}$ & $-0.012^{+0.020}_{-0.019}$\\
SN2007cq & $ 63.7^{+3.9}_{-3.9}$ & $ 11.6^{+0.8}_{-0.8}$ & $ 6138^{+  3}_{-  3}$ & $  0.05^{+  0.13}_{-  0.13}$ & $-0.111^{+0.029}_{-0.027}$  & $0.008^{+0.011}_{-0.011}$ & $-0.015^{+0.015}_{-0.015}$ & $-0.015^{+0.015}_{-0.015}$\\
SNF20080918-004 & $ 91.9^{+6.6}_{-6.6}$ & $ 21.2^{+0.9}_{-0.9}$ & $ 6141^{+  4}_{-  4}$ & $ -1.37^{+  0.15}_{-  0.15}$ & $0.035^{+0.142}_{-0.087}$  & $-0.021^{+0.014}_{-0.013}$ & $-0.012^{+0.019}_{-0.018}$ & $-0.012^{+0.019}_{-0.018}$\\
CSS120424\_01 & $138.5^{+2.1}_{-2.1}$ & $ 11.5^{+0.6}_{-0.6}$ & $ 6138^{+  3}_{-  3}$ & $  0.58^{+  0.14}_{-  0.14}$ & $-0.010^{+0.028}_{-0.028}$  & $-0.061^{+0.009}_{-0.009}$ & $-0.024^{+0.013}_{-0.012}$ & $-0.024^{+0.013}_{-0.012}$\\
SNF20080610-000 & $119.5^{+8.6}_{-8.8}$ & $ 17.0^{+1.4}_{-1.4}$ & $ 6131^{+  6}_{-  6}$ & $  0.16^{+  0.14}_{-  0.14}$ & $0.013^{+0.068}_{-0.068}$  & $-0.019^{+0.014}_{-0.015}$ & $-0.027^{+0.017}_{-0.013}$ & $-0.027^{+0.017}_{-0.013}$\\
SNF20070701-005 & $101.6^{+2.6}_{-2.6}$ & $ 13.6^{+0.9}_{-1.0}$ & $ 6155^{+  5}_{-  5}$ & $  0.39^{+  0.14}_{-  0.14}$ & $-0.187^{+0.021}_{-0.021}$  & $-0.012^{+0.011}_{-0.011}$ & $0.037^{+0.015}_{-0.015}$ & $0.037^{+0.015}_{-0.015}$\\
SN2007kk & $128.6^{+1.4}_{-1.4}$ & $ 11.0^{+0.9}_{-0.9}$ & $ 6099^{+  4}_{-  4}$ & $  0.59^{+  0.14}_{-  0.14}$ & $-0.087^{+0.029}_{-0.029}$  & $-0.048^{+0.012}_{-0.012}$ & $-0.035^{+0.012}_{-0.010}$ & $-0.035^{+0.012}_{-0.010}$\\
SNF20060908-004 & $114.4^{+1.2}_{-1.2}$ & $ 12.4^{+0.6}_{-0.6}$ & $ 6136^{+  3}_{-  3}$ & $  0.63^{+  0.15}_{-  0.15}$ & $-0.099^{+0.138}_{-0.081}$  & $-0.068^{+0.009}_{-0.008}$ & $0.021^{+0.017}_{-0.017}$ & $0.021^{+0.017}_{-0.017}$\\
SNF20080909-030 & $ 93.8^{+1.0}_{-0.9}$ & $  7.7^{+0.4}_{-0.4}$ & $ 6171^{+  3}_{-  3}$ & $  0.99^{+  0.13}_{-  0.13}$ & $0.030^{+0.030}_{-0.030}$  & $0.008^{+0.013}_{-0.012}$ & $0.008^{+0.015}_{-0.014}$ & $0.008^{+0.015}_{-0.014}$\\
PTF11bgv & $ 78.3^{+3.1}_{-3.1}$ & $ 12.7^{+0.6}_{-0.6}$ & $ 6146^{+  3}_{-  3}$ & $ -0.27^{+  0.11}_{-  0.10}$ & $0.057^{+0.028}_{-0.028}$  & $0.081^{+0.010}_{-0.010}$ & $0.006^{+0.015}_{-0.015}$ & $0.006^{+0.015}_{-0.015}$\\
SNNGC2691 & $ 55.7^{+14.4}_{-14.5}$ & $  4.5^{+0.2}_{-0.2}$ & $ 6138^{+  8}_{-  8}$ & $  0.82^{+  0.16}_{-  0.16}$ & $-0.031^{+0.039}_{-0.038}$  & $0.069^{+0.018}_{-0.019}$ & $0.041^{+0.020}_{-0.020}$ & $0.041^{+0.020}_{-0.020}$\\
PTF13asv & $ 75.4^{+1.1}_{-1.1}$ & $  2.4^{+0.4}_{-0.4}$ & $ 6149^{+  4}_{-  4}$ & $  0.66^{+  0.10}_{-  0.10}$ & $-0.018^{+0.184}_{-0.105}$  & $-0.068^{+0.012}_{-0.009}$ & $0.015^{+0.015}_{-0.015}$ & $0.015^{+0.015}_{-0.015}$\\
SNF20070806-026 & $102.2^{+8.4}_{-9.0}$ & $ 25.3^{+0.7}_{-0.7}$ & $ 6112^{+  7}_{-  7}$ & $ -1.93^{+  0.14}_{-  0.14}$ & $0.010^{+0.032}_{-0.033}$  & $-0.071^{+0.011}_{-0.008}$ & $0.015^{+0.017}_{-0.017}$ & $0.015^{+0.017}_{-0.017}$\\
SNF20070427-001 & $ 80.7^{+2.3}_{-2.3}$ & $  6.5^{+0.8}_{-0.8}$ & $ 6143^{+  5}_{-  5}$ & $  0.82^{+  0.17}_{-  0.17}$ & $-0.009^{+0.031}_{-0.031}$  & $-0.066^{+0.011}_{-0.009}$ & $-0.018^{+0.017}_{-0.016}$ & $-0.018^{+0.017}_{-0.016}$\\
SNF20061108-004 & $128.2^{+5.4}_{-5.4}$ & $  6.8^{+2.1}_{-2.1}$ & $ 6110^{+  6}_{-  6}$ & $  2.36^{+  0.27}_{-  0.27}$ & $-0.089^{+0.038}_{-0.036}$  & $-0.025^{+0.021}_{-0.023}$ & $-0.005^{+0.024}_{-0.022}$ & $-0.005^{+0.024}_{-0.022}$\\
SNF20060912-000 & $106.9^{+1.7}_{-1.7}$ & $ 19.8^{+1.5}_{-1.5}$ & $ 6158^{+  7}_{-  7}$ & $  0.06^{+  0.17}_{-  0.17}$ & $-0.177^{+0.032}_{-0.033}$  & $0.034^{+0.018}_{-0.019}$ & $0.080^{+0.021}_{-0.020}$ & $0.080^{+0.021}_{-0.020}$\\
CSS110918\_02 & $109.1^{+8.1}_{-8.1}$ & $ 15.0^{+0.6}_{-0.6}$ & $ 6137^{+  3}_{-  3}$ & $ -0.26^{+  0.13}_{-  0.13}$ & $-0.052^{+0.027}_{-0.027}$  & $-0.039^{+0.010}_{-0.010}$ & $-0.009^{+0.014}_{-0.014}$ & $-0.009^{+0.014}_{-0.014}$\\
SNF20080918-002 & $ 96.6^{+2.8}_{-2.7}$ & $ 13.3^{+1.2}_{-1.2}$ & $ 6143^{+  6}_{-  6}$ & $ -0.47^{+  0.08}_{-  0.08}$ & $0.021^{+0.034}_{-0.035}$  & $-0.056^{+0.014}_{-0.013}$ & $-0.005^{+0.020}_{-0.021}$ & $-0.005^{+0.020}_{-0.021}$\\
SNIC3573 & $103.0^{+1.8}_{-1.7}$ & $ 12.1^{+0.9}_{-0.9}$ & $ 6141^{+  5}_{-  5}$ & $  0.39^{+  0.16}_{-  0.16}$ & $-0.109^{+0.026}_{-0.024}$  & $0.015^{+0.009}_{-0.010}$ & $-0.025^{+0.014}_{-0.012}$ & $-0.025^{+0.014}_{-0.012}$\\
SNF20080725-004 & $133.8^{+2.1}_{-2.1}$ & $  6.7^{+0.9}_{-0.9}$ & $ 6131^{+  6}_{-  5}$ & $  0.59^{+  0.13}_{-  0.13}$ & $0.078^{+0.032}_{-0.032}$  & $0.025^{+0.012}_{-0.011}$ & $0.016^{+0.015}_{-0.014}$ & $0.016^{+0.015}_{-0.014}$\\
SNF20050728-006 & $127.7^{+2.5}_{-2.4}$ & $ 15.3^{+1.2}_{-1.2}$ & $ 6124^{+  6}_{-  6}$ & $  0.11^{+  0.19}_{-  0.19}$ & $-0.088^{+0.034}_{-0.032}$  & $0.064^{+0.011}_{-0.012}$ & $0.012^{+0.017}_{-0.017}$ & $0.012^{+0.017}_{-0.017}$\\
SN2012fr & $134.2^{+0.5}_{-0.5}$ & $  7.4^{+0.2}_{-0.2}$ & $ 6101^{+  1}_{-  1}$ & $  0.88^{+  0.08}_{-  0.09}$ & $-0.248^{+0.021}_{-0.021}$  & $-0.077^{+0.005}_{-0.005}$ & $0.044^{+0.012}_{-0.011}$ & $0.044^{+0.012}_{-0.011}$\\
SNF20060512-002 & $100.4^{+2.7}_{-2.8}$ & $ 13.5^{+1.1}_{-1.1}$ & $ 6106^{+  7}_{-  7}$ & $ -0.87^{+  0.15}_{-  0.16}$ & $-0.186^{+0.030}_{-0.028}$  & $0.018^{+0.012}_{-0.012}$ & $0.056^{+0.017}_{-0.016}$ & $0.056^{+0.017}_{-0.016}$\\
SNF20060512-001 & $ 88.2^{+1.2}_{-1.2}$ & $  5.5^{+0.4}_{-0.4}$ & $ 6169^{+  3}_{-  3}$ & $  0.87^{+  0.11}_{-  0.11}$ & $-0.025^{+0.025}_{-0.025}$  & $0.050^{+0.011}_{-0.011}$ & $-0.010^{+0.011}_{-0.012}$ & $-0.010^{+0.011}_{-0.012}$\\
SNF20071003-016 & $126.5^{+4.5}_{-4.4}$ & $ 16.7^{+1.5}_{-1.5}$ & $ 6123^{+ 10}_{- 10}$ & $ -0.30^{+  0.14}_{-  0.14}$ & $0.058^{+0.032}_{-0.032}$  & $0.068^{+0.013}_{-0.014}$ & $-0.030^{+0.016}_{-0.012}$ & $-0.030^{+0.016}_{-0.012}$\\
SNF20050821-007 & $141.8^{+2.6}_{-2.6}$ & $  7.8^{+0.9}_{-0.9}$ & $ 6140^{+  9}_{-  8}$ & $  0.34^{+  0.20}_{-  0.19}$ & $-0.024^{+0.034}_{-0.033}$  & $-0.030^{+0.014}_{-0.013}$ & $-0.012^{+0.017}_{-0.017}$ & $-0.012^{+0.017}_{-0.017}$\\
SNF20070803-005 & $ 45.4^{+14.0}_{-13.8}$ & $  0.8^{+0.6}_{-0.6}$ & $ 6138^{+ 22}_{- 22}$ & $  0.81^{+  0.15}_{-  0.15}$ & $-0.137^{+0.031}_{-0.031}$  & $0.043^{+0.019}_{-0.019}$ & $-0.005^{+0.021}_{-0.021}$ & $-0.005^{+0.021}_{-0.021}$\\
PTF09foz & $127.2^{+1.9}_{-1.9}$ & $ 22.1^{+1.1}_{-1.1}$ & $ 6135^{+  4}_{-  4}$ & $ -0.92^{+  0.16}_{-  0.16}$ & $-0.141^{+0.019}_{-0.018}$  & $-0.042^{+0.009}_{-0.010}$ & $0.031^{+0.013}_{-0.012}$ & $0.031^{+0.013}_{-0.012}$\\
PTF12grk & $158.1^{+8.3}_{-8.4}$ & $ 22.1^{+1.2}_{-1.2}$ & $ 6089^{+  8}_{-  8}$ & $ -1.02^{+  0.12}_{-  0.12}$ & $-0.117^{+0.027}_{-0.026}$  & $-0.025^{+0.016}_{-0.017}$ & $-0.039^{+0.010}_{-0.008}$ & $-0.039^{+0.010}_{-0.008}$\\
SNF20080720-001 & $138.3^{+3.9}_{-4.0}$ & $ 14.0^{+1.5}_{-1.5}$ & $ 6107^{+  3}_{-  3}$ & $  0.19^{+  0.12}_{-  0.12}$ & $-0.176^{+0.028}_{-0.028}$  & $0.325^{+0.018}_{-0.018}$ & $0.007^{+0.019}_{-0.018}$ & $0.007^{+0.019}_{-0.018}$\\
SNF20080810-001 & $ 98.8^{+12.2}_{-12.3}$ & $ 21.7^{+1.0}_{-1.0}$ & $ 6145^{+  5}_{-  5}$ & $ -0.93^{+  0.10}_{-  0.11}$ & $-0.058^{+0.024}_{-0.024}$  & $-0.023^{+0.016}_{-0.016}$ & $-0.000^{+0.017}_{-0.016}$ & $-0.000^{+0.017}_{-0.016}$\\
SNF20050729-002 & $110.4^{+2.3}_{-2.2}$ & $ 10.7^{+1.5}_{-1.4}$ & $ 6141^{+  6}_{-  6}$ & $ -0.22^{+  0.21}_{-  0.22}$ & $0.052^{+0.102}_{-0.075}$  & $-0.039^{+0.015}_{-0.014}$ & $-0.039^{+0.011}_{-0.008}$ & $-0.039^{+0.011}_{-0.008}$\\
SN2008ec & $ 97.7^{+10.6}_{-10.6}$ & $ 23.0^{+0.4}_{-0.4}$ & $ 6125^{+  3}_{-  3}$ & $ -1.09^{+  0.10}_{-  0.10}$ & $-0.076^{+0.025}_{-0.026}$  & $0.064^{+0.012}_{-0.012}$ & $0.017^{+0.015}_{-0.015}$ & $0.017^{+0.015}_{-0.015}$\\
SNF20070902-018 & $108.6^{+9.5}_{-9.7}$ & $ 18.1^{+1.7}_{-1.7}$ & $ 6119^{+  7}_{-  7}$ & $ -0.81^{+  0.10}_{-  0.09}$ & $0.032^{+0.030}_{-0.030}$  & $0.023^{+0.016}_{-0.017}$ & $-0.035^{+0.014}_{-0.010}$ & $-0.035^{+0.014}_{-0.010}$\\
SNF20070424-003 & $123.6^{+3.6}_{-3.6}$ & $ 14.2^{+1.2}_{-1.1}$ & $ 6134^{+  6}_{-  6}$ & $  0.30^{+  0.10}_{-  0.10}$ & $0.008^{+0.023}_{-0.023}$  & $-0.010^{+0.010}_{-0.010}$ & $-0.041^{+0.008}_{-0.007}$ & $-0.041^{+0.008}_{-0.007}$\\
SN2006cj & $101.4^{+1.3}_{-1.3}$ & $  5.5^{+0.8}_{-0.8}$ & $ 6128^{+  3}_{-  3}$ & $  0.86^{+  0.13}_{-  0.13}$ & $0.028^{+0.025}_{-0.025}$  & $0.006^{+0.008}_{-0.009}$ & $-0.017^{+0.013}_{-0.013}$ & $-0.017^{+0.013}_{-0.013}$\\
SN2007nq & $ 91.2^{+9.1}_{-9.0}$ & $ 23.3^{+1.0}_{-1.0}$ & $ 6109^{+  5}_{-  5}$ & $ -1.90^{+  0.20}_{-  0.20}$ & $-0.075^{+0.038}_{-0.035}$  & $-0.022^{+0.015}_{-0.015}$ & $0.031^{+0.021}_{-0.020}$ & $0.031^{+0.021}_{-0.020}$\\
SNF20070817-003 & $ 94.5^{+2.4}_{-2.4}$ & $ 18.1^{+1.1}_{-1.1}$ & $ 6115^{+  6}_{-  6}$ & $ -0.61^{+  0.12}_{-  0.12}$ & $0.050^{+0.031}_{-0.032}$  & $-0.031^{+0.012}_{-0.012}$ & $-0.024^{+0.015}_{-0.014}$ & $-0.024^{+0.015}_{-0.014}$\\
SNF20070403-000 & $ 62.8^{+6.1}_{-6.2}$ & $ 27.6^{+1.6}_{-1.6}$ & $ 6154^{+  7}_{-  7}$ & $ -1.88^{+  0.17}_{-  0.17}$ & $0.147^{+0.040}_{-0.041}$  & $0.081^{+0.021}_{-0.021}$ & $0.041^{+0.025}_{-0.023}$ & $0.041^{+0.025}_{-0.023}$\\
SNF20061022-005 & $ 63.4^{+3.6}_{-3.8}$ & $  3.9^{+1.4}_{-1.4}$ & $ 6147^{+  7}_{-  7}$ & $  2.03^{+  0.25}_{-  0.25}$ & $0.056^{+0.038}_{-0.038}$  & $0.041^{+0.018}_{-0.020}$ & $-0.000^{+0.022}_{-0.021}$ & $-0.000^{+0.022}_{-0.021}$\\
SNNGC4076 & $127.1^{+2.4}_{-2.4}$ & $ 15.7^{+1.1}_{-1.1}$ & $ 6152^{+  4}_{-  4}$ & $  0.04^{+  0.10}_{-  0.10}$ & $-0.004^{+0.030}_{-0.029}$  & $0.049^{+0.013}_{-0.013}$ & $0.045^{+0.016}_{-0.015}$ & $0.045^{+0.016}_{-0.015}$\\
SNF20070727-016 & $ 78.3^{+2.4}_{-2.5}$ & $  4.6^{+0.8}_{-0.8}$ & $ 6139^{+  4}_{-  4}$ & $  0.16^{+  0.11}_{-  0.10}$ & $-0.097^{+0.043}_{-0.036}$  & $-0.037^{+0.011}_{-0.011}$ & $-0.025^{+0.015}_{-0.014}$ & $-0.025^{+0.015}_{-0.014}$\\
PTF12fuu & $105.4^{+3.0}_{-3.0}$ & $  6.5^{+1.0}_{-1.0}$ & $ 6124^{+  5}_{-  5}$ & $  0.37^{+  0.11}_{-  0.11}$ & $0.178^{+0.026}_{-0.026}$  & $-0.057^{+0.010}_{-0.010}$ & $-0.026^{+0.014}_{-0.013}$ & $-0.026^{+0.014}_{-0.013}$\\
SNF20070820-000 & $107.1^{+3.4}_{-3.4}$ & $ 18.7^{+1.2}_{-1.2}$ & $ 6133^{+ 13}_{- 13}$ & $ -0.28^{+  0.12}_{-  0.12}$ & $-0.054^{+0.031}_{-0.030}$  & $0.051^{+0.012}_{-0.013}$ & $-0.015^{+0.018}_{-0.016}$ & $-0.015^{+0.018}_{-0.016}$\\
SNF20070725-001 & $108.3^{+1.9}_{-1.9}$ & $  9.5^{+1.0}_{-1.1}$ & $ 6139^{+  6}_{-  6}$ & $  0.61^{+  0.15}_{-  0.15}$ & $0.058^{+0.033}_{-0.033}$  & $-0.072^{+0.010}_{-0.007}$ & $-0.018^{+0.016}_{-0.015}$ & $-0.018^{+0.016}_{-0.015}$\\
SNF20071108-021 & $ 98.7^{+2.7}_{-2.7}$ & $  6.0^{+0.8}_{-0.8}$ & $ 6165^{+  5}_{-  5}$ & $  1.15^{+  0.16}_{-  0.15}$ & $-0.045^{+0.028}_{-0.028}$  & $-0.028^{+0.012}_{-0.012}$ & $0.024^{+0.015}_{-0.014}$ & $0.024^{+0.015}_{-0.014}$\\
SNF20080914-001 & $126.4^{+1.2}_{-1.2}$ & $ 16.3^{+1.0}_{-1.0}$ & $ 6159^{+  3}_{-  3}$ & $  0.10^{+  0.14}_{-  0.14}$ & $-0.144^{+0.026}_{-0.024}$  & $0.094^{+0.012}_{-0.012}$ & $0.022^{+0.016}_{-0.015}$ & $0.022^{+0.016}_{-0.015}$\\
SNF20060609-002 & $ 86.3^{+3.5}_{-3.5}$ & $  8.1^{+1.2}_{-1.1}$ & $ 6132^{+  4}_{-  4}$ & $  0.18^{+  0.12}_{-  0.13}$ & $-0.069^{+0.033}_{-0.032}$  & $0.192^{+0.014}_{-0.014}$ & $-0.008^{+0.018}_{-0.017}$ & $-0.008^{+0.018}_{-0.017}$\\
SNF20050624-000 & $119.9^{+5.1}_{-5.1}$ & $  7.9^{+2.4}_{-2.6}$ & $ 6126^{+  5}_{-  5}$ & $  0.47^{+  0.29}_{-  0.29}$ & $-0.171^{+0.026}_{-0.024}$  & $-0.051^{+0.019}_{-0.017}$ & $0.022^{+0.019}_{-0.021}$ & $0.022^{+0.019}_{-0.021}$\\
SNF20060618-023 & $ 73.3^{+4.7}_{-4.7}$ & $  5.3^{+1.6}_{-1.5}$ & $ 6140^{+ 18}_{- 18}$ & $  0.18^{+  0.16}_{-  0.16}$ & $-0.229^{+0.039}_{-0.038}$  & $0.059^{+0.020}_{-0.020}$ & $0.092^{+0.023}_{-0.022}$ & $0.092^{+0.023}_{-0.022}$\\
SNF20080531-000 & $133.2^{+1.5}_{-1.5}$ & $ 17.2^{+0.8}_{-0.8}$ & $ 6113^{+  5}_{-  5}$ & $ -0.63^{+  0.13}_{-  0.13}$ & $0.002^{+0.024}_{-0.024}$  & $-0.040^{+0.009}_{-0.009}$ & $-0.009^{+0.012}_{-0.013}$ & $-0.009^{+0.012}_{-0.013}$\\
SN2006do & $106.3^{+2.1}_{-2.1}$ & $ 26.8^{+1.3}_{-1.2}$ & $ 6101^{+  2}_{-  2}$ & $ -1.95^{+  0.34}_{-  0.34}$ & $-0.077^{+0.031}_{-0.030}$  & $-0.038^{+0.012}_{-0.013}$ & $0.035^{+0.018}_{-0.017}$ & $0.035^{+0.018}_{-0.017}$\\
PTF12ikt & $110.3^{+1.5}_{-1.6}$ & $ 14.0^{+0.7}_{-0.7}$ & $ 6141^{+  4}_{-  4}$ & $ -0.40^{+  0.10}_{-  0.10}$ & $-0.023^{+0.025}_{-0.025}$  & $-0.034^{+0.009}_{-0.008}$ & $-0.009^{+0.014}_{-0.014}$ & $-0.009^{+0.014}_{-0.014}$\\
SN2006dm & $ 99.6^{+1.6}_{-1.6}$ & $ 29.8^{+0.7}_{-0.7}$ & $ 6118^{+  3}_{-  3}$ & $ -2.35^{+  0.18}_{-  0.18}$ & $0.084^{+0.029}_{-0.029}$  & $-0.047^{+0.011}_{-0.011}$ & $0.047^{+0.017}_{-0.016}$ & $0.047^{+0.017}_{-0.016}$\\
PTF13azs & $140.6^{+4.8}_{-4.8}$ & $ 15.2^{+1.3}_{-1.3}$ & $ 6124^{+ 10}_{- 10}$ & $ -0.12^{+  0.11}_{-  0.11}$ & $0.148^{+0.029}_{-0.029}$  & $0.195^{+0.012}_{-0.013}$ & $-0.037^{+0.012}_{-0.009}$ & $-0.037^{+0.012}_{-0.009}$\\
SN2005hj & $ 80.4^{+2.4}_{-2.4}$ & $  4.4^{+0.8}_{-0.8}$ & $ 6139^{+  4}_{-  4}$ & $  2.58^{+  0.24}_{-  0.24}$ & $0.076^{+0.039}_{-0.037}$  & $0.009^{+0.018}_{-0.020}$ & $0.066^{+0.022}_{-0.022}$ & $0.066^{+0.022}_{-0.022}$\\
PTF12iiq & $149.1^{+2.1}_{-2.2}$ & $ 24.8^{+0.8}_{-0.7}$ & $ 6048^{+  7}_{-  7}$ & $ -0.83^{+  0.10}_{-  0.11}$ & $0.005^{+0.033}_{-0.033}$  & $-0.005^{+0.019}_{-0.020}$ & $-0.034^{+0.017}_{-0.010}$ & $-0.034^{+0.017}_{-0.010}$\\
PTF10ndc & $124.1^{+2.4}_{-2.4}$ & $  6.7^{+1.0}_{-1.0}$ & $ 6119^{+  3}_{-  3}$ & $  0.86^{+  0.14}_{-  0.14}$ & $0.007^{+0.026}_{-0.025}$  & $-0.040^{+0.011}_{-0.011}$ & $-0.023^{+0.014}_{-0.013}$ & $-0.023^{+0.014}_{-0.013}$\\
SNF20080919-002 & $100.4^{+6.7}_{-6.6}$ & $ 26.3^{+1.7}_{-1.7}$ & $ 6133^{+  8}_{-  8}$ & $ -1.54^{+  0.13}_{-  0.13}$ & $0.340^{+0.043}_{-0.044}$  & $-0.058^{+0.017}_{-0.014}$ & $0.114^{+0.020}_{-0.021}$ & $0.114^{+0.020}_{-0.021}$\\
SNPGC027923 & $ 85.4^{+0.6}_{-0.6}$ & $  5.8^{+0.3}_{-0.3}$ & $ 6130^{+  4}_{-  4}$ & $  0.45^{+  0.17}_{-  0.17}$ & $0.078^{+0.032}_{-0.031}$  & $-0.077^{+0.006}_{-0.005}$ & $-0.026^{+0.015}_{-0.013}$ & $-0.026^{+0.015}_{-0.013}$\\
SNF20070330-024 & $118.3^{+2.1}_{-2.1}$ & $  5.5^{+1.8}_{-1.8}$ & $ 6101^{+  3}_{-  3}$ & $  0.48^{+  0.19}_{-  0.21}$ & $-0.166^{+0.025}_{-0.024}$  & $-0.040^{+0.014}_{-0.014}$ & $-0.018^{+0.017}_{-0.016}$ & $-0.018^{+0.017}_{-0.016}$\\
SNF20061030-010 & $131.6^{+2.2}_{-2.2}$ & $ 16.9^{+1.1}_{-1.1}$ & $ 6116^{+  4}_{-  4}$ & $ -0.10^{+  0.20}_{-  0.20}$ & $0.103^{+0.041}_{-0.041}$  & $0.049^{+0.012}_{-0.012}$ & $0.019^{+0.021}_{-0.021}$ & $0.019^{+0.021}_{-0.021}$\\
SNhunt46 & $ 93.8^{+2.0}_{-2.0}$ & $ 11.2^{+0.6}_{-0.6}$ & $ 6133^{+  4}_{-  4}$ & $  0.22^{+  0.17}_{-  0.17}$ & $0.150^{+0.032}_{-0.032}$  & $-0.062^{+0.010}_{-0.010}$ & $0.035^{+0.016}_{-0.015}$ & $0.035^{+0.016}_{-0.015}$\\
SN2005hc & $127.2^{+2.4}_{-2.5}$ & $  9.7^{+0.7}_{-0.7}$ & $ 6122^{+  3}_{-  3}$ & $  0.47^{+  0.16}_{-  0.15}$ & $0.137^{+0.025}_{-0.025}$  & $-0.062^{+0.008}_{-0.008}$ & $0.001^{+0.012}_{-0.012}$ & $0.001^{+0.012}_{-0.012}$\\
LSQ12dbr & $106.9^{+0.6}_{-0.6}$ & $  6.4^{+0.6}_{-0.6}$ & $ 6137^{+  4}_{-  4}$ & $  0.86^{+  0.11}_{-  0.11}$ & $0.004^{+0.226}_{-0.109}$  & $-0.071^{+0.008}_{-0.007}$ & $-0.034^{+0.011}_{-0.010}$ & $-0.034^{+0.011}_{-0.010}$\\
LSQ12hjm & $ 65.3^{+10.8}_{-11.3}$ & $ 15.3^{+1.1}_{-1.1}$ & $ 6147^{+  5}_{-  5}$ & $ -0.16^{+  0.17}_{-  0.17}$ & $0.112^{+0.030}_{-0.031}$  & $-0.062^{+0.013}_{-0.011}$ & $-0.042^{+0.008}_{-0.007}$ & $-0.042^{+0.008}_{-0.007}$\\
SNF20060521-001 & $ 89.1^{+11.4}_{-12.2}$ & $ 19.1^{+1.1}_{-1.1}$ & $ 6119^{+ 10}_{-  9}$ & $ -1.70^{+  0.20}_{-  0.20}$ & $0.029^{+0.061}_{-0.059}$  & $-0.068^{+0.014}_{-0.010}$ & $0.010^{+0.019}_{-0.020}$ & $0.010^{+0.019}_{-0.020}$\\
SNF20070630-006 & $126.3^{+3.1}_{-3.1}$ & $ 10.3^{+1.1}_{-1.2}$ & $ 6126^{+  4}_{-  4}$ & $  0.30^{+  0.11}_{-  0.11}$ & $-0.018^{+0.025}_{-0.025}$  & $-0.027^{+0.010}_{-0.010}$ & $-0.040^{+0.010}_{-0.008}$ & $-0.040^{+0.010}_{-0.008}$\\
PTF11drz & $132.6^{+1.4}_{-1.4}$ & $ 14.9^{+0.9}_{-0.9}$ & $ 6116^{+  4}_{-  4}$ & $ -0.11^{+  0.17}_{-  0.18}$ & $-0.007^{+0.025}_{-0.026}$  & $-0.021^{+0.010}_{-0.009}$ & $-0.016^{+0.013}_{-0.013}$ & $-0.016^{+0.013}_{-0.013}$\\
SNF20080323-009 & $ 96.0^{+2.2}_{-2.2}$ & $ 11.1^{+1.0}_{-1.0}$ & $ 6145^{+  5}_{-  5}$ & $ -0.15^{+  0.14}_{-  0.14}$ & $-0.029^{+0.030}_{-0.029}$  & $-0.051^{+0.011}_{-0.011}$ & $-0.041^{+0.009}_{-0.008}$ & $-0.041^{+0.009}_{-0.008}$\\
SNF20071021-000 & $167.9^{+2.2}_{-2.2}$ & $ 20.2^{+0.6}_{-0.6}$ & $ 6111^{+  4}_{-  4}$ & $ -0.66^{+  0.13}_{-  0.13}$ & $-0.105^{+0.025}_{-0.023}$  & $-0.001^{+0.010}_{-0.010}$ & $-0.004^{+0.013}_{-0.013}$ & $-0.004^{+0.013}_{-0.013}$\\
SNNGC0927 & $155.2^{+1.3}_{-1.3}$ & $ 10.9^{+0.7}_{-0.7}$ & $ 6109^{+  4}_{-  4}$ & $  0.39^{+  0.12}_{-  0.12}$ & $-0.031^{+0.027}_{-0.027}$  & $0.039^{+0.010}_{-0.010}$ & $-0.018^{+0.014}_{-0.013}$ & $-0.018^{+0.014}_{-0.013}$\\
SNF20060526-003 & $111.8^{+2.5}_{-2.5}$ & $  9.9^{+1.0}_{-1.0}$ & $ 6121^{+  3}_{-  3}$ & $  0.34^{+  0.13}_{-  0.13}$ & $-0.004^{+0.026}_{-0.027}$  & $-0.010^{+0.010}_{-0.010}$ & $-0.011^{+0.014}_{-0.014}$ & $-0.011^{+0.014}_{-0.014}$\\
SNF20080806-002 & $135.4^{+1.8}_{-1.9}$ & $  8.3^{+0.9}_{-0.9}$ & $ 6136^{+  4}_{-  4}$ & $  0.83^{+  0.14}_{-  0.14}$ & $-0.106^{+0.028}_{-0.027}$  & $0.020^{+0.010}_{-0.010}$ & $-0.036^{+0.013}_{-0.009}$ & $-0.036^{+0.013}_{-0.009}$\\
SNF20080803-000 & $118.3^{+2.5}_{-2.6}$ & $  8.4^{+1.3}_{-1.4}$ & $ 6125^{+  4}_{-  4}$ & $  0.25^{+  0.12}_{-  0.12}$ & $-0.030^{+0.023}_{-0.024}$  & $0.100^{+0.011}_{-0.011}$ & $-0.040^{+0.009}_{-0.008}$ & $-0.040^{+0.009}_{-0.008}$\\
SNF20080822-005 & $ 78.6^{+1.8}_{-1.8}$ & $  6.2^{+0.8}_{-0.8}$ & $ 6138^{+  4}_{-  4}$ & $  0.51^{+  0.16}_{-  0.16}$ & $-0.027^{+0.030}_{-0.030}$  & $-0.018^{+0.011}_{-0.010}$ & $0.014^{+0.015}_{-0.015}$ & $0.014^{+0.015}_{-0.015}$\\
SNF20060618-014 & $136.8^{+2.4}_{-2.5}$ & $  9.8^{+1.1}_{-1.1}$ & $ 6113^{+  7}_{-  7}$ & $  1.20^{+  0.18}_{-  0.18}$ & $0.077^{+0.047}_{-0.047}$  & $-0.037^{+0.014}_{-0.014}$ & $0.019^{+0.025}_{-0.024}$ & $0.019^{+0.025}_{-0.024}$\\
PTF12ghy & $100.6^{+3.5}_{-3.5}$ & $ 16.4^{+0.7}_{-0.7}$ & $ 6133^{+  3}_{-  3}$ & $ -0.58^{+  0.10}_{-  0.10}$ & $-0.138^{+0.023}_{-0.022}$  & $0.196^{+0.011}_{-0.011}$ & $0.001^{+0.013}_{-0.013}$ & $0.001^{+0.013}_{-0.013}$\\
SNF20070531-011 & $122.2^{+2.6}_{-2.6}$ & $ 21.5^{+0.8}_{-0.8}$ & $ 6114^{+  3}_{-  4}$ & $ -1.56^{+  0.20}_{-  0.20}$ & $-0.081^{+0.027}_{-0.025}$  & $-0.032^{+0.011}_{-0.011}$ & $-0.000^{+0.015}_{-0.014}$ & $-0.000^{+0.015}_{-0.014}$\\
SNF20070831-015 & $112.9^{+2.6}_{-2.6}$ & $  6.9^{+0.9}_{-0.9}$ & $ 6144^{+  6}_{-  6}$ & $  1.38^{+  0.16}_{-  0.16}$ & $0.042^{+0.027}_{-0.027}$  & $-0.016^{+0.010}_{-0.010}$ & $0.002^{+0.014}_{-0.014}$ & $0.002^{+0.014}_{-0.014}$\\
SNF20070417-002 & $103.8^{+5.0}_{-4.9}$ & $ 24.1^{+1.3}_{-1.3}$ & $ 6125^{+  8}_{-  9}$ & $ -1.37^{+  0.10}_{-  0.10}$ & $-0.032^{+0.050}_{-0.044}$  & $-0.072^{+0.010}_{-0.007}$ & $-0.039^{+0.010}_{-0.008}$ & $-0.039^{+0.010}_{-0.008}$\\
PTF11cao & $143.4^{+1.5}_{-1.6}$ & $ 19.0^{+1.1}_{-1.1}$ & $ 6104^{+  4}_{-  4}$ & $  0.20^{+  0.13}_{-  0.13}$ & $-0.014^{+0.032}_{-0.031}$  & $0.009^{+0.013}_{-0.013}$ & $-0.017^{+0.017}_{-0.015}$ & $-0.017^{+0.017}_{-0.015}$\\
SNF20080522-000 & $ 62.1^{+3.5}_{-3.5}$ & $  3.1^{+0.8}_{-0.8}$ & $ 6130^{+  7}_{-  7}$ & $  0.99^{+  0.11}_{-  0.11}$ & $-0.021^{+0.028}_{-0.028}$  & $-0.009^{+0.012}_{-0.013}$ & $0.047^{+0.014}_{-0.013}$ & $0.047^{+0.014}_{-0.013}$\\
PTF10qjq & $ 74.4^{+2.3}_{-2.4}$ & $ 12.0^{+0.7}_{-0.7}$ & $ 6132^{+  3}_{-  3}$ & $ -0.50^{+  0.10}_{-  0.09}$ & $-0.127^{+0.026}_{-0.024}$  & $0.049^{+0.010}_{-0.011}$ & $-0.010^{+0.015}_{-0.015}$ & $-0.010^{+0.015}_{-0.015}$\\
PTF12dxm & $119.3^{+15.5}_{-16.8}$ & $ 31.3^{+1.9}_{-2.1}$ & $ 6136^{+  4}_{-  4}$ & $ -2.14^{+  0.18}_{-  0.18}$ & $0.034^{+0.034}_{-0.033}$  & $-0.047^{+0.027}_{-0.021}$ & $0.067^{+0.023}_{-0.024}$ & $0.067^{+0.023}_{-0.024}$\\
SNF20061021-003 & $122.5^{+2.2}_{-2.2}$ & $ 10.8^{+1.5}_{-1.5}$ & $ 6131^{+  4}_{-  4}$ & $  0.39^{+  0.15}_{-  0.15}$ & $-0.012^{+0.035}_{-0.033}$  & $0.036^{+0.014}_{-0.014}$ & $0.047^{+0.019}_{-0.017}$ & $0.047^{+0.019}_{-0.017}$\\
SNF20080510-005 & $112.2^{+2.5}_{-2.5}$ & $  7.0^{+1.0}_{-1.0}$ & $ 6117^{+  4}_{-  4}$ & $  1.68^{+  0.22}_{-  0.22}$ & $0.116^{+0.030}_{-0.030}$  & $-0.029^{+0.013}_{-0.015}$ & $-0.042^{+0.008}_{-0.007}$ & $-0.042^{+0.008}_{-0.007}$\\
SNF20080507-000 & $ 98.4^{+1.6}_{-1.6}$ & $  8.0^{+1.7}_{-1.6}$ & $ 6142^{+  5}_{-  5}$ & $  0.56^{+  0.18}_{-  0.18}$ & $-0.126^{+0.023}_{-0.022}$  & $0.063^{+0.014}_{-0.013}$ & $-0.003^{+0.015}_{-0.014}$ & $-0.003^{+0.015}_{-0.014}$\\
SNF20080913-031 & $117.9^{+1.5}_{-1.5}$ & $ 14.7^{+1.6}_{-1.6}$ & $ 6159^{+  5}_{-  5}$ & $  0.49^{+  0.17}_{-  0.17}$ & $-0.048^{+0.067}_{-0.052}$  & $-0.003^{+0.015}_{-0.015}$ & $-0.009^{+0.020}_{-0.018}$ & $-0.009^{+0.020}_{-0.018}$\\
SNF20080510-001 & $118.9^{+2.0}_{-2.1}$ & $ 15.0^{+1.1}_{-1.0}$ & $ 6115^{+  4}_{-  4}$ & $  0.34^{+  0.14}_{-  0.14}$ & $0.001^{+0.026}_{-0.026}$  & $-0.058^{+0.011}_{-0.012}$ & $-0.032^{+0.012}_{-0.011}$ & $-0.032^{+0.012}_{-0.011}$\\
SNF20070712-003 & $108.1^{+2.7}_{-2.6}$ & $ 14.1^{+0.9}_{-0.9}$ & $ 6157^{+  6}_{-  6}$ & $  0.62^{+  0.14}_{-  0.14}$ & $-0.034^{+0.030}_{-0.029}$  & $-0.052^{+0.010}_{-0.011}$ & $-0.021^{+0.015}_{-0.014}$ & $-0.021^{+0.015}_{-0.014}$\\
\enddata
\end{deluxetable}
}




The distributions of the Monte Carlo links for all supernovae are shown in Figure~\ref{perobject3:fig}.
Each supernova is represented by a cloud of its parameters' links.
There is a core concentration in the  parameter-space, with around eight objects that occupy its outskirts.
Many outliers appear in the red tail of $E_{\gamma^0}({\hat{B}}-{\hat{V}})$, as would be expected for the (infrequent) selection of supernovae
heavily extinguished by host-galaxy dust.

The Pearson correlation coefficients of the per-supernova parameters are given in the matrix
\begin{multline}
Cor(EW_{\mathrm{Ca}}, EW_{\mathrm{Si}}, \lambda_{\mathrm{Si}}, x_1, E_{\gamma^0}({\hat{B}}-{\hat{V}}), E_{\gamma^1}({\hat{B}}-{\hat{V}}), A_{p,V}) =\\
\begin{pmatrix}
\begin{array}{rrrrrrr}
EW_{\mathrm{Ca}} & 0.10^{+0.03}_{-0.03} & -0.25^{+0.03}_{-0.03} & 0.07^{+0.03}_{-0.03} & -0.10^{+0.05}_{-0.05} & -0.18^{+0.06}_{-0.06} & 0.00^{+0.08}_{-0.08} \\
\ldots &EW_{\mathrm{Si}} & -0.12^{+0.03}_{-0.03} & -0.84^{+0.01}_{-0.01} & -0.13^{+0.04}_{-0.04} & 0.27^{+0.08}_{-0.09} & -0.00^{+0.08}_{-0.08} \\
\ldots & \ldots& \lambda_{\mathrm{Si}} & 0.14^{+0.03}_{-0.03} & 0.01^{+0.06}_{-0.05} & 0.00^{+0.06}_{-0.06} & -0.00^{+0.08}_{-0.08} \\
\ldots& \ldots& \ldots& x_1 & 0.13^{+0.04}_{-0.04} & -0.20^{+0.09}_{-0.09} & -0.00^{+0.08}_{-0.08} \\
\ldots& \ldots& \ldots & \ldots& E_{\gamma^0}({\hat{B}}-{\hat{V}})& 0.04^{+0.06}_{-0.06} & 0.00^{+0.08}_{-0.08} \\
\ldots & \ldots & \ldots& \ldots&\ldots & E_{\gamma^1}({\hat{B}}-{\hat{V}})  & 0.00^{+0.08}_{-0.08} \\
\ldots & \ldots & \ldots&\ldots &\ldots &\ldots &  A_{p,V} \\
\end{array}
\end{pmatrix}.
\label{corr:eqn}
\end{multline}
Recall that the $\Delta$-cofactor degeneracy is broken by enforcing zero correlation between $\Delta$ and the features.
To aid readability, only the upper triangular part of the otherwise symmetric matrix is shown and the corresponding feature label is given on
the diagonal.

\begin{figure}[htbp] %  figure placement: here, top, bottom, or page
   \centering
   \includegraphics[width=5.2in]{perobject_corner.pdf} 
   \caption{
   Distributions for the supernova parameters $\Delta$, $EW_{\mathrm{Ca}}$, $EW_{\mathrm{Si}}$, $\lambda_{\mathrm{Si}}$, $x_1$, $E_{\gamma^0}({\hat{B}}-{\hat{V}})$,  $E_{\gamma^1}({\hat{B}}-{\hat{V}})$,  and  $A_{p,V}$, as well as the grey offset
$\Delta$.  All Monte Carlo links are plotted, so that each supernova contributes a cloud of points.
   \label{perobject3:fig}}
\end{figure}

\added{
The correlation matrix in Eq.~\ref{corr:eqn} is not close to being singular, meaning that the each of the features conveys
independent supernova variability.  Nevertheless, a strong correlation between two features means that the determination
of one would reduce the residual dispersion in the other.  The amount of residual dispersion is conveyed in the matrix
\begin{equation}
Z =
\begin{pmatrix}
\scriptsize
\begin{array}{rrrrrrr}
27.589  \pm 0.554 & 27.451  \pm 0.572 & 26.668  \pm 0.589 & 27.509  \pm 0.542 & 27.408  \pm 0.563 & 27.099  \pm 0.640 & 27.506  \pm 0.562 \\
7.041  \pm 0.160 & 7.076  \pm 0.157 & 7.019  \pm 0.165 & 3.860  \pm 0.186 & 7.011  \pm 0.158 & 6.790  \pm 0.206 & 7.056  \pm 0.159 \\
19.779  \pm 0.449 & 20.295  \pm 0.458 & 20.463  \pm 0.460 & 20.267  \pm 0.456 & 20.430  \pm 0.461 & 20.423  \pm 0.462 & 20.404  \pm 0.465 \\
0.992  \pm 0.018 & 0.543  \pm 0.023 & 0.986  \pm 0.019 & 0.995  \pm 0.018 & 0.986  \pm 0.019 & 0.971  \pm 0.026 & 0.992  \pm 0.019 \\
0.073  \pm 0.003 & 0.073  \pm 0.003 & 0.074  \pm 0.003 & 0.073  \pm 0.003 & 0.074  \pm 0.003 & 0.074  \pm 0.003 & 0.074  \pm 0.003 \\
0.038  \pm 0.004 & 0.037  \pm 0.004 & 0.039  \pm 0.004 & 0.038  \pm 0.004 & 0.039  \pm 0.004 & 0.039  \pm 0.004 & 0.039  \pm 0.004 \\
0.050  \pm 0.011 & 0.050  \pm 0.011 & 0.050  \pm 0.011 & 0.050  \pm 0.011 & 0.050  \pm 0.011 & 0.050  \pm 0.011 & 0.050  \pm 0.011 
\\\end{array}
\end{pmatrix},
\label{ellipse:eqn}
\end{equation}
where the associated units are [\AA, \AA, \AA, 1, mag, mag, mag]$\cdot$[\AA, \AA, \AA, 1, mag, mag, mag]$^{T}$.

For indices $i, j \in \{EW_{\mathrm{Ca}}, EW_{\mathrm{Si}}, \lambda_{\mathrm{Si}}, x_1, E_{\gamma^0}({\hat{B}}-{\hat{V}}), E_{\gamma^1}({\hat{B}}-{\hat{V}}), A_{p,V}\}$, $M_{ii}$ is the standard deviation of the parameter $i$ distribution, and for $i \ne j$ $Z_{ij}$ is the standard deviation in parameter $i$
if parameter $j$ were known perfectly.  For example, for $i=EW_{\mathrm{Si}}$ the standard deviation is 7.08\AA\ but when $j=x_1$
is fixed its
standard deviation drops to 3.86\AA.  Therefore, an analysis that considers $x_1$ but not $EW_{\mathrm{Si}}$ would not account for
3.86\AA$\vec{\beta}=\{0.14 ,  0.09,  0.08,  0.07,  0.04 \}$~mag ${\hat{U}}{\hat{B}}{\hat{V}}{\hat{R}}{\hat{I}}$ magnitude dispersions
that are not intrinsically random.
}

An extensive discussion of the correlations between the spectral features  and light-curve parameters of the SNfactory data
set can be found in \citet{chotard:thesis, leget:thesis, 2017Chotard}. 
We here confine ourselves to noting the strong anti-correlation between $EW_{\mathrm{Si}}$ and $x_1$, which was alluded
to in \S\ref{results3global:sec} when interpreting the correlation between the cofactors $\vec{\beta}$ and $\vec{\zeta}$ that connect them with magnitudes.

The $\Delta$ parameter will be discussed in its own subsection \S\ref{Delta:sec}.

\subsubsection{Results for the Latent  Parameters $g_0$, $g_1$}
\label{g0g1:sec}
The vector
$\vec{\gamma}^0$ describes color changes for an effective $\langle R^F_{\mathit{eff}}\rangle=2.43$ dust, as shown 
in  \S\ref{results3gamma:sec}.
Hence, the per-supernova parameter $\gamma^0_{\hat{V}} g_0$ is associated with the $\hat{V}$-band
extinction for a constant $R^F=2.43$.
Figure~\ref{k0_med:fig} shows the histogram of
per-supernova
median values
from the Monte Carlo chains
and the  stack of the posteriors of all supernovae  for $\gamma^0_{\hat{V}} g_0$ 
relative to that of an arbitrary supernova  $\gamma^0_{\hat{V}} g_0|_0$.
The distributions are non-Gaussian, having a sharp rise in the blue and an extended tail in the red.  This is consistent
with
simulations based on expected dust distributions
within galaxies and the distribution of galaxy orientations with respect to the observer,
which determine column densities along the lines of sight toward supernovae that are consistent with observations
\citep{1998ApJ...502..177H, 2007ApJ...659..122J}.  
We emphasize that unlike other analyses,
this distribution is given by the data, and so is not dependent on any knowledge or prior
for the distribution of dust extinction the SN population is expected to suffer. 
There is no sign of two populations in $\gamma^1_{\hat{V}} g_1$, which would be expected from a subpopulation
experiencing extinction and scattering by circumstellar dust.

\begin{figure}[htbp] %  figure placement: here, top, bottom, or page
   \centering
   \includegraphics[width=2.8in]{deltagamma0_med.pdf}
   \includegraphics[width=2.8in]{deltagamma1_med.pdf}
      \caption{
      Normalized stack of the posteriors (includes parameter uncertainty) of all supernovae  and histogram 
      of  per-supernova 
median values of: (left) $\gamma^0_{\hat{V}} g_0-\gamma^0_{\hat{V}} g_0|_0$, which is associated with the $A^F_V$ of an $\langle R^F_{\mathit{eff}}\rangle \sim 2.43$ dust model;
(right) $\gamma^1_{\hat{V}} g_1-\gamma^1_{\hat{V}} g_1|_0$,
which  is associated with extinction corrections due to deviations away from the canonical $\langle R^F_{\mathit{eff}}\rangle$ value.
   \label{k0_med:fig}}
\end{figure}

An instructive way to consider the  $\{g_0, g_1\}$ parameters is
to transform them to $\{A_V^F, E^F(B-V)\}$ (modulo additive constants)   using the
matrix $M$ in  Eq.~\ref{trans_I:eqn}.
A plot showing the expected values and 68\% credible intervals of these parameters
for our supernova
sample is shown in Figure~\ref{kk:fig}. 
For reference, a line that represents $\langle R^F_{\mathit{eff}}\rangle=2.43$ is overplotted; points above
the line have larger $R^F$, points below have smaller $R^F$.  The majority of supernovae lie within a narrow range above the line,
while the remaining fraction fall in a broader range below the line.
Comparison with  $\gamma^1_{\hat{V}} g_1-\gamma^1_{\hat{V}} g_1|_0$ of the right plot of Figure~\ref{k0_med:fig}
shows that the 
negative tail corresponds to smaller $R^F$, and the sharper positive edge to larger $R^F$.
These findings are qualitatively consistent with previous results:
using fixed dust-extinction models,
\citet{2014ApJ...789...32B, 2015MNRAS.453.3300A} deduce a wide range of dust behavior $1.5<R^F<3$ encountered by the SN~Ia population.
Our model cannot provide the range of $R^F$ in our sample nor per-supernova determinations of $R^F$ 
without further assumptions.

\begin{figure}[htbp] %  figure placement: here, top, bottom, or page
   \centering
   \includegraphics[width=4in]{avebv_synth.pdf}
      \caption{
      Expected values and 68\% credible intervals of effective $E^F(B-V)$ and $A_V^F$ after transformation from our model $g_0$ and $g_1$ parameters, for the supernova in our sample.
      Overplotted is a line with the slope expected for $R^F=2.43$.
   \label{kk:fig}}
\end{figure}



There are $>2 \sigma$ correlations between our  color-excess parameters $g_0$, $g_1$
and the input features  $EW_{\mathrm{Ca}}$,
$EW_{\mathrm{Si}}$, and $x_1$. 
Given the association
of  $g$ with dust-extinction parameters, this implies that the host environment and the input features are connected.
The correlations between $EW_{\mathrm{Si}}$ and  $EW_{\mathrm{Ca}}$  with host have already been noted \citep{2011ApJ...734...42N, 2015MNRAS.451.1973S}.
There is a lack of correlation between  $g_0$, $g_1$ and  our velocity parameter $\lambda_{\mathrm{Si}}$
(measured at peak brightness).
This contrasts with the subset of supernovae identified by their high-velocity features (HVF) in
Si~II~$\lambda$6355
(typically detected pre-maximum), who are found to be distinguished by their host galaxy properties
\citep[e.g.][]{2014MNRAS.444.3258M, 2015MNRAS.446..354P, 2015ApJS..220...20Z}.
The population and properties of HVF SNe within the SNfactory sample will be presented in \citet{2018lin}.
The correlation of $x_1$ with  host-galaxy (including dust) properties is  well established 
\citep{2000AJ....120.1479H, 2003MNRAS.340.1057S}.


\subsubsection{Results for the Parameter $p$}
\label{p:sec}
The model assumes that $p$ has a Normal distribution, which is shown in Figure~\ref{ebv3:fig} as the 
stack of the posteriors of all supernovae of the color-excess $E_p({\hat{B}}-{\hat{V}})$.
Also shown for comparison is the distribution of
the   color-excess $E_\gamma({\hat{B}}-{\hat{V}}) \equiv (\gamma^0_{\hat{B}} -\gamma^0_{\hat{V}}) g_0
+ (\gamma^1_{\hat{B}} -\gamma^1_{\hat{V}}) g_1$.
The standard deviations of  $E_\gamma({\hat{B}}-{\hat{V}})$ and $E_p({\hat{B}}-{\hat{V}})$ are
%-----
0.085 %new
and 0.010
%-----
mag respectively.
The latter is comparable to the typical measurement uncertainty.
In $\hat{B}-\hat{V}$, 
the $g$-based color variation is much larger than that of the $p$-based and its effect per-supernova would be difficult to discern.

\begin{figure}[htbp] %  figure placement: here, top, bottom, or page
   \centering
   \includegraphics[width=4in]{ebv.pdf}
      \caption{Stack of the posteriors  (includes parameter uncertainty) of all supernovae of the  $E_{\gamma}({\hat{B}}-{\hat{V}})$ 
      and the 
 $E_p({\hat{B}}-{\hat{V}})$  contributions to color excess relative to an arbitrary supernova.
   \label{ebv3:fig}}
\end{figure}


There is no significant correlation between the parameter $p$ and the other features.
Their independence from the  $g_0$ and $g_1$ parameters
implies that two physical parameters are not being artificially attributed to three
model parameters.  One could have
worried this might
occur since our linear model does not precisely
describe the non-linearity between broad-band magnitudes and dust parameters. A
correlation 
between the parameters would have complicated any claims of the detection of a third independent supernova parameter.

Detection of intrinsic color has been previously reported by \citetalias{2017ApJ...842...93M}.
For data, they use the outputs of SALT2 fits: the $c$ color parameter;
the
supernova absolute magnitude $M$.
In the \citetalias{2017ApJ...842...93M}
model, intrinsic color contributes a linear   $\beta_{\mathit{int}} c$ to  the absolute
magnitude.
They find a significant $\beta_{\mathit{int}} = 2.2\pm 0.3$.
While we too claim the detection of a color parameter incompatible with dust,
the differences between our models make it
difficult to determine whether the results are consistent.   Our results show effects on color that are not monotonic
with wavelength.  We find that the statistic that nominally best corresponds to  $\beta_{\mathit{int}}$
to be
$\phi_{\hat{B}}/(\phi_{\hat{B}}-\phi_{\hat{V}}) =-4.4_{  -6.4}^{+   2.4}$, which appears to be inconsistent with  \citetalias{2017ApJ...842...93M}.  
However a change of the color baseline gives a significantly different
$\phi_{\hat{B}}/(\phi_{\hat{B}}-\phi_{\hat{R}}) =4.3_{  -1.4}^{+   3.8}$.  Our results indicate that  $\beta_{\mathit{int}}$
is sensitive to the color that $c$ corresponds to.  Although SALT2 $c$ is calibrated to correspond with $B-V$, it is
determined using data from all bands with the wavelength extremes  typically providing the strongest leverage.
The $c$ parameter is thus an  amalgamation of all input colors.
We therefore make no conclusions on the consistency between our $\vec{\phi}$-color properties
and those found by \citetalias{2017ApJ...842...93M}.


An independent piece of information available for a subset of our supernova sample is host-galaxy mass \citep{2013ApJ...770..108C}.  
This statistic is of interest, as there is a correlation between Hubble residual and host-galaxy mass
\citep[first noted by][]{2010ApJ...715..743K,2010MNRAS.406..782S}, a signal confirmed to exist in the SNfactory
sample \citep{2013ApJ...770..108C}.
This host-mass bias could be the result of a parameter that was not accounted for in the inference of SN~Ia absolute magnitude.
Indeed  \citetalias{2017ApJ...842...93M} find that the  introduction of intrinsic color as a latent parameter
reduces the strength of  this bias.

It is possible that the  $p$ parameter is a supernova-tracer of a population that to date has only been identified through host-galaxy tracers.
We plot in Figure~\ref{childress3:fig} our parameter
 $\sigma_p\phi_{\hat{V}}(p-p|_0) $  (subtracting out a random supernova
to suppress correlated errors) versus host mass
for the subset of supernovae whose host measurements are given in \citet{2013ApJ...770..108C}.
Low- and high-mass galaxies, divided by $\log{(M/M_\sun)}=10$, have high probability of hosting supernovae with different $p$ distributions, as
the Kolmogorov-Smirnov test gives a two-tailed $p$-value of $  0.016 _ {     0.011 } ^{     0.044 }$.
The median and median uncertainties of the two subsamples are
%---
$\langle \sigma_p\phi_{\hat{V}}(p-p|_0) \rangle=  0.041 \pm {     0.007 }$ mag,
$\langle \sigma_p\phi_{\hat{V}}(p-p|_0)  \rangle=0.065 \pm {     0.007 }$ mag
for low- and high-mass hosts respectively, with a difference of $ 0.024 \pm {     0.010 }$~mag that is significant at $>2 \sigma$.

In this article we do not associate $p$ directly with Hubble residuals. This is left for future work.

\begin{figure}[htbp] %  figure placement: here, top, bottom, or page
   \centering
   \includegraphics[width=4in]{childress.pdf} 
   \caption{Parameter $\sigma_p\phi_{\hat{V}}p$  versus host galaxy mass. Overplotted are the mean and 1$\sigma$ uncertainty on the mean for supernovae with hosts
      less than and greater than  $\log{(M/M_\sun)}=10$.
 $p$ as a function of host-galaxy mass
    \label{childress3:fig}}
\end{figure}


\subsubsection{Results for the Magnitude Offset $\Delta$}
\label{Delta:sec}

The color-standardization leaves magnitude residuals $\Delta$.
The histogram of the per-supernova  medians of these relative  grey offsets $\Delta-\Delta|_0$,
is shown in Figure~\ref{hist:fig}.  The distribution
has a total standard deviation of
%-----
$0.13$
%-----
mag, and a tail in the positive (fainter) direction.
Supernova peculiar velocities contribute to the standard deviation; of the four
supernovae with the highest values of $\Delta$, two have 
heliocentric
redshifts of 0.0015 and 0.0086. 
\replaced{
Eq.~\ref{corr:eqn} shows $>2 \sigma$ correlations between $\Delta$ and each of $\lambda_{\mathrm{Si}}$, $x_1$, $E_{\gamma^1}({\hat{B}}-{\hat{V}})$, and
$\sigma_p\phi_{\hat{V}}p$, showing that in addition to standardizing colors, these features can also be used to standardize
absolute magnitudes.  }{Recall that the model degeneracy between $\Delta$ and the cofactors (e.g.\ $\alpha$, $\beta$) is removed by enforcing
zero correlation between $\Delta$ and the per-supernova parameters (e.g.\ $EW_{\mathrm{Ca}}$,  $EW_{\mathrm{Si}}$).
An alternative method to break the degeneracy is to minimize the dispersion in $\Delta$.
}
The focus of this article is color standardization so we defer consideration of absolute-magnitude standardization to future work.
 
\begin{figure}[htbp] %  figure placement: here, top, bottom, or page
   \centering
   \includegraphics[width=4in]{deltaDelta_hist.pdf} 
   \caption{
   Normalized stack of the posteriors  (includes parameter uncertainty) of all supernovae and histogram of the per-supernova medians of the grey offset $\Delta$. 
   To help null correlated errors, we select
    an arbitrary supernova and subtract out its $\Delta|_0$ from those of all other objects at the level of each MCMC link.
  Of the four
supernovae with the highest values of $\Delta$, two have 
heliocentric
redshifts of 0.0015 and 0.0086. 
   \label{hist:fig}}
\end{figure}



\section{Conclusions}
\label{conclusions:sec}
To summarize, we model SN~Ia broadband optical peak magnitudes allowing for correlations with spectral features at peak,
light-curve shape, and
latent color parameters.  Analyzing SNfactory data with this model, we find significant evidence that the above parameters do
affect supernova magnitudes and colors.  Two of the latent color parameters are consistent with the
\citetalias{1999PASP..111...63F} dust-extinction model, making this
the first determination of the dust extinction curve outside the Local Group
derived entirely independently of assumptions about the shape of the extinction curve and/or assumptions about the
distribution of $A_V$.  
We find a broad distribution for total-to-selective extinction
with an effective
$\langle R^F_{\mathit{eff}}\rangle \sim 2.43$ and an asymmetric tail extending toward lower values.
We identify a new
parameter that affects supernova
colors in a manner that is distinct from the expectations for dust
and the SALT2 shape-parameter $x_1$.
This new parameter correlates with host-galaxy mass, meaning it is a candidate supernova observable that may be linked
with the Hubble diagram mass-step.
A significant bias in $R^F$ is obtained when this third source of color variation
is unaccounted for.

Just as supernova spectral and light curve features are associated with distinct color variations, the latent parameter $p$
identified through color variation may be associated with a spectral and/or light curve feature not included
in our analysis.  It
is of interest to look for tracers other than color to account for this supernova diversity.
We leave to future work the search for correlations between $p$ and other
observable features accessed by SNfactory data.

Using different arguments based on expected populations, \citetalias{2017ApJ...842...93M} infer
separate intrinsic and dust-based color-magnitude relations based on an independent sample of 248 SNe~Ia
mainly from the CfA \citep{1999AJ....117..707R, 2006AJ....131..527J, 2009ApJ...700.1097H, 2012ApJS..200...12H}
and CSP \citep{2010AJ....139..519C, 2011AJ....142..156S} surveys.
Finding slight color perturbations in the multi-band light curves 
of those samples that are  consistent with our $\sigma_p \vec{\phi}$ would provide strong evidence 
that we and  \citetalias{2017ApJ...842...93M}  are seeing the same color effect.

The current data shows a statistically significant difference in the distribution of our new parameter for supernovae
in low- and high-mass host-galaxies.
Using the SNfactory sample,
\citet{2017Rigault} find that a step in Hubble residuals is better related to local 
specific
star 
formation rate, rather than
global host mass.  The search for a correlation between our parameter and local star formation rate is planned for future work.

An interesting direction for future work is to use narrower bands for generating synthetic photometry
from the SNfactory spectrophotometry.  When the bandwidth becomes small enough
to resolve spectral features, the analysis would produce a spectroscopic model.   Higher resolution would allow us to forego 
the use of a fiducial template necessary to predict broadband fluxes generated by dust models, and also allow direct incorporation
of dust models into our framework.


The approach of our analysis is to mine for new supernova properties based on colors and spectral features.
Our model and results do carry information on absolute magnitude but are not tailored for its study.
The grey parameter $\Delta$, while containing information on absolute magnitude, 
also 
includes
contributions from peculiar velocities and measurement uncertainties.  Studies
focused on improving SNe~Ia as standard candles 
would need to
modify  our model to
distinguish between these sources of greyness and dispersion.
Our
mining exercise uses all supernovae that pass quality cuts, the approach taken by previous
exploratory work.  In contrast, the calibration of SNe~Ia as standard candles needs procedures such
as cross-validation to avoid overtraining.
An absolute magnitude calibration using the parameters identified in this analysis is left to future work.

The existence of three color parameters  has important implications when SNe~Ia are used as distance indicators.
Measurements in at least four bands are necessary to obtain three colors.  
Supernova distance determinations that do not account for three color parameters are susceptible to 
systematic errors.  
Parameter populations evolving with redshift would result in errors in
Hubble diagrams.
The use of  correlators with the SN parameters, such as host-galaxy properties e.g.\ as shown for our $p$ parameter,
may partially mitigate such errors.

\acknowledgments
We thank the STAN team for providing the statistical tool without which this analysis would not have been possible,
and Michael Betancourt specifically for his helpful guidance.  
Distribution surfaces are plotted using the ChainConsumer package \citep{Hinton2016}.
We thank Danny Goldstein and
Xiaosheng Huang for useful discussions.
We thank Dan Birchall for observing assistance, the technical and
scientific staffs of the Palomar Observatory, the High Performance
Wireless Radio Network (HPWREN), and the University of Hawaii 2.2~m
telescope.  We recognize the significant cultural role of Mauna Kea
within the indigenous Hawaiian community, and we appreciate the
opportunity to conduct observations from this revered site.  This
work was supported in part by the Director, Office of Science,
Office of High Energy Physics, of the U.S. Department of Energy
under Contract No. DE-AC02- 05CH11231.  Support in France was
provided by CNRS/IN2P3, CNRS/INSU, and PNC; LPNHE acknowledges
support from LABEX ILP, supported by French state funds managed by
the ANR within the Investissements d'Avenir programme under reference
ANR-11-IDEX-0004-02.  Support in Germany was provided by the DFG
through TRR33 ``The Dark Universe;'' and in China from Tsinghua
University 985 grant and NSFC grant No~11173017.  Some results were
obtained using resources and support from the National Energy
Research Scientific Computing Center, supported by the Director,
Office of Science, Office of Advanced Scientific Computing Research,
of the U.S. Department of Energy under Contract No. DE-AC02-05CH11231.
HPWREN is funded by National Science Foundation Grant Number
ANI-0087344, and the University of California, San Diego.
This project has received funding from the European Research Council (ERC) under the European Union's Horizon 2020 research and innovation programme (grant agreement No 759194 - USNAC 

\appendix
\section{Results from Model~I and II}
\label{m12results:sec}
For each of the five filters, the 68\%  equal-tailed credible intervals for the global parameters $\vec{\alpha}$, $\vec{\beta}$, $\vec{\eta}$,
$\vec{\gamma}^0$ and $\vec{\gamma}^1$ of Model~I are given in  Table~\ref{global2:tab}.
The results for the same parameters and $\sigma_p\vec{\phi}$
of Model~II are given in Table~\ref{global2:tab}.  In contrast to Model~III,
Figure~\ref{M1ev:fig} shows that the $\hat{I}$-band has
the largest separation between the degenerate solution of $\sigma_p \phi_X$.  To break the degeneracy we enforce 
$\vec{\phi} = \text{sign}(\phi_{\hat{I}}) \vec{\phi}$.


\begin{table}
\centering
\begin{tabular}{|c|c|c|c|c|c|}
\hline
Parameters& $X={\hat{U}}$ &${\hat{B}}$&${\hat{V}}$&${\hat{R}}$&${\hat{I}}$\\ \hline
$\alpha_X$
& $0.0037^{+0.0003}_{-0.0004}$
& $0.0011^{+0.0003}_{-0.0003}$
& $0.0012^{+0.0003}_{-0.0003}$
& $0.0013^{+0.0003}_{-0.0003}$
& $0.0026^{+0.0002}_{-0.0003}$
\\
${\alpha_X/\alpha_V-1}$
& $   2.0^{+   0.5}_{  -0.3}$
& $  -0.1^{+   0.1}_{  -0.1}$
& \ldots
& $   0.1^{+   0.1}_{  -0.0}$
& $   1.1^{+   0.4}_{  -0.2}$
\\
$\beta_X$
& $ 0.026^{+ 0.002}_{-0.002}$
& $ 0.019^{+ 0.001}_{-0.001}$
& $ 0.022^{+ 0.001}_{-0.001}$
& $ 0.018^{+ 0.001}_{-0.001}$
& $ 0.018^{+ 0.001}_{-0.001}$
\\
${\beta_X/\beta_V-1}$
& $  0.17^{+  0.02}_{ -0.02}$
& $ -0.14^{+  0.02}_{ -0.02}$
& \ldots
& $ -0.18^{+  0.01}_{ -0.01}$
& $ -0.16^{+  0.01}_{ -0.01}$
\\
$\eta_X$
& $-0.0006^{+0.0005}_{-0.0005}$
& $-0.0007^{+0.0004}_{-0.0004}$
& \ldots
& $0.0002^{+0.0003}_{-0.0004}$
& $-0.0007^{+0.0003}_{-0.0003}$
\\
${\eta_X/\eta_V-1}$
& $ -0.30^{+  4.06}_{ -3.18}$
& $ -0.73^{+  5.16}_{ -3.69}$
& \ldots
& $ -0.02^{+  0.81}_{ -0.97}$
& $ -1.03^{+  5.25}_{ -3.77}$
\\
$\gamma^0_X$
& $ 76.37^{+  3.19}_{ -3.23}$
& $ 60.22^{+  3.09}_{ -3.20}$
& $ 43.06^{+  2.95}_{ -3.11}$
& $ 32.93^{+  2.60}_{ -2.87}$
& $ 22.89^{+  2.27}_{ -2.65}$
\\
${\gamma^0_X/\gamma^0_V-1}$
& $  0.78^{+  0.06}_{ -0.06}$
& $  0.40^{+  0.03}_{ -0.03}$
& \ldots
& $ -0.24^{+  0.01}_{ -0.01}$
& $ -0.47^{+  0.02}_{ -0.03}$
\\
$\gamma^1_X$
& $-18.64^{+  4.86}_{ -4.86}$
& $-28.10^{+  4.23}_{ -4.20}$
& $-28.71^{+  3.75}_{ -3.70}$
& $-21.86^{+  3.49}_{ -3.32}$
& $-11.49^{+  3.38}_{ -3.05}$
\\
${\gamma^1_X/\gamma^1_V-1}$
& $ -0.35^{+  0.08}_{ -0.10}$
& $ -0.02^{+  0.03}_{ -0.03}$
& \ldots
& $ -0.24^{+  0.02}_{ -0.03}$
& $ -0.60^{+  0.06}_{ -0.08}$
\\
\hline
\end{tabular}
\caption{68\% credible intervals for the global fit parameters of the Two Color Parameter Model~I in \S\ref{modelI:sec}.\label{global1:tab}}
\end{table}


\begin{table}
\centering
\begin{tabular}{|c|c|c|c|c|c|}
\hline
Parameters& $X={\hat{U}}$ &${\hat{B}}$&${\hat{V}}$&${\hat{R}}$&${\hat{I}}$\\ \hline
$\alpha_X$
& $0.0048^{+0.0010}_{-0.0011}$
& $0.0019^{+0.0008}_{-0.0009}$
& $0.0018^{+0.0007}_{-0.0007}$
& $0.0017^{+0.0005}_{-0.0005}$
& $0.0027^{+0.0004}_{-0.0004}$
\\
${\alpha_X}/\alpha_{\hat{V}-1}$
& $   1.7^{+   0.8}_{  -0.4}$
& $   0.1^{+   0.1}_{  -0.1}$
& \ldots
& $  -0.0^{+   0.1}_{  -0.1}$
& $   0.5^{+   0.6}_{  -0.3}$
\\
$\beta_X$
& $ 0.035^{+ 0.002}_{-0.003}$
& $ 0.026^{+ 0.002}_{-0.002}$
& $ 0.027^{+ 0.002}_{-0.002}$
& $ 0.022^{+ 0.001}_{-0.001}$
& $ 0.021^{+ 0.001}_{-0.001}$
\\
${\beta_X}/\beta_{\hat{V}-1}$
& $  0.27^{+  0.04}_{ -0.04}$
& $ -0.05^{+  0.02}_{ -0.02}$
&  \ldots
& $ -0.19^{+  0.01}_{ -0.01}$
& $ -0.22^{+  0.03}_{ -0.02}$
\\
$\eta_X$
& $-0.0005^{+0.0008}_{-0.0008}$
& $-0.0003^{+0.0007}_{-0.0007}$
& $0.0006^{+0.0006}_{-0.0006}$
& $0.0006^{+0.0005}_{-0.0005}$
& $0.0000^{+0.0004}_{-0.0004}$
\\
${\eta_X}/\eta_{\hat{V}-1}$
& $ -1.17^{+  1.20}_{ -2.44}$
& $ -1.01^{+  0.76}_{ -1.89}$
&  \ldots
& $ -0.03^{+  0.33}_{ -0.19}$
& $ -0.78^{+  0.51}_{ -0.78}$
\\
$\gamma^0_X$
& $ 71.40^{+  3.29}_{ -3.14}$
& $ 58.39^{+  2.79}_{ -2.62}$
& $ 43.99^{+  2.27}_{ -2.12}$
& $ 33.29^{+  1.87}_{ -1.72}$
& $ 24.81^{+  1.64}_{ -1.53}$
\\
${\gamma^0_X}/\gamma^0_{\hat{V}-1}$
& $  0.62^{+  0.04}_{ -0.04}$
& $  0.33^{+  0.02}_{ -0.02}$
& \ldots
& $ -0.24^{+  0.01}_{ -0.01}$
& $ -0.44^{+  0.02}_{ -0.02}$
\\
$\gamma^1_X$
& $  9.83^{+  5.44}_{ -4.90}$
& $  2.34^{+  4.44}_{ -4.18}$
& $ -4.58^{+  3.40}_{ -3.32}$
& $ -2.41^{+  2.72}_{ -2.63}$
& $ -5.77^{+  2.22}_{ -2.15}$
\\
${\gamma^1_X}/\gamma^1_{\hat{V}-1}$
& $ -2.62^{+  1.37}_{ -4.02}$
& $ -1.25^{+  0.68}_{ -1.99}$
& \ldots
& $ -0.43^{+  0.12}_{ -0.32}$
& $  0.15^{+  0.88}_{ -0.37}$
\\
$\sigma_p \phi_X$
& $-0.007^{+ 0.012}_{-0.012}$
& $-0.039^{+ 0.010}_{-0.010}$
& $-0.039^{+ 0.009}_{-0.008}$
& $-0.026^{+ 0.007}_{-0.007}$
& $ 0.018^{+ 0.007}_{-0.007}$
\\
${\phi_X/\phi_{\hat{V}}-1}$
& $-0.804^{+ 0.245}_{-0.342}$
& $-0.009^{+ 0.095}_{-0.109}$
& \ldots
& $-0.328^{+ 0.047}_{-0.066}$
& $-1.465^{+ 0.220}_{-0.349}$
\\\hline
\end{tabular}
\caption{68\% credible intervals for the global fit parameters of the Three Color Parameter Model~II in \S\ref{modelII:sec}.\label{global2:tab}}
\end{table}
%
%\section{Covariance Matrix for Model~III Global Parameters}
%\added{The following is the  covariance matrix for the global cofactors $\{ \alpha_{\hat{U}}, \beta_{\hat{U}}, \eta_{\hat{U}}, \zeta_{\hat{U}}, \gamma^0_{\hat{U}},\gamma^1_{\hat{U}}, \sigma_p \phi_{\hat{U}}, 
%\ldots,
%\alpha_{\hat{I}}, \beta_{\hat{I}}, \eta_{\hat{I}}, \zeta_{\hat{I}}, \gamma^0_{\hat{I}},\gamma^1_{\hat{I}}, \sigma_p \phi_{\hat{I}}
% \}$.  The matrix 35x35 is big so will have to go in supplemental material.}
%\setcounter{MaxMatrixCols}{40}
%\begin{equation}
%\tiny
%\begin{bmatrix}
%5.5\times 10^{-7} & -1.8\times 10^{-6} & 3.9\times 10^{-7} & -1.6\times 10^{-5} & 2.1\times 10^{-4} & -2.6\times 10^{-4} & 3.6\times 10^{-6} & 4.5\times 10^{-7} & -1.5\times 10^{-6} & 3.1\times 10^{-7} & -1.3\times 10^{-5} & 2.6\times 10^{-4} & -1.3\times 10^{-4} & 2.8\times 10^{-6} & 3.6\times 10^{-7} & -1.2\times 10^{-6} & 2.4\times 10^{-7} & -1.0\times 10^{-5} & 1.9\times 10^{-4} & -5.5\times 10^{-5} & 1.7\times 10^{-6} & 2.9\times 10^{-7} & -9.2\times 10^{-7} & 1.9\times 10^{-7} & -8.1\times 10^{-6} & 1.1\times 10^{-4} & -1.7\times 10^{-5} & 1.1\times 10^{-6} & 2.2\times 10^{-7} & -7.2\times 10^{-7} & 1.4\times 10^{-7} & -6.3\times 10^{-6} & -3.6\times 10^{-5} & -6.7\times 10^{-6} & 4.7\times 10^{-7} \\
%-1.8\times 10^{-6} & 2.7\times 10^{-5} & -2.6\times 10^{-6} & 1.9\times 10^{-4} & -4.0\times 10^{-3} & -3.0\times 10^{-3} & -1.8\times 10^{-5} & -1.4\times 10^{-6} & 2.2\times 10^{-5} & -2.1\times 10^{-6} & 1.5\times 10^{-4} & -3.2\times 10^{-3} & -2.7\times 10^{-3} & -1.4\times 10^{-5} & -1.1\times 10^{-6} & 1.7\times 10^{-5} & -1.6\times 10^{-6} & 1.2\times 10^{-4} & -2.6\times 10^{-3} & -2.1\times 10^{-3} & -1.0\times 10^{-5} & -9.2\times 10^{-7} & 1.4\times 10^{-5} & -1.2\times 10^{-6} & 9.4\times 10^{-5} & -2.0\times 10^{-3} & -1.6\times 10^{-3} & -7.9\times 10^{-6} & -7.3\times 10^{-7} & 1.1\times 10^{-5} & -8.8\times 10^{-7} & 7.6\times 10^{-5} & -1.1\times 10^{-3} & -1.2\times 10^{-3} & -5.2\times 10^{-6} \\
%3.9\times 10^{-7} & -2.6\times 10^{-6} & 1.3\times 10^{-6} & -2.2\times 10^{-5} & 3.9\times 10^{-4} & 1.7\times 10^{-3} & 1.0\times 10^{-5} & 3.2\times 10^{-7} & -2.1\times 10^{-6} & 1.0\times 10^{-6} & -1.8\times 10^{-5} & 5.0\times 10^{-4} & 1.4\times 10^{-3} & 8.3\times 10^{-6} & 2.4\times 10^{-7} & -1.5\times 10^{-6} & 7.8\times 10^{-7} & -1.3\times 10^{-5} & 5.4\times 10^{-4} & 1.0\times 10^{-3} & 6.0\times 10^{-6} & 1.9\times 10^{-7} & -1.1\times 10^{-6} & 6.1\times 10^{-7} & -1.0\times 10^{-5} & 4.1\times 10^{-4} & 8.0\times 10^{-4} & 4.6\times 10^{-6} & 1.4\times 10^{-7} & -8.1\times 10^{-7} & 4.7\times 10^{-7} & -7.4\times 10^{-6} & 1.6\times 10^{-4} & 5.9\times 10^{-4} & 3.1\times 10^{-6} \\
%-1.6\times 10^{-5} & 1.9\times 10^{-4} & -2.2\times 10^{-5} & 1.6\times 10^{-3} & -2.9\times 10^{-2} & -1.4\times 10^{-2} & -2.7\times 10^{-4} & -1.3\times 10^{-5} & 1.5\times 10^{-4} & -1.8\times 10^{-5} & 1.3\times 10^{-3} & -2.8\times 10^{-2} & -1.3\times 10^{-2} & -2.1\times 10^{-4} & -1.0\times 10^{-5} & 1.2\times 10^{-4} & -1.4\times 10^{-5} & 9.9\times 10^{-4} & -2.3\times 10^{-2} & -1.0\times 10^{-2} & -1.5\times 10^{-4} & -8.3\times 10^{-6} & 9.5\times 10^{-5} & -1.1\times 10^{-5} & 7.9\times 10^{-4} & -1.8\times 10^{-2} & -7.7\times 10^{-3} & -1.2\times 10^{-4} & -6.4\times 10^{-6} & 7.7\times 10^{-5} & -7.7\times 10^{-6} & 6.3\times 10^{-4} & -8.1\times 10^{-3} & -4.9\times 10^{-3} & -7.9\times 10^{-5} \\
%2.1\times 10^{-4} & -4.0\times 10^{-3} & 3.9\times 10^{-4} & -2.9\times 10^{-2} & 6.9\times 10^{+0} & 1.3\times 10^{+0} & 6.7\times 10^{-3} & 1.4\times 10^{-4} & -3.0\times 10^{-3} & 3.2\times 10^{-4} & -2.2\times 10^{-2} & 5.5\times 10^{+0} & 1.1\times 10^{+0} & 5.1\times 10^{-3} & 9.6\times 10^{-5} & -2.2\times 10^{-3} & 2.3\times 10^{-4} & -1.6\times 10^{-2} & 4.5\times 10^{+0} & 9.7\times 10^{-1} & 3.7\times 10^{-3} & 6.0\times 10^{-5} & -1.7\times 10^{-3} & 1.7\times 10^{-4} & -1.3\times 10^{-2} & 3.7\times 10^{+0} & 8.5\times 10^{-1} & 3.0\times 10^{-3} & 3.9\times 10^{-5} & -1.4\times 10^{-3} & 1.3\times 10^{-4} & -1.1\times 10^{-2} & 2.9\times 10^{+0} & 6.7\times 10^{-1} & 2.1\times 10^{-3} \\
%-2.6\times 10^{-4} & -3.0\times 10^{-3} & 1.7\times 10^{-3} & -1.4\times 10^{-2} & 1.3\times 10^{+0} & 1.7\times 10^{+1} & 8.7\times 10^{-3} & -1.9\times 10^{-4} & -2.8\times 10^{-3} & 1.3\times 10^{-3} & -1.3\times 10^{-2} & 1.1\times 10^{+0} & 1.4\times 10^{+1} & 7.1\times 10^{-3} & -1.7\times 10^{-4} & -1.5\times 10^{-3} & 9.7\times 10^{-4} & -6.8\times 10^{-3} & 1.8\times 10^{+0} & 1.1\times 10^{+1} & 4.9\times 10^{-3} & -1.2\times 10^{-4} & -1.0\times 10^{-3} & 7.6\times 10^{-4} & -4.4\times 10^{-3} & 1.6\times 10^{+0} & 9.6\times 10^{+0} & 3.2\times 10^{-3} & -10.0\times 10^{-5} & -3.8\times 10^{-4} & 5.6\times 10^{-4} & -7.6\times 10^{-4} & 1.4\times 10^{+0} & 7.9\times 10^{+0} & 1.8\times 10^{-3} \\
%3.6\times 10^{-6} & -1.8\times 10^{-5} & 1.0\times 10^{-5} & -2.7\times 10^{-4} & 6.7\times 10^{-3} & 8.7\times 10^{-3} & 2.8\times 10^{-4} & 2.9\times 10^{-6} & -1.4\times 10^{-5} & 8.4\times 10^{-6} & -2.1\times 10^{-4} & 1.1\times 10^{-2} & 6.8\times 10^{-3} & 2.3\times 10^{-4} & 2.0\times 10^{-6} & -9.4\times 10^{-6} & 6.4\times 10^{-6} & -1.6\times 10^{-4} & 1.1\times 10^{-2} & 4.1\times 10^{-3} & 1.8\times 10^{-4} & 1.4\times 10^{-6} & -7.4\times 10^{-6} & 4.9\times 10^{-6} & -1.2\times 10^{-4} & 8.0\times 10^{-3} & 3.0\times 10^{-3} & 1.4\times 10^{-4} & 9.2\times 10^{-7} & -7.2\times 10^{-6} & 3.6\times 10^{-6} & -9.9\times 10^{-5} & 2.6\times 10^{-3} & 2.3\times 10^{-3} & 1.1\times 10^{-4} \\
%4.5\times 10^{-7} & -1.4\times 10^{-6} & 3.2\times 10^{-7} & -1.3\times 10^{-5} & 1.4\times 10^{-4} & -1.9\times 10^{-4} & 2.9\times 10^{-6} & 4.0\times 10^{-7} & -1.3\times 10^{-6} & 2.7\times 10^{-7} & -1.1\times 10^{-5} & 2.0\times 10^{-4} & -2.5\times 10^{-5} & 2.3\times 10^{-6} & 3.2\times 10^{-7} & -1.0\times 10^{-6} & 2.1\times 10^{-7} & -9.1\times 10^{-6} & 1.4\times 10^{-4} & 4.3\times 10^{-5} & 1.3\times 10^{-6} & 2.6\times 10^{-7} & -8.2\times 10^{-7} & 1.7\times 10^{-7} & -7.1\times 10^{-6} & 7.4\times 10^{-5} & 5.1\times 10^{-5} & 6.9\times 10^{-7} & 1.9\times 10^{-7} & -6.0\times 10^{-7} & 1.2\times 10^{-7} & -5.3\times 10^{-6} & -5.2\times 10^{-5} & 9.5\times 10^{-6} & 2.4\times 10^{-7} \\
%-1.5\times 10^{-6} & 2.2\times 10^{-5} & -2.1\times 10^{-6} & 1.5\times 10^{-4} & -3.0\times 10^{-3} & -2.8\times 10^{-3} & -1.4\times 10^{-5} & -1.3\times 10^{-6} & 1.9\times 10^{-5} & -1.8\times 10^{-6} & 1.3\times 10^{-4} & -2.4\times 10^{-3} & -3.0\times 10^{-3} & -1.2\times 10^{-5} & -1.0\times 10^{-6} & 1.6\times 10^{-5} & -1.4\times 10^{-6} & 1.0\times 10^{-4} & -2.0\times 10^{-3} & -2.6\times 10^{-3} & -7.6\times 10^{-6} & -8.4\times 10^{-7} & 1.2\times 10^{-5} & -1.0\times 10^{-6} & 8.4\times 10^{-5} & -1.5\times 10^{-3} & -2.0\times 10^{-3} & -5.6\times 10^{-6} & -6.2\times 10^{-7} & 9.3\times 10^{-6} & -7.7\times 10^{-7} & 6.3\times 10^{-5} & -7.7\times 10^{-4} & -1.1\times 10^{-3} & -4.0\times 10^{-6} \\
%3.1\times 10^{-7} & -2.1\times 10^{-6} & 1.0\times 10^{-6} & -1.8\times 10^{-5} & 3.2\times 10^{-4} & 1.3\times 10^{-3} & 8.4\times 10^{-6} & 2.7\times 10^{-7} & -1.8\times 10^{-6} & 8.5\times 10^{-7} & -1.5\times 10^{-5} & 4.1\times 10^{-4} & 1.1\times 10^{-3} & 6.7\times 10^{-6} & 2.0\times 10^{-7} & -1.2\times 10^{-6} & 6.7\times 10^{-7} & -1.1\times 10^{-5} & 4.4\times 10^{-4} & 8.2\times 10^{-4} & 4.9\times 10^{-6} & 1.6\times 10^{-7} & -9.4\times 10^{-7} & 5.2\times 10^{-7} & -8.5\times 10^{-6} & 3.4\times 10^{-4} & 6.4\times 10^{-4} & 3.7\times 10^{-6} & 1.1\times 10^{-7} & -6.9\times 10^{-7} & 3.8\times 10^{-7} & -6.2\times 10^{-6} & 1.3\times 10^{-4} & 4.8\times 10^{-4} & 2.6\times 10^{-6} \\
%-1.3\times 10^{-5} & 1.5\times 10^{-4} & -1.8\times 10^{-5} & 1.3\times 10^{-3} & -2.2\times 10^{-2} & -1.3\times 10^{-2} & -2.1\times 10^{-4} & -1.1\times 10^{-5} & 1.3\times 10^{-4} & -1.5\times 10^{-5} & 1.1\times 10^{-3} & -2.2\times 10^{-2} & -1.3\times 10^{-2} & -1.7\times 10^{-4} & -9.0\times 10^{-6} & 1.0\times 10^{-4} & -1.2\times 10^{-5} & 8.7\times 10^{-4} & -1.8\times 10^{-2} & -1.1\times 10^{-2} & -1.2\times 10^{-4} & -7.2\times 10^{-6} & 8.3\times 10^{-5} & -8.9\times 10^{-6} & 6.9\times 10^{-4} & -1.4\times 10^{-2} & -8.3\times 10^{-3} & -9.2\times 10^{-5} & -5.3\times 10^{-6} & 6.2\times 10^{-5} & -6.6\times 10^{-6} & 5.1\times 10^{-4} & -6.0\times 10^{-3} & -4.1\times 10^{-3} & -6.7\times 10^{-5} \\
%2.6\times 10^{-4} & -3.2\times 10^{-3} & 5.0\times 10^{-4} & -2.8\times 10^{-2} & 5.5\times 10^{+0} & 1.1\times 10^{+0} & 1.1\times 10^{-2} & 2.0\times 10^{-4} & -2.4\times 10^{-3} & 4.1\times 10^{-4} & -2.2\times 10^{-2} & 5.1\times 10^{+0} & 9.4\times 10^{-1} & 8.4\times 10^{-3} & 1.4\times 10^{-4} & -1.8\times 10^{-3} & 3.0\times 10^{-4} & -1.6\times 10^{-2} & 4.3\times 10^{+0} & 8.6\times 10^{-1} & 6.0\times 10^{-3} & 9.8\times 10^{-5} & -1.4\times 10^{-3} & 2.3\times 10^{-4} & -1.3\times 10^{-2} & 3.4\times 10^{+0} & 7.8\times 10^{-1} & 4.6\times 10^{-3} & 5.8\times 10^{-5} & -1.1\times 10^{-3} & 1.7\times 10^{-4} & -1.0\times 10^{-2} & 2.4\times 10^{+0} & 6.5\times 10^{-1} & 3.2\times 10^{-3} \\
%-1.3\times 10^{-4} & -2.7\times 10^{-3} & 1.4\times 10^{-3} & -1.3\times 10^{-2} & 1.1\times 10^{+0} & 1.4\times 10^{+1} & 6.8\times 10^{-3} & -2.5\times 10^{-5} & -3.0\times 10^{-3} & 1.1\times 10^{-3} & -1.3\times 10^{-2} & 9.4\times 10^{-1} & 1.4\times 10^{+1} & 6.4\times 10^{-3} & 1.8\times 10^{-5} & -2.2\times 10^{-3} & 8.6\times 10^{-4} & -1.1\times 10^{-2} & 1.5\times 10^{+0} & 1.2\times 10^{+1} & 1.6\times 10^{-3} & 2.0\times 10^{-5} & -1.6\times 10^{-3} & 6.7\times 10^{-4} & -7.7\times 10^{-3} & 1.3\times 10^{+0} & 9.5\times 10^{+0} & 3.9\times 10^{-6} & -2.3\times 10^{-5} & -6.0\times 10^{-4} & 5.1\times 10^{-4} & -3.1\times 10^{-3} & 1.1\times 10^{+0} & 7.1\times 10^{+0} & -1.3\times 10^{-3} \\
%2.8\times 10^{-6} & -1.4\times 10^{-5} & 8.3\times 10^{-6} & -2.1\times 10^{-4} & 5.1\times 10^{-3} & 7.1\times 10^{-3} & 2.3\times 10^{-4} & 2.3\times 10^{-6} & -1.2\times 10^{-5} & 6.7\times 10^{-6} & -1.7\times 10^{-4} & 8.4\times 10^{-3} & 6.4\times 10^{-3} & 1.9\times 10^{-4} & 1.6\times 10^{-6} & -8.8\times 10^{-6} & 5.2\times 10^{-6} & -1.4\times 10^{-4} & 8.6\times 10^{-3} & 4.5\times 10^{-3} & 1.5\times 10^{-4} & 1.1\times 10^{-6} & -7.0\times 10^{-6} & 4.0\times 10^{-6} & -1.1\times 10^{-4} & 6.3\times 10^{-3} & 3.2\times 10^{-3} & 1.2\times 10^{-4} & 6.7\times 10^{-7} & -6.3\times 10^{-6} & 3.0\times 10^{-6} & -8.3\times 10^{-5} & 2.1\times 10^{-3} & 2.0\times 10^{-3} & 8.9\times 10^{-5} \\
%3.6\times 10^{-7} & -1.1\times 10^{-6} & 2.4\times 10^{-7} & -1.0\times 10^{-5} & 9.6\times 10^{-5} & -1.7\times 10^{-4} & 2.0\times 10^{-6} & 3.2\times 10^{-7} & -1.0\times 10^{-6} & 2.0\times 10^{-7} & -9.0\times 10^{-6} & 1.4\times 10^{-4} & 1.8\times 10^{-5} & 1.6\times 10^{-6} & 2.8\times 10^{-7} & -9.3\times 10^{-7} & 1.7\times 10^{-7} & -7.8\times 10^{-6} & 9.4\times 10^{-5} & 1.0\times 10^{-4} & 6.8\times 10^{-7} & 2.3\times 10^{-7} & -7.4\times 10^{-7} & 1.3\times 10^{-7} & -6.2\times 10^{-6} & 3.6\times 10^{-5} & 8.4\times 10^{-5} & 2.4\times 10^{-7} & 1.7\times 10^{-7} & -5.5\times 10^{-7} & 9.8\times 10^{-8} & -4.6\times 10^{-6} & -6.3\times 10^{-5} & 3.2\times 10^{-5} & -5.5\times 10^{-8} \\
%-1.2\times 10^{-6} & 1.7\times 10^{-5} & -1.5\times 10^{-6} & 1.2\times 10^{-4} & -2.2\times 10^{-3} & -1.5\times 10^{-3} & -9.4\times 10^{-6} & -1.0\times 10^{-6} & 1.6\times 10^{-5} & -1.2\times 10^{-6} & 1.0\times 10^{-4} & -1.8\times 10^{-3} & -2.2\times 10^{-3} & -8.8\times 10^{-6} & -9.3\times 10^{-7} & 1.4\times 10^{-5} & -9.7\times 10^{-7} & 9.4\times 10^{-5} & -1.3\times 10^{-3} & -2.2\times 10^{-3} & -5.0\times 10^{-6} & -7.6\times 10^{-7} & 1.1\times 10^{-5} & -7.5\times 10^{-7} & 7.6\times 10^{-5} & -9.6\times 10^{-4} & -1.6\times 10^{-3} & -3.4\times 10^{-6} & -5.7\times 10^{-7} & 8.7\times 10^{-6} & -5.6\times 10^{-7} & 5.8\times 10^{-5} & -4.2\times 10^{-4} & -9.0\times 10^{-4} & -2.8\times 10^{-6} \\
%2.4\times 10^{-7} & -1.6\times 10^{-6} & 7.8\times 10^{-7} & -1.4\times 10^{-5} & 2.3\times 10^{-4} & 9.7\times 10^{-4} & 6.4\times 10^{-6} & 2.1\times 10^{-7} & -1.4\times 10^{-6} & 6.7\times 10^{-7} & -1.2\times 10^{-5} & 3.0\times 10^{-4} & 8.6\times 10^{-4} & 5.2\times 10^{-6} & 1.7\times 10^{-7} & -9.7\times 10^{-7} & 5.5\times 10^{-7} & -8.8\times 10^{-6} & 3.3\times 10^{-4} & 6.6\times 10^{-4} & 3.7\times 10^{-6} & 1.3\times 10^{-7} & -7.4\times 10^{-7} & 4.3\times 10^{-7} & -6.8\times 10^{-6} & 2.5\times 10^{-4} & 5.1\times 10^{-4} & 2.8\times 10^{-6} & 9.4\times 10^{-8} & -5.4\times 10^{-7} & 3.2\times 10^{-7} & -5.0\times 10^{-6} & 9.2\times 10^{-5} & 3.6\times 10^{-4} & 2.0\times 10^{-6} \\
%-1.0\times 10^{-5} & 1.2\times 10^{-4} & -1.3\times 10^{-5} & 9.9\times 10^{-4} & -1.6\times 10^{-2} & -6.8\times 10^{-3} & -1.6\times 10^{-4} & -9.1\times 10^{-6} & 1.0\times 10^{-4} & -1.1\times 10^{-5} & 8.7\times 10^{-4} & -1.6\times 10^{-2} & -1.1\times 10^{-2} & -1.4\times 10^{-4} & -7.8\times 10^{-6} & 9.4\times 10^{-5} & -8.8\times 10^{-6} & 7.8\times 10^{-4} & -1.4\times 10^{-2} & -1.0\times 10^{-2} & -9.2\times 10^{-5} & -6.3\times 10^{-6} & 7.6\times 10^{-5} & -6.8\times 10^{-6} & 6.2\times 10^{-4} & -1.0\times 10^{-2} & -7.0\times 10^{-3} & -6.7\times 10^{-5} & -4.7\times 10^{-6} & 5.8\times 10^{-5} & -5.1\times 10^{-6} & 4.7\times 10^{-4} & -4.0\times 10^{-3} & -3.9\times 10^{-3} & -5.1\times 10^{-5} \\
%1.9\times 10^{-4} & -2.6\times 10^{-3} & 5.4\times 10^{-4} & -2.3\times 10^{-2} & 4.5\times 10^{+0} & 1.8\times 10^{+0} & 1.1\times 10^{-2} & 1.4\times 10^{-4} & -2.0\times 10^{-3} & 4.4\times 10^{-4} & -1.8\times 10^{-2} & 4.3\times 10^{+0} & 1.5\times 10^{+0} & 8.6\times 10^{-3} & 9.4\times 10^{-5} & -1.3\times 10^{-3} & 3.3\times 10^{-4} & -1.4\times 10^{-2} & 4.0\times 10^{+0} & 1.2\times 10^{+0} & 6.5\times 10^{-3} & 5.8\times 10^{-5} & -9.9\times 10^{-4} & 2.5\times 10^{-4} & -1.1\times 10^{-2} & 3.2\times 10^{+0} & 1.0\times 10^{+0} & 5.0\times 10^{-3} & 2.6\times 10^{-5} & -7.8\times 10^{-4} & 1.8\times 10^{-4} & -8.2\times 10^{-3} & 2.2\times 10^{+0} & 8.9\times 10^{-1} & 3.6\times 10^{-3} \\
%-5.5\times 10^{-5} & -2.1\times 10^{-3} & 1.0\times 10^{-3} & -1.0\times 10^{-2} & 9.7\times 10^{-1} & 1.1\times 10^{+1} & 4.1\times 10^{-3} & 4.3\times 10^{-5} & -2.6\times 10^{-3} & 8.2\times 10^{-4} & -1.1\times 10^{-2} & 8.6\times 10^{-1} & 1.2\times 10^{+1} & 4.5\times 10^{-3} & 1.0\times 10^{-4} & -2.2\times 10^{-3} & 6.6\times 10^{-4} & -1.0\times 10^{-2} & 1.2\times 10^{+0} & 1.0\times 10^{+1} & -3.8\times 10^{-4} & 8.4\times 10^{-5} & -1.6\times 10^{-3} & 5.2\times 10^{-4} & -7.3\times 10^{-3} & 1.1\times 10^{+0} & 8.6\times 10^{+0} & -1.6\times 10^{-3} & 2.1\times 10^{-5} & -6.0\times 10^{-4} & 4.0\times 10^{-4} & -3.1\times 10^{-3} & 8.8\times 10^{-1} & 6.4\times 10^{+0} & -2.5\times 10^{-3} \\
%1.7\times 10^{-6} & -1.0\times 10^{-5} & 6.0\times 10^{-6} & -1.5\times 10^{-4} & 3.7\times 10^{-3} & 4.9\times 10^{-3} & 1.8\times 10^{-4} & 1.3\times 10^{-6} & -7.6\times 10^{-6} & 4.9\times 10^{-6} & -1.2\times 10^{-4} & 6.0\times 10^{-3} & 1.6\times 10^{-3} & 1.5\times 10^{-4} & 6.8\times 10^{-7} & -5.0\times 10^{-6} & 3.7\times 10^{-6} & -9.2\times 10^{-5} & 6.5\times 10^{-3} & -3.8\times 10^{-4} & 1.3\times 10^{-4} & 3.9\times 10^{-7} & -4.1\times 10^{-6} & 2.9\times 10^{-6} & -7.2\times 10^{-5} & 4.9\times 10^{-3} & -3.9\times 10^{-4} & 1.1\times 10^{-4} & 2.1\times 10^{-7} & -4.4\times 10^{-6} & 2.1\times 10^{-6} & -5.8\times 10^{-5} & 1.9\times 10^{-3} & 6.8\times 10^{-4} & 8.3\times 10^{-5} \\
%2.9\times 10^{-7} & -9.2\times 10^{-7} & 1.9\times 10^{-7} & -8.3\times 10^{-6} & 6.0\times 10^{-5} & -1.2\times 10^{-4} & 1.4\times 10^{-6} & 2.6\times 10^{-7} & -8.4\times 10^{-7} & 1.6\times 10^{-7} & -7.2\times 10^{-6} & 9.8\times 10^{-5} & 2.0\times 10^{-5} & 1.1\times 10^{-6} & 2.3\times 10^{-7} & -7.6\times 10^{-7} & 1.3\times 10^{-7} & -6.3\times 10^{-6} & 5.8\times 10^{-5} & 8.4\times 10^{-5} & 3.9\times 10^{-7} & 1.9\times 10^{-7} & -6.2\times 10^{-7} & 1.0\times 10^{-7} & -5.0\times 10^{-6} & 1.2\times 10^{-5} & 7.2\times 10^{-5} & 4.6\times 10^{-8} & 1.4\times 10^{-7} & -4.6\times 10^{-7} & 8.0\times 10^{-8} & -3.9\times 10^{-6} & -6.4\times 10^{-5} & 3.0\times 10^{-5} & -1.9\times 10^{-7} \\
%-9.2\times 10^{-7} & 1.4\times 10^{-5} & -1.1\times 10^{-6} & 9.5\times 10^{-5} & -1.7\times 10^{-3} & -1.0\times 10^{-3} & -7.4\times 10^{-6} & -8.2\times 10^{-7} & 1.2\times 10^{-5} & -9.4\times 10^{-7} & 8.3\times 10^{-5} & -1.4\times 10^{-3} & -1.6\times 10^{-3} & -7.0\times 10^{-6} & -7.4\times 10^{-7} & 1.1\times 10^{-5} & -7.4\times 10^{-7} & 7.6\times 10^{-5} & -9.9\times 10^{-4} & -1.6\times 10^{-3} & -4.1\times 10^{-6} & -6.2\times 10^{-7} & 9.5\times 10^{-6} & -5.8\times 10^{-7} & 6.3\times 10^{-5} & -7.1\times 10^{-4} & -1.1\times 10^{-3} & -3.0\times 10^{-6} & -4.8\times 10^{-7} & 7.5\times 10^{-6} & -4.4\times 10^{-7} & 5.0\times 10^{-5} & -3.0\times 10^{-4} & -6.5\times 10^{-4} & -2.4\times 10^{-6} \\
%1.9\times 10^{-7} & -1.2\times 10^{-6} & 6.1\times 10^{-7} & -1.1\times 10^{-5} & 1.7\times 10^{-4} & 7.6\times 10^{-4} & 4.9\times 10^{-6} & 1.7\times 10^{-7} & -1.0\times 10^{-6} & 5.2\times 10^{-7} & -8.9\times 10^{-6} & 2.3\times 10^{-4} & 6.7\times 10^{-4} & 4.0\times 10^{-6} & 1.3\times 10^{-7} & -7.5\times 10^{-7} & 4.3\times 10^{-7} & -6.8\times 10^{-6} & 2.5\times 10^{-4} & 5.2\times 10^{-4} & 2.9\times 10^{-6} & 1.0\times 10^{-7} & -5.8\times 10^{-7} & 3.5\times 10^{-7} & -5.3\times 10^{-6} & 1.9\times 10^{-4} & 4.1\times 10^{-4} & 2.2\times 10^{-6} & 7.7\times 10^{-8} & -4.2\times 10^{-7} & 2.6\times 10^{-7} & -4.0\times 10^{-6} & 7.3\times 10^{-5} & 2.8\times 10^{-4} & 1.6\times 10^{-6} \\
%-8.1\times 10^{-6} & 9.4\times 10^{-5} & -1.0\times 10^{-5} & 7.9\times 10^{-4} & -1.3\times 10^{-2} & -4.4\times 10^{-3} & -1.2\times 10^{-4} & -7.1\times 10^{-6} & 8.4\times 10^{-5} & -8.5\times 10^{-6} & 6.9\times 10^{-4} & -1.3\times 10^{-2} & -7.7\times 10^{-3} & -1.1\times 10^{-4} & -6.2\times 10^{-6} & 7.6\times 10^{-5} & -6.8\times 10^{-6} & 6.2\times 10^{-4} & -1.1\times 10^{-2} & -7.3\times 10^{-3} & -7.2\times 10^{-5} & -5.0\times 10^{-6} & 6.3\times 10^{-5} & -5.3\times 10^{-6} & 5.1\times 10^{-4} & -7.8\times 10^{-3} & -4.8\times 10^{-3} & -5.3\times 10^{-5} & -3.9\times 10^{-6} & 5.0\times 10^{-5} & -4.0\times 10^{-6} & 4.0\times 10^{-4} & -3.2\times 10^{-3} & -2.6\times 10^{-3} & -4.0\times 10^{-5} \\
%1.1\times 10^{-4} & -2.0\times 10^{-3} & 4.1\times 10^{-4} & -1.8\times 10^{-2} & 3.7\times 10^{+0} & 1.6\times 10^{+0} & 8.0\times 10^{-3} & 7.4\times 10^{-5} & -1.5\times 10^{-3} & 3.4\times 10^{-4} & -1.4\times 10^{-2} & 3.4\times 10^{+0} & 1.3\times 10^{+0} & 6.3\times 10^{-3} & 3.6\times 10^{-5} & -9.6\times 10^{-4} & 2.5\times 10^{-4} & -1.0\times 10^{-2} & 3.2\times 10^{+0} & 1.1\times 10^{+0} & 4.9\times 10^{-3} & 1.2\times 10^{-5} & -7.1\times 10^{-4} & 1.9\times 10^{-4} & -7.8\times 10^{-3} & 2.6\times 10^{+0} & 9.3\times 10^{-1} & 3.9\times 10^{-3} & -6.0\times 10^{-6} & -5.7\times 10^{-4} & 1.4\times 10^{-4} & -6.2\times 10^{-3} & 1.9\times 10^{+0} & 8.1\times 10^{-1} & 2.9\times 10^{-3} \\
%-1.7\times 10^{-5} & -1.6\times 10^{-3} & 8.0\times 10^{-4} & -7.7\times 10^{-3} & 8.5\times 10^{-1} & 9.6\times 10^{+0} & 3.0\times 10^{-3} & 5.1\times 10^{-5} & -2.0\times 10^{-3} & 6.4\times 10^{-4} & -8.3\times 10^{-3} & 7.8\times 10^{-1} & 9.5\times 10^{+0} & 3.2\times 10^{-3} & 8.4\times 10^{-5} & -1.6\times 10^{-3} & 5.1\times 10^{-4} & -7.0\times 10^{-3} & 1.0\times 10^{+0} & 8.6\times 10^{+0} & -3.9\times 10^{-4} & 7.2\times 10^{-5} & -1.1\times 10^{-3} & 4.1\times 10^{-4} & -4.8\times 10^{-3} & 9.3\times 10^{-1} & 7.3\times 10^{+0} & -1.4\times 10^{-3} & 2.3\times 10^{-5} & -3.9\times 10^{-4} & 3.1\times 10^{-4} & -1.6\times 10^{-3} & 7.8\times 10^{-1} & 5.7\times 10^{+0} & -2.1\times 10^{-3} \\
%1.1\times 10^{-6} & -7.9\times 10^{-6} & 4.6\times 10^{-6} & -1.2\times 10^{-4} & 3.0\times 10^{-3} & 3.2\times 10^{-3} & 1.4\times 10^{-4} & 6.9\times 10^{-7} & -5.6\times 10^{-6} & 3.7\times 10^{-6} & -9.2\times 10^{-5} & 4.6\times 10^{-3} & 3.9\times 10^{-6} & 1.2\times 10^{-4} & 2.4\times 10^{-7} & -3.4\times 10^{-6} & 2.8\times 10^{-6} & -6.7\times 10^{-5} & 5.0\times 10^{-3} & -1.6\times 10^{-3} & 1.1\times 10^{-4} & 4.6\times 10^{-8} & -3.0\times 10^{-6} & 2.2\times 10^{-6} & -5.3\times 10^{-5} & 3.9\times 10^{-3} & -1.4\times 10^{-3} & 8.9\times 10^{-5} & -3.2\times 10^{-8} & -3.6\times 10^{-6} & 1.6\times 10^{-6} & -4.5\times 10^{-5} & 1.7\times 10^{-3} & -4.8\times 10^{-5} & 7.2\times 10^{-5} \\
%2.2\times 10^{-7} & -7.3\times 10^{-7} & 1.4\times 10^{-7} & -6.4\times 10^{-6} & 3.9\times 10^{-5} & -10.0\times 10^{-5} & 9.2\times 10^{-7} & 1.9\times 10^{-7} & -6.2\times 10^{-7} & 1.1\times 10^{-7} & -5.3\times 10^{-6} & 5.8\times 10^{-5} & -2.3\times 10^{-5} & 6.7\times 10^{-7} & 1.7\times 10^{-7} & -5.7\times 10^{-7} & 9.4\times 10^{-8} & -4.7\times 10^{-6} & 2.6\times 10^{-5} & 2.1\times 10^{-5} & 2.1\times 10^{-7} & 1.4\times 10^{-7} & -4.8\times 10^{-7} & 7.7\times 10^{-8} & -3.9\times 10^{-6} & -6.0\times 10^{-6} & 2.3\times 10^{-5} & -3.2\times 10^{-8} & 1.3\times 10^{-7} & -4.2\times 10^{-7} & 6.6\times 10^{-8} & -3.3\times 10^{-6} & -5.8\times 10^{-5} & 3.4\times 10^{-5} & -2.4\times 10^{-7} \\
%-7.2\times 10^{-7} & 1.1\times 10^{-5} & -8.1\times 10^{-7} & 7.7\times 10^{-5} & -1.4\times 10^{-3} & -3.8\times 10^{-4} & -7.2\times 10^{-6} & -6.0\times 10^{-7} & 9.3\times 10^{-6} & -6.9\times 10^{-7} & 6.2\times 10^{-5} & -1.1\times 10^{-3} & -6.0\times 10^{-4} & -6.3\times 10^{-6} & -5.5\times 10^{-7} & 8.7\times 10^{-6} & -5.4\times 10^{-7} & 5.8\times 10^{-5} & -7.8\times 10^{-4} & -6.0\times 10^{-4} & -4.4\times 10^{-6} & -4.6\times 10^{-7} & 7.5\times 10^{-6} & -4.2\times 10^{-7} & 5.0\times 10^{-5} & -5.7\times 10^{-4} & -3.9\times 10^{-4} & -3.6\times 10^{-6} & -4.2\times 10^{-7} & 7.0\times 10^{-6} & -3.3\times 10^{-7} & 4.6\times 10^{-5} & -2.7\times 10^{-4} & -3.8\times 10^{-4} & -2.8\times 10^{-6} \\
%1.4\times 10^{-7} & -8.8\times 10^{-7} & 4.7\times 10^{-7} & -7.7\times 10^{-6} & 1.3\times 10^{-4} & 5.6\times 10^{-4} & 3.6\times 10^{-6} & 1.2\times 10^{-7} & -7.7\times 10^{-7} & 3.8\times 10^{-7} & -6.6\times 10^{-6} & 1.7\times 10^{-4} & 5.1\times 10^{-4} & 3.0\times 10^{-6} & 9.8\times 10^{-8} & -5.6\times 10^{-7} & 3.2\times 10^{-7} & -5.1\times 10^{-6} & 1.8\times 10^{-4} & 4.0\times 10^{-4} & 2.1\times 10^{-6} & 8.0\times 10^{-8} & -4.4\times 10^{-7} & 2.6\times 10^{-7} & -4.0\times 10^{-6} & 1.4\times 10^{-4} & 3.1\times 10^{-4} & 1.6\times 10^{-6} & 6.6\times 10^{-8} & -3.3\times 10^{-7} & 2.2\times 10^{-7} & -3.2\times 10^{-6} & 5.7\times 10^{-5} & 2.2\times 10^{-4} & 1.2\times 10^{-6} \\
%-6.3\times 10^{-6} & 7.6\times 10^{-5} & -7.4\times 10^{-6} & 6.3\times 10^{-4} & -1.1\times 10^{-2} & -7.6\times 10^{-4} & -9.9\times 10^{-5} & -5.3\times 10^{-6} & 6.3\times 10^{-5} & -6.2\times 10^{-6} & 5.1\times 10^{-4} & -1.0\times 10^{-2} & -3.1\times 10^{-3} & -8.3\times 10^{-5} & -4.6\times 10^{-6} & 5.8\times 10^{-5} & -5.0\times 10^{-6} & 4.7\times 10^{-4} & -8.2\times 10^{-3} & -3.1\times 10^{-3} & -5.8\times 10^{-5} & -3.9\times 10^{-6} & 5.0\times 10^{-5} & -4.0\times 10^{-6} & 4.0\times 10^{-4} & -6.2\times 10^{-3} & -1.6\times 10^{-3} & -4.5\times 10^{-5} & -3.3\times 10^{-6} & 4.6\times 10^{-5} & -3.2\times 10^{-6} & 3.7\times 10^{-4} & -3.0\times 10^{-3} & -1.2\times 10^{-3} & -3.3\times 10^{-5} \\
%-3.6\times 10^{-5} & -1.1\times 10^{-3} & 1.6\times 10^{-4} & -8.1\times 10^{-3} & 2.9\times 10^{+0} & 1.4\times 10^{+0} & 2.6\times 10^{-3} & -5.2\times 10^{-5} & -7.7\times 10^{-4} & 1.3\times 10^{-4} & -6.0\times 10^{-3} & 2.4\times 10^{+0} & 1.1\times 10^{+0} & 2.1\times 10^{-3} & -6.3\times 10^{-5} & -4.2\times 10^{-4} & 9.2\times 10^{-5} & -4.0\times 10^{-3} & 2.2\times 10^{+0} & 8.8\times 10^{-1} & 1.9\times 10^{-3} & -6.4\times 10^{-5} & -3.0\times 10^{-4} & 7.3\times 10^{-5} & -3.2\times 10^{-3} & 1.9\times 10^{+0} & 7.8\times 10^{-1} & 1.7\times 10^{-3} & -5.8\times 10^{-5} & -2.7\times 10^{-4} & 5.7\times 10^{-5} & -3.0\times 10^{-3} & 1.8\times 10^{+0} & 6.9\times 10^{-1} & 1.5\times 10^{-3} \\
%-6.7\times 10^{-6} & -1.2\times 10^{-3} & 5.9\times 10^{-4} & -4.9\times 10^{-3} & 6.7\times 10^{-1} & 7.9\times 10^{+0} & 2.3\times 10^{-3} & 9.5\times 10^{-6} & -1.1\times 10^{-3} & 4.8\times 10^{-4} & -4.1\times 10^{-3} & 6.5\times 10^{-1} & 7.1\times 10^{+0} & 2.0\times 10^{-3} & 3.2\times 10^{-5} & -9.0\times 10^{-4} & 3.6\times 10^{-4} & -3.9\times 10^{-3} & 8.9\times 10^{-1} & 6.4\times 10^{+0} & 6.8\times 10^{-4} & 3.0\times 10^{-5} & -6.5\times 10^{-4} & 2.8\times 10^{-4} & -2.6\times 10^{-3} & 8.1\times 10^{-1} & 5.7\times 10^{+0} & -4.8\times 10^{-5} & 3.4\times 10^{-5} & -3.8\times 10^{-4} & 2.2\times 10^{-4} & -1.2\times 10^{-3} & 6.9\times 10^{-1} & 5.4\times 10^{+0} & -6.4\times 10^{-4} \\
%4.7\times 10^{-7} & -5.2\times 10^{-6} & 3.1\times 10^{-6} & -7.9\times 10^{-5} & 2.1\times 10^{-3} & 1.8\times 10^{-3} & 1.1\times 10^{-4} & 2.4\times 10^{-7} & -4.0\times 10^{-6} & 2.6\times 10^{-6} & -6.7\times 10^{-5} & 3.2\times 10^{-3} & -1.3\times 10^{-3} & 8.9\times 10^{-5} & -5.5\times 10^{-8} & -2.8\times 10^{-6} & 2.0\times 10^{-6} & -5.1\times 10^{-5} & 3.6\times 10^{-3} & -2.5\times 10^{-3} & 8.3\times 10^{-5} & -1.9\times 10^{-7} & -2.4\times 10^{-6} & 1.6\times 10^{-6} & -4.0\times 10^{-5} & 2.9\times 10^{-3} & -2.1\times 10^{-3} & 7.2\times 10^{-5} & -2.4\times 10^{-7} & -2.8\times 10^{-6} & 1.2\times 10^{-6} & -3.3\times 10^{-5} & 1.5\times 10^{-3} & -6.4\times 10^{-4} & 6.6\times 10^{-5} \\
%\end{bmatrix}
%\end{equation}
\bibliographystyle{aasjournal}
\bibliography{alex}
%\documentclass[trackchanges]{aastex62}   	% use "amsart" instead of "article" for AMSLaTeX format
\usepackage{geometry}                		% See geometry.pdf to learn the layout options. There are lots.
\geometry{letterpaper}                   		% ... or a4paper or a5paper or ... 
\usepackage{graphicx}				% Use pdf, png, jpg, or eps§ with pdflatex; use eps in DVI mode
\usepackage{amsmath}
\usepackage{amssymb}
\usepackage{natbib}
\usepackage{lineno}
\usepackage{color}
\defcitealias{1999PASP..111...63F}{F99}
\defcitealias{2017ApJ...842...93M}{M17}
\linenumbers

\begin{document}

\title{Evidence for a Third Color  Parameter Within the Type~Ia Supernovae of the Nearby Supernova Factory}
\author[0000-0001-6315-8743]{A.~G.~Kim}
\affiliation{    Physics Division, Lawrence Berkeley National Laboratory, 
    1 Cyclotron Road, Berkeley, CA, 94720}
    
\author{     G.~Aldering}
\affiliation{    Physics Division, Lawrence Berkeley National Laboratory, 
    1 Cyclotron Road, Berkeley, CA, 94720}

\author{     P.~Antilogus}
\affiliation{    Sorbonne Universit\'e, Universit\'e Paris Diderot, CNRS/IN2P3, Laboratoire de Physique Nucl\'eaire et de Hautes Energies, LPNHE, 4 Place Jussieu, F-75252, Paris, France}
    
\author{     S.~Bailey}
\affiliation{    Physics Division, Lawrence Berkeley National Laboratory, 
    1 Cyclotron Road, Berkeley, CA, 94720}

\author{     C.~Baltay}
\affiliation{    Department of Physics, Yale University, 
    New Haven, CT, 06250-8121}

\author{     K.~Barbary}
\affiliation{
    Department of Physics, University of California Berkeley,
    366 LeConte Hall MC 7300, Berkeley, CA, 94720-7300}

\author{    D.~Baugh}
\affiliation{   Tsinghua Center for Astrophysics, Tsinghua University, Beijing 100084, China }

\author{     K.~Boone}
\affiliation{    Physics Division, Lawrence Berkeley National Laboratory, 
    1 Cyclotron Road, Berkeley, CA, 94720}
\affiliation{
    Department of Physics, University of California Berkeley,
    366 LeConte Hall MC 7300, Berkeley, CA, 94720-7300}

\author{     S.~Bongard}
\affiliation{ Sorbonne Universit\'e, Universit\'e Paris Diderot, CNRS/IN2P3, Laboratoire de Physique Nucl\'eaire et de Hautes Energies, LPNHE, 4 Place Jussieu, F-75252, Paris, France}

\author{     C.~Buton}
\affiliation{    Universit\'e de Lyon, F-69622, Lyon, France ; Universit\'e de Lyon 1, Villeurbanne ; 
    CNRS/IN2P3, Institut de Physique Nucl\'eaire de Lyon}
    
\author{     J.~Chen}
\affiliation{   Tsinghua Center for Astrophysics, Tsinghua University, Beijing 100084, China }

\author{     N.~Chotard}
\affiliation{    Universit\'e de Lyon, F-69622, Lyon, France ; Universit\'e de Lyon 1, Villeurbanne ; 
    CNRS/IN2P3, Institut de Physique Nucl\'eaire de Lyon}
    
\author[0000-0002-5317-7518]{     Y.~Copin}
\affiliation{    Universit\'e de Lyon, F-69622, Lyon, France ; Universit\'e de Lyon 1, Villeurbanne ; 
    CNRS/IN2P3, Institut de Physique Nucl\'eaire de Lyon}

\author{ S.~Dixon}
\affiliation{
    Department of Physics, University of California Berkeley,
    366 LeConte Hall MC 7300, Berkeley, CA, 94720-7300}

\author{     P.~Fagrelius}
\affiliation{    Physics Division, Lawrence Berkeley National Laboratory, 
    1 Cyclotron Road, Berkeley, CA, 94720}
\affiliation{
    Department of Physics, University of California Berkeley,
    366 LeConte Hall MC 7300, Berkeley, CA, 94720-7300}

\author{     H.~K.~Fakhouri}
\affiliation{    Physics Division, Lawrence Berkeley National Laboratory, 
    1 Cyclotron Road, Berkeley, CA, 94720}
  \affiliation{
    Department of Physics, University of California Berkeley,
    366 LeConte Hall MC 7300, Berkeley, CA, 94720-7300}

\author{     U.~Feindt}
\affiliation{The Oskar Klein Centre, Department of Physics, AlbaNova, Stockholm University, SE-106 91 Stockholm, Sweden}

\author{     D.~Fouchez}
\affiliation{ Aix Marseille Univ, CNRS/IN2P3, CPPM, Marseille, France
}
    
\author[0000-0001-6728-1423]{     E.~Gangler}  
\affiliation{    Clermont Universit\'e, Universit\'e Blaise Pascal, CNRS/IN2P3, Laboratoire de Physique Corpusculaire,
    BP 10448, F-63000 Clermont-Ferrand, France}
    
\author{     B.~Hayden}
\affiliation{    Physics Division, Lawrence Berkeley National Laboratory, 
    1 Cyclotron Road, Berkeley, CA, 94720}

\author{     W.~Hillebrandt}
\affiliation{    Max-Planck-Institut f\"ur Astrophysik, Karl-Schwarzschild-Str. 1,
D-85748 Garching, Germany}

\author{     M.~Kowalski}
\affiliation{    Institut fur Physik,  Humboldt-Universitat zu Berlin,
    Newtonstr. 15, 12489 Berlin}
\affiliation{ DESY, D-15735 Zeuthen, Germany}

\author{     P.-F.~Leget}
\affiliation{    Clermont Universit\'e, Universit\'e Blaise Pascal, CNRS/IN2P3, Laboratoire de Physique Corpusculaire,
    BP 10448, F-63000 Clermont-Ferrand, France}
    
\author{     S.~Lombardo}
\affiliation{    Institut fur Physik,  Humboldt-Universitat zu Berlin,
    Newtonstr. 15, 12489 Berlin}
    
\author{     J.~Nordin}
\affiliation{    Institut fur Physik,  Humboldt-Universitat zu Berlin,
    Newtonstr. 15, 12489 Berlin}
    
\author{     R.~Pain}
\affiliation{ Sorbonne Universit\'e, Universit\'e Paris Diderot, CNRS/IN2P3, Laboratoire de Physique Nucl\'eaire et de Hautes Energies, LPNHE, 4 Place Jussieu, F-75252, Paris, France}
     
\author{     E.~Pecontal}
\affiliation{   Centre de Recherche Astronomique de Lyon, Universit\'e Lyon 1,
    9 Avenue Charles Andr\'e, 69561 Saint Genis Laval Cedex, France}
    
\author{    R.~Pereira}
 \affiliation{    Universit\'e de Lyon, F-69622, Lyon, France ; Universit\'e de Lyon 1, Villeurbanne ; 
    CNRS/IN2P3, Institut de Physique Nucl\'eaire de Lyon}
 
 \author{    S.~Perlmutter}
 \affiliation{    Physics Division, Lawrence Berkeley National Laboratory, 
    1 Cyclotron Road, Berkeley, CA, 94720} 
\affiliation{
    Department of Physics, University of California Berkeley,
    366 LeConte Hall MC 7300, Berkeley, CA, 94720-7300}
    
 \author{    D.~Rabinowitz}
 \affiliation{    Department of Physics, Yale University, 
    New Haven, CT, 06250-8121}
    
 \author[0000-0002-8121-2560]{    M.~Rigault} 
\affiliation{ Clermont Universit\'e, Universit\'e Blaise Pascal, CNRS/IN2P3, Laboratoire de Physique Corpusculaire,
    BP 10448, F-63000 Clermont-Ferrand, France}
     
 \author{    D.~Rubin}
 \affiliation{    Physics Division, Lawrence Berkeley National Laboratory, 
    1 Cyclotron Road, Berkeley, CA, 94720}
    \affiliation{   Space Telescope Science Institute, 3700 San Martin Drive, Baltimore, MD 21218}
 
 \author{    K.~Runge}
 \affiliation{    Physics Division, Lawrence Berkeley National Laboratory, 
    1 Cyclotron Road, Berkeley, CA, 94720}
 
 \author{    C.~Saunders}
 \affiliation{    Physics Division, Lawrence Berkeley National Laboratory, 
    1 Cyclotron Road, Berkeley, CA, 94720}
 \affiliation{Sorbonne Universit\'es, Institut Lagrange de Paris (ILP), 98 bis Boulevard Arago, 75014 Paris, France}

\author{    C.~Sofiatti}
\affiliation{    Physics Division, Lawrence Berkeley National Laboratory, 
    1 Cyclotron Road, Berkeley, CA, 94720} 
\affiliation{
    Department of Physics, University of California Berkeley,
    366 LeConte Hall MC 7300, Berkeley, CA, 94720-7300}

\author{    N.~Suzuki}
\affiliation{    Physics Division, Lawrence Berkeley National Laboratory, 
    1 Cyclotron Road, Berkeley, CA, 94720}

\author{     S.~Taubenberger}
\affiliation{    Max-Planck-Institut f\"ur Astrophysik, Karl-Schwarzschild-Str. 1,
D-85748 Garching, Germany}

\author{     C.~Tao}
\affiliation{   Tsinghua Center for Astrophysics, Tsinghua University, Beijing 100084, China }
\affiliation{ Aix Marseille Univ, CNRS/IN2P3, CPPM, Marseille, France
}
   
\author{     R.~C.~Thomas}
\affiliation{    Computational Cosmology Center, Computational Research Division, Lawrence Berkeley National Laboratory, 
    1 Cyclotron Road MS 50B-4206, Berkeley, CA, 94720}
    
\collaboration{(The Nearby Supernova Factory)}


\begin{abstract}
Through empirical modeling of its observed signal, the peak absolute magnitude of a Type~Ia supernova (SN~Ia) can be accurately determined,
making SNe~Ia  excellent distance indicators.  Improved modeling of SN~Ia colors and magnitudes can account for
further physical
diversity that is expected but not included
in current models, and so consequently can lead to more precise per-object distances with smaller systematic uncertainties.  In this article, we present   
an empirical model for SN~Ia peak colors with
three latent color parameters and dependence on the equivalent widths of Ca~II and Si~II, the Si~II velocity,
and the light-curve shape.
This model is applied to the supernova sample of the Nearby Supernova Factory.  The peak magnitudes in synthetic
broadband photometry and their colors are found to be 
dependent on the observable features and on the three color parameters.
The color changes allowed by two of the color parameters are similar to those allowed by the extinction model of  \citet{1999PASP..111...63F}. 
Correspondingly the sample has an effective value of $\langle R^F_{\mathit{eff}}\rangle \sim 2.43$, though with a broad distribution of 
total-to-selective extinctions. 
We detect the influence on color by a third parameter at
%--- table.py
$> 99\%$
%-----
confidence.
The third parameter has  unique color characteristics that distinguish it from 
those already  tracked by the spectral features, light-curve shape, and two-parameter dust models.
These color parameters must be accounted for when using SNe~Ia as distance indicators.
\end{abstract}

\keywords{supernovae: general; cosmology: distance scale; methods: data analysis}

\section{Introduction}
Type~Ia supernovae (SNe~Ia) form a homogenous set of exploding stars and as such were early recognized and utilized as a powerful distance indicator 
and probe of cosmology \citep[e.g.][]{1992ARA&A..30..359B, 1993ApJ...415....1S}.  After further careful consideration of supernova data, it was recognized
that SN~Ia light-curve shapes \citep{1993ApJ...413L.105P} and colors \citep{1996ApJ...473...88R, 1998A&A...331..815T} exhibit subtle signs of heterogeneity
that are correlated with absolute magnitude, and must be considered when inferring distances.  Empirical models parameterizing SNe~Ia by their light-curve shape \citep{1996ApJ...473...88R,
1997ApJ...483..565P,
1999ApJ...517..565P}
and color  \citep{1996ApJ...473...88R}  were developed that enabled absolute magnitude corrections
and accurate distance measurements of cosmological supernovae,
which 
were subsequently used in the discovery of the accelerating expansion of the Universe \citep{1998AJ....116.1009R,1999ApJ...517..565P}.

The two most commonly used supernova-cosmology light-curve fitters today are SALT2 \citep{2007A&A...466...11G} and MLCS2k2
\citep{2007ApJ...659..122J}.\footnote{Light-curve fitters with more flexible degrees of freedom
\citep[e.g.][]{2008ApJ...681..482C, 2011AJ....141...19B, 2011ApJ...731..120M} are available and have for
the most part been used to study SN~Ia heterogeneity.}
They remain two-parameter models, with one parameter
primarily
characterizing light-curve shape and the other
color.
In SALT2 the light curve shapes are described by phase-dependent flux corrections,
 whereas MLCS2k2 varies shapes through additive magnitude corrections.
The physical cause of the color diversity is interpreted differently by the two sets of authors: 
\citet{2007A&A...466...11G} pragmatically extract color variation empirically from SNe that span a wide range of colors, with no attribution
to either intrinsic or extrinsic origins;
\citet{2007ApJ...659..122J}
attribute changes in color
partially to intrinsic variations linked to light-curve shape, and partially
to the reddening of light from host-galaxy dust.  Differences between these models produce differences in the results of
analyses of both low-redshift \citep{2007ApJ...664L..13C} and high-redshift \citep{2009ApJS..185...32K} supernovae.

There is evidence that supports the expectation that a single parameter beyond light-curve shape  cannot describe the full range
of colors seen in the SN~Ia population.  One approach to look for color diversity is to find correlations between color and spectral features.
\citet{2009ApJ...699L.139W, 2011ApJ...729...55F} find two subpopulations distinguished
by Si~II velocity that exhibit differing $B_{\mathit{\mathit{max}}}-V_{\mathit{max}}$; this color correlation, in addition to one with $B-R$, is confirmed by
\citet{2014ApJ...797...75M}.
\citet{2009ApJ...699L.139W,2015MNRAS.451.1973S}
find that high-velocity Si~II~$\lambda$6355 is found in objects that have red ultraviolet/optical colors near maximum brightness.
\citet{2011MNRAS.413.3075M} show evidence that supernova asymmetry and viewing angle,
traced by wavelength shifts in nebular emission lines, is an important determinant in controlling supernova color; such correlations are also seen by \citet{2011A&A...534L..15C}.

Another approach to probe color diversity is through multiple colors (at least 3 bands)
of individual supernovae.  Color ratios are sensitive to processes of the responsible physics.   For example,
relative dust absorption varies as a function of wavelength depending on grain size, distribution, composition and shape,
independent (to first order) of the amount of dust along the line of sight.
Near and mid-UV photometry obtained by the Ultra-Violet/Optical Telescope on the Swift spacecraft cannot be entirely explained
by dust absorption and hence imply intrinsic variability of supernova UV colors
\citep{2017ApJ...836..232B}.
Measurements of color ratios are being advanced with the development of flexible empirical light curve models that accommodate flexibility in multi-band colors
\citep[e.g.][]{2011ApJ...731..120M}.
\citet{2014ApJ...789...32B, 2015MNRAS.453.3300A} find wide
ranges of total-to-selective extinction with average values significantly lower than $R_V = 3.1$,
the canonical value for diffuse Milky Way dust.
They also confirm the \citet{2006MNRAS.369.1880E, 2011ApJ...731..120M} finding that low $R_V$ is associated with high-extinction supernovae.
In contrast, \citet{2011A&A...529L...4C} argue that after accounting for the diversity of spectral features,
the $R_V=3.1$ measured for the diffuse Milky Way dust is recovered on average and \citet{2017ApJ...836..157H}
find $R_V=2.95 \pm 0.08$ for the highly-extincted SN~2012cu.

Even two parameters are probably not enough to capture supernova color diversity.  SNe~Ia 
are affected by interstellar Milky Way-like dust  and by ``intrinsic'' color variations from the supernova itself and its surrounding circumstellar dust. Physical modeling
shows that the complexity of interstellar dust shouldn't be expected to be captured by a single parameter \citep{2015ApJ...807L..26G,
2017ApJ...836...13H}. Supernova models \citep[e.g.][]{2007ApJ...662..487W,2013MNRAS.436..333S,2014MNRAS.441..532D,2015MNRAS.454.2549B,
2017ApJ...846...58H} and circumstellar dust \citep{2005ApJ...635L..33W,
2008ApJ...686L.103G} produce color variations beyond those due to interstellar dust.  A third color
parameter is therefore expected.

Hierarchical modeling has recently enabled
the study of intrinsic supernova color based on SN~Ia Hubble diagrams
by allowing the addition of a supernova population model as an extra level within in the larger model.
Latent parameters that are not directly tied to observables
but  influence color can be included in such models.
\citet{2017ApJ...842...93M} \citepalias[henceforth referred to as][]{2017ApJ...842...93M}) take the approach of modeling the distribution of their parameters to find that
scatter in the Hubble diagram is better explained by a combination of 
an intrinsic color-magnitude relationship with slope $\beta_{\mathit{int}}\sim 2.3$ intrinsic
color dispersion,
and
$R_V=2.7$ dust, rather than by dust with no color dispersion.
They draw these conclusions by using only the SALT2 (v2.4)
 $c$
parameter as the summary statistic that describes color.

The Nearby Supernova Factory \citep[SNfactory;][]{2002SPIE.4836...61A} has systematically observed the
spectrophotometric time series of hundreds of Hubble-flow $0.03<z<0.08$ SNe~Ia.   The $3200$--$10000$~\AA\ spectral coverage
provides measurements of an array of supernova spectral features while also enablng synthetic broadband photometry
spanning near-UV to near-IR SN-frame wavelengths.  SNfactory specifically targeted objects
early in their temporal evolution, so that well over a hundred of these supernovae have  coverage over
peak brightness.  This dataset provides a homogenous sample with which to study SN~Ia colors and spectral features simultaneously.

In this article we use the idea that spectral indicators carry information on intrinsic supernova colors at peak magnitude.
This approach is taken by \citet{2011A&A...529L...4C}, who find that after standardization based on Ca and Si features, remaining residual color
variation is consistent with Milky Way dust models.
We accommodate up to  three independent color parameters.
The data used in this analysis are described in \S\ref{data:sec}.
The analysis itself is presented as a series of three models of increasing complexity.
\S\ref{modelI:sec}  and  \S\ref{modelII:sec} present the first two models, providing an introduction to
our methodology and yielding results to be compared with those from the third model.
That ultimate model and its results are discussed in detail in \S\ref{modelIII:sec}. 
In \S\ref{modelI:sec} we present Model~I, 
a first analysis using spectral features to standardize SN colors
and two additional latent parameters to account for color diversity. 
By latent, we mean that the parameters are not directly associated with a single observable but rather deduced
as part of a hierarchical
model.
For Model~II described in \S\ref{modelII:sec}, we add a third latent color parameter to the analysis.
In \S\ref{modelIII:sec} we present our final Model~III, which  includes light-curve shape as an additional standardization parameter.
Conclusions are presented in \S\ref{conclusions:sec}.

\section{Data}
\label{data:sec}

Our analysis uses the spectrophotometric data set obtained by
the SNfactory with the SuperNova Integral Field
Spectrograph \citep[SNIFS,][]{2002SPIE.4836...61A, 2004SPIE.5249..146L}.  SNIFS is a fully integrated
instrument optimized for automated observation of point sources on a
structured background over the full ground-based optical window at
moderate spectral resolution ($R \sim 500$).  It consists of a
high-throughput wide-band lenslet integral field spectrograph, a multi-band
imager that covers the field in the vicinity of
the IFS for atmospheric transmission monitoring simultaneous with
spectroscopy, and an acquisition/guiding channel.  The IFS possesses a
fully-filled $6\farcs 4 \times 6\farcs 4$ spectroscopic field of view
subdivided into a grid of $15 \times 15$ spatial elements, a
dual-channel spectrograph covering 3200--5200~\AA\ and 5100--10000~\AA\
simultaneously, and an internal calibration unit (continuum and arc
lamps).  SNIFS is mounted on the south bent Cassegrain port of the
University of Hawaii 2.2~m telescope on Mauna Kea, and is operated
remotely.  Observations are reduced using the SNfactory's dedicated data
reduction pipeline, similar to that presented in \S4 of \citet{2001MNRAS.326...23B}.
A discussion of the software pipeline is presented in
\citet{2006ApJ...650..510A} and is updated in \citet{2010ApJ...713.1073S}. 
The flux calibration is presented in \citet{2013A&A...549A...8B}.
A detailed
description of host-galaxy subtraction is given in \citet{2011MNRAS.418..258B}.

\added{The SNfactory sample used here consists of 198 SNe~Ia
having fully-processed spectrophotometric lightcurve data, including
observations on at least two photometric nights, final references, and a
host spectroscopic redshift.   This sample was defined for use in \citet{leget:thesis}.
These supernovae have well-sampled lightcurves as defined by having
at minimum 5 spectra while still active, with at least four within
epochs  $[-10, 35]$ days, at least one within $[-10, 7]$ days, at least one within $[7, 20]$ days,
and at least one color (Anyone know what this means? See slide 3 of
\url{https://snf-doc.lbl.gov/twiki/pub/Conf/VideoConf20160317/IDR_03-17-2016.pdf}) within $[-8, 10]$ days of peak $B$ brightness.
Synthetic-photometric lightcurves derived from the spectra are fit to the SALT2~(v2.4)
model; we require that supernovae have no more than 20\% points that are
$>0.2$ mag from the best-fit SALT2 model, and require a Normalized Median Absolute Deviation (nMAD) of residuals
of  $< 0.12$ mag relative to this model.  \citep[It is known that more flexibility than available to SALT2 is required to capture
the full range of observed SN~Ia behavior, e.g.\ ][]{2011AJ....141...19B, 2011ApJ...731..120M}.
%These all have at least 5 spectra while the SN is
%active, and pass the quality cuts suggested by \citet{2010A&A...523A...7G}.
More than 80\% of our SNe are from searches where there was no
pre-selection based on host galaxy properties (those
whose names start with ``SNF'', ``LSQ'', or ``PTF'' in Table~\ref{data:tab}).

\citet{2013ApJ...770..108C}  present the distribution of host-galaxy parameters for a subset of the  above sample.
\citet{2015MNRAS.447.1247S} show scatter plots of the peak Si~II~$\lambda 5972$ and
Si~II~$\lambda 6355$ velocities and pseudo-equivalent
widths, absolute $B$ and $V$ magnitudes, $B-V$, SALTII $c$ and $x_1$ parameters, and $\Delta m_{15}$ for 
an preceding sample of 151 SNe.   

%Quality criteria for a 'good' lightcurve are the following: At least 5 nights of observation
%No more than 20\% of rejection (after outlier cut>0.2mag) nMAD of residuals < 0.12 mag
%Phase coverage (4 epochs in [-10 < p < +35] days, 1 epochs in [-10 < p < +7] days, 1 epochs in [+7 < p < +20] days,1 color in [-8 < p < +10] days)

From that sample, we further require at least one spectrum with $\pm 2.5$ days of B-band peak brightness.
We exclude six SNe~Ia in the
SN~1991T, SN~1991bg and SN~2002cx subclasses,
SNF20080905-005 described in \citet{leget:thesis} whose spectral features make it a $>3\sigma$ outlier in the
sample population, and
the extremely red SN2012cu  who has a SALT2 $C \approx 1$ \citep{2017ApJ...836..157H} that is several times larger than the colors
of in-sample supernovae.

Considering that our empirical parameters
are not fundamental supernova parameters, their best-fit values are dependent on sample selection.
Nevertheless, this  article seeks
underlying parameters that significantly
affect SN~Ia colors; any such parameters found in a subsample are still
relevant parameters when dealing with the population as a whole.
Within this context, it is not so important whether our sample population is representative
of the underlying population.

There is a fraction of SNfactory-observed supernovae that do have the
observations to satisfy the well-sampled light-curve criteria, but are excluded due to some observations
having not completed full processing.  Incomplete processing can occur for data taken during poor
observing conditions, when there are instrumental artifacts, and late in a supernova's
light-curve when the signal-to-noise and contrast relative to the host are low.  The sample-selection exclusion of these supernovae are not related,
at least directly, to the at-maximum observables used in this analysis.  Our results are insensitive to potential sample-selection
bias due to stochasticity in the measurements.
}


A recent description of the data is presented in \citet{2015ApJ...815...58F}.
We provide a brief summary of the points important for this analysis.
The spectral time-series  are corrected for Milky Way dust
extinction \citep{1989ApJ...345..245C,1998ApJ...500..525S}.  
Each spectral time series is
blue-shifted to rest-frame
based on the systemic redshift of the host \citep[c.f.][]{2013ApJ...770..107C}, and the fluxes are converted to luminosity assuming
distances expected for the supernova redshifts given a flat
$\Lambda$CDM cosmology with $\Omega_M = 0.28$ (with an arbitrarily selected
$H_0$ since the current analysis does not depend on absolute magnitude).

This article presents the first application of a computationally intensive analysis, so although  high spectral resolution
is available we here compress the information into low-resolution broadband photometry. 
Synthetic supernova-frame photometry is generated for a top-hat filter system
comprised of five 
bands with the following wavelength ranges: ${\hat{U}}$ $[3300.00 - 3978.02]$\AA;
${\hat{B}}$ $[3978.02-4795.35]$\AA;
$\hat{V}$ $[4795.35-5780.60]$\AA;
$\hat{R}$ $[5780.60-6968.29]$\AA;
$\hat{I}$ $[6968.29-8400.00]$\AA.
(The diacritic hat serves as a reminder that these are not standard Johnson-Cousins filters.)
For each supernova, the magnitudes within 5-days of peak brightness are used to fit single-band magnitudes
at $B$-band peak brightness.
The equivalent widths of the Si~II~$\lambda 4141$ and Ca~II H\&K features are computed as
in \citet{2008A&A...477..717B} and the 
wavelength of the Si~II~$\lambda 6355$ feature
as in \citet{chotard:thesis, 2017Chotard}.
Equivalent widths and the
Si~II~$\lambda 6355$ wavelength are taken from spectra  within $\pm 2.5$ days from $B$-band maximum;
the average is used  in cases where there are multiple spectral measurements within that time window.
All the above spectral features have units of \AA ngstroms.

Our analysis sample is comprised of the
172
supernovae that have been fully processed and have the data coverage to 
give photometric and spectroscopic statistics described above.
The 
spectroscopic-feature measurements, 
\replaced{SALT2 light-curve shape parameters  $x_1$,
and their uncertainties
are presented
in \citet{2017Chotard}}{
synthetic magnitudes (plus a constant offset), SALT2 light-curve shape parameters  $x_1$,
and their uncertainties
are presented
in 
Table~\ref{data:tab}. 
For the subset supernovae for whom it is available, we use the host-galaxy mass given in \citet{2013ApJ...770..108C}.

\startlongtable
\begin{deluxetable}{crrrrrrrrr}
\tabletypesize{\tiny}
\tablecaption{Supernova Spectral-Feature and Peak-Magnitude Data
\label{data:tab}}
\tablehead{
\colhead{Name} & \colhead{$EW_{\mathrm{Ca}}$ (\AA)} & \colhead{$EW_{\mathrm{Si}}$ (\AA)} & \colhead{$\lambda_{\mathrm{Si}}$ (\AA)} & \colhead{$U$+offset} & \colhead{$B$+offset} & \colhead{$V$+offset} & \colhead{$R$+offset} & \colhead{$I$+offset} & \colhead{$x_1$}
}
\startdata
SN2007bd & $109.7 \pm 5.9$ & $ 17.5 \pm 0.7$& $ 6101 \pm   3$ & $-29.31 \pm   0.01$ & $-29.12 \pm   0.01$& $-28.60 \pm   0.01$& $-28.35 \pm   0.01$& $-27.60 \pm   0.01$ & $ -0.85 \pm   0.34$\\
PTF10zdk & $149.7 \pm 1.2$ & $ 14.3 \pm 0.6$& $ 6150 \pm   3$ & $-28.61 \pm   0.02$ & $-28.69 \pm   0.02$& $-28.32 \pm   0.02$& $-28.08 \pm   0.02$& $-27.40 \pm   0.02$ & $  0.18 \pm   0.17$\\
SNF20080815-017 & $ 63.8 \pm 21.5$ & $ 27.6 \pm 3.8$& $ 6132 \pm   6$ & $-29.04 \pm   0.07$ & $-28.79 \pm   0.07$& $-28.32 \pm   0.07$& $-28.12 \pm   0.07$& $-27.41 \pm   0.07$ & $ -1.78 \pm   0.18$\\
PTF09dnl & $129.9 \pm 0.9$ & $  9.5 \pm 0.7$& $ 6093 \pm   3$ & $-29.23 \pm   0.01$ & $-29.07 \pm   0.01$& $-28.72 \pm   0.01$& $-28.44 \pm   0.01$& $-27.69 \pm   0.01$ & $  0.72 \pm   0.13$\\
SN2010ex & $114.4 \pm 0.9$ & $  8.4 \pm 0.4$& $ 6129 \pm   6$ & $-29.26 \pm   0.01$ & $-28.99 \pm   0.01$& $-28.50 \pm   0.01$& $-28.20 \pm   0.01$& $-27.44 \pm   0.01$ & $  0.27 \pm   0.14$\\
PTF09dnp & $ 64.9 \pm 4.5$ & $ 16.5 \pm 0.7$& $ 6098 \pm   4$ & $-29.55 \pm   0.02$ & $-29.19 \pm   0.02$& $-28.68 \pm   0.02$& $-28.48 \pm   0.02$& $-27.93 \pm   0.02$ & $ -0.74 \pm   0.24$\\
PTF11bnx & $151.4 \pm 3.0$ & $ 13.9 \pm 1.1$& $ 6142 \pm   5$ & $-28.63 \pm   0.02$ & $-28.57 \pm   0.01$& $-28.20 \pm   0.01$& $-27.99 \pm   0.01$& $-27.34 \pm   0.01$ & $ -0.07 \pm   0.17$\\
PTF12jqh & $151.9 \pm 1.5$ & $  7.9 \pm 0.7$& $ 6116 \pm  10$ & $-29.37 \pm   0.01$ & $-29.14 \pm   0.01$& $-28.71 \pm   0.01$& $-28.40 \pm   0.01$& $-27.64 \pm   0.01$ & $  0.69 \pm   0.19$\\
SNF20080802-006 & $108.2 \pm 6.0$ & $ 20.6 \pm 1.9$& $ 6122 \pm   5$ & $-29.02 \pm   0.06$ & $-28.80 \pm   0.06$& $-28.40 \pm   0.06$& $-28.20 \pm   0.06$& $-27.50 \pm   0.06$ & $ -1.56 \pm   0.13$\\
PTF10xyt & $123.7 \pm 6.6$ & $ 16.4 \pm 4.3$& $ 6101 \pm   4$ & $-28.26 \pm   0.02$ & $-28.20 \pm   0.02$& $-27.93 \pm   0.02$& $-27.74 \pm   0.02$& $-27.22 \pm   0.04$ & $  0.03 \pm   0.21$\\
PTF11qmo & $101.7 \pm 1.1$ & $  7.7 \pm 0.7$& $ 6150 \pm   8$ & $-29.77 \pm   0.02$ & $-29.43 \pm   0.02$& $-28.97 \pm   0.02$& $-28.64 \pm   0.02$& $-27.93 \pm   0.02$ & $ -0.00 \pm   0.34$\\
SNF20070331-025 & $119.8 \pm 7.4$ & $ 14.2 \pm 2.7$& $ 6120 \pm  10$ & $-28.94 \pm   0.02$ & $-28.75 \pm   0.02$& $-28.32 \pm   0.02$& $-28.07 \pm   0.02$& $-27.31 \pm   0.02$ & $ -1.28 \pm   0.16$\\
SNF20070818-001 & $157.5 \pm 7.5$ & $ 16.7 \pm 1.8$& $ 6115 \pm   5$ & $-28.97 \pm   0.02$ & $-28.96 \pm   0.01$& $-28.61 \pm   0.01$& $-28.37 \pm   0.01$& $-27.62 \pm   0.01$ & $  0.15 \pm   0.16$\\
SNBOSS38 & $ 57.1 \pm 0.4$ & $ 17.9 \pm 0.3$& $ 6127 \pm   3$ & $-29.20 \pm   0.01$ & $-28.84 \pm   0.01$& $-28.47 \pm   0.01$& $-28.23 \pm   0.01$& $-27.73 \pm   0.04$ & $  1.10 \pm   0.11$\\
SN2006ob & $ 90.0 \pm 16.5$ & $ 26.5 \pm 1.5$& $ 6112 \pm   5$ & $-29.11 \pm   0.02$ & $-28.82 \pm   0.01$& $-28.42 \pm   0.01$& $-28.19 \pm   0.01$& $-27.54 \pm   0.01$ & $ -2.45 \pm   0.28$\\
PTF12eer & $165.6 \pm 10.7$ & $ 12.7 \pm 2.8$& $ 6150 \pm  10$ & $-28.76 \pm   0.01$ & $-28.76 \pm   0.01$& $-28.40 \pm   0.01$& $-28.17 \pm   0.01$& $-27.45 \pm   0.02$ & $  0.68 \pm   0.16$\\
PTF10ops & $ 38.7 \pm 9.9$ & $  7.2 \pm 8.7$& $ 6140 \pm   5$ & $-27.93 \pm   0.38$ & $-27.76 \pm   0.38$& $-27.73 \pm   0.38$& $-27.59 \pm   0.38$& $-27.21 \pm   0.38$ & $ -0.62 \pm   0.13$\\
SNF20080514-002 & $ 83.2 \pm 0.7$ & $ 19.4 \pm 0.6$& $ 6131 \pm   3$ & $-29.30 \pm   0.01$ & $-28.95 \pm   0.01$& $-28.44 \pm   0.01$& $-28.17 \pm   0.01$& $-27.49 \pm   0.01$ & $ -1.57 \pm   0.11$\\
PTF12evo & $129.2 \pm 2.8$ & $  9.1 \pm 1.3$& $ 6156 \pm   4$ & $-29.14 \pm   0.02$ & $-28.98 \pm   0.01$& $-28.56 \pm   0.01$& $-28.28 \pm   0.01$& $-27.61 \pm   0.01$ & $  0.74 \pm   0.15$\\
SNF20080614-010 & $125.4 \pm 5.1$ & $ 26.9 \pm 1.6$& $ 6128 \pm   3$ & $-29.04 \pm   0.04$ & $-28.81 \pm   0.04$& $-28.38 \pm   0.04$& $-28.16 \pm   0.04$& $-27.57 \pm   0.04$ & $ -2.12 \pm   0.17$\\
PTF10icb & $104.8 \pm 0.9$ & $ 12.7 \pm 0.3$& $ 6138 \pm   3$ & $-28.58 \pm   0.02$ & $-28.36 \pm   0.02$& $-27.98 \pm   0.02$& $-27.77 \pm   0.02$& $-27.17 \pm   0.02$ & $  0.37 \pm   0.13$\\
PTF12efn & $144.9 \pm 3.4$ & $  7.1 \pm 1.8$& $ 6115 \pm   3$ & $-29.40 \pm   0.01$ & $-29.17 \pm   0.01$& $-28.79 \pm   0.01$& $-28.45 \pm   0.01$& $-27.64 \pm   0.01$ & $  1.21 \pm   0.20$\\
SNNGC4424 & $109.0 \pm 0.3$ & $  8.6 \pm 0.1$& $ 6138 \pm   2$ & $-28.35 \pm   0.01$ & $-28.15 \pm   0.01$& $-27.79 \pm   0.01$& $-27.58 \pm   0.01$& $-26.97 \pm   0.01$ & $  0.37 \pm   0.12$\\
SNF20080516-022 & $100.1 \pm 2.1$ & $ 13.7 \pm 1.1$& $ 6158 \pm   3$ & $-29.46 \pm   0.01$ & $-29.19 \pm   0.01$& $-28.71 \pm   0.01$& $-28.39 \pm   0.01$& $-27.77 \pm   0.01$ & $  0.20 \pm   0.14$\\
PTF12hwb & $ 21.1 \pm 78.0$ & $ -1.8 \pm 8.9$& $ 6090 \pm  14$ & $-28.32 \pm   0.02$ & $-28.24 \pm   0.02$& $-28.03 \pm   0.02$& $-27.79 \pm   0.02$& $-27.05 \pm   0.04$ & $  1.34 \pm   0.17$\\
PTF10qyz & $106.4 \pm 2.1$ & $ 23.0 \pm 1.0$& $ 6120 \pm   5$ & $-29.05 \pm   0.17$ & $-28.92 \pm   0.17$& $-28.41 \pm   0.17$& $-28.14 \pm   0.17$& $-27.30 \pm   0.17$ & $ -1.09 \pm   0.16$\\
SNF20060907-000 & $106.1 \pm 10.4$ & $ 17.0 \pm 0.9$& $ 6149 \pm   4$ & $-29.54 \pm   0.02$ & $-29.28 \pm   0.01$& $-28.76 \pm   0.01$& $-28.42 \pm   0.01$& $-27.74 \pm   0.04$ & $  0.17 \pm   0.17$\\
LSQ12fxd & $122.9 \pm 1.7$ & $ 11.4 \pm 0.8$& $ 6119 \pm   4$ & $-29.62 \pm   0.07$ & $-29.39 \pm   0.07$& $-28.95 \pm   0.07$& $-28.64 \pm   0.07$& $-27.91 \pm   0.07$ & $  0.05 \pm   0.15$\\
SNF20080821-000 & $105.1 \pm 2.2$ & $  8.6 \pm 1.3$& $ 6121 \pm   4$ & $-29.34 \pm   0.01$ & $-29.10 \pm   0.01$& $-28.73 \pm   0.01$& $-28.46 \pm   0.01$& $-27.82 \pm   0.01$ & $  0.70 \pm   0.19$\\
SNF20070802-000 & $158.3 \pm 3.3$ & $ 16.3 \pm 1.7$& $ 6102 \pm   5$ & $-28.90 \pm   0.01$ & $-28.81 \pm   0.01$& $-28.45 \pm   0.01$& $-28.22 \pm   0.01$& $-27.52 \pm   0.01$ & $  0.08 \pm   0.16$\\
PTF10wnm & $105.8 \pm 2.3$ & $  6.5 \pm 1.0$& $ 6124 \pm   3$ & $-29.38 \pm   0.01$ & $-29.07 \pm   0.01$& $-28.68 \pm   0.01$& $-28.37 \pm   0.01$& $-27.69 \pm   0.01$ & $  0.63 \pm   0.16$\\
PTF10mwb & $116.5 \pm 1.2$ & $ 19.8 \pm 0.9$& $ 6138 \pm   2$ & $-29.02 \pm   0.07$ & $-28.84 \pm   0.07$& $-28.40 \pm   0.07$& $-28.14 \pm   0.07$& $-27.52 \pm   0.07$ & $ -0.68 \pm   0.10$\\
SN2010dt & $116.2 \pm 14.9$ & $ 15.5 \pm 0.7$& $ 6138 \pm   6$ & $-29.30 \pm   0.01$ & $-29.15 \pm   0.01$& $-28.64 \pm   0.01$& $-28.35 \pm   0.01$& $-27.63 \pm   0.01$ & $ -0.03 \pm   0.10$\\
SNF20080623-001 & $149.1 \pm 1.4$ & $ 14.9 \pm 0.7$& $ 6131 \pm   3$ & $-29.11 \pm   0.01$ & $-28.97 \pm   0.01$& $-28.50 \pm   0.01$& $-28.22 \pm   0.01$& $-27.46 \pm   0.01$ & $  0.01 \pm   0.13$\\
LSQ12fhe & $ 42.8 \pm 1.2$ & $  4.0 \pm 3.1$& $ 6108 \pm   4$ & $-29.76 \pm   0.02$ & $-29.40 \pm   0.02$& $-29.04 \pm   0.02$& $-28.74 \pm   0.02$& $-28.11 \pm   0.02$ & $  1.07 \pm   0.19$\\
PTF11bju & $ 30.2 \pm 4.4$ & $  4.0 \pm 3.0$& $ 6139 \pm   5$ & $-29.47 \pm   0.02$ & $-29.10 \pm   0.01$& $-28.75 \pm   0.01$& $-28.45 \pm   0.01$& $-27.87 \pm   0.01$ & $  1.04 \pm   0.20$\\
PTF09fox & $117.6 \pm 2.7$ & $  9.1 \pm 1.0$& $ 6116 \pm   3$ & $-29.44 \pm   0.03$ & $-29.21 \pm   0.03$& $-28.72 \pm   0.03$& $-28.42 \pm   0.03$& $-27.68 \pm   0.03$ & $  0.68 \pm   0.27$\\
PTF13ayw & $104.6 \pm 2.4$ & $ 26.6 \pm 3.2$& $ 6115 \pm   6$ & $-29.16 \pm   0.02$ & $-28.82 \pm   0.02$& $-28.43 \pm   0.02$& $-28.20 \pm   0.02$& $-27.55 \pm   0.02$ & $ -1.72 \pm   0.29$\\
SNF20070810-004 & $126.7 \pm 1.8$ & $ 21.1 \pm 1.1$& $ 6118 \pm   7$ & $-29.22 \pm   0.01$ & $-29.10 \pm   0.01$& $-28.63 \pm   0.01$& $-28.34 \pm   0.01$& $-27.62 \pm   0.01$ & $  0.10 \pm   0.12$\\
PTF11mty & $111.4 \pm 2.3$ & $ 10.6 \pm 1.5$& $ 6138 \pm   5$ & $-29.54 \pm   0.01$ & $-29.23 \pm   0.01$& $-28.80 \pm   0.01$& $-28.46 \pm   0.01$& $-27.82 \pm   0.01$ & $  0.83 \pm   0.16$\\
SNF20080512-010 & $ 95.3 \pm 3.5$ & $ 23.3 \pm 1.5$& $ 6129 \pm   5$ & $-29.22 \pm   0.08$ & $-28.96 \pm   0.08$& $-28.50 \pm   0.08$& $-28.26 \pm   0.08$& $-27.56 \pm   0.08$ & $ -1.24 \pm   0.26$\\
PTF11mkx & $ 31.5 \pm 3.7$ & $  4.5 \pm 1.3$& $ 6169 \pm   5$ & $-29.50 \pm   0.45$ & $-29.25 \pm   0.45$& $-28.89 \pm   0.45$& $-28.61 \pm   0.45$& $-27.97 \pm   0.45$ & $  0.60 \pm   0.15$\\
PTF10tce & $135.7 \pm 1.1$ & $ 11.2 \pm 1.5$& $ 6090 \pm   4$ & $-29.13 \pm   0.02$ & $-28.99 \pm   0.01$& $-28.59 \pm   0.01$& $-28.31 \pm   0.01$& $-27.55 \pm   0.01$ & $  0.43 \pm   0.14$\\
SNF20061020-000 & $ 95.4 \pm 18.8$ & $ 24.1 \pm 1.0$& $ 6120 \pm   5$ & $-29.01 \pm   0.03$ & $-28.78 \pm   0.03$& $-28.35 \pm   0.03$& $-28.17 \pm   0.03$& $-27.54 \pm   0.03$ & $ -1.67 \pm   0.20$\\
SN2005ir & $115.6 \pm 2.8$ & $ 13.5 \pm 6.9$& $ 6069 \pm   5$ & $-29.33 \pm   0.02$ & $-29.12 \pm   0.02$& $-28.84 \pm   0.02$& $-28.49 \pm   0.02$& $-27.77 \pm   0.02$ & $  1.71 \pm   0.69$\\
SNF20080717-000 & $ 93.3 \pm 2.6$ & $  8.3 \pm 2.2$& $ 6104 \pm   3$ & $-28.58 \pm   0.01$ & $-28.47 \pm   0.01$& $-28.29 \pm   0.01$& $-28.05 \pm   0.01$& $-27.50 \pm   0.01$ & $  0.70 \pm   0.14$\\
PTF12ena & $101.1 \pm 1.6$ & $  7.4 \pm 1.0$& $ 6129 \pm   4$ & $-28.01 \pm   0.01$ & $-28.00 \pm   0.01$& $-27.85 \pm   0.01$& $-27.77 \pm   0.01$& $-27.31 \pm   0.01$ & $  0.65 \pm   0.16$\\
PTF13anh & $166.8 \pm 1.8$ & $ 21.8 \pm 1.2$& $ 6175 \pm   4$ & $-28.67 \pm   0.20$ & $-28.74 \pm   0.20$& $-28.30 \pm   0.20$& $-28.05 \pm   0.20$& $-27.28 \pm   0.20$ & $ -0.28 \pm   0.17$\\
CSS110918\_01 & $110.6 \pm 1.0$ & $  8.0 \pm 1.3$& $ 6101 \pm   2$ & $-29.88 \pm   0.76$ & $-29.58 \pm   0.76$& $-29.09 \pm   0.76$& $-28.71 \pm   0.76$& $-27.91 \pm   0.76$ & $  0.92 \pm   0.17$\\
SNF20061024-000 & $ 86.9 \pm 26.8$ & $ 30.0 \pm 1.5$& $ 6127 \pm   5$ & $-28.88 \pm   0.04$ & $-28.70 \pm   0.04$& $-28.26 \pm   0.04$& $-28.05 \pm   0.04$& $-27.40 \pm   0.04$ & $ -1.39 \pm   0.19$\\
SNF20070506-006 & $ 94.1 \pm 1.3$ & $  6.7 \pm 0.6$& $ 6153 \pm   3$ & $-29.72 \pm   0.01$ & $-29.39 \pm   0.01$& $-28.97 \pm   0.01$& $-28.64 \pm   0.01$& $-27.96 \pm   0.01$ & $  1.17 \pm   0.13$\\
SNF20070403-001 & $105.9 \pm 5.4$ & $ 18.3 \pm 1.8$& $ 6124 \pm   4$ & $-29.23 \pm   0.02$ & $-29.04 \pm   0.01$& $-28.63 \pm   0.01$& $-28.35 \pm   0.01$& $-27.62 \pm   0.01$ & $ -0.54 \pm   0.13$\\
PTF10hmv & $109.6 \pm 1.3$ & $  8.9 \pm 0.7$& $ 6143 \pm   3$ & $-28.54 \pm   0.01$ & $-28.40 \pm   0.01$& $-28.11 \pm   0.01$& $-27.89 \pm   0.01$& $-27.31 \pm   0.01$ & $  1.39 \pm   0.18$\\
SNF20071015-000 & $105.0 \pm 3.2$ & $  6.9 \pm 1.1$& $ 6124 \pm   7$ & $-27.89 \pm   0.02$ & $-27.82 \pm   0.02$& $-27.69 \pm   0.02$& $-27.63 \pm   0.02$& $-27.16 \pm   0.04$ & $  1.36 \pm   0.37$\\
SNhunt89 & $ 88.0 \pm 2.7$ & $ 32.2 \pm 1.9$& $ 6111 \pm   7$ & $-28.37 \pm   0.03$ & $-28.26 \pm   0.03$& $-27.92 \pm   0.03$& $-27.77 \pm   0.03$& $-27.13 \pm   0.03$ & $ -1.83 \pm   0.18$\\
SNF20070902-021 & $108.9 \pm 3.5$ & $ 17.1 \pm 1.0$& $ 6131 \pm   6$ & $-29.25 \pm   0.02$ & $-29.02 \pm   0.02$& $-28.56 \pm   0.02$& $-28.32 \pm   0.01$& $-27.65 \pm   0.02$ & $ -0.43 \pm   0.10$\\
PTF09dlc & $143.5 \pm 2.2$ & $ 10.2 \pm 0.9$& $ 6143 \pm   3$ & $-29.38 \pm   0.01$ & $-29.17 \pm   0.01$& $-28.69 \pm   0.01$& $-28.40 \pm   0.01$& $-27.62 \pm   0.01$ & $  0.11 \pm   0.12$\\
PTF13ajv & $150.5 \pm 8.9$ & $ 46.3 \pm 8.6$& $ 6110 \pm  21$ & $-28.70 \pm   0.02$ & $-28.61 \pm   0.02$& $-28.16 \pm   0.02$& $-27.91 \pm   0.02$& $-27.07 \pm   0.04$ & $ -1.59 \pm   0.22$\\
SNF20080919-000 & $114.7 \pm 2.8$ & $  9.4 \pm 0.9$& $ 6145 \pm   5$ & $-28.53 \pm   0.02$ & $-28.41 \pm   0.01$& $-28.11 \pm   0.01$& $-27.99 \pm   0.01$& $-27.38 \pm   0.01$ & $ -0.13 \pm   0.11$\\
SNF20080919-001 & $ 85.0 \pm 1.1$ & $  6.0 \pm 0.4$& $ 6150 \pm   5$ & $-29.73 \pm   0.01$ & $-29.43 \pm   0.01$& $-29.04 \pm   0.01$& $-28.72 \pm   0.01$& $-28.07 \pm   0.01$ & $  0.84 \pm   0.18$\\
SN2010kg & $ 95.1 \pm 28.5$ & $ 21.7 \pm 0.7$& $ 6077 \pm   5$ & $-28.85 \pm   0.01$ & $-28.74 \pm   0.01$& $-28.41 \pm   0.01$& $-28.20 \pm   0.01$& $-27.47 \pm   0.01$ & $ -2.26 \pm   0.40$\\
SNF20080714-008 & $134.8 \pm 15.7$ & $ 19.7 \pm 3.7$& $ 6100 \pm   6$ & $-28.56 \pm   0.02$ & $-28.63 \pm   0.01$& $-28.32 \pm   0.01$& $-28.13 \pm   0.01$& $-27.42 \pm   0.01$ & $  0.01 \pm   0.15$\\
SNF20070714-007 & $129.6 \pm 5.6$ & $ 31.1 \pm 23.8$& $ 6146 \pm   5$ & $-27.88 \pm   0.02$ & $-28.12 \pm   0.01$& $-28.02 \pm   0.01$& $-27.86 \pm   0.01$& $-27.24 \pm   0.03$ & $ -1.10 \pm   0.13$\\
SNF20080522-011 & $122.1 \pm 1.7$ & $  8.3 \pm 0.5$& $ 6125 \pm   3$ & $-29.63 \pm   0.01$ & $-29.38 \pm   0.01$& $-28.92 \pm   0.01$& $-28.60 \pm   0.01$& $-27.88 \pm   0.01$ & $  0.81 \pm   0.19$\\
SNF20061111-002 & $110.8 \pm 10.7$ & $ 20.4 \pm 1.0$& $ 6145 \pm   6$ & $-29.16 \pm   0.01$ & $-28.99 \pm   0.01$& $-28.59 \pm   0.01$& $-28.29 \pm   0.01$& $-27.61 \pm   0.01$ & $  0.27 \pm   0.21$\\
SNNGC6343 & $ 87.0 \pm 1.4$ & $ 20.7 \pm 0.7$& $ 6136 \pm   3$ & $-28.78 \pm   0.01$ & $-28.66 \pm   0.01$& $-28.30 \pm   0.01$& $-28.08 \pm   0.01$& $-27.41 \pm   0.01$ & $ -2.09 \pm   0.25$\\
SNF20061011-005 & $120.6 \pm 1.1$ & $  9.3 \pm 0.4$& $ 6132 \pm   4$ & $-29.72 \pm   0.04$ & $-29.43 \pm   0.03$& $-28.99 \pm   0.03$& $-28.64 \pm   0.03$& $-27.90 \pm   0.03$ & $  0.74 \pm   0.29$\\
SNF20080825-010 & $102.4 \pm 13.4$ & $ 19.2 \pm 0.6$& $ 6116 \pm   4$ & $-29.46 \pm   0.01$ & $-29.17 \pm   0.01$& $-28.71 \pm   0.01$& $-28.47 \pm   0.01$& $-27.83 \pm   0.01$ & $ -0.77 \pm   0.18$\\
PTF10ufj & $141.1 \pm 3.4$ & $ 11.7 \pm 1.2$& $ 6131 \pm   6$ & $-29.28 \pm   0.15$ & $-29.16 \pm   0.15$& $-28.72 \pm   0.15$& $-28.41 \pm   0.15$& $-27.65 \pm   0.15$ & $  0.53 \pm   0.16$\\
PTF10wof & $129.6 \pm 2.7$ & $ 17.3 \pm 1.0$& $ 6102 \pm   2$ & $-28.91 \pm   0.01$ & $-28.84 \pm   0.01$& $-28.46 \pm   0.01$& $-28.18 \pm   0.01$& $-27.43 \pm   0.01$ & $  0.16 \pm   0.13$\\
SNF20080918-000 & $146.8 \pm 3.5$ & $  7.5 \pm 2.5$& $ 6110 \pm   5$ & $-28.79 \pm   0.02$ & $-28.65 \pm   0.02$& $-28.35 \pm   0.02$& $-28.12 \pm   0.02$& $-27.46 \pm   0.02$ & $  1.50 \pm   0.30$\\
SNF20080516-000 & $117.4 \pm 2.2$ & $  9.0 \pm 1.2$& $ 6135 \pm   3$ & $-29.50 \pm   0.01$ & $-29.23 \pm   0.01$& $-28.80 \pm   0.01$& $-28.47 \pm   0.01$& $-27.74 \pm   0.01$ & $  1.24 \pm   0.26$\\
SN2005cf & $159.1 \pm 0.7$ & $ 15.7 \pm 0.8$& $ 6141 \pm   3$ & $-29.37 \pm   0.02$ & $-29.16 \pm   0.02$& $-28.68 \pm   0.02$& $-28.41 \pm   0.02$& $-27.69 \pm   0.02$ & $ -0.31 \pm   0.17$\\
CSS130502\_01 & $ 91.5 \pm 10.9$ & $ 15.6 \pm 0.5$& $ 6128 \pm   3$ & $-29.43 \pm   0.02$ & $-29.09 \pm   0.02$& $-28.60 \pm   0.01$& $-28.30 \pm   0.01$& $-27.62 \pm   0.04$ & $ -0.00 \pm   0.24$\\
SNF20080620-000 & $107.8 \pm 14.1$ & $ 20.0 \pm 0.7$& $ 6132 \pm   3$ & $-28.82 \pm   0.02$ & $-28.78 \pm   0.01$& $-28.32 \pm   0.01$& $-28.09 \pm   0.01$& $-27.39 \pm   0.01$ & $ -1.04 \pm   0.16$\\
SNPGC51271 & $ 92.1 \pm 16.5$ & $ 21.1 \pm 0.7$& $ 6121 \pm   2$ & $-29.28 \pm   0.02$ & $-28.95 \pm   0.02$& $-28.46 \pm   0.02$& $-28.20 \pm   0.02$& $-27.62 \pm   0.04$ & $ -2.08 \pm   0.18$\\
PTF11pdk & $128.6 \pm 2.8$ & $ 15.6 \pm 1.7$& $ 6153 \pm   5$ & $-29.35 \pm   0.02$ & $-29.11 \pm   0.02$& $-28.61 \pm   0.02$& $-28.32 \pm   0.02$& $-27.67 \pm   0.02$ & $  0.06 \pm   0.19$\\
SNF20060511-014 & $102.6 \pm 2.8$ & $ 15.6 \pm 1.1$& $ 6141 \pm   8$ & $-29.16 \pm   0.07$ & $-29.04 \pm   0.06$& $-28.56 \pm   0.06$& $-28.30 \pm   0.06$& $-27.63 \pm   0.06$ & $ -0.68 \pm   0.18$\\
SNF20080612-003 & $120.0 \pm 1.1$ & $  7.3 \pm 0.6$& $ 6123 \pm   3$ & $-29.64 \pm   0.02$ & $-29.41 \pm   0.02$& $-28.99 \pm   0.02$& $-28.70 \pm   0.02$& $-28.00 \pm   0.02$ & $  0.71 \pm   0.21$\\
SNF20080626-002 & $130.0 \pm 1.0$ & $  6.1 \pm 4.2$& $ 6111 \pm   3$ & $-29.42 \pm   0.01$ & $-29.24 \pm   0.01$& $-28.84 \pm   0.01$& $-28.52 \pm   0.01$& $-27.76 \pm   0.01$ & $  1.04 \pm   0.23$\\
SNF20060621-015 & $111.9 \pm 1.3$ & $  9.8 \pm 0.7$& $ 6144 \pm   3$ & $-29.63 \pm   0.01$ & $-29.36 \pm   0.01$& $-28.88 \pm   0.01$& $-28.54 \pm   0.01$& $-27.81 \pm   0.01$ & $  0.43 \pm   0.20$\\
SNF20080920-000 & $135.2 \pm 1.4$ & $  5.6 \pm 1.6$& $ 6085 \pm   3$ & $-29.44 \pm   0.02$ & $-29.19 \pm   0.02$& $-28.79 \pm   0.02$& $-28.49 \pm   0.02$& $-27.74 \pm   0.02$ & $  1.17 \pm   0.34$\\
SN2007cq & $ 65.8 \pm 4.1$ & $ 10.2 \pm 0.9$& $ 6137 \pm   3$ & $-29.53 \pm   0.02$ & $-29.30 \pm   0.02$& $-28.89 \pm   0.02$& $-28.56 \pm   0.02$& $-27.90 \pm   0.02$ & $ -0.43 \pm   0.16$\\
SNF20080918-004 & $ 87.8 \pm 7.2$ & $ 21.5 \pm 0.9$& $ 6141 \pm   4$ & $-29.00 \pm   0.22$ & $-28.82 \pm   0.22$& $-28.37 \pm   0.22$& $-28.13 \pm   0.22$& $-27.43 \pm   0.22$ & $ -1.27 \pm   0.18$\\
CSS120424\_01 & $138.1 \pm 2.1$ & $ 11.7 \pm 0.7$& $ 6138 \pm   3$ & $-29.40 \pm   0.02$ & $-29.23 \pm   0.02$& $-28.77 \pm   0.01$& $-28.45 \pm   0.02$& $-27.68 \pm   0.02$ & $  0.76 \pm   0.19$\\
SNF20080610-000 & $119.9 \pm 10.4$ & $ 16.4 \pm 1.7$& $ 6131 \pm   6$ & $-29.05 \pm   0.07$ & $-28.92 \pm   0.07$& $-28.50 \pm   0.07$& $-28.22 \pm   0.07$& $-27.55 \pm   0.07$ & $  0.12 \pm   0.16$\\
SNF20070701-005 & $101.8 \pm 2.6$ & $ 12.4 \pm 1.0$& $ 6158 \pm   5$ & $-29.46 \pm   0.02$ & $-29.27 \pm   0.02$& $-28.87 \pm   0.02$& $-28.60 \pm   0.02$& $-27.96 \pm   0.02$ & $  0.06 \pm   0.17$\\
SN2007kk & $128.5 \pm 1.4$ & $ 10.6 \pm 1.0$& $ 6098 \pm   4$ & $-29.48 \pm   0.02$ & $-29.31 \pm   0.02$& $-28.87 \pm   0.01$& $-28.54 \pm   0.01$& $-27.77 \pm   0.02$ & $  0.52 \pm   0.18$\\
SNF20060908-004 & $114.4 \pm 1.2$ & $ 12.6 \pm 0.6$& $ 6136 \pm   3$ & $-29.59 \pm   0.23$ & $-29.34 \pm   0.23$& $-28.91 \pm   0.23$& $-28.58 \pm   0.23$& $-27.87 \pm   0.23$ & $  0.70 \pm   0.19$\\
SNF20080909-030 & $ 93.7 \pm 1.0$ & $  7.8 \pm 0.4$& $ 6171 \pm   3$ & $-29.38 \pm   0.02$ & $-29.12 \pm   0.01$& $-28.74 \pm   0.01$& $-28.44 \pm   0.01$& $-27.78 \pm   0.01$ & $  1.17 \pm   0.16$\\
PTF11bgv & $ 79.4 \pm 3.2$ & $ 12.6 \pm 0.7$& $ 6146 \pm   3$ & $-28.90 \pm   0.02$ & $-28.62 \pm   0.01$& $-28.27 \pm   0.01$& $-28.08 \pm   0.01$& $-27.54 \pm   0.01$ & $ -0.33 \pm   0.12$\\
SNNGC2691 & $ 39.0 \pm 22.2$ & $  4.5 \pm 0.2$& $ 6139 \pm   8$ & $-29.46 \pm   0.02$ & $-29.06 \pm   0.02$& $-28.75 \pm   0.02$& $-28.49 \pm   0.02$& $-27.93 \pm   0.02$ & $  0.98 \pm   0.18$\\
PTF13asv & $ 75.6 \pm 1.1$ & $  2.2 \pm 0.4$& $ 6148 \pm   4$ & $-29.92 \pm   0.32$ & $-29.49 \pm   0.32$& $-29.02 \pm   0.32$& $-28.63 \pm   0.32$& $-27.90 \pm   0.32$ & $  0.50 \pm   0.11$\\
SNF20070806-026 & $ 98.8 \pm 12.1$ & $ 25.9 \pm 0.7$& $ 6114 \pm   7$ & $-29.14 \pm   0.02$ & $-28.91 \pm   0.02$& $-28.44 \pm   0.02$& $-28.21 \pm   0.02$& $-27.49 \pm   0.02$ & $ -1.73 \pm   0.17$\\
SNF20070427-001 & $ 81.3 \pm 2.3$ & $  6.3 \pm 0.9$& $ 6142 \pm   5$ & $-29.89 \pm   0.02$ & $-29.46 \pm   0.02$& $-28.97 \pm   0.02$& $-28.62 \pm   0.02$& $-27.97 \pm   0.02$ & $  0.62 \pm   0.21$\\
SNF20061108-004 & $129.5 \pm 5.6$ & $  6.3 \pm 2.5$& $ 6110 \pm   6$ & $-29.53 \pm   0.02$ & $-29.31 \pm   0.02$& $-28.95 \pm   0.02$& $-28.60 \pm   0.02$& $-27.96 \pm   0.02$ & $  2.14 \pm   0.42$\\
SNF20060912-000 & $106.5 \pm 1.8$ & $ 21.4 \pm 1.7$& $ 6163 \pm   7$ & $-28.98 \pm   0.02$ & $-28.92 \pm   0.02$& $-28.66 \pm   0.02$& $-28.42 \pm   0.02$& $-27.77 \pm   0.02$ & $  0.24 \pm   0.20$\\
CSS110918\_02 & $109.1 \pm 9.4$ & $ 15.0 \pm 0.6$& $ 6137 \pm   3$ & $-29.36 \pm   0.02$ & $-29.14 \pm   0.01$& $-28.69 \pm   0.01$& $-28.41 \pm   0.01$& $-27.70 \pm   0.01$ & $ -0.27 \pm   0.17$\\
SNF20080918-002 & $ 97.7 \pm 2.8$ & $ 12.6 \pm 1.4$& $ 6141 \pm   6$ & $-29.50 \pm   0.02$ & $-29.11 \pm   0.02$& $-28.61 \pm   0.02$& $-28.34 \pm   0.02$& $-27.71 \pm   0.02$ & $ -0.50 \pm   0.09$\\
SNIC3573 & $102.7 \pm 1.8$ & $ 11.9 \pm 1.0$& $ 6142 \pm   5$ & $-29.28 \pm   0.02$ & $-29.14 \pm   0.02$& $-28.74 \pm   0.02$& $-28.46 \pm   0.01$& $-27.76 \pm   0.03$ & $  0.35 \pm   0.19$\\
SNF20080725-004 & $133.6 \pm 2.1$ & $  6.9 \pm 0.9$& $ 6131 \pm   6$ & $-29.09 \pm   0.01$ & $-28.93 \pm   0.01$& $-28.59 \pm   0.01$& $-28.31 \pm   0.01$& $-27.55 \pm   0.03$ & $  0.60 \pm   0.14$\\
SNF20050728-006 & $127.8 \pm 2.5$ & $ 15.8 \pm 1.3$& $ 6124 \pm   6$ & $-28.80 \pm   0.02$ & $-28.68 \pm   0.02$& $-28.37 \pm   0.02$& $-28.18 \pm   0.02$& $-27.55 \pm   0.02$ & $  0.27 \pm   0.27$\\
SN2012fr & $134.2 \pm 0.5$ & $  7.4 \pm 0.2$& $ 6102 \pm   1$ & $-29.91 \pm   0.01$ & $-29.70 \pm   0.01$& $-29.31 \pm   0.01$& $-28.94 \pm   0.01$& $-28.10 \pm   0.01$ & $  0.96 \pm   0.10$\\
SNF20060512-002 & $100.2 \pm 2.8$ & $ 13.4 \pm 1.1$& $ 6107 \pm   8$ & $-29.33 \pm   0.02$ & $-29.11 \pm   0.02$& $-28.77 \pm   0.02$& $-28.52 \pm   0.02$& $-27.80 \pm   0.02$ & $ -0.92 \pm   0.19$\\
SNF20060512-001 & $ 88.4 \pm 1.2$ & $  5.4 \pm 0.4$& $ 6169 \pm   3$ & $-29.33 \pm   0.01$ & $-29.05 \pm   0.01$& $-28.68 \pm   0.01$& $-28.40 \pm   0.01$& $-27.79 \pm   0.01$ & $  0.69 \pm   0.15$\\
SNF20071003-016 & $125.2 \pm 4.6$ & $ 17.1 \pm 2.0$& $ 6124 \pm  11$ & $-28.58 \pm   0.02$ & $-28.54 \pm   0.02$& $-28.19 \pm   0.02$& $-27.99 \pm   0.02$& $-27.31 \pm   0.02$ & $ -0.25 \pm   0.15$\\
SNF20050821-007 & $141.7 \pm 2.6$ & $  7.7 \pm 1.0$& $ 6140 \pm   9$ & $-29.38 \pm   0.02$ & $-29.20 \pm   0.02$& $-28.77 \pm   0.02$& $-28.46 \pm   0.02$& $-27.67 \pm   0.02$ & $  0.17 \pm   0.32$\\
SNF20070803-005 & $ 22.7 \pm 21.4$ & $  0.9 \pm 0.6$& $ 6157 \pm  27$ & $-29.87 \pm   0.01$ & $-29.43 \pm   0.01$& $-29.04 \pm   0.01$& $-28.74 \pm   0.01$& $-28.11 \pm   0.01$ & $  0.88 \pm   0.18$\\
PTF09foz & $127.2 \pm 1.9$ & $ 21.7 \pm 1.2$& $ 6136 \pm   4$ & $-29.14 \pm   0.01$ & $-29.00 \pm   0.01$& $-28.59 \pm   0.01$& $-28.35 \pm   0.01$& $-27.65 \pm   0.01$ & $ -1.14 \pm   0.26$\\
PTF12grk & $162.3 \pm 9.8$ & $ 19.6 \pm 1.4$& $ 6085 \pm   8$ & $-28.86 \pm   0.02$ & $-28.87 \pm   0.01$& $-28.42 \pm   0.01$& $-28.19 \pm   0.01$& $-27.50 \pm   0.03$ & $ -1.17 \pm   0.13$\\
SNF20080720-001 & $138.5 \pm 4.0$ & $ 14.0 \pm 2.0$& $ 6107 \pm   3$ & $-27.59 \pm   0.02$ & $-27.78 \pm   0.01$& $-27.73 \pm   0.01$& $-27.71 \pm   0.01$& $-27.19 \pm   0.02$ & $  0.18 \pm   0.13$\\
SNF20080810-001 & $ 88.4 \pm 21.6$ & $ 22.3 \pm 1.1$& $ 6145 \pm   5$ & $-29.11 \pm   0.01$ & $-28.89 \pm   0.01$& $-28.45 \pm   0.01$& $-28.23 \pm   0.01$& $-27.60 \pm   0.01$ & $ -0.87 \pm   0.11$\\
SNF20050729-002 & $109.4 \pm 2.2$ & $ 11.5 \pm 1.7$& $ 6142 \pm   6$ & $-29.35 \pm   0.13$ & $-29.17 \pm   0.13$& $-28.68 \pm   0.13$& $-28.38 \pm   0.13$& $-27.56 \pm   0.13$ & $  0.57 \pm   0.34$\\
SN2008ec & $103.7 \pm 17.0$ & $ 23.1 \pm 0.4$& $ 6125 \pm   3$ & $-28.67 \pm   0.01$ & $-28.52 \pm   0.01$& $-28.18 \pm   0.01$& $-28.03 \pm   0.01$& $-27.47 \pm   0.01$ & $ -1.05 \pm   0.12$\\
SNF20070902-018 & $ 93.8 \pm 12.2$ & $ 23.8 \pm 3.0$& $ 6120 \pm   8$ & $-28.87 \pm   0.02$ & $-28.70 \pm   0.01$& $-28.26 \pm   0.01$& $-28.08 \pm   0.01$& $-27.41 \pm   0.02$ & $ -0.72 \pm   0.10$\\
SNF20070424-003 & $122.5 \pm 3.8$ & $ 12.7 \pm 1.6$& $ 6132 \pm   6$ & $-29.10 \pm   0.01$ & $-28.96 \pm   0.01$& $-28.51 \pm   0.01$& $-28.25 \pm   0.01$& $-27.57 \pm   0.01$ & $  0.31 \pm   0.11$\\
SN2006cj & $101.7 \pm 1.3$ & $  4.8 \pm 0.8$& $ 6127 \pm   3$ & $-29.43 \pm   0.01$ & $-29.14 \pm   0.01$& $-28.74 \pm   0.01$& $-28.43 \pm   0.01$& $-27.76 \pm   0.01$ & $  0.48 \pm   0.18$\\
SN2007nq & $ 89.8 \pm 9.9$ & $ 23.4 \pm 1.1$& $ 6109 \pm   5$ & $-29.11 \pm   0.02$ & $-28.91 \pm   0.02$& $-28.50 \pm   0.02$& $-28.27 \pm   0.02$& $-27.57 \pm   0.02$ & $ -1.89 \pm   0.26$\\
SNF20070817-003 & $ 93.9 \pm 2.4$ & $ 18.5 \pm 1.3$& $ 6116 \pm   6$ & $-29.19 \pm   0.02$ & $-29.03 \pm   0.01$& $-28.59 \pm   0.01$& $-28.30 \pm   0.01$& $-27.55 \pm   0.02$ & $ -0.57 \pm   0.13$\\
SNF20070403-000 & $ 61.8 \pm 6.5$ & $ 27.1 \pm 1.8$& $ 6154 \pm   8$ & $-28.37 \pm   0.02$ & $-28.27 \pm   0.02$& $-27.97 \pm   0.02$& $-27.80 \pm   0.02$& $-27.24 \pm   0.02$ & $ -1.96 \pm   0.19$\\
SNF20061022-005 & $ 64.6 \pm 3.8$ & $  3.7 \pm 1.4$& $ 6146 \pm   7$ & $-29.49 \pm   0.02$ & $-29.06 \pm   0.02$& $-28.71 \pm   0.02$& $-28.42 \pm   0.02$& $-27.93 \pm   0.02$ & $  1.72 \pm   0.44$\\
SNNGC4076 & $127.3 \pm 2.4$ & $ 15.5 \pm 1.2$& $ 6152 \pm   4$ & $-28.77 \pm   0.01$ & $-28.66 \pm   0.01$& $-28.37 \pm   0.01$& $-28.15 \pm   0.01$& $-27.52 \pm   0.01$ & $  0.01 \pm   0.12$\\
SNF20070727-016 & $ 77.5 \pm 2.5$ & $  5.1 \pm 0.8$& $ 6140 \pm   4$ & $-29.96 \pm   0.06$ & $-29.56 \pm   0.06$& $-29.06 \pm   0.06$& $-28.75 \pm   0.06$& $-28.01 \pm   0.06$ & $  0.27 \pm   0.12$\\
PTF12fuu & $105.5 \pm 3.0$ & $  6.2 \pm 1.2$& $ 6124 \pm   5$ & $-29.54 \pm   0.01$ & $-29.23 \pm   0.01$& $-28.74 \pm   0.01$& $-28.40 \pm   0.01$& $-27.64 \pm   0.01$ & $  0.34 \pm   0.12$\\
SNF20070820-000 & $107.2 \pm 3.5$ & $ 18.6 \pm 1.3$& $ 6132 \pm  14$ & $-28.80 \pm   0.02$ & $-28.69 \pm   0.02$& $-28.34 \pm   0.02$& $-28.13 \pm   0.02$& $-27.52 \pm   0.02$ & $ -0.29 \pm   0.13$\\
SNF20070725-001 & $108.4 \pm 2.0$ & $ 11.1 \pm 1.5$& $ 6140 \pm   7$ & $-29.61 \pm   0.02$ & $-29.32 \pm   0.02$& $-28.84 \pm   0.02$& $-28.50 \pm   0.02$& $-27.76 \pm   0.02$ & $  0.68 \pm   0.17$\\
SNF20071108-021 & $ 99.1 \pm 2.7$ & $  5.8 \pm 0.8$& $ 6164 \pm   5$ & $-29.67 \pm   0.01$ & $-29.34 \pm   0.01$& $-28.94 \pm   0.01$& $-28.60 \pm   0.01$& $-27.96 \pm   0.01$ & $  1.00 \pm   0.21$\\
SNF20080914-001 & $126.5 \pm 1.2$ & $ 15.4 \pm 1.1$& $ 6159 \pm   3$ & $-28.67 \pm   0.02$ & $-28.60 \pm   0.02$& $-28.31 \pm   0.02$& $-28.13 \pm   0.02$& $-27.58 \pm   0.02$ & $ -0.11 \pm   0.17$\\
SNF20060609-002 & $ 87.7 \pm 3.6$ & $  7.3 \pm 1.3$& $ 6132 \pm   4$ & $-28.60 \pm   0.02$ & $-28.42 \pm   0.02$& $-28.19 \pm   0.02$& $-28.05 \pm   0.02$& $-27.53 \pm   0.02$ & $  0.07 \pm   0.15$\\
SNF20050624-000 & $121.0 \pm 5.3$ & $  9.3 \pm 3.1$& $ 6126 \pm   6$ & $-29.75 \pm   0.01$ & $-29.42 \pm   0.01$& $-28.99 \pm   0.01$& $-28.68 \pm   0.01$& $-27.97 \pm   0.01$ & $  0.69 \pm   0.62$\\
SNF20060618-023 & $ 74.9 \pm 4.9$ & $  5.0 \pm 1.8$& $ 6137 \pm  21$ & $-29.61 \pm   0.02$ & $-29.18 \pm   0.02$& $-28.89 \pm   0.02$& $-28.66 \pm   0.02$& $-28.08 \pm   0.02$ & $  0.12 \pm   0.18$\\
SNF20080531-000 & $133.0 \pm 1.5$ & $ 17.6 \pm 0.8$& $ 6114 \pm   5$ & $-29.12 \pm   0.01$ & $-28.98 \pm   0.01$& $-28.54 \pm   0.01$& $-28.28 \pm   0.01$& $-27.51 \pm   0.01$ & $ -0.49 \pm   0.18$\\
SN2006do & $106.4 \pm 2.1$ & $ 26.7 \pm 1.3$& $ 6101 \pm   2$ & $-29.00 \pm   0.01$ & $-28.83 \pm   0.01$& $-28.42 \pm   0.01$& $-28.20 \pm   0.01$& $-27.53 \pm   0.04$ & $ -2.32 \pm   0.54$\\
PTF12ikt & $110.3 \pm 1.6$ & $ 14.2 \pm 0.7$& $ 6141 \pm   4$ & $-29.34 \pm   0.01$ & $-29.04 \pm   0.01$& $-28.57 \pm   0.01$& $-28.32 \pm   0.01$& $-27.66 \pm   0.01$ & $ -0.35 \pm   0.11$\\
SN2006dm & $ 99.5 \pm 1.6$ & $ 30.0 \pm 0.7$& $ 6118 \pm   3$ & $-28.81 \pm   0.01$ & $-28.65 \pm   0.01$& $-28.23 \pm   0.01$& $-28.02 \pm   0.01$& $-27.33 \pm   0.01$ & $ -1.88 \pm   0.34$\\
PTF13azs & $138.0 \pm 5.1$ & $ 16.2 \pm 1.6$& $ 6125 \pm  10$ & $-27.84 \pm   0.02$ & $-27.92 \pm   0.02$& $-27.69 \pm   0.02$& $-27.60 \pm   0.02$& $-26.99 \pm   0.02$ & $ -0.02 \pm   0.12$\\
SN2005hj & $ 80.8 \pm 2.4$ & $  4.3 \pm 0.8$& $ 6138 \pm   4$ & $-29.54 \pm   0.02$ & $-29.16 \pm   0.01$& $-28.87 \pm   0.01$& $-28.54 \pm   0.01$& $-28.01 \pm   0.01$ & $  1.85 \pm   0.59$\\
PTF12iiq & $150.4 \pm 2.2$ & $ 22.5 \pm 0.8$& $ 6041 \pm   6$ & $-28.60 \pm   0.01$ & $-28.77 \pm   0.01$& $-28.41 \pm   0.01$& $-28.10 \pm   0.01$& $-27.29 \pm   0.01$ & $ -1.30 \pm   0.12$\\
PTF10ndc & $124.2 \pm 2.4$ & $  6.8 \pm 1.1$& $ 6119 \pm   3$ & $-29.52 \pm   0.01$ & $-29.25 \pm   0.01$& $-28.80 \pm   0.01$& $-28.49 \pm   0.01$& $-27.76 \pm   0.01$ & $  0.87 \pm   0.18$\\
SNF20080919-002 & $103.6 \pm 7.2$ & $ 27.2 \pm 1.9$& $ 6133 \pm   8$ & $-28.74 \pm   0.02$ & $-28.46 \pm   0.01$& $-28.09 \pm   0.01$& $-27.87 \pm   0.01$& $-27.26 \pm   0.04$ & $ -1.52 \pm   0.14$\\
SNPGC027923 & $ 85.5 \pm 0.6$ & $  5.9 \pm 0.3$& $ 6130 \pm   4$ & $-29.87 \pm   0.02$ & $-29.45 \pm   0.02$& $-28.94 \pm   0.02$& $-28.57 \pm   0.02$& $-27.85 \pm   0.02$ & $  0.22 \pm   0.26$\\
SNF20070330-024 & $118.1 \pm 2.1$ & $  4.6 \pm 2.2$& $ 6101 \pm   3$ & $-29.77 \pm   0.02$ & $-29.52 \pm   0.02$& $-29.08 \pm   0.02$& $-28.74 \pm   0.01$& $-27.94 \pm   0.02$ & $  0.34 \pm   0.29$\\
SNF20061030-010 & $131.4 \pm 2.2$ & $ 17.4 \pm 1.1$& $ 6116 \pm   4$ & $-28.60 \pm   0.02$ & $-28.55 \pm   0.02$& $-28.25 \pm   0.02$& $-28.03 \pm   0.02$& $-27.34 \pm   0.02$ & $  0.14 \pm   0.25$\\
SNhunt46 & $ 94.1 \pm 2.0$ & $ 11.2 \pm 0.6$& $ 6132 \pm   4$ & $-29.50 \pm   0.02$ & $-29.11 \pm   0.02$& $-28.67 \pm   0.02$& $-28.37 \pm   0.02$& $-27.71 \pm   0.02$ & $  0.20 \pm   0.24$\\
SN2005hc & $126.9 \pm 2.5$ & $ 10.0 \pm 0.7$& $ 6123 \pm   3$ & $-29.38 \pm   0.01$ & $-29.13 \pm   0.01$& $-28.69 \pm   0.01$& $-28.37 \pm   0.01$& $-27.61 \pm   0.01$ & $  0.87 \pm   0.31$\\
LSQ12dbr & $106.9 \pm 0.6$ & $  7.1 \pm 0.7$& $ 6138 \pm   4$ & $-29.29 \pm   0.73$ & $-29.00 \pm   0.73$& $-28.51 \pm   0.73$& $-28.15 \pm   0.73$& $-27.38 \pm   0.73$ & $  1.12 \pm   0.14$\\
LSQ12hjm & $ 82.6 \pm 17.5$ & $ 12.2 \pm 1.4$& $ 6144 \pm   5$ & $-29.51 \pm   0.02$ & $-29.14 \pm   0.01$& $-28.60 \pm   0.01$& $-28.30 \pm   0.01$& $-27.71 \pm   0.02$ & $ -0.50 \pm   0.21$\\
SNF20060521-001 & $ 78.9 \pm 20.2$ & $ 21.1 \pm 1.4$& $ 6123 \pm  10$ & $-29.37 \pm   0.05$ & $-29.04 \pm   0.05$& $-28.54 \pm   0.05$& $-28.30 \pm   0.05$& $-27.57 \pm   0.05$ & $ -1.22 \pm   0.26$\\
SNF20070630-006 & $125.5 \pm 3.2$ & $ 10.1 \pm 1.6$& $ 6126 \pm   4$ & $-29.34 \pm   0.01$ & $-29.12 \pm   0.01$& $-28.65 \pm   0.01$& $-28.38 \pm   0.01$& $-27.66 \pm   0.01$ & $  0.35 \pm   0.12$\\
PTF11drz & $132.6 \pm 1.4$ & $ 15.2 \pm 1.0$& $ 6116 \pm   5$ & $-29.12 \pm   0.01$ & $-28.95 \pm   0.01$& $-28.53 \pm   0.01$& $-28.27 \pm   0.01$& $-27.55 \pm   0.01$ & $  0.21 \pm   0.37$\\
SNF20080323-009 & $ 95.9 \pm 2.3$ & $ 10.6 \pm 1.1$& $ 6143 \pm   6$ & $-29.59 \pm   0.02$ & $-29.22 \pm   0.02$& $-28.68 \pm   0.02$& $-28.42 \pm   0.02$& $-27.77 \pm   0.02$ & $ -0.10 \pm   0.17$\\
SNF20071021-000 & $167.5 \pm 2.2$ & $ 20.4 \pm 0.6$& $ 6112 \pm   4$ & $-28.75 \pm   0.02$ & $-28.78 \pm   0.02$& $-28.40 \pm   0.02$& $-28.18 \pm   0.02$& $-27.41 \pm   0.02$ & $ -0.53 \pm   0.18$\\
SNNGC0927 & $155.2 \pm 1.4$ & $ 11.0 \pm 0.7$& $ 6109 \pm   4$ & $-28.87 \pm   0.02$ & $-28.81 \pm   0.01$& $-28.46 \pm   0.01$& $-28.22 \pm   0.01$& $-27.48 \pm   0.01$ & $  0.41 \pm   0.14$\\
SNF20060526-003 & $112.1 \pm 2.5$ & $  9.8 \pm 1.0$& $ 6121 \pm   3$ & $-29.34 \pm   0.01$ & $-29.09 \pm   0.01$& $-28.68 \pm   0.01$& $-28.39 \pm   0.01$& $-27.70 \pm   0.01$ & $  0.29 \pm   0.17$\\
SNF20080806-002 & $135.8 \pm 1.8$ & $  7.5 \pm 0.9$& $ 6135 \pm   4$ & $-29.22 \pm   0.02$ & $-29.02 \pm   0.02$& $-28.61 \pm   0.02$& $-28.35 \pm   0.01$& $-27.71 \pm   0.02$ & $  0.64 \pm   0.16$\\
SNF20080803-000 & $117.6 \pm 2.6$ & $  8.9 \pm 2.0$& $ 6125 \pm   4$ & $-28.84 \pm   0.01$ & $-28.70 \pm   0.01$& $-28.35 \pm   0.01$& $-28.16 \pm   0.01$& $-27.50 \pm   0.01$ & $  0.37 \pm   0.14$\\
SNF20080822-005 & $ 78.5 \pm 1.8$ & $  6.3 \pm 0.9$& $ 6138 \pm   4$ & $-29.71 \pm   0.01$ & $-29.34 \pm   0.01$& $-28.93 \pm   0.01$& $-28.61 \pm   0.01$& $-27.92 \pm   0.01$ & $  0.56 \pm   0.23$\\
SNF20060618-014 & $137.2 \pm 2.5$ & $  9.3 \pm 1.1$& $ 6112 \pm   7$ & $-29.27 \pm   0.03$ & $-29.09 \pm   0.03$& $-28.73 \pm   0.03$& $-28.38 \pm   0.03$& $-27.68 \pm   0.03$ & $  0.97 \pm   0.21$\\
PTF12ghy & $ 99.3 \pm 3.6$ & $ 16.8 \pm 0.7$& $ 6134 \pm   3$ & $-28.29 \pm   0.02$ & $-28.27 \pm   0.01$& $-28.05 \pm   0.01$& $-27.95 \pm   0.01$& $-27.40 \pm   0.01$ & $ -0.48 \pm   0.12$\\
SNF20070531-011 & $122.4 \pm 2.7$ & $ 21.2 \pm 0.8$& $ 6114 \pm   4$ & $-29.07 \pm   0.01$ & $-28.94 \pm   0.01$& $-28.50 \pm   0.01$& $-28.26 \pm   0.01$& $-27.53 \pm   0.03$ & $ -1.82 \pm   0.25$\\
SNF20070831-015 & $112.2 \pm 2.7$ & $  7.8 \pm 1.0$& $ 6145 \pm   6$ & $-29.42 \pm   0.01$ & $-29.17 \pm   0.01$& $-28.78 \pm   0.01$& $-28.46 \pm   0.01$& $-27.78 \pm   0.01$ & $  1.85 \pm   0.23$\\
SNF20070417-002 & $104.5 \pm 5.5$ & $ 24.4 \pm 2.2$& $ 6123 \pm   9$ & $-29.20 \pm   0.05$ & $-29.01 \pm   0.05$& $-28.48 \pm   0.05$& $-28.23 \pm   0.05$& $-27.54 \pm   0.05$ & $ -1.37 \pm   0.11$\\
PTF11cao & $143.3 \pm 1.6$ & $ 18.9 \pm 1.3$& $ 6104 \pm   5$ & $-28.78 \pm   0.02$ & $-28.79 \pm   0.02$& $-28.44 \pm   0.02$& $-28.18 \pm   0.02$& $-27.45 \pm   0.02$ & $  0.18 \pm   0.15$\\
SNF20080522-000 & $ 61.8 \pm 3.5$ & $  3.3 \pm 0.9$& $ 6131 \pm   7$ & $-29.86 \pm   0.01$ & $-29.41 \pm   0.01$& $-29.03 \pm   0.01$& $-28.70 \pm   0.01$& $-28.06 \pm   0.01$ & $  1.02 \pm   0.13$\\
PTF10qjq & $ 73.9 \pm 2.4$ & $ 12.8 \pm 0.8$& $ 6133 \pm   3$ & $-29.29 \pm   0.02$ & $-28.94 \pm   0.02$& $-28.53 \pm   0.01$& $-28.35 \pm   0.01$& $-27.76 \pm   0.01$ & $ -0.35 \pm   0.11$\\
PTF12dxm & $ 95.4 \pm 41.8$ & $ 35.7 \pm 2.8$& $ 6136 \pm   4$ & $-28.71 \pm   0.01$ & $-28.58 \pm   0.01$& $-28.19 \pm   0.01$& $-27.99 \pm   0.01$& $-27.34 \pm   0.01$ & $ -1.98 \pm   0.20$\\
SNF20061021-003 & $122.8 \pm 2.3$ & $  9.7 \pm 1.7$& $ 6131 \pm   4$ & $-29.04 \pm   0.02$ & $-28.86 \pm   0.02$& $-28.56 \pm   0.02$& $-28.30 \pm   0.02$& $-27.64 \pm   0.02$ & $  0.25 \pm   0.18$\\
SNF20080510-005 & $111.6 \pm 2.6$ & $  6.4 \pm 1.1$& $ 6115 \pm   4$ & $-29.41 \pm   0.01$ & $-29.15 \pm   0.01$& $-28.70 \pm   0.01$& $-28.38 \pm   0.01$& $-27.73 \pm   0.04$ & $  1.90 \pm   0.26$\\
SNF20080507-000 & $ 98.1 \pm 1.6$ & $ 10.6 \pm 2.1$& $ 6143 \pm   5$ & $-29.23 \pm   0.01$ & $-29.05 \pm   0.01$& $-28.71 \pm   0.01$& $-28.45 \pm   0.01$& $-27.79 \pm   0.01$ & $  0.89 \pm   0.25$\\
SNF20080913-031 & $118.2 \pm 1.5$ & $ 11.3 \pm 1.8$& $ 6158 \pm   5$ & $-29.13 \pm   0.08$ & $-29.01 \pm   0.07$& $-28.62 \pm   0.07$& $-28.32 \pm   0.07$& $-27.68 \pm   0.07$ & $  0.03 \pm   0.21$\\
SNF20080510-001 & $118.8 \pm 2.1$ & $ 15.3 \pm 1.3$& $ 6115 \pm   4$ & $-29.35 \pm   0.01$ & $-29.15 \pm   0.01$& $-28.69 \pm   0.01$& $-28.38 \pm   0.01$& $-27.68 \pm   0.01$ & $  0.43 \pm   0.18$\\
SNF20070712-003 & $108.8 \pm 2.7$ & $ 13.5 \pm 0.9$& $ 6155 \pm   6$ & $-29.44 \pm   0.02$ & $-29.19 \pm   0.01$& $-28.74 \pm   0.01$& $-28.42 \pm   0.01$& $-27.78 \pm   0.01$ & $  0.42 \pm   0.18$\\
\enddata
\end{deluxetable}

The distributions of the central values of the inputs  $EW_{\mathrm{Ca}}$, $EW_{\mathrm{Si}}$,
$\lambda_{\mathrm{Si}}$ , $x_1$, and the SALT2 color parameter $C$ are shown in Table~\ref{input:fig}.
For reference, the $x_1$ and $C$ distributions of the $0.03<z< 0.08$ Pantheon sample,
predominantly supernovae from CFA3 \citep{2009ApJ...700..331H}, CFA4 \citep{2012ApJS..200...12H}, and CSP
\citep{2010AJ....139..519C, 2011AJ....142..156S}, that were used as part of the low-redshift anchor in the recent cosmology analysis of \citet{2017arXiv171000845S}.

\begin{figure}[htbp] %  figure placement: here, top, bottom, or page
   \centering
   \includegraphics[width=4in]{perobject_input.pdf}
   \caption{\added{Distributions of the central values of the inputs  $EW_{\mathrm{Ca}}$, $EW_{\mathrm{Si}}$,
$\lambda_{\mathrm{Si}}$ , $x_1$, and $C$ of the sample used in this article, and
the $x_1$ and $C$ distributions of the $0.03<z< 0.08$ Pantheon sample.}
   \label{input:fig}}
\end{figure}
}

\section{Model~I: Two Color Parameters}
\label{modelI:sec}
We begin by considering Model~I, in which supernova magnitudes  have linear dependence on spectral parameters
and two color parameters. The purpose of this section is to introduce some basic concepts of our models,
to show how the two  color terms can be associated with existing dust-extinction models, and to give Model~I results on dust extinction.
Deficiencies of this model are addressed in the subsequent models presented in the article.
To avoid repetitiveness,
we refrain from going into full detail on the results of Model~I in deference to Model~III, which is presented in Section~\ref{modelIII:sec}.

\added{It is important to note that the model and those in subsequent sections are applied to our supernova sample;
the ensuing results thus correspond to our SN~Ia sample's underlying population. }

\subsection{Model}
We assume 
that  peak
underlying
 ${\hat{U}}{\hat{B}}\hat{V}\hat{R}\hat{I}$ magnitudes of a supernova  with effects from  astrophysical foregrounds are linearly dependent
on its
 equivalent widths of the Ca~II H\&K and Si~II~$\lambda$4141 spectral features
$EW_{\mathrm{Ca}}$ and $EW_{\mathrm{Si}}$,
and the wavelength of the minimum of 
the Si~II~$\lambda6355$ feature $\lambda_{\mathrm{Si}}$
around $B$-band peak brightness:
these spectral features are associated with SN~Ia  spectroscopic diversity  
\citep{2005ApJ...623.1011B, 2006PASP..118..560B, 2008A&A...492..535A, 2009A&A...500L..17B, 2009PASP..121..238B, 2009ApJ...699L.139W, 2011ApJ...729...55F,2012AJ....143..126B}.
The explicit omission of light-curve shape in our model is compensated by its proxy,
$EW_{\mathrm{Si}}$, at peak brightness
\citep{2008A&A...492..535A, 2011A&A...529L...4C}. 
The underlying magnitudes are also linearly dependent on the per-supernova latent parameters $g_0$ and $g_1$.
Unlike
the spectral parameters ($EW_{\mathrm{Ca}}$, $EW_{\mathrm{Si}}$ and $\lambda_{\mathrm{Si}}$),  $g_0$ and $g_1$ are not directly associated
with observables but rather are inferred as part of the analysis.
A grey magnitude offset, $\Delta$, is included for each supernova
to capture 
potential
band-independent intrinsic dispersion, while also absorbing peculiar-velocity errors introduced when converting
fluxes to luminosities.
With this grey offset the model standardizes colors, not absolute magnitude.
The underlying magnitudes are
\begin{equation}
\begin{pmatrix}
{\hat{U}}\\{\hat{B}}\\{\hat{V}}\\{\hat{R}}\\{\hat{I}}
\end{pmatrix}
=
\Delta \vec{1} +
\begin{pmatrix}
c_{\hat{U}}+\alpha_{\hat{U}} EW_{\mathrm{Ca}} + \beta_{\hat{U}} EW_{\mathrm{Si}} + \eta_{\hat{U}} \lambda_{\mathrm{Si}}  +\gamma^0_{{\hat{U}}} g_0 +\gamma^1_{{\hat{U}}} g_1 \\
c_{\hat{B}}+\alpha_{\hat{B}} EW_{\mathrm{Ca}} + \beta_{\hat{B}} EW_{\mathrm{Si}} + \eta_{\hat{B}} \lambda_{\mathrm{Si}}  +\gamma^0_{{\hat{B}}} g_0 +\gamma^1_{{\hat{B}}} g_1 \\
c_{\hat{V}}+\alpha_{\hat{V}} EW_{\mathrm{Ca}} + \beta_{\hat{V}} EW_{\mathrm{Si}} + \eta_{\hat{V}} \lambda_{\mathrm{Si}} +\gamma^0_{{\hat{V}}} g_0 +\gamma^1_{{\hat{V}}} g_1 \\
c_{\hat{R}}+\alpha_{\hat{R}} EW_{\mathrm{Ca}} + \beta_{\hat{R}} EW_{\mathrm{Si}} + \eta_{\hat{R}} \lambda_{\mathrm{Si}}  +\gamma^0_{{\hat{R}}} g_0 +\gamma^1_{{\hat{R}}} g_1 \\
c_{\hat{I}}+\alpha_{\hat{I}} EW_{\mathrm{Ca}} + \beta_{\hat{I}} EW_{\mathrm{Si}}+ \eta_{\hat{I}} \lambda_{\mathrm{Si}}  +\gamma^0_{{\hat{I}}} g_0 +\gamma^1_{{\hat{I}}} g_1
\end{pmatrix}.
\label{ewsiv:eqn}
\end{equation}
$\vec{1}$ is the Identity vector.
The global parameter vectors\footnote{Global parameters act on each of the five bands: a single component is written with a subscript for the corresponding band, e.g.\  $c_{\hat{U}}$; 
the combination of all components are written as a vector, e.g.\ $\vec{c}=(c_{\hat{U}}, c_{\hat{B}}, c_{\hat{V}}, c_{\hat{R}}, c_{\hat{I}}) $.  The
band subscript or arrow
notationally identify global parameters, in contrast with the per-supernova parameters.
} that describe the SN~Ia population
$\vec{c}$  (units of mag);
$\vec{\alpha}$, $\vec{\beta}$,
$\vec{\eta}$ (units of mag~\AA$^{-1}$), $\vec{\gamma}^0$ and $\vec{\gamma}^1$ (units of mag)
 are the intercept
and slopes
of the linear relationships that
relate
per-supernova
parameters with 
underlying magnitudes.
The notation used in this article is not to be associated with those from previous works.


The observables
$\hat{U}_o, {\hat{B}}_o, {\hat{V}}_o, {\hat{R}}_o, {\hat{I}}_o$, $EW_{Ca,o}$, $EW_{Si,o}$, $\lambda_{Si,o}$
have Gaussian measurement uncertainty with covariance $C$.
The
likelihood density for the described model
corresponds to the probability distribution function described by
\begin{equation}
\begin{pmatrix}
{\hat{U}}_o\\{\hat{B}}_o\\ {\hat{V}}_o\\{\hat{R}}_o\\{\hat{I}}_o\\EW_{Si, o}\\ EW_{Ca, o} \\ \lambda_{Si, o}
\end{pmatrix}
\sim \mathcal{N}
\left(
\begin{pmatrix}
{\hat{U}}  \\{\hat{B}}  \\
{\hat{V}}\\{\hat{R}}\\{\hat{I}}\\
EW_{\mathrm{Si}}\\ EW_{\mathrm{Ca}} \\ \lambda_{\mathrm{Si}}
\end{pmatrix}
,C
\right).
\label{dust:eqn}
\end{equation}


The model as written has degeneracies that need to be constrained in order for fits to converge.
The model does not specify the absolute magnitude nor the 
color of a $g_0=0$, $g_1=0$ supernova;
in their place the zeropoints of the per-supernova magnitude and color parameters are set to the sample means
\begin{equation}
\langle \Delta \rangle=0,\ \langle g_0 \rangle=0,\ \langle g_1 \rangle=0.
\label{zero:eqn}
\end{equation}
The model contains the product of parameters
$\vec{\gamma} g$, which leads to the degeneracy $\vec{\gamma} \rightarrow a\vec{\gamma}$, $g \rightarrow a^{-1} g$.
To aid in the convergence of $\vec{\gamma}$ and $g$ we impose a prior on the rms of $g$, though
our physical interpretations are ultimately independent of this scaling.
This prior does not specify the sign of $a$, which leaves a parity degeneracy. 
Results in this article are given in terms of $\vec{\gamma}g$, not $g$ alone.
As will be seen, the signal-to-noise in $\vec{\gamma}$ is sufficiently
high that our finite MCMC chains do not migrate between the degenerate solutions; to simplify the merging of multiple chains
we impose one of the degenerate solutions
\begin{equation}
\gamma^0_{\hat{U}} > 0,\ \gamma^1_{\hat{U}} < 0.
\end{equation}
As  in mixture models, the combinations $\vec{\gamma}^0 g_0$ and $\vec{\gamma}^1 g_1$ terms are degenerate under exchange of the 0 and 1 indices. For a cleaner presentation of our results, 
we break that degeneracy  by setting consistent initial conditions that succeed in keeping indices consistent over all analysis chains.  

Each of the multiplicative cofactors has a degeneracy with the zeropoint.  Consider
$\vec{\alpha} \rightarrow \vec{\alpha} +   \epsilon_{\alpha}$,
$\vec{\beta} \rightarrow \vec{\beta} +   \epsilon_{\beta}$,
$\vec{\eta} \rightarrow \vec{\eta} +   \epsilon_{\eta}$,
$\vec{\gamma}^0 \rightarrow \vec{\gamma}^0 +   \epsilon_{\gamma^0}$,
$\vec{\gamma}^1 \rightarrow \vec{\gamma}^1 +   \epsilon_{\gamma^1}$.
Then
each supernova's grey term can shift as
$\Delta \rightarrow \Delta  + \epsilon_\Delta$
with
$$
\epsilon_\Delta  = -  \epsilon_{\alpha} (EW_{\text{Ca}}-\langle EW_{\text{Ca}}\rangle)- 
 \epsilon_{\beta} (EW_{\mathrm{Si}}-\langle EW_{\mathrm{Si}}\rangle) -  \epsilon_{\eta} (\lambda_{\mathrm{Si}}-\langle \lambda_{\mathrm{Si}}\rangle)
- \epsilon_{\gamma^0}g_0 -  \epsilon_{\gamma^1}g_1.
$$
We break this degeneracy by fixing the values of $\epsilon$ such that $\Delta$ is
uncorrelated with  $EW_{\text{Ca}}$, $EW_{\mathrm{Si}}$, $\lambda_{\text{Ca}}$,
$g_0$, and $g_1$ reflecting
our preference for feature affects to be confined to the cofactors and not the grey magnitude offsets.

All parameters not  otherwise noted above have flat priors.

For $N$ supernovae there are $8N$ observables.  There are $3N$ spectral parameters, each of
$\Delta$, $g_0$, $g_1$ contributes $N-1$ parameters
(recall the constraint in Eq.~\ref{zero:eqn}),  and there are $5 \times 7$ global coefficients.
For $N=172$ supernovae, there are 1376 observables and 1064  parameters.


\subsection{Results for the Latent Parameter Cofactors  $\vec{\gamma}^0$ and $\vec{\gamma}^1$  }
\label{results1:sec}
\added{The MCMC chains for all parameters from this and subsequent sections are accessible from the SNfactory website\footnote{For the referee the parameters
are available at \url{https://drive.google.com/open?id=1d52nxfF7eS3HblfbmT3fxgiZod_Z-Gjg}. Concurrent with publication the chain will be
made available
at \url{https://snfactory.lbl.gov/snf/data/index.html}.}.}
In this subsection we concentrate on the results for $\vec{\gamma}^0$ and $\vec{\gamma}^1$.
In our model, $g_0$ and $g_1$ are introduced as latent parameters that are related to linear shifts in band magnitudes through
the  coefficients $\vec{\gamma}^0$ and
$\vec{\gamma}^1$.  Otherwise, no physically-motivated suppositions on the properties of $g$ and $\vec{\gamma}$ are imposed.
Since dust extinction is a dominant determinant of supernova colors that is often described by a 2-parameter linear model, we expect
for our two  color terms to capture the effects of dust.
In this section we reexpress our model $\vec{\gamma}$'s in terms of
the dust-extinction model of \citet{1999PASP..111...63F} \citepalias[henceforth referred to as][]{1999PASP..111...63F}.

The 68\% credible intervals for $\vec{\gamma}^0$ and $\vec{\gamma}^1$ (and all other global parameters) are given in \S\ref{m12results:sec}.
All elements of the two $\vec{\gamma}$ parameter vectors 
describing color
are significantly non-zero.
None of the 20000 links of 
our Monte Carlo chains for $\vec{\gamma}$ come close to the origin.   We claim 
a probability $(1-5\times 10^{-5})$-detection of
a magnitude term in the form of a 5-dimensional vector spanned by two basis vectors
\begin{equation}
\vec{A} = \vec{\gamma}^0 g_0 +  \vec{\gamma}^1 g_1.
\end{equation}
All possible magnitude changes in our five bands due to $g_0$ and $g_1$ are confined to a two-dimensional
plane defined by $\vec{\gamma}^0$ and  $\vec{\gamma}^1$. 

Dust-extinction
models are also approximately described by a 2-parameter linear model, here 
using the commonly-used parameters $A_V$ and $E(B-V)$
\begin{equation}
\vec{A} =\vec{ a}  A_{V} + \vec{b} E(B-V).
\label{f99:eqn}
\end{equation}

For example,   the wavelength-dependent model of  \citet{1989ApJ...345..245C} is linear,
though its linearity is lost when integrated over broad-band filters.
For the case of
$R^F=2.5$ and $A^F_V=0.1$ dust attenuating light from the SALT2
\citep{2007A&A...466...11G} $s=1$, $x_1=0$ SN~Ia template at ${\hat{B}}$-band peak, the 
\citetalias{1999PASP..111...63F} model
can be approximated with
$\vec{a} = (0.96,   1.00,   1.00,   0.97,   0.77)$ and $\vec{b}=(  1.77,   0.98,   0.12,  -0.50,  -0.53)$.
(The $F$ superscript is used to distinguish parameters of the \citetalias{1999PASP..111...63F} model.)
Over the ranges
 $0\le A^F_V\le 1$ and $2 \le R^F \le 3.5$ 
and the wavelengths under consideration here,  the values of the elements of $a$ and $b$ vary by $<5$\%
for the
\citetalias{1999PASP..111...63F} model with
 the largest error in extinction from the linear approximation having amplitude $<0.008$ mag.
All possible magnitude changes in our five bands due to  \citetalias{1999PASP..111...63F}  dust extinction are confined to a two-dimensional
plane defined by $\vec{a}$ and  $\vec{b}$. 


The credible interval for $\vec{\gamma}$'s and other parameters are presented in Table~\ref{global1:tab}.
Our model $\vec{\gamma}$ vectors can be written as a linear combination (described by the $2 \times 2$ matrix M) of the
dust-based $\vec{a}$ and $\vec{b}$ vectors plus residual vectors perpendicular to the plane spanned by $\vec{a}$ and $\vec{b}$,
\begin{equation}
\begin{pmatrix}
\vec{\gamma}^0 \\
\vec{\gamma}^1
\end{pmatrix}=
M
\begin{pmatrix}
\vec{a} \\
\vec{b}
\end{pmatrix}+
\begin{pmatrix}
\vec{\epsilon}_{\gamma^0} \\
\vec{\epsilon}_{\gamma^1}
\end{pmatrix}.
\label{trans_I:eqn}
\end{equation}
We find
\begin{equation}
M=
\begin{pmatrix}
\begin{array}{rr}
43.7^{+2.3}_{-2.1} & 19.4^{+0.5}_{-0.5} \\
-18.7^{+3.0}_{-3.1} & -0.8^{+0.9}_{-1.0} \\
\end{array}
\end{pmatrix} ,
\end{equation}
\begin{align}
\begin{split}
\vec{\epsilon}_{\gamma^0} &=\left(1.5^{+0.3}_{-0.3} , -1.4^{+0.3}_{-0.3} , -1.8^{+0.4}_{-0.3} , 1.4^{+0.1}_{-0.1} , 0.5^{+0.4}_{-0.5} \right) \text{mag}, \\
\vec{\epsilon}_{\gamma^1} & =\left(4.8^{+0.3}_{-0.3} , -4.3^{+0.3}_{-0.3} , -5.7^{+0.4}_{-0.4} , 0.1^{+0.2}_{-0.2} , 6.8^{+0.5}_{-0.5}\right) \text{mag},
\end{split}
\label{res_1:eqn}
\end{align}
which correspond to
a quadratic fraction of $0.9993^{+0.0002}_{-0.0003}$ of $\vec{\gamma}^0$ and
$0.9303^{+0.0210}_{-0.0341}$ of $\vec{\gamma}^1$ projecting onto the $\vec{a}$--$\vec{b}$ plane.
The  contributions of these residuals to magnitudes are relatively small compared to
the values of $\vec{\gamma}^0$ and $\vec{\gamma}^1$ in Table~\ref{global1:tab}.
For the standard deviation values of $g_0$ and $g_1$ of the supernovae in our sample, the amplitude of the magnitude residuals due to $\vec{\epsilon}$ are 
$\left(0.025, 0.023,0.029, 0.008, 0.027\right)$ mag.
The allowed color
variations in ${\hat{U}}{\hat{B}}{\hat{V}}{\hat{R}}{\hat{I}}$ allowed by the \citetalias{1999PASP..111...63F} model and our best-fit model are confined to almost identical
2-dimensional
planes within
the 5-dimensional magnitude space.
With no prior assumptions of dust extinction behavior or the distribution of $A_V$, the supernova data themselves exhibit
2-dimensional color variations that are closely aligned with the 2-dimensional color variations predicted by the \citetalias{1999PASP..111...63F} dust model.


The above result is visualized in Figure~\ref{plane:fig}, which  shows  in the
${\hat{U}}{\hat{V}}\hat{I}$-subspace
two perspectives (left and right panels)
of the unit vectors corresponding to $\vec{\gamma}^0$ and  $\vec{\gamma}^1$ of our model (solid lines),
and $\vec{a}$, $\vec{b}$ of the \citetalias{1999PASP..111...63F} model (dashed lines).  All  vectors are set to intersect the origin.
The two perspectives show that while the four vectors point in different directions for each band combination,
they are almost coplanar in ${\hat{U}}{\hat{V}}{\hat{I}}$ (they are slightly less coplanar in ${\hat{B}}{\hat{V}}{\hat{R}}$).  The $\vec{a}$ and $\vec{b}$ vectors and the $\vec{\gamma}^0$ and $\vec{\gamma}^1$
vectors span near-parallel planes in the 5-dimensional color space, and hence the latter can be almost entirely expressed in terms of
the former.

\begin{figure}[htbp] %  figure placement: here, top, bottom, or page
   \centering
   \includegraphics[width=2.95in]{fix1plane0.pdf}
   \includegraphics[width=2.95in]{fix1plane1.pdf}
   \caption{
   Visualization of how supernova magnitudes can vary in Model~I and that of \citetalias{1999PASP..111...63F}.  While the models describe
   magnitudes in 5-bands ${\hat{U}}{\hat{B}}{\hat{V}}{\hat{R}}{\hat{I}}$ this visualization shows only ${\hat{U}}{\hat{V}}{\hat{I}}$.   The left and right plots show the same information from
   two different perspectives.
   Our model is shown in solid-line unit vectors in the directions of $\vec{\gamma}^0$ and $\vec{\gamma}^1$. The only possible magnitudes
   are a linear combination of these two vectors, and hence are confined to the plane that contains both.
   The  \citetalias{1999PASP..111...63F} model is shown in dashed-line unit vectors
   in the directions of  $\vec{a}$, $\vec{b}$; the only possible magnitudes are confined to the plane that contains both.
   The perspective on the right visually shows that all four vectors are nearly coplanar.  
   The combination $\vec{a}+\vec{b}/2.25$ is shown in the dotted red
   line: it is almost perfectly superimposed on $\vec{\gamma}^0$.
   \label{plane:fig}}
\end{figure}


We
now turn to
the specific choice of $\vec{\gamma}$'s  returned by the fit in terms of the \citetalias{1999PASP..111...63F}  dust-extinction model. 
Our linear model would be satisfied by any two independent vectors  that span
the plane defined by the $\vec{\gamma}$ vectors. 
While
Eq.~\ref{f99:eqn} expresses that model using basis vectors $\vec{a}$ and $\vec{b}$ and
their corresponding parameters 
$A_V$ and 
$E(B-V)$, it can be written more generally using arbitrary bases $\vec{a}+\kappa_1 \vec{b}$
and $\kappa_2 \vec{a} + \vec{b}$, and parameter sets
$A^F_V - \kappa_2 E^F(B-V)$ and $-\kappa_1 A^F_V + E^F(B-V)$  such that 
\begin{equation}
A_X =  (1-\kappa_1 \kappa_2)^{-1} [(A^F_V - \kappa_2 E^F(B-V))\left(a_X+\kappa_1 b_X \right) +  (-\kappa_1 A^F_V + E^F(B-V)) (\kappa_2 a_X + b_X)],
\label{newdust:eqn}
\end{equation}
where $\kappa_1$ and $\kappa_2$ are free to float.
In our analysis, the $\vec{\gamma}$ vectors converge to a specific direction, meaning that our result prefers specific values of
$\kappa_1=M[1,2]/M[1,1]$ and $\kappa_2=M[2,1]/M[2,2]$, where $M$ is the matrix in  Eq.~\ref{trans_I:eqn}.

The values of $\kappa$'s are determined by the conditions that the components $A^F_V - \kappa_2 E^F(B-V)$ and $-\kappa_1 A^F_V + E^F(B-V)$ be
uncorrelated with $\Delta$.   The corresponding basis vectors are $a_X+\kappa_1 b_X $ and $\kappa_2 a_X + b_X$.
Defining an effective total to selective extinction $\langle R^F \rangle \equiv \kappa_1^{-1}$ so that the first basis vector is
$a_X+ b_X/\langle R^F \rangle  $, we find from the 68\%-ile credible interval of $\kappa_1^{-1}$ that
$\langle R^F \rangle=2.25^{+0.11}_{-0.10}$.
The non-correlation condition does not strongly constrain the  second basis vector as
$\kappa_2 =  15.60^{+26.19}_{-32.47}$.
Figure~\ref{plane:fig} visualizes the near-perfect alignment of $\vec{\gamma}^0$ and  $\vec{a}+\vec{b}/2.25$.


\subsection{Model~I Summary}
There are two points to highlight from the analysis of
Model~I.
\begin{itemize}
\item Model~I introduces two latent supernova parameters that can influence supernova colors.
These parameters come  with ``non-informative'' priors.  Fitting
of our model gives a space of possible color variations highly consistent with those expected from two-parameter dust-extinction models. 
That we recover a physically-motivated  result lends confidence in the modeling and analysis.
\item In Model~I, any  color variation that remains after spectral corrections is allocated to two parameters.
Dust extinction is not the only effect that can contribute to $g_0$ and $g_1$ so interpreting
them as pure dust parameters is subject to systematic errors.
\item We have defined an effective parameter $\langle R^F\rangle$ whose value may be compared with dust models.
Such comparisons must be taken with a grain of salt,
as this value is not fundamentally constrained by the model but is fixed by the added condition that $\Delta$ is
uncorrelated with  $EW_{\text{Ca}}$, $EW_{\mathrm{Si}}$, $\lambda_{\text{Ca}}$,
$g_0$, and $g_1$.  The value of  $\langle R^F\rangle$ changes given different ansatzes  for breaking the $\Delta$--cofactor degeneracy 
\end{itemize}

\section{Model~II: Three Color Parameters}
\label{modelII:sec}
Model~II extends Model~I to include a third color parameter.  This increase in the number of degrees of freedom 
goes beyond the two of dust-extinction models used in supernova analysis.
 In this section we  focus on describing the consequences of the addition of the
new parameter.

\subsection{Model}
\label{modelIImodel:sec}
Model~II is identical to Model~I except for the 
addition of a third color parameter $p$.  The distribution of $p$ in our sample
is assumed to form a Normal distribution with unit standard deviation.  An unconstrained distribution,
as taken for $g_0$ and $g_1$, yields multiple local maxima in the posterior and leads to difficulty in the convergence of the MCMC.
The relative effect on the five magnitudes is specified by the global parameters 
$\sigma_p\vec{\phi}$, where $\vec{\phi}$ is a 5-dimensional unit vector.
Model~II reads as
\begin{equation}
p  \sim \mathcal{N}(0,1 ),
\end{equation}
\begin{equation}
\sigma_p  \sim \text{Cauchy}(0.1,0.1),
\end{equation}
\begin{equation}
\begin{pmatrix}
{\hat{U}}\\{\hat{B}}\\{\hat{V}}\\{\hat{R}}\\{\hat{I}}
\end{pmatrix}
=
\Delta  \vec{1} +
\begin{pmatrix}
c_{\hat{U}}+\alpha_{\hat{U}} EW_{\mathrm{Ca}} + \beta_{\hat{U}} EW_{\mathrm{Si}} + \eta_{\hat{U}} \lambda_{\mathrm{Si}}+\gamma^0_{{\hat{U}}} g_0 +\gamma^1_{{\hat{U}}} g_1  + \sigma_p\phi_{\hat{U}} p\\
c_{\hat{B}}+\alpha_{\hat{B}} EW_{\mathrm{Ca}} + \beta_{\hat{B}} EW_{\mathrm{Si}} + \eta_{\hat{B}} \lambda_{\mathrm{Si}} +\gamma^0_{{\hat{B}}} g_0 +\gamma^1_{{\hat{B}}} g_1  + \sigma_p\phi_{\hat{B}} p\\
c_{\hat{V}}+\alpha_{\hat{V}} EW_{\mathrm{Ca}} + \beta_{\hat{V}} EW_{\mathrm{Si}} + \eta_{\hat{V}} \lambda_{\mathrm{Si}} +\gamma^0_{{\hat{V}}} g_0 +\gamma^1_{{\hat{V}}} g_1 + \sigma_p\phi_{\hat{V}}  p\\
c_{\hat{R}}+\alpha_{\hat{R}} EW_{\mathrm{Ca}} + \beta_{\hat{R}} EW_{\mathrm{Si}} + \eta_{\hat{R}} \lambda_{\mathrm{Si}} +\gamma^0_{{\hat{R}}} g_0 +\gamma^1_{{\hat{R}}} g_1 + \sigma_p\phi_{\hat{R}} p\\
c_{\hat{I}}+\alpha_{\hat{I}} EW_{\mathrm{Ca}} + \beta_{\hat{I}} EW_{\mathrm{Si}}+ \eta_{\hat{I}} \lambda_{\mathrm{Si}}+\gamma^0_{{\hat{I}}} g_0 +\gamma^1_{{\hat{I}}} g_1 + \sigma_p\phi_{\hat{I}}  p
\end{pmatrix}.
\label{ewsiv2:eqn}
\end{equation}

The Cauchy prior is commonly used to concentrate probability around a specific value while having fat tails with infinite variance.


The model contains the product 
$\vec{\phi} p$, which has the degeneracy $\vec{\phi} \rightarrow -\vec{\phi}$, $p \rightarrow -p$.
No conditions are added to break this degeneracy to leave open the possibility that
$\sigma_p\vec{\phi}$ is consistent with zero.  The degeneracy
in $\vec{\phi}$ is apparent later in the article, but it does not affect the magnitude predictions. 
As in \S\ref{modelI:sec}, the degeneracy between $\vec{\phi}$ and $\Delta$ is fixed by enforcing zero correlation between $\Delta$ and $p$.

For $N$ supernovae there are $8N$ observables.  There are $3N$ spectral parameters, each of
$\Delta$, $g_0$, $g_1$ contributes $N-1$ parameters while $p$  contributes $N$ parameters,  and  there are $5 \times 7$ global coefficients.
For $N=172$ supernovae, there are 1376 observables and 1236  parameters.

\subsection{Results for the Latent Parameter Cofactor $\sigma_p \vec{\phi}$}
\label{results2:sec}
The credible intervals of all Model~II global parameters are given in  \S\ref{m12results:sec}.
In this subsection we focus on the effect of adding the third color term  $\sigma_p \vec{\phi}$.
There is strong evidence for a third color parameter.
The 68\% credible interval for $\sigma_p$ is $0.051^{+0.005}_{-0.004}$~mag, highly incompatible with zero. 
The posterior for $\sigma_p \vec{\phi}$ is shown in Figure~\ref{M1ev:fig}.  
Two maxima are allowed due to the sign degeneracy for $\vec{\phi}$.  Practically each of the MCMC chains converges to one of the solutions and does
not migrate between them by crossing zero.  To aid visibility of this figure we thus impose $\phi_I > 0$.  
 The new parameter does affect colors in a  manner distinct from $g_0$ and $g_1$.
A quadratic fraction of
$0.48^{+0.25}_{-0.17}$ of 
 $\vec{\phi}$ lies perpendicular to the $\vec{\gamma}_0$--$\vec{\gamma}_1$ plane.


\begin{figure}[htbp] %  figure placement: here, top, bottom, or page
   \centering
   \includegraphics[width=4in]{fix3sigev.pdf} 
            \caption{Model~II posterior  contours for $\sigma_p \vec{\phi}$. 
            The solid line shows the location of zero.          \label{M1ev:fig}}
\end{figure}

Model~II introduces a new parameter  that can absorb color variability
that in Model~I is restricted to $g_0$ and $g_1$, and whose distribution comes from a Normal distribution.
The change in the distributions of the $\gamma^0_{\hat{V}} g_0$ and $\gamma^1_{\hat{V}}g_1$ parameters from
 Model~I to Model~II
is apparent in Figure~\ref{kcomp:fig}.  (Here and in similar plots, the parameter value of one supernova,
e.g.\ $\gamma^0_{\hat{V}} g_0|_0$, is subtracted
out to null out correlated
parameter uncertainty between supernovae; recall the $\langle \Delta \rangle=0$, $\langle g_0 \rangle=0$, $\langle g_1 \rangle=0$,
and $p  \sim \mathcal{N}(0,1)$ conditions.) The  change from Model~I to II is pronounced in  $\gamma^1_{\hat{V}}g_1$,
where  a Normal distribution component apparent in Model~I disappears in Model~II, presumably having been  assigned to $p$.


\begin{figure}[htbp] %  figure placement: here, top, bottom, or page
   \centering
   \includegraphics[width=2.6in]{fix1deltagamma0_med.pdf} 
   \includegraphics[width=2.6in]{fix1deltagamma1_med.pdf} 
   \includegraphics[width=2.6in]{fix3deltagamma0_med.pdf} 
   \includegraphics[width=2.6in]{fix3deltagamma1_med.pdf} 
   \caption{
   Left:
   Normalized stack of the posteriors (includes parameter uncertainties) of all supernovae  and histogram 
      of  per-supernova 
median values of $\gamma^0_{\hat{V}} g_0-\gamma^0_{\hat{V}} g_0|_0$.  Right:  The same plots for $\gamma^1_{\hat{V}} g_1-\gamma^1_{\hat{V}} g_1|_0$,
Top: Model~I. Bottom: Model~II.
    \label{kcomp:fig}}
\end{figure}

The $\vec{\gamma}$ vectors, like those from Model~I, are almost entirely projected
onto the plane defined by
$\vec{a}$ and $\vec{b}$ of the \citetalias{1999PASP..111...63F} model.
The direction of the  $\vec{\gamma}^0$ vector corresponds to an effective
$\langle R^F_{\mathit{eff}}\rangle = 2.54^{+0.13}_{-0.12}$.
The change of $\langle R^F_{\mathit{eff}}\rangle$ from Model~I to II is anticipated by the change in the distribution of $g_1$,
whose connection with $R^F$ is shown in \S\ref{g0g1:sec}.


\subsection{Model~II Summary}
The inclusion of a third color term beyond the two of Model~I has important consequences.

\begin{itemize}
\item The data strongly supports the influence of an additional supernova parameter that affects supernova colors in a manner orthogonal to
 dust extinction.
\item The effective $\langle R^F_{\mathit{eff}}\rangle$ shifts by $>1 \sigma$ when including the new color term.  This demonstrates,
as has been noted by \citet{2009ApJ...704.1036F,2009ApJS..185...32K}, that dust analyses
that do not allow for additional color dispersion are susceptible to bias when inferring dust parameters.
\end{itemize}

\section{Model~III: Three Color Parameters and Light-Curve Shape}
\label{modelIII:sec}

Light-curve shape is an established indicator of SN~Ia diversity 
\citep{1993ApJ...413L.105P, 1996ApJ...473...88R,
1997ApJ...483..565P}
with its own color correlations \citep{2005A&A...443..781G, 2007ApJ...659..122J}.
Model~III explicitly includes light-curve shape as a feature used to standardize supernova colors, to complement the
spectral features considered in the previous two models.
We will show that light-curve shape does have a non-trivial relationship with color that is not already encoded
in the spectral features.  Model~III is thus the focus of
this article and its results are  described here in depth.


\subsection{Model}
\label{modelIImodel:sec}
Model~III is identical to Model~II except for the 
addition of a new term  that relates magnitudes with the SALT2 light-curve-shape parameter,
$x_1$, through linear coefficients $\vec{\zeta}$. 
The model reads as
\begin{equation}
p  \sim \mathcal{N}(0,1 ),
\end{equation}
\begin{equation}
\sigma_p  \sim \text{Cauchy}(0.1,0.1),
\end{equation}
\begin{equation}
\begin{pmatrix}
{\hat{U}}\\{\hat{B}}\\{\hat{V}}\\{\hat{R}}\\{\hat{I}}
\end{pmatrix}
=
\Delta  \vec{1} +
\begin{pmatrix}
c_{\hat{U}}+\alpha_{\hat{U}} EW_{\mathrm{Ca}} + \beta_{\hat{U}} EW_{\mathrm{Si}} + \eta_{\hat{U}} \lambda_{\mathrm{Si}} +\zeta_{\hat{U}} x_1 +\gamma^0_{{\hat{U}}} g_0 +\gamma^1_{{\hat{U}}} g_1 + \sigma_p\phi_{\hat{U}} p\\
c_{\hat{B}}+\alpha_{\hat{B}} EW_{\mathrm{Ca}} + \beta_{\hat{B}} EW_{\mathrm{Si}} + \eta_{\hat{B}} \lambda_{\mathrm{Si}} +\zeta_{\hat{B}} x_1 +\gamma^0_{{\hat{B}}} g_0 +\gamma^1_{{\hat{B}}} g_1 + \sigma_p\phi_{\hat{B}} p\\
c_{\hat{V}}+\alpha_{\hat{V}} EW_{\mathrm{Ca}} + \beta_{\hat{V}} EW_{\mathrm{Si}} + \eta_{\hat{V}} \lambda_{\mathrm{Si}} +\zeta_{\hat{V}} x_1+\gamma^0_{{\hat{V}}} g_0 +\gamma^1_{{\hat{V}}} g_1 + \sigma_p\phi_{\hat{V}}  p\\
c_{\hat{R}}+\alpha_{\hat{R}} EW_{\mathrm{Ca}} + \beta_{\hat{R}} EW_{\mathrm{Si}} + \eta_{\hat{R}} \lambda_{\mathrm{Si}} +\zeta_{\hat{R}} x_1+\gamma^0_{{\hat{R}}} g_0 +\gamma^1_{{\hat{R}}} g_1 + \sigma_p\phi_{\hat{R}} p\\
c_{\hat{I}}+\alpha_{\hat{I}} EW_{\mathrm{Ca}} + \beta_{\hat{I}} EW_{\mathrm{Si}}+ \eta_{\hat{I}} \lambda_{\mathrm{Si}}+\zeta_{\hat{I}} x_1+\gamma^0_{{\hat{I}}} g_0 +\gamma^1_{{\hat{I}}} g_1 + \sigma_p\phi_{\hat{I}}  p
\end{pmatrix},
\label{ewsiv3:eqn}
\end{equation}

\begin{equation}
\begin{pmatrix}
{\hat{U}}_o\\{\hat{B}}_o\\ {\hat{V}}_o\\{\hat{R}}_o\\{\hat{I}}_o\\EW_{Si, o}\\ EW_{Ca, o} \\ \lambda_{Si, o} \\ x_{1, o}
\end{pmatrix}
\sim \mathcal{N}
\left(
\begin{pmatrix}
{\hat{U}}  \\{\hat{B}}  \\
{\hat{V}}\\{\hat{R}}\\{\hat{I}}\\
EW_{\mathrm{Si}}\\ EW_{\mathrm{Ca}} \\ \lambda_{\mathrm{Si}} \\ x_1
\end{pmatrix}
,C
\right).
\label{dust3:eqn}
\end{equation}
The changes from Model~II are the addition of the $\vec{\zeta} x_1$ terms in Eq.~\ref{ewsiv3:eqn}, and
the addition of the shape measurement and uncertainty (implicit in the covariance matrix $C$) and $x_1$   in Eq.~\ref{dust3:eqn}.
As in \S\ref{modelI:sec}, the degeneracy between $\vec{\zeta}$ and $\Delta$ is fixed by enforcing zero correlation between $\Delta$ and $x_1$.
For $N$ supernovae there are $9N$ observables.  There are $4N$ spectral and light-curve shape parameters, each of
$\Delta$, $g_0$, $g_1$ contributes $N-1$ parameters while $p$  contributes $N$ parameters,  and there are $5 \times 8$ global coefficients.
For $N=172$ supernovae, this makes 1548 observables and 1413  parameters.


\subsection{Results}
\subsubsection{Analysis Method and Validation}

In the analysis of this and previous sections of this article,
the posterior of the model parameters is evaluated using Hamiltonian Monte Carlo with a No-U-Turn
Sampler as implemented in
STAN \citep{JSSv076i01}.  We run eight chains, each with 5000 iterations of which
half are used for warmup.
STAN provides output statistics to assess
the convergence of the output Markov chains.
The 
potential scale reduction statistic, $\hat{R}$
\citep[][in this paragraph not to be confused with the synthetic $\hat{R}$-band magnitude]{Gelman92}, measures the convergence of the target distribution
in iterative simulations 
by using multiple independent sequences to estimate how much that distribution would sharpen if the simulations were run longer.
$N_{\mathit{eff}}$ is an estimate of the number of independent draws. The STAN output gives $\hat{R} \sim 1.0$ for all parameters, meaning there is no evidence for non-convergence.  The
output also gives  $N_{\mathit{eff}} \gg 100$ for all parameters, indicating that they are densely sampled.


Our analysis pipeline is run with input simulated data with known  signals and measurement uncertainties similar to those of our dataset.
Stacks of the posteriors from 100 simulated datasets  exhibit no biases that are significant relative to the statistical uncertainties of a single posterior.

As will be seen in  \S\ref{results3global:sec}, the posterior derived from the analysis has a distinct bounded  
peak showing no indication that the MCMC has failed to converge.

The model provides a fair representation of the data as seen in the residual differences between the data and the model prediction.
Figure~\ref{residual:fig} shows plots of 
residuals between measured and model-expectation colors as a function of observed $\hat{B}_o-\hat{V}_o$.
This and similar plots for the other permutations of observed colors show  no apparent trend between residuals and observed
color nor any catastrophic outliers.

\begin{figure}[htbp] %  figure placement: here, top, bottom, or page
   \centering
   \includegraphics[width=4in]{cpull.pdf} 
               \caption{
               Residuals between measured and model-expectation colors as a function of observed $\hat{B}_o-\hat{V}_o$.  The colors are
               relative to on arbitrary supernova. Plots that show
               residual
               colors containing $\hat{B}_o$ or $\hat{V}_o$ are shaded to highlight that the abscissa and ordinate have contributions
               from  a common measurement.
            \label{residual:fig}}
\end{figure}

This suite of tests lend confidence in our model and the validity of the derived posterior.

\subsubsection{Global Parameter Posterior Results}
\label{results3global:sec}
Results from our analysis are shown in Figures~\ref{global1:fig} -- \ref{global5:fig}
as contours of the posterior surface for pairs of global parameters grouped by filter.
The confidence regions are localized and unimodal, except for the combination $\sigma_p \vec{\phi}$ that
has the sign degeneracy explained in \S\ref{results2:sec} and plotted in Figure~\ref{M1ev:fig}.
Within the finite number of generated links, the MCMC chains do not migrate between the parity-degenerate $\vec{\gamma}$--$g$ solutions.
Each chain converges to one of the  $\vec{\phi}$--$p$ degenerate solutions, away from which it does not depart.  

\begin{figure}[htbp] %  figure placement: here, top, bottom, or page
   \centering
   \includegraphics[width=5.2in]{coeff0.pdf} 
            \caption{Model~III posterior contours for $\vec{c}$, $\vec{\alpha}$, $\vec{\beta}$, $\vec{\eta}$, $\vec{\gamma}^0$, $\vec{\gamma}^1$, and $\sigma_p \vec{\phi}$ in the ${\hat{U}}$ band.
            The contours shown here and in future plots represent 1-$\sigma$ in the parameter distribution (i.e.\ they should be
            projected onto the corresponding 1-d parameter axis), not to 68\%, 95\%, etc.\
            enclosed probability.  Lines for zero value for $\alpha_{\hat{U}}$, $\beta_{\hat{U}}$, $\eta_{\hat{U}}$, $\gamma_{\hat{U}}^0$, $\gamma_{\hat{U}}^1$, and $\sigma_p \phi_{\hat{U}}$ are shown for reference --
            in most cases they are outside the range of the plot.
            \label{global1:fig}}
\end{figure}

\begin{figure}[htbp] %  figure placement: here, top, bottom, or page
   \centering
   \includegraphics[width=5.2in]{coeff1.pdf} 
            \caption{Model~III posterior contours for $\vec{c}$, $\vec{\alpha}$, $\vec{\beta}$, $\vec{\eta}$,  $\vec{\gamma}^0$, $\vec{\gamma}^1$, and $\sigma_p \vec{\phi}$ in the ${\hat{B}}$ band.
 \label{global2:fig}}
\end{figure}

\begin{figure}[htbp] %  figure placement: here, top, bottom, or page
   \centering
   \includegraphics[width=5.2in]{coeff2.pdf} 
            \caption{Model~III posterior contours for $\vec{c}$, $\vec{\alpha}$, $\vec{\beta}$, $\vec{\eta}$, $\vec{\gamma}^0$, $\vec{\gamma}^1$, and $\sigma_p \vec{\phi}$ in the ${\hat{V}}$ band.
 \label{global3:fig}}
\end{figure}

\begin{figure}[htbp] %  figure placement: here, top, bottom, or page
   \centering
      \includegraphics[width=5.2in]{coeff3.pdf} 
            \caption{Model~III posterior contours for  $\vec{c}$, $\vec{\alpha}$, $\vec{\beta}$, $\vec{\eta}$,  $\vec{\gamma}^0$, $\vec{\gamma}^1$, and $\sigma_p \vec{\phi}$ in the ${\hat{R}}$ band.
 \label{global4:fig}}
\end{figure}

\begin{figure}[htbp] %  figure placement: here, top, bottom, or page
   \centering
         \includegraphics[width=5.2in]{coeff4.pdf} 
            \caption{Model~III posterior contours for  $\vec{c}$, $\vec{\alpha}$, $\vec{\beta}$, $\vec{\eta}$, $\vec{\gamma}^0$, $\vec{\gamma}^1$, and $\sigma_p \vec{\phi}$ in the ${\hat{I}}$ band.
 \label{global5:fig}}
\end{figure}


For each of the five filters, the 68\%  equal-tailed credible intervals for the global parameters $\vec{\alpha}$, $\vec{\beta}$, $\vec{\eta}$, $\vec{\zeta}$,
$\vec{\gamma}^0$, $\vec{\gamma}^1$, and $\sigma_p\vec{\phi}$
are given in Table~\ref{global2:tab}.
In constructing the credible interval of $\sigma_p\vec{\phi}$ 
we set $\vec{\phi} = -\text{sign}(\phi_{\hat{V}}) \vec{\phi}$ to break its parity degeneracy;
Table~\ref{global3:tab} shows it has the largest signal-to-noise in
in the $\hat{V}$-band.
The effect of spectral parameters on color (as opposed to magnitude)
is shown in the rows of $\alpha_X/\alpha_{\hat{V}}-1$,  $\beta_X/\beta_{\hat{V}}-1$, and  $\eta_X/\eta_{\hat{V}}-1$.
The normalization freedom of $\vec{\gamma}$ is nulled out in the statistic
 $\gamma_X/\gamma_{\hat{V}}-1$.
 These statistics give the change in $X-\hat{V}$ color normalized to the change in $\hat{V}$ magnitude, and are analogous to
$R_V^{-1}=A_B/A_V-1$ as an observational descriptor of dust properties.

\begin{table}
\centering
\begin{tabular}{|c|c|c|c|c|c|}
\hline
Parameters& $X={\hat{U}}$ &${\hat{B}}$&${\hat{V}}$&${\hat{R}}$&${\hat{I}}$\\ \hline
$\alpha_X$
& $0.0051^{+0.0007}_{-0.0008}$
& $0.0025^{+0.0006}_{-0.0007}$
& $0.0022^{+0.0005}_{-0.0005}$
& $0.0019^{+0.0004}_{-0.0004}$
& $0.0034^{+0.0003}_{-0.0004}$
\\
${\alpha_X}/\alpha_{\hat{V}}-1$
& $   1.3^{+   0.4}_{  -0.2}$
& $   0.1^{+   0.1}_{  -0.1}$
& \ldots
& $  -0.1^{+   0.0}_{  -0.0}$
& $   0.5^{+   0.3}_{  -0.2}$
\\
$\beta_X$
& $ 0.035^{+ 0.005}_{-0.005}$
& $ 0.024^{+ 0.004}_{-0.004}$
& $ 0.021^{+ 0.004}_{-0.004}$
& $ 0.019^{+ 0.003}_{-0.003}$
& $ 0.010^{+ 0.003}_{-0.003}$
\\
${\beta_X}/\beta_{\hat{V}}-1$
& $  0.68^{+  0.16}_{ -0.13}$
& $  0.14^{+  0.07}_{ -0.07}$
&\ldots
& $ -0.11^{+  0.03}_{ -0.02}$
& $ -0.50^{+  0.06}_{ -0.07}$
\\
$\eta_X$
& $0.0003^{+0.0011}_{-0.0011}$
& $0.0005^{+0.0009}_{-0.0009}$
& $0.0008^{+0.0007}_{-0.0007}$
& $0.0007^{+0.0006}_{-0.0006}$
& $-0.0000^{+0.0005}_{-0.0005}$
\\
${\eta_X}/\eta_{\hat{V}}-1$
& $ -0.39^{+  0.64}_{ -1.36}$
& $ -0.26^{+  0.35}_{ -0.77}$
&\ldots
& $ -0.19^{+  0.12}_{ -0.11}$
& $ -0.86^{+  0.33}_{ -0.71}$
\\
$\zeta_X$
& $ -0.00^{+  0.04}_{ -0.04}$
& $ -0.02^{+  0.03}_{ -0.03}$
& $ -0.05^{+  0.03}_{ -0.03}$
& $ -0.03^{+  0.02}_{ -0.02}$
& $ -0.08^{+  0.02}_{ -0.02}$
\\
${\zeta_X}/\zeta_{\hat{V}}-1$
& $ -0.94^{+  0.48}_{ -1.15}$
& $ -0.54^{+  0.28}_{ -0.65}$
& \ldots
& $ -0.42^{+  0.09}_{ -0.21}$
& $  0.47^{+  0.83}_{ -0.32}$
\\
$\gamma^0_X$
& $ 68.91^{+  2.62}_{ -2.56}$
& $ 55.44^{+  2.22}_{ -2.25}$
& $ 41.41^{+  1.98}_{ -1.98}$
& $ 31.49^{+  1.61}_{ -1.62}$
& $ 22.92^{+  1.35}_{ -1.30}$
\\
${\gamma^0_X}/\gamma^0_{\hat{V}}-1$
& $  0.67^{+  0.04}_{ -0.04}$
& $  0.34^{+  0.02}_{ -0.02}$
&\ldots
& $ -0.24^{+  0.01}_{ -0.01}$
& $ -0.45^{+  0.02}_{ -0.02}$
\\
$\gamma^1_X$
& $ -1.85^{+  4.11}_{ -4.05}$
& $  1.47^{+  3.74}_{ -3.58}$
& $ -6.75^{+  3.21}_{ -3.15}$
& $ -6.60^{+  2.71}_{ -2.65}$
& $ -7.65^{+  2.33}_{ -2.24}$
\\
${\gamma^1_X}/\gamma^1_{\hat{V}}-1$
& $ -0.70^{+  0.38}_{ -0.76}$
& $ -1.18^{+  0.43}_{ -1.08}$
& \ldots
& $ -0.03^{+  0.14}_{ -0.08}$
& $  0.12^{+  0.45}_{ -0.22}$
\\
$\sigma_p \phi_X$
& $ 0.002^{+ 0.016}_{-0.017}$
& $-0.041^{+ 0.013}_{-0.014}$
& $-0.050^{+ 0.011}_{-0.012}$
& $-0.032^{+ 0.009}_{-0.010}$
& $ 0.002^{+ 0.008}_{-0.008}$
\\
${\phi_X/\phi_{\hat{V}}-1}$
& $-1.034^{+ 0.291}_{-0.409}$
& $-0.174^{+ 0.103}_{-0.134}$
& \ldots
& $-0.347^{+ 0.042}_{-0.060}$
& $-1.032^{+ 0.141}_{-0.208}$
\\

\hline
\end{tabular}
\caption{68\% credible intervals for the global fit parameters of Model~III in \S\ref{modelIII:sec}.\label{global3:tab}}
\end{table}



\subsubsection{Results for the Observables Cofactors $\vec{\alpha}$, $\vec{\beta}$, $\vec{\eta}$,  $\vec{\zeta}$}

The cofactors  $\vec{\alpha}$, $\vec{\beta}$, and $\vec{\eta}$ have  significant non-zero values,  meaning that
$EW_{\mathrm{Ca}}$, $EW_{\mathrm{Si}}$, and $\lambda_{\mathrm{Si}}$ are indicators of broadband
colors at peak.  The significance of their influence varies between  colors; in $\hat{B}-\hat{V}$, roughly the color
where previous studies have focused, only $\vec{\beta}$ has a strong $>2\sigma$ correlation.  A positive
dependence (consistent with the positive sign
of $\beta_{\hat{B}}/\beta_{\hat{V}}-1$)
of $B-V$ on the pseudo-equivalent width of Si~II~$\lambda$4130 has previously been reported by
\citet{2011ApJ...734...42N, 2013ApJ...773...53F}.
The insignificance of  our $\alpha_{\hat{B}}/\alpha_{\hat{V}}-1$ is consistent with the
lack of correlation between the  pseudo-equivalent width of Ca~II~H\&K measurements and
intrinsic color reported by \citet{2011ApJ...742...89F}.

The story on Si~II~$\lambda$6355 is not as clear.
 \citet{2009ApJ...699L.139W,2011ApJ...742...89F, 2012ApJ...748..127F} find that
higher-velocity SNe~Ia tend to be redder than those with lower velocity. \citet{2012AJ....143..126B} find a
weaker relation than that reported by \citet{2011ApJ...742...89F},
a difference attributed to a better treatment of uncertainties and host-galaxy dust. In contrast,
\citet{2013ApJ...773...53F} find no significant relation.
Our result for $\eta_{\hat{B}}/\eta_{\hat{V}}-1$ also gives no significant relation. 

The significantly non-zero values for  $\alpha_{\hat{U}}$ and $\beta_{\hat{U}}$ from this study can be compared with the $U$-band study of
\citet{2018arXiv180101834N} carried out on a subset of the same supernovae.
Both analyses find a strong correlation between $U$ and the Si and Ca features.

The non-zero values for $\vec{\zeta}$ and their effect on color $\vec{\zeta}-\zeta_{\hat{V}}$  demonstrate that light-curve shape $x_1$ is  an indicator
of broadband colors, in the sense that SNe~Ia with broader light curves are bluer in all colors.
This is as expected based on the results of \citet{2007A&A...466...11G, 2007ApJ...659..122J}.


The signal in  $\vec{\alpha}$ and $\vec{\beta}$ cannot
entirely
be attributed to the equivalent widths themselves.
The range of Si~II~$\lambda$4130 equivalent widths is $\pm 20$~\AA\ whereas the width of the ${\hat{B}}$-band is 851~\AA, so that its direct affect on magnitude
is
$2.5 \log{(20/850)} \sim 0.03$ mag.  
The implied span in ${\hat{B}}$ magnitude based on $\beta_{\hat{B}}$ is 0.54~mag.  Therefore $\beta_{\hat{B}}$ cannot wholly be attributed to the flux deficit
from the line itself.
The Ca~II H\&K equivalent widths have range $\pm 50$~\AA, while the width of the ${\hat{U}}$ band is
701~\AA, so that its direct affect on magnitude
is
$2.5 \log{(50/701)} \sim 0.08$ mags.   The implied span in ${\hat{U}}$ magnitude is  0.21~mag, 
which would imply that  $\alpha_{\hat{U}}$ cannot be completely due to the 
presence of the line itself.  
Supernova flux has a large gradient in our $\hat{U}$ and there is a variation in the effective, flux-weighted,  $\hat{U}$ bandwidth. 
Nevertheless, the  correlation between spectral features with magnitudes in bands with no wavelength overlap is a non-trivial
signal detected in the analysis.

The posterior visualized in
Figures~\ref{global1:fig} -- \ref{global5:fig} shows
 that $\vec{\zeta}$, the magnitude vector associated with $x_1$, is highly correlated with $\vec{\beta}$, the vector
 associated with  $EW_{\mathrm{Si}}$.
The direct correlation between  $x_1$ and $EW_{\mathrm{Si}}$ has already been established
\citep{2008A&A...492..535A, 2011A&A...529L...4C}
and exists
in our sample as is shown later in \S\ref{results3per:sec}.  
Nevertheless the $\vec{\zeta}$ and $\vec{\beta}$   posteriors are bounded and inconsistent with zero, indicating that
 $x_1$ does convey color information independent of our set of spectral features.
For this reason  Model~III is highlighted in this article.

\subsubsection{Results for the Latent Parameter Cofactors $\vec{\gamma}^0$, $\vec{\gamma}^1$}
\label{results3gamma:sec}
As with Model~I in \S\ref{modelI:sec}, all elements of the two $\vec{\gamma}$ parameter vectors 
are significantly non-zero with 
none of the 20000 links of 
our Monte Carlo chains for $\vec{\gamma}$ extending to 0 (see Figures~\ref{global1:fig}--\ref{global5:fig}).
The direction of the vector in the ${\hat{U}}{\hat{V}}{\hat{I}}$-subspace is shown in Figure~\ref{plane3:fig}.

The transformation between the $\vec{\gamma}$  vectors and the   \citetalias{1999PASP..111...63F} vector given in
Eq.~\ref{trans_I:eqn}, is satisfied by the transformation matrix
\begin{equation}
M=
\begin{pmatrix}
\begin{array}{rr}
40.3^{+1.8}_{-1.8} & 16.6^{+0.6}_{-0.6} \\
-5.8^{+3.1}_{-3.0} & 3.4^{+1.0}_{-0.9}
\end{array}
\end{pmatrix} 
\end{equation}
and the two residual vectors
\begin{align}
\begin{split}
\vec{\epsilon}_{\gamma^0} &=\left(1.1^{+0.4}_{-0.4} , -1.2^{+0.4}_{-0.4} , -0.9^{+0.4}_{-0.4} , 0.7^{+0.2}_{-0.2} , 0.5^{+0.6}_{-0.5}\right) \text{mag}, \\
\vec{\epsilon}_{\gamma^1} & =\left(-2.3^{+0.6}_{-0.5} , 4.0^{+0.6}_{-0.6} , -1.3^{+0.6}_{-0.6} , 0.7^{+0.3}_{-0.3} , -1.4^{+0.8}_{-0.8}\right) \text{mag}.
\end{split}
\label{res_3:eqn}
\end{align}
This corresponds to a quadratic fraction of $0.9996^{+0.0002}_{-0.0003}$ of $\vec{\gamma}^0$ that projects onto the plane of
color changes allowed by the \citetalias{1999PASP..111...63F} dust-extinction model
as described by
$\vec{a}$ and $\vec{b}$. The quadratic fraction of $\vec{\gamma}^1$ projected
onto this plane is
$0.8426^{+0.0780}_{-0.1236}$.
For the standard deviation values of $g_0$ and $g_1$ of the supernovae in our sample, the amplitude of the magnitude residuals from $\vec{\epsilon}$ are 
$\left(0.005, 0.012,0.011, 0.007, 0.004 \right)$~mag.
The   $\vec{\gamma}^0$  vector points in the $\langle R^F_{\mathit{eff}}\rangle=2.43^{+0.10}_{-0.10}$ dust-extinction direction.
This value is consistent with the $R_V=2.5$ found in the supernova samples of \citet{2011ApJ...729...55F}, and the
center  of the range of per-supernova $R^F$ values found in the samples of \citet{2014ApJ...789...32B, 2015MNRAS.453.3300A}.
Although the model $\vec{\gamma}$'s cannot be attributed to dust, they behave like the   \citetalias{1999PASP..111...63F} model.

\begin{figure}[htbp] %  figure placement: here, top, bottom, or page
   \centering
   \includegraphics[width=2.95in]{plane0.pdf}
   \includegraphics[width=2.95in]{plane1.pdf}
   \caption{
   Visualization of how supernova magnitudes can vary in Model~III and the dust extinction of \citetalias{1999PASP..111...63F}.
   See Figure~\ref{plane:fig}.
   The combination $\vec{a}+\vec{b}/2.43$ is shown in the dotted red
   line: it is almost perfectly superimposed on $\vec{\gamma}^0$.
   \label{plane3:fig}}
\end{figure}


\subsubsection{Results for the Latent Parameter Cofactor $\sigma_p \vec{\phi}$}

The  new  $p$ parameter has a significant influence on color.  The magnitude of its color effect is non-zero,
with $\sigma_p = 0.053^{+ 0.010}_{-0.005}$~mag.  The smallest value of $\sigma_p$ % new
over all Monte Carlo chains is 0.040~mag.
These numbers are significantly larger than the median data color uncertainty of 0.010~mag (the maximum is 0.042~mag).
The evaluation of 100 simulated data sets with no third color parameter produces no median $\sigma_p$'s  as
high as that of the data, so we assign a $>99$\% confidence in the detection of this third color term.

The direction of $\vec{\phi}$ is not aligned with the plane defined by
the $\vec{\gamma}$ vectors (nor of the  \citetalias{1999PASP..111...63F} dust-extinction model); a quadratic fraction of
$0.49^{+0.27}_{-0.19}$ of  $\vec{\phi}$ lies perpendicular to the $\vec{\gamma}^0$--$\vec{\gamma}^1$ plane.

The values of $\vec{\phi}$ does not change  monotonically with wavelength, as is apparent  in 
the combination $\phi_X/\phi_{\hat{V}}-1$ shown in Figure~\ref{phiratio:fig}.  An object that has fainter  $\hat{U}$ or $\hat{I}$-band peak magnitudes
has brighter  $\hat{B}$$\hat{V}$$\hat{R}$  peaks.  That brightening is non-uniform, in that
the reddening of $\hat{U}-\hat{B}$ and $\hat{B}-\hat{V}$ is accompanied
by a bluing in $\hat{V}-\hat{R}$ and  $\hat{R}-\hat{I}$. 

What is measured as $\sigma_p \vec{\phi}$ likely captures the effect of many physical processes.
To simplify the model it is assumed that the $p$'s are drawn from a Normal distribution, which may not be a fair description of what occurs in nature
and result in model bias.
The physical interpretation of this color should be tempered with caution.

\begin{figure}[htbp] %  figure placement: here, top, bottom, or page
   \centering
      \includegraphics[width=4in]{phiratio.pdf}
   \caption{Model~III median values  and corresponding 68\% intervals for $\phi_X/\phi_{\hat{V}}-1$ in the 5 bands.
   A dotted line at zero is shown for reference.
   \label{phiratio:fig}}
\end{figure}


\subsubsection{Results for the Per-Supernova Parameters}
\label{results3per:sec}
Each supernova is described by its parameters $\Delta$, $EW_{\mathrm{Ca}}$, $EW_{\mathrm{Si}}$, $\lambda_{\mathrm{Si}}$, $x_1$,
$E_{\gamma^0}({\hat{B}}-{\hat{V}})=(\gamma^0_{\hat{B}}-\gamma^0_{\hat{V}})g_0$, $E_{\gamma^1}({\hat{B}}-{\hat{V}})=(\gamma^1_{\hat{B}}-\gamma^1_{\hat{V}})g_1$, and $A_{p,V} =  \sigma_p\phi_{\hat{V}}p$.  (The  three latent parameters are cast as physically meaning
quantities by multiplying them by select cofactors.)
\added{The 68\% credible intervals for these parameters for all supernovae are given in Table~\ref{results:tab}.
\startlongtable
\begin{deluxetable}{crrrrrrrrr}
\tabletypesize{\tiny}
\tablecaption{Supernova Parameter Credible Intervals
\label{results:tab}}
\tablehead{
\colhead{Name} & \colhead{$EW_{\mathrm{Ca}}$ (\AA)} & \colhead{$EW_{\mathrm{Si}}$ (\AA)} & \colhead{$\lambda_{\mathrm{Si}}$ (\AA)} & \colhead{$x_1$} & \colhead{$\Delta$}
& \colhead{$E_{\gamma^0}({\hat{B}}-{\hat{V}})$} & \colhead{$E_{\gamma^1}({\hat{B}}-{\hat{V}})$} & \colhead{$A_{p,V} $}
}
\startdata
SN2007bd & $113.3^{+5.4}_{-5.4}$ & $ 17.8^{+0.7}_{-0.7}$ & $ 6101^{+  3}_{-  3}$ & $ -1.02^{+  0.18}_{-  0.18}$ & $-0.071^{+0.027}_{-0.027}$  & $-0.050^{+0.011}_{-0.012}$ & $-0.042^{+0.007}_{-0.007}$ & $-0.042^{+0.007}_{-0.007}$\\
PTF10zdk & $149.7^{+1.2}_{-1.2}$ & $ 14.9^{+0.6}_{-0.6}$ & $ 6150^{+  3}_{-  3}$ & $  0.37^{+  0.14}_{-  0.13}$ & $-0.029^{+0.026}_{-0.025}$  & $0.062^{+0.012}_{-0.012}$ & $-0.042^{+0.007}_{-0.007}$ & $-0.042^{+0.007}_{-0.007}$\\
SNF20080815-017 & $ 90.9^{+14.3}_{-14.5}$ & $ 19.6^{+2.4}_{-2.3}$ & $ 6131^{+  6}_{-  6}$ & $ -1.96^{+  0.16}_{-  0.16}$ & $0.041^{+0.070}_{-0.064}$  & $-0.008^{+0.025}_{-0.026}$ & $-0.011^{+0.028}_{-0.022}$ & $-0.011^{+0.028}_{-0.022}$\\
PTF09dnl & $130.0^{+0.8}_{-0.9}$ & $  9.1^{+0.7}_{-0.7}$ & $ 6092^{+  3}_{-  3}$ & $  0.59^{+  0.10}_{-  0.10}$ & $-0.061^{+0.021}_{-0.021}$  & $0.014^{+0.009}_{-0.009}$ & $0.000^{+0.011}_{-0.011}$ & $0.000^{+0.011}_{-0.011}$\\
SN2010ex & $114.5^{+0.9}_{-0.9}$ & $  8.4^{+0.4}_{-0.4}$ & $ 6129^{+  6}_{-  6}$ & $  0.17^{+  0.12}_{-  0.11}$ & $0.266^{+0.026}_{-0.026}$  & $-0.042^{+0.008}_{-0.008}$ & $-0.039^{+0.010}_{-0.008}$ & $-0.039^{+0.010}_{-0.008}$\\
PTF09dnp & $ 66.9^{+4.5}_{-4.6}$ & $ 17.5^{+0.7}_{-0.7}$ & $ 6098^{+  5}_{-  4}$ & $ -0.53^{+  0.17}_{-  0.17}$ & $-0.249^{+0.026}_{-0.026}$  & $-0.008^{+0.020}_{-0.021}$ & $-0.044^{+0.006}_{-0.006}$ & $-0.044^{+0.006}_{-0.006}$\\
PTF11bnx & $150.7^{+3.0}_{-2.9}$ & $ 14.6^{+1.0}_{-1.0}$ & $ 6143^{+  5}_{-  5}$ & $  0.07^{+  0.14}_{-  0.14}$ & $0.043^{+0.029}_{-0.029}$  & $0.049^{+0.011}_{-0.011}$ & $-0.030^{+0.014}_{-0.011}$ & $-0.030^{+0.014}_{-0.011}$\\
PTF12jqh & $151.8^{+1.4}_{-1.5}$ & $  7.9^{+0.6}_{-0.7}$ & $ 6118^{+ 10}_{- 10}$ & $  0.73^{+  0.14}_{-  0.14}$ & $0.069^{+0.028}_{-0.028}$  & $-0.056^{+0.010}_{-0.010}$ & $0.009^{+0.014}_{-0.015}$ & $0.009^{+0.014}_{-0.015}$\\
SNF20080802-006 & $108.7^{+5.6}_{-5.8}$ & $ 19.1^{+1.6}_{-1.6}$ & $ 6122^{+  5}_{-  5}$ & $ -1.61^{+  0.12}_{-  0.12}$ & $-0.042^{+0.059}_{-0.055}$  & $-0.001^{+0.016}_{-0.016}$ & $0.033^{+0.020}_{-0.020}$ & $0.033^{+0.020}_{-0.020}$\\
PTF10xyt & $120.2^{+6.3}_{-6.4}$ & $ 21.5^{+3.2}_{-3.1}$ & $ 6102^{+  4}_{-  4}$ & $  0.18^{+  0.20}_{-  0.20}$ & $0.241^{+0.049}_{-0.049}$  & $0.070^{+0.026}_{-0.028}$ & $0.014^{+0.029}_{-0.027}$ & $0.014^{+0.029}_{-0.027}$\\
PTF11qmo & $101.7^{+1.1}_{-1.1}$ & $  7.8^{+0.7}_{-0.7}$ & $ 6151^{+  7}_{-  8}$ & $  0.41^{+  0.18}_{-  0.18}$ & $-0.078^{+0.031}_{-0.028}$  & $-0.058^{+0.010}_{-0.010}$ & $0.004^{+0.014}_{-0.015}$ & $0.004^{+0.014}_{-0.015}$\\
SNF20070331-025 & $118.7^{+6.9}_{-7.0}$ & $ 15.4^{+2.0}_{-2.0}$ & $ 6121^{+ 10}_{- 10}$ & $ -1.22^{+  0.15}_{-  0.15}$ & $0.190^{+0.037}_{-0.037}$  & $-0.015^{+0.020}_{-0.020}$ & $0.001^{+0.023}_{-0.022}$ & $0.001^{+0.023}_{-0.022}$\\
SNF20070818-001 & $160.6^{+6.7}_{-6.8}$ & $ 16.4^{+1.5}_{-1.5}$ & $ 6113^{+  5}_{-  5}$ & $  0.14^{+  0.14}_{-  0.14}$ & $-0.178^{+0.021}_{-0.020}$  & $0.005^{+0.014}_{-0.014}$ & $0.014^{+0.016}_{-0.015}$ & $0.014^{+0.016}_{-0.015}$\\
SNBOSS38 & $ 57.1^{+0.4}_{-0.4}$ & $ 17.8^{+0.3}_{-0.3}$ & $ 6127^{+  3}_{-  3}$ & $  1.02^{+  0.11}_{-  0.11}$ & $0.018^{+0.038}_{-0.039}$  & $0.017^{+0.021}_{-0.023}$ & $-0.002^{+0.019}_{-0.020}$ & $-0.002^{+0.019}_{-0.020}$\\
SN2006ob & $ 75.9^{+12.1}_{-12.4}$ & $ 26.8^{+1.3}_{-1.4}$ & $ 6113^{+  5}_{-  5}$ & $ -2.26^{+  0.21}_{-  0.21}$ & $0.039^{+0.035}_{-0.035}$  & $-0.051^{+0.019}_{-0.017}$ & $0.084^{+0.021}_{-0.021}$ & $0.084^{+0.021}_{-0.021}$\\
PTF12eer & $169.4^{+8.9}_{-8.8}$ & $ 12.2^{+1.8}_{-1.8}$ & $ 6148^{+ 10}_{- 10}$ & $  0.64^{+  0.14}_{-  0.15}$ & $-0.053^{+0.028}_{-0.027}$  & $0.039^{+0.018}_{-0.019}$ & $-0.029^{+0.017}_{-0.012}$ & $-0.029^{+0.017}_{-0.012}$\\
PTF10ops & $ 38.0^{+9.2}_{-9.2}$ & $ 21.9^{+4.6}_{-4.6}$ & $ 6140^{+  6}_{-  5}$ & $ -0.58^{+  0.12}_{-  0.12}$ & $-0.091^{+0.221}_{-0.154}$  & $0.190^{+0.040}_{-0.041}$ & $0.181^{+0.038}_{-0.036}$ & $0.181^{+0.038}_{-0.036}$\\
SNF20080514-002 & $ 83.1^{+0.7}_{-0.7}$ & $ 19.2^{+0.5}_{-0.5}$ & $ 6131^{+  3}_{-  3}$ & $ -1.55^{+  0.09}_{-  0.09}$ & $0.174^{+0.019}_{-0.019}$  & $-0.074^{+0.007}_{-0.006}$ & $0.006^{+0.011}_{-0.012}$ & $0.006^{+0.011}_{-0.012}$\\
PTF12evo & $129.1^{+2.6}_{-2.7}$ & $  9.9^{+1.1}_{-1.1}$ & $ 6157^{+  4}_{-  4}$ & $  0.81^{+  0.13}_{-  0.12}$ & $0.024^{+0.027}_{-0.027}$  & $0.004^{+0.011}_{-0.011}$ & $-0.034^{+0.012}_{-0.010}$ & $-0.034^{+0.012}_{-0.010}$\\
SNF20080614-010 & $118.9^{+4.9}_{-4.8}$ & $ 28.9^{+1.2}_{-1.2}$ & $ 6129^{+  3}_{-  3}$ & $ -1.81^{+  0.15}_{-  0.15}$ & $-0.114^{+0.046}_{-0.042}$  & $-0.068^{+0.012}_{-0.009}$ & $0.049^{+0.019}_{-0.019}$ & $0.049^{+0.019}_{-0.019}$\\
PTF10icb & $104.8^{+0.9}_{-0.9}$ & $ 12.6^{+0.3}_{-0.3}$ & $ 6138^{+  2}_{-  2}$ & $  0.23^{+  0.10}_{-  0.10}$ & $0.410^{+0.024}_{-0.025}$  & $0.046^{+0.006}_{-0.006}$ & $-0.019^{+0.010}_{-0.010}$ & $-0.019^{+0.010}_{-0.010}$\\
PTF12efn & $145.5^{+3.3}_{-3.2}$ & $  5.4^{+1.4}_{-1.5}$ & $ 6115^{+  3}_{-  3}$ & $  1.04^{+  0.16}_{-  0.16}$ & $0.177^{+0.030}_{-0.029}$  & $-0.058^{+0.014}_{-0.012}$ & $0.051^{+0.017}_{-0.017}$ & $0.051^{+0.017}_{-0.017}$\\
SNNGC4424 & $109.0^{+0.3}_{-0.3}$ & $  8.6^{+0.1}_{-0.1}$ & $ 6138^{+  2}_{-  2}$ & $  0.44^{+  0.09}_{-  0.09}$ & $0.600^{+0.022}_{-0.023}$  & $0.078^{+0.007}_{-0.006}$ & $-0.025^{+0.011}_{-0.011}$ & $-0.025^{+0.011}_{-0.011}$\\
SNF20080516-022 & $ 99.1^{+2.2}_{-2.1}$ & $ 15.3^{+0.9}_{-0.9}$ & $ 6159^{+  3}_{-  3}$ & $  0.48^{+  0.12}_{-  0.11}$ & $0.001^{+0.026}_{-0.026}$  & $-0.064^{+0.010}_{-0.009}$ & $-0.025^{+0.014}_{-0.013}$ & $-0.025^{+0.014}_{-0.013}$\\
PTF12hwb & $139.4^{+25.9}_{-23.8}$ & $ -0.9^{+4.4}_{-4.3}$ & $ 6090^{+ 14}_{- 14}$ & $  1.34^{+  0.17}_{-  0.17}$ & $0.471^{+0.042}_{-0.044}$  & $0.151^{+0.037}_{-0.043}$ & $0.004^{+0.040}_{-0.032}$ & $0.004^{+0.040}_{-0.032}$\\
PTF10qyz & $108.1^{+2.1}_{-2.1}$ & $ 21.0^{+0.8}_{-0.8}$ & $ 6117^{+  5}_{-  5}$ & $ -1.67^{+  0.13}_{-  0.13}$ & $0.099^{+0.133}_{-0.095}$  & $-0.067^{+0.014}_{-0.010}$ & $-0.038^{+0.012}_{-0.008}$ & $-0.038^{+0.012}_{-0.008}$\\
SNF20060907-000 & $ 87.8^{+7.5}_{-7.3}$ & $ 15.9^{+0.9}_{-0.9}$ & $ 6150^{+  4}_{-  4}$ & $  0.16^{+  0.17}_{-  0.17}$ & $0.037^{+0.033}_{-0.033}$  & $-0.079^{+0.005}_{-0.005}$ & $-0.041^{+0.008}_{-0.007}$ & $-0.041^{+0.008}_{-0.007}$\\
LSQ12fxd & $122.8^{+1.6}_{-1.7}$ & $ 11.7^{+0.8}_{-0.7}$ & $ 6119^{+  4}_{-  4}$ & $  0.16^{+  0.12}_{-  0.12}$ & $-0.154^{+0.044}_{-0.032}$  & $-0.052^{+0.008}_{-0.008}$ & $0.002^{+0.013}_{-0.013}$ & $0.002^{+0.013}_{-0.013}$\\
SNF20080821-000 & $104.8^{+2.2}_{-2.2}$ & $  8.8^{+1.2}_{-1.2}$ & $ 6121^{+  4}_{-  4}$ & $  0.77^{+  0.15}_{-  0.14}$ & $-0.104^{+0.023}_{-0.023}$  & $0.025^{+0.011}_{-0.011}$ & $-0.006^{+0.013}_{-0.013}$ & $-0.006^{+0.013}_{-0.013}$\\
SNF20070802-000 & $157.9^{+3.3}_{-3.3}$ & $ 16.0^{+1.4}_{-1.4}$ & $ 6103^{+  5}_{-  5}$ & $  0.07^{+  0.13}_{-  0.13}$ & $-0.074^{+0.028}_{-0.028}$  & $0.005^{+0.014}_{-0.014}$ & $0.005^{+0.016}_{-0.016}$ & $0.005^{+0.016}_{-0.016}$\\
PTF10wnm & $105.5^{+2.3}_{-2.3}$ & $  6.8^{+0.9}_{-0.9}$ & $ 6125^{+  3}_{-  3}$ & $  0.73^{+  0.12}_{-  0.12}$ & $0.123^{+0.022}_{-0.021}$  & $-0.014^{+0.009}_{-0.009}$ & $0.015^{+0.012}_{-0.012}$ & $0.015^{+0.012}_{-0.012}$\\
PTF10mwb & $116.2^{+1.2}_{-1.2}$ & $ 21.1^{+0.7}_{-0.7}$ & $ 6139^{+  2}_{-  2}$ & $ -0.48^{+  0.09}_{-  0.09}$ & $-0.000^{+0.071}_{-0.070}$  & $-0.036^{+0.008}_{-0.008}$ & $-0.011^{+0.011}_{-0.011}$ & $-0.011^{+0.011}_{-0.011}$\\
SN2010dt & $133.3^{+8.4}_{-8.6}$ & $ 16.0^{+0.6}_{-0.6}$ & $ 6141^{+  5}_{-  5}$ & $ -0.02^{+  0.09}_{-  0.09}$ & $-0.046^{+0.024}_{-0.024}$  & $-0.063^{+0.010}_{-0.009}$ & $-0.043^{+0.007}_{-0.007}$ & $-0.043^{+0.007}_{-0.007}$\\
SNF20080623-001 & $149.2^{+1.4}_{-1.4}$ & $ 14.8^{+0.6}_{-0.6}$ & $ 6131^{+  3}_{-  3}$ & $ -0.07^{+  0.11}_{-  0.10}$ & $0.072^{+0.025}_{-0.025}$  & $-0.054^{+0.010}_{-0.009}$ & $-0.034^{+0.011}_{-0.010}$ & $-0.034^{+0.011}_{-0.010}$\\
LSQ12fhe & $ 42.9^{+1.2}_{-1.2}$ & $  2.3^{+1.9}_{-1.9}$ & $ 6108^{+  4}_{-  4}$ & $  1.01^{+  0.17}_{-  0.17}$ & $-0.183^{+0.032}_{-0.033}$  & $0.067^{+0.017}_{-0.018}$ & $-0.017^{+0.018}_{-0.016}$ & $-0.017^{+0.018}_{-0.016}$\\
PTF11bju & $ 31.3^{+4.3}_{-4.2}$ & $  3.3^{+2.0}_{-2.0}$ & $ 6138^{+  5}_{-  5}$ & $  1.02^{+  0.17}_{-  0.18}$ & $0.073^{+0.036}_{-0.036}$  & $0.077^{+0.019}_{-0.019}$ & $-0.014^{+0.020}_{-0.018}$ & $-0.014^{+0.020}_{-0.018}$\\
PTF09fox & $118.5^{+2.7}_{-2.7}$ & $  9.8^{+0.9}_{-0.8}$ & $ 6117^{+  3}_{-  3}$ & $  0.52^{+  0.16}_{-  0.15}$ & $0.027^{+0.039}_{-0.039}$  & $-0.047^{+0.009}_{-0.009}$ & $-0.043^{+0.007}_{-0.006}$ & $-0.043^{+0.007}_{-0.006}$\\
PTF13ayw & $103.8^{+2.4}_{-2.4}$ & $ 23.1^{+1.8}_{-1.9}$ & $ 6116^{+  6}_{-  6}$ & $ -1.67^{+  0.23}_{-  0.24}$ & $0.068^{+0.041}_{-0.041}$  & $-0.069^{+0.014}_{-0.009}$ & $0.101^{+0.019}_{-0.020}$ & $0.101^{+0.019}_{-0.020}$\\
SNF20070810-004 & $127.0^{+1.8}_{-1.8}$ & $ 20.1^{+0.9}_{-0.9}$ & $ 6117^{+  7}_{-  7}$ & $ -0.01^{+  0.11}_{-  0.11}$ & $-0.044^{+0.026}_{-0.025}$  & $-0.068^{+0.011}_{-0.009}$ & $-0.033^{+0.012}_{-0.010}$ & $-0.033^{+0.012}_{-0.010}$\\
PTF11mty & $110.5^{+2.2}_{-2.2}$ & $ 11.3^{+1.0}_{-1.1}$ & $ 6140^{+  5}_{-  5}$ & $  1.03^{+  0.13}_{-  0.13}$ & $0.062^{+0.025}_{-0.024}$  & $-0.072^{+0.009}_{-0.007}$ & $0.024^{+0.013}_{-0.014}$ & $0.024^{+0.013}_{-0.014}$\\
SNF20080512-010 & $ 96.0^{+3.4}_{-3.4}$ & $ 21.9^{+1.3}_{-1.3}$ & $ 6128^{+  5}_{-  5}$ & $ -1.64^{+  0.18}_{-  0.18}$ & $-0.032^{+0.072}_{-0.061}$  & $-0.051^{+0.013}_{-0.012}$ & $0.018^{+0.016}_{-0.016}$ & $0.018^{+0.016}_{-0.016}$\\
PTF11mkx & $ 34.8^{+3.5}_{-3.5}$ & $  5.5^{+1.3}_{-1.3}$ & $ 6169^{+  5}_{-  5}$ & $  0.53^{+  0.14}_{-  0.14}$ & $0.022^{+0.135}_{-0.080}$  & $0.102^{+0.017}_{-0.016}$ & $-0.041^{+0.008}_{-0.007}$ & $-0.041^{+0.008}_{-0.007}$\\
PTF10tce & $135.8^{+1.1}_{-1.1}$ & $ 10.6^{+1.2}_{-1.2}$ & $ 6090^{+  4}_{-  4}$ & $  0.38^{+  0.12}_{-  0.12}$ & $0.041^{+0.028}_{-0.027}$  & $-0.005^{+0.013}_{-0.014}$ & $-0.019^{+0.015}_{-0.015}$ & $-0.019^{+0.015}_{-0.015}$\\
SNF20061020-000 & $ 90.3^{+13.4}_{-13.3}$ & $ 23.9^{+0.9}_{-0.9}$ & $ 6120^{+  4}_{-  5}$ & $ -1.75^{+  0.17}_{-  0.17}$ & $-0.096^{+0.039}_{-0.034}$  & $0.000^{+0.016}_{-0.016}$ & $0.006^{+0.022}_{-0.020}$ & $0.006^{+0.022}_{-0.020}$\\
SN2005ir & $115.7^{+2.8}_{-2.7}$ & $  8.7^{+4.6}_{-4.8}$ & $ 6069^{+  5}_{-  5}$ & $  1.64^{+  0.49}_{-  0.48}$ & $0.112^{+0.042}_{-0.043}$  & $-0.022^{+0.036}_{-0.033}$ & $0.067^{+0.037}_{-0.037}$ & $0.067^{+0.037}_{-0.037}$\\
SNF20080717-000 & $ 92.7^{+2.6}_{-2.6}$ & $ 12.4^{+1.7}_{-1.7}$ & $ 6105^{+  3}_{-  3}$ & $  0.87^{+  0.13}_{-  0.12}$ & $0.115^{+0.034}_{-0.033}$  & $0.126^{+0.018}_{-0.019}$ & $0.050^{+0.021}_{-0.019}$ & $0.050^{+0.021}_{-0.019}$\\
PTF12ena & $101.0^{+1.6}_{-1.6}$ & $  8.2^{+0.9}_{-0.9}$ & $ 6130^{+  3}_{-  3}$ & $  0.76^{+  0.13}_{-  0.13}$ & $-0.039^{+0.028}_{-0.028}$  & $0.289^{+0.011}_{-0.012}$ & $-0.038^{+0.010}_{-0.008}$ & $-0.038^{+0.010}_{-0.008}$\\
PTF13anh & $167.2^{+1.8}_{-1.8}$ & $ 21.6^{+1.0}_{-1.0}$ & $ 6174^{+  3}_{-  4}$ & $ -0.38^{+  0.14}_{-  0.14}$ & $0.092^{+0.108}_{-0.067}$  & $-0.032^{+0.020}_{-0.019}$ & $-0.034^{+0.014}_{-0.010}$ & $-0.034^{+0.014}_{-0.010}$\\
CSS110918\_01 & $110.4^{+1.0}_{-1.0}$ & $  5.0^{+0.9}_{-0.9}$ & $ 6101^{+  2}_{-  2}$ & $  0.82^{+  0.13}_{-  0.13}$ & $-0.077^{+0.187}_{-0.094}$  & $-0.078^{+0.006}_{-0.005}$ & $-0.028^{+0.013}_{-0.012}$ & $-0.028^{+0.013}_{-0.012}$\\
SNF20061024-000 & $ 98.9^{+15.7}_{-16.7}$ & $ 29.1^{+1.4}_{-1.4}$ & $ 6127^{+  5}_{-  5}$ & $ -1.51^{+  0.18}_{-  0.18}$ & $0.045^{+0.062}_{-0.059}$  & $-0.049^{+0.021}_{-0.019}$ & $0.003^{+0.031}_{-0.028}$ & $0.003^{+0.031}_{-0.028}$\\
SNF20070506-006 & $ 94.3^{+1.3}_{-1.2}$ & $  6.4^{+0.6}_{-0.5}$ & $ 6152^{+  3}_{-  3}$ & $  1.02^{+  0.11}_{-  0.11}$ & $-0.075^{+0.027}_{-0.026}$  & $-0.029^{+0.009}_{-0.009}$ & $0.006^{+0.013}_{-0.013}$ & $0.006^{+0.013}_{-0.013}$\\
SNF20070403-001 & $107.8^{+5.0}_{-5.1}$ & $ 17.1^{+1.4}_{-1.4}$ & $ 6123^{+  4}_{-  4}$ & $ -0.60^{+  0.12}_{-  0.12}$ & $0.007^{+0.029}_{-0.028}$  & $-0.036^{+0.014}_{-0.014}$ & $0.020^{+0.017}_{-0.017}$ & $0.020^{+0.017}_{-0.017}$\\
PTF10hmv & $109.8^{+1.3}_{-1.4}$ & $  8.5^{+0.7}_{-0.6}$ & $ 6143^{+  3}_{-  3}$ & $  1.15^{+  0.13}_{-  0.12}$ & $0.284^{+0.024}_{-0.023}$  & $0.106^{+0.009}_{-0.009}$ & $-0.010^{+0.012}_{-0.012}$ & $-0.010^{+0.012}_{-0.012}$\\
SNF20071015-000 & $104.8^{+3.1}_{-3.1}$ & $  6.5^{+1.1}_{-1.1}$ & $ 6125^{+  7}_{-  7}$ & $  0.90^{+  0.28}_{-  0.29}$ & $0.116^{+0.040}_{-0.040}$  & $0.293^{+0.016}_{-0.017}$ & $-0.018^{+0.021}_{-0.017}$ & $-0.018^{+0.021}_{-0.017}$\\
SNhunt89 & $ 88.6^{+2.7}_{-2.7}$ & $ 29.8^{+1.7}_{-1.7}$ & $ 6109^{+  7}_{-  7}$ & $ -2.02^{+  0.16}_{-  0.16}$ & $0.202^{+0.049}_{-0.047}$  & $0.030^{+0.018}_{-0.019}$ & $0.032^{+0.027}_{-0.026}$ & $0.032^{+0.027}_{-0.026}$\\
SNF20070902-021 & $109.0^{+3.3}_{-3.3}$ & $ 16.8^{+0.9}_{-0.9}$ & $ 6131^{+  5}_{-  5}$ & $ -0.46^{+  0.09}_{-  0.09}$ & $-0.082^{+0.027}_{-0.026}$  & $-0.023^{+0.010}_{-0.010}$ & $-0.028^{+0.014}_{-0.012}$ & $-0.028^{+0.014}_{-0.012}$\\
PTF09dlc & $143.8^{+2.1}_{-2.1}$ & $  9.7^{+0.8}_{-0.8}$ & $ 6143^{+  3}_{-  3}$ & $  0.01^{+  0.10}_{-  0.10}$ & $0.008^{+0.027}_{-0.026}$  & $-0.054^{+0.012}_{-0.012}$ & $-0.017^{+0.014}_{-0.015}$ & $-0.017^{+0.014}_{-0.015}$\\
PTF13ajv & $149.9^{+8.1}_{-8.2}$ & $ 22.0^{+2.7}_{-3.0}$ & $ 6107^{+ 20}_{- 20}$ & $ -1.64^{+  0.20}_{-  0.21}$ & $0.263^{+0.046}_{-0.045}$  & $-0.062^{+0.021}_{-0.014}$ & $-0.005^{+0.023}_{-0.022}$ & $-0.005^{+0.023}_{-0.022}$\\
SNF20080919-000 & $115.6^{+2.7}_{-2.7}$ & $  8.8^{+0.9}_{-0.8}$ & $ 6144^{+  5}_{-  5}$ & $ -0.24^{+  0.10}_{-  0.10}$ & $-0.043^{+0.027}_{-0.026}$  & $0.165^{+0.013}_{-0.012}$ & $-0.035^{+0.013}_{-0.010}$ & $-0.035^{+0.013}_{-0.010}$\\
SNF20080919-001 & $ 85.0^{+1.1}_{-1.1}$ & $  6.1^{+0.4}_{-0.4}$ & $ 6149^{+  4}_{-  4}$ & $  1.05^{+  0.12}_{-  0.12}$ & $-0.183^{+0.020}_{-0.019}$  & $0.001^{+0.008}_{-0.008}$ & $0.006^{+0.012}_{-0.012}$ & $0.006^{+0.012}_{-0.012}$\\
SN2010kg & $ 99.4^{+17.2}_{-17.5}$ & $ 21.8^{+0.7}_{-0.7}$ & $ 6077^{+  5}_{-  5}$ & $ -1.87^{+  0.24}_{-  0.25}$ & $-0.073^{+0.030}_{-0.029}$  & $0.035^{+0.021}_{-0.020}$ & $0.034^{+0.020}_{-0.021}$ & $0.034^{+0.020}_{-0.021}$\\
SNF20080714-008 & $155.6^{+11.0}_{-10.8}$ & $ 16.6^{+1.9}_{-1.8}$ & $ 6099^{+  6}_{-  6}$ & $ -0.07^{+  0.14}_{-  0.14}$ & $-0.141^{+0.024}_{-0.022}$  & $0.082^{+0.018}_{-0.020}$ & $-0.029^{+0.017}_{-0.013}$ & $-0.029^{+0.017}_{-0.013}$\\
SNF20070714-007 & $132.6^{+5.5}_{-5.5}$ & $ 33.5^{+4.3}_{-4.2}$ & $ 6145^{+  4}_{-  4}$ & $ -1.09^{+  0.13}_{-  0.13}$ & $-0.094^{+0.046}_{-0.047}$  & $0.118^{+0.035}_{-0.038}$ & $0.117^{+0.036}_{-0.032}$ & $0.117^{+0.036}_{-0.032}$\\
SNF20080522-011 & $122.1^{+1.7}_{-1.7}$ & $  8.3^{+0.5}_{-0.5}$ & $ 6125^{+  2}_{-  2}$ & $  0.82^{+  0.13}_{-  0.13}$ & $-0.124^{+0.023}_{-0.023}$  & $-0.045^{+0.008}_{-0.008}$ & $-0.031^{+0.011}_{-0.010}$ & $-0.031^{+0.011}_{-0.010}$\\
SNF20061111-002 & $117.1^{+8.6}_{-8.8}$ & $ 19.8^{+0.9}_{-0.9}$ & $ 6144^{+  6}_{-  6}$ & $  0.07^{+  0.17}_{-  0.17}$ & $0.056^{+0.032}_{-0.032}$  & $-0.062^{+0.013}_{-0.011}$ & $0.026^{+0.017}_{-0.017}$ & $0.026^{+0.017}_{-0.017}$\\
SNNGC6343 & $ 86.9^{+1.5}_{-1.4}$ & $ 21.2^{+0.7}_{-0.7}$ & $ 6136^{+  3}_{-  3}$ & $ -1.49^{+  0.16}_{-  0.16}$ & $0.055^{+0.029}_{-0.030}$  & $0.036^{+0.013}_{-0.012}$ & $0.020^{+0.016}_{-0.014}$ & $0.020^{+0.016}_{-0.014}$\\
SNF20061011-005 & $120.5^{+1.1}_{-1.1}$ & $  9.2^{+0.3}_{-0.3}$ & $ 6132^{+  4}_{-  4}$ & $  0.66^{+  0.22}_{-  0.21}$ & $-0.091^{+0.043}_{-0.041}$  & $-0.073^{+0.008}_{-0.007}$ & $0.003^{+0.021}_{-0.021}$ & $0.003^{+0.021}_{-0.021}$\\
SNF20080825-010 & $ 96.4^{+10.2}_{-10.2}$ & $ 19.2^{+0.6}_{-0.6}$ & $ 6116^{+  4}_{-  4}$ & $ -0.76^{+  0.13}_{-  0.13}$ & $-0.192^{+0.020}_{-0.020}$  & $-0.045^{+0.013}_{-0.013}$ & $0.002^{+0.013}_{-0.013}$ & $0.002^{+0.013}_{-0.013}$\\
PTF10ufj & $141.4^{+3.3}_{-3.3}$ & $ 12.0^{+1.1}_{-1.0}$ & $ 6131^{+  6}_{-  5}$ & $  0.56^{+  0.13}_{-  0.13}$ & $-0.043^{+0.106}_{-0.068}$  & $-0.040^{+0.011}_{-0.011}$ & $-0.030^{+0.014}_{-0.011}$ & $-0.030^{+0.014}_{-0.011}$\\
PTF10wof & $130.5^{+2.6}_{-2.7}$ & $ 16.5^{+0.9}_{-0.9}$ & $ 6102^{+  2}_{-  2}$ & $  0.06^{+  0.11}_{-  0.11}$ & $0.096^{+0.023}_{-0.023}$  & $-0.009^{+0.011}_{-0.011}$ & $-0.013^{+0.013}_{-0.012}$ & $-0.013^{+0.013}_{-0.012}$\\
SNF20080918-000 & $146.6^{+3.5}_{-3.5}$ & $  5.6^{+2.1}_{-2.0}$ & $ 6110^{+  5}_{-  5}$ & $  1.28^{+  0.23}_{-  0.23}$ & $0.112^{+0.038}_{-0.037}$  & $0.073^{+0.019}_{-0.019}$ & $-0.004^{+0.023}_{-0.022}$ & $-0.004^{+0.023}_{-0.022}$\\
SNF20080516-000 & $117.6^{+2.1}_{-2.1}$ & $  8.2^{+1.2}_{-1.1}$ & $ 6135^{+  3}_{-  3}$ & $  0.96^{+  0.18}_{-  0.18}$ & $0.084^{+0.029}_{-0.028}$  & $-0.054^{+0.011}_{-0.010}$ & $0.004^{+0.015}_{-0.015}$ & $0.004^{+0.015}_{-0.015}$\\
SN2005cf & $158.9^{+0.7}_{-0.7}$ & $ 15.2^{+0.8}_{-0.7}$ & $ 6140^{+  3}_{-  3}$ & $ -0.19^{+  0.13}_{-  0.13}$ & $-0.122^{+0.028}_{-0.027}$  & $-0.077^{+0.007}_{-0.005}$ & $0.002^{+0.014}_{-0.015}$ & $0.002^{+0.014}_{-0.015}$\\
CSS130502\_01 & $ 87.3^{+8.0}_{-8.7}$ & $ 15.4^{+0.5}_{-0.5}$ & $ 6128^{+  3}_{-  3}$ & $ -0.15^{+  0.20}_{-  0.20}$ & $0.121^{+0.032}_{-0.032}$  & $-0.069^{+0.011}_{-0.009}$ & $-0.017^{+0.015}_{-0.015}$ & $-0.017^{+0.015}_{-0.015}$\\
SNF20080620-000 & $125.3^{+9.5}_{-9.7}$ & $ 20.6^{+0.6}_{-0.6}$ & $ 6133^{+  3}_{-  3}$ & $ -0.97^{+  0.14}_{-  0.14}$ & $-0.007^{+0.029}_{-0.028}$  & $0.001^{+0.012}_{-0.012}$ & $-0.043^{+0.007}_{-0.007}$ & $-0.043^{+0.007}_{-0.007}$\\
SNPGC51271 & $ 64.0^{+12.0}_{-13.2}$ & $ 21.4^{+0.7}_{-0.7}$ & $ 6121^{+  2}_{-  2}$ & $ -1.85^{+  0.16}_{-  0.17}$ & $0.125^{+0.034}_{-0.034}$  & $-0.057^{+0.017}_{-0.014}$ & $0.011^{+0.018}_{-0.019}$ & $0.011^{+0.018}_{-0.019}$\\
PTF11pdk & $127.7^{+2.6}_{-2.7}$ & $ 16.1^{+1.2}_{-1.2}$ & $ 6154^{+  5}_{-  5}$ & $  0.16^{+  0.16}_{-  0.17}$ & $-0.027^{+0.035}_{-0.035}$  & $-0.070^{+0.010}_{-0.008}$ & $-0.033^{+0.015}_{-0.011}$ & $-0.033^{+0.015}_{-0.011}$\\
SNF20060511-014 & $102.9^{+2.8}_{-2.8}$ & $ 17.1^{+1.0}_{-1.0}$ & $ 6144^{+  8}_{-  7}$ & $ -0.50^{+  0.15}_{-  0.15}$ & $-0.042^{+0.047}_{-0.037}$  & $-0.008^{+0.011}_{-0.010}$ & $-0.041^{+0.009}_{-0.007}$ & $-0.041^{+0.009}_{-0.007}$\\
SNF20080612-003 & $120.0^{+1.1}_{-1.1}$ & $  7.5^{+0.6}_{-0.6}$ & $ 6122^{+  3}_{-  3}$ & $  0.96^{+  0.15}_{-  0.15}$ & $-0.221^{+0.018}_{-0.017}$  & $-0.015^{+0.007}_{-0.007}$ & $-0.001^{+0.013}_{-0.014}$ & $-0.001^{+0.013}_{-0.014}$\\
SNF20080626-002 & $130.0^{+1.0}_{-1.0}$ & $  5.8^{+2.1}_{-2.0}$ & $ 6111^{+  3}_{-  3}$ & $  1.03^{+  0.19}_{-  0.19}$ & $-0.043^{+0.026}_{-0.025}$  & $-0.009^{+0.015}_{-0.016}$ & $-0.022^{+0.017}_{-0.015}$ & $-0.022^{+0.017}_{-0.015}$\\
SNF20060621-015 & $111.8^{+1.3}_{-1.3}$ & $  9.9^{+0.7}_{-0.7}$ & $ 6144^{+  3}_{-  3}$ & $  0.52^{+  0.14}_{-  0.14}$ & $-0.015^{+0.026}_{-0.027}$  & $-0.067^{+0.008}_{-0.008}$ & $-0.020^{+0.013}_{-0.013}$ & $-0.020^{+0.013}_{-0.013}$\\
SNF20080920-000 & $135.1^{+1.4}_{-1.4}$ & $  5.3^{+1.5}_{-1.5}$ & $ 6085^{+  3}_{-  3}$ & $  1.08^{+  0.23}_{-  0.23}$ & $-0.002^{+0.038}_{-0.037}$  & $-0.020^{+0.017}_{-0.017}$ & $-0.012^{+0.020}_{-0.019}$ & $-0.012^{+0.020}_{-0.019}$\\
SN2007cq & $ 63.7^{+3.9}_{-3.9}$ & $ 11.6^{+0.8}_{-0.8}$ & $ 6138^{+  3}_{-  3}$ & $  0.05^{+  0.13}_{-  0.13}$ & $-0.111^{+0.029}_{-0.027}$  & $0.008^{+0.011}_{-0.011}$ & $-0.015^{+0.015}_{-0.015}$ & $-0.015^{+0.015}_{-0.015}$\\
SNF20080918-004 & $ 91.9^{+6.6}_{-6.6}$ & $ 21.2^{+0.9}_{-0.9}$ & $ 6141^{+  4}_{-  4}$ & $ -1.37^{+  0.15}_{-  0.15}$ & $0.035^{+0.142}_{-0.087}$  & $-0.021^{+0.014}_{-0.013}$ & $-0.012^{+0.019}_{-0.018}$ & $-0.012^{+0.019}_{-0.018}$\\
CSS120424\_01 & $138.5^{+2.1}_{-2.1}$ & $ 11.5^{+0.6}_{-0.6}$ & $ 6138^{+  3}_{-  3}$ & $  0.58^{+  0.14}_{-  0.14}$ & $-0.010^{+0.028}_{-0.028}$  & $-0.061^{+0.009}_{-0.009}$ & $-0.024^{+0.013}_{-0.012}$ & $-0.024^{+0.013}_{-0.012}$\\
SNF20080610-000 & $119.5^{+8.6}_{-8.8}$ & $ 17.0^{+1.4}_{-1.4}$ & $ 6131^{+  6}_{-  6}$ & $  0.16^{+  0.14}_{-  0.14}$ & $0.013^{+0.068}_{-0.068}$  & $-0.019^{+0.014}_{-0.015}$ & $-0.027^{+0.017}_{-0.013}$ & $-0.027^{+0.017}_{-0.013}$\\
SNF20070701-005 & $101.6^{+2.6}_{-2.6}$ & $ 13.6^{+0.9}_{-1.0}$ & $ 6155^{+  5}_{-  5}$ & $  0.39^{+  0.14}_{-  0.14}$ & $-0.187^{+0.021}_{-0.021}$  & $-0.012^{+0.011}_{-0.011}$ & $0.037^{+0.015}_{-0.015}$ & $0.037^{+0.015}_{-0.015}$\\
SN2007kk & $128.6^{+1.4}_{-1.4}$ & $ 11.0^{+0.9}_{-0.9}$ & $ 6099^{+  4}_{-  4}$ & $  0.59^{+  0.14}_{-  0.14}$ & $-0.087^{+0.029}_{-0.029}$  & $-0.048^{+0.012}_{-0.012}$ & $-0.035^{+0.012}_{-0.010}$ & $-0.035^{+0.012}_{-0.010}$\\
SNF20060908-004 & $114.4^{+1.2}_{-1.2}$ & $ 12.4^{+0.6}_{-0.6}$ & $ 6136^{+  3}_{-  3}$ & $  0.63^{+  0.15}_{-  0.15}$ & $-0.099^{+0.138}_{-0.081}$  & $-0.068^{+0.009}_{-0.008}$ & $0.021^{+0.017}_{-0.017}$ & $0.021^{+0.017}_{-0.017}$\\
SNF20080909-030 & $ 93.8^{+1.0}_{-0.9}$ & $  7.7^{+0.4}_{-0.4}$ & $ 6171^{+  3}_{-  3}$ & $  0.99^{+  0.13}_{-  0.13}$ & $0.030^{+0.030}_{-0.030}$  & $0.008^{+0.013}_{-0.012}$ & $0.008^{+0.015}_{-0.014}$ & $0.008^{+0.015}_{-0.014}$\\
PTF11bgv & $ 78.3^{+3.1}_{-3.1}$ & $ 12.7^{+0.6}_{-0.6}$ & $ 6146^{+  3}_{-  3}$ & $ -0.27^{+  0.11}_{-  0.10}$ & $0.057^{+0.028}_{-0.028}$  & $0.081^{+0.010}_{-0.010}$ & $0.006^{+0.015}_{-0.015}$ & $0.006^{+0.015}_{-0.015}$\\
SNNGC2691 & $ 55.7^{+14.4}_{-14.5}$ & $  4.5^{+0.2}_{-0.2}$ & $ 6138^{+  8}_{-  8}$ & $  0.82^{+  0.16}_{-  0.16}$ & $-0.031^{+0.039}_{-0.038}$  & $0.069^{+0.018}_{-0.019}$ & $0.041^{+0.020}_{-0.020}$ & $0.041^{+0.020}_{-0.020}$\\
PTF13asv & $ 75.4^{+1.1}_{-1.1}$ & $  2.4^{+0.4}_{-0.4}$ & $ 6149^{+  4}_{-  4}$ & $  0.66^{+  0.10}_{-  0.10}$ & $-0.018^{+0.184}_{-0.105}$  & $-0.068^{+0.012}_{-0.009}$ & $0.015^{+0.015}_{-0.015}$ & $0.015^{+0.015}_{-0.015}$\\
SNF20070806-026 & $102.2^{+8.4}_{-9.0}$ & $ 25.3^{+0.7}_{-0.7}$ & $ 6112^{+  7}_{-  7}$ & $ -1.93^{+  0.14}_{-  0.14}$ & $0.010^{+0.032}_{-0.033}$  & $-0.071^{+0.011}_{-0.008}$ & $0.015^{+0.017}_{-0.017}$ & $0.015^{+0.017}_{-0.017}$\\
SNF20070427-001 & $ 80.7^{+2.3}_{-2.3}$ & $  6.5^{+0.8}_{-0.8}$ & $ 6143^{+  5}_{-  5}$ & $  0.82^{+  0.17}_{-  0.17}$ & $-0.009^{+0.031}_{-0.031}$  & $-0.066^{+0.011}_{-0.009}$ & $-0.018^{+0.017}_{-0.016}$ & $-0.018^{+0.017}_{-0.016}$\\
SNF20061108-004 & $128.2^{+5.4}_{-5.4}$ & $  6.8^{+2.1}_{-2.1}$ & $ 6110^{+  6}_{-  6}$ & $  2.36^{+  0.27}_{-  0.27}$ & $-0.089^{+0.038}_{-0.036}$  & $-0.025^{+0.021}_{-0.023}$ & $-0.005^{+0.024}_{-0.022}$ & $-0.005^{+0.024}_{-0.022}$\\
SNF20060912-000 & $106.9^{+1.7}_{-1.7}$ & $ 19.8^{+1.5}_{-1.5}$ & $ 6158^{+  7}_{-  7}$ & $  0.06^{+  0.17}_{-  0.17}$ & $-0.177^{+0.032}_{-0.033}$  & $0.034^{+0.018}_{-0.019}$ & $0.080^{+0.021}_{-0.020}$ & $0.080^{+0.021}_{-0.020}$\\
CSS110918\_02 & $109.1^{+8.1}_{-8.1}$ & $ 15.0^{+0.6}_{-0.6}$ & $ 6137^{+  3}_{-  3}$ & $ -0.26^{+  0.13}_{-  0.13}$ & $-0.052^{+0.027}_{-0.027}$  & $-0.039^{+0.010}_{-0.010}$ & $-0.009^{+0.014}_{-0.014}$ & $-0.009^{+0.014}_{-0.014}$\\
SNF20080918-002 & $ 96.6^{+2.8}_{-2.7}$ & $ 13.3^{+1.2}_{-1.2}$ & $ 6143^{+  6}_{-  6}$ & $ -0.47^{+  0.08}_{-  0.08}$ & $0.021^{+0.034}_{-0.035}$  & $-0.056^{+0.014}_{-0.013}$ & $-0.005^{+0.020}_{-0.021}$ & $-0.005^{+0.020}_{-0.021}$\\
SNIC3573 & $103.0^{+1.8}_{-1.7}$ & $ 12.1^{+0.9}_{-0.9}$ & $ 6141^{+  5}_{-  5}$ & $  0.39^{+  0.16}_{-  0.16}$ & $-0.109^{+0.026}_{-0.024}$  & $0.015^{+0.009}_{-0.010}$ & $-0.025^{+0.014}_{-0.012}$ & $-0.025^{+0.014}_{-0.012}$\\
SNF20080725-004 & $133.8^{+2.1}_{-2.1}$ & $  6.7^{+0.9}_{-0.9}$ & $ 6131^{+  6}_{-  5}$ & $  0.59^{+  0.13}_{-  0.13}$ & $0.078^{+0.032}_{-0.032}$  & $0.025^{+0.012}_{-0.011}$ & $0.016^{+0.015}_{-0.014}$ & $0.016^{+0.015}_{-0.014}$\\
SNF20050728-006 & $127.7^{+2.5}_{-2.4}$ & $ 15.3^{+1.2}_{-1.2}$ & $ 6124^{+  6}_{-  6}$ & $  0.11^{+  0.19}_{-  0.19}$ & $-0.088^{+0.034}_{-0.032}$  & $0.064^{+0.011}_{-0.012}$ & $0.012^{+0.017}_{-0.017}$ & $0.012^{+0.017}_{-0.017}$\\
SN2012fr & $134.2^{+0.5}_{-0.5}$ & $  7.4^{+0.2}_{-0.2}$ & $ 6101^{+  1}_{-  1}$ & $  0.88^{+  0.08}_{-  0.09}$ & $-0.248^{+0.021}_{-0.021}$  & $-0.077^{+0.005}_{-0.005}$ & $0.044^{+0.012}_{-0.011}$ & $0.044^{+0.012}_{-0.011}$\\
SNF20060512-002 & $100.4^{+2.7}_{-2.8}$ & $ 13.5^{+1.1}_{-1.1}$ & $ 6106^{+  7}_{-  7}$ & $ -0.87^{+  0.15}_{-  0.16}$ & $-0.186^{+0.030}_{-0.028}$  & $0.018^{+0.012}_{-0.012}$ & $0.056^{+0.017}_{-0.016}$ & $0.056^{+0.017}_{-0.016}$\\
SNF20060512-001 & $ 88.2^{+1.2}_{-1.2}$ & $  5.5^{+0.4}_{-0.4}$ & $ 6169^{+  3}_{-  3}$ & $  0.87^{+  0.11}_{-  0.11}$ & $-0.025^{+0.025}_{-0.025}$  & $0.050^{+0.011}_{-0.011}$ & $-0.010^{+0.011}_{-0.012}$ & $-0.010^{+0.011}_{-0.012}$\\
SNF20071003-016 & $126.5^{+4.5}_{-4.4}$ & $ 16.7^{+1.5}_{-1.5}$ & $ 6123^{+ 10}_{- 10}$ & $ -0.30^{+  0.14}_{-  0.14}$ & $0.058^{+0.032}_{-0.032}$  & $0.068^{+0.013}_{-0.014}$ & $-0.030^{+0.016}_{-0.012}$ & $-0.030^{+0.016}_{-0.012}$\\
SNF20050821-007 & $141.8^{+2.6}_{-2.6}$ & $  7.8^{+0.9}_{-0.9}$ & $ 6140^{+  9}_{-  8}$ & $  0.34^{+  0.20}_{-  0.19}$ & $-0.024^{+0.034}_{-0.033}$  & $-0.030^{+0.014}_{-0.013}$ & $-0.012^{+0.017}_{-0.017}$ & $-0.012^{+0.017}_{-0.017}$\\
SNF20070803-005 & $ 45.4^{+14.0}_{-13.8}$ & $  0.8^{+0.6}_{-0.6}$ & $ 6138^{+ 22}_{- 22}$ & $  0.81^{+  0.15}_{-  0.15}$ & $-0.137^{+0.031}_{-0.031}$  & $0.043^{+0.019}_{-0.019}$ & $-0.005^{+0.021}_{-0.021}$ & $-0.005^{+0.021}_{-0.021}$\\
PTF09foz & $127.2^{+1.9}_{-1.9}$ & $ 22.1^{+1.1}_{-1.1}$ & $ 6135^{+  4}_{-  4}$ & $ -0.92^{+  0.16}_{-  0.16}$ & $-0.141^{+0.019}_{-0.018}$  & $-0.042^{+0.009}_{-0.010}$ & $0.031^{+0.013}_{-0.012}$ & $0.031^{+0.013}_{-0.012}$\\
PTF12grk & $158.1^{+8.3}_{-8.4}$ & $ 22.1^{+1.2}_{-1.2}$ & $ 6089^{+  8}_{-  8}$ & $ -1.02^{+  0.12}_{-  0.12}$ & $-0.117^{+0.027}_{-0.026}$  & $-0.025^{+0.016}_{-0.017}$ & $-0.039^{+0.010}_{-0.008}$ & $-0.039^{+0.010}_{-0.008}$\\
SNF20080720-001 & $138.3^{+3.9}_{-4.0}$ & $ 14.0^{+1.5}_{-1.5}$ & $ 6107^{+  3}_{-  3}$ & $  0.19^{+  0.12}_{-  0.12}$ & $-0.176^{+0.028}_{-0.028}$  & $0.325^{+0.018}_{-0.018}$ & $0.007^{+0.019}_{-0.018}$ & $0.007^{+0.019}_{-0.018}$\\
SNF20080810-001 & $ 98.8^{+12.2}_{-12.3}$ & $ 21.7^{+1.0}_{-1.0}$ & $ 6145^{+  5}_{-  5}$ & $ -0.93^{+  0.10}_{-  0.11}$ & $-0.058^{+0.024}_{-0.024}$  & $-0.023^{+0.016}_{-0.016}$ & $-0.000^{+0.017}_{-0.016}$ & $-0.000^{+0.017}_{-0.016}$\\
SNF20050729-002 & $110.4^{+2.3}_{-2.2}$ & $ 10.7^{+1.5}_{-1.4}$ & $ 6141^{+  6}_{-  6}$ & $ -0.22^{+  0.21}_{-  0.22}$ & $0.052^{+0.102}_{-0.075}$  & $-0.039^{+0.015}_{-0.014}$ & $-0.039^{+0.011}_{-0.008}$ & $-0.039^{+0.011}_{-0.008}$\\
SN2008ec & $ 97.7^{+10.6}_{-10.6}$ & $ 23.0^{+0.4}_{-0.4}$ & $ 6125^{+  3}_{-  3}$ & $ -1.09^{+  0.10}_{-  0.10}$ & $-0.076^{+0.025}_{-0.026}$  & $0.064^{+0.012}_{-0.012}$ & $0.017^{+0.015}_{-0.015}$ & $0.017^{+0.015}_{-0.015}$\\
SNF20070902-018 & $108.6^{+9.5}_{-9.7}$ & $ 18.1^{+1.7}_{-1.7}$ & $ 6119^{+  7}_{-  7}$ & $ -0.81^{+  0.10}_{-  0.09}$ & $0.032^{+0.030}_{-0.030}$  & $0.023^{+0.016}_{-0.017}$ & $-0.035^{+0.014}_{-0.010}$ & $-0.035^{+0.014}_{-0.010}$\\
SNF20070424-003 & $123.6^{+3.6}_{-3.6}$ & $ 14.2^{+1.2}_{-1.1}$ & $ 6134^{+  6}_{-  6}$ & $  0.30^{+  0.10}_{-  0.10}$ & $0.008^{+0.023}_{-0.023}$  & $-0.010^{+0.010}_{-0.010}$ & $-0.041^{+0.008}_{-0.007}$ & $-0.041^{+0.008}_{-0.007}$\\
SN2006cj & $101.4^{+1.3}_{-1.3}$ & $  5.5^{+0.8}_{-0.8}$ & $ 6128^{+  3}_{-  3}$ & $  0.86^{+  0.13}_{-  0.13}$ & $0.028^{+0.025}_{-0.025}$  & $0.006^{+0.008}_{-0.009}$ & $-0.017^{+0.013}_{-0.013}$ & $-0.017^{+0.013}_{-0.013}$\\
SN2007nq & $ 91.2^{+9.1}_{-9.0}$ & $ 23.3^{+1.0}_{-1.0}$ & $ 6109^{+  5}_{-  5}$ & $ -1.90^{+  0.20}_{-  0.20}$ & $-0.075^{+0.038}_{-0.035}$  & $-0.022^{+0.015}_{-0.015}$ & $0.031^{+0.021}_{-0.020}$ & $0.031^{+0.021}_{-0.020}$\\
SNF20070817-003 & $ 94.5^{+2.4}_{-2.4}$ & $ 18.1^{+1.1}_{-1.1}$ & $ 6115^{+  6}_{-  6}$ & $ -0.61^{+  0.12}_{-  0.12}$ & $0.050^{+0.031}_{-0.032}$  & $-0.031^{+0.012}_{-0.012}$ & $-0.024^{+0.015}_{-0.014}$ & $-0.024^{+0.015}_{-0.014}$\\
SNF20070403-000 & $ 62.8^{+6.1}_{-6.2}$ & $ 27.6^{+1.6}_{-1.6}$ & $ 6154^{+  7}_{-  7}$ & $ -1.88^{+  0.17}_{-  0.17}$ & $0.147^{+0.040}_{-0.041}$  & $0.081^{+0.021}_{-0.021}$ & $0.041^{+0.025}_{-0.023}$ & $0.041^{+0.025}_{-0.023}$\\
SNF20061022-005 & $ 63.4^{+3.6}_{-3.8}$ & $  3.9^{+1.4}_{-1.4}$ & $ 6147^{+  7}_{-  7}$ & $  2.03^{+  0.25}_{-  0.25}$ & $0.056^{+0.038}_{-0.038}$  & $0.041^{+0.018}_{-0.020}$ & $-0.000^{+0.022}_{-0.021}$ & $-0.000^{+0.022}_{-0.021}$\\
SNNGC4076 & $127.1^{+2.4}_{-2.4}$ & $ 15.7^{+1.1}_{-1.1}$ & $ 6152^{+  4}_{-  4}$ & $  0.04^{+  0.10}_{-  0.10}$ & $-0.004^{+0.030}_{-0.029}$  & $0.049^{+0.013}_{-0.013}$ & $0.045^{+0.016}_{-0.015}$ & $0.045^{+0.016}_{-0.015}$\\
SNF20070727-016 & $ 78.3^{+2.4}_{-2.5}$ & $  4.6^{+0.8}_{-0.8}$ & $ 6139^{+  4}_{-  4}$ & $  0.16^{+  0.11}_{-  0.10}$ & $-0.097^{+0.043}_{-0.036}$  & $-0.037^{+0.011}_{-0.011}$ & $-0.025^{+0.015}_{-0.014}$ & $-0.025^{+0.015}_{-0.014}$\\
PTF12fuu & $105.4^{+3.0}_{-3.0}$ & $  6.5^{+1.0}_{-1.0}$ & $ 6124^{+  5}_{-  5}$ & $  0.37^{+  0.11}_{-  0.11}$ & $0.178^{+0.026}_{-0.026}$  & $-0.057^{+0.010}_{-0.010}$ & $-0.026^{+0.014}_{-0.013}$ & $-0.026^{+0.014}_{-0.013}$\\
SNF20070820-000 & $107.1^{+3.4}_{-3.4}$ & $ 18.7^{+1.2}_{-1.2}$ & $ 6133^{+ 13}_{- 13}$ & $ -0.28^{+  0.12}_{-  0.12}$ & $-0.054^{+0.031}_{-0.030}$  & $0.051^{+0.012}_{-0.013}$ & $-0.015^{+0.018}_{-0.016}$ & $-0.015^{+0.018}_{-0.016}$\\
SNF20070725-001 & $108.3^{+1.9}_{-1.9}$ & $  9.5^{+1.0}_{-1.1}$ & $ 6139^{+  6}_{-  6}$ & $  0.61^{+  0.15}_{-  0.15}$ & $0.058^{+0.033}_{-0.033}$  & $-0.072^{+0.010}_{-0.007}$ & $-0.018^{+0.016}_{-0.015}$ & $-0.018^{+0.016}_{-0.015}$\\
SNF20071108-021 & $ 98.7^{+2.7}_{-2.7}$ & $  6.0^{+0.8}_{-0.8}$ & $ 6165^{+  5}_{-  5}$ & $  1.15^{+  0.16}_{-  0.15}$ & $-0.045^{+0.028}_{-0.028}$  & $-0.028^{+0.012}_{-0.012}$ & $0.024^{+0.015}_{-0.014}$ & $0.024^{+0.015}_{-0.014}$\\
SNF20080914-001 & $126.4^{+1.2}_{-1.2}$ & $ 16.3^{+1.0}_{-1.0}$ & $ 6159^{+  3}_{-  3}$ & $  0.10^{+  0.14}_{-  0.14}$ & $-0.144^{+0.026}_{-0.024}$  & $0.094^{+0.012}_{-0.012}$ & $0.022^{+0.016}_{-0.015}$ & $0.022^{+0.016}_{-0.015}$\\
SNF20060609-002 & $ 86.3^{+3.5}_{-3.5}$ & $  8.1^{+1.2}_{-1.1}$ & $ 6132^{+  4}_{-  4}$ & $  0.18^{+  0.12}_{-  0.13}$ & $-0.069^{+0.033}_{-0.032}$  & $0.192^{+0.014}_{-0.014}$ & $-0.008^{+0.018}_{-0.017}$ & $-0.008^{+0.018}_{-0.017}$\\
SNF20050624-000 & $119.9^{+5.1}_{-5.1}$ & $  7.9^{+2.4}_{-2.6}$ & $ 6126^{+  5}_{-  5}$ & $  0.47^{+  0.29}_{-  0.29}$ & $-0.171^{+0.026}_{-0.024}$  & $-0.051^{+0.019}_{-0.017}$ & $0.022^{+0.019}_{-0.021}$ & $0.022^{+0.019}_{-0.021}$\\
SNF20060618-023 & $ 73.3^{+4.7}_{-4.7}$ & $  5.3^{+1.6}_{-1.5}$ & $ 6140^{+ 18}_{- 18}$ & $  0.18^{+  0.16}_{-  0.16}$ & $-0.229^{+0.039}_{-0.038}$  & $0.059^{+0.020}_{-0.020}$ & $0.092^{+0.023}_{-0.022}$ & $0.092^{+0.023}_{-0.022}$\\
SNF20080531-000 & $133.2^{+1.5}_{-1.5}$ & $ 17.2^{+0.8}_{-0.8}$ & $ 6113^{+  5}_{-  5}$ & $ -0.63^{+  0.13}_{-  0.13}$ & $0.002^{+0.024}_{-0.024}$  & $-0.040^{+0.009}_{-0.009}$ & $-0.009^{+0.012}_{-0.013}$ & $-0.009^{+0.012}_{-0.013}$\\
SN2006do & $106.3^{+2.1}_{-2.1}$ & $ 26.8^{+1.3}_{-1.2}$ & $ 6101^{+  2}_{-  2}$ & $ -1.95^{+  0.34}_{-  0.34}$ & $-0.077^{+0.031}_{-0.030}$  & $-0.038^{+0.012}_{-0.013}$ & $0.035^{+0.018}_{-0.017}$ & $0.035^{+0.018}_{-0.017}$\\
PTF12ikt & $110.3^{+1.5}_{-1.6}$ & $ 14.0^{+0.7}_{-0.7}$ & $ 6141^{+  4}_{-  4}$ & $ -0.40^{+  0.10}_{-  0.10}$ & $-0.023^{+0.025}_{-0.025}$  & $-0.034^{+0.009}_{-0.008}$ & $-0.009^{+0.014}_{-0.014}$ & $-0.009^{+0.014}_{-0.014}$\\
SN2006dm & $ 99.6^{+1.6}_{-1.6}$ & $ 29.8^{+0.7}_{-0.7}$ & $ 6118^{+  3}_{-  3}$ & $ -2.35^{+  0.18}_{-  0.18}$ & $0.084^{+0.029}_{-0.029}$  & $-0.047^{+0.011}_{-0.011}$ & $0.047^{+0.017}_{-0.016}$ & $0.047^{+0.017}_{-0.016}$\\
PTF13azs & $140.6^{+4.8}_{-4.8}$ & $ 15.2^{+1.3}_{-1.3}$ & $ 6124^{+ 10}_{- 10}$ & $ -0.12^{+  0.11}_{-  0.11}$ & $0.148^{+0.029}_{-0.029}$  & $0.195^{+0.012}_{-0.013}$ & $-0.037^{+0.012}_{-0.009}$ & $-0.037^{+0.012}_{-0.009}$\\
SN2005hj & $ 80.4^{+2.4}_{-2.4}$ & $  4.4^{+0.8}_{-0.8}$ & $ 6139^{+  4}_{-  4}$ & $  2.58^{+  0.24}_{-  0.24}$ & $0.076^{+0.039}_{-0.037}$  & $0.009^{+0.018}_{-0.020}$ & $0.066^{+0.022}_{-0.022}$ & $0.066^{+0.022}_{-0.022}$\\
PTF12iiq & $149.1^{+2.1}_{-2.2}$ & $ 24.8^{+0.8}_{-0.7}$ & $ 6048^{+  7}_{-  7}$ & $ -0.83^{+  0.10}_{-  0.11}$ & $0.005^{+0.033}_{-0.033}$  & $-0.005^{+0.019}_{-0.020}$ & $-0.034^{+0.017}_{-0.010}$ & $-0.034^{+0.017}_{-0.010}$\\
PTF10ndc & $124.1^{+2.4}_{-2.4}$ & $  6.7^{+1.0}_{-1.0}$ & $ 6119^{+  3}_{-  3}$ & $  0.86^{+  0.14}_{-  0.14}$ & $0.007^{+0.026}_{-0.025}$  & $-0.040^{+0.011}_{-0.011}$ & $-0.023^{+0.014}_{-0.013}$ & $-0.023^{+0.014}_{-0.013}$\\
SNF20080919-002 & $100.4^{+6.7}_{-6.6}$ & $ 26.3^{+1.7}_{-1.7}$ & $ 6133^{+  8}_{-  8}$ & $ -1.54^{+  0.13}_{-  0.13}$ & $0.340^{+0.043}_{-0.044}$  & $-0.058^{+0.017}_{-0.014}$ & $0.114^{+0.020}_{-0.021}$ & $0.114^{+0.020}_{-0.021}$\\
SNPGC027923 & $ 85.4^{+0.6}_{-0.6}$ & $  5.8^{+0.3}_{-0.3}$ & $ 6130^{+  4}_{-  4}$ & $  0.45^{+  0.17}_{-  0.17}$ & $0.078^{+0.032}_{-0.031}$  & $-0.077^{+0.006}_{-0.005}$ & $-0.026^{+0.015}_{-0.013}$ & $-0.026^{+0.015}_{-0.013}$\\
SNF20070330-024 & $118.3^{+2.1}_{-2.1}$ & $  5.5^{+1.8}_{-1.8}$ & $ 6101^{+  3}_{-  3}$ & $  0.48^{+  0.19}_{-  0.21}$ & $-0.166^{+0.025}_{-0.024}$  & $-0.040^{+0.014}_{-0.014}$ & $-0.018^{+0.017}_{-0.016}$ & $-0.018^{+0.017}_{-0.016}$\\
SNF20061030-010 & $131.6^{+2.2}_{-2.2}$ & $ 16.9^{+1.1}_{-1.1}$ & $ 6116^{+  4}_{-  4}$ & $ -0.10^{+  0.20}_{-  0.20}$ & $0.103^{+0.041}_{-0.041}$  & $0.049^{+0.012}_{-0.012}$ & $0.019^{+0.021}_{-0.021}$ & $0.019^{+0.021}_{-0.021}$\\
SNhunt46 & $ 93.8^{+2.0}_{-2.0}$ & $ 11.2^{+0.6}_{-0.6}$ & $ 6133^{+  4}_{-  4}$ & $  0.22^{+  0.17}_{-  0.17}$ & $0.150^{+0.032}_{-0.032}$  & $-0.062^{+0.010}_{-0.010}$ & $0.035^{+0.016}_{-0.015}$ & $0.035^{+0.016}_{-0.015}$\\
SN2005hc & $127.2^{+2.4}_{-2.5}$ & $  9.7^{+0.7}_{-0.7}$ & $ 6122^{+  3}_{-  3}$ & $  0.47^{+  0.16}_{-  0.15}$ & $0.137^{+0.025}_{-0.025}$  & $-0.062^{+0.008}_{-0.008}$ & $0.001^{+0.012}_{-0.012}$ & $0.001^{+0.012}_{-0.012}$\\
LSQ12dbr & $106.9^{+0.6}_{-0.6}$ & $  6.4^{+0.6}_{-0.6}$ & $ 6137^{+  4}_{-  4}$ & $  0.86^{+  0.11}_{-  0.11}$ & $0.004^{+0.226}_{-0.109}$  & $-0.071^{+0.008}_{-0.007}$ & $-0.034^{+0.011}_{-0.010}$ & $-0.034^{+0.011}_{-0.010}$\\
LSQ12hjm & $ 65.3^{+10.8}_{-11.3}$ & $ 15.3^{+1.1}_{-1.1}$ & $ 6147^{+  5}_{-  5}$ & $ -0.16^{+  0.17}_{-  0.17}$ & $0.112^{+0.030}_{-0.031}$  & $-0.062^{+0.013}_{-0.011}$ & $-0.042^{+0.008}_{-0.007}$ & $-0.042^{+0.008}_{-0.007}$\\
SNF20060521-001 & $ 89.1^{+11.4}_{-12.2}$ & $ 19.1^{+1.1}_{-1.1}$ & $ 6119^{+ 10}_{-  9}$ & $ -1.70^{+  0.20}_{-  0.20}$ & $0.029^{+0.061}_{-0.059}$  & $-0.068^{+0.014}_{-0.010}$ & $0.010^{+0.019}_{-0.020}$ & $0.010^{+0.019}_{-0.020}$\\
SNF20070630-006 & $126.3^{+3.1}_{-3.1}$ & $ 10.3^{+1.1}_{-1.2}$ & $ 6126^{+  4}_{-  4}$ & $  0.30^{+  0.11}_{-  0.11}$ & $-0.018^{+0.025}_{-0.025}$  & $-0.027^{+0.010}_{-0.010}$ & $-0.040^{+0.010}_{-0.008}$ & $-0.040^{+0.010}_{-0.008}$\\
PTF11drz & $132.6^{+1.4}_{-1.4}$ & $ 14.9^{+0.9}_{-0.9}$ & $ 6116^{+  4}_{-  4}$ & $ -0.11^{+  0.17}_{-  0.18}$ & $-0.007^{+0.025}_{-0.026}$  & $-0.021^{+0.010}_{-0.009}$ & $-0.016^{+0.013}_{-0.013}$ & $-0.016^{+0.013}_{-0.013}$\\
SNF20080323-009 & $ 96.0^{+2.2}_{-2.2}$ & $ 11.1^{+1.0}_{-1.0}$ & $ 6145^{+  5}_{-  5}$ & $ -0.15^{+  0.14}_{-  0.14}$ & $-0.029^{+0.030}_{-0.029}$  & $-0.051^{+0.011}_{-0.011}$ & $-0.041^{+0.009}_{-0.008}$ & $-0.041^{+0.009}_{-0.008}$\\
SNF20071021-000 & $167.9^{+2.2}_{-2.2}$ & $ 20.2^{+0.6}_{-0.6}$ & $ 6111^{+  4}_{-  4}$ & $ -0.66^{+  0.13}_{-  0.13}$ & $-0.105^{+0.025}_{-0.023}$  & $-0.001^{+0.010}_{-0.010}$ & $-0.004^{+0.013}_{-0.013}$ & $-0.004^{+0.013}_{-0.013}$\\
SNNGC0927 & $155.2^{+1.3}_{-1.3}$ & $ 10.9^{+0.7}_{-0.7}$ & $ 6109^{+  4}_{-  4}$ & $  0.39^{+  0.12}_{-  0.12}$ & $-0.031^{+0.027}_{-0.027}$  & $0.039^{+0.010}_{-0.010}$ & $-0.018^{+0.014}_{-0.013}$ & $-0.018^{+0.014}_{-0.013}$\\
SNF20060526-003 & $111.8^{+2.5}_{-2.5}$ & $  9.9^{+1.0}_{-1.0}$ & $ 6121^{+  3}_{-  3}$ & $  0.34^{+  0.13}_{-  0.13}$ & $-0.004^{+0.026}_{-0.027}$  & $-0.010^{+0.010}_{-0.010}$ & $-0.011^{+0.014}_{-0.014}$ & $-0.011^{+0.014}_{-0.014}$\\
SNF20080806-002 & $135.4^{+1.8}_{-1.9}$ & $  8.3^{+0.9}_{-0.9}$ & $ 6136^{+  4}_{-  4}$ & $  0.83^{+  0.14}_{-  0.14}$ & $-0.106^{+0.028}_{-0.027}$  & $0.020^{+0.010}_{-0.010}$ & $-0.036^{+0.013}_{-0.009}$ & $-0.036^{+0.013}_{-0.009}$\\
SNF20080803-000 & $118.3^{+2.5}_{-2.6}$ & $  8.4^{+1.3}_{-1.4}$ & $ 6125^{+  4}_{-  4}$ & $  0.25^{+  0.12}_{-  0.12}$ & $-0.030^{+0.023}_{-0.024}$  & $0.100^{+0.011}_{-0.011}$ & $-0.040^{+0.009}_{-0.008}$ & $-0.040^{+0.009}_{-0.008}$\\
SNF20080822-005 & $ 78.6^{+1.8}_{-1.8}$ & $  6.2^{+0.8}_{-0.8}$ & $ 6138^{+  4}_{-  4}$ & $  0.51^{+  0.16}_{-  0.16}$ & $-0.027^{+0.030}_{-0.030}$  & $-0.018^{+0.011}_{-0.010}$ & $0.014^{+0.015}_{-0.015}$ & $0.014^{+0.015}_{-0.015}$\\
SNF20060618-014 & $136.8^{+2.4}_{-2.5}$ & $  9.8^{+1.1}_{-1.1}$ & $ 6113^{+  7}_{-  7}$ & $  1.20^{+  0.18}_{-  0.18}$ & $0.077^{+0.047}_{-0.047}$  & $-0.037^{+0.014}_{-0.014}$ & $0.019^{+0.025}_{-0.024}$ & $0.019^{+0.025}_{-0.024}$\\
PTF12ghy & $100.6^{+3.5}_{-3.5}$ & $ 16.4^{+0.7}_{-0.7}$ & $ 6133^{+  3}_{-  3}$ & $ -0.58^{+  0.10}_{-  0.10}$ & $-0.138^{+0.023}_{-0.022}$  & $0.196^{+0.011}_{-0.011}$ & $0.001^{+0.013}_{-0.013}$ & $0.001^{+0.013}_{-0.013}$\\
SNF20070531-011 & $122.2^{+2.6}_{-2.6}$ & $ 21.5^{+0.8}_{-0.8}$ & $ 6114^{+  3}_{-  4}$ & $ -1.56^{+  0.20}_{-  0.20}$ & $-0.081^{+0.027}_{-0.025}$  & $-0.032^{+0.011}_{-0.011}$ & $-0.000^{+0.015}_{-0.014}$ & $-0.000^{+0.015}_{-0.014}$\\
SNF20070831-015 & $112.9^{+2.6}_{-2.6}$ & $  6.9^{+0.9}_{-0.9}$ & $ 6144^{+  6}_{-  6}$ & $  1.38^{+  0.16}_{-  0.16}$ & $0.042^{+0.027}_{-0.027}$  & $-0.016^{+0.010}_{-0.010}$ & $0.002^{+0.014}_{-0.014}$ & $0.002^{+0.014}_{-0.014}$\\
SNF20070417-002 & $103.8^{+5.0}_{-4.9}$ & $ 24.1^{+1.3}_{-1.3}$ & $ 6125^{+  8}_{-  9}$ & $ -1.37^{+  0.10}_{-  0.10}$ & $-0.032^{+0.050}_{-0.044}$  & $-0.072^{+0.010}_{-0.007}$ & $-0.039^{+0.010}_{-0.008}$ & $-0.039^{+0.010}_{-0.008}$\\
PTF11cao & $143.4^{+1.5}_{-1.6}$ & $ 19.0^{+1.1}_{-1.1}$ & $ 6104^{+  4}_{-  4}$ & $  0.20^{+  0.13}_{-  0.13}$ & $-0.014^{+0.032}_{-0.031}$  & $0.009^{+0.013}_{-0.013}$ & $-0.017^{+0.017}_{-0.015}$ & $-0.017^{+0.017}_{-0.015}$\\
SNF20080522-000 & $ 62.1^{+3.5}_{-3.5}$ & $  3.1^{+0.8}_{-0.8}$ & $ 6130^{+  7}_{-  7}$ & $  0.99^{+  0.11}_{-  0.11}$ & $-0.021^{+0.028}_{-0.028}$  & $-0.009^{+0.012}_{-0.013}$ & $0.047^{+0.014}_{-0.013}$ & $0.047^{+0.014}_{-0.013}$\\
PTF10qjq & $ 74.4^{+2.3}_{-2.4}$ & $ 12.0^{+0.7}_{-0.7}$ & $ 6132^{+  3}_{-  3}$ & $ -0.50^{+  0.10}_{-  0.09}$ & $-0.127^{+0.026}_{-0.024}$  & $0.049^{+0.010}_{-0.011}$ & $-0.010^{+0.015}_{-0.015}$ & $-0.010^{+0.015}_{-0.015}$\\
PTF12dxm & $119.3^{+15.5}_{-16.8}$ & $ 31.3^{+1.9}_{-2.1}$ & $ 6136^{+  4}_{-  4}$ & $ -2.14^{+  0.18}_{-  0.18}$ & $0.034^{+0.034}_{-0.033}$  & $-0.047^{+0.027}_{-0.021}$ & $0.067^{+0.023}_{-0.024}$ & $0.067^{+0.023}_{-0.024}$\\
SNF20061021-003 & $122.5^{+2.2}_{-2.2}$ & $ 10.8^{+1.5}_{-1.5}$ & $ 6131^{+  4}_{-  4}$ & $  0.39^{+  0.15}_{-  0.15}$ & $-0.012^{+0.035}_{-0.033}$  & $0.036^{+0.014}_{-0.014}$ & $0.047^{+0.019}_{-0.017}$ & $0.047^{+0.019}_{-0.017}$\\
SNF20080510-005 & $112.2^{+2.5}_{-2.5}$ & $  7.0^{+1.0}_{-1.0}$ & $ 6117^{+  4}_{-  4}$ & $  1.68^{+  0.22}_{-  0.22}$ & $0.116^{+0.030}_{-0.030}$  & $-0.029^{+0.013}_{-0.015}$ & $-0.042^{+0.008}_{-0.007}$ & $-0.042^{+0.008}_{-0.007}$\\
SNF20080507-000 & $ 98.4^{+1.6}_{-1.6}$ & $  8.0^{+1.7}_{-1.6}$ & $ 6142^{+  5}_{-  5}$ & $  0.56^{+  0.18}_{-  0.18}$ & $-0.126^{+0.023}_{-0.022}$  & $0.063^{+0.014}_{-0.013}$ & $-0.003^{+0.015}_{-0.014}$ & $-0.003^{+0.015}_{-0.014}$\\
SNF20080913-031 & $117.9^{+1.5}_{-1.5}$ & $ 14.7^{+1.6}_{-1.6}$ & $ 6159^{+  5}_{-  5}$ & $  0.49^{+  0.17}_{-  0.17}$ & $-0.048^{+0.067}_{-0.052}$  & $-0.003^{+0.015}_{-0.015}$ & $-0.009^{+0.020}_{-0.018}$ & $-0.009^{+0.020}_{-0.018}$\\
SNF20080510-001 & $118.9^{+2.0}_{-2.1}$ & $ 15.0^{+1.1}_{-1.0}$ & $ 6115^{+  4}_{-  4}$ & $  0.34^{+  0.14}_{-  0.14}$ & $0.001^{+0.026}_{-0.026}$  & $-0.058^{+0.011}_{-0.012}$ & $-0.032^{+0.012}_{-0.011}$ & $-0.032^{+0.012}_{-0.011}$\\
SNF20070712-003 & $108.1^{+2.7}_{-2.6}$ & $ 14.1^{+0.9}_{-0.9}$ & $ 6157^{+  6}_{-  6}$ & $  0.62^{+  0.14}_{-  0.14}$ & $-0.034^{+0.030}_{-0.029}$  & $-0.052^{+0.010}_{-0.011}$ & $-0.021^{+0.015}_{-0.014}$ & $-0.021^{+0.015}_{-0.014}$\\
\enddata
\end{deluxetable}
}




The distributions of the Monte Carlo links for all supernovae are shown in Figure~\ref{perobject3:fig}.
Each supernova is represented by a cloud of its parameters' links.
There is a core concentration in the  parameter-space, with around eight objects that occupy its outskirts.
Many outliers appear in the red tail of $E_{\gamma^0}({\hat{B}}-{\hat{V}})$, as would be expected for the (infrequent) selection of supernovae
heavily extinguished by host-galaxy dust.

The Pearson correlation coefficients of the per-supernova parameters are given in the matrix
\begin{multline}
Cor(EW_{\mathrm{Ca}}, EW_{\mathrm{Si}}, \lambda_{\mathrm{Si}}, x_1, E_{\gamma^0}({\hat{B}}-{\hat{V}}), E_{\gamma^1}({\hat{B}}-{\hat{V}}), A_{p,V}) =\\
\begin{pmatrix}
\begin{array}{rrrrrrr}
EW_{\mathrm{Ca}} & 0.10^{+0.03}_{-0.03} & -0.25^{+0.03}_{-0.03} & 0.07^{+0.03}_{-0.03} & -0.10^{+0.05}_{-0.05} & -0.18^{+0.06}_{-0.06} & 0.00^{+0.08}_{-0.08} \\
\ldots &EW_{\mathrm{Si}} & -0.12^{+0.03}_{-0.03} & -0.84^{+0.01}_{-0.01} & -0.13^{+0.04}_{-0.04} & 0.27^{+0.08}_{-0.09} & -0.00^{+0.08}_{-0.08} \\
\ldots & \ldots& \lambda_{\mathrm{Si}} & 0.14^{+0.03}_{-0.03} & 0.01^{+0.06}_{-0.05} & 0.00^{+0.06}_{-0.06} & -0.00^{+0.08}_{-0.08} \\
\ldots& \ldots& \ldots& x_1 & 0.13^{+0.04}_{-0.04} & -0.20^{+0.09}_{-0.09} & -0.00^{+0.08}_{-0.08} \\
\ldots& \ldots& \ldots & \ldots& E_{\gamma^0}({\hat{B}}-{\hat{V}})& 0.04^{+0.06}_{-0.06} & 0.00^{+0.08}_{-0.08} \\
\ldots & \ldots & \ldots& \ldots&\ldots & E_{\gamma^1}({\hat{B}}-{\hat{V}})  & 0.00^{+0.08}_{-0.08} \\
\ldots & \ldots & \ldots&\ldots &\ldots &\ldots &  A_{p,V} \\
\end{array}
\end{pmatrix}.
\label{corr:eqn}
\end{multline}
Recall that the $\Delta$-cofactor degeneracy is broken by enforcing zero correlation between $\Delta$ and the features.
To aid readability, only the upper triangular part of the otherwise symmetric matrix is shown and the corresponding feature label is given on
the diagonal.

\begin{figure}[htbp] %  figure placement: here, top, bottom, or page
   \centering
   \includegraphics[width=5.2in]{perobject_corner.pdf} 
   \caption{
   Distributions for the supernova parameters $\Delta$, $EW_{\mathrm{Ca}}$, $EW_{\mathrm{Si}}$, $\lambda_{\mathrm{Si}}$, $x_1$, $E_{\gamma^0}({\hat{B}}-{\hat{V}})$,  $E_{\gamma^1}({\hat{B}}-{\hat{V}})$,  and  $A_{p,V}$, as well as the grey offset
$\Delta$.  All Monte Carlo links are plotted, so that each supernova contributes a cloud of points.
   \label{perobject3:fig}}
\end{figure}

\added{
The correlation matrix in Eq.~\ref{corr:eqn} is not close to being singular, meaning that the each of the features conveys
independent supernova variability.  Nevertheless, a strong correlation between two features means that the determination
of one would reduce the residual dispersion in the other.  The amount of residual dispersion is conveyed in the matrix
\begin{equation}
Z =
\begin{pmatrix}
\scriptsize
\begin{array}{rrrrrrr}
27.589  \pm 0.554 & 27.451  \pm 0.572 & 26.668  \pm 0.589 & 27.509  \pm 0.542 & 27.408  \pm 0.563 & 27.099  \pm 0.640 & 27.506  \pm 0.562 \\
7.041  \pm 0.160 & 7.076  \pm 0.157 & 7.019  \pm 0.165 & 3.860  \pm 0.186 & 7.011  \pm 0.158 & 6.790  \pm 0.206 & 7.056  \pm 0.159 \\
19.779  \pm 0.449 & 20.295  \pm 0.458 & 20.463  \pm 0.460 & 20.267  \pm 0.456 & 20.430  \pm 0.461 & 20.423  \pm 0.462 & 20.404  \pm 0.465 \\
0.992  \pm 0.018 & 0.543  \pm 0.023 & 0.986  \pm 0.019 & 0.995  \pm 0.018 & 0.986  \pm 0.019 & 0.971  \pm 0.026 & 0.992  \pm 0.019 \\
0.073  \pm 0.003 & 0.073  \pm 0.003 & 0.074  \pm 0.003 & 0.073  \pm 0.003 & 0.074  \pm 0.003 & 0.074  \pm 0.003 & 0.074  \pm 0.003 \\
0.038  \pm 0.004 & 0.037  \pm 0.004 & 0.039  \pm 0.004 & 0.038  \pm 0.004 & 0.039  \pm 0.004 & 0.039  \pm 0.004 & 0.039  \pm 0.004 \\
0.050  \pm 0.011 & 0.050  \pm 0.011 & 0.050  \pm 0.011 & 0.050  \pm 0.011 & 0.050  \pm 0.011 & 0.050  \pm 0.011 & 0.050  \pm 0.011 
\\\end{array}
\end{pmatrix},
\label{ellipse:eqn}
\end{equation}
where the associated units are [\AA, \AA, \AA, 1, mag, mag, mag]$\cdot$[\AA, \AA, \AA, 1, mag, mag, mag]$^{T}$.

For indices $i, j \in \{EW_{\mathrm{Ca}}, EW_{\mathrm{Si}}, \lambda_{\mathrm{Si}}, x_1, E_{\gamma^0}({\hat{B}}-{\hat{V}}), E_{\gamma^1}({\hat{B}}-{\hat{V}}), A_{p,V}\}$, $M_{ii}$ is the standard deviation of the parameter $i$ distribution, and for $i \ne j$ $Z_{ij}$ is the standard deviation in parameter $i$
if parameter $j$ were known perfectly.  For example, for $i=EW_{\mathrm{Si}}$ the standard deviation is 7.08\AA\ but when $j=x_1$
is fixed its
standard deviation drops to 3.86\AA.  Therefore, an analysis that considers $x_1$ but not $EW_{\mathrm{Si}}$ would not account for
3.86\AA$\vec{\beta}=\{0.14 ,  0.09,  0.08,  0.07,  0.04 \}$~mag ${\hat{U}}{\hat{B}}{\hat{V}}{\hat{R}}{\hat{I}}$ magnitude dispersions
that are not intrinsically random.
}

An extensive discussion of the correlations between the spectral features  and light-curve parameters of the SNfactory data
set can be found in \citet{chotard:thesis, leget:thesis, 2017Chotard}. 
We here confine ourselves to noting the strong anti-correlation between $EW_{\mathrm{Si}}$ and $x_1$, which was alluded
to in \S\ref{results3global:sec} when interpreting the correlation between the cofactors $\vec{\beta}$ and $\vec{\zeta}$ that connect them with magnitudes.

The $\Delta$ parameter will be discussed in its own subsection \S\ref{Delta:sec}.

\subsubsection{Results for the Latent  Parameters $g_0$, $g_1$}
\label{g0g1:sec}
The vector
$\vec{\gamma}^0$ describes color changes for an effective $\langle R^F_{\mathit{eff}}\rangle=2.43$ dust, as shown 
in  \S\ref{results3gamma:sec}.
Hence, the per-supernova parameter $\gamma^0_{\hat{V}} g_0$ is associated with the $\hat{V}$-band
extinction for a constant $R^F=2.43$.
Figure~\ref{k0_med:fig} shows the histogram of
per-supernova
median values
from the Monte Carlo chains
and the  stack of the posteriors of all supernovae  for $\gamma^0_{\hat{V}} g_0$ 
relative to that of an arbitrary supernova  $\gamma^0_{\hat{V}} g_0|_0$.
The distributions are non-Gaussian, having a sharp rise in the blue and an extended tail in the red.  This is consistent
with
simulations based on expected dust distributions
within galaxies and the distribution of galaxy orientations with respect to the observer,
which determine column densities along the lines of sight toward supernovae that are consistent with observations
\citep{1998ApJ...502..177H, 2007ApJ...659..122J}.  
We emphasize that unlike other analyses,
this distribution is given by the data, and so is not dependent on any knowledge or prior
for the distribution of dust extinction the SN population is expected to suffer. 
There is no sign of two populations in $\gamma^1_{\hat{V}} g_1$, which would be expected from a subpopulation
experiencing extinction and scattering by circumstellar dust.

\begin{figure}[htbp] %  figure placement: here, top, bottom, or page
   \centering
   \includegraphics[width=2.8in]{deltagamma0_med.pdf}
   \includegraphics[width=2.8in]{deltagamma1_med.pdf}
      \caption{
      Normalized stack of the posteriors (includes parameter uncertainty) of all supernovae  and histogram 
      of  per-supernova 
median values of: (left) $\gamma^0_{\hat{V}} g_0-\gamma^0_{\hat{V}} g_0|_0$, which is associated with the $A^F_V$ of an $\langle R^F_{\mathit{eff}}\rangle \sim 2.43$ dust model;
(right) $\gamma^1_{\hat{V}} g_1-\gamma^1_{\hat{V}} g_1|_0$,
which  is associated with extinction corrections due to deviations away from the canonical $\langle R^F_{\mathit{eff}}\rangle$ value.
   \label{k0_med:fig}}
\end{figure}

An instructive way to consider the  $\{g_0, g_1\}$ parameters is
to transform them to $\{A_V^F, E^F(B-V)\}$ (modulo additive constants)   using the
matrix $M$ in  Eq.~\ref{trans_I:eqn}.
A plot showing the expected values and 68\% credible intervals of these parameters
for our supernova
sample is shown in Figure~\ref{kk:fig}. 
For reference, a line that represents $\langle R^F_{\mathit{eff}}\rangle=2.43$ is overplotted; points above
the line have larger $R^F$, points below have smaller $R^F$.  The majority of supernovae lie within a narrow range above the line,
while the remaining fraction fall in a broader range below the line.
Comparison with  $\gamma^1_{\hat{V}} g_1-\gamma^1_{\hat{V}} g_1|_0$ of the right plot of Figure~\ref{k0_med:fig}
shows that the 
negative tail corresponds to smaller $R^F$, and the sharper positive edge to larger $R^F$.
These findings are qualitatively consistent with previous results:
using fixed dust-extinction models,
\citet{2014ApJ...789...32B, 2015MNRAS.453.3300A} deduce a wide range of dust behavior $1.5<R^F<3$ encountered by the SN~Ia population.
Our model cannot provide the range of $R^F$ in our sample nor per-supernova determinations of $R^F$ 
without further assumptions.

\begin{figure}[htbp] %  figure placement: here, top, bottom, or page
   \centering
   \includegraphics[width=4in]{avebv_synth.pdf}
      \caption{
      Expected values and 68\% credible intervals of effective $E^F(B-V)$ and $A_V^F$ after transformation from our model $g_0$ and $g_1$ parameters, for the supernova in our sample.
      Overplotted is a line with the slope expected for $R^F=2.43$.
   \label{kk:fig}}
\end{figure}



There are $>2 \sigma$ correlations between our  color-excess parameters $g_0$, $g_1$
and the input features  $EW_{\mathrm{Ca}}$,
$EW_{\mathrm{Si}}$, and $x_1$. 
Given the association
of  $g$ with dust-extinction parameters, this implies that the host environment and the input features are connected.
The correlations between $EW_{\mathrm{Si}}$ and  $EW_{\mathrm{Ca}}$  with host have already been noted \citep{2011ApJ...734...42N, 2015MNRAS.451.1973S}.
There is a lack of correlation between  $g_0$, $g_1$ and  our velocity parameter $\lambda_{\mathrm{Si}}$
(measured at peak brightness).
This contrasts with the subset of supernovae identified by their high-velocity features (HVF) in
Si~II~$\lambda$6355
(typically detected pre-maximum), who are found to be distinguished by their host galaxy properties
\citep[e.g.][]{2014MNRAS.444.3258M, 2015MNRAS.446..354P, 2015ApJS..220...20Z}.
The population and properties of HVF SNe within the SNfactory sample will be presented in \citet{2018lin}.
The correlation of $x_1$ with  host-galaxy (including dust) properties is  well established 
\citep{2000AJ....120.1479H, 2003MNRAS.340.1057S}.


\subsubsection{Results for the Parameter $p$}
\label{p:sec}
The model assumes that $p$ has a Normal distribution, which is shown in Figure~\ref{ebv3:fig} as the 
stack of the posteriors of all supernovae of the color-excess $E_p({\hat{B}}-{\hat{V}})$.
Also shown for comparison is the distribution of
the   color-excess $E_\gamma({\hat{B}}-{\hat{V}}) \equiv (\gamma^0_{\hat{B}} -\gamma^0_{\hat{V}}) g_0
+ (\gamma^1_{\hat{B}} -\gamma^1_{\hat{V}}) g_1$.
The standard deviations of  $E_\gamma({\hat{B}}-{\hat{V}})$ and $E_p({\hat{B}}-{\hat{V}})$ are
%-----
0.085 %new
and 0.010
%-----
mag respectively.
The latter is comparable to the typical measurement uncertainty.
In $\hat{B}-\hat{V}$, 
the $g$-based color variation is much larger than that of the $p$-based and its effect per-supernova would be difficult to discern.

\begin{figure}[htbp] %  figure placement: here, top, bottom, or page
   \centering
   \includegraphics[width=4in]{ebv.pdf}
      \caption{Stack of the posteriors  (includes parameter uncertainty) of all supernovae of the  $E_{\gamma}({\hat{B}}-{\hat{V}})$ 
      and the 
 $E_p({\hat{B}}-{\hat{V}})$  contributions to color excess relative to an arbitrary supernova.
   \label{ebv3:fig}}
\end{figure}


There is no significant correlation between the parameter $p$ and the other features.
Their independence from the  $g_0$ and $g_1$ parameters
implies that two physical parameters are not being artificially attributed to three
model parameters.  One could have
worried this might
occur since our linear model does not precisely
describe the non-linearity between broad-band magnitudes and dust parameters. A
correlation 
between the parameters would have complicated any claims of the detection of a third independent supernova parameter.

Detection of intrinsic color has been previously reported by \citetalias{2017ApJ...842...93M}.
For data, they use the outputs of SALT2 fits: the $c$ color parameter;
the
supernova absolute magnitude $M$.
In the \citetalias{2017ApJ...842...93M}
model, intrinsic color contributes a linear   $\beta_{\mathit{int}} c$ to  the absolute
magnitude.
They find a significant $\beta_{\mathit{int}} = 2.2\pm 0.3$.
While we too claim the detection of a color parameter incompatible with dust,
the differences between our models make it
difficult to determine whether the results are consistent.   Our results show effects on color that are not monotonic
with wavelength.  We find that the statistic that nominally best corresponds to  $\beta_{\mathit{int}}$
to be
$\phi_{\hat{B}}/(\phi_{\hat{B}}-\phi_{\hat{V}}) =-4.4_{  -6.4}^{+   2.4}$, which appears to be inconsistent with  \citetalias{2017ApJ...842...93M}.  
However a change of the color baseline gives a significantly different
$\phi_{\hat{B}}/(\phi_{\hat{B}}-\phi_{\hat{R}}) =4.3_{  -1.4}^{+   3.8}$.  Our results indicate that  $\beta_{\mathit{int}}$
is sensitive to the color that $c$ corresponds to.  Although SALT2 $c$ is calibrated to correspond with $B-V$, it is
determined using data from all bands with the wavelength extremes  typically providing the strongest leverage.
The $c$ parameter is thus an  amalgamation of all input colors.
We therefore make no conclusions on the consistency between our $\vec{\phi}$-color properties
and those found by \citetalias{2017ApJ...842...93M}.


An independent piece of information available for a subset of our supernova sample is host-galaxy mass \citep{2013ApJ...770..108C}.  
This statistic is of interest, as there is a correlation between Hubble residual and host-galaxy mass
\citep[first noted by][]{2010ApJ...715..743K,2010MNRAS.406..782S}, a signal confirmed to exist in the SNfactory
sample \citep{2013ApJ...770..108C}.
This host-mass bias could be the result of a parameter that was not accounted for in the inference of SN~Ia absolute magnitude.
Indeed  \citetalias{2017ApJ...842...93M} find that the  introduction of intrinsic color as a latent parameter
reduces the strength of  this bias.

It is possible that the  $p$ parameter is a supernova-tracer of a population that to date has only been identified through host-galaxy tracers.
We plot in Figure~\ref{childress3:fig} our parameter
 $\sigma_p\phi_{\hat{V}}(p-p|_0) $  (subtracting out a random supernova
to suppress correlated errors) versus host mass
for the subset of supernovae whose host measurements are given in \citet{2013ApJ...770..108C}.
Low- and high-mass galaxies, divided by $\log{(M/M_\sun)}=10$, have high probability of hosting supernovae with different $p$ distributions, as
the Kolmogorov-Smirnov test gives a two-tailed $p$-value of $  0.016 _ {     0.011 } ^{     0.044 }$.
The median and median uncertainties of the two subsamples are
%---
$\langle \sigma_p\phi_{\hat{V}}(p-p|_0) \rangle=  0.041 \pm {     0.007 }$ mag,
$\langle \sigma_p\phi_{\hat{V}}(p-p|_0)  \rangle=0.065 \pm {     0.007 }$ mag
for low- and high-mass hosts respectively, with a difference of $ 0.024 \pm {     0.010 }$~mag that is significant at $>2 \sigma$.

In this article we do not associate $p$ directly with Hubble residuals. This is left for future work.

\begin{figure}[htbp] %  figure placement: here, top, bottom, or page
   \centering
   \includegraphics[width=4in]{childress.pdf} 
   \caption{Parameter $\sigma_p\phi_{\hat{V}}p$  versus host galaxy mass. Overplotted are the mean and 1$\sigma$ uncertainty on the mean for supernovae with hosts
      less than and greater than  $\log{(M/M_\sun)}=10$.
 $p$ as a function of host-galaxy mass
    \label{childress3:fig}}
\end{figure}


\subsubsection{Results for the Magnitude Offset $\Delta$}
\label{Delta:sec}

The color-standardization leaves magnitude residuals $\Delta$.
The histogram of the per-supernova  medians of these relative  grey offsets $\Delta-\Delta|_0$,
is shown in Figure~\ref{hist:fig}.  The distribution
has a total standard deviation of
%-----
$0.13$
%-----
mag, and a tail in the positive (fainter) direction.
Supernova peculiar velocities contribute to the standard deviation; of the four
supernovae with the highest values of $\Delta$, two have 
heliocentric
redshifts of 0.0015 and 0.0086. 
\replaced{
Eq.~\ref{corr:eqn} shows $>2 \sigma$ correlations between $\Delta$ and each of $\lambda_{\mathrm{Si}}$, $x_1$, $E_{\gamma^1}({\hat{B}}-{\hat{V}})$, and
$\sigma_p\phi_{\hat{V}}p$, showing that in addition to standardizing colors, these features can also be used to standardize
absolute magnitudes.  }{Recall that the model degeneracy between $\Delta$ and the cofactors (e.g.\ $\alpha$, $\beta$) is removed by enforcing
zero correlation between $\Delta$ and the per-supernova parameters (e.g.\ $EW_{\mathrm{Ca}}$,  $EW_{\mathrm{Si}}$).
An alternative method to break the degeneracy is to minimize the dispersion in $\Delta$.
}
The focus of this article is color standardization so we defer consideration of absolute-magnitude standardization to future work.
 
\begin{figure}[htbp] %  figure placement: here, top, bottom, or page
   \centering
   \includegraphics[width=4in]{deltaDelta_hist.pdf} 
   \caption{
   Normalized stack of the posteriors  (includes parameter uncertainty) of all supernovae and histogram of the per-supernova medians of the grey offset $\Delta$. 
   To help null correlated errors, we select
    an arbitrary supernova and subtract out its $\Delta|_0$ from those of all other objects at the level of each MCMC link.
  Of the four
supernovae with the highest values of $\Delta$, two have 
heliocentric
redshifts of 0.0015 and 0.0086. 
   \label{hist:fig}}
\end{figure}



\section{Conclusions}
\label{conclusions:sec}
To summarize, we model SN~Ia broadband optical peak magnitudes allowing for correlations with spectral features at peak,
light-curve shape, and
latent color parameters.  Analyzing SNfactory data with this model, we find significant evidence that the above parameters do
affect supernova magnitudes and colors.  Two of the latent color parameters are consistent with the
\citetalias{1999PASP..111...63F} dust-extinction model, making this
the first determination of the dust extinction curve outside the Local Group
derived entirely independently of assumptions about the shape of the extinction curve and/or assumptions about the
distribution of $A_V$.  
We find a broad distribution for total-to-selective extinction
with an effective
$\langle R^F_{\mathit{eff}}\rangle \sim 2.43$ and an asymmetric tail extending toward lower values.
We identify a new
parameter that affects supernova
colors in a manner that is distinct from the expectations for dust
and the SALT2 shape-parameter $x_1$.
This new parameter correlates with host-galaxy mass, meaning it is a candidate supernova observable that may be linked
with the Hubble diagram mass-step.
A significant bias in $R^F$ is obtained when this third source of color variation
is unaccounted for.

Just as supernova spectral and light curve features are associated with distinct color variations, the latent parameter $p$
identified through color variation may be associated with a spectral and/or light curve feature not included
in our analysis.  It
is of interest to look for tracers other than color to account for this supernova diversity.
We leave to future work the search for correlations between $p$ and other
observable features accessed by SNfactory data.

Using different arguments based on expected populations, \citetalias{2017ApJ...842...93M} infer
separate intrinsic and dust-based color-magnitude relations based on an independent sample of 248 SNe~Ia
mainly from the CfA \citep{1999AJ....117..707R, 2006AJ....131..527J, 2009ApJ...700.1097H, 2012ApJS..200...12H}
and CSP \citep{2010AJ....139..519C, 2011AJ....142..156S} surveys.
Finding slight color perturbations in the multi-band light curves 
of those samples that are  consistent with our $\sigma_p \vec{\phi}$ would provide strong evidence 
that we and  \citetalias{2017ApJ...842...93M}  are seeing the same color effect.

The current data shows a statistically significant difference in the distribution of our new parameter for supernovae
in low- and high-mass host-galaxies.
Using the SNfactory sample,
\citet{2017Rigault} find that a step in Hubble residuals is better related to local 
specific
star 
formation rate, rather than
global host mass.  The search for a correlation between our parameter and local star formation rate is planned for future work.

An interesting direction for future work is to use narrower bands for generating synthetic photometry
from the SNfactory spectrophotometry.  When the bandwidth becomes small enough
to resolve spectral features, the analysis would produce a spectroscopic model.   Higher resolution would allow us to forego 
the use of a fiducial template necessary to predict broadband fluxes generated by dust models, and also allow direct incorporation
of dust models into our framework.


The approach of our analysis is to mine for new supernova properties based on colors and spectral features.
Our model and results do carry information on absolute magnitude but are not tailored for its study.
The grey parameter $\Delta$, while containing information on absolute magnitude, 
also 
includes
contributions from peculiar velocities and measurement uncertainties.  Studies
focused on improving SNe~Ia as standard candles 
would need to
modify  our model to
distinguish between these sources of greyness and dispersion.
Our
mining exercise uses all supernovae that pass quality cuts, the approach taken by previous
exploratory work.  In contrast, the calibration of SNe~Ia as standard candles needs procedures such
as cross-validation to avoid overtraining.
An absolute magnitude calibration using the parameters identified in this analysis is left to future work.

The existence of three color parameters  has important implications when SNe~Ia are used as distance indicators.
Measurements in at least four bands are necessary to obtain three colors.  
Supernova distance determinations that do not account for three color parameters are susceptible to 
systematic errors.  
Parameter populations evolving with redshift would result in errors in
Hubble diagrams.
The use of  correlators with the SN parameters, such as host-galaxy properties e.g.\ as shown for our $p$ parameter,
may partially mitigate such errors.

\acknowledgments
We thank the STAN team for providing the statistical tool without which this analysis would not have been possible,
and Michael Betancourt specifically for his helpful guidance.  
Distribution surfaces are plotted using the ChainConsumer package \citep{Hinton2016}.
We thank Danny Goldstein and
Xiaosheng Huang for useful discussions.
We thank Dan Birchall for observing assistance, the technical and
scientific staffs of the Palomar Observatory, the High Performance
Wireless Radio Network (HPWREN), and the University of Hawaii 2.2~m
telescope.  We recognize the significant cultural role of Mauna Kea
within the indigenous Hawaiian community, and we appreciate the
opportunity to conduct observations from this revered site.  This
work was supported in part by the Director, Office of Science,
Office of High Energy Physics, of the U.S. Department of Energy
under Contract No. DE-AC02- 05CH11231.  Support in France was
provided by CNRS/IN2P3, CNRS/INSU, and PNC; LPNHE acknowledges
support from LABEX ILP, supported by French state funds managed by
the ANR within the Investissements d'Avenir programme under reference
ANR-11-IDEX-0004-02.  Support in Germany was provided by the DFG
through TRR33 ``The Dark Universe;'' and in China from Tsinghua
University 985 grant and NSFC grant No~11173017.  Some results were
obtained using resources and support from the National Energy
Research Scientific Computing Center, supported by the Director,
Office of Science, Office of Advanced Scientific Computing Research,
of the U.S. Department of Energy under Contract No. DE-AC02-05CH11231.
HPWREN is funded by National Science Foundation Grant Number
ANI-0087344, and the University of California, San Diego.
This project has received funding from the European Research Council (ERC) under the European Union's Horizon 2020 research and innovation programme (grant agreement No 759194 - USNAC 

\appendix
\section{Results from Model~I and II}
\label{m12results:sec}
For each of the five filters, the 68\%  equal-tailed credible intervals for the global parameters $\vec{\alpha}$, $\vec{\beta}$, $\vec{\eta}$,
$\vec{\gamma}^0$ and $\vec{\gamma}^1$ of Model~I are given in  Table~\ref{global2:tab}.
The results for the same parameters and $\sigma_p\vec{\phi}$
of Model~II are given in Table~\ref{global2:tab}.  In contrast to Model~III,
Figure~\ref{M1ev:fig} shows that the $\hat{I}$-band has
the largest separation between the degenerate solution of $\sigma_p \phi_X$.  To break the degeneracy we enforce 
$\vec{\phi} = \text{sign}(\phi_{\hat{I}}) \vec{\phi}$.


\begin{table}
\centering
\begin{tabular}{|c|c|c|c|c|c|}
\hline
Parameters& $X={\hat{U}}$ &${\hat{B}}$&${\hat{V}}$&${\hat{R}}$&${\hat{I}}$\\ \hline
$\alpha_X$
& $0.0037^{+0.0003}_{-0.0004}$
& $0.0011^{+0.0003}_{-0.0003}$
& $0.0012^{+0.0003}_{-0.0003}$
& $0.0013^{+0.0003}_{-0.0003}$
& $0.0026^{+0.0002}_{-0.0003}$
\\
${\alpha_X/\alpha_V-1}$
& $   2.0^{+   0.5}_{  -0.3}$
& $  -0.1^{+   0.1}_{  -0.1}$
& \ldots
& $   0.1^{+   0.1}_{  -0.0}$
& $   1.1^{+   0.4}_{  -0.2}$
\\
$\beta_X$
& $ 0.026^{+ 0.002}_{-0.002}$
& $ 0.019^{+ 0.001}_{-0.001}$
& $ 0.022^{+ 0.001}_{-0.001}$
& $ 0.018^{+ 0.001}_{-0.001}$
& $ 0.018^{+ 0.001}_{-0.001}$
\\
${\beta_X/\beta_V-1}$
& $  0.17^{+  0.02}_{ -0.02}$
& $ -0.14^{+  0.02}_{ -0.02}$
& \ldots
& $ -0.18^{+  0.01}_{ -0.01}$
& $ -0.16^{+  0.01}_{ -0.01}$
\\
$\eta_X$
& $-0.0006^{+0.0005}_{-0.0005}$
& $-0.0007^{+0.0004}_{-0.0004}$
& \ldots
& $0.0002^{+0.0003}_{-0.0004}$
& $-0.0007^{+0.0003}_{-0.0003}$
\\
${\eta_X/\eta_V-1}$
& $ -0.30^{+  4.06}_{ -3.18}$
& $ -0.73^{+  5.16}_{ -3.69}$
& \ldots
& $ -0.02^{+  0.81}_{ -0.97}$
& $ -1.03^{+  5.25}_{ -3.77}$
\\
$\gamma^0_X$
& $ 76.37^{+  3.19}_{ -3.23}$
& $ 60.22^{+  3.09}_{ -3.20}$
& $ 43.06^{+  2.95}_{ -3.11}$
& $ 32.93^{+  2.60}_{ -2.87}$
& $ 22.89^{+  2.27}_{ -2.65}$
\\
${\gamma^0_X/\gamma^0_V-1}$
& $  0.78^{+  0.06}_{ -0.06}$
& $  0.40^{+  0.03}_{ -0.03}$
& \ldots
& $ -0.24^{+  0.01}_{ -0.01}$
& $ -0.47^{+  0.02}_{ -0.03}$
\\
$\gamma^1_X$
& $-18.64^{+  4.86}_{ -4.86}$
& $-28.10^{+  4.23}_{ -4.20}$
& $-28.71^{+  3.75}_{ -3.70}$
& $-21.86^{+  3.49}_{ -3.32}$
& $-11.49^{+  3.38}_{ -3.05}$
\\
${\gamma^1_X/\gamma^1_V-1}$
& $ -0.35^{+  0.08}_{ -0.10}$
& $ -0.02^{+  0.03}_{ -0.03}$
& \ldots
& $ -0.24^{+  0.02}_{ -0.03}$
& $ -0.60^{+  0.06}_{ -0.08}$
\\
\hline
\end{tabular}
\caption{68\% credible intervals for the global fit parameters of the Two Color Parameter Model~I in \S\ref{modelI:sec}.\label{global1:tab}}
\end{table}


\begin{table}
\centering
\begin{tabular}{|c|c|c|c|c|c|}
\hline
Parameters& $X={\hat{U}}$ &${\hat{B}}$&${\hat{V}}$&${\hat{R}}$&${\hat{I}}$\\ \hline
$\alpha_X$
& $0.0048^{+0.0010}_{-0.0011}$
& $0.0019^{+0.0008}_{-0.0009}$
& $0.0018^{+0.0007}_{-0.0007}$
& $0.0017^{+0.0005}_{-0.0005}$
& $0.0027^{+0.0004}_{-0.0004}$
\\
${\alpha_X}/\alpha_{\hat{V}-1}$
& $   1.7^{+   0.8}_{  -0.4}$
& $   0.1^{+   0.1}_{  -0.1}$
& \ldots
& $  -0.0^{+   0.1}_{  -0.1}$
& $   0.5^{+   0.6}_{  -0.3}$
\\
$\beta_X$
& $ 0.035^{+ 0.002}_{-0.003}$
& $ 0.026^{+ 0.002}_{-0.002}$
& $ 0.027^{+ 0.002}_{-0.002}$
& $ 0.022^{+ 0.001}_{-0.001}$
& $ 0.021^{+ 0.001}_{-0.001}$
\\
${\beta_X}/\beta_{\hat{V}-1}$
& $  0.27^{+  0.04}_{ -0.04}$
& $ -0.05^{+  0.02}_{ -0.02}$
&  \ldots
& $ -0.19^{+  0.01}_{ -0.01}$
& $ -0.22^{+  0.03}_{ -0.02}$
\\
$\eta_X$
& $-0.0005^{+0.0008}_{-0.0008}$
& $-0.0003^{+0.0007}_{-0.0007}$
& $0.0006^{+0.0006}_{-0.0006}$
& $0.0006^{+0.0005}_{-0.0005}$
& $0.0000^{+0.0004}_{-0.0004}$
\\
${\eta_X}/\eta_{\hat{V}-1}$
& $ -1.17^{+  1.20}_{ -2.44}$
& $ -1.01^{+  0.76}_{ -1.89}$
&  \ldots
& $ -0.03^{+  0.33}_{ -0.19}$
& $ -0.78^{+  0.51}_{ -0.78}$
\\
$\gamma^0_X$
& $ 71.40^{+  3.29}_{ -3.14}$
& $ 58.39^{+  2.79}_{ -2.62}$
& $ 43.99^{+  2.27}_{ -2.12}$
& $ 33.29^{+  1.87}_{ -1.72}$
& $ 24.81^{+  1.64}_{ -1.53}$
\\
${\gamma^0_X}/\gamma^0_{\hat{V}-1}$
& $  0.62^{+  0.04}_{ -0.04}$
& $  0.33^{+  0.02}_{ -0.02}$
& \ldots
& $ -0.24^{+  0.01}_{ -0.01}$
& $ -0.44^{+  0.02}_{ -0.02}$
\\
$\gamma^1_X$
& $  9.83^{+  5.44}_{ -4.90}$
& $  2.34^{+  4.44}_{ -4.18}$
& $ -4.58^{+  3.40}_{ -3.32}$
& $ -2.41^{+  2.72}_{ -2.63}$
& $ -5.77^{+  2.22}_{ -2.15}$
\\
${\gamma^1_X}/\gamma^1_{\hat{V}-1}$
& $ -2.62^{+  1.37}_{ -4.02}$
& $ -1.25^{+  0.68}_{ -1.99}$
& \ldots
& $ -0.43^{+  0.12}_{ -0.32}$
& $  0.15^{+  0.88}_{ -0.37}$
\\
$\sigma_p \phi_X$
& $-0.007^{+ 0.012}_{-0.012}$
& $-0.039^{+ 0.010}_{-0.010}$
& $-0.039^{+ 0.009}_{-0.008}$
& $-0.026^{+ 0.007}_{-0.007}$
& $ 0.018^{+ 0.007}_{-0.007}$
\\
${\phi_X/\phi_{\hat{V}}-1}$
& $-0.804^{+ 0.245}_{-0.342}$
& $-0.009^{+ 0.095}_{-0.109}$
& \ldots
& $-0.328^{+ 0.047}_{-0.066}$
& $-1.465^{+ 0.220}_{-0.349}$
\\\hline
\end{tabular}
\caption{68\% credible intervals for the global fit parameters of the Three Color Parameter Model~II in \S\ref{modelII:sec}.\label{global2:tab}}
\end{table}
%
%\section{Covariance Matrix for Model~III Global Parameters}
%\added{The following is the  covariance matrix for the global cofactors $\{ \alpha_{\hat{U}}, \beta_{\hat{U}}, \eta_{\hat{U}}, \zeta_{\hat{U}}, \gamma^0_{\hat{U}},\gamma^1_{\hat{U}}, \sigma_p \phi_{\hat{U}}, 
%\ldots,
%\alpha_{\hat{I}}, \beta_{\hat{I}}, \eta_{\hat{I}}, \zeta_{\hat{I}}, \gamma^0_{\hat{I}},\gamma^1_{\hat{I}}, \sigma_p \phi_{\hat{I}}
% \}$.  The matrix 35x35 is big so will have to go in supplemental material.}
%\setcounter{MaxMatrixCols}{40}
%\begin{equation}
%\tiny
%\begin{bmatrix}
%5.5\times 10^{-7} & -1.8\times 10^{-6} & 3.9\times 10^{-7} & -1.6\times 10^{-5} & 2.1\times 10^{-4} & -2.6\times 10^{-4} & 3.6\times 10^{-6} & 4.5\times 10^{-7} & -1.5\times 10^{-6} & 3.1\times 10^{-7} & -1.3\times 10^{-5} & 2.6\times 10^{-4} & -1.3\times 10^{-4} & 2.8\times 10^{-6} & 3.6\times 10^{-7} & -1.2\times 10^{-6} & 2.4\times 10^{-7} & -1.0\times 10^{-5} & 1.9\times 10^{-4} & -5.5\times 10^{-5} & 1.7\times 10^{-6} & 2.9\times 10^{-7} & -9.2\times 10^{-7} & 1.9\times 10^{-7} & -8.1\times 10^{-6} & 1.1\times 10^{-4} & -1.7\times 10^{-5} & 1.1\times 10^{-6} & 2.2\times 10^{-7} & -7.2\times 10^{-7} & 1.4\times 10^{-7} & -6.3\times 10^{-6} & -3.6\times 10^{-5} & -6.7\times 10^{-6} & 4.7\times 10^{-7} \\
%-1.8\times 10^{-6} & 2.7\times 10^{-5} & -2.6\times 10^{-6} & 1.9\times 10^{-4} & -4.0\times 10^{-3} & -3.0\times 10^{-3} & -1.8\times 10^{-5} & -1.4\times 10^{-6} & 2.2\times 10^{-5} & -2.1\times 10^{-6} & 1.5\times 10^{-4} & -3.2\times 10^{-3} & -2.7\times 10^{-3} & -1.4\times 10^{-5} & -1.1\times 10^{-6} & 1.7\times 10^{-5} & -1.6\times 10^{-6} & 1.2\times 10^{-4} & -2.6\times 10^{-3} & -2.1\times 10^{-3} & -1.0\times 10^{-5} & -9.2\times 10^{-7} & 1.4\times 10^{-5} & -1.2\times 10^{-6} & 9.4\times 10^{-5} & -2.0\times 10^{-3} & -1.6\times 10^{-3} & -7.9\times 10^{-6} & -7.3\times 10^{-7} & 1.1\times 10^{-5} & -8.8\times 10^{-7} & 7.6\times 10^{-5} & -1.1\times 10^{-3} & -1.2\times 10^{-3} & -5.2\times 10^{-6} \\
%3.9\times 10^{-7} & -2.6\times 10^{-6} & 1.3\times 10^{-6} & -2.2\times 10^{-5} & 3.9\times 10^{-4} & 1.7\times 10^{-3} & 1.0\times 10^{-5} & 3.2\times 10^{-7} & -2.1\times 10^{-6} & 1.0\times 10^{-6} & -1.8\times 10^{-5} & 5.0\times 10^{-4} & 1.4\times 10^{-3} & 8.3\times 10^{-6} & 2.4\times 10^{-7} & -1.5\times 10^{-6} & 7.8\times 10^{-7} & -1.3\times 10^{-5} & 5.4\times 10^{-4} & 1.0\times 10^{-3} & 6.0\times 10^{-6} & 1.9\times 10^{-7} & -1.1\times 10^{-6} & 6.1\times 10^{-7} & -1.0\times 10^{-5} & 4.1\times 10^{-4} & 8.0\times 10^{-4} & 4.6\times 10^{-6} & 1.4\times 10^{-7} & -8.1\times 10^{-7} & 4.7\times 10^{-7} & -7.4\times 10^{-6} & 1.6\times 10^{-4} & 5.9\times 10^{-4} & 3.1\times 10^{-6} \\
%-1.6\times 10^{-5} & 1.9\times 10^{-4} & -2.2\times 10^{-5} & 1.6\times 10^{-3} & -2.9\times 10^{-2} & -1.4\times 10^{-2} & -2.7\times 10^{-4} & -1.3\times 10^{-5} & 1.5\times 10^{-4} & -1.8\times 10^{-5} & 1.3\times 10^{-3} & -2.8\times 10^{-2} & -1.3\times 10^{-2} & -2.1\times 10^{-4} & -1.0\times 10^{-5} & 1.2\times 10^{-4} & -1.4\times 10^{-5} & 9.9\times 10^{-4} & -2.3\times 10^{-2} & -1.0\times 10^{-2} & -1.5\times 10^{-4} & -8.3\times 10^{-6} & 9.5\times 10^{-5} & -1.1\times 10^{-5} & 7.9\times 10^{-4} & -1.8\times 10^{-2} & -7.7\times 10^{-3} & -1.2\times 10^{-4} & -6.4\times 10^{-6} & 7.7\times 10^{-5} & -7.7\times 10^{-6} & 6.3\times 10^{-4} & -8.1\times 10^{-3} & -4.9\times 10^{-3} & -7.9\times 10^{-5} \\
%2.1\times 10^{-4} & -4.0\times 10^{-3} & 3.9\times 10^{-4} & -2.9\times 10^{-2} & 6.9\times 10^{+0} & 1.3\times 10^{+0} & 6.7\times 10^{-3} & 1.4\times 10^{-4} & -3.0\times 10^{-3} & 3.2\times 10^{-4} & -2.2\times 10^{-2} & 5.5\times 10^{+0} & 1.1\times 10^{+0} & 5.1\times 10^{-3} & 9.6\times 10^{-5} & -2.2\times 10^{-3} & 2.3\times 10^{-4} & -1.6\times 10^{-2} & 4.5\times 10^{+0} & 9.7\times 10^{-1} & 3.7\times 10^{-3} & 6.0\times 10^{-5} & -1.7\times 10^{-3} & 1.7\times 10^{-4} & -1.3\times 10^{-2} & 3.7\times 10^{+0} & 8.5\times 10^{-1} & 3.0\times 10^{-3} & 3.9\times 10^{-5} & -1.4\times 10^{-3} & 1.3\times 10^{-4} & -1.1\times 10^{-2} & 2.9\times 10^{+0} & 6.7\times 10^{-1} & 2.1\times 10^{-3} \\
%-2.6\times 10^{-4} & -3.0\times 10^{-3} & 1.7\times 10^{-3} & -1.4\times 10^{-2} & 1.3\times 10^{+0} & 1.7\times 10^{+1} & 8.7\times 10^{-3} & -1.9\times 10^{-4} & -2.8\times 10^{-3} & 1.3\times 10^{-3} & -1.3\times 10^{-2} & 1.1\times 10^{+0} & 1.4\times 10^{+1} & 7.1\times 10^{-3} & -1.7\times 10^{-4} & -1.5\times 10^{-3} & 9.7\times 10^{-4} & -6.8\times 10^{-3} & 1.8\times 10^{+0} & 1.1\times 10^{+1} & 4.9\times 10^{-3} & -1.2\times 10^{-4} & -1.0\times 10^{-3} & 7.6\times 10^{-4} & -4.4\times 10^{-3} & 1.6\times 10^{+0} & 9.6\times 10^{+0} & 3.2\times 10^{-3} & -10.0\times 10^{-5} & -3.8\times 10^{-4} & 5.6\times 10^{-4} & -7.6\times 10^{-4} & 1.4\times 10^{+0} & 7.9\times 10^{+0} & 1.8\times 10^{-3} \\
%3.6\times 10^{-6} & -1.8\times 10^{-5} & 1.0\times 10^{-5} & -2.7\times 10^{-4} & 6.7\times 10^{-3} & 8.7\times 10^{-3} & 2.8\times 10^{-4} & 2.9\times 10^{-6} & -1.4\times 10^{-5} & 8.4\times 10^{-6} & -2.1\times 10^{-4} & 1.1\times 10^{-2} & 6.8\times 10^{-3} & 2.3\times 10^{-4} & 2.0\times 10^{-6} & -9.4\times 10^{-6} & 6.4\times 10^{-6} & -1.6\times 10^{-4} & 1.1\times 10^{-2} & 4.1\times 10^{-3} & 1.8\times 10^{-4} & 1.4\times 10^{-6} & -7.4\times 10^{-6} & 4.9\times 10^{-6} & -1.2\times 10^{-4} & 8.0\times 10^{-3} & 3.0\times 10^{-3} & 1.4\times 10^{-4} & 9.2\times 10^{-7} & -7.2\times 10^{-6} & 3.6\times 10^{-6} & -9.9\times 10^{-5} & 2.6\times 10^{-3} & 2.3\times 10^{-3} & 1.1\times 10^{-4} \\
%4.5\times 10^{-7} & -1.4\times 10^{-6} & 3.2\times 10^{-7} & -1.3\times 10^{-5} & 1.4\times 10^{-4} & -1.9\times 10^{-4} & 2.9\times 10^{-6} & 4.0\times 10^{-7} & -1.3\times 10^{-6} & 2.7\times 10^{-7} & -1.1\times 10^{-5} & 2.0\times 10^{-4} & -2.5\times 10^{-5} & 2.3\times 10^{-6} & 3.2\times 10^{-7} & -1.0\times 10^{-6} & 2.1\times 10^{-7} & -9.1\times 10^{-6} & 1.4\times 10^{-4} & 4.3\times 10^{-5} & 1.3\times 10^{-6} & 2.6\times 10^{-7} & -8.2\times 10^{-7} & 1.7\times 10^{-7} & -7.1\times 10^{-6} & 7.4\times 10^{-5} & 5.1\times 10^{-5} & 6.9\times 10^{-7} & 1.9\times 10^{-7} & -6.0\times 10^{-7} & 1.2\times 10^{-7} & -5.3\times 10^{-6} & -5.2\times 10^{-5} & 9.5\times 10^{-6} & 2.4\times 10^{-7} \\
%-1.5\times 10^{-6} & 2.2\times 10^{-5} & -2.1\times 10^{-6} & 1.5\times 10^{-4} & -3.0\times 10^{-3} & -2.8\times 10^{-3} & -1.4\times 10^{-5} & -1.3\times 10^{-6} & 1.9\times 10^{-5} & -1.8\times 10^{-6} & 1.3\times 10^{-4} & -2.4\times 10^{-3} & -3.0\times 10^{-3} & -1.2\times 10^{-5} & -1.0\times 10^{-6} & 1.6\times 10^{-5} & -1.4\times 10^{-6} & 1.0\times 10^{-4} & -2.0\times 10^{-3} & -2.6\times 10^{-3} & -7.6\times 10^{-6} & -8.4\times 10^{-7} & 1.2\times 10^{-5} & -1.0\times 10^{-6} & 8.4\times 10^{-5} & -1.5\times 10^{-3} & -2.0\times 10^{-3} & -5.6\times 10^{-6} & -6.2\times 10^{-7} & 9.3\times 10^{-6} & -7.7\times 10^{-7} & 6.3\times 10^{-5} & -7.7\times 10^{-4} & -1.1\times 10^{-3} & -4.0\times 10^{-6} \\
%3.1\times 10^{-7} & -2.1\times 10^{-6} & 1.0\times 10^{-6} & -1.8\times 10^{-5} & 3.2\times 10^{-4} & 1.3\times 10^{-3} & 8.4\times 10^{-6} & 2.7\times 10^{-7} & -1.8\times 10^{-6} & 8.5\times 10^{-7} & -1.5\times 10^{-5} & 4.1\times 10^{-4} & 1.1\times 10^{-3} & 6.7\times 10^{-6} & 2.0\times 10^{-7} & -1.2\times 10^{-6} & 6.7\times 10^{-7} & -1.1\times 10^{-5} & 4.4\times 10^{-4} & 8.2\times 10^{-4} & 4.9\times 10^{-6} & 1.6\times 10^{-7} & -9.4\times 10^{-7} & 5.2\times 10^{-7} & -8.5\times 10^{-6} & 3.4\times 10^{-4} & 6.4\times 10^{-4} & 3.7\times 10^{-6} & 1.1\times 10^{-7} & -6.9\times 10^{-7} & 3.8\times 10^{-7} & -6.2\times 10^{-6} & 1.3\times 10^{-4} & 4.8\times 10^{-4} & 2.6\times 10^{-6} \\
%-1.3\times 10^{-5} & 1.5\times 10^{-4} & -1.8\times 10^{-5} & 1.3\times 10^{-3} & -2.2\times 10^{-2} & -1.3\times 10^{-2} & -2.1\times 10^{-4} & -1.1\times 10^{-5} & 1.3\times 10^{-4} & -1.5\times 10^{-5} & 1.1\times 10^{-3} & -2.2\times 10^{-2} & -1.3\times 10^{-2} & -1.7\times 10^{-4} & -9.0\times 10^{-6} & 1.0\times 10^{-4} & -1.2\times 10^{-5} & 8.7\times 10^{-4} & -1.8\times 10^{-2} & -1.1\times 10^{-2} & -1.2\times 10^{-4} & -7.2\times 10^{-6} & 8.3\times 10^{-5} & -8.9\times 10^{-6} & 6.9\times 10^{-4} & -1.4\times 10^{-2} & -8.3\times 10^{-3} & -9.2\times 10^{-5} & -5.3\times 10^{-6} & 6.2\times 10^{-5} & -6.6\times 10^{-6} & 5.1\times 10^{-4} & -6.0\times 10^{-3} & -4.1\times 10^{-3} & -6.7\times 10^{-5} \\
%2.6\times 10^{-4} & -3.2\times 10^{-3} & 5.0\times 10^{-4} & -2.8\times 10^{-2} & 5.5\times 10^{+0} & 1.1\times 10^{+0} & 1.1\times 10^{-2} & 2.0\times 10^{-4} & -2.4\times 10^{-3} & 4.1\times 10^{-4} & -2.2\times 10^{-2} & 5.1\times 10^{+0} & 9.4\times 10^{-1} & 8.4\times 10^{-3} & 1.4\times 10^{-4} & -1.8\times 10^{-3} & 3.0\times 10^{-4} & -1.6\times 10^{-2} & 4.3\times 10^{+0} & 8.6\times 10^{-1} & 6.0\times 10^{-3} & 9.8\times 10^{-5} & -1.4\times 10^{-3} & 2.3\times 10^{-4} & -1.3\times 10^{-2} & 3.4\times 10^{+0} & 7.8\times 10^{-1} & 4.6\times 10^{-3} & 5.8\times 10^{-5} & -1.1\times 10^{-3} & 1.7\times 10^{-4} & -1.0\times 10^{-2} & 2.4\times 10^{+0} & 6.5\times 10^{-1} & 3.2\times 10^{-3} \\
%-1.3\times 10^{-4} & -2.7\times 10^{-3} & 1.4\times 10^{-3} & -1.3\times 10^{-2} & 1.1\times 10^{+0} & 1.4\times 10^{+1} & 6.8\times 10^{-3} & -2.5\times 10^{-5} & -3.0\times 10^{-3} & 1.1\times 10^{-3} & -1.3\times 10^{-2} & 9.4\times 10^{-1} & 1.4\times 10^{+1} & 6.4\times 10^{-3} & 1.8\times 10^{-5} & -2.2\times 10^{-3} & 8.6\times 10^{-4} & -1.1\times 10^{-2} & 1.5\times 10^{+0} & 1.2\times 10^{+1} & 1.6\times 10^{-3} & 2.0\times 10^{-5} & -1.6\times 10^{-3} & 6.7\times 10^{-4} & -7.7\times 10^{-3} & 1.3\times 10^{+0} & 9.5\times 10^{+0} & 3.9\times 10^{-6} & -2.3\times 10^{-5} & -6.0\times 10^{-4} & 5.1\times 10^{-4} & -3.1\times 10^{-3} & 1.1\times 10^{+0} & 7.1\times 10^{+0} & -1.3\times 10^{-3} \\
%2.8\times 10^{-6} & -1.4\times 10^{-5} & 8.3\times 10^{-6} & -2.1\times 10^{-4} & 5.1\times 10^{-3} & 7.1\times 10^{-3} & 2.3\times 10^{-4} & 2.3\times 10^{-6} & -1.2\times 10^{-5} & 6.7\times 10^{-6} & -1.7\times 10^{-4} & 8.4\times 10^{-3} & 6.4\times 10^{-3} & 1.9\times 10^{-4} & 1.6\times 10^{-6} & -8.8\times 10^{-6} & 5.2\times 10^{-6} & -1.4\times 10^{-4} & 8.6\times 10^{-3} & 4.5\times 10^{-3} & 1.5\times 10^{-4} & 1.1\times 10^{-6} & -7.0\times 10^{-6} & 4.0\times 10^{-6} & -1.1\times 10^{-4} & 6.3\times 10^{-3} & 3.2\times 10^{-3} & 1.2\times 10^{-4} & 6.7\times 10^{-7} & -6.3\times 10^{-6} & 3.0\times 10^{-6} & -8.3\times 10^{-5} & 2.1\times 10^{-3} & 2.0\times 10^{-3} & 8.9\times 10^{-5} \\
%3.6\times 10^{-7} & -1.1\times 10^{-6} & 2.4\times 10^{-7} & -1.0\times 10^{-5} & 9.6\times 10^{-5} & -1.7\times 10^{-4} & 2.0\times 10^{-6} & 3.2\times 10^{-7} & -1.0\times 10^{-6} & 2.0\times 10^{-7} & -9.0\times 10^{-6} & 1.4\times 10^{-4} & 1.8\times 10^{-5} & 1.6\times 10^{-6} & 2.8\times 10^{-7} & -9.3\times 10^{-7} & 1.7\times 10^{-7} & -7.8\times 10^{-6} & 9.4\times 10^{-5} & 1.0\times 10^{-4} & 6.8\times 10^{-7} & 2.3\times 10^{-7} & -7.4\times 10^{-7} & 1.3\times 10^{-7} & -6.2\times 10^{-6} & 3.6\times 10^{-5} & 8.4\times 10^{-5} & 2.4\times 10^{-7} & 1.7\times 10^{-7} & -5.5\times 10^{-7} & 9.8\times 10^{-8} & -4.6\times 10^{-6} & -6.3\times 10^{-5} & 3.2\times 10^{-5} & -5.5\times 10^{-8} \\
%-1.2\times 10^{-6} & 1.7\times 10^{-5} & -1.5\times 10^{-6} & 1.2\times 10^{-4} & -2.2\times 10^{-3} & -1.5\times 10^{-3} & -9.4\times 10^{-6} & -1.0\times 10^{-6} & 1.6\times 10^{-5} & -1.2\times 10^{-6} & 1.0\times 10^{-4} & -1.8\times 10^{-3} & -2.2\times 10^{-3} & -8.8\times 10^{-6} & -9.3\times 10^{-7} & 1.4\times 10^{-5} & -9.7\times 10^{-7} & 9.4\times 10^{-5} & -1.3\times 10^{-3} & -2.2\times 10^{-3} & -5.0\times 10^{-6} & -7.6\times 10^{-7} & 1.1\times 10^{-5} & -7.5\times 10^{-7} & 7.6\times 10^{-5} & -9.6\times 10^{-4} & -1.6\times 10^{-3} & -3.4\times 10^{-6} & -5.7\times 10^{-7} & 8.7\times 10^{-6} & -5.6\times 10^{-7} & 5.8\times 10^{-5} & -4.2\times 10^{-4} & -9.0\times 10^{-4} & -2.8\times 10^{-6} \\
%2.4\times 10^{-7} & -1.6\times 10^{-6} & 7.8\times 10^{-7} & -1.4\times 10^{-5} & 2.3\times 10^{-4} & 9.7\times 10^{-4} & 6.4\times 10^{-6} & 2.1\times 10^{-7} & -1.4\times 10^{-6} & 6.7\times 10^{-7} & -1.2\times 10^{-5} & 3.0\times 10^{-4} & 8.6\times 10^{-4} & 5.2\times 10^{-6} & 1.7\times 10^{-7} & -9.7\times 10^{-7} & 5.5\times 10^{-7} & -8.8\times 10^{-6} & 3.3\times 10^{-4} & 6.6\times 10^{-4} & 3.7\times 10^{-6} & 1.3\times 10^{-7} & -7.4\times 10^{-7} & 4.3\times 10^{-7} & -6.8\times 10^{-6} & 2.5\times 10^{-4} & 5.1\times 10^{-4} & 2.8\times 10^{-6} & 9.4\times 10^{-8} & -5.4\times 10^{-7} & 3.2\times 10^{-7} & -5.0\times 10^{-6} & 9.2\times 10^{-5} & 3.6\times 10^{-4} & 2.0\times 10^{-6} \\
%-1.0\times 10^{-5} & 1.2\times 10^{-4} & -1.3\times 10^{-5} & 9.9\times 10^{-4} & -1.6\times 10^{-2} & -6.8\times 10^{-3} & -1.6\times 10^{-4} & -9.1\times 10^{-6} & 1.0\times 10^{-4} & -1.1\times 10^{-5} & 8.7\times 10^{-4} & -1.6\times 10^{-2} & -1.1\times 10^{-2} & -1.4\times 10^{-4} & -7.8\times 10^{-6} & 9.4\times 10^{-5} & -8.8\times 10^{-6} & 7.8\times 10^{-4} & -1.4\times 10^{-2} & -1.0\times 10^{-2} & -9.2\times 10^{-5} & -6.3\times 10^{-6} & 7.6\times 10^{-5} & -6.8\times 10^{-6} & 6.2\times 10^{-4} & -1.0\times 10^{-2} & -7.0\times 10^{-3} & -6.7\times 10^{-5} & -4.7\times 10^{-6} & 5.8\times 10^{-5} & -5.1\times 10^{-6} & 4.7\times 10^{-4} & -4.0\times 10^{-3} & -3.9\times 10^{-3} & -5.1\times 10^{-5} \\
%1.9\times 10^{-4} & -2.6\times 10^{-3} & 5.4\times 10^{-4} & -2.3\times 10^{-2} & 4.5\times 10^{+0} & 1.8\times 10^{+0} & 1.1\times 10^{-2} & 1.4\times 10^{-4} & -2.0\times 10^{-3} & 4.4\times 10^{-4} & -1.8\times 10^{-2} & 4.3\times 10^{+0} & 1.5\times 10^{+0} & 8.6\times 10^{-3} & 9.4\times 10^{-5} & -1.3\times 10^{-3} & 3.3\times 10^{-4} & -1.4\times 10^{-2} & 4.0\times 10^{+0} & 1.2\times 10^{+0} & 6.5\times 10^{-3} & 5.8\times 10^{-5} & -9.9\times 10^{-4} & 2.5\times 10^{-4} & -1.1\times 10^{-2} & 3.2\times 10^{+0} & 1.0\times 10^{+0} & 5.0\times 10^{-3} & 2.6\times 10^{-5} & -7.8\times 10^{-4} & 1.8\times 10^{-4} & -8.2\times 10^{-3} & 2.2\times 10^{+0} & 8.9\times 10^{-1} & 3.6\times 10^{-3} \\
%-5.5\times 10^{-5} & -2.1\times 10^{-3} & 1.0\times 10^{-3} & -1.0\times 10^{-2} & 9.7\times 10^{-1} & 1.1\times 10^{+1} & 4.1\times 10^{-3} & 4.3\times 10^{-5} & -2.6\times 10^{-3} & 8.2\times 10^{-4} & -1.1\times 10^{-2} & 8.6\times 10^{-1} & 1.2\times 10^{+1} & 4.5\times 10^{-3} & 1.0\times 10^{-4} & -2.2\times 10^{-3} & 6.6\times 10^{-4} & -1.0\times 10^{-2} & 1.2\times 10^{+0} & 1.0\times 10^{+1} & -3.8\times 10^{-4} & 8.4\times 10^{-5} & -1.6\times 10^{-3} & 5.2\times 10^{-4} & -7.3\times 10^{-3} & 1.1\times 10^{+0} & 8.6\times 10^{+0} & -1.6\times 10^{-3} & 2.1\times 10^{-5} & -6.0\times 10^{-4} & 4.0\times 10^{-4} & -3.1\times 10^{-3} & 8.8\times 10^{-1} & 6.4\times 10^{+0} & -2.5\times 10^{-3} \\
%1.7\times 10^{-6} & -1.0\times 10^{-5} & 6.0\times 10^{-6} & -1.5\times 10^{-4} & 3.7\times 10^{-3} & 4.9\times 10^{-3} & 1.8\times 10^{-4} & 1.3\times 10^{-6} & -7.6\times 10^{-6} & 4.9\times 10^{-6} & -1.2\times 10^{-4} & 6.0\times 10^{-3} & 1.6\times 10^{-3} & 1.5\times 10^{-4} & 6.8\times 10^{-7} & -5.0\times 10^{-6} & 3.7\times 10^{-6} & -9.2\times 10^{-5} & 6.5\times 10^{-3} & -3.8\times 10^{-4} & 1.3\times 10^{-4} & 3.9\times 10^{-7} & -4.1\times 10^{-6} & 2.9\times 10^{-6} & -7.2\times 10^{-5} & 4.9\times 10^{-3} & -3.9\times 10^{-4} & 1.1\times 10^{-4} & 2.1\times 10^{-7} & -4.4\times 10^{-6} & 2.1\times 10^{-6} & -5.8\times 10^{-5} & 1.9\times 10^{-3} & 6.8\times 10^{-4} & 8.3\times 10^{-5} \\
%2.9\times 10^{-7} & -9.2\times 10^{-7} & 1.9\times 10^{-7} & -8.3\times 10^{-6} & 6.0\times 10^{-5} & -1.2\times 10^{-4} & 1.4\times 10^{-6} & 2.6\times 10^{-7} & -8.4\times 10^{-7} & 1.6\times 10^{-7} & -7.2\times 10^{-6} & 9.8\times 10^{-5} & 2.0\times 10^{-5} & 1.1\times 10^{-6} & 2.3\times 10^{-7} & -7.6\times 10^{-7} & 1.3\times 10^{-7} & -6.3\times 10^{-6} & 5.8\times 10^{-5} & 8.4\times 10^{-5} & 3.9\times 10^{-7} & 1.9\times 10^{-7} & -6.2\times 10^{-7} & 1.0\times 10^{-7} & -5.0\times 10^{-6} & 1.2\times 10^{-5} & 7.2\times 10^{-5} & 4.6\times 10^{-8} & 1.4\times 10^{-7} & -4.6\times 10^{-7} & 8.0\times 10^{-8} & -3.9\times 10^{-6} & -6.4\times 10^{-5} & 3.0\times 10^{-5} & -1.9\times 10^{-7} \\
%-9.2\times 10^{-7} & 1.4\times 10^{-5} & -1.1\times 10^{-6} & 9.5\times 10^{-5} & -1.7\times 10^{-3} & -1.0\times 10^{-3} & -7.4\times 10^{-6} & -8.2\times 10^{-7} & 1.2\times 10^{-5} & -9.4\times 10^{-7} & 8.3\times 10^{-5} & -1.4\times 10^{-3} & -1.6\times 10^{-3} & -7.0\times 10^{-6} & -7.4\times 10^{-7} & 1.1\times 10^{-5} & -7.4\times 10^{-7} & 7.6\times 10^{-5} & -9.9\times 10^{-4} & -1.6\times 10^{-3} & -4.1\times 10^{-6} & -6.2\times 10^{-7} & 9.5\times 10^{-6} & -5.8\times 10^{-7} & 6.3\times 10^{-5} & -7.1\times 10^{-4} & -1.1\times 10^{-3} & -3.0\times 10^{-6} & -4.8\times 10^{-7} & 7.5\times 10^{-6} & -4.4\times 10^{-7} & 5.0\times 10^{-5} & -3.0\times 10^{-4} & -6.5\times 10^{-4} & -2.4\times 10^{-6} \\
%1.9\times 10^{-7} & -1.2\times 10^{-6} & 6.1\times 10^{-7} & -1.1\times 10^{-5} & 1.7\times 10^{-4} & 7.6\times 10^{-4} & 4.9\times 10^{-6} & 1.7\times 10^{-7} & -1.0\times 10^{-6} & 5.2\times 10^{-7} & -8.9\times 10^{-6} & 2.3\times 10^{-4} & 6.7\times 10^{-4} & 4.0\times 10^{-6} & 1.3\times 10^{-7} & -7.5\times 10^{-7} & 4.3\times 10^{-7} & -6.8\times 10^{-6} & 2.5\times 10^{-4} & 5.2\times 10^{-4} & 2.9\times 10^{-6} & 1.0\times 10^{-7} & -5.8\times 10^{-7} & 3.5\times 10^{-7} & -5.3\times 10^{-6} & 1.9\times 10^{-4} & 4.1\times 10^{-4} & 2.2\times 10^{-6} & 7.7\times 10^{-8} & -4.2\times 10^{-7} & 2.6\times 10^{-7} & -4.0\times 10^{-6} & 7.3\times 10^{-5} & 2.8\times 10^{-4} & 1.6\times 10^{-6} \\
%-8.1\times 10^{-6} & 9.4\times 10^{-5} & -1.0\times 10^{-5} & 7.9\times 10^{-4} & -1.3\times 10^{-2} & -4.4\times 10^{-3} & -1.2\times 10^{-4} & -7.1\times 10^{-6} & 8.4\times 10^{-5} & -8.5\times 10^{-6} & 6.9\times 10^{-4} & -1.3\times 10^{-2} & -7.7\times 10^{-3} & -1.1\times 10^{-4} & -6.2\times 10^{-6} & 7.6\times 10^{-5} & -6.8\times 10^{-6} & 6.2\times 10^{-4} & -1.1\times 10^{-2} & -7.3\times 10^{-3} & -7.2\times 10^{-5} & -5.0\times 10^{-6} & 6.3\times 10^{-5} & -5.3\times 10^{-6} & 5.1\times 10^{-4} & -7.8\times 10^{-3} & -4.8\times 10^{-3} & -5.3\times 10^{-5} & -3.9\times 10^{-6} & 5.0\times 10^{-5} & -4.0\times 10^{-6} & 4.0\times 10^{-4} & -3.2\times 10^{-3} & -2.6\times 10^{-3} & -4.0\times 10^{-5} \\
%1.1\times 10^{-4} & -2.0\times 10^{-3} & 4.1\times 10^{-4} & -1.8\times 10^{-2} & 3.7\times 10^{+0} & 1.6\times 10^{+0} & 8.0\times 10^{-3} & 7.4\times 10^{-5} & -1.5\times 10^{-3} & 3.4\times 10^{-4} & -1.4\times 10^{-2} & 3.4\times 10^{+0} & 1.3\times 10^{+0} & 6.3\times 10^{-3} & 3.6\times 10^{-5} & -9.6\times 10^{-4} & 2.5\times 10^{-4} & -1.0\times 10^{-2} & 3.2\times 10^{+0} & 1.1\times 10^{+0} & 4.9\times 10^{-3} & 1.2\times 10^{-5} & -7.1\times 10^{-4} & 1.9\times 10^{-4} & -7.8\times 10^{-3} & 2.6\times 10^{+0} & 9.3\times 10^{-1} & 3.9\times 10^{-3} & -6.0\times 10^{-6} & -5.7\times 10^{-4} & 1.4\times 10^{-4} & -6.2\times 10^{-3} & 1.9\times 10^{+0} & 8.1\times 10^{-1} & 2.9\times 10^{-3} \\
%-1.7\times 10^{-5} & -1.6\times 10^{-3} & 8.0\times 10^{-4} & -7.7\times 10^{-3} & 8.5\times 10^{-1} & 9.6\times 10^{+0} & 3.0\times 10^{-3} & 5.1\times 10^{-5} & -2.0\times 10^{-3} & 6.4\times 10^{-4} & -8.3\times 10^{-3} & 7.8\times 10^{-1} & 9.5\times 10^{+0} & 3.2\times 10^{-3} & 8.4\times 10^{-5} & -1.6\times 10^{-3} & 5.1\times 10^{-4} & -7.0\times 10^{-3} & 1.0\times 10^{+0} & 8.6\times 10^{+0} & -3.9\times 10^{-4} & 7.2\times 10^{-5} & -1.1\times 10^{-3} & 4.1\times 10^{-4} & -4.8\times 10^{-3} & 9.3\times 10^{-1} & 7.3\times 10^{+0} & -1.4\times 10^{-3} & 2.3\times 10^{-5} & -3.9\times 10^{-4} & 3.1\times 10^{-4} & -1.6\times 10^{-3} & 7.8\times 10^{-1} & 5.7\times 10^{+0} & -2.1\times 10^{-3} \\
%1.1\times 10^{-6} & -7.9\times 10^{-6} & 4.6\times 10^{-6} & -1.2\times 10^{-4} & 3.0\times 10^{-3} & 3.2\times 10^{-3} & 1.4\times 10^{-4} & 6.9\times 10^{-7} & -5.6\times 10^{-6} & 3.7\times 10^{-6} & -9.2\times 10^{-5} & 4.6\times 10^{-3} & 3.9\times 10^{-6} & 1.2\times 10^{-4} & 2.4\times 10^{-7} & -3.4\times 10^{-6} & 2.8\times 10^{-6} & -6.7\times 10^{-5} & 5.0\times 10^{-3} & -1.6\times 10^{-3} & 1.1\times 10^{-4} & 4.6\times 10^{-8} & -3.0\times 10^{-6} & 2.2\times 10^{-6} & -5.3\times 10^{-5} & 3.9\times 10^{-3} & -1.4\times 10^{-3} & 8.9\times 10^{-5} & -3.2\times 10^{-8} & -3.6\times 10^{-6} & 1.6\times 10^{-6} & -4.5\times 10^{-5} & 1.7\times 10^{-3} & -4.8\times 10^{-5} & 7.2\times 10^{-5} \\
%2.2\times 10^{-7} & -7.3\times 10^{-7} & 1.4\times 10^{-7} & -6.4\times 10^{-6} & 3.9\times 10^{-5} & -10.0\times 10^{-5} & 9.2\times 10^{-7} & 1.9\times 10^{-7} & -6.2\times 10^{-7} & 1.1\times 10^{-7} & -5.3\times 10^{-6} & 5.8\times 10^{-5} & -2.3\times 10^{-5} & 6.7\times 10^{-7} & 1.7\times 10^{-7} & -5.7\times 10^{-7} & 9.4\times 10^{-8} & -4.7\times 10^{-6} & 2.6\times 10^{-5} & 2.1\times 10^{-5} & 2.1\times 10^{-7} & 1.4\times 10^{-7} & -4.8\times 10^{-7} & 7.7\times 10^{-8} & -3.9\times 10^{-6} & -6.0\times 10^{-6} & 2.3\times 10^{-5} & -3.2\times 10^{-8} & 1.3\times 10^{-7} & -4.2\times 10^{-7} & 6.6\times 10^{-8} & -3.3\times 10^{-6} & -5.8\times 10^{-5} & 3.4\times 10^{-5} & -2.4\times 10^{-7} \\
%-7.2\times 10^{-7} & 1.1\times 10^{-5} & -8.1\times 10^{-7} & 7.7\times 10^{-5} & -1.4\times 10^{-3} & -3.8\times 10^{-4} & -7.2\times 10^{-6} & -6.0\times 10^{-7} & 9.3\times 10^{-6} & -6.9\times 10^{-7} & 6.2\times 10^{-5} & -1.1\times 10^{-3} & -6.0\times 10^{-4} & -6.3\times 10^{-6} & -5.5\times 10^{-7} & 8.7\times 10^{-6} & -5.4\times 10^{-7} & 5.8\times 10^{-5} & -7.8\times 10^{-4} & -6.0\times 10^{-4} & -4.4\times 10^{-6} & -4.6\times 10^{-7} & 7.5\times 10^{-6} & -4.2\times 10^{-7} & 5.0\times 10^{-5} & -5.7\times 10^{-4} & -3.9\times 10^{-4} & -3.6\times 10^{-6} & -4.2\times 10^{-7} & 7.0\times 10^{-6} & -3.3\times 10^{-7} & 4.6\times 10^{-5} & -2.7\times 10^{-4} & -3.8\times 10^{-4} & -2.8\times 10^{-6} \\
%1.4\times 10^{-7} & -8.8\times 10^{-7} & 4.7\times 10^{-7} & -7.7\times 10^{-6} & 1.3\times 10^{-4} & 5.6\times 10^{-4} & 3.6\times 10^{-6} & 1.2\times 10^{-7} & -7.7\times 10^{-7} & 3.8\times 10^{-7} & -6.6\times 10^{-6} & 1.7\times 10^{-4} & 5.1\times 10^{-4} & 3.0\times 10^{-6} & 9.8\times 10^{-8} & -5.6\times 10^{-7} & 3.2\times 10^{-7} & -5.1\times 10^{-6} & 1.8\times 10^{-4} & 4.0\times 10^{-4} & 2.1\times 10^{-6} & 8.0\times 10^{-8} & -4.4\times 10^{-7} & 2.6\times 10^{-7} & -4.0\times 10^{-6} & 1.4\times 10^{-4} & 3.1\times 10^{-4} & 1.6\times 10^{-6} & 6.6\times 10^{-8} & -3.3\times 10^{-7} & 2.2\times 10^{-7} & -3.2\times 10^{-6} & 5.7\times 10^{-5} & 2.2\times 10^{-4} & 1.2\times 10^{-6} \\
%-6.3\times 10^{-6} & 7.6\times 10^{-5} & -7.4\times 10^{-6} & 6.3\times 10^{-4} & -1.1\times 10^{-2} & -7.6\times 10^{-4} & -9.9\times 10^{-5} & -5.3\times 10^{-6} & 6.3\times 10^{-5} & -6.2\times 10^{-6} & 5.1\times 10^{-4} & -1.0\times 10^{-2} & -3.1\times 10^{-3} & -8.3\times 10^{-5} & -4.6\times 10^{-6} & 5.8\times 10^{-5} & -5.0\times 10^{-6} & 4.7\times 10^{-4} & -8.2\times 10^{-3} & -3.1\times 10^{-3} & -5.8\times 10^{-5} & -3.9\times 10^{-6} & 5.0\times 10^{-5} & -4.0\times 10^{-6} & 4.0\times 10^{-4} & -6.2\times 10^{-3} & -1.6\times 10^{-3} & -4.5\times 10^{-5} & -3.3\times 10^{-6} & 4.6\times 10^{-5} & -3.2\times 10^{-6} & 3.7\times 10^{-4} & -3.0\times 10^{-3} & -1.2\times 10^{-3} & -3.3\times 10^{-5} \\
%-3.6\times 10^{-5} & -1.1\times 10^{-3} & 1.6\times 10^{-4} & -8.1\times 10^{-3} & 2.9\times 10^{+0} & 1.4\times 10^{+0} & 2.6\times 10^{-3} & -5.2\times 10^{-5} & -7.7\times 10^{-4} & 1.3\times 10^{-4} & -6.0\times 10^{-3} & 2.4\times 10^{+0} & 1.1\times 10^{+0} & 2.1\times 10^{-3} & -6.3\times 10^{-5} & -4.2\times 10^{-4} & 9.2\times 10^{-5} & -4.0\times 10^{-3} & 2.2\times 10^{+0} & 8.8\times 10^{-1} & 1.9\times 10^{-3} & -6.4\times 10^{-5} & -3.0\times 10^{-4} & 7.3\times 10^{-5} & -3.2\times 10^{-3} & 1.9\times 10^{+0} & 7.8\times 10^{-1} & 1.7\times 10^{-3} & -5.8\times 10^{-5} & -2.7\times 10^{-4} & 5.7\times 10^{-5} & -3.0\times 10^{-3} & 1.8\times 10^{+0} & 6.9\times 10^{-1} & 1.5\times 10^{-3} \\
%-6.7\times 10^{-6} & -1.2\times 10^{-3} & 5.9\times 10^{-4} & -4.9\times 10^{-3} & 6.7\times 10^{-1} & 7.9\times 10^{+0} & 2.3\times 10^{-3} & 9.5\times 10^{-6} & -1.1\times 10^{-3} & 4.8\times 10^{-4} & -4.1\times 10^{-3} & 6.5\times 10^{-1} & 7.1\times 10^{+0} & 2.0\times 10^{-3} & 3.2\times 10^{-5} & -9.0\times 10^{-4} & 3.6\times 10^{-4} & -3.9\times 10^{-3} & 8.9\times 10^{-1} & 6.4\times 10^{+0} & 6.8\times 10^{-4} & 3.0\times 10^{-5} & -6.5\times 10^{-4} & 2.8\times 10^{-4} & -2.6\times 10^{-3} & 8.1\times 10^{-1} & 5.7\times 10^{+0} & -4.8\times 10^{-5} & 3.4\times 10^{-5} & -3.8\times 10^{-4} & 2.2\times 10^{-4} & -1.2\times 10^{-3} & 6.9\times 10^{-1} & 5.4\times 10^{+0} & -6.4\times 10^{-4} \\
%4.7\times 10^{-7} & -5.2\times 10^{-6} & 3.1\times 10^{-6} & -7.9\times 10^{-5} & 2.1\times 10^{-3} & 1.8\times 10^{-3} & 1.1\times 10^{-4} & 2.4\times 10^{-7} & -4.0\times 10^{-6} & 2.6\times 10^{-6} & -6.7\times 10^{-5} & 3.2\times 10^{-3} & -1.3\times 10^{-3} & 8.9\times 10^{-5} & -5.5\times 10^{-8} & -2.8\times 10^{-6} & 2.0\times 10^{-6} & -5.1\times 10^{-5} & 3.6\times 10^{-3} & -2.5\times 10^{-3} & 8.3\times 10^{-5} & -1.9\times 10^{-7} & -2.4\times 10^{-6} & 1.6\times 10^{-6} & -4.0\times 10^{-5} & 2.9\times 10^{-3} & -2.1\times 10^{-3} & 7.2\times 10^{-5} & -2.4\times 10^{-7} & -2.8\times 10^{-6} & 1.2\times 10^{-6} & -3.3\times 10^{-5} & 1.5\times 10^{-3} & -6.4\times 10^{-4} & 6.6\times 10^{-5} \\
%\end{bmatrix}
%\end{equation}
\bibliographystyle{aasjournal}
\bibliography{alex}
%\input{fix.bbl}
\end{document} 

\end{document} 

\end{document} 

\end{document} 
