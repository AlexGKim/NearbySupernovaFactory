\documentclass[11pt, oneside]{article}   	% use "amsart" instead of "article" for AMSLaTeX format
\usepackage{geometry}                		% See geometry.pdf to learn the layout options. There are lots.
\geometry{letterpaper}                   		% ... or a4paper or a5paper or ... 
%\geometry{landscape}                		% Activate for rotated page geometry
%\usepackage[parfill]{parskip}    		% Activate to begin paragraphs with an empty line rather than an indent
\usepackage{graphicx}				% Use pdf, png, jpg, or eps§ with pdflatex; use eps in DVI mode
								% TeX will automatically convert eps --> pdf in pdflatex		
\usepackage{amssymb}
\usepackage{amsmath}

\begin{document}
The model for the intrinsic absolute magnitudes $U$, $B$, $V$, $R$, and $I$ is that they are drawn from a Normal distribution
\begin{equation*}
\left(
\begin{matrix}
U\\B\\V\\R\\I
\end{matrix}
\right) \sim \mathcal{N}
\left(
\Delta +
\left(
\begin{matrix}
c_0+\alpha_0 EW_{Ca} + \beta_0 EW_{Si} \\
c_1+\alpha_1 EW_{Ca} + \beta_1 EW_{Si}  \\
c_2+\alpha_2 EW_{Ca} + \beta_2 EW_{Si} \\
c_3+\alpha_3 EW_{Ca} + \beta_3 EW_{Si} \\
c_4+\alpha_4 EW_{Ca} + \beta_4 EW_{Si}
\end{matrix}
\right)
,C_{c}
\right).
\end{equation*}
The mean of the normal distribution linearly relates absolute magnitudes with
individual spectral parameters  $EW_{Ca}$, and $EW_{Si}$ via the  global coefficients
 $c_i$, $\alpha_i$, $\beta_i$.  Residual variations in absolute magnitude is described
 by  $C_c$ the intrinsic dispersion of the Normal distribution.  To constrain the problem we
 set $\langle \Delta \rangle=0$.

The observed magnitudes  $U_o$, $B_o$, $V_o$, $R_o$, and $I_o$  and spectral features  $EW_{Ca,o}$, and $EW_{Si,o}$ are modeled
as being drawn from a Normal distribution. 
\begin{equation*}
\left(
\begin{matrix}
U_o\\B_o\\ V_o\\R_o\\I_o\\EW_{Si, o}\\ EW_{Ca, o}
\end{matrix}
\right) \sim \mathcal{N}
\left(
\left(
\begin{matrix}
U +\gamma_0 k \\B +\gamma_1 k \\V+\gamma_2 k\\R+\gamma_3 k\\I+\gamma_4 k\\
EW_{Si}\\ EW_{Ca}
\end{matrix}
\right)
,C
\right).
\end{equation*}
The mean of the normal distribution is the intrinsic magnitudes plus an extrinsic magnitude change described by a parameter for each individual supernova
$k$ and global parameters $\gamma_i$.  
To constrain the problem we set  $\langle k \rangle=0$.  We also set  $\gamma_2=1$ meaning that $k$ can be interpreted as $A_V$.

As I perform an Bayesian analysis some mention of priors is required.  A flat prior is used for all parameters except
for the covariance matrix $C_c$, which is constructed from a correlation matrix with  $\nu=2$  LKJ prior and standard
deviations with
 $\mu=0.1$, $\sigma=2.5$ Cauchy (Lorentz) distribution prior restricted to non-negative values.
The nature of the LKJ prior is rather mysterious; one thing to note is that for $\nu=2$ the correlation prior is broad but
approaches zero at the extremum.  The prior disfavors uniform correlations allowing grey offsets to be attributed to the $\Delta$ parameters.

Figure~\ref{sigma:fig} shows the confidence regions for 
 the overall scale off the grey offsets
 ``$\Delta$ scale''
and of the standard deviations of the $C_c$ diagonal.  This parameter subset is constrained and the  ``$\Delta$ scale'' parameter
is not highly correlated with the others.  The average
of the covariance matrix draws (there must be something better to use than this) is 
\begin{equation}
\begin{pmatrix}
0.0050 & 0.0026 & 0.0017 & 0.0015 & 0.0010 \\
0.0026 & 0.0038 & 0.0035 & 0.0029 & 0.0013 \\
0.0017 & 0.0035 & 0.0052 & 0.0045 & 0.0032 \\
0.0015 & 0.0029 & 0.0045 & 0.0046 & 0.0038 \\
0.0010 & 0.0013 & 0.0032 & 0.0038 & 0.0055
 \end{pmatrix}.
 \end{equation}
 In terms of colors $U-V$, $B-V$, $V-R$, and $V-I$ the covariance is
\begin{equation}
\begin{pmatrix}
 0.0068 & 0.0025 & -0.0005 & -0.0012 \\
0.0025 & 0.0020 & -0.0000 & 0.0003 \\
-0.0005 & -0.0000 & 0.0008 & 0.0012 \\
-0.0012 & 0.0003 & 0.0012 & 0.0042
  \end{pmatrix}.
 \end{equation}
These correlations indicate that intrinsic color parameters are available. 
  
\begin{figure}[htbp] %  figure placement: here, top, bottom, or page
   \centering
   \includegraphics[width=4in]{output2/sigma_corner.pdf} 
   \caption{Confidence regions for the parameter ``$\Delta$ scale'' that describes the overall scale off the $\Delta$'s
and of the standard deviations of the diagonal of $C_c$.}
   \label{sigma:fig}
\end{figure}


The ideogram for the grey offsets $\Delta$ for all supernovae is shown in Figure~\ref{Delta_hist:fig}.  The intrinsic dispersion
is given by the standard deviation of $0.10$ mag.  The distribution has a broad tail; a model assumption of a normal intrinsic dispersion
would have been inappropriate.
\begin{figure}[htbp] %  figure placement: here, top, bottom, or page
   \centering
   \includegraphics[width=4in]{output2/Delta_hist.pdf} 
   \caption{Ideogram for the grey offset $\Delta$ for all supernovae.}
   \label{Delta_hist:fig}
\end{figure}


To illustrate how the linear model reduces the dispersion in observed absolute magnitude, we take the
observed $V_o$, and successive correction of $\gamma_2k$,  $\beta EW_{Si}$, and  $\alpha EW_{Ca}$, 
where the 50\%-ile fit value (of the full model) is used for each parameter.  The histograms over the supernovae for these magnitudes are shown in
Figure~\ref{magresidual:fig}.  The benefit of each  linear correction results in the reduction in dispersion.  The amount
of successive reduction in dispersion is similar for all bands.  This is shown for illustration only: in my attempt to reproduce Gerard's previous
work there were no supernovae reserved for out-of-sample model-comparison tests.  AIC and BIC can be computed.
\begin{figure}[htbp] %  figure placement: here, top, bottom, or page
   \centering
   \includegraphics[width=4in]{output2/magresidual_V.pdf} 
   \caption{Distributions of $V$-band corrected magnitudes.}
   \label{magresidual:fig}
\end{figure}

The best-fit distributions for the parameters $\alpha$ and $\beta$, which describe the linear relation between absolute magnitude
and $EW_{Ca}$ and $EW_{Si}$ respectively, are shown in Figure~\ref{alphabeta:fig}.  The 1-$\sigma$ 
credible intervals are $\alpha_0=3.0e-3 \pm 8.6e-4$,
$\alpha_1 = 4.7e-4  \pm 7.2e-4$, $\alpha_2 = 5.3e-4  \pm 5.9e-4$, $\alpha_3 = 7.0e-4  \pm 5.0e-4$,
$\alpha_4 = 2.1e-3  \pm 4.4e-4$,  $\beta_0=0.03  \pm 2.8e-3$,
$\beta_1 =0.03  \pm 2.4e-3$, $\beta_2 = 0.03  \pm 2.0e-3$, $\beta_3 =0.02  \pm 1.8e-3$,
$\beta_4 =0.02  \pm 1.6e-3$.
\begin{figure}[htbp] %  figure placement: here, top, bottom, or page
   \centering
   \includegraphics[width=2.9in]{output2/alpha_corner.pdf} 
   \includegraphics[width=2.9in]{output2/beta_corner.pdf} 
   \caption{Distributions for $\alpha$ and $\beta$.}
   \label{alphabeta:fig}
\end{figure}

The best-fit distribution for the parameters $\gamma$  and a comparison with the predictions for extrinsic
magnitude changes being to CCM dust, are shown in Figure~\ref{gamma:fig}.  The 1-$\sigma$ 
credible intervals are $\gamma_0=1.65 \pm 0.05$,
$\gamma_1 = 1.33  \pm 0.03$, $\gamma_2 = 0.78  \pm 0.1$, $\gamma_3 = 0.57  \pm 0.3$.
The CCM dust predictions are based 
on the central wavelengths of the synthetic broad-band filters, and are expected to be shifted slightly.
By eye the model fit results appear consistent with $R_V=3.1$, although $R_V=2.8$ appears to match better.
\begin{figure}[htbp] %  figure placement: here, top, bottom, or page
   \centering
   \includegraphics[width=2.9in]{output2/gamma_corner.pdf} 
   \includegraphics[width=2.9in]{output2/ccm.pdf} 
   \caption{Best-fit distribution for the parameters $\gamma$ and $\beta$, and a comparison with the predictions for extrinsic
magnitude changes being to CCM dust.}
   \label{gamma:fig}
\end{figure}

\begin{equation*}
\left(
\begin{matrix}
U\\B\\V\\R\\I
\end{matrix}
\right) \sim \mathcal{N}
\left(
\Delta +
\left(
\begin{matrix}
c_0+\alpha_0 EW_{Ca} + \beta_0 EW_{Si} +\rho_{00} R +\rho_{10} R^2\\
c_1+\alpha_1 EW_{Ca} + \beta_1 EW_{Si} +\rho_{01} R  +\rho_{11} R^2\\
c_2+\alpha_2 EW_{Ca} + \beta_2 EW_{Si} +\rho_{02} R +\rho_{12} R^2\\
c_3+\alpha_3 EW_{Ca} + \beta_3 EW_{Si} +\rho_{03} R +\rho_{13} R^2\\
c_4+\alpha_4 EW_{Ca} + \beta_4 EW_{Si}+\rho_{04} R +\rho_{14} R^2
\end{matrix}
\right)
,C_{c}
\right).
\end{equation*}
\begin{equation*}
R \sim \mathcal{N}(\bar{R},1)
\end{equation*}
\begin{equation*}
\left(
\begin{matrix}
U_o\\B_o\\ V_o\\R_o\\I_o\\EW_{Si, o}\\ EW_{Ca, o}
\end{matrix}
\right) \sim \mathcal{N}
\left(
\left(
\begin{matrix}
U +\gamma_0 k \\B +\gamma_1 k \\V+\gamma_2 k\\R+\gamma_3 k\\I+\gamma_4 k\\
EW_{Si}\\ EW_{Ca}
\end{matrix}
\right)
,C
\right).
\end{equation*}

\begin{equation*}
\begin{pmatrix}
0.0060 & 0.0024 & -0.0003 & -0.0008 \\
0.0024 & 0.0017 & -0.0001 & 0.0001 \\
-0.0003 & -0.0001 & 0.0006 & 0.0007 \\
-0.0008 & 0.0001 & 0.0007 & 0.0022
  \end{pmatrix}.
 \end{equation*}
%
%
%A model for a single object is that its intrinsic properties (captured by equivalent width) are drawn from a Normal distribution
%\begin{equation}
%\left(
%\begin{matrix}
%EW_{Si}\\ EW_{Ca}
%\end{matrix}
%\right) \sim \mathcal{N}
%\left(
%\left(
%\begin{matrix}
%EW_{0, Si}\\ EW_{0, Ca}
%\end{matrix}
%\right)
%,C_{EW}
%\right),
%\end{equation}
%with global parameters $EW_{0, Si}$, $EW_{0, Ca}$, and the independent elements of $C_{EW}$.
%The colors, which depend on the intrinsic properties and the extrinsic effect of dust, are modeled as
%\begin{equation}
%\left(
%\begin{matrix}
%U-V\\B-V\\V-R\\V-I
%\end{matrix}
%\right) \sim \mathcal{N}
%\left(
%\left(
%\begin{matrix}
%c_0+\alpha_0 EW_{Si} + \beta_0 EW_{Ca} +\gamma_0 k+ E(U-V) \\
%c_1+\alpha_1 EW_{Si} + \beta_1 EW_{Ca}  +k +E(B-V) \\
%c_2+\alpha_2 EW_{Si} + \beta_2 EW_{Ca} +\gamma_2 k + E(V-R)\\
%c_3+\alpha_3 EW_{Si} + \beta_3 EW_{Ca} +\gamma_3 k +E(V-I)\\
%\end{matrix}
%\right)
%,C_c
%\right),
%\end{equation}
%where there are global parameters $\alpha_X$, $\beta_X$, $c_X$,  and the independent elements of $C_c$,
%while each supernova has independent parameters $E(X-V)$ and intrinsic color parameter $k$.
%
%The measurements are drawn from
%\begin{equation}
%\left(
%\begin{matrix}
%U-V\\B-V\\V-R\\V-I\\EW_{Si}\\ EW_{Ca}
%\end{matrix}
%\right)_o \sim \mathcal{N}
%\left(
%\left(
%\begin{matrix}
%U-V\\B-V\\V-R\\V-I\\EW_{Si}\\ EW_{Ca}
%\end{matrix}
%\right)
%,C_o
%\right),
%\end{equation}
%where $C_o$ is the observation covariance matrix.
%
%Given enough supernovae to constrain the global parameters, the observed colors of each supernova are used to determine its
%color excess parameters.
\end{document} 