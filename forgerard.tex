\documentclass[11pt, oneside]{article}   	% use "amsart" instead of "article" for AMSLaTeX format
\usepackage{geometry}                		% See geometry.pdf to learn the layout options. There are lots.
\geometry{letterpaper}                   		% ... or a4paper or a5paper or ... 
%\geometry{landscape}                		% Activate for rotated page geometry
%\usepackage[parfill]{parskip}    		% Activate to begin paragraphs with an empty line rather than an indent
\usepackage{graphicx}				% Use pdf, png, jpg, or eps§ with pdflatex; use eps in DVI mode
								% TeX will automatically convert eps --> pdf in pdflatex		
\usepackage{amssymb}
\usepackage{amsmath}

\begin{document}
\section{Data}
The data set consists of synthetic $UBVRI$ (whether SNF or Nico system yet to be elucidated)
peak brightnesses and Si and Ca equivalent widths at peak
of 172 SNf supernovae.  The photometry is normalized to place all supernovae at a common distance, as
inferred by redshift.
The data was provided by Manu Gangler.

The SiII$\lambda6355$ velocity within 3-days of peak brightness is taken from the file phrenology\_2016\_12\_01\_CABALLOv1.pkl.
We use the weighted sum for supernovae with multiple velocity measurements.  A subset of 163
supernovae have this velocity measurement.

\section{Model Fits}

We hypothesize that Type Ia supernova spectral features are predictive of intrinsic
magnitudes.  We use feature statistics that are insensitive to broad-band reddening.
The intrinsic magnitudes are not perfectly deterministic, but have some intrinsic dispersion.
An extrinsic physical process common to all supernovae can influence the propagation
of light along the line of sight to influence apparent magnitudes.
We allow for additional color variation left unaccounted for by the spectral features and external extinction.

\subsection{Setup}
We assume 
that  intrinsic $UBVRI$ magnitudes are linearly dependent
on the equivalent widths of the Ca and Si spectral features
$EW_{Ca}$, $EW_{Si}$,
the SiII$\lambda6355$ velocity $v_{Si}$, and an intrinsic color-variation parameter $D$.
Residual Gaussian dispersion is described by covariance $C_c$.  A grey magnitude offset $\Delta$ is included for each supernova
to capture grey intrinsic dispersion and distance errors introduced in the normalization of the photometry.
The observed magnitudes are linearly dependent on an
extrinsic color-variation parameter $k$  parameter.  The observables
$U_o, B_o, V_o, R_o, I_o$, $EW_{Ca,o}$, $EW_{Si,o}$, $v_{Si,o}$ have Gaussian uncertainty with covariance $C$.  The model is written as
\begin{equation}
\left(
\begin{matrix}
U\\B\\V\\R\\I
\end{matrix}
\right) \sim \mathcal{N}
\left(
\Delta +
\left(
\begin{matrix}
c_0+\alpha_0 EW_{Ca} + \beta_0 EW_{Si} + \eta_0 v_{Si} + \delta_0 D\\
c_1+\alpha_1 EW_{Ca} + \beta_1 EW_{Si} + \eta_1 v_{Si} + \delta_1 D \\
c_2+\alpha_2 EW_{Ca} + \beta_2 EW_{Si} + \eta_2 v_{Si} + \delta_2 D\\
c_3+\alpha_3 EW_{Ca} + \beta_3 EW_{Si} + \eta_3 v_{Si} + \delta_3 D\\
c_4+\alpha_4 EW_{Ca} + \beta_4 EW_{Si}+ \eta_4 v_{Si} + \delta_4 D
\end{matrix}
\right)
,C_{c}
\right)
\label{ewsiv:eqn}
\end{equation}
\begin{equation}
\left(
\begin{matrix}
U_o\\B_o\\ V_o\\R_o\\I_o\\EW_{Si, o}\\ EW_{Ca, o} \\ v_{Si, o}\end{matrix}
\right) \sim \mathcal{N}
\left(
\left(
\begin{matrix}
U +\gamma_0 k \\B +\gamma_1 k \\V+\gamma_2 k\\R+\gamma_3 k\\I+\gamma_4 k\\
EW_{Si}\\ EW_{Ca} \\ v_{Si}
\end{matrix}
\right)
,C
\right).
\label{dust:eqn}
\end{equation}
The global parameters $c$, $\alpha$, $\beta$, $\eta$, and $\delta$  are the intercepts and slopes of the bilinear relationships between the spectral
features and intrinsic color with intrinsic magnitudes.  The extrinsic piece is split into the properties $\gamma$ of the physical process and the per-supernova 
parameters $k$.  To break the degeneracies inherent in the model we impose
\begin{equation*}
\langle \Delta \rangle=0, \langle k \rangle=0, \langle D \rangle=0, \gamma_0 \ge 0, \delta_0\le0.
\end{equation*}

Having the intrinsic dispersion $C_c$ as fit parameters seemingly introduces degeneracy in the model, as magnitude and color variation
ascribed to $\Delta$, $\delta D$, and $\gamma k$ could also be attributed to intrinsic dispersion.  There are several features of the model
that drives the assignation of variations away from $C_c$:  The probability distribution functions decrease
with increasing $\det{C_c}$ so growing $C_c$ is disfavored; The distributions of $\Delta$ and $k$ are turn out to
be non-Gaussian, and so are not well described by a Normal covariance; The Bayesian prior selected for $C_c$ disfavors
grey offsets.

If interpreted as being due to dust, the extrinsic term could be associated with extinction $A_X = \gamma_X k$.  In
this parlance the color excess
is $E(B-V) = (\gamma_1-\gamma_2) k$ and $R_X = \frac{\gamma_X}{\gamma_1-\gamma_2}$.
Similarly we introduce for the intrinsic color term $E_\delta(B-V) = (\delta_1-\delta_2) D$ and $R_{\delta X} = \frac{\delta_X}{\delta_1-\delta_2}$.

As I perform an Bayesian analysis some mention of priors is required.  A flat prior is used for all parameters except
for the covariance matrix $C_c$, which is constructed from a correlation matrix with  $\nu=4$  LKJ prior and standard
deviations with
 $\mu=0.1 $, $\sigma=0.1$ Cauchy (Lorentz) distribution prior restricted to non-negative values.
The nature of the LKJ prior is rather mysterious; one thing to note is that for $\nu=4$ the correlation prior is broad but
approaches zero at the extremum.

\subsubsection{Results}
STAN is used to determine the parameter posteriors.  Visual inspection of the posterior and the $\hat{R}$ and $N_{eff}$ statistics
show no signs for non-convergence or sparse sampling.
The $\hat{R}$ statistic is equal to $1.00$ for almost all the parameters, the largest value is $1.03$ for one of the parameters describing the covariance matrix $C_c$;  $N_{eff} \gg 100$ for all parameters, as recommended.

The 68\% intervals for the global parameters $\alpha$, $\beta$, $\eta$, $R$, $R_\delta$, and $\sigma = \sqrt{C_{c,ii}}$
for each of the five filters are given in Table~\ref{global:tab}.
The two dimensional contours for the parameters, grouped by filter, are shown in Figure~\ref{global:fig}.  The Table does not show
the results for $c$ as their values are not significant for the aims of this analysis, the Figure does show $c$ to indicate the convergence
and correlations of these parameters in the fit.

\begin{table}
\centering
\begin{tabular}{|c|c|c|c|c|c|}
\hline
& $X=0$ &1&2&3&4\\ \hline
$\alpha_{X}$
&
$0.0042^{0.0009}_{-0.0009}$
&
$0.0014^{0.0007}_{-0.0008}$
&
$0.0014^{0.0006}_{-0.0006}$
&
$0.0014^{0.0005}_{-0.0005}$
&
$0.0025^{0.0005}_{-0.0005}$
\\
$\alpha_X/\alpha_2-1$
&
$   2.0^{   1.3}_{  -0.6}$
&
$   0.0^{   0.1}_{  -0.2}$
&
$   0.0^{   0.0}_{   0.0}$
&
$   0.0^{   0.2}_{  -0.1}$
&
$   0.8^{   0.9}_{  -0.4}$
\\
$\beta_{X}$
&
$ 0.033^{ 0.003}_{-0.003}$
&
$ 0.027^{ 0.003}_{-0.003}$
&
$ 0.027^{ 0.002}_{-0.002}$
&
$ 0.021^{ 0.002}_{-0.002}$
&
$ 0.020^{ 0.002}_{-0.002}$
\\
$\beta_X/\beta_2-1$
&
$  0.26^{  0.05}_{ -0.05}$
&
$  0.00^{  0.03}_{ -0.03}$
&
$  0.00^{  0.00}_{  0.00}$
&
$ -0.19^{  0.01}_{ -0.01}$
&
$ -0.25^{  0.03}_{ -0.03}$
\\
$\eta_{X}$
&
$0.0010^{0.0012}_{-0.0013}$
&
$0.0006^{0.0010}_{-0.0011}$
&
$0.0008^{0.0008}_{-0.0009}$
&
$0.0008^{0.0007}_{-0.0008}$
&
$0.0002^{0.0006}_{-0.0007}$
\\
$\eta_X/\eta_2-1$
&
$  0.26^{  0.65}_{ -0.65}$
&
$ -0.09^{  0.53}_{ -0.55}$
&
$  0.00^{  0.00}_{  0.00}$
&
$ -0.08^{  0.24}_{ -0.21}$
&
$ -0.57^{  0.51}_{ -0.59}$
\\
$R_{X}$
&
$  4.87^{  0.48}_{ -0.43}$
&
$  3.96^{  0.40}_{ -0.35}$
&
$  2.96^{  0.31}_{ -0.28}$
&
$  2.29^{  0.26}_{ -0.23}$
&
$  1.64^{  0.21}_{ -0.19}$
\\
$R_{\delta X}$
&
$ -2.70^{  1.32}_{ -2.44}$
&
$ -3.46^{  1.29}_{ -2.65}$
&
$ -4.46^{  1.43}_{ -3.11}$
&
$ -4.20^{  1.32}_{ -2.89}$
&
$ -3.80^{  1.20}_{ -2.61}$
\\
$\sigma_{X}$
&
$ 0.059^{ 0.012}_{-0.011}$
&
$ 0.031^{ 0.007}_{-0.007}$
&
$ 0.020^{ 0.004}_{-0.006}$
&
$ 0.013^{ 0.007}_{-0.008}$
&
$ 0.043^{ 0.005}_{-0.004}$
\\
\hline
\end{tabular}
\caption{68\% Intervals for Global Fit Parameters  \label{global:tab}}
\end{table}

\begin{figure}[htbp] %  figure placement: here, top, bottom, or page
   \centering
   \includegraphics[width=2.5in]{output11/coeff0.pdf} 
   \includegraphics[width=2.5in]{output11/coeff1.pdf} 
   \includegraphics[width=2.5in]{output11/coeff2.pdf} 
      \includegraphics[width=2.5in]{output11/coeff3.pdf} 
         \includegraphics[width=2.5in]{output11/coeff4.pdf} 
            \caption{Distributions for $c$, $\alpha$, $\beta$, $\eta$, $\gamma$, $\delta$, and $\sigma$ in each of the 5 bands.   \label{global:fig}}

\end{figure}

We find a small but significant non-zero values for $\alpha_0$ and $\alpha_4$, indicating that $EW_{Ca}$ is an indicator of $U$ and $I$
magnitudes.  All bands show larger and significant non-zero values for $\beta$.  This confirms the part of our hypothesis that spectral indicators
are tracers of supernova absolute magnitude.  On the other hand, the values of $\eta$ are consistent with zero within one sigma.
The effect in color can be seen in $\alpha_X/\alpha_V-1$,  $\beta_X/\beta_V-1$, and  $\eta_X/\eta_V-1$.
Both $EW_{Ca}$ and $EW_{Ca}$ are associated with color changes though not in $B-V$ specifically, while
$v_{Si}$ does not have a significant affect on colors.

The best-fit distributions for $R_X=\frac{\gamma_X}{\gamma_1-\gamma_2}$ are shown in Figure~\ref{rx:fig}.  Figure~\ref{ccm:fig}
shows the expectation of CCM dust, plotted together with our measurements of
$R_X$ and $\frac{\gamma_X}{\gamma_2}$ placed at the central wavelength of the synthetic bands.
the $R_X$ are consistent with the  $R_V=3.1$ CCM model (assuming we are using
SNf synthetic magnitudes).  The uncertainties in the 5 measurements
of $R_X$ are correlated: measurements of $\gamma_X/\gamma_2$ have significantly smaller
uncertainties and are also consistent with $R_V=3.1$.

\begin{figure}[htbp] %  figure placement: here, top, bottom, or page
   \centering
   \includegraphics[width=2.8in]{output11/rx_corner.pdf}
      \includegraphics[width=2.8in]{output11/rxdelta_corner.pdf} 
   \caption{Best-fit distribution for (left)  $R_X=\frac{\gamma_X}{\gamma_1-\gamma_2}$ and (right)  $R_{\delta X}=\frac{\delta_X}{\delta_1-\delta_2}$.
   \label{rx:fig}}
\end{figure}

\begin{figure}[htbp] %  figure placement: here, top, bottom, or page
   \centering
   \includegraphics[width=2.8in]{output11/ccm.pdf}
      \includegraphics[width=2.8in]{output11/ccm2.pdf} 
   \caption{68\% measurements of $R_X=\frac{\gamma_X}{\gamma_1-\gamma_2}$ (left) and $\frac{\gamma_X}{\gamma_2}$ (right).  CCM
   predictions for these parameters are overlaid for different values of $R_V$.
   \label{ccm:fig}}
\end{figure}

The best-fit distributions for $R_{\delta X}=\frac{\delta_X}{\delta_1-\delta_2}$ are shown in Figure~\ref{rx:fig}.
Although the posteriors of $\delta$ are fairly circular, $\delta_1-\delta_2$ can get close to zero, resulting in the
elongated tail in negative  $R_{\delta X}$.
All values are inconsistent with zero and negative: a color excess in $B-V$ produces a brightening of the supernova.
The smallest (most negative) value is $R_{\delta V}$  with  values increasing in both bluer and redder directions; a consequence
of which is that
a reddening in $B-V$ comes with a blueing in $V-R$.
These are not the behaviors expected from normal dust or a blackbody in the Raleigh-Jeans tail, and so may be indicative of
more complicated underlying supernova physics.

The distribution of color excess $E(B-V) =(\gamma_1-\gamma_2)k$ and
 $E_\delta(B-V) = (\delta_1-\delta_2)D$ are shown in Figure~\ref{ebv:fig}.
Recall that given our boundary conditions the means of both distributions are equal to zero. 
The standard deviation of $E(B-V)$ is 0.078 mag while that of $E_\delta(B-V)$ is 0.019 mag.
The range of $B-V$ colors is almost 4 times larger for the external contribution:
extreme colors can thus be attributed to the external source. 
Both distributions are non-Gaussian extended tails in the red.
\begin{figure}[htbp] %  figure placement: here, top, bottom, or page
   \centering
   \includegraphics[width=3.2in]{output11/ebv.pdf}
   \caption{Histograms of $E(B-V) =(\gamma_1-\gamma_2)k$ and
   $E_\delta(B-V) = (\delta_1-\delta_2)D$.
   \label{ebv:fig}}
\end{figure}

The two separate color effects are conflated into a combined color excess with a single effective extinction behavior
\begin{equation}
R_{V,eff}  = \frac{\gamma_1 k + \delta_1 D}{(\gamma_1-\gamma_2) k + (\delta_1-\delta_2)R}.
\end{equation}
The distribution for $R_{V,eff}$ is shown in Figure~\ref{rveff:fig}
\begin{figure}[htbp] %  figure placement: here, top, bottom, or page
   \centering
   \includegraphics[width=3.2in]{output11/rveff.pdf}
   \caption{The effective dust law of the two color terms combined $R_{V,eff}  = \frac{\gamma_1 k + \delta_1 D}{(\gamma_1-\gamma_2) k + (\delta_1-\delta_2)D}$,
   \label{rveff:fig}}
\end{figure}
and has mean and standard deviation  $R_{V,eff}  = 1.52 \pm 0.24$.  
Rerunning the posterior calculation with initial conditions for $\gamma$  near $R_{X,eff}$ reproduced the original results, indicating that
the effective and true $R$ values are not separated by local posterior extrema.

Non-trivial residual magnitude dispersions are captured in $C_c$.   Figure~\ref{sigma:fig} shows the confidence regions for the
square root of the diagonal elements of $C_c$.  The residual intrinsic dispersion ranges from 0.01 to 0.06 mag, significantly smaller
than the dispersion in $\Delta$.

%A representative example is given by the matrix comprised by the average of each element in the Monte Carlo chain.
%For $UBVRI$ the matrix is
%\begin{equation}
%\begin{pmatrix}
%0.0050 & 0.0026 & 0.0017 & 0.0014 & 0.0009 \\
%0.0026 & 0.0034 & 0.0032 & 0.0026 & 0.0011 \\
%0.0017 & 0.0032 & 0.0047 & 0.0041 & 0.0029 \\
%0.0014 & 0.0026 & 0.0041 & 0.0042 & 0.0035 \\
%0.0009 & 0.0011 & 0.0029 & 0.0035 & 0.0050
% \end{pmatrix}.
% \end{equation}
% In terms of colors $U-V$, $B-V$, $V-R$, and $V-I$ the covariance is
%\begin{equation}
%\begin{pmatrix}
%0.0064 & 0.0025 & -0.0004 & -0.0010 \\
%0.0025 & 0.0018 & -0.0000 & 0.0003 \\
%-0.0004 & -0.0000 & 0.0007 & 0.0011 \\
%-0.0010 & 0.0003 & 0.0011 & 0.0039
%  \end{pmatrix}.
% \end{equation}
% The $B-V$ and $V-R$ colors have small standard deviations of  0.04 mag and 0.03 mag with no correlation.  The
% residuals not accounted by the
% model are stronger in the $U-V$ and $V-I$ bands.
 
 \begin{figure}[htbp] %  figure placement: here, top, bottom, or page
   \centering
   \includegraphics[width=3.2in]{output11/sigma_corner.pdf} 
   \caption{Best-fit distribution for the parameters $\sigma$, the square root of the diagonal elements of $C_c$.
   \label{sigma:fig}}
\end{figure}

The ideogram for the grey offsets $\Delta$ for all supernovae is shown in Figure~\ref{hist:fig}.  The intrinsic dispersion
is given by the standard deviation of $0.096$ mag.  The distribution is non-Gaussian, with a broad tail. 
\begin{figure}[htbp] %  figure placement: here, top, bottom, or page
   \centering
   \includegraphics[width=3.2in]{output11/Delta_hist.pdf} 
   \caption{Ideogram for the grey offset $\Delta$.
   \label{hist:fig}}
\end{figure}

Each supernova is described by its parameters $EW_{Ca}$, $EW_{Si}$, $v_{Si}$, $E(B-V)$, and $E_\delta(B-V)$, as well as its grey offset
$\Delta$: the distributions (for all Monte Carlo chain links for all supernovae) are shown in Figure~\ref{perobject:fig}.
There is a core distribution within the supernova parameter-space, with around ten objects that live on the outskirts of that
distribution.  Many outliers appear in the red tail of $E(B-V)$, as would be expected in the infrequent selection of supernovae
heavily extinguished by host-galaxy dust; note that $E_\delta(B-V)$ does not have an analogous tail.
Otherwise the distribution in $EW_{Ca}$--$v_{Si}$ space is the view that
shows the most outliers. 

\begin{figure}[htbp] %  figure placement: here, top, bottom, or page
   \centering
   \includegraphics[width=3.2in]{output11/perobject_corner.pdf} 
   \caption{Distributions for the supernova parameters $EW_{Ca}$, $EW_{Si}$, $v_{Si}$, $E(B-V)$, and $E_\delta(B-V)$, as well as its grey offset
$\Delta$.
   \label{perobject:fig}}
\end{figure}

The Pearson correlation coefficients for $\Delta$, $EW_{Ca}$, $EW_{Si}$, $v_{Si}$, $E(B-V)$, and $E_\delta(B-V)$ are shown in the matrix
\begin{equation}
\begin{pmatrix}
1.000 & -0.012 & -0.031 & -0.026 & 0.070 & 0.002 \\
-0.012 & 1.000 & 0.128 & -0.331 & -0.064 & -0.004 \\
-0.031 & 0.128 & 1.000 & -0.182 & -0.155 & -0.024 \\
-0.026 & -0.331 & -0.182 & 1.000 & 0.018 & 0.036 \\
0.070 & -0.064 & -0.155 & 0.018 & 1.000 & 0.062 \\
0.002 & -0.004 & -0.024 & 0.036 & 0.062 & 1.000
\end{pmatrix}.
\end{equation}
Again  $EW_{Ca}$--$v_{Si}$ is singled out, this time as having the strongest (anti-)correlation.  The second strongest is in
$EW_{Si}$--$v_{Si}$, which are statistics that describe the same spectral feature.
External $E(B-V)$ dust most strongly correlated with $EW_{Si}$, while the internal $E_\delta(B-V)$  has no correlation stronger with internal parameters stronger than
0.04.  Although $E(B-V)$ and $E_\delta(B-V)$ are determined simultaneously for each supernova from colors, they have a low correlation coefficient of 0.062.

%
%To illustrate how the linear model reduces the dispersion in observed absolute magnitude, plot the observed colors as a
%function of the observed  $EW_{Ca, o}$, $EW_{Si,o}$ and $B_o-V_o$ in Fig.~\ref{magresidual:fig}.  Overplotted are lines with the
%best-fit slope.  The data in the color magnitude diagram indicate that the slope at bluer colors is different from that at redder colors
%and that our one-color linear model could be improved upon.
%
%\begin{figure}[htbp] %  figure placement: here, top, bottom, or page
%   \centering
%   \includegraphics[width=2.8in]{output7/speccamag.pdf} 
%      \includegraphics[width=2.8in]{output7/specsimag.pdf} 
%   \includegraphics[width=2.8in]{output7/colormag.pdf} 
%   \caption{Observed $UBVRI$ as a function of observed   $EW_{Ca, o}$ (top left), $EW_{Si,o}$ (top right) and $B_o-V_o$ (bottom).
%   Overplotted are lines with the best-fit slope from the model fit.}
%   \label{magresidual:fig}
%\end{figure}
%
%\subsubsection{Notes}
%Some equivalent widths go positive.  Is this an emission?
%
%
%\subsection{One Dust With Si Velocity}
%\label{onecolor_si:sec}
%Sil$\lambda3655$ velocity is spectral indicator that has been identified to correlated with peak brightness.  The analysis is expanded to include
%this line. We use the weighted mean of all spectral measurements within 3 days of peak brightness.  A handful of supernovae do not have a
%spectrum that close to peak.
%\subsubsection{Setup}
%With the inclusion of SiII$\lambda3655$ velocity  $v_{Si}$ the model looks like
%\begin{equation}
%\left(
%\begin{matrix}
%U\\B\\V\\R\\I
%\end{matrix}
%\right) \sim \mathcal{N}
%\left(
%\Delta +
%\left(
%\begin{matrix}
%c_0+\alpha_0 EW_{Ca} + \beta_0 EW_{Si} + \eta_0 v_{Si}\\
%c_1+\alpha_1 EW_{Ca} + \beta_1 EW_{Si} + \eta_1 v_{Si} \\
%c_2+\alpha_2 EW_{Ca} + \beta_2 EW_{Si} + \eta_2 v_{Si}\\
%c_3+\alpha_3 EW_{Ca} + \beta_3 EW_{Si} + \eta_3 v_{Si}\\
%c_4+\alpha_4 EW_{Ca} + \beta_4 EW_{Si}+ \eta_4 v_{Si}
%\end{matrix}
%\right)
%,C_{c}
%\right)
%\label{ewsiv:eqn}
%\end{equation}
%\begin{equation}
%\left(
%\begin{matrix}
%U_o\\B_o\\ V_o\\R_o\\I_o\\EW_{Si, o}\\ EW_{Ca, o} \\ v_{Si, o}\end{matrix}
%\right) \sim \mathcal{N}
%\left(
%\left(
%\begin{matrix}
%U +\gamma_0 k \\B +\gamma_1 k \\V+\gamma_2 k\\R+\gamma_3 k\\I+\gamma_4 k\\
%EW_{Si}\\ EW_{Ca} \\ v_{Si}
%\end{matrix}
%\right)
%,C
%\right).
%\label{dust:eqn}
%\end{equation}
%The global parameters $\eta$ are the slopes of the linear relationships between Si velocity
%and intrinsic magnitudes. 
%
%\subsubsection{Results}
%The 68\% intervals for the global parameters $\alpha$, $\beta$, $\eta$, $R$, and $\sigma = \sqrt{C_{c,ii}}$  are given in Table~\ref{globalvsi:tab}.
%The two dimensional contours for the parameters of each filter are shown in Figure~\ref{globalvsi:fig}.  
%All bands show large and significant non-zero values for $\eta$ with some non-trivial changes to $\beta$ and a lowering
%if $\sigma$.  The values
%of $\alpha$ are unchanged.
%
%\begin{table}
%\centering
%\begin{tabular}{|c|c|c|c|c|c|}
%\hline
%& $X=0$ &1&2&3&4\\ \hline
%$\alpha_X$
%&
%$0.0031^{0.0009}_{-0.0009}$
%&
%$0.0005^{0.0008}_{-0.0007}$
%&
%$0.0006^{0.0006}_{-0.0006}$
%&
%$0.0008^{0.0005}_{-0.0005}$
%&
%$0.0020^{0.0005}_{-0.0004}$
%\\
%$\beta_X$
%&
%$0.0375^{0.0030}_{-0.0030}$
%&
%$0.0303^{0.0026}_{-0.0025}$
%&
%$0.0298^{0.0023}_{-0.0021}$
%&
%$0.0243^{0.0020}_{-0.0019}$
%&
%$0.0224^{0.0017}_{-0.0017}$
%\\
%$\eta_X$
%&
%$0.0032^{0.0011}_{-0.0011}$
%&
%$0.0024^{0.0009}_{-0.0009}$
%&
%$0.0022^{0.0008}_{-0.0008}$
%&
%$0.0019^{0.0007}_{-0.0007}$
%&
%$0.0011^{0.0006}_{-0.0006}$
%\\
%$R_X$
%&
%$5.1004^{0.2929}_{-0.2517}$
%&
%$4.1677^{0.2656}_{-0.2272}$
%&
%$3.1677^{0.2656}_{-0.2272}$
%&
%$2.4828^{0.2327}_{-0.1964}$
%&
%$1.8320^{0.2050}_{-0.1838}$
%\\
%$\sigma_X$
%&
%$0.0537^{0.0162}_{-0.0185}$
%&
%$0.0405^{0.0107}_{-0.0131}$
%&
%$0.0565^{0.0093}_{-0.0109}$
%&
%$0.0538^{0.0084}_{-0.0084}$
%&
%$0.0636^{0.0076}_{-0.0079}$
%\\
%\hline
%\end{tabular}
%\caption{68\% Intervals for Global Fit Parameters for the One-Dust Model with $v_{Si}$  \label{globalvsi:tab}}
%\end{table}
%
%\begin{figure}[htbp] %  figure placement: here, top, bottom, or page
%   \centering
%   \includegraphics[width=2.5in]{output10/coeff0.pdf} 
%   \includegraphics[width=2.5in]{output10/coeff1.pdf} 
%   \includegraphics[width=2.5in]{output10/coeff2.pdf} 
%      \includegraphics[width=2.5in]{output10/coeff3.pdf} 
%         \includegraphics[width=2.5in]{output10/coeff4.pdf} 
%            \caption{Distributions for $c$, $\alpha$, $\beta$, $\eta$, $\gamma$, and $\sigma$ in each of the 5 bands.}
%   \label{globalvsi:fig}
%\end{figure}
%
%
%
%The best-fit distribution for $R_X=\frac{\gamma_X}{\gamma_1-\gamma_2}$ are shown in Figure~\ref{rxsiv:fig}.  Our measurement of
%$R_X$ and $\frac{\gamma_X}{\gamma_2}$  compared with the predictions of CCM dust:
%our fit values for $\gamma$ are consistent with the  $R_V=3.1$ CCM model.
%
%\begin{figure}[htbp] %  figure placement: here, top, bottom, or page
%   \centering
%   \includegraphics[width=3in]{output10/gamma_corner.pdf} 
%   \caption{Best-fit distribution for the   $R_X=\frac{\gamma_X}{\gamma_1-\gamma_2}$.
%   \label{rxsiv:fig}}
%\end{figure}
%
%\begin{figure}[htbp] %  figure placement: here, top, bottom, or page
%   \centering
%   \includegraphics[width=2.8in]{output10/ccm.pdf}
%      \includegraphics[width=2.8in]{output10/ccm2.pdf} 
%   \caption{68\% measurements of $R_X=\frac{\gamma_X}{\gamma_1-\gamma_2}$ (left) and $\frac{\gamma_X}{\gamma_2}$ (right).  CCM
%   predictions for these parameters are overlaid for different values of $R_V$.}
%   \label{ccmsiv:fig}
%\end{figure}
%
% \begin{figure}[htbp] %  figure placement: here, top, bottom, or page
%   \centering
%   \includegraphics[width=2.4in]{output10/sigma_corner.pdf} 
%   \caption{Best-fit distribution for the parameters $\sigma$, the square root of the diagonal elements of $C_c$.}
%   \label{sigmasiv:fig}
%\end{figure}
%
%The ideogram for the grey offsets $\Delta$ for all supernovae is shown in Figure~\ref{histsiv:fig}.  The intrinsic dispersion
%is given by the standard deviation of $0.09$ mag.  The distribution is non-Gaussian, with a broad tail. 
%\begin{figure}[htbp] %  figure placement: here, top, bottom, or page
%   \centering
%   \includegraphics[width=2.8in]{output10/Delta_hist.pdf} 
%   \includegraphics[width=2.8in]{output10/k_hist.pdf} 
%   \caption{Ideograms for Left: the grey offset $\Delta$; Right: the color term $k$ for all supernovae.}
%   \label{histsiv:fig}
%\end{figure}
%
%\subsection{Two Dust}
%
%Motivated by the consistency of the external extinction with CCV $R_V=3.1$ dust, and the hint of the color-magnitude
%diagrams in Figure~\ref{magresidual:fig}  not following single slopes, we develop a new model in which
%each supernova encounters one of two sources of external extinction.
%
%\subsubsection{Setup}
%The model is identical to that of \S\ref{onecolor:sec} except with the addition of an alternative extrinsic source of extinction.
%The intrinsic magnitudes of the supernova are determined by the spectral properties expressed as Eqn.~\ref{ew:eqn}.
%One extrinsic source of extinction  (Dust 1) is modeled as before with Eqn.~\ref{dust:eqn};
%the second source of extinction (Dust 2)
%is modeled similarly but with the introduction of new parameters $\gamma_{2X}$
%\begin{equation}
%\left(
%\begin{matrix}
%U_o\\B_o\\ V_o\\R_o\\I_o\\EW_{Si, o}\\ EW_{Ca, o}\end{matrix}
%\right) \sim \mathcal{N}
%\left(
%\left(
%\begin{matrix}
%U +\gamma_{20} k \\B +\gamma_{21} k \\V+\gamma_{22} k\\R+\gamma_{23} k\\I+\gamma_{24} k\\
%EW_{Si}\\ EW_{Ca}
%\end{matrix}
%\right)
%,C
%\right).
%\label{dust2:eqn}
%\end{equation}
%The probability that a supernova experiences  Dust 1 is given by the free parameter $p_0$,
%the probability of Dust 2 is $1-p_0$.  
%The same $k$ parameters are used to describe supernovae in both extinction sources.  In order to do this we change
%some of the conditions that constrain the degeneracies of the model:
%\begin{equation*}
%\langle \Delta \rangle=0, \gamma_1 = 1+\gamma_2, \gamma_{21}=1+\gamma_{22}.
%\end{equation*}
%These conditions are useful for interpreting the extinction as being due to dust, as $R_X = \gamma_X$,
%the color excess $E(B-V) \equiv (\gamma_1-\gamma_2)k =k$.
%The $c$ parameters  are fit such that both extinction models
%revert to the same intrinsic magnitudes for $k=0$: the $\langle k \rangle=0$ condition is removed. 
%
%To alleviate the degeneracies inherent to mixture models, we restrict the range of $\alpha$, $\beta$, and $\gamma$ (the parameters for Dust 1)
%to be in the neighborhood of the pdf's of \S\ref{onecolor:sec}.  The one-dust model is embedded in the two-dust model when $p_0=1$.
%We therefore do not claim to find the global optimum of the parameter
%pdf, but we can conclude whether the two-dust model is preferred over the one-dust model.
%The intrinsic covariance $C_c$ is taken to be diagonal.
%
%\subsubsection{Results}
%
%The probability of an object experiencing the first dust is   $p_0=0.3334^{0.0532}_{-0.0721}$,
%with the pdf shown in Figure~\ref{prob0:fig}.
%A single dust with properties centered around the one-dust parameter optimum is not only disfavored, but is most likely the
%minority component.  
%\begin{figure}[htbp] %  figure placement: here, top, bottom, or page
%   \centering
%   \includegraphics[width=3.3in]{output9/prob0.pdf}
%   \caption{pdf of $p_0$, the probability that a supernova experiences Dust 1.  \label{prob0:fig}}
%\end{figure}
%
%
%The 68\% intervals for the global parameters $\alpha$, $\beta$, $R$, $R_2$ and $\sigma = \sqrt{C_{c,ii}}$  are given in Table~\ref{global2:tab}.
%The two dimensional contours for the parameters of each filter are shown in Figure~\ref{global2:fig}.  
%\begin{table}
%\centering
%\begin{tabular}{|c|c|c|c|c|c|}
%\hline
%& $X=0$ &1&2&3&4\\ \hline
%$\alpha_{X}$
%&
%$0.0077^{0.0014}_{-0.0018}$
%&
%$0.0040^{0.0011}_{-0.0013}$
%&
%$0.0025^{0.0008}_{-0.0008}$
%&
%$0.0020^{0.0006}_{-0.0007}$
%&
%$0.0027^{0.0006}_{-0.0006}$
%\\
%$\beta_{X}$
%&
%$0.0276^{0.0048}_{-0.0053}$
%&
%$0.0225^{0.0037}_{-0.0042}$
%&
%$0.0249^{0.0025}_{-0.0026}$
%&
%$0.0213^{0.0019}_{-0.0020}$
%&
%$0.0216^{0.0015}_{-0.0016}$
%\\
%$R_{X}$
%&
%$5.0878^{0.4233}_{-0.4082}$
%&
%$4.3163^{0.3615}_{-0.3327}$
%&
%$3.3163^{0.3615}_{-0.3327}$
%&
%$2.6600^{0.2685}_{-0.2691}$
%&
%$1.9355^{0.2604}_{-0.2474}$
%\\
%$R_{2X}$
%&
%$2.2199^{0.2110}_{-0.2301}$
%&
%$1.5919^{0.1801}_{-0.2246}$
%&
%$0.5919^{0.1801}_{-0.2246}$
%&
%$0.2486^{0.1650}_{-0.1788}$
%&
%$-0.1484^{0.1802}_{-0.1472}$
%\\
%$\sigma_{X}$
%&
%$0.0570^{0.0042}_{-0.0040}$
%&
%$0.0103^{0.0059}_{-0.0048}$
%&
%$0.0229^{0.0021}_{-0.0018}$
%&
%$0.0094^{0.0031}_{-0.0039}$
%&
%$0.0418^{0.0031}_{-0.0030}$
%\\
%\hline
%\end{tabular}
%\caption{68\% Intervals for Global Fit Parameters for the Two-Dust Model \label{global2:tab}}
%\end{table}
%
%\begin{figure}[htbp] %  figure placement: here, top, bottom, or page
%   \centering
%   \includegraphics[width=2.5in]{output9/coeff0.pdf} 
%   \includegraphics[width=2.5in]{output9/coeff1.pdf} 
%   \includegraphics[width=2.5in]{output9/coeff2.pdf} 
%      \includegraphics[width=2.5in]{output9/coeff3.pdf} 
%         \includegraphics[width=2.5in]{output9/coeff4.pdf} 
%            \caption{Distributions for $c$, $\alpha$, $\beta$, $\gamma$, $\gamma_2$, and $\sigma$ in each of the 5 bands.}
%   \label{global2:fig}
%\end{figure}
%
%Our determinations of $R$ and $R_2$ are compared to the expectations of CCM dust in Figure~\ref{ccm2:fig}.
%Within the context of this model there are two distinct types of dust: Dust 1 behaves similarly to $R_V=3.1$ CCM dust.
%In contrast, Dust 2 is more consistent with extremely low values of $R_V$.  Negative values of $R_V$ are unphysical.
%\begin{figure}[htbp] %  figure placement: here, top, bottom, or page
%   \centering
%   \includegraphics[width=2.8in]{output9/ccm.pdf}
%      \includegraphics[width=2.8in]{output9/ccm2.pdf} 
%         \includegraphics[width=2.8in]{output9/ccm12.pdf}
%      \includegraphics[width=2.8in]{output9/ccm22.pdf} 
%   \caption{68\% measurements of $R_X=\frac{\gamma_X}{\gamma_1-\gamma_2}$ (left) and $\frac{\gamma_X}{\gamma_2}$ (right)
%   for Dust 1 (top row) and Dust 2 (bottom row).  CCM
%   predictions for these parameters are overlaid for different values of $R_V$.}
%   \label{ccm2:fig}
%\end{figure}
%
%While the values for $\beta$ are consistent between the one- and two-dust models, the $\alpha$ values change significantly.
%This implies a correlation between $EW_{Ca}$ and the dust environment.
%
\end{document} 