\documentclass[11pt, oneside]{article}   	% use "amsart" instead of "article" for AMSLaTeX format
\usepackage{geometry}                		% See geometry.pdf to learn the layout options. There are lots.
\geometry{letterpaper}                   		% ... or a4paper or a5paper or ... 
%\geometry{landscape}                		% Activate for rotated page geometry
%\usepackage[parfill]{parskip}    		% Activate to begin paragraphs with an empty line rather than an indent
\usepackage{graphicx}				% Use pdf, png, jpg, or eps§ with pdflatex; use eps in DVI mode
								% TeX will automatically convert eps --> pdf in pdflatex		
\usepackage{amssymb}
\usepackage{amsmath}

\begin{document}

The simplified implemented model is
\begin{equation}
\left(
\begin{matrix}
U_o-V_o\\B_o-V_o\\V_o-R_o\\V_o-I_o\\EW_{Si, o}\\ EW_{Ca, o}
\end{matrix}
\right) \sim \mathcal{N}
\left(
\left(
\begin{matrix}
c_0+\alpha_0 EW_{Si} + \beta_0 EW_{Ca} +\gamma_0 k  \\
c_1+\alpha_1 EW_{Si} + \beta_1 EW_{Ca}  +\gamma_1 k  \\
c_2+\alpha_2 EW_{Si} + \beta_2 EW_{Ca} +\gamma_2 k\\
c_3+\alpha_3 EW_{Si} + \beta_3 EW_{Ca} +\gamma_3 k\\
EW_{Si}\\ EW_{Ca}
\end{matrix}
\right)
,C
\right),
\end{equation}
The data are the colors and equivalent widths on the left side, $C$ is the corresponding measurement
covariance.  There are global parameters $c$, $\alpha$, $\beta$, and $\gamma$.  Each
supernova has parameters $EW_{Si}$, $EW_{Ca}$, and $k$.  To constrain the problem $\gamma_1=1$ and one supernova is selected to have
$k=0$.

The next level of complexity to add is intrinsic dispersion through stochastic parameters that describe the global  covariance matrices
$C_{c}$.  The model would then look like
\begin{equation}
\left(
\begin{matrix}
U-V\\B-V\\V-R\\V-I\\V
\end{matrix}
\right) \sim \mathcal{N}
\left(
\left(
\begin{matrix}
c_0+\alpha_0 EW_{Si} + \beta_0 EW_{Ca} \\
c_1+\alpha_1 EW_{Si} + \beta_1 EW_{Ca}  \\
c_2+\alpha_2 EW_{Si} + \beta_2 EW_{Ca} \\
c_3+\alpha_3 EW_{Si} + \beta_3 EW_{Ca} \\
c_4+\alpha_4 EW_{Si} + \beta_4 EW_{Ca}
\end{matrix}
\right)
,C_{c}
\right),
\end{equation}

\begin{equation}
\left(
\begin{matrix}
U_o-V_o\\B_o-V_o\\V_o-R_o\\V_o-I_o\\V_o\\EW_{Si, o}\\ EW_{Ca, o}
\end{matrix}
\right) \sim \mathcal{N}
\left(
\left(
\begin{matrix}
U-V +\gamma_0 k \\B-V +\gamma_1 k \\V-R+\gamma_2 k\\V-I+\gamma_3 k\\V+\gamma_4 k\\
EW_{Si}\\ EW_{Ca}
\end{matrix}
\right)
,C
\right).
\end{equation}

%
%
%A model for a single object is that its intrinsic properties (captured by equivalent width) are drawn from a Normal distribution
%\begin{equation}
%\left(
%\begin{matrix}
%EW_{Si}\\ EW_{Ca}
%\end{matrix}
%\right) \sim \mathcal{N}
%\left(
%\left(
%\begin{matrix}
%EW_{0, Si}\\ EW_{0, Ca}
%\end{matrix}
%\right)
%,C_{EW}
%\right),
%\end{equation}
%with global parameters $EW_{0, Si}$, $EW_{0, Ca}$, and the independent elements of $C_{EW}$.
%The colors, which depend on the intrinsic properties and the extrinsic effect of dust, are modeled as
%\begin{equation}
%\left(
%\begin{matrix}
%U-V\\B-V\\V-R\\V-I
%\end{matrix}
%\right) \sim \mathcal{N}
%\left(
%\left(
%\begin{matrix}
%c_0+\alpha_0 EW_{Si} + \beta_0 EW_{Ca} +\gamma_0 k+ E(U-V) \\
%c_1+\alpha_1 EW_{Si} + \beta_1 EW_{Ca}  +k +E(B-V) \\
%c_2+\alpha_2 EW_{Si} + \beta_2 EW_{Ca} +\gamma_2 k + E(V-R)\\
%c_3+\alpha_3 EW_{Si} + \beta_3 EW_{Ca} +\gamma_3 k +E(V-I)\\
%\end{matrix}
%\right)
%,C_c
%\right),
%\end{equation}
%where there are global parameters $\alpha_X$, $\beta_X$, $c_X$,  and the independent elements of $C_c$,
%while each supernova has independent parameters $E(X-V)$ and intrinsic color parameter $k$.
%
%The measurements are drawn from
%\begin{equation}
%\left(
%\begin{matrix}
%U-V\\B-V\\V-R\\V-I\\EW_{Si}\\ EW_{Ca}
%\end{matrix}
%\right)_o \sim \mathcal{N}
%\left(
%\left(
%\begin{matrix}
%U-V\\B-V\\V-R\\V-I\\EW_{Si}\\ EW_{Ca}
%\end{matrix}
%\right)
%,C_o
%\right),
%\end{equation}
%where $C_o$ is the observation covariance matrix.
%
%Given enough supernovae to constrain the global parameters, the observed colors of each supernova are used to determine its
%color excess parameters.
\end{document} 