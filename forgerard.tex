\documentclass[11pt, oneside]{article}   	% use "amsart" instead of "article" for AMSLaTeX format
\usepackage{geometry}                		% See geometry.pdf to learn the layout options. There are lots.
\geometry{letterpaper}                   		% ... or a4paper or a5paper or ... 
%\geometry{landscape}                		% Activate for rotated page geometry
%\usepackage[parfill]{parskip}    		% Activate to begin paragraphs with an empty line rather than an indent
\usepackage{graphicx}				% Use pdf, png, jpg, or eps§ with pdflatex; use eps in DVI mode
								% TeX will automatically convert eps --> pdf in pdflatex		
\usepackage{amssymb}
\usepackage{amsmath}

\begin{document}
\section{Data}
The data set consists of synthetic $UBVRI$ peak brightnesses and Si and Ca equivalent widths at peak
of 172 SNf supernovae.  The photometry is normalized to place all supernovae at a common distance, as
inferred by redshift.
The data was provided by Manu Gangler.

\section{Model Fits}
\subsection{One-Color Model}
We hypothesize that Type Ia supernova spectral features are predictive of intrinsic
magnitudes.  The intrinsic magnitudes are not perfectly deterministic, but have some intrinsic dispersion.
An extrinsic physical process common to all supernovae can influence the propagation
of light along the line of sight to incluence apparent magnitudes. 

\subsubsection{Setup}
Specifically, we assume a linear relationship between the equivalent widths of the Ca and Si spectral features
$EW_{Ca}$ and $EW_{Si}$, and the intrinsic $UBVRI$ magnitudes
with residual Gaussian dispersion with covariance $C_c$.  A grey magnitude offset $\Delta$ is included for each supernova
to capture grey intrinsic dispersion and distance errors introduced in the normalization of the photometry.
The  extrinsic process is also assumed to be linear.  The observables
$U_o, B_o, V_o, R_o, I_o$, $EW_{Ca,o}$ and $EW_{Si,o}$ have Gaussian uncertainty with covariance $C$.  The model can be expressed
\begin{equation*}
\left(
\begin{matrix}
U\\B\\V\\R\\I
\end{matrix}
\right) \sim \mathcal{N}
\left(
\Delta +
\left(
\begin{matrix}
c_0+\alpha_0 EW_{Ca} + \beta_0 EW_{Si} \\
c_1+\alpha_1 EW_{Ca} + \beta_1 EW_{Si}  \\
c_2+\alpha_2 EW_{Ca} + \beta_2 EW_{Si} \\
c_3+\alpha_3 EW_{Ca} + \beta_3 EW_{Si} \\
c_4+\alpha_4 EW_{Ca} + \beta_4 EW_{Si}
\end{matrix}
\right)
,C_{c}
\right)
\end{equation*}
\begin{equation*}
\left(
\begin{matrix}
U_o\\B_o\\ V_o\\R_o\\I_o\\EW_{Si, o}\\ EW_{Ca, o}\end{matrix}
\right) \sim \mathcal{N}
\left(
\left(
\begin{matrix}
U +\gamma_0 k \\B +\gamma_1 k \\V+\gamma_2 k\\R+\gamma_3 k\\I+\gamma_4 k\\
EW_{Si}\\ EW_{Ca}
\end{matrix}
\right)
,C
\right)
\end{equation*}
The global parameters $c$, $\alpha$, and $\beta$ are the intercepts and slopes of the bilinear relationships between equivalent widths
and intrinsic magnitudes.  The extrinsic piece is split into the properties $\gamma$ of the physical process and the per-supernova 
parameters $k$ that indicate how much of that process each object undergoes.  To break the degeneracies inherent in the model we impose
\begin{equation*}
\langle \Delta \rangle=0, \langle k \rangle=0, \langle k^2 \rangle=1, \gamma_0 \ge 0.
\end{equation*}

Having the intrinsic dispersion $C_c$ as fit parameters seemingly introduces degeneracy in the model, as magnitude and color variation
ascribed to $\Delta$ and $\gamma k$ could also be attributed to intrinsic dispersion.  There are several features of the model
that drives the assignation of variations away from $C_c$:  The probability distribution functions decrease
with increasing $\det{C_c}$ so growing $C_c$ is disfavored; The distributions of $\Delta$ and $k$ are turn out to
be non-Gaussian, and so are not well described by a Normal covariance; The Bayesian prior selected for $C_c$ disfavors
grey offsets.

If interpreted as being due to dust, the extrinsic term could be associated with extinction $A_X = \gamma_X k$.  In
this parlance $E(B-V) = (\gamma_1-\gamma_2) k$ and $R_X = \frac{\gamma_X}{\gamma_1-\gamma_2}$.
Note that these values of $A_X$, $E(B-V)$, and $R_X$ do not depend on the normalization condition $ \langle k^2 \rangle$, i.e.\
are independent of the transformation $\gamma \rightarrow a\gamma$, $k \rightarrow a^{-1} k$.

As I perform an Bayesian analysis some mention of priors is required.  A flat prior is used for all parameters except
for the covariance matrix $C_c$, which is constructed from a correlation matrix with  $\nu=2$  LKJ prior and standard
deviations with
 $\mu=0$, $\sigma=1$ Cauchy (Lorentz) distribution prior restricted to non-negative values.
The nature of the LKJ prior is rather mysterious; one thing to note is that for $\nu=2$ the correlation prior is broad but
approaches zero at the extremum.

\subsubsection{Results}
The 68\% intervals for the global parameters $\alpha$, $\beta$, $\gamma$, and $\sigma = \sqrt{C_{c,ii}}$  are given in Table~\ref{global:tab}.
The two dimensional contours for the parameters of each filter are shown in Figure~\ref{global:fig}.  The Table does not show
the results for $c$ as their values are not significant for the aims of this analysis, the Figure does show $c$ to indicate the convergence
and correlations of these parameters in the fit.

\begin{table}
\centering
\begin{tabular}{|c|c|c|c|c|c|}
\hline
& $X=0$ &1&2&3&4\\ \hline
$\alpha_X$
&
$0.0031^{0.0008}_{-0.0009}$
&
$0.0005^{0.0007}_{-0.0007}$
&
$0.0006^{0.0006}_{-0.0006}$
&
$0.0007^{0.0005}_{-0.0005}$
&
$0.0021^{0.0004}_{-0.0004}$
\\
$\beta_X$
&
$0.0345^{0.0027}_{-0.0029}$
&
$0.0274^{0.0022}_{-0.0025}$
&
$0.0274^{0.0021}_{-0.0021}$
&
$0.0223^{0.0018}_{-0.0018}$
&
$0.0213^{0.0016}_{-0.0017}$
\\
$\gamma_X$
&
$0.3818^{0.0138}_{-0.0132}$
&
$0.3108^{0.0117}_{-0.0113}$
&
$0.2342^{0.0118}_{-0.0115}$
&
$0.1827^{0.0107}_{-0.0111}$
&
$0.1357^{0.0110}_{-0.0119}$
\\
$\sigma_X$
&
$0.0693^{0.0157}_{-0.0152}$
&
$0.0575^{0.0099}_{-0.0089}$
&
$0.0679^{0.0083}_{-0.0080}$
&
$0.0639^{0.0073}_{-0.0069}$
&
$0.0703^{0.0075}_{-0.0075}$
\\
\hline
\end{tabular}
\caption{68\% Intervals for Global Fit Parameters \label{global:tab}}
\end{table}

\begin{figure}[htbp] %  figure placement: here, top, bottom, or page
   \centering
   \includegraphics[width=2.5in]{output7/coeff0.pdf} 
   \includegraphics[width=2.5in]{output7/coeff1.pdf} 
   \includegraphics[width=2.5in]{output7/coeff2.pdf} 
      \includegraphics[width=2.5in]{output7/coeff3.pdf} 
         \includegraphics[width=2.5in]{output7/coeff4.pdf} 
            \caption{Distributions for $c$, $\alpha$, $\beta$, $\gamma$, and $\sigma$ in each of the 5 bands.}
   \label{global:fig}
\end{figure}

We find a small but significant non-zero values for $\alpha_0$ and $\alpha_4$, indicating that $EW_{Ca}$ is an indicator of $U$ and $I$
magnitudes.  All bands show larger and significant non-zero values for $\beta$.  This confirms the part of our hypothesis that spectral indicators
are tracers of supernova absolute magnitude.

The best-fit distribution for the parameters $\gamma$ are shown in Figure~\ref{gamma:fig}.  Our measurement of
$\frac{\gamma_X}{\gamma_1-\gamma_2}$ and $\frac{\gamma_X}{\gamma_2}$  compared with the predictions of CCM dust:
our fit values for $\gamma$ are consistent with the  $R_V=3.1$ CCM model.  The uncertainties in the 5 measurements
of $\frac{\gamma_X}{\gamma_1-\gamma_2}$ are correlated: measurements of $\gamma_X/\gamma_2$ have significantly smaller
uncertainties and are consistent with $R_V=3.1$.

\begin{figure}[htbp] %  figure placement: here, top, bottom, or page
   \centering
   \includegraphics[width=3in]{output7/gamma_corner.pdf} 
   \caption{Best-fit distribution for the parameters $\gamma$.  68\% measurements of $\frac{\gamma_X}{\gamma_1-\gamma_2}$ and CCM
   predictions for different values of $R_V$.}
   \label{gamma:fig}
\end{figure}

\begin{figure}[htbp] %  figure placement: here, top, bottom, or page
   \centering
   \includegraphics[width=2.8in]{output7/ccm.pdf}
      \includegraphics[width=2.8in]{output7/ccm2.pdf} 
   \caption{68\% measurements of $\frac{\gamma_X}{\gamma_1-\gamma_2}$ (left) and $\frac{\gamma_X}{\gamma_2}$ (right).  CCM
   predictions for these parameters are overlaid for different values of $R_V$.}
   \label{ccm:fig}
\end{figure}

Non-trivial residual magnitude dispersions are captured in $C_c$.   Figure~\ref{sigma:fig} shows the confidence regions for the
square root of the diagonal elements of $C_c$.
A representative example is given by the matrix comprised by the average of each element in the Monte Carlo chain.
For $UBVRI$ the matrix is
\begin{equation}
\begin{pmatrix}
0.0050 & 0.0026 & 0.0017 & 0.0014 & 0.0009 \\
0.0026 & 0.0034 & 0.0032 & 0.0026 & 0.0011 \\
0.0017 & 0.0032 & 0.0047 & 0.0041 & 0.0029 \\
0.0014 & 0.0026 & 0.0041 & 0.0042 & 0.0035 \\
0.0009 & 0.0011 & 0.0029 & 0.0035 & 0.0050
 \end{pmatrix}.
 \end{equation}
 In terms of colors $U-V$, $B-V$, $V-R$, and $V-I$ the covariance is
\begin{equation}
\begin{pmatrix}
0.0064 & 0.0025 & -0.0004 & -0.0010 \\
0.0025 & 0.0018 & -0.0000 & 0.0003 \\
-0.0004 & -0.0000 & 0.0007 & 0.0011 \\
-0.0010 & 0.0003 & 0.0011 & 0.0039
  \end{pmatrix}.
 \end{equation}
 The $B-V$ and $V-R$ colors have small standard deviations of  0.04 mag and 0.03 mag with no correlation.  The
 residuals not accounted by the
 model are stronger in the $U-V$ and $V-I$ bands.
 
 \begin{figure}[htbp] %  figure placement: here, top, bottom, or page
   \centering
   \includegraphics[width=2.4in]{output7/sigma_corner.pdf} 
   \caption{Best-fit distribution for the parameters $\sigma$, the square root of the diagonal elements of $C_c$.}
   \label{sigma:fig}
\end{figure}

The ideogram for the grey offsets $\Delta$ for all supernovae is shown in Figure~\ref{hist:fig}.  The intrinsic dispersion
is given by the standard deviation of $0.10$ mag.  The distribution is non-Gaussian, with a broad tail. 
\begin{figure}[htbp] %  figure placement: here, top, bottom, or page
   \centering
   \includegraphics[width=2.8in]{output7/Delta_hist.pdf} 
   \includegraphics[width=2.8in]{output7/k_hist.pdf} 
   \caption{Ideograms for Left: the grey offset $\Delta$; Right: the color term $k$ for all supernovae.}
   \label{hist:fig}
\end{figure}

The ideogram for the color term  $k$ for all supernovae is shown in Figure~\ref{hist:fig}.The distribution is non-Gaussian, with a broad tail. 

To illustrate how the linear model reduces the dispersion in observed absolute magnitude, plot the observed colors as a
function of the observed  $EW_{Ca, o}$, $EW_{Si,o}$ and $B_o-V_o$ in Fig.~\ref{magresidual:fig}.  Overplotted are lines with the
best-fit slope.  The data in the color magnitude diagram indicate that the slope at bluer colors is different from that at redder colors
and that our one-color linear model could be improved upon.

\begin{figure}[htbp] %  figure placement: here, top, bottom, or page
   \centering
   \includegraphics[width=2.8in]{output7/speccamag.pdf} 
      \includegraphics[width=2.8in]{output7/specsimag.pdf} 
   \includegraphics[width=2.8in]{output7/colormag.pdf} 
   \caption{Observed $UBVRI$ as a function of observed   $EW_{Ca, o}$ (top left), $EW_{Si,o}$ (top right) and $B_o-V_o$ (bottom).
   Overplotted are lines with the best-fit slope from the model fit.}
   \label{magresidual:fig}
\end{figure}

\subsubsection{Notes}
Some equivalent widths go positive.  Is this an emission?

Add Si line velocity.
%
%\subsection{Two-Color Model}
%Motivated by the indication that the one-parameter linear model does not fully describe the color-magnitude data, we develop a new model.
%Considering that dust does predict a linear extinction in magnitudes, we preserve $k$ as a color parameter and the $\gamma k$ relation
%with magnitude.   A second color parameter $D$ is introduced to be quadratically related with magnitude.  Practically I have found
%that the data are not sufficient to constrain both the linear and quadratic terms of $D$, so only the latter is considered.
%
%\subsubsection{Setup}
%\begin{equation*}
%\left(
%\begin{matrix}
%U\\B\\V\\R\\I
%\end{matrix}
%\right) \sim \mathcal{N}
%\left(
%\Delta +
%\left(
%\begin{matrix}
%c_0+\alpha_0 EW_{Ca} + \beta_0 EW_{Si} +\delta_{10} D^2\\
%c_1+\alpha_1 EW_{Ca} + \beta_1 EW_{Si} +\delta_{11} D^2\\
%c_2+\alpha_2 EW_{Ca} + \beta_2 EW_{Si} +\delta_{12} D^2\\
%c_3+\alpha_3 EW_{Ca} + \beta_3 EW_{Si} +\delta_{13} D^2\\
%c_4+\alpha_4 EW_{Ca} + \beta_4 EW_{Si} +\delta_{14} D^2
%\end{matrix}
%\right)
%,C_{c}
%\right).
%\end{equation*}
%%\begin{equation*}
%%D \sim \mathcal{N}(\bar{D},1)
%%\end{equation*}
%\begin{equation*}
%\left(
%\begin{matrix}
%U_o  \\B_o\\ V_o\\R_o\\I_o\\EW_{Si, o}\\ EW_{Ca, o}
%\end{matrix}
%\right) \sim \mathcal{N}
%\left(
%\left(
%\begin{matrix}
%U +\gamma_0 k \\B +\gamma_1 k \\V+\gamma_2 k\\R+\gamma_3 k\\I+\gamma_4 k\\
%EW_{Si}\\ EW_{Ca}
%\end{matrix}
%\right)
%,C
%\right).
%\end{equation*}
%\begin{equation*}
%\langle \Delta \rangle=0, \langle k \rangle=0, \langle k^2 \rangle=1, \langle (D-\bar{D})^2 \rangle=1, \gamma_0 \ge 0.
%\end{equation*}

%
%
%A model for a single object is that its intrinsic properties (captured by equivalent width) are drawn from a Normal distribution
%\begin{equation}
%\left(
%\begin{matrix}
%EW_{Si}\\ EW_{Ca}
%\end{matrix}
%\right) \sim \mathcal{N}
%\left(
%\left(
%\begin{matrix}
%EW_{0, Si}\\ EW_{0, Ca}
%\end{matrix}
%\right)
%,C_{EW}
%\right),
%\end{equation}
%with global parameters $EW_{0, Si}$, $EW_{0, Ca}$, and the independent elements of $C_{EW}$.
%The colors, which depend on the intrinsic properties and the extrinsic effect of dust, are modeled as
%\begin{equation}
%\left(
%\begin{matrix}
%U-V\\B-V\\V-R\\V-I
%\end{matrix}
%\right) \sim \mathcal{N}
%\left(
%\left(
%\begin{matrix}
%c_0+\alpha_0 EW_{Si} + \beta_0 EW_{Ca} +\gamma_0 k+ E(U-V) \\
%c_1+\alpha_1 EW_{Si} + \beta_1 EW_{Ca}  +k +E(B-V) \\
%c_2+\alpha_2 EW_{Si} + \beta_2 EW_{Ca} +\gamma_2 k + E(V-R)\\
%c_3+\alpha_3 EW_{Si} + \beta_3 EW_{Ca} +\gamma_3 k +E(V-I)\\
%\end{matrix}
%\right)
%,C_c
%\right),
%\end{equation}
%where there are global parameters $\alpha_X$, $\beta_X$, $c_X$,  and the independent elements of $C_c$,
%while each supernova has independent parameters $E(X-V)$ and intrinsic color parameter $k$.
%
%The measurements are drawn from
%\begin{equation}
%\left(
%\begin{matrix}
%U-V\\B-V\\V-R\\V-I\\EW_{Si}\\ EW_{Ca}
%\end{matrix}
%\right)_o \sim \mathcal{N}
%\left(
%\left(
%\begin{matrix}
%U-V\\B-V\\V-R\\V-I\\EW_{Si}\\ EW_{Ca}
%\end{matrix}
%\right)
%,C_o
%\right),
%\end{equation}
%where $C_o$ is the observation covariance matrix.
%
%Given enough supernovae to constrain the global parameters, the observed colors of each supernova are used to determine its
%color excess parameters.
\end{document} 