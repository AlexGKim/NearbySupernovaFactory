\documentclass[11pt, oneside]{article}   	% use "amsart" instead of "article" for AMSLaTeX format
\usepackage{geometry}                		% See geometry.pdf to learn the layout options. There are lots.
\geometry{letterpaper}                   		% ... or a4paper or a5paper or ... 
%\geometry{landscape}                		% Activate for rotated page geometry
%\usepackage[parfill]{parskip}    		% Activate to begin paragraphs with an empty line rather than an indent
\usepackage{graphicx}				% Use pdf, png, jpg, or eps§ with pdflatex; use eps in DVI mode
								% TeX will automatically convert eps --> pdf in pdflatex		
\usepackage{amssymb}
\usepackage{amsmath}

\begin{document}
A model for a single object is that its intrinsic properties (captured by equivalent width) are drawn from a Normal distribution
\begin{equation}
\left(
\begin{matrix}
EW_{Si}\\ EW_{Ca}
\end{matrix}
\right) \sim \mathcal{N}
\left(
\left(
\begin{matrix}
EW_{0, Si}\\ EW_{0, Ca}
\end{matrix}
\right)
,C_{EW}
\right),
\end{equation}
with global parameters $EW_{0, Si}$, $EW_{0, Ca}$, and the independent elements of $C_{EW}$.
The colors, which depend on the intrinsic properties and the extrinsic effect of dust, are modeled as
\begin{equation}
\left(
\begin{matrix}
U-V\\B-V\\V-R\\V-I
\end{matrix}
\right) \sim \mathcal{N}
\left(
\left(
\begin{matrix}
c_0+\alpha_0 EW_{Si} + \beta_0 EW_{Ca} +\gamma_0 k+ E(U-V) \\
c_1+\alpha_1 EW_{Si} + \beta_1 EW_{Ca}  +k +E(B-V) \\
c_2+\alpha_2 EW_{Si} + \beta_2 EW_{Ca} +\gamma_2 k + E(V-R)\\
c_3+\alpha_3 EW_{Si} + \beta_3 EW_{Ca} +\gamma_3 k +E(V-I)\\
\end{matrix}
\right)
,C_c
\right),
\end{equation}
where there are global parameters $\alpha_X$, $\beta_X$, $c_X$,  and the independent elements of $C_c$,
while each supernova has independent parameters $E(X-V)$ and intrinsic color parameter $k$.

The measurements are drawn from
\begin{equation}
\left(
\begin{matrix}
U-V\\B-V\\V-R\\V-I\\EW_{Si}\\ EW_{Ca}
\end{matrix}
\right)_o \sim \mathcal{N}
\left(
\left(
\begin{matrix}
U-V\\B-V\\V-R\\V-I\\EW_{Si}\\ EW_{Ca}
\end{matrix}
\right)
,C_o
\right),
\end{equation}
where $C_o$ is the observation covariance matrix.

Given enough supernovae to constrain the global parameters, the observed colors of each supernova are used to determine its
color excess parameters.
\end{document} 