\documentclass{aastex}   	% use "amsart" instead of "article" for AMSLaTeX format
\usepackage{geometry}                		% See geometry.pdf to learn the layout options. There are lots.
\geometry{letterpaper}                   		% ... or a4paper or a5paper or ... 
%\geometry{landscape}                		% Activate for rotated page geometry
%\usepackage[parfill]{parskip}    		% Activate to begin paragraphs with an empty line rather than an indent
\usepackage{graphicx}				% Use pdf, png, jpg, or eps§ with pdflatex; use eps in DVI mode
								% TeX will automatically convert eps --> pdf in pdflatex		
%\usepackage{amssymb}
\usepackage{amsmath}
\usepackage{natbib}
\usepackage{lineno}

\usepackage{color}
\linenumbers


\begin{document}

\title{Detection of an Intrinsic Color Parameter within the Type~Ia Supernovae of the Nearby Supernova Factory}
\author
{
    A.~G.~Kim,\altaffilmark{1}
    G.~Smadja,\altaffilmark{2} 
    G.~Aldering,\altaffilmark{1}
    P.~Antilogus,\altaffilmark{4}
    S.~Bailey,\altaffilmark{1}
    C.~Baltay,\altaffilmark{5}
    K.~Barbary,\altaffilmark{3}
    D.~Baugh,\altaffilmark{6}
    K.~Boone,\altaffilmark{1,3}
    S.~Bongard,\altaffilmark{4}
    C.~Buton,\altaffilmark{2}
    J.~Chen,\altaffilmark{6}
    N.~Chotard,\altaffilmark{2}
    Y.~Copin,\altaffilmark{2}
    P.~Fagrelius,\altaffilmark{1,3}
    H.~K.~Fakhouri,\altaffilmark{1,3}
    U.~Feindt,\altaffilmark{9}
    D.~Fouchez,\altaffilmark{10}
    E.~Gangler,\altaffilmark{11}  
    B.~Hayden,\altaffilmark{1}
    W.~Hillebrandt,\altaffilmark{16}
    M.~Kowalski,\altaffilmark{9,12}
    P.-F.~Leget,\altaffilmark{11}
    S.~Lombardo,\altaffilmark{9}
    J.~Nordin,\altaffilmark{1,9}
    R.~Pain,\altaffilmark{4} 
    E.~Pecontal,\altaffilmark{13}
    R.~Pereira,\altaffilmark{2}
    S.~Perlmutter,\altaffilmark{1,3}
    D.~Rabinowitz,\altaffilmark{5}
    M.~Rigault,\altaffilmark{9} 
    D.~Rubin,\altaffilmark{1,14}
    K.~Runge,\altaffilmark{1}
    C.~Saunders,\altaffilmark{1,3}
    C.~Sofiatti,\altaffilmark{1,3} 
    N.~Suzuki,\altaffilmark{1}
    S.~Taubenberger,\altaffilmark{16}
    C.~Tao,\altaffilmark{6,10}
    R.~C.~Thomas,\altaffilmark{1} \\
    (The Nearby Supernova Factory)
}


\altaffiltext{1}
{
    Physics Division, Lawrence Berkeley National Laboratory, 
    1 Cyclotron Road, Berkeley, CA, 94720
}

\altaffiltext{2}
{
    Universit\'e de Lyon, F-69622, Lyon, France ; Universit\'e de Lyon 1, Villeurbanne ; 
    CNRS/IN2P3, Institut de Physique Nucl\'eaire de Lyon.
}

\altaffiltext{3}
{
    Department of Physics, University of California Berkeley,
    366 LeConte Hall MC 7300, Berkeley, CA, 94720-7300
}
\altaffiltext{4}
{
    Laboratoire de Physique Nucl\'eaire et des Hautes \'Energies,
    Universit\'e Pierre et Marie Curie Paris 6, Universit\'e Paris Diderot Paris 7, CNRS-IN2P3, 
    4 place Jussieu, 75252 Paris Cedex 05, France
}
\altaffiltext{5}
{
    Department of Physics, Yale University, 
    New Haven, CT, 06250-8121
}
\altaffiltext{6}
{
    Tsinghua Center for Astrophysics, Tsinghua University, Beijing 100084, China 
}

\altaffiltext{9}
{
    Institut fur Physik,  Humboldt-Universitat zu Berlin,
    Newtonstr. 15, 12489 Berlin
}
\altaffiltext{10}
{
    Centre de Physique des Particules de Marseille, 
    Aix-Marseille Universit\'e , CNRS/IN2P3, 
    163 avenue de Luminy - Case 902 - 13288 Marseille Cedex 09, France
}
\altaffiltext{11}
{
    Clermont Universit\'e, Universit\'e Blaise Pascal, CNRS/IN2P3, Laboratoire de Physique Corpusculaire,
    BP 10448, F-63000 Clermont-Ferrand, France
}
\altaffiltext{12}
{
    DESY, D-15735 Zeuthen, Germany
}
\altaffiltext{13}
{
    Centre de Recherche Astronomique de Lyon, Universit\'e Lyon 1,
    9 Avenue Charles Andr\'e, 69561 Saint Genis Laval Cedex, France
}
\altaffiltext{14}
{
    Department of Physics, Florida State University,
    315 Keen Building, Tallahassee, FL 32306-4350
}
\altaffiltext{15}
{
    Computational Cosmology Center, Computational Research Division, Lawrence Berkeley National Laboratory, 
    1 Cyclotron Road MS 50B-4206, Berkeley, CA, 94720
}

\altaffiltext{16}
{
    Max-Planck-Institut f\"ur Astrophysik, Karl-Schwarzschild-Str. 1,
D-85748 Garching, Germany
}

\begin{abstract}
An empirical model for SN~Ia peak magnitudes with two color parameters and dependence on the equivalent widths of CaII, SiII, and SiII velocity
is applied to the supernova sample of the Nearby Supernova Factory.  The peak magnitudes and their colors are found to be 
dependent on the spectral equivalent widths, and two independent color parameters with better than 0.999 confidence.
One parameter, interpreted as due to extrinsic host-galaxy
dust extinction, is consistent with an $R_V=3.06$  \citet{1999PASP..111...63F} dust model.  The second parameter is inconsistent with dust and
is inferred to have origins intrinsic to supernovae.  Our model explains features of external data: 
\textcolor{red}{
the wide range of inferred $R_V$ among
SNe~Ia, the systematically lower inferred $R_V$ of redder supernovae relative to the upper bound of the full population,} and the colors of SN~2014J.
\end{abstract}

\keywords{supernovae: general; supernovae: SN 2014J}

\section{Introduction}
Type~Ia supernovae (SNe~Ia) form a homogenous set of exploding stars and as such were early recognized and utilized as a powerful distance indicator 
and probe of cosmology \citep[e.g.][]{1992ARA&A..30..359B, 1993ApJ...415....1S}.  After further careful consideration of supernova data, it was recognized
that SN~Ia light-curve shapes \citep{1993ApJ...413L.105P} and colors \citep{1998A&A...331..815T} exhibit subtle signs of heterogeneity,
which being correlated with absolute magnitude
need to be considered when inferring distances.  Empirical models parameterizing SNe~Ia by their light-curve shape
and color were developed \citep{1996ApJ...473...88R, 1999ApJ...517..565P} that enabled absolute magnitude corrections
and distance measurements of cosmological supernovae,
which 
were subsequently used in the discovery of the accelerating expansion of the Universe \citep{1998AJ....116.1009R,1999ApJ...517..565P}.

The two most commonly used supernova-cosmology light-curve fitters today are SALT2 \citep{2007A&A...466...11G} and MLCS2k2
\citep{2007ApJ...659..122J}.\footnote{Light-curve fitters with more flexible degrees of freedom
\citep[e.g.][]{2008ApJ...681..482C, 2011AJ....141...19B} are used to study SN~Ia heterogeneity.}
They remain two-parameter models, with one parameter characterizing light-curve shape and the other
 color.  The physical cause of the color diversity is treated differently by the two sets of authors: 
\citet{2007A&A...466...11G} pragmatically extract color variation empirically from SNe that span a wide range of colors, with no attribution
to either intrinsic or extrinsic origins;
\citet{2007ApJ...659..122J}
attribute changes in color to the attenuation of light from host-galaxy dust. Neither model accommodates the naive expectation of independent
intrinsic and extrinsic color variability.

There is evidence that supports the expectation that a single parameter cannot describe the full range
of colors seen in the SN~Ia population.  One approach to look for color diversity is to find correlations between color and spectral features.
\citet{2009ApJ...699L.139W, 2011ApJ...729...55F} find two populations distinguished
by Si velocity with differing $B_{max}-V_{max}$.
\citet{2015MNRAS.451.1973S}
find that high-velocity SiII$\lambda$6355 is found in objects that have red ultraviolet/optical colors near maximum brightness.
Another approach to probe color diversity is to use multiple colors of a single supernova to infer individual differential reddening. 
\citet{2014ApJ...789...32B, 2015MNRAS.453.3300A} find a wide
range of total-to-selective extinction with average values significantly lower than $R_V = 3.1$,
the canonical value for diffuse Milky Way dust.
The also confirm the \citet{2011ApJ...729...55F} finding that low $R_V$ is associated with high-extinction supernovae.
In contrast, \citet{2011A&A...529L...4C} argue that after accounting for the diversity of spectral features,
the standard $R_V=3.1$ is recovered for supernovae on average.

The Nearby Supernova Factory \citep[SNfactory;][]{2004SPIE.5249..146L} has collected  
spectrophotometric time series of hundreds of Hubble-flow $0.03<z<0.08$ SNe~Ia.   The $3200$--$10000$~\AA\ spectral coverage
provides measurements of an array of supernova spectral features while also providing synthetic broadband photometry
spanning near-UV to near-IR SN-frame wavelengths.  Well over a hundred these supernovae have temporal coverage over
peak brightness.  This dataset provides a homogenous sample with which to study SN~Ia colors and spectral features together.

In this article we use the idea that spectral indicators carry information on intrinsic supernova colors at peak magnitude.
We allow for two independent color parameters, one due to intrinsic supernova
properties and the other due to
extrinsic processes, which we attribute to dust.  The data used in this analysis are described in \S\ref{data:sec}.  Our supernova model is presented in
\S\ref{model:sec} and results are given in \S\ref{results:sec}.  In \S\ref{sn2014j:sec} our results are applied to the out-of-sample
supernova SN2014J that has been previously identified as having extreme total-to-selective extinction behavior.
We conclude with a discussion of our findings in \S\ref{discussion:sec}.


\section{Data}
\label{data:sec}

Our analysis uses the spectrophotometric data set obtained by
the SNfactory with the SuperNova Integral Field
Spectrograph \citep[SNIFS,][]{2004SPIE.5249..146L}.  SNIFS is a fully integrated
instrument optimized for automated observation of point sources on a
structured background over the full ground-based optical window at
moderate spectral resolution ($R \sim 500$).  It consists of a
high-throughput wide-band pure-lenslet integral field spectrograph
\citep[IFS, ``\`a la TIGER;''][]{1995A&AS..113..347B,2000ASPC..195..173B,2001MNRAS.326...23B}, a
multi-filter photometric channel to image the field in the vicinity of
the IFS for atmospheric transmission monitoring simultaneous with
spectroscopy, and an acquisition/guiding channel.  The IFS possesses a
fully-filled $6\farcs 4 \times 6\farcs 4$ spectroscopic field of view
subdivided into a grid of $15 \times 15$ spatial elements, a
dual-channel spectrograph covering 3200--5200~\AA\ and 5100--10000~\AA\
simultaneously, and an internal calibration unit (continuum and arc
lamps).  SNIFS is mounted on the south bent Cassegrain port of the
University of Hawaii 2.2~m telescope on Mauna Kea, and is operated
remotely.  Observations are reduced using the SNfactory's dedicated data
reduction pipeline, similar to that presented in \S4 of \citet{2001MNRAS.326...23B}.
A discussion of the software pipeline is presented in
\citet{2006ApJ...650..510A} and is updated in \citet{2010ApJ...713.1073S}.  A detailed
description of host-galaxy subtraction is given in \citet{2011MNRAS.418..258B}.

The current status of the data and its treatment s presented in \citet{2015ApJ...815...58F}.
We provide a brief summary of the points important for this analysis.
The spectral time-series  are corrected for Milky Way dust
extinction \citep{1989ApJ...345..245C,1998ApJ...500..525S} but not for the
effects of circumstellar or host-galaxy dust extinction.  
Each spectral time series is
blue-shifted to rest-frame, and the fluxes converted to luminosity assuming
distances expected for the supernova redshifts given a flat
$\Lambda$CDM cosmology with $\Omega_M = 0.28$ (with an arbitrarily selected
$H_0$ since the current analysis does not depend on the absolute magnitude scale).

Supernova-frame synthetic photometry is generated for a top-hat filter system
comprised of five 
\color{red}
bands with the following wavelength ranges: $U$ $[3300.00 - 3978.02]$\AA;
$B$ $[3978.02-4795.35]$\AA;
$V$ $[4795.35-5780.60]$\AA;
$R$ $[5780.60-6968.29]$\AA;
$I$ $[6968.29-8400.00]$\AA.
\color{black}
For each supernova, the magnitudes at peak brightness are determined using single-band SALT2 fits.
The equivalent widths of SiII$\lambda 4141$ and the CaII H\&K features are computed as
in \citet{2008A&A...477..717B} and the  velocity of SiII$\lambda 6355$ (expressed as the
observed location of the feature $\lambda_{Si} = (1+v_{Si}/c) 6355$\AA)
as in \citet{chotard:thesis}.
Equivalent widths with 2.5 days of maximum are considered, in the case of multiple spectra the one
closest to maximum is used \citep{chotard:thesis}. [Confirm!]
The weighted mean from phases within 3 days of maximum light is used for the  SiII$\lambda 6355$ velocity.

Our analysis sample is constituted with the  163 supernovae that have the data coverage to 
give photometric and spectroscopic statistics described above.
The data are given in Table~\ref{data:tab}.
The peak magnitude uncertainties do have covariance, which are accounted
for in the analysis; only the variance is included in the table.

\begin{deluxetable}{ccccccccc}
\tabletypesize{\tiny}
\tablecaption{Supernova Spectral-Feature and Peak-Magnitude Datat\label{data:tab}}
\tablehead{
\colhead{$n$} & \colhead{$EW_{Ca}$} & \colhead{$EW_{Si}$} & \colhead{$\lambda_{Si}$} & \colhead{$U$} & \colhead{$B$} & \colhead{$V$} & \colhead{$R$} & \colhead{$I$} \\
 & (\AA) & (\AA)&  \colhead{(\AA)} & \multicolumn{5}{c}{}
}
\startdata
0 & $109.7 \pm 5.9$ & $ 17.5 \pm 0.7$& $ 6101 \pm   3$ & $-29.31 \pm   0.01$ & $-29.12 \pm   0.01$& $-28.60 \pm   0.01$& $-28.35 \pm   0.01$& $-27.60 \pm   0.01$ \\
1 & $149.7 \pm 1.2$ & $ 14.3 \pm 0.6$& $ 6139 \pm   2$ & $-28.61 \pm   0.02$ & $-28.69 \pm   0.02$& $-28.32 \pm   0.02$& $-28.08 \pm   0.02$& $-27.40 \pm   0.02$ \\
2 & $ 63.8 \pm 21.5$ & $ 27.6 \pm 3.8$& $ 6131 \pm   6$ & $-29.04 \pm   0.07$ & $-28.79 \pm   0.07$& $-28.32 \pm   0.07$& $-28.12 \pm   0.07$& $-27.41 \pm   0.07$ \\
3 & $129.9 \pm 0.9$ & $  9.5 \pm 0.7$& $ 6093 \pm   2$ & $-29.23 \pm   0.01$ & $-29.07 \pm   0.01$& $-28.72 \pm   0.01$& $-28.44 \pm   0.01$& $-27.69 \pm   0.01$ \\
4 & $114.4 \pm 0.9$ & $  8.4 \pm 0.4$& $ 6129 \pm   6$ & $-29.26 \pm   0.01$ & $-28.99 \pm   0.01$& $-28.50 \pm   0.01$& $-28.20 \pm   0.01$& $-27.44 \pm   0.01$ \\
5 & $ 64.9 \pm 4.5$ & $ 16.5 \pm 0.7$& $ 6098 \pm   4$ & $-29.55 \pm   0.02$ & $-29.19 \pm   0.02$& $-28.68 \pm   0.02$& $-28.48 \pm   0.02$& $-27.93 \pm   0.02$ \\
6 & $151.4 \pm 3.0$ & $ 13.9 \pm 1.1$& $ 6140 \pm   4$ & $-28.63 \pm   0.02$ & $-28.57 \pm   0.01$& $-28.20 \pm   0.01$& $-27.99 \pm   0.01$& $-27.34 \pm   0.01$ \\
7 & $151.9 \pm 1.5$ & $  7.9 \pm 0.7$& $ 6099 \pm   4$ & $-29.37 \pm   0.01$ & $-29.14 \pm   0.01$& $-28.71 \pm   0.01$& $-28.40 \pm   0.01$& $-27.64 \pm   0.01$ \\
8 & $108.2 \pm 6.0$ & $ 20.6 \pm 1.9$& $ 6122 \pm   5$ & $-29.02 \pm   0.06$ & $-28.80 \pm   0.06$& $-28.40 \pm   0.06$& $-28.20 \pm   0.06$& $-27.50 \pm   0.06$ \\
9 & $123.7 \pm 6.6$ & $ 16.4 \pm 4.3$& $ 6099 \pm   4$ & $-28.26 \pm   0.02$ & $-28.20 \pm   0.02$& $-27.93 \pm   0.02$& $-27.74 \pm   0.02$& $-27.22 \pm   0.04$ \\
10 & $101.7 \pm 1.1$ & $  7.7 \pm 0.7$& $ 6144 \pm   2$ & $-29.77 \pm   0.02$ & $-29.43 \pm   0.02$& $-28.97 \pm   0.02$& $-28.64 \pm   0.02$& $-27.93 \pm   0.02$ \\
11 & $119.8 \pm 7.4$ & $ 14.2 \pm 2.7$& $ 6120 \pm  10$ & $-28.94 \pm   0.02$ & $-28.75 \pm   0.02$& $-28.32 \pm   0.02$& $-28.07 \pm   0.02$& $-27.31 \pm   0.02$ \\
12 & $157.5 \pm 7.5$ & $ 16.7 \pm 1.8$& $ 6112 \pm   4$ & $-28.97 \pm   0.02$ & $-28.96 \pm   0.01$& $-28.61 \pm   0.01$& $-28.37 \pm   0.01$& $-27.62 \pm   0.01$ \\
13 & $ 57.1 \pm 0.4$ & $ 17.9 \pm 0.3$& $ 6115 \pm   2$ & $-29.20 \pm   0.01$ & $-28.84 \pm   0.01$& $-28.47 \pm   0.01$& $-28.23 \pm   0.01$& $-27.73 \pm   0.04$ \\
14 & $ 90.0 \pm 16.5$ & $ 26.5 \pm 1.5$& $ 6116 \pm   3$ & $-29.11 \pm   0.02$ & $-28.82 \pm   0.01$& $-28.42 \pm   0.01$& $-28.19 \pm   0.01$& $-27.54 \pm   0.01$ \\
15 & $165.6 \pm 10.7$ & $ 12.7 \pm 2.8$& $ 6150 \pm   4$ & $-28.76 \pm   0.01$ & $-28.76 \pm   0.01$& $-28.40 \pm   0.01$& $-28.17 \pm   0.01$& $-27.45 \pm   0.02$ \\
16 & $ 38.7 \pm 9.9$ & $  7.2 \pm 8.7$& $ 6134 \pm   4$ & $-27.93 \pm   0.38$ & $-27.76 \pm   0.38$& $-27.73 \pm   0.38$& $-27.59 \pm   0.38$& $-27.21 \pm   0.38$ \\
17 & $ 83.2 \pm 0.7$ & $ 19.4 \pm 0.6$& $ 6125 \pm   1$ & $-29.30 \pm   0.01$ & $-28.95 \pm   0.01$& $-28.44 \pm   0.01$& $-28.17 \pm   0.01$& $-27.49 \pm   0.01$ \\
18 & $129.2 \pm 2.8$ & $  9.1 \pm 1.3$& $ 6155 \pm   4$ & $-29.14 \pm   0.02$ & $-28.98 \pm   0.01$& $-28.56 \pm   0.01$& $-28.28 \pm   0.01$& $-27.61 \pm   0.01$ \\
19 & $125.4 \pm 5.1$ & $ 26.9 \pm 1.6$& $ 6127 \pm   3$ & $-29.04 \pm   0.04$ & $-28.81 \pm   0.04$& $-28.38 \pm   0.04$& $-28.16 \pm   0.04$& $-27.57 \pm   0.04$ \\
20 & $104.8 \pm 0.9$ & $ 12.7 \pm 0.3$& $ 6132 \pm   1$ & $-28.58 \pm   0.02$ & $-28.36 \pm   0.02$& $-27.98 \pm   0.02$& $-27.77 \pm   0.02$& $-27.17 \pm   0.02$ \\
21 & $109.0 \pm 0.3$ & $  8.6 \pm 0.1$& $ 6130 \pm   2$ & $-28.35 \pm   0.01$ & $-28.15 \pm   0.01$& $-27.79 \pm   0.01$& $-27.58 \pm   0.01$& $-26.97 \pm   0.01$ \\
22 & $100.1 \pm 2.1$ & $ 13.7 \pm 1.1$& $ 6158 \pm   3$ & $-29.46 \pm   0.01$ & $-29.19 \pm   0.01$& $-28.71 \pm   0.01$& $-28.39 \pm   0.01$& $-27.77 \pm   0.01$ \\
23 & $106.1 \pm 10.4$ & $ 17.0 \pm 0.9$& $ 6148 \pm   4$ & $-29.54 \pm   0.02$ & $-29.28 \pm   0.01$& $-28.76 \pm   0.01$& $-28.42 \pm   0.01$& $-27.74 \pm   0.04$ \\
24 & $122.9 \pm 1.7$ & $ 11.4 \pm 0.8$& $ 6116 \pm   2$ & $-29.62 \pm   0.07$ & $-29.39 \pm   0.07$& $-28.95 \pm   0.07$& $-28.64 \pm   0.07$& $-27.91 \pm   0.07$ \\
25 & $105.1 \pm 2.2$ & $  8.6 \pm 1.3$& $ 6118 \pm   3$ & $-29.34 \pm   0.01$ & $-29.10 \pm   0.01$& $-28.73 \pm   0.01$& $-28.46 \pm   0.01$& $-27.82 \pm   0.01$ \\
26 & $158.3 \pm 3.3$ & $ 16.3 \pm 1.7$& $ 6099 \pm   4$ & $-28.90 \pm   0.01$ & $-28.81 \pm   0.01$& $-28.45 \pm   0.01$& $-28.22 \pm   0.01$& $-27.52 \pm   0.01$ \\
27 & $105.8 \pm 2.3$ & $  6.5 \pm 1.0$& $ 6122 \pm   2$ & $-29.38 \pm   0.01$ & $-29.07 \pm   0.01$& $-28.68 \pm   0.01$& $-28.37 \pm   0.01$& $-27.69 \pm   0.01$ \\
28 & $116.5 \pm 1.2$ & $ 19.8 \pm 0.9$& $ 6128 \pm   1$ & $-29.02 \pm   0.07$ & $-28.84 \pm   0.07$& $-28.40 \pm   0.07$& $-28.14 \pm   0.07$& $-27.52 \pm   0.07$ \\
29 & $116.2 \pm 14.9$ & $ 15.5 \pm 0.7$& $ 6125 \pm   3$ & $-29.30 \pm   0.01$ & $-29.15 \pm   0.01$& $-28.64 \pm   0.01$& $-28.35 \pm   0.01$& $-27.63 \pm   0.01$ \\
30 & $149.1 \pm 1.4$ & $ 14.9 \pm 0.7$& $ 6120 \pm   2$ & $-29.11 \pm   0.01$ & $-28.97 \pm   0.01$& $-28.50 \pm   0.01$& $-28.22 \pm   0.01$& $-27.46 \pm   0.01$ \\
31 & $ 42.8 \pm 1.2$ & $  4.0 \pm 3.1$& $ 6112 \pm   3$ & $-29.76 \pm   0.02$ & $-29.40 \pm   0.02$& $-29.04 \pm   0.02$& $-28.74 \pm   0.02$& $-28.11 \pm   0.02$ \\
32 & $ 30.2 \pm 4.4$ & $  4.0 \pm 3.0$& $ 6141 \pm   3$ & $-29.47 \pm   0.02$ & $-29.10 \pm   0.01$& $-28.75 \pm   0.01$& $-28.45 \pm   0.01$& $-27.87 \pm   0.01$ \\
33 & $117.6 \pm 2.7$ & $  9.1 \pm 1.0$& $ 6113 \pm   2$ & $-29.44 \pm   0.03$ & $-29.21 \pm   0.03$& $-28.72 \pm   0.03$& $-28.42 \pm   0.03$& $-27.68 \pm   0.03$ \\
34 & $104.6 \pm 2.4$ & $ 26.6 \pm 3.2$& $ 6104 \pm   4$ & $-29.16 \pm   0.02$ & $-28.82 \pm   0.02$& $-28.43 \pm   0.02$& $-28.20 \pm   0.02$& $-27.55 \pm   0.02$ \\
35 & $126.7 \pm 1.8$ & $ 21.1 \pm 1.1$& $ 6108 \pm   5$ & $-29.22 \pm   0.01$ & $-29.10 \pm   0.01$& $-28.63 \pm   0.01$& $-28.34 \pm   0.01$& $-27.62 \pm   0.01$ \\
36 & $111.4 \pm 2.3$ & $ 10.6 \pm 1.5$& $ 6130 \pm   3$ & $-29.54 \pm   0.01$ & $-29.23 \pm   0.01$& $-28.80 \pm   0.01$& $-28.46 \pm   0.01$& $-27.82 \pm   0.01$ \\
37 & $ 95.3 \pm 3.5$ & $ 23.3 \pm 1.5$& $ 6125 \pm   5$ & $-29.22 \pm   0.08$ & $-28.96 \pm   0.08$& $-28.50 \pm   0.08$& $-28.26 \pm   0.08$& $-27.56 \pm   0.08$ \\
38 & $ 31.5 \pm 3.7$ & $  4.5 \pm 1.3$& $ 6168 \pm   5$ & $-29.50 \pm   0.45$ & $-29.25 \pm   0.45$& $-28.89 \pm   0.45$& $-28.61 \pm   0.45$& $-27.97 \pm   0.45$ \\
39 & $135.7 \pm 1.1$ & $ 11.2 \pm 1.5$& $ 6088 \pm   2$ & $-29.13 \pm   0.02$ & $-28.99 \pm   0.01$& $-28.59 \pm   0.01$& $-28.31 \pm   0.01$& $-27.55 \pm   0.01$ \\
40 & $ 95.4 \pm 18.8$ & $ 24.1 \pm 1.0$& $ 6122 \pm   4$ & $-29.01 \pm   0.03$ & $-28.78 \pm   0.03$& $-28.35 \pm   0.03$& $-28.17 \pm   0.03$& $-27.54 \pm   0.03$ \\
41 & $115.6 \pm 2.8$ & $ 13.5 \pm 6.9$& $ 6068 \pm   4$ & $-29.33 \pm   0.02$ & $-29.12 \pm   0.02$& $-28.84 \pm   0.02$& $-28.49 \pm   0.02$& $-27.77 \pm   0.02$ \\
42 & $ 93.3 \pm 2.6$ & $  8.3 \pm 2.2$& $ 6102 \pm   3$ & $-28.58 \pm   0.01$ & $-28.47 \pm   0.01$& $-28.29 \pm   0.01$& $-28.05 \pm   0.01$& $-27.50 \pm   0.01$ \\
43 & $101.1 \pm 1.6$ & $  7.4 \pm 1.0$& $ 6122 \pm   3$ & $-28.01 \pm   0.01$ & $-28.00 \pm   0.01$& $-27.85 \pm   0.01$& $-27.77 \pm   0.01$& $-27.31 \pm   0.01$ \\
44 & $166.8 \pm 1.8$ & $ 21.8 \pm 1.2$& $ 6169 \pm   3$ & $-28.67 \pm   0.20$ & $-28.74 \pm   0.20$& $-28.30 \pm   0.20$& $-28.05 \pm   0.20$& $-27.28 \pm   0.20$ \\
45 & $110.6 \pm 1.0$ & $  8.0 \pm 1.3$& $ 6098 \pm   1$ & $-29.88 \pm   0.76$ & $-29.58 \pm   0.76$& $-29.09 \pm   0.76$& $-28.71 \pm   0.76$& $-27.91 \pm   0.76$ \\
46 & $ 94.1 \pm 1.3$ & $  6.7 \pm 0.6$& $ 6150 \pm   2$ & $-29.72 \pm   0.01$ & $-29.39 \pm   0.01$& $-28.97 \pm   0.01$& $-28.64 \pm   0.01$& $-27.96 \pm   0.01$ \\
47 & $105.9 \pm 5.4$ & $ 18.3 \pm 1.8$& $ 6124 \pm   4$ & $-29.23 \pm   0.02$ & $-29.04 \pm   0.01$& $-28.63 \pm   0.01$& $-28.35 \pm   0.01$& $-27.62 \pm   0.01$ \\
48 & $109.6 \pm 1.3$ & $  8.9 \pm 0.7$& $ 6142 \pm   2$ & $-28.54 \pm   0.01$ & $-28.40 \pm   0.01$& $-28.11 \pm   0.01$& $-27.89 \pm   0.01$& $-27.31 \pm   0.01$ \\
49 & $105.0 \pm 3.2$ & $  6.9 \pm 1.1$& $ 6124 \pm   7$ & $-27.89 \pm   0.02$ & $-27.82 \pm   0.02$& $-27.69 \pm   0.02$& $-27.63 \pm   0.02$& $-27.16 \pm   0.04$ \\
50 & $ 88.0 \pm 2.7$ & $ 32.2 \pm 1.9$& $ 6111 \pm   7$ & $-28.37 \pm   0.03$ & $-28.26 \pm   0.03$& $-27.92 \pm   0.03$& $-27.77 \pm   0.03$& $-27.13 \pm   0.03$ \\
51 & $108.9 \pm 3.5$ & $ 17.1 \pm 1.0$& $ 6131 \pm   6$ & $-29.25 \pm   0.02$ & $-29.02 \pm   0.02$& $-28.56 \pm   0.02$& $-28.32 \pm   0.01$& $-27.65 \pm   0.02$ \\
52 & $143.5 \pm 2.2$ & $ 10.2 \pm 0.9$& $ 6142 \pm   3$ & $-29.38 \pm   0.01$ & $-29.17 \pm   0.01$& $-28.69 \pm   0.01$& $-28.40 \pm   0.01$& $-27.62 \pm   0.01$ \\
53 & $150.5 \pm 8.9$ & $ 46.3 \pm 8.6$& $ 6110 \pm   8$ & $-28.70 \pm   0.02$ & $-28.61 \pm   0.02$& $-28.16 \pm   0.02$& $-27.91 \pm   0.02$& $-27.07 \pm   0.04$ \\
54 & $114.7 \pm 2.8$ & $  9.4 \pm 0.9$& $ 6145 \pm   3$ & $-28.53 \pm   0.02$ & $-28.41 \pm   0.01$& $-28.11 \pm   0.01$& $-27.99 \pm   0.01$& $-27.38 \pm   0.01$ \\
55 & $ 85.0 \pm 1.1$ & $  6.0 \pm 0.4$& $ 6140 \pm   3$ & $-29.73 \pm   0.01$ & $-29.43 \pm   0.01$& $-29.04 \pm   0.01$& $-28.72 \pm   0.01$& $-28.07 \pm   0.01$ \\
56 & $134.8 \pm 15.7$ & $ 19.7 \pm 3.7$& $ 6095 \pm   5$ & $-28.56 \pm   0.02$ & $-28.63 \pm   0.01$& $-28.32 \pm   0.01$& $-28.13 \pm   0.01$& $-27.42 \pm   0.01$ \\
57 & $129.6 \pm 5.6$ & $ 31.1 \pm 23.8$& $ 6151 \pm   4$ & $-27.88 \pm   0.02$ & $-28.12 \pm   0.01$& $-28.02 \pm   0.01$& $-27.86 \pm   0.01$& $-27.24 \pm   0.03$ \\
58 & $122.1 \pm 1.7$ & $  8.3 \pm 0.5$& $ 6123 \pm   2$ & $-29.63 \pm   0.01$ & $-29.38 \pm   0.01$& $-28.92 \pm   0.01$& $-28.60 \pm   0.01$& $-27.88 \pm   0.01$ \\
59 & $110.8 \pm 10.7$ & $ 20.4 \pm 1.0$& $ 6135 \pm   4$ & $-29.16 \pm   0.01$ & $-28.99 \pm   0.01$& $-28.59 \pm   0.01$& $-28.29 \pm   0.01$& $-27.61 \pm   0.01$ \\
60 & $ 87.0 \pm 1.4$ & $ 20.7 \pm 0.7$& $ 6132 \pm   2$ & $-28.78 \pm   0.01$ & $-28.66 \pm   0.01$& $-28.30 \pm   0.01$& $-28.08 \pm   0.01$& $-27.41 \pm   0.01$ \\
61 & $120.6 \pm 1.1$ & $  9.3 \pm 0.4$& $ 6132 \pm   4$ & $-29.72 \pm   0.04$ & $-29.43 \pm   0.03$& $-28.99 \pm   0.03$& $-28.64 \pm   0.03$& $-27.90 \pm   0.03$ \\
62 & $102.4 \pm 13.4$ & $ 19.2 \pm 0.6$& $ 6112 \pm   3$ & $-29.46 \pm   0.01$ & $-29.17 \pm   0.01$& $-28.71 \pm   0.01$& $-28.47 \pm   0.01$& $-27.83 \pm   0.01$ \\
63 & $141.1 \pm 3.4$ & $ 11.7 \pm 1.2$& $ 6128 \pm   3$ & $-29.28 \pm   0.15$ & $-29.16 \pm   0.15$& $-28.72 \pm   0.15$& $-28.41 \pm   0.15$& $-27.65 \pm   0.15$ \\
64 & $129.6 \pm 2.7$ & $ 17.3 \pm 1.0$& $ 6100 \pm   2$ & $-28.91 \pm   0.01$ & $-28.84 \pm   0.01$& $-28.46 \pm   0.01$& $-28.18 \pm   0.01$& $-27.43 \pm   0.01$ \\
65 & $146.8 \pm 3.5$ & $  7.5 \pm 2.5$& $ 6108 \pm   4$ & $-28.79 \pm   0.02$ & $-28.65 \pm   0.02$& $-28.35 \pm   0.02$& $-28.12 \pm   0.02$& $-27.46 \pm   0.02$ \\
66 & $117.4 \pm 2.2$ & $  9.0 \pm 1.2$& $ 6135 \pm   3$ & $-29.50 \pm   0.01$ & $-29.23 \pm   0.01$& $-28.80 \pm   0.01$& $-28.47 \pm   0.01$& $-27.74 \pm   0.01$ \\
67 & $159.1 \pm 0.7$ & $ 15.7 \pm 0.8$& $ 6130 \pm   1$ & $-29.37 \pm   0.02$ & $-29.16 \pm   0.02$& $-28.68 \pm   0.02$& $-28.41 \pm   0.02$& $-27.69 \pm   0.02$ \\
68 & $ 91.5 \pm 10.9$ & $ 15.6 \pm 0.5$& $ 6127 \pm   2$ & $-29.43 \pm   0.02$ & $-29.09 \pm   0.02$& $-28.60 \pm   0.01$& $-28.30 \pm   0.01$& $-27.62 \pm   0.04$ \\
69 & $107.8 \pm 14.1$ & $ 20.0 \pm 0.7$& $ 6132 \pm   3$ & $-28.82 \pm   0.02$ & $-28.78 \pm   0.01$& $-28.32 \pm   0.01$& $-28.09 \pm   0.01$& $-27.39 \pm   0.01$ \\
70 & $ 92.1 \pm 16.5$ & $ 21.1 \pm 0.7$& $ 6118 \pm   2$ & $-29.28 \pm   0.02$ & $-28.95 \pm   0.02$& $-28.46 \pm   0.02$& $-28.20 \pm   0.02$& $-27.62 \pm   0.04$ \\
71 & $128.6 \pm 2.8$ & $ 15.6 \pm 1.7$& $ 6150 \pm   4$ & $-29.35 \pm   0.02$ & $-29.11 \pm   0.02$& $-28.61 \pm   0.02$& $-28.32 \pm   0.02$& $-27.67 \pm   0.02$ \\
72 & $102.6 \pm 2.8$ & $ 15.6 \pm 1.1$& $ 6141 \pm   8$ & $-29.16 \pm   0.07$ & $-29.04 \pm   0.06$& $-28.56 \pm   0.06$& $-28.30 \pm   0.06$& $-27.63 \pm   0.06$ \\
73 & $120.0 \pm 1.1$ & $  7.3 \pm 0.6$& $ 6123 \pm   3$ & $-29.64 \pm   0.02$ & $-29.41 \pm   0.02$& $-28.99 \pm   0.02$& $-28.70 \pm   0.02$& $-28.00 \pm   0.02$ \\
74 & $130.0 \pm 1.0$ & $  6.1 \pm 4.2$& $ 6109 \pm   3$ & $-29.42 \pm   0.01$ & $-29.24 \pm   0.01$& $-28.84 \pm   0.01$& $-28.52 \pm   0.01$& $-27.76 \pm   0.01$ \\
75 & $111.9 \pm 1.3$ & $  9.8 \pm 0.7$& $ 6142 \pm   2$ & $-29.63 \pm   0.01$ & $-29.36 \pm   0.01$& $-28.88 \pm   0.01$& $-28.54 \pm   0.01$& $-27.81 \pm   0.01$ \\
76 & $135.2 \pm 1.4$ & $  5.6 \pm 1.6$& $ 6084 \pm   3$ & $-29.44 \pm   0.02$ & $-29.19 \pm   0.02$& $-28.79 \pm   0.02$& $-28.49 \pm   0.02$& $-27.74 \pm   0.02$ \\
77 & $ 65.8 \pm 4.1$ & $ 10.2 \pm 0.9$& $ 6133 \pm   2$ & $-29.53 \pm   0.02$ & $-29.30 \pm   0.02$& $-28.89 \pm   0.02$& $-28.56 \pm   0.02$& $-27.90 \pm   0.02$ \\
78 & $ 87.8 \pm 7.2$ & $ 21.5 \pm 0.9$& $ 6137 \pm   3$ & $-29.00 \pm   0.22$ & $-28.82 \pm   0.22$& $-28.37 \pm   0.22$& $-28.13 \pm   0.22$& $-27.43 \pm   0.22$ \\
79 & $138.1 \pm 2.1$ & $ 11.7 \pm 0.7$& $ 6133 \pm   3$ & $-29.40 \pm   0.02$ & $-29.23 \pm   0.02$& $-28.77 \pm   0.01$& $-28.45 \pm   0.02$& $-27.68 \pm   0.02$ \\
80 & $119.9 \pm 10.4$ & $ 16.4 \pm 1.7$& $ 6132 \pm   6$ & $-29.05 \pm   0.07$ & $-28.92 \pm   0.07$& $-28.50 \pm   0.07$& $-28.22 \pm   0.07$& $-27.55 \pm   0.07$ \\
81 & $101.8 \pm 2.6$ & $ 12.4 \pm 1.0$& $ 6158 \pm   5$ & $-29.46 \pm   0.02$ & $-29.27 \pm   0.02$& $-28.87 \pm   0.02$& $-28.60 \pm   0.02$& $-27.96 \pm   0.02$ \\
82 & $128.5 \pm 1.4$ & $ 10.6 \pm 1.0$& $ 6098 \pm   4$ & $-29.48 \pm   0.02$ & $-29.31 \pm   0.02$& $-28.87 \pm   0.01$& $-28.54 \pm   0.01$& $-27.77 \pm   0.02$ \\
83 & $114.4 \pm 1.2$ & $ 12.6 \pm 0.6$& $ 6136 \pm   3$ & $-29.59 \pm   0.23$ & $-29.34 \pm   0.23$& $-28.91 \pm   0.23$& $-28.58 \pm   0.23$& $-27.87 \pm   0.23$ \\
84 & $ 93.7 \pm 1.0$ & $  7.8 \pm 0.4$& $ 6171 \pm   2$ & $-29.38 \pm   0.02$ & $-29.12 \pm   0.01$& $-28.74 \pm   0.01$& $-28.44 \pm   0.01$& $-27.78 \pm   0.01$ \\
85 & $ 79.4 \pm 3.2$ & $ 12.6 \pm 0.7$& $ 6144 \pm   2$ & $-28.90 \pm   0.02$ & $-28.62 \pm   0.01$& $-28.27 \pm   0.01$& $-28.08 \pm   0.01$& $-27.54 \pm   0.01$ \\
86 & $ 39.0 \pm 22.2$ & $  4.5 \pm 0.2$& $ 6139 \pm   3$ & $-29.46 \pm   0.02$ & $-29.06 \pm   0.02$& $-28.75 \pm   0.02$& $-28.49 \pm   0.02$& $-27.93 \pm   0.02$ \\
87 & $ 75.6 \pm 1.1$ & $  2.2 \pm 0.4$& $ 6142 \pm   3$ & $-29.92 \pm   0.32$ & $-29.49 \pm   0.32$& $-29.02 \pm   0.32$& $-28.63 \pm   0.32$& $-27.90 \pm   0.32$ \\
88 & $ 98.8 \pm 12.1$ & $ 25.9 \pm 0.7$& $ 6114 \pm   7$ & $-29.14 \pm   0.02$ & $-28.91 \pm   0.02$& $-28.44 \pm   0.02$& $-28.21 \pm   0.02$& $-27.49 \pm   0.02$ \\
89 & $ 81.3 \pm 2.3$ & $  6.3 \pm 0.9$& $ 6142 \pm   5$ & $-29.89 \pm   0.02$ & $-29.46 \pm   0.02$& $-28.97 \pm   0.02$& $-28.62 \pm   0.02$& $-27.97 \pm   0.02$ \\
90 & $129.5 \pm 5.6$ & $  6.3 \pm 2.5$& $ 6110 \pm   6$ & $-29.53 \pm   0.02$ & $-29.31 \pm   0.02$& $-28.95 \pm   0.02$& $-28.60 \pm   0.02$& $-27.96 \pm   0.02$ \\
91 & $106.5 \pm 1.8$ & $ 21.4 \pm 1.7$& $ 6162 \pm   4$ & $-28.98 \pm   0.02$ & $-28.92 \pm   0.02$& $-28.66 \pm   0.02$& $-28.42 \pm   0.02$& $-27.77 \pm   0.02$ \\
92 & $109.1 \pm 9.4$ & $ 15.0 \pm 0.6$& $ 6134 \pm   2$ & $-29.36 \pm   0.02$ & $-29.14 \pm   0.01$& $-28.69 \pm   0.01$& $-28.41 \pm   0.01$& $-27.70 \pm   0.01$ \\
93 & $ 97.7 \pm 2.8$ & $ 12.6 \pm 1.4$& $ 6142 \pm   5$ & $-29.50 \pm   0.02$ & $-29.11 \pm   0.02$& $-28.61 \pm   0.02$& $-28.34 \pm   0.02$& $-27.71 \pm   0.02$ \\
94 & $102.7 \pm 1.8$ & $ 11.9 \pm 1.0$& $ 6137 \pm   4$ & $-29.28 \pm   0.02$ & $-29.14 \pm   0.02$& $-28.74 \pm   0.02$& $-28.46 \pm   0.01$& $-27.76 \pm   0.03$ \\
95 & $133.6 \pm 2.1$ & $  6.9 \pm 0.9$& $ 6127 \pm   3$ & $-29.09 \pm   0.01$ & $-28.93 \pm   0.01$& $-28.59 \pm   0.01$& $-28.31 \pm   0.01$& $-27.55 \pm   0.03$ \\
96 & $127.8 \pm 2.5$ & $ 15.8 \pm 1.3$& $ 6117 \pm   4$ & $-28.80 \pm   0.02$ & $-28.68 \pm   0.02$& $-28.37 \pm   0.02$& $-28.18 \pm   0.02$& $-27.55 \pm   0.02$ \\
97 & $134.2 \pm 0.5$ & $  7.4 \pm 0.2$& $ 6097 \pm   1$ & $-29.91 \pm   0.01$ & $-29.70 \pm   0.01$& $-29.31 \pm   0.01$& $-28.94 \pm   0.01$& $-28.10 \pm   0.01$ \\
98 & $100.2 \pm 2.8$ & $ 13.4 \pm 1.1$& $ 6107 \pm   8$ & $-29.33 \pm   0.02$ & $-29.11 \pm   0.02$& $-28.77 \pm   0.02$& $-28.52 \pm   0.02$& $-27.80 \pm   0.02$ \\
99 & $ 88.4 \pm 1.2$ & $  5.4 \pm 0.4$& $ 6168 \pm   2$ & $-29.33 \pm   0.01$ & $-29.05 \pm   0.01$& $-28.68 \pm   0.01$& $-28.40 \pm   0.01$& $-27.79 \pm   0.01$ \\
100 & $125.2 \pm 4.6$ & $ 17.1 \pm 2.0$& $ 6124 \pm  11$ & $-28.58 \pm   0.02$ & $-28.54 \pm   0.02$& $-28.19 \pm   0.02$& $-27.99 \pm   0.02$& $-27.31 \pm   0.02$ \\
101 & $141.7 \pm 2.6$ & $  7.7 \pm 1.0$& $ 6134 \pm   5$ & $-29.38 \pm   0.02$ & $-29.20 \pm   0.02$& $-28.77 \pm   0.02$& $-28.46 \pm   0.02$& $-27.67 \pm   0.02$ \\
102 & $ 22.7 \pm 21.4$ & $  0.9 \pm 0.6$& $ 6197 \pm   7$ & $-29.87 \pm   0.01$ & $-29.43 \pm   0.01$& $-29.04 \pm   0.01$& $-28.74 \pm   0.01$& $-28.11 \pm   0.01$ \\
103 & $127.2 \pm 1.9$ & $ 21.7 \pm 1.2$& $ 6134 \pm   3$ & $-29.14 \pm   0.01$ & $-29.00 \pm   0.01$& $-28.59 \pm   0.01$& $-28.35 \pm   0.01$& $-27.65 \pm   0.01$ \\
104 & $162.3 \pm 9.8$ & $ 19.6 \pm 1.4$& $ 6076 \pm   6$ & $-28.86 \pm   0.02$ & $-28.87 \pm   0.01$& $-28.42 \pm   0.01$& $-28.19 \pm   0.01$& $-27.50 \pm   0.03$ \\
105 & $138.5 \pm 4.0$ & $ 14.0 \pm 2.0$& $ 6111 \pm   2$ & $-27.59 \pm   0.02$ & $-27.78 \pm   0.01$& $-27.73 \pm   0.01$& $-27.71 \pm   0.01$& $-27.19 \pm   0.02$ \\
106 & $ 88.4 \pm 21.6$ & $ 22.3 \pm 1.1$& $ 6141 \pm   3$ & $-29.11 \pm   0.01$ & $-28.89 \pm   0.01$& $-28.45 \pm   0.01$& $-28.23 \pm   0.01$& $-27.60 \pm   0.01$ \\
107 & $109.4 \pm 2.2$ & $ 11.5 \pm 1.7$& $ 6142 \pm   6$ & $-29.35 \pm   0.13$ & $-29.17 \pm   0.13$& $-28.68 \pm   0.13$& $-28.38 \pm   0.13$& $-27.56 \pm   0.13$ \\
108 & $103.7 \pm 17.0$ & $ 23.1 \pm 0.4$& $ 6118 \pm   2$ & $-28.67 \pm   0.01$ & $-28.52 \pm   0.01$& $-28.18 \pm   0.01$& $-28.03 \pm   0.01$& $-27.47 \pm   0.01$ \\
109 & $ 93.8 \pm 12.2$ & $ 23.8 \pm 3.0$& $ 6122 \pm   4$ & $-28.87 \pm   0.02$ & $-28.70 \pm   0.01$& $-28.26 \pm   0.01$& $-28.08 \pm   0.01$& $-27.41 \pm   0.02$ \\
110 & $122.5 \pm 3.8$ & $ 12.7 \pm 1.6$& $ 6131 \pm   4$ & $-29.10 \pm   0.01$ & $-28.96 \pm   0.01$& $-28.51 \pm   0.01$& $-28.25 \pm   0.01$& $-27.57 \pm   0.01$ \\
111 & $101.7 \pm 1.3$ & $  4.8 \pm 0.8$& $ 6126 \pm   2$ & $-29.43 \pm   0.01$ & $-29.14 \pm   0.01$& $-28.74 \pm   0.01$& $-28.43 \pm   0.01$& $-27.76 \pm   0.01$ \\
112 & $ 89.8 \pm 9.9$ & $ 23.4 \pm 1.1$& $ 6109 \pm   5$ & $-29.11 \pm   0.02$ & $-28.91 \pm   0.02$& $-28.50 \pm   0.02$& $-28.27 \pm   0.02$& $-27.57 \pm   0.02$ \\
113 & $ 93.9 \pm 2.4$ & $ 18.5 \pm 1.3$& $ 6108 \pm   4$ & $-29.19 \pm   0.02$ & $-29.03 \pm   0.01$& $-28.59 \pm   0.01$& $-28.30 \pm   0.01$& $-27.55 \pm   0.02$ \\
114 & $ 61.8 \pm 6.5$ & $ 27.1 \pm 1.8$& $ 6154 \pm   8$ & $-28.37 \pm   0.02$ & $-28.27 \pm   0.02$& $-27.97 \pm   0.02$& $-27.80 \pm   0.02$& $-27.24 \pm   0.02$ \\
115 & $ 64.6 \pm 3.8$ & $  3.7 \pm 1.4$& $ 6147 \pm   4$ & $-29.49 \pm   0.02$ & $-29.06 \pm   0.02$& $-28.71 \pm   0.02$& $-28.42 \pm   0.02$& $-27.93 \pm   0.02$ \\
116 & $127.3 \pm 2.4$ & $ 15.5 \pm 1.2$& $ 6127 \pm   2$ & $-28.77 \pm   0.01$ & $-28.66 \pm   0.01$& $-28.37 \pm   0.01$& $-28.15 \pm   0.01$& $-27.52 \pm   0.01$ \\
117 & $ 77.5 \pm 2.5$ & $  5.1 \pm 0.8$& $ 6140 \pm   4$ & $-29.96 \pm   0.06$ & $-29.56 \pm   0.06$& $-29.06 \pm   0.06$& $-28.75 \pm   0.06$& $-28.01 \pm   0.06$ \\
118 & $105.5 \pm 3.0$ & $  6.2 \pm 1.2$& $ 6120 \pm   5$ & $-29.54 \pm   0.01$ & $-29.23 \pm   0.01$& $-28.74 \pm   0.01$& $-28.40 \pm   0.01$& $-27.64 \pm   0.01$ \\
119 & $107.2 \pm 3.5$ & $ 18.6 \pm 1.3$& $ 6132 \pm  14$ & $-28.80 \pm   0.02$ & $-28.69 \pm   0.02$& $-28.34 \pm   0.02$& $-28.13 \pm   0.02$& $-27.52 \pm   0.02$ \\
120 & $ 99.1 \pm 2.7$ & $  5.8 \pm 0.8$& $ 6164 \pm   5$ & $-29.67 \pm   0.01$ & $-29.34 \pm   0.01$& $-28.94 \pm   0.01$& $-28.60 \pm   0.01$& $-27.96 \pm   0.01$ \\
121 & $126.5 \pm 1.2$ & $ 15.4 \pm 1.1$& $ 6155 \pm   2$ & $-28.67 \pm   0.02$ & $-28.60 \pm   0.02$& $-28.31 \pm   0.02$& $-28.13 \pm   0.02$& $-27.58 \pm   0.02$ \\
122 & $ 87.7 \pm 3.6$ & $  7.3 \pm 1.3$& $ 6132 \pm   3$ & $-28.60 \pm   0.02$ & $-28.42 \pm   0.02$& $-28.19 \pm   0.02$& $-28.05 \pm   0.02$& $-27.53 \pm   0.02$ \\
123 & $121.0 \pm 5.3$ & $  9.3 \pm 3.1$& $ 6126 \pm   3$ & $-29.75 \pm   0.01$ & $-29.42 \pm   0.01$& $-28.99 \pm   0.01$& $-28.68 \pm   0.01$& $-27.97 \pm   0.01$ \\
124 & $133.0 \pm 1.5$ & $ 17.6 \pm 0.8$& $ 6112 \pm   4$ & $-29.12 \pm   0.01$ & $-28.98 \pm   0.01$& $-28.54 \pm   0.01$& $-28.28 \pm   0.01$& $-27.51 \pm   0.01$ \\
125 & $106.4 \pm 2.1$ & $ 26.7 \pm 1.3$& $ 6100 \pm   2$ & $-29.00 \pm   0.01$ & $-28.83 \pm   0.01$& $-28.42 \pm   0.01$& $-28.20 \pm   0.01$& $-27.53 \pm   0.04$ \\
126 & $110.3 \pm 1.6$ & $ 14.2 \pm 0.7$& $ 6133 \pm   3$ & $-29.34 \pm   0.01$ & $-29.04 \pm   0.01$& $-28.57 \pm   0.01$& $-28.32 \pm   0.01$& $-27.66 \pm   0.01$ \\
127 & $ 99.5 \pm 1.6$ & $ 30.0 \pm 0.7$& $ 6114 \pm   2$ & $-28.81 \pm   0.01$ & $-28.65 \pm   0.01$& $-28.23 \pm   0.01$& $-28.02 \pm   0.01$& $-27.33 \pm   0.01$ \\
128 & $138.0 \pm 5.1$ & $ 16.2 \pm 1.6$& $ 6111 \pm   3$ & $-27.84 \pm   0.02$ & $-27.92 \pm   0.02$& $-27.69 \pm   0.02$& $-27.60 \pm   0.02$& $-26.99 \pm   0.02$ \\
129 & $ 80.8 \pm 2.4$ & $  4.3 \pm 0.8$& $ 6139 \pm   3$ & $-29.54 \pm   0.02$ & $-29.16 \pm   0.01$& $-28.87 \pm   0.01$& $-28.54 \pm   0.01$& $-28.01 \pm   0.01$ \\
130 & $124.2 \pm 2.4$ & $  6.8 \pm 1.1$& $ 6120 \pm   2$ & $-29.52 \pm   0.01$ & $-29.25 \pm   0.01$& $-28.80 \pm   0.01$& $-28.49 \pm   0.01$& $-27.76 \pm   0.01$ \\
131 & $103.6 \pm 7.2$ & $ 27.2 \pm 1.9$& $ 6120 \pm   5$ & $-28.74 \pm   0.02$ & $-28.46 \pm   0.01$& $-28.09 \pm   0.01$& $-27.87 \pm   0.01$& $-27.26 \pm   0.04$ \\
132 & $ 85.5 \pm 0.6$ & $  5.9 \pm 0.3$& $ 6130 \pm   4$ & $-29.87 \pm   0.02$ & $-29.45 \pm   0.02$& $-28.94 \pm   0.02$& $-28.57 \pm   0.02$& $-27.85 \pm   0.02$ \\
133 & $118.1 \pm 2.1$ & $  4.6 \pm 2.2$& $ 6101 \pm   3$ & $-29.77 \pm   0.02$ & $-29.52 \pm   0.02$& $-29.08 \pm   0.02$& $-28.74 \pm   0.01$& $-27.94 \pm   0.02$ \\
134 & $131.4 \pm 2.2$ & $ 17.4 \pm 1.1$& $ 6116 \pm   4$ & $-28.60 \pm   0.02$ & $-28.55 \pm   0.02$& $-28.25 \pm   0.02$& $-28.03 \pm   0.02$& $-27.34 \pm   0.02$ \\
135 & $ 94.1 \pm 2.0$ & $ 11.2 \pm 0.6$& $ 6131 \pm   3$ & $-29.50 \pm   0.02$ & $-29.11 \pm   0.02$& $-28.67 \pm   0.02$& $-28.37 \pm   0.02$& $-27.71 \pm   0.02$ \\
136 & $126.9 \pm 2.5$ & $ 10.0 \pm 0.7$& $ 6121 \pm   2$ & $-29.38 \pm   0.01$ & $-29.13 \pm   0.01$& $-28.69 \pm   0.01$& $-28.37 \pm   0.01$& $-27.61 \pm   0.01$ \\
137 & $106.9 \pm 0.6$ & $  7.1 \pm 0.7$& $ 6134 \pm   3$ & $-29.29 \pm   0.73$ & $-29.00 \pm   0.73$& $-28.51 \pm   0.73$& $-28.15 \pm   0.73$& $-27.38 \pm   0.73$ \\
138 & $ 82.6 \pm 17.5$ & $ 12.2 \pm 1.4$& $ 6139 \pm   3$ & $-29.51 \pm   0.02$ & $-29.14 \pm   0.01$& $-28.60 \pm   0.01$& $-28.30 \pm   0.01$& $-27.71 \pm   0.02$ \\
139 & $ 78.9 \pm 20.2$ & $ 21.1 \pm 1.4$& $ 6128 \pm   6$ & $-29.37 \pm   0.05$ & $-29.04 \pm   0.05$& $-28.54 \pm   0.05$& $-28.30 \pm   0.05$& $-27.57 \pm   0.05$ \\
140 & $125.5 \pm 3.2$ & $ 10.1 \pm 1.6$& $ 6121 \pm   3$ & $-29.34 \pm   0.01$ & $-29.12 \pm   0.01$& $-28.65 \pm   0.01$& $-28.38 \pm   0.01$& $-27.66 \pm   0.01$ \\
141 & $132.6 \pm 1.4$ & $ 15.2 \pm 1.0$& $ 6113 \pm   2$ & $-29.12 \pm   0.01$ & $-28.95 \pm   0.01$& $-28.53 \pm   0.01$& $-28.27 \pm   0.01$& $-27.55 \pm   0.01$ \\
142 & $ 95.9 \pm 2.3$ & $ 10.6 \pm 1.1$& $ 6143 \pm   6$ & $-29.59 \pm   0.02$ & $-29.22 \pm   0.02$& $-28.68 \pm   0.02$& $-28.42 \pm   0.02$& $-27.77 \pm   0.02$ \\
143 & $167.5 \pm 2.2$ & $ 20.4 \pm 0.6$& $ 6090 \pm   2$ & $-28.75 \pm   0.02$ & $-28.78 \pm   0.02$& $-28.40 \pm   0.02$& $-28.18 \pm   0.02$& $-27.41 \pm   0.02$ \\
144 & $112.1 \pm 2.5$ & $  9.8 \pm 1.0$& $ 6118 \pm   3$ & $-29.34 \pm   0.01$ & $-29.09 \pm   0.01$& $-28.68 \pm   0.01$& $-28.39 \pm   0.01$& $-27.70 \pm   0.01$ \\
145 & $135.8 \pm 1.8$ & $  7.5 \pm 0.9$& $ 6135 \pm   4$ & $-29.22 \pm   0.02$ & $-29.02 \pm   0.02$& $-28.61 \pm   0.02$& $-28.35 \pm   0.01$& $-27.71 \pm   0.02$ \\
146 & $117.6 \pm 2.6$ & $  8.9 \pm 2.0$& $ 6119 \pm   3$ & $-28.84 \pm   0.01$ & $-28.70 \pm   0.01$& $-28.35 \pm   0.01$& $-28.16 \pm   0.01$& $-27.50 \pm   0.01$ \\
147 & $ 78.5 \pm 1.8$ & $  6.3 \pm 0.9$& $ 6140 \pm   3$ & $-29.71 \pm   0.01$ & $-29.34 \pm   0.01$& $-28.93 \pm   0.01$& $-28.61 \pm   0.01$& $-27.92 \pm   0.01$ \\
148 & $137.2 \pm 2.5$ & $  9.3 \pm 1.1$& $ 6112 \pm   7$ & $-29.27 \pm   0.03$ & $-29.09 \pm   0.03$& $-28.73 \pm   0.03$& $-28.38 \pm   0.03$& $-27.68 \pm   0.03$ \\
149 & $ 99.3 \pm 3.6$ & $ 16.8 \pm 0.7$& $ 6129 \pm   3$ & $-28.29 \pm   0.02$ & $-28.27 \pm   0.01$& $-28.05 \pm   0.01$& $-27.95 \pm   0.01$& $-27.40 \pm   0.01$ \\
150 & $122.4 \pm 2.7$ & $ 21.2 \pm 0.8$& $ 6114 \pm   4$ & $-29.07 \pm   0.01$ & $-28.94 \pm   0.01$& $-28.50 \pm   0.01$& $-28.26 \pm   0.01$& $-27.53 \pm   0.03$ \\
151 & $112.2 \pm 2.7$ & $  7.8 \pm 1.0$& $ 6143 \pm   4$ & $-29.42 \pm   0.01$ & $-29.17 \pm   0.01$& $-28.78 \pm   0.01$& $-28.46 \pm   0.01$& $-27.78 \pm   0.01$ \\
152 & $104.5 \pm 5.5$ & $ 24.4 \pm 2.2$& $ 6123 \pm   9$ & $-29.20 \pm   0.05$ & $-29.01 \pm   0.05$& $-28.48 \pm   0.05$& $-28.23 \pm   0.05$& $-27.54 \pm   0.05$ \\
153 & $143.3 \pm 1.6$ & $ 18.9 \pm 1.3$& $ 6103 \pm   4$ & $-28.78 \pm   0.02$ & $-28.79 \pm   0.02$& $-28.44 \pm   0.02$& $-28.18 \pm   0.02$& $-27.45 \pm   0.02$ \\
154 & $ 61.8 \pm 3.5$ & $  3.3 \pm 0.9$& $ 6149 \pm   4$ & $-29.86 \pm   0.01$ & $-29.41 \pm   0.01$& $-29.03 \pm   0.01$& $-28.70 \pm   0.01$& $-28.06 \pm   0.01$ \\
155 & $ 73.9 \pm 2.4$ & $ 12.8 \pm 0.8$& $ 6126 \pm   2$ & $-29.29 \pm   0.02$ & $-28.94 \pm   0.02$& $-28.53 \pm   0.01$& $-28.35 \pm   0.01$& $-27.76 \pm   0.01$ \\
156 & $ 95.4 \pm 41.8$ & $ 35.7 \pm 2.8$& $ 6135 \pm   4$ & $-28.71 \pm   0.01$ & $-28.58 \pm   0.01$& $-28.19 \pm   0.01$& $-27.99 \pm   0.01$& $-27.34 \pm   0.01$ \\
157 & $122.8 \pm 2.3$ & $  9.7 \pm 1.7$& $ 6124 \pm   2$ & $-29.04 \pm   0.02$ & $-28.86 \pm   0.02$& $-28.56 \pm   0.02$& $-28.30 \pm   0.02$& $-27.64 \pm   0.02$ \\
158 & $111.6 \pm 2.6$ & $  6.4 \pm 1.1$& $ 6116 \pm   3$ & $-29.41 \pm   0.01$ & $-29.15 \pm   0.01$& $-28.70 \pm   0.01$& $-28.38 \pm   0.01$& $-27.73 \pm   0.04$ \\
159 & $ 98.1 \pm 1.6$ & $ 10.6 \pm 2.1$& $ 6146 \pm   4$ & $-29.23 \pm   0.01$ & $-29.05 \pm   0.01$& $-28.71 \pm   0.01$& $-28.45 \pm   0.01$& $-27.79 \pm   0.01$ \\
160 & $118.2 \pm 1.5$ & $ 11.3 \pm 1.8$& $ 6157 \pm   5$ & $-29.13 \pm   0.08$ & $-29.01 \pm   0.07$& $-28.62 \pm   0.07$& $-28.32 \pm   0.07$& $-27.68 \pm   0.07$ \\
161 & $118.8 \pm 2.1$ & $ 15.3 \pm 1.3$& $ 6112 \pm   3$ & $-29.35 \pm   0.01$ & $-29.15 \pm   0.01$& $-28.69 \pm   0.01$& $-28.38 \pm   0.01$& $-27.68 \pm   0.01$ \\
162 & $108.8 \pm 2.7$ & $ 13.5 \pm 0.9$& $ 6153 \pm   4$ & $-29.44 \pm   0.02$ & $-29.19 \pm   0.01$& $-28.74 \pm   0.01$& $-28.42 \pm   0.01$& $-27.78 \pm   0.01$ \\\enddata
\end{deluxetable}

Absolute photometric calibration is of extreme importance for the SNfactory.  This analysis, however, is confined to the relative colors for a low-redshift set of supernovae
and is thus insensitive to uncertainties in these zeropoints.


\section{Supernova Model}
\label{model:sec}
We hypothesize that at peak brightness
SN~Ia broadband magnitudes and colors are correlated with
spectral features: equivalent widths and line velocities are considered as statistics localized in wavelength that are insensitive to variations in
broadband colors.
Spectra may not efficiently capture all variations in broadband magnitude, so an additional source
of intrinsic variation is accommodated.
The intrinsic spectral and color parameters may not deterministically predict magnitudes, but rather do so with some dispersion.
An extrinsic physical process (i.e.\ dust) subsequently determines the apparent magnitudes.

We assume 
that  peak intrinsic $UBVRI$ magnitudes are linearly dependent
on the equivalent widths of the CaII H\&K and SiII$\lambda$4141 spectral features
$EW_{Ca}$, $EW_{Si}$,
the SiII$\lambda6355$ position $\lambda_{Si}$ (so also linear with $v_{Si}$):
these spectral features are associated with SN~Ia  spectroscopic diversity  
\citep{2006PASP..118..560B,2009A&A...500L..17B, 2009PASP..121..238B, 2009ApJ...699L.139W, 2011ApJ...729...55F}.
$EW_{Si}$ at peak brightness is an effective proxy for the light-curve shape parameter
\citep{2011A&A...529L...4C}. 
An intrinsic-color parameter $D$ describes  intrinsic magnitude/color variation unaccounted for by the spectral features.
Residual dispersion is described by a Normal distribution with a parameterized covariance matrix
$C_c$.  A grey magnitude offset $\Delta$ is included for each supernova
to capture intrinsic dispersion and peculiar-velocity errors introduced when converting fluxes to luminosity.
The observed magnitudes are linearly dependent on an
extrinsic-color parameter $k$.  The observables
$U_o, B_o, V_o, R_o, I_o$, $EW_{Ca,o}$, $EW_{Si,o}$, $v_{Si,o}$
shown in Table~\ref{data:tab} have Gaussian measurement uncertainty with covariance $C$.
The model is written as
\begin{equation}
\begin{pmatrix}
U\\B\\V\\R\\I
\end{pmatrix}
\sim \mathcal{N}
\left(
\Delta +
\begin{pmatrix}
c_0+\alpha_0 EW_{Ca} + \beta_0 EW_{Si} + \eta_0 \lambda_{Si} + \delta_0 D\\
c_1+\alpha_1 EW_{Ca} + \beta_1 EW_{Si} + \eta_1 \lambda_{Si} + \delta_1 D \\
c_2+\alpha_2 EW_{Ca} + \beta_2 EW_{Si} + \eta_2 \lambda_{Si} + \delta_2 D\\
c_3+\alpha_3 EW_{Ca} + \beta_3 EW_{Si} + \eta_3 \lambda_{Si} + \delta_3 D\\
c_4+\alpha_4 EW_{Ca} + \beta_4 EW_{Si}+ \eta_4 \lambda_{Si} + \delta_4 D
\end{pmatrix}
,C_{c}
\right)
\label{ewsiv:eqn}
\end{equation}
\begin{equation}
\begin{pmatrix}
U_o\\B_o\\ V_o\\R_o\\I_o\\EW_{Si, o}\\ EW_{Ca, o} \\ v_{Si, o}
\end{pmatrix}
\sim \mathcal{N}
\left(
\begin{pmatrix}
U +\gamma_0 k \\B +\gamma_1 k \\V+\gamma_2 k\\R+\gamma_3 k\\I+\gamma_4 k\\
EW_{Si}\\ EW_{Ca} \\ \lambda_{Si}
\end{pmatrix}
,C
\right).
\label{dust:eqn}
\end{equation}
The global parameters $c$, $\alpha$, $\beta$, $\eta$, and $\delta$  are the intercepts and slopes of the linear relationships that
relate intrinsic magnitudes with the spectral-feature parameters and intrinsic-color parameter.
(In this article we freely interchange the parameter indices with  $01234$ or $UBVRI$.)
The global parameters $\gamma$ are the slopes that connect the extrinsic-color
parameter and observed magnitudes.  To constrain the degrees of freedom and degeneracies inherent in the model we impose that
\color{red}
each of the vectors $k+1/N$ and $D+1/N$ are simplexes, where $N$ are the number of objects in our sample: these
conditions break the 
degeneracies  $\gamma \rightarrow a\gamma$, $k \rightarrow a^{-1}k$ and $\delta \rightarrow a\delta$, $D \rightarrow a^{-1}D$.
Furthermore we impose
\color{black} 
\begin{equation}
\langle \Delta \rangle=0, \langle k \rangle=0, \langle D \rangle=0, \gamma_0 > 0, \delta_0 < 0.
\end{equation}
The first two specify the definition of zero color relative to which the color excess is measured.    The latter two exclude degenerate posterior space
associated
with  simultaneous sign flips of
$\delta$--$D$ and $\gamma$--$k$
\color{red}
(when running our fits, $k$ and $D$ are set to
identical initial conditions, so the difference in the signs of $\gamma_0$ and $\delta_0$ is non-trivial.)
Although not formally a degeneracy in the model, practically the parameter pairs  $\delta$--$D$ and $\gamma$--$k$ are degenerate with
each other, as is apparent after 
running the Monte Carlo with exchange of the initial  and boundary conditions of $\delta$ and $\gamma$.
The different initial and boundary conditions for $\gamma_0$ and $\delta_0$ break the degeneracy of our mixture model;
otherwise identical initial conditions for $\gamma$, $\delta$ produce indistinguishable posteriors
with relatively broader credible intervals and lower maximum likelihood.
\color{black}
As the model cannot distinguish which effect is intrinsic or extrinsic, we do so
by assigning the parameters whose values are consistent with dust to $\gamma$ and the others to $\delta$.



Having the intrinsic dispersion $C_c$ as fit parameters seemingly introduces degeneracy in the model, as magnitude and color variation
ascribed to $\Delta$, $\delta D$, and $\gamma k$ could also be attributed to intrinsic dispersion.  There are several features of the model
that drives the assignation of variations away from $C_c$:  Maximizing the posterior disfavors the increase of $\det{(C_c)}$;
The distributions of $\Delta$ and $k$ turn out to
be non-Gaussian, and so are not well described by a Normal covariance; Grey offsets would appear as a constant
in all elements of the covariance matrix, which is disfavored for the Bayesian prior selected for $C_c$.

When interpreted as being due to dust, the extrinsic terms are associated with extinction $A_X = \gamma_X k$.  In
this parlance our parameters can be re-expressed as the  familiar color excess
$E_\gamma(B-V) = (\gamma_1-\gamma_2) k$ and total-to-selective extinction $R_X = \frac{\gamma_X}{\gamma_1-\gamma_2}$.
Similar reparameterizatons of the intrinsic color parameters are
$E_\delta(B-V) = (\delta_1-\delta_2) D$ and $R_{\delta X} = \frac{\delta_X}{\delta_1-\delta_2}$.

In a Bayesian analysis the priors must be described.  A flat prior is used for all parameters except
for the covariance matrix $C_c$, which is constructed from a correlation matrix with  $\nu=4$  LKJ prior\footnote{
Visualization of the LKJ correlation distribution can be found in \url{http://www.psychstatistics.com/2014/12/27/d-lkj-priors/}.}
\citep{Lewandowski20091989} and standard
deviations $\sigma_i = \sqrt{C_{c,ii}}$ with a  Cauchy distribution prior with location
 $0.1$ and scale $0.1$ mag restricted to positive values.  (We find that imposing a stricter assumption of a
 diagonal covariance matrix, with no reference to a correlation matrix with LKJ prior, yields the same parameter distributions within
 uncertainty).

\section{Results}
\label{results:sec}
The posterior of the model parameters is evaluated using Hamiltonian Monte Carlo as implemented in
STAN \citep{stan}. STAN provides output statistics to assess
the convergence of the output Markov chains.  The output gives $\hat{R} \sim 1.0$ for all parameters, meaning there is no evidence for non-convergence.  The
output also gives  $N_{eff} \gg 100$ for all parameters, indicating that are all densely sampled.
Empirically, the confidence regions are localized and unimodal as is seen in  Figure~\ref{global:fig}.  We find no evidence that
the Monte Carlo chains have not converged to the stationary posterior distribution.
We have rurun the analysis with a variety of initial conditions, including one with all the $\gamma$'s and $\delta$'s equal to zero, except for a small positive 
$\gamma_0$ and small negative $\delta_0$ to satisfy our limit conditions.  Parameter credible intervals
remain consistent within uncertainty.

\begin{figure}[htbp] %  figure placement: here, top, bottom, or page
   \centering
   \includegraphics[width=2.5in]{output11/coeff0.pdf} 
   \includegraphics[width=2.5in]{output11/coeff1.pdf} 
   \includegraphics[width=2.5in]{output11/coeff2.pdf} 
      \includegraphics[width=2.5in]{output11/coeff3.pdf} 
         \includegraphics[width=2.5in]{output11/coeff4.pdf} 
            \caption{Posterior contours for $c$, $\alpha$, $\beta$, $\eta$, $\gamma$, $\delta$, and $\sigma$ in each of the 5 bands.
            The contours shown here and in future plots represent 1-$\sigma$ in the parameter distribution, not to 68\%, 95\%, etc.\
            enclosed probability.  \label{global:fig}}
\end{figure}

\color{red}

The differences between observed and model-predicted supernova colors are shown in Figure~\ref{residual:fig}:
no systematic trends are apparent (in \S\ref{discussion:sec} we look for trends with light-curve shape.) 
The scatters in the plots, i.e.\ the residual intrinsic color dispersions from our model, are described
later when we discuss the inferred $C_c$.
\begin{figure}[htbp] %  figure placement: here, top, bottom, or page
   \centering
   \includegraphics[width=5in]{output11/residual.pdf} 
            \caption{\textcolor{red}{Differences between observed and  model-predicted supernova colors  as a function
            of predicted colors.  For reference a dotted line is plotted at zero difference.}
            \label{residual:fig}}
\end{figure}

\color{black}

For each of the five filters, the 68\%  equal-tailed credible intervals for the global parameters $\alpha$, $\beta$, $\eta$, $R$, $R_\delta$, and $\sigma$
(and other interesting derived parameters)
are given in Table~\ref{global:tab}.
Contours of the posterior surface for parameter pairs grouped by filter are shown in Figure~\ref{global:fig}.

\begin{table}
\centering
\begin{tabular}{|c|c|c|c|c|c|}
\hline
& $X=0$ &1&2&3&4\\ \hline
$\alpha_{X}$
&
$0.0042^{0.0009}_{-0.0009}$
&
$0.0014^{0.0007}_{-0.0008}$
&
$0.0014^{0.0006}_{-0.0006}$
&
$0.0014^{0.0005}_{-0.0005}$
&
$0.0025^{0.0005}_{-0.0005}$
\\
$\alpha_X/\alpha_2-1$
&
$   2.0^{   1.3}_{  -0.6}$
&
$   0.0^{   0.1}_{  -0.2}$
&
0
&
$   0.0^{   0.2}_{  -0.1}$
&
$   0.8^{   0.9}_{  -0.4}$
\\
$\beta_{X}$
&
$ 0.033^{ 0.003}_{-0.003}$
&
$ 0.027^{ 0.003}_{-0.003}$
&
$ 0.027^{ 0.002}_{-0.002}$
&
$ 0.021^{ 0.002}_{-0.002}$
&
$ 0.020^{ 0.002}_{-0.002}$
\\
$\beta_X/\beta_2-1$
&
$  0.26^{  0.05}_{ -0.05}$
&
$  0.00^{  0.03}_{ -0.03}$
&
0
&
$ -0.19^{  0.01}_{ -0.01}$
&
$ -0.25^{  0.03}_{ -0.03}$
\\
$\eta_{X}$
&
$0.0010^{0.0012}_{-0.0013}$
&
$0.0006^{0.0010}_{-0.0011}$
&
$0.0008^{0.0008}_{-0.0009}$
&
$0.0008^{0.0007}_{-0.0008}$
&
$0.0002^{0.0006}_{-0.0007}$
\\
$\eta_X/\eta_2-1$
&
$  0.26^{  0.65}_{ -0.65}$
&
$ -0.09^{  0.53}_{ -0.55}$
&
0
&
$ -0.08^{  0.24}_{ -0.21}$
&
$ -0.57^{  0.51}_{ -0.59}$
\\
$\gamma_X/\gamma_2-1$
&
$  0.64^{  0.05}_{ -0.05}$
&
$  0.34^{  0.03}_{ -0.03}$
&
0
&
$ -0.23^{  0.01}_{ -0.01}$
&
$ -0.45^{  0.03}_{ -0.03}$
\\
$R_{X}$
&
$  4.87^{  0.48}_{ -0.43}$
&
$  3.96^{  0.40}_{ -0.35}$
&
$  2.96^{  0.31}_{ -0.28}$
&
$  2.29^{  0.26}_{ -0.23}$
&
$  1.64^{  0.21}_{ -0.19}$
\\
$\delta_X/\delta_2-1$
&
$ -0.39^{  0.18}_{ -0.21}$
&
$ -0.22^{  0.10}_{ -0.12}$
&
0
&
$ -0.06^{  0.05}_{ -0.05}$
&
$ -0.15^{  0.10}_{ -0.09}$
\\
$R_{\delta X}$
&
$ -2.70^{  1.32}_{ -2.44}$
&
$ -3.46^{  1.29}_{ -2.65}$
&
$ -4.46^{  1.43}_{ -3.11}$
&
$ -4.20^{  1.32}_{ -2.89}$
&
$ -3.80^{  1.20}_{ -2.61}$
\\
$\sigma_{X}$
&
$ 0.059^{ 0.012}_{-0.011}$
&
$ 0.031^{ 0.007}_{-0.007}$
&
$ 0.020^{ 0.004}_{-0.006}$
&
$ 0.013^{ 0.007}_{-0.008}$
&
$ 0.043^{ 0.005}_{-0.004}$
\\
\hline
\end{tabular}
\caption{68\% credible Intervals for the Global Fit Parameters \label{global:tab}}
\end{table}


We find small but significant non-zero values for $\alpha_0$ and $\alpha_4$, indicating that $EW_{Ca}$ is an indicator of $U$ and $I$
magnitudes.  All bands show larger and significant non-zero values for $\beta$, indicating their sensitivity to
$EW_{Si}$.  This validates our hypothesis that spectral indicators
are tracers of supernova absolute magnitude.  On the other hand, the values of $\eta$ (the coefficients attached to $\lambda_{Si}$) are consistent with zero within one sigma.
The effect of spectral parameters on color is shown in $\alpha_X/\alpha_V-1$,  $\beta_X/\beta_V-1$, and  $\eta_X/\eta_V-1$:
values of zero signify no color changes associated with magnitude changes.
Both $EW_{Ca}$ and $EW_{Ca}$ are associated with color changes though not in $B-V$ specifically.
We do not detect a significant association between
$\lambda_{Si}$ and color.

%
%The best-fit distributions for $R_X=\frac{\gamma_X}{\gamma_1-\gamma_2}$ are shown in Figure~\ref{rx:fig}.  
%Our measurements of
%$R_X$ and $\frac{\gamma_X}{\gamma_2}$, placed at the effective wavelength of the supernova flux that passes
%through the synthetic bands, are shown in
%Figure~\ref{ccm:fig}.
%
%

%
%\begin{figure}[htbp] %  figure placement: here, top, bottom, or page
%   \centering
%   \includegraphics[width=2.8in]{output11/ccm.pdf}
%      \includegraphics[width=2.8in]{output11/ccm2.pdf} 
%   \caption{68\% credible intervals of Left: $R_X=\frac{\gamma_X}{\gamma_1-\gamma_2}$; Right: $\frac{\gamma_X}{\gamma_2}$. 
%   \label{ccm:fig}}
%\end{figure}

None of the 4000 links of
our Monte Carlo chains for $\delta$ exceed 0 (see Figure~\ref{global:fig}), we thus claim detection of an affect of the
intrinsic parameter on supernova magnitudes
with probability $>3999/4000$. 
The posterior distributions for $R_{\delta X}=\frac{\delta_X}{\delta_1-\delta_2}$ are shown in Figure~\ref{rx:fig}.
Although the posteriors of $\delta$ are close to Gaussian, $\delta_1-\delta_2$ can approach zero, resulting in the
elongated tail in negative  $R_{\delta X}$.
All values are negative and inconsistent meaning a color excess in $B-V$ produces a brightening of the supernova.
The values of $\delta_X/\delta_V-1$ are not monotonic with color: they increase from the $U$ to $V$, and then decrease
from $V$ to $I$,
a consequence
of which is that
a reddening in $B-V$ comes with a blueing in $V-R$.
These are not the behaviors expected from normal dust,
circumstellar dust suggested as a source of SN extinction \citep{2005ApJ...635L..33W,2008ApJ...686L.103G,
2015ApJ...807L..26G}, or a blackbody in the Raleigh-Jeans tail, and so may be indicative of
more complicated underlying supernova physics.

\begin{figure}[htbp] %  figure placement: here, top, bottom, or page
   \centering
   \includegraphics[width=2.8in]{output11/rx_corner.pdf}
      \includegraphics[width=2.8in]{output11/rxdelta_corner.pdf} 
   \caption{Posterior contours for Left:  $R_X=\frac{\gamma_X}{\gamma_1-\gamma_2}$;  Right:  $R_{\delta X}=\frac{\delta_X}{\delta_1-\delta_2}$.
   \label{rx:fig}}
\end{figure}

The distribution of both sources of color excess $E_\gamma(B-V) =(\gamma_1-\gamma_2)k$ and
 $E_\delta(B-V) = (\delta_1-\delta_2)D$ are shown as the ideograms in Figure~\ref{ebv:fig}.
Recall that our boundary conditions set the means of both distributions to zero. 
The standard deviation of $E_\gamma(B-V)$ is 0.078 mag while that of $E_\delta(B-V)$ is 0.019 mag.
The range of $B-V$ colors is $\sim 4$ times larger for the external contribution. 
Both distributions are non-Gaussian with sharp rises in the blue and extended tails in the red, consistent
with the expectation for extinction due to host-galaxy dust \citep{1998ApJ...502..177H}.

\begin{figure}[htbp] %  figure placement: here, top, bottom, or page
   \centering
   \includegraphics[width=3.2in]{output11/ebv.pdf}
   \caption{Ideograms of the external $E_\gamma(B-V) =(\gamma_1-\gamma_2)k$ and
   internal $E_\delta(B-V) = (\delta_1-\delta_2)D$  contributions to color excess  for the supernovae in our sample.
   \label{ebv:fig}}
\end{figure}



The ideogram for the grey offsets $\Delta$ for all supernovae is shown in Figure~\ref{hist:fig}.  The distribution is non-Gaussian, 
has a standard deviation of $0.096$ mag, and a broad tail in the positive (fainter) direction.
\begin{figure}[htbp] %  figure placement: here, top, bottom, or page
   \centering
   \includegraphics[width=3.2in]{output11/Delta_hist.pdf} 
   \caption{Ideogram for the grey offset $\Delta$.
   \label{hist:fig}}
\end{figure}


Non-trivial residual magnitude dispersions are captured in $C_c$.   Figure~\ref{sigma:fig} shows the confidence regions for $\sigma$, the
square root of the diagonal elements of $C_c$.  The residual intrinsic dispersion ranges from 0.01 to 0.06 mag, significantly smaller
than the dispersion in $\Delta$.
The off-diagonal elements of $C_c$ are parameterized by the Cholesky factors of a correlation matrix.
Assembling
a new set of Cholesky factors based on the distributions of each individual factor will not generally satisfy the condition of representing a correlation matrix.  
To characterize a typical posterior draw of $C_c$ we use the matrix that is the mean of all covariance realizations in the
chain, element by element.
For $UBVRI$ the matrix is
\begin{equation}
\begin{pmatrix}
0.0037 & 0.0010 & -0.0001 & -0.0001 & 0.0003 \\
0.0010 & 0.0010 & 0.0002 & -0.0000 & -0.0004 \\
-0.0001 & 0.0002 & 0.0004 & 0.0001 & -0.0001 \\
-0.0001 & -0.0000 & 0.0001 & 0.0002 & 0.0002 \\
0.0003 & -0.0004 & -0.0001 & 0.0002 & 0.0019
 \end{pmatrix} \text{mag}^2.
 \label{mag_cov:eqn}
 \end{equation}
The  covariance of the colors $U-V$, $B-V$, $V-R$, and $V-I$ is
expressed as the standard deviations and
 correlation matrix
 \begin{equation}
 \begin{pmatrix}
0.066 , 0.033 , 0.022 , 0.050 
  \end{pmatrix} \text{mag}
 \label{color_sd:eqn}
   \end{equation}
 \begin{equation}
\begin{pmatrix}
1.000 & 0.634 & -0.256 & -0.300 \\
0.634 & 1.000 & -0.203 & 0.032 \\
-0.256 & -0.203 & 1.000 & 0.604 \\
-0.300 & 0.032 & 0.604 & 1.000
  \end{pmatrix}.
  \label{color_cor:eqn}
 \end{equation}
 
 \begin{figure}[htbp] %  figure placement: here, top, bottom, or page
   \centering
   \includegraphics[width=3.2in]{output11/sigma_corner.pdf} 
   \caption{Posterior contours for the parameters $\sigma$, the square root of the diagonal elements of $C_c$.
   \label{sigma:fig}}
\end{figure}

Each supernova is described by its parameters $EW_{Ca}$, $EW_{Si}$, $\lambda_{Si}$, $E_\gamma(B-V)$, and $E_\delta(B-V)$, as well as its grey offset
$\Delta$: their distributions for all Monte Carlo links for all supernovae are shown in Figure~\ref{perobject:fig}.
There is a core concentration in the  parameter-space, with around ten objects that occupy its outskirts.
Many outliers appear in the red tail of $E_\gamma(B-V)$, as would be expected for the infrequent selection of supernovae
heavily extinguished by host-galaxy dust.
Outliers  are also clearly distinguished in  $EW_{Ca}$--$\lambda_{Si}$ space.

\begin{figure}[htbp] %  figure placement: here, top, bottom, or page
   \centering
   \includegraphics[width=3.2in]{output11/perobject_corner.pdf} 
   \caption{Distributions for the supernova parameters $EW_{Ca}$, $EW_{Si}$, $\lambda_{Si}$, $E_\gamma(B-V)$, and $E_\delta(B-V)$, as well as the grey offset
$\Delta$.
   \label{perobject:fig}}
\end{figure}

The Pearson correlation coefficients for $\Delta$, $EW_{Ca}$, $EW_{Si}$, $\lambda_{Si}$, $E_\gamma(B-V)$, and $E_\delta(B-V)$ are given in the matrix
\begin{equation}
\begin{pmatrix}
1.000 & -0.012 & -0.031 & -0.026 & 0.070 & 0.002 \\
-0.012 & 1.000 & 0.128 & -0.331 & -0.064 & -0.004 \\
-0.031 & 0.128 & 1.000 & -0.182 & -0.155 & -0.024 \\
-0.026 & -0.331 & -0.182 & 1.000 & 0.018 & 0.036 \\
0.070 & -0.064 & -0.155 & 0.018 & 1.000 & 0.062 \\
0.002 & -0.004 & -0.024 & 0.036 & 0.062 & 1.000
\end{pmatrix}.
\end{equation}
Again  $EW_{Ca}$--$\lambda_{Si}$ is singled out, this time as having the strongest (anti-)correlation.  
External $E_\gamma(B-V)$ dust is
most strongly correlated with $EW_{Si}$; recall that $EW_{Si}$ is correlated with light-curve shape, which is correlated with host-galaxy properties
\citep{2003MNRAS.340.1057S}.
The internal $E_\delta(B-V)$  has no correlation  with internal parameters stronger than
0.04.  Although $E_\gamma(B-V)$ and $E_\delta(B-V)$ are determined simultaneously for each supernova from colors, they have a low correlation coefficient of 0.062;
this low correlation, and the broad distribution in $E_\delta(B-V)$ for low $E_\gamma(B-V)$,  indicate that what we have been calling intrinsic
indeed is not related to extrinsic dust.
An extensive analysis of the correlations between the spectral features of the SNfactory data set can be found in \citet{chotard:thesis}.

The range of SiII equivalent widths is $\pm 20$~\AA\ whereas the width of the $B$-band is 851~\AA.  
The implied span in $B$ magnitude based on $\beta_1$ is 0.54~mag.  Therefore $\beta_1$ cannot wholly be attributed to the line itself.
Similarly, the CaII equivalent widths have range $\pm 50$~\AA, while the width of the $U$ band is
701~\AA.  The implied span in $U$ magnitude is is 0.21~mag, so $\alpha_0$ cannot be completely due to the line iteslf.

\section{SN 2014J in the Context of Our Model}
\label{sn2014j:sec}

SN~2014J   is one of several SNe~Ia that exhibits colors that imply a low $R_V<2.0$ \citep{2014ApJ...788L..21A, 2014MNRAS.443.2887F, 
2014ApJ...795L...4K, 2015ApJ...805...74B}.
The colors of this supernova can be studied in comparison to SN~2011fe, an object with similar
spectral evolution that 
suffers low Galactic and host reddening
\citep[e.g.][]{2006MNRAS.369.1880E,2007AJ....133...58K,2008MNRAS.384..107E,2010AJ....139..120F, 2014ApJ...788L..21A}.
The overall higher photospheric velocities of
SN~2014J are unimportant  based on the findings of  \S\ref{results:sec}.


Data of SN~2014J are  analyzed using our model and the parameter values found with the SNfactory sample.
The data for the color excesses  in $UBRi$  relative to $V$ at peak brightness  are taken from \citet{2014ApJ...788L..21A},
following their perscription of averaging measurements within 5-days of peak $B$ magnitude.
Their values 
$E_o(U-V) =   2.23 \pm   0.03$,
$E_o(B-V) =   1.28 \pm   0.04$,
$E_o(R-V) =  -0.47 \pm   0.03$,
$E_o(i-V) =  -0.92 \pm   0.03$
are plotted in Figure~\ref{sn2014j:fig}.
These data are fit to the model
\begin{equation}
E_o(X-V) =  \left(\frac{\gamma_X}{\gamma_2}-1\right)k +  \left(\frac{\delta_X}{\delta_2}-1\right)D,
\end{equation}
where we use the median values of the $\gamma$- and $\delta$-terms from Table~\ref{global:tab}.

\begin{figure}[htbp] %  figure placement: here, top, bottom, or page
   \centering
   \includegraphics[width=3.2in]{output11/sn2014j.pdf} 
   \caption{Measured color excess for SN~2014J and the best-fit predictions from this article and  \citet{2014ApJ...788L..21A}  
   \label{sn2014j:fig}}
\end{figure}

The best-fit parameters for SN~2014J are $k= 2.56$, $ D=-1.57$ with covariance
\begin{equation}
\begin{pmatrix}
0.058 & 0.059 \\
0.059 & 0.204
\end{pmatrix}.
\end{equation}
The predicted values of observed $E_o(B-V)$ from the fit are shown in Figure~\ref{sn2014j:fig}, where they are found to
overlap with the data within the error bars.   For comparison, the best-fit model determined by  \citet{2014ApJ...788L..21A} using
UV through NIR data,
a  \citet{1999PASP..111...63F} dust with $R_V^F=1.4$ and $E(B-V)=1.37$, 
is also plotted.  Our model predictions provide a significantly better match to the data.

In the fit to our model, the observed color excess is attributed to   extrinsic
$E_\gamma(B-V)=  0.86 \pm   0.08$ and intrinsic $E_\delta(B-V)=  0.35 \pm   0.10$ contributions.
Among the supernovae in the SNfactory  set used to determine $\gamma$ and $\delta$, the
\color{red}
extreme median values for 
$E_\gamma(B-V)$ are $-0.07 \pm 0.01$ and  $  0.33 \pm 0.04$,
and for $E_\delta(B-V)$ $-0.01 \pm 0.01$  and
$  0.07 \pm 0.03$ 
\color{black}
(see Figure~\ref{ebv:fig}).
\textcolor{red}{The deduced parameters for SN~2014J, which are relative to SN~2011fe, live well outside the 
span} of supernovae used to train the coefficients of the model.

\section{Discussion}
\label{discussion:sec}
We model SNe~Ia broadband optical peak magnitudes allowing for correlations with spectral features at peak and distinct intrinsic and
extrinsic color parameters.  Analyzing SNfactory data with the model, we find significant evidence that the above parameters do
affect supernova magnitudes and colors.  The external color parameter has behavior consistent with $R^F_V=3.06$ \citet{1999PASP..111...63F} dust,
whereas the internal color parameter is inconsistent with dust or a blackbody.  This model  does an excellent job in
describing SN~2014J, a supernova not used in the training that has atypical colors.

\color{red}
Our model does not explicitly include light-curve shape, which is known to track SN~Ia diversity.
The correlation between $EW_{Si}$ and the SALT2 $X_1$ light-curve shape parameter is verified
for our sample, as shown in Figure~\ref{x1:fig}.
On the other hand, there is no such correlation seen between $E_\delta(B-V)$ and $X_1$. 
Our magnitude residuals $\sigma_X$ are comparable to those of SALT2
\citep{2010A&A...523A...7G}, suffering no  disadvantage from the neglect of $X_1$.
Figure~\ref{x1res:fig} shows the difference between observed and model-predicted colors,
as a function of $X_1$: for all colors, there appear non-linear correlations 
with a negative slope for $X_1<0$ and positive slope for $X_1>0$.
We  run an extension of our model to include light-curve shape by adding a new linear term $\zeta X_1$; the credible intervals
from this analysis are given in Table~\ref{globalx1:tab}.  
The $\sigma_X$ values decrease at the per-mil level and there $<1\sigma$
shifts in parameters as compared to the previous results. 
The light-curve shape is not a strong predictor of supernova magnitude relative to the spectral features: the most significant non-zero values for
$\zeta$ are in the $V$ and $I$ bands at less than $1.5 \sigma$, while the $\alpha$'s and $\beta$'s remain significantly
non-zero.  This finding could be due to a deficiency of our model, 
as it  is incapable of capturing the non-linear trends in $X_1$ in Figure~\ref{x1res:fig}.
\begin{figure}[htbp] %  figure placement: here, top, bottom, or page
   \centering
   \includegraphics[width=2.8in]{output11/x1si.pdf}
   \includegraphics[width=2.8in]{output11/x1D.pdf}
    \caption{\textcolor{red}{Left: SALT2 $X_1$ versus $EW_{Si}$.  Right: $X_1$ versus $E_\delta(B-V)$.}
   \label{x1:fig}}
\end{figure}

\begin{figure}[htbp] %  figure placement: here, top, bottom, or page
   \centering
   \includegraphics[width=5in]{output11/residualx1.pdf}
    \caption{\textcolor{red}{Differences between observed colors and the colors predicted from the analysis, as a function
            of the SALT2 light-curve shape parameter $X_1$.  For reference a dotted line is plotted at zero difference.}
   \label{x1res:fig}}
\end{figure}


\begin{table}
\centering
\begin{tabular}{|c|c|c|c|c|c|}
\hline
& $X=0$ &1&2&3&4\\ \hline
$\alpha_{X}$
&
$0.0039^{0.0010}_{-0.0010}$
&
$0.0013^{0.0008}_{-0.0008}$
&
$0.0015^{0.0007}_{-0.0007}$
&
$0.0014^{0.0006}_{-0.0006}$
&
$0.0026^{0.0005}_{-0.0005}$
\\
${\alpha_X/\alpha_2-1}$
&
$   1.6^{   1.1}_{  -0.5}$
&
$  -0.1^{   0.2}_{  -0.3}$
&
0
&
$  -0.1^{   0.1}_{  -0.1}$
&
$   0.8^{   0.9}_{  -0.4}$
\\
$\beta_{X}$
&
$ 0.036^{ 0.005}_{-0.005}$
&
$ 0.027^{ 0.004}_{-0.004}$
&
$ 0.023^{ 0.004}_{-0.004}$
&
$ 0.020^{ 0.003}_{-0.003}$
&
$ 0.017^{ 0.003}_{-0.003}$
\\
${\beta_X/\beta_2-1}$
&
$  0.56^{  0.12}_{ -0.10}$
&
$  0.18^{  0.06}_{ -0.06}$
&
0
&
$ -0.13^{  0.03}_{ -0.02}$
&
$ -0.29^{  0.06}_{ -0.06}$
\\
$\eta_{X}$
&
$0.0013^{0.0012}_{-0.0012}$
&
$0.0009^{0.0010}_{-0.0010}$
&
$0.0011^{0.0008}_{-0.0008}$
&
$0.0010^{0.0007}_{-0.0007}$
&
$0.0004^{0.0006}_{-0.0006}$
\\
${\eta_X/\eta_2-1}$
&
$  0.23^{  0.49}_{ -0.51}$
&
$ -0.10^{  0.26}_{ -0.43}$
&
0
&
$ -0.10^{  0.17}_{ -0.12}$
&
$ -0.58^{  0.28}_{ -0.49}$
\\
$\zeta_{X}$
&
$ 0.012^{ 0.040}_{-0.040}$
&
$-0.002^{ 0.033}_{-0.034}$
&
$-0.032^{ 0.028}_{-0.028}$
&
$-0.014^{ 0.024}_{-0.025}$
&
$-0.029^{ 0.021}_{-0.022}$
\\
${\zeta_X-\zeta_2-1}$
&
$  -0.9^{   0.8}_{  -2.2}$
&
$  -0.7^{   0.5}_{  -1.4}$
&
0
&
$  -0.4^{   0.2}_{  -0.5}$
&
$  -0.1^{   0.5}_{  -0.4}$

\\
${\gamma_X/\gamma_2-1}$
&
$  0.61^{  0.05}_{ -0.05}$
&
$  0.33^{  0.03}_{ -0.03}$
&
0
&
$ -0.23^{  0.01}_{ -0.01}$
&
$ -0.45^{  0.03}_{ -0.03}$
\\
$R_{X}$
&
$  4.93^{  0.52}_{ -0.46}$
&
$  4.06^{  0.43}_{ -0.38}$
&
$  3.06^{  0.34}_{ -0.30}$
&
$  2.35^{  0.28}_{ -0.25}$
&
$  1.67^{  0.23}_{ -0.20}$
\\
${\delta_X/\delta_2-1}$
&
$ -0.45^{  0.17}_{ -0.19}$
&
$ -0.29^{  0.10}_{ -0.11}$
&
0
&
$ -0.05^{  0.05}_{ -0.04}$
&
$ -0.13^{  0.10}_{ -0.08}$
\\
$R_{\delta X}$
&
$ -1.92^{  0.90}_{ -1.30}$
&
$ -2.48^{  0.82}_{ -1.30}$
&
$ -3.48^{  0.90}_{ -1.52}$
&
$ -3.30^{  0.83}_{ -1.42}$
&
$ -3.03^{  0.78}_{ -1.31}$
\\
$\sigma_{X}$
&
$ 0.058^{ 0.012}_{-0.011}$
&
$ 0.030^{ 0.007}_{-0.007}$
&
$ 0.017^{ 0.004}_{-0.007}$
&
$ 0.013^{ 0.006}_{-0.008}$
&
$ 0.042^{ 0.005}_{-0.004}$
\\
\hline
\end{tabular}
\caption{68\% credible Intervals for the Global Fit Parameters including $X_1$ \label{globalx1:tab}}
\end{table}



\color{black}

\color{red}
The extrinsic parameters $\gamma$ attributed to dust are fit for without reference to a specific dust model
we now do so in a fit to the 
model of \citet{1999PASP..111...63F},
assuming the fixed 3693,  4369,  5287,  6319, 7611 \AA\  for the effective wavelengths of the supernova flux through the bands.
Figure~\ref{fitz:fig} shows the   68\% credible intervals of $\gamma$ and our result
$R^F_V=3.06\pm 0.18$.  
We do the analysis in $\gamma$, as opposed to $R_V$ say, because the posteriors of $\gamma$ are well-approximated by a normal distribution that is used for the fit.
\color{black}

\begin{figure}[htbp] %  figure placement: here, top, bottom, or page
   \centering
   \includegraphics[width=3.2in]{output11/fitz.pdf}
   \caption{\textcolor{red}{68\% credible intervals of $\gamma_X$ and the best-fit to the  \citet{1999PASP..111...63F}  model $R^F_V=3.06$.}
   \label{fitz:fig}}
\end{figure}


\citet{2014ApJ...789...32B, 2015MNRAS.453.3300A} deduce a wide range of dust behavior $1.5<R^F_V<3$ encountered by the SN~Ia population.
\citet{2011ApJ...731..120M, 2011ApJ...729...55F} and the above authors
find that supernovae with large $E_o(B-V)$ 
preferentially exhibit low $R^F_V$. 
\color{red}
In our model internal and external effects for each supernova  conflate into an effective color excess
$E_{eff}(X-V) = E_\gamma(X-V) + E_\delta(X-V)$ and effective $V$ extinction $A_{eff,V} = \gamma_2 k  + \delta_2 D$ 
Recall that the colors where $E_\gamma(B-V)=0$,  $E_\delta(B-V)=0$ are free and arbitrarily
fixed using the  boundary conditions $\langle \Delta \rangle = 0$, $\langle k \rangle = 0$:
there is no unique value for the statistic $R_{eff,V} = A_{eff,V} /E_{eff}(B-V)$, as it depends on the choice of  boundary condition.

For specificity in the current qualitative discussion, 
let us consider a set of SNe~Ia  with parameters set to the median fit values of the SNfactory sample.
We specify   $E_{eff}(U-V)$  and $E_{eff}(B-V)$ by
assigning zero color excess close to the blue-extreme colors of this set, 
as shown in bottom-left corner in the left plot of Figure~\ref{rveff:fig}.
For this choice of color zeropoints,
$R_{eff,U} - R_{eff,V} = E_{eff}(U-V) /E_{eff}(B-V)$ is plotted
in the right of Figure~\ref{rveff:fig}.
Around $E_{eff}(B-V) \sim 0$ the $R_{eff,U} - R_{eff,V}$ values are high, but this is also the regime where their  uncertainties 
blow up due to a division of an uncertain quantity with value close to zero.
The distribution spreads to significantly 
lower   $R_{eff,U} - R_{eff,V}$ as supernovae appear redder but
then reconverge to higher, tighter values red effective color, with asymptote  $R_{eff,U} - R_{eff,V} >1.8$.
This cause of this asymptotic behavior is due to the relatively higher contribution of extrinsic over intrinsic color excess to the effective color excess, as
seen in the bottom plot of
Figure~\ref{rveff:fig}. 
Since the standard deviation of the $E_\delta(B-V)$ distribution is  narrow compared to that of
 $E_\gamma(B-V)$, low values of $R_{eff,U} - R_{eff,V}$ can only occur
at low $E_{eff}(X-V)$.
 The  \citet{1999PASP..111...63F} dust model interprets
 low values of $R_{eff,U} - R_{eff,V}$ as implying high values of $R_V$,
for example $R_{eff,U} - R_{eff,V}=1.70$ for $R_V=3.06$,  $R_{eff,U} - R_{eff,V}=1.81$ for $R_V=1.5$ (see Figure~\ref{ebv:fig}).
The $R_V$ inferred for the reddest supernovae would be $R_V<1.5$, even though their colors are dominated by 
extrinsic $R^F_V=3.06$ dust; this is due our zero colors being (qualitatively) assigned to supernovae that have bluest extrinsic AND intrinsic color excess.
\begin{figure}[htbp] %  figure placement: here, top, bottom, or page
   \centering
   \includegraphics[width=2.8in]{output11/rveff.pdf}
   \includegraphics[width=2.8in]{output11/rveff2.pdf}
   \includegraphics[width=2.8in]{output11/rveff3.pdf}
      \caption{\textcolor{red}{Left: Effective color excess $E_{eff}(B-V)$ versus $E_{eff}(U-V)$
   where $E_{eff}(X-V)   = E_\gamma(X-V) + E_\delta(X-V) - z_X$
   for a set of SNe with parameters set to those of the medians of the objects in our analysis.
   The $z_X$ represents a choice of colors at zero color-excess so that the lower left corner points have values approaching zero. 
Right:  The inferred $R_{eff,U}- R_{eff,V} = E_{eff}(U-V)/E_{eff}(B-V)$.
Bottom:  $R_{eff,U}- R_{eff,V}$ as a function of the relative contributions of extrinsic and intrinsic color effects $E_\gamma(B-V)/E_\delta(B-V)$.}
   \label{rveff:fig}}
\end{figure}

An alternative way to specify $R_V$ is to consider colors relative to a canonical supernova,
 the model prediction being
\begin{equation}
R_{eff, V} = \frac{\gamma_1  (k-k_0) + \delta_1 (D-D_0) }{(\gamma_1-\gamma_2) (k-k_0) + (\delta_1-\delta_2) (D-D_0) },
\end{equation}
where $k_0$ and $D_0$ are the parameter values of the canonical supernova.
If $D = D_0$, the $R_V = \gamma_1/(\gamma_1-\gamma_2)$ of the extrinsic contribution is recovered, but this does not hold true
when $D \ne D_0$.  Again considering a set of SNe~Ia  with parameters equal to the median fit values of the SNfactory sample,
the values for $R_{eff,V}$ choosing a canonical supernova with $k_0=0$, $\delta_0=0$  is shown in Figure~\ref{rv4:fig}.
The reddest supernovae exhibit a range of $1.5< R_{eff,V}<3.5$. 

\begin{figure}[htbp] %  figure placement: here, top, bottom, or page
   \centering
   \includegraphics[width=3.2in]{output11/rveff4.pdf}
      \caption{\textcolor{red}{Effective total-to-selective extinction $R_{eff, V}$
      with color excess defined relative to  a
      $k_0=0$, $\delta_0=0$ supernove,  for  a set of SNe~Ia  with parameters equal to the median fit values of the SNfactory sample.}
   \label{rv4:fig}}
\end{figure}


\color{black}

%The trend in Figure~\ref{rveff:fig} showing an increased range of brighter-than-expected magnitudes  with increasing $E_{eff}(B-V)$  can
%be compared
%to Figure~\ref{betoule:fig}, which shows the $B$-band  Hubble residuals as a function of the color $C$ parameter (directly
%comparable to $E(B-V)$) from the Hubble diagram analysis of \citet{2014A&A...568A..22B}.
%A linear fit for the relationship between color and Hubble residual has a significant slope ($-0.57 \pm 0.07$), consistent with the slope
%found for the SNfactory data.
%
%\begin{figure}[htbp] %  figure placement: here, top, bottom, or page
%   \centering
%   \includegraphics[width=3.2in]{output11/betoule.pdf} 
%   \caption{Best-fit SN~Ia  $B$-band Hubble residuals as a function of the best-fit color parameter $C$ from the cosmology analysis of \citet{2014A&A...568A..22B}.
%   \label{betoule:fig}}
%\end{figure}


\citet{2009ApJ...699L.139W, 2011ApJ...729...55F} find a connection between $v_{Si}$ and color, and  
\citet{2015MNRAS.447.1247S} show that $v_{Si}$ is an important spectral classifier within the SNfactory data themselves.
The values of $\eta$ in this article are consistent with zero. Instead,  $EW_{Ca}$ and $EW_{Si}$ have a significant affect on color, not particularly 
surprising as they
are strong features in the $B$ band.
Recall that we find a correlation between $\lambda_{Si}$ and $EW_{Ca}$
Removing from our model the dependence on equivalent widths (eliminating the  $\alpha$ and $\beta$ parameters), we recover
non-zero $\eta$ values at  $\gtrsim 2\sigma$.  We conclude that $v_{Si}$ is correlated with color, 
\textcolor{red}{but the equivalent-width corrections are preferred as they give lower residual dispersions $C_c$},
and  anticipate that in the previous analysis the equivalent widths of CaII and 
SiII would also have shown correlations with color.


We find an intrinsic supernova parameter that does not have a monotonic influence on supernova magnitudes as a function
of wavelength but rather has an inflection in the $V$ band.  In addition, red $B-V$ colors imply brighter magnitudes.  Line opacities
that operate over broad bands are the most obvious candidate for producing this behavior.
\citet{2006ApJ...649..939K} points out that at a temperature of 7000~K, the iron/cobalt gas in the ejecta transition
from doubly and singly ionized states, making the gas phosphorescent and efficient in redistributing energy from the UV/blue to redder
wavelengths.  Qualitatively this would result in the structure for $\gamma_X/\gamma_2-1$ that we measure.
We conjecture that varying $^{56}$Ni mass affects the rise-time to peak $B$ brightness and hence the optical depth
of the reionization front with respect to the core of the ejecta, yielding the wavelength-dependent color behavior
detected in this analysis.

The model introduced in this article is limited, in that it doesn't accommodate the expected diversity
in extrinsic dust properties that supernovae in different hosts may encounter.  Extending our model to 
include  a parameterized  $\gamma$ distribution would allow a range of host-dust behavior, and
offers a direction of further study.
For the moment, we assert that variations in extrinsic dust across supernovae are limited in
size by Eqn.~\ref{color_sd:eqn} and positive correlations possible within Eqn.~\ref{color_cor:eqn}.


The model and results presented here
carry information on the calibration of absolute magnitude.  The grey parameter $\Delta$ represents Hubble residuals after
application of the model, and its  $0.096$ mag dispersion has contributions from intrinsic dispersion, peculiar velocities, and
measurement uncertainties.  It is also subject to overtraining.
The  analysis of this article was intended to model the color behavior of supernovae in our dataset.
We defer to a new model to determine how
well this model can be used to measure supernova distances
and examine correlations between its Hubble residuals host-galaxy parameters, e.g.\ mass,
local star formation rate \citep{2010ApJ...715..743K, 2013A&A...560A..66R}.


\acknowledgments
We thank the STAN team for providing the statistical tool without which this analysis would not have been possible,
and Michael Betancourt specifically for his helpful guidance.  We thank Danny Goldstein for useful discussions.
This work was supported by the Director, Office of Science, Office of High Energy Physics, 
of the U.S.\ Department of Energy under Contract No. DE-AC02-05CH11231.

\bibliographystyle{apj}
\bibliography{/Users/akim/Documents/alex}


\end{document} 
