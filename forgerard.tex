\documentclass{aastex}   	% use "amsart" instead of "article" for AMSLaTeX format
\usepackage{geometry}                		% See geometry.pdf to learn the layout options. There are lots.
\geometry{letterpaper}                   		% ... or a4paper or a5paper or ... 
%\geometry{landscape}                		% Activate for rotated page geometry
%\usepackage[parfill]{parskip}    		% Activate to begin paragraphs with an empty line rather than an indent
\usepackage{graphicx}				% Use pdf, png, jpg, or eps§ with pdflatex; use eps in DVI mode
								% TeX will automatically convert eps --> pdf in pdflatex		
%\usepackage{amssymb}
\usepackage{amsmath}
\usepackage{natbib}
\usepackage{lineno}
\linenumbers


\begin{document}

\title{text}
\author
{
    G.~Aldering,\altaffilmark{2}
    P.~Antilogus,\altaffilmark{4}
    S.~Bailey,\altaffilmark{2}
    C.~Baltay,\altaffilmark{5}
    K.~Barbary,\altaffilmark{3}
    D.~Baugh,\altaffilmark{6}
    K.~Boone,\altaffilmark{2,3}
    S.~Bongard,\altaffilmark{4}
    C.~Buton,\altaffilmark{7}
    J.~Chen,\altaffilmark{6}
    N.~Chotard,\altaffilmark{7}
    Y.~Copin,\altaffilmark{7}
    P.~Fagrelius,\altaffilmark{2,3}
    H.~K.~Fakhouri,\altaffilmark{2,3}
    U.~Feindt,\altaffilmark{9}
    D.~Fouchez,\altaffilmark{10}
    E.~Gangler,\altaffilmark{11}  
    B.~Hayden,\altaffilmark{2}
    W.~Hillebrandt,\altaffilmark{16}
    A.~G.~Kim,\altaffilmark{2}
    M.~Kowalski,\altaffilmark{9,12}
    P.-F.~Leget,\altaffilmark{11}
    S.~Lombardo,\altaffilmark{9}
    J.~Nordin,\altaffilmark{2,9}
    R.~Pain,\altaffilmark{4} 
    E.~Pecontal,\altaffilmark{13}
    R.~Pereira,\altaffilmark{7}
    S.~Perlmutter,\altaffilmark{2,3}
    D.~Rabinowitz,\altaffilmark{5}
    M.~Rigault,\altaffilmark{9} 
    D.~Rubin,\altaffilmark{2,14}
    K.~Runge,\altaffilmark{2}
    C.~Saunders,\altaffilmark{2,3}
   R.~Scalzo,\altaffilmark{8}
    G.~Smadja,\altaffilmark{7} 
    C.~Sofiatti,\altaffilmark{2,3} 
    A.~Stocker,\altaffilmark{1}
    N.~Suzuki,\altaffilmark{2}
    S.~Taubenberger,\altaffilmark{16}
    C.~Tao,\altaffilmark{6,10}
    R.~C.~Thomas,\altaffilmark{15} \\
    (The Nearby Supernova Factory)
}


\altaffiltext{2}
{
    Physics Division, Lawrence Berkeley National Laboratory, 
    1 Cyclotron Road, Berkeley, CA, 94720
}
\altaffiltext{3}
{
    Department of Physics, University of California Berkeley,
    366 LeConte Hall MC 7300, Berkeley, CA, 94720-7300
}
\altaffiltext{4}
{
    Laboratoire de Physique Nucl\'eaire et des Hautes \'Energies,
    Universit\'e Pierre et Marie Curie Paris 6, Universit\'e Paris Diderot Paris 7, CNRS-IN2P3, 
    4 place Jussieu, 75252 Paris Cedex 05, France
}
\altaffiltext{5}
{
    Department of Physics, Yale University, 
    New Haven, CT, 06250-8121
}
\altaffiltext{6}
{
    Tsinghua Center for Astrophysics, Tsinghua University, Beijing 100084, China 
}
\altaffiltext{7}
{
    Universit\'e de Lyon, F-69622, Lyon, France ; Universit\'e de Lyon 1, Villeurbanne ; 
    CNRS/IN2P3, Institut de Physique Nucl\'eaire de Lyon.
}
\altaffiltext{8}
{
    Research School of Astronomy and Astrophysics,
    Australian National University,
    Canberra, ACT 2611, Australia
}
\altaffiltext{9}
{
    Institut fur Physik,  Humboldt-Universitat zu Berlin,
    Newtonstr. 15, 12489 Berlin
}
\altaffiltext{10}
{
    Centre de Physique des Particules de Marseille, 
    Aix-Marseille Universit\'e , CNRS/IN2P3, 
    163 avenue de Luminy - Case 902 - 13288 Marseille Cedex 09, France
}
\altaffiltext{11}
{
    Clermont Universit\'e, Universit\'e Blaise Pascal, CNRS/IN2P3, Laboratoire de Physique Corpusculaire,
    BP 10448, F-63000 Clermont-Ferrand, France
}
\altaffiltext{12}
{
    DESY, D-15735 Zeuthen, Germany
}
\altaffiltext{13}
{
    Centre de Recherche Astronomique de Lyon, Universit\'e Lyon 1,
    9 Avenue Charles Andr\'e, 69561 Saint Genis Laval Cedex, France
}
\altaffiltext{14}
{
    Department of Physics, Florida State University,
    315 Keen Building, Tallahassee, FL 32306-4350
}
\altaffiltext{15}
{
    Computational Cosmology Center, Computational Research Division, Lawrence Berkeley National Laboratory, 
    1 Cyclotron Road MS 50B-4206, Berkeley, CA, 94720
}

\altaffiltext{16}
{
    Max-Planck-Institut f\"ur Astrophysik, Karl-Schwarzschild-Str. 1,
D-85748 Garching, Germany
}

\begin{abstract}
abstract text
\end{abstract}

\keywords{text}

\section{Introduction}
Type~Ia supernovae (SNe~Ia) form a homogenous set of astronomical objects and as such were recognized as a powerful distance indicator 
and probe of cosmology \citep{1992ARA&A..30..359B}.  After further careful consideration of supernova data, it was recognized
that SN~Ia light-curve shapes \citep{1984SvA....28..658P, 1993ApJ...413L.105P} and colors \citep{1998A&A...331..815T} exhibit subtle signs of heterogeneity,
which need to be considered when inferring SN~Ia distances.  Empirical models parameterizing SNe~Ia by their light-curve shape
and color were developed \citep{1996ApJ...473...88R, 1999ApJ...517..565P} to enable distance measurements of cosmological supernovae
that subsequently were used in the discovery of the accelerating expansion of the Universe \citep{1998AJ....116.1009R,1999ApJ...517..565P}.

The two most commonly used supernova light-curve fitters for cosmological purposes today are SALT2 \citep{2007A&A...466...11G} and MLCS2k2
\citep{2007ApJ...659..122J}.\footnote{More flexible light-curve fitters
\citep[e.g.][]{2011AJ....141...19B} are used to study SN~Ia heterogeneity.}
They remain two-parameter models, with one parameter characterizing light-curve shape and the other
supernova color.  SALT2 training extracts color variation empirically from SNe that span a wide range of colors whereas MLCS2k2
attributes color to the attenuation of light from host-galaxy dust. Neither model accommodates the naive expectation that
both intrinsic and extrinsic color variation may occur.

There is evidence that supports the expectation that one parameter in insufficient to describe the full range
of supernova color behavior.  One approach to look for color diversity is to find correlations between color and spectral features.
\citet{2009ApJ...699L.139W, 2011ApJ...729...55F} find two populations with differing $B_{max}-V_{max}$ identified by Si velocity.
\citet{2015MNRAS.451.1973S}
find that high-velocity Si II$\lambda$6355 are found in objects that have red ultraviolet/optical colors near maximum brightness.
Another approach to probe color diversity is to use multiple colors of a single supernova to infer individual differential reddening. 
\citet{2014ApJ...789...32B, 2015MNRAS.453.3300A} find on average values for $R_V \ll 3.1$,
much smaller than that of diffuse Milky Way dust, though the dispersion is large.
The also confirm the \citet{2011ApJ...729...55F} finding hat low $R_V$ is associated with high-extinction supernovae.
The above evidence has lead to the hypothesis that multiple scattering from circumstellar dust
plays a role in observed supernova colors \citep{2005ApJ...635L..33W, 2008ApJ...686L.103G}.
In contrast, \citet{2011A&A...529L...4C} argue that after accounting for the diversity of spectral features,
the standard $R_V=3.1$ is recovered.


In this article we use the idea that spectral indicators carry information on intrinsic supernova colors,
and allow for the possibility of there being two further color parameters, one intrinsic and the other
extrinsic.  The data used in this analysis are described in \S\ref{data:sec}.  Our supernova model is presented in
\S\ref{model:sec} and results are given in \S\ref{results:sec}.  In \S\ref{sn2014j:sec} our results are applied to the out-of-sample
supernova SN2014J that has been previously identified as having extreme color properties.
We finish with a discussion of our findings in \S\ref{discussion:sec}.


\section{Data}
\label{data:sec}

Our analysis uses the spectrophotometric data set obtained by
the SNfactory with the SuperNova Integral Field
Spectrograph \citep[SNIFS,][]{2004SPIE.5249..146L}.  SNIFS is a fully integrated
instrument optimized for automated observation of point sources on a
structured background over the full ground-based optical window at
moderate spectral resolution ($R \sim 500$).  It consists of a
high-throughput wide-band pure-lenslet integral field spectrograph
\citep[IFS, ``\`a la TIGER;''][]{1995A&AS..113..347B,2000ASPC..195..173B,2001MNRAS.326...23B}, a
multi-filter photometric channel to image the field in the vicinity of
the IFS for atmospheric transmission monitoring simultaneous with
spectroscopy, and an acquisition/guiding channel.  The IFS possesses a
fully-filled $6\farcs 4 \times 6\farcs 4$ spectroscopic field of view
subdivided into a grid of $15 \times 15$ spatial elements, a
dual-channel spectrograph covering 3200--5200~\AA\ and 5100--10000~\AA\
simultaneously, and an internal calibration unit (continuum and arc
lamps).  SNIFS is mounted on the south bent Cassegrain port of the
University of Hawaii 2.2~m telescope on Mauna Kea, and is operated
remotely.  Observations are reduced using the SNfactory's dedicated data
reduction pipeline, similar to that presented in \S4 of \citet{2001MNRAS.326...23B}.
A discussion of the software pipeline is presented in
\citet{2006ApJ...650..510A} and is updated in \citet{2010ApJ...713.1073S}.  A detailed
description of host-galaxy subtraction is given in \citet{2011MNRAS.418..258B}.

The spectral time-series used are corrected for Milky Way dust
extinction \citep{1989ApJ...345..245C,1998ApJ...500..525S} but not for the
effects of circumstellar or host galaxy dust extinction.  
Each spectral time series is
blue-shifted to rest-frame, and the fluxes normalized to represent that
observed at a common distance for all supernovae --- thus, ratios
in flux between supernovae can be interpreted as ratios in
luminosity or equivalently differences in absolute magnitude.

Supernova-frame synthetic photometry is generated for the top-hat filter system
described in \citet{2011A&A...529L...4C}, which is comprised of five logarithmically-spaced bands spanning
3276--8635 \AA; the bands have approximate correspondence to $UBVRI$ and are thus referred to as such.
For each supernova, the magnitudes at peak brightness are determined using single-band SALT2 fits.
The equivalent widths of SiII$\lambda 4141$ and the CaII H\&K features are computed as
in \citet{2008A&A...477..717B} and the velocity of SiII$\lambda 6355$ as in \citet{chotard:thesis}.
For  SiII$\lambda 6355$ the weighted mean from phases within 3 days of maximum light is used.
(What is done for EW?)

There are 163 supernovae that have spectroscopic data within three days of peak brightness:
these constitute our analysis sample.  The data are given in Table~\ref{data:tab}.
The peak magnitude uncertainties do have covariance, which are accounted
for in the analysis; only the variance is included in the table.

\begin{deluxetable}{ccccccccc}
\tabletypesize{\tiny}
\tablecaption{Supernova Spectral-Feature and Peak-Magnitude Datat\label{data:tab}}
\tablehead{
\colhead{$n$} & \colhead{$EW_{Ca}$} & \colhead{$EW_{Si}$} & \colhead{$v_{Si}$} & \colhead{$U$} & \colhead{$B$} & \colhead{$V$} & \colhead{$R$} & \colhead{$I$} \\
\multicolumn{3}{c}{} &  \colhead{(km s$^{-1}$)} & \multicolumn{5}{c}{}
}
\startdata
0 & $109.7 \pm 5.9$ & $ 17.5 \pm 0.7$& $ 6101 \pm   3$ & $-29.31 \pm   0.01$ & $-29.12 \pm   0.01$& $-28.60 \pm   0.01$& $-28.35 \pm   0.01$& $-27.60 \pm   0.01$ \\
1 & $149.7 \pm 1.2$ & $ 14.3 \pm 0.6$& $ 6139 \pm   2$ & $-28.61 \pm   0.02$ & $-28.69 \pm   0.02$& $-28.32 \pm   0.02$& $-28.08 \pm   0.02$& $-27.40 \pm   0.02$ \\
2 & $ 63.8 \pm 21.5$ & $ 27.6 \pm 3.8$& $ 6131 \pm   6$ & $-29.04 \pm   0.07$ & $-28.79 \pm   0.07$& $-28.32 \pm   0.07$& $-28.12 \pm   0.07$& $-27.41 \pm   0.07$ \\
3 & $129.9 \pm 0.9$ & $  9.5 \pm 0.7$& $ 6093 \pm   2$ & $-29.23 \pm   0.01$ & $-29.07 \pm   0.01$& $-28.72 \pm   0.01$& $-28.44 \pm   0.01$& $-27.69 \pm   0.01$ \\
4 & $114.4 \pm 0.9$ & $  8.4 \pm 0.4$& $ 6129 \pm   6$ & $-29.26 \pm   0.01$ & $-28.99 \pm   0.01$& $-28.50 \pm   0.01$& $-28.20 \pm   0.01$& $-27.44 \pm   0.01$ \\
5 & $ 64.9 \pm 4.5$ & $ 16.5 \pm 0.7$& $ 6098 \pm   4$ & $-29.55 \pm   0.02$ & $-29.19 \pm   0.02$& $-28.68 \pm   0.02$& $-28.48 \pm   0.02$& $-27.93 \pm   0.02$ \\
6 & $151.4 \pm 3.0$ & $ 13.9 \pm 1.1$& $ 6140 \pm   4$ & $-28.63 \pm   0.02$ & $-28.57 \pm   0.01$& $-28.20 \pm   0.01$& $-27.99 \pm   0.01$& $-27.34 \pm   0.01$ \\
7 & $151.9 \pm 1.5$ & $  7.9 \pm 0.7$& $ 6099 \pm   4$ & $-29.37 \pm   0.01$ & $-29.14 \pm   0.01$& $-28.71 \pm   0.01$& $-28.40 \pm   0.01$& $-27.64 \pm   0.01$ \\
8 & $108.2 \pm 6.0$ & $ 20.6 \pm 1.9$& $ 6122 \pm   5$ & $-29.02 \pm   0.06$ & $-28.80 \pm   0.06$& $-28.40 \pm   0.06$& $-28.20 \pm   0.06$& $-27.50 \pm   0.06$ \\
9 & $123.7 \pm 6.6$ & $ 16.4 \pm 4.3$& $ 6099 \pm   4$ & $-28.26 \pm   0.02$ & $-28.20 \pm   0.02$& $-27.93 \pm   0.02$& $-27.74 \pm   0.02$& $-27.22 \pm   0.04$ \\
10 & $101.7 \pm 1.1$ & $  7.7 \pm 0.7$& $ 6144 \pm   2$ & $-29.77 \pm   0.02$ & $-29.43 \pm   0.02$& $-28.97 \pm   0.02$& $-28.64 \pm   0.02$& $-27.93 \pm   0.02$ \\
11 & $119.8 \pm 7.4$ & $ 14.2 \pm 2.7$& $ 6120 \pm  10$ & $-28.94 \pm   0.02$ & $-28.75 \pm   0.02$& $-28.32 \pm   0.02$& $-28.07 \pm   0.02$& $-27.31 \pm   0.02$ \\
12 & $157.5 \pm 7.5$ & $ 16.7 \pm 1.8$& $ 6112 \pm   4$ & $-28.97 \pm   0.02$ & $-28.96 \pm   0.01$& $-28.61 \pm   0.01$& $-28.37 \pm   0.01$& $-27.62 \pm   0.01$ \\
13 & $ 57.1 \pm 0.4$ & $ 17.9 \pm 0.3$& $ 6115 \pm   2$ & $-29.20 \pm   0.01$ & $-28.84 \pm   0.01$& $-28.47 \pm   0.01$& $-28.23 \pm   0.01$& $-27.73 \pm   0.04$ \\
14 & $ 90.0 \pm 16.5$ & $ 26.5 \pm 1.5$& $ 6116 \pm   3$ & $-29.11 \pm   0.02$ & $-28.82 \pm   0.01$& $-28.42 \pm   0.01$& $-28.19 \pm   0.01$& $-27.54 \pm   0.01$ \\
15 & $165.6 \pm 10.7$ & $ 12.7 \pm 2.8$& $ 6150 \pm   4$ & $-28.76 \pm   0.01$ & $-28.76 \pm   0.01$& $-28.40 \pm   0.01$& $-28.17 \pm   0.01$& $-27.45 \pm   0.02$ \\
16 & $ 38.7 \pm 9.9$ & $  7.2 \pm 8.7$& $ 6134 \pm   4$ & $-27.93 \pm   0.38$ & $-27.76 \pm   0.38$& $-27.73 \pm   0.38$& $-27.59 \pm   0.38$& $-27.21 \pm   0.38$ \\
17 & $ 83.2 \pm 0.7$ & $ 19.4 \pm 0.6$& $ 6125 \pm   1$ & $-29.30 \pm   0.01$ & $-28.95 \pm   0.01$& $-28.44 \pm   0.01$& $-28.17 \pm   0.01$& $-27.49 \pm   0.01$ \\
18 & $129.2 \pm 2.8$ & $  9.1 \pm 1.3$& $ 6155 \pm   4$ & $-29.14 \pm   0.02$ & $-28.98 \pm   0.01$& $-28.56 \pm   0.01$& $-28.28 \pm   0.01$& $-27.61 \pm   0.01$ \\
19 & $125.4 \pm 5.1$ & $ 26.9 \pm 1.6$& $ 6127 \pm   3$ & $-29.04 \pm   0.04$ & $-28.81 \pm   0.04$& $-28.38 \pm   0.04$& $-28.16 \pm   0.04$& $-27.57 \pm   0.04$ \\
20 & $104.8 \pm 0.9$ & $ 12.7 \pm 0.3$& $ 6132 \pm   1$ & $-28.58 \pm   0.02$ & $-28.36 \pm   0.02$& $-27.98 \pm   0.02$& $-27.77 \pm   0.02$& $-27.17 \pm   0.02$ \\
21 & $109.0 \pm 0.3$ & $  8.6 \pm 0.1$& $ 6130 \pm   2$ & $-28.35 \pm   0.01$ & $-28.15 \pm   0.01$& $-27.79 \pm   0.01$& $-27.58 \pm   0.01$& $-26.97 \pm   0.01$ \\
22 & $100.1 \pm 2.1$ & $ 13.7 \pm 1.1$& $ 6158 \pm   3$ & $-29.46 \pm   0.01$ & $-29.19 \pm   0.01$& $-28.71 \pm   0.01$& $-28.39 \pm   0.01$& $-27.77 \pm   0.01$ \\
23 & $106.1 \pm 10.4$ & $ 17.0 \pm 0.9$& $ 6148 \pm   4$ & $-29.54 \pm   0.02$ & $-29.28 \pm   0.01$& $-28.76 \pm   0.01$& $-28.42 \pm   0.01$& $-27.74 \pm   0.04$ \\
24 & $122.9 \pm 1.7$ & $ 11.4 \pm 0.8$& $ 6116 \pm   2$ & $-29.62 \pm   0.07$ & $-29.39 \pm   0.07$& $-28.95 \pm   0.07$& $-28.64 \pm   0.07$& $-27.91 \pm   0.07$ \\
25 & $105.1 \pm 2.2$ & $  8.6 \pm 1.3$& $ 6118 \pm   3$ & $-29.34 \pm   0.01$ & $-29.10 \pm   0.01$& $-28.73 \pm   0.01$& $-28.46 \pm   0.01$& $-27.82 \pm   0.01$ \\
26 & $158.3 \pm 3.3$ & $ 16.3 \pm 1.7$& $ 6099 \pm   4$ & $-28.90 \pm   0.01$ & $-28.81 \pm   0.01$& $-28.45 \pm   0.01$& $-28.22 \pm   0.01$& $-27.52 \pm   0.01$ \\
27 & $105.8 \pm 2.3$ & $  6.5 \pm 1.0$& $ 6122 \pm   2$ & $-29.38 \pm   0.01$ & $-29.07 \pm   0.01$& $-28.68 \pm   0.01$& $-28.37 \pm   0.01$& $-27.69 \pm   0.01$ \\
28 & $116.5 \pm 1.2$ & $ 19.8 \pm 0.9$& $ 6128 \pm   1$ & $-29.02 \pm   0.07$ & $-28.84 \pm   0.07$& $-28.40 \pm   0.07$& $-28.14 \pm   0.07$& $-27.52 \pm   0.07$ \\
29 & $116.2 \pm 14.9$ & $ 15.5 \pm 0.7$& $ 6125 \pm   3$ & $-29.30 \pm   0.01$ & $-29.15 \pm   0.01$& $-28.64 \pm   0.01$& $-28.35 \pm   0.01$& $-27.63 \pm   0.01$ \\
30 & $149.1 \pm 1.4$ & $ 14.9 \pm 0.7$& $ 6120 \pm   2$ & $-29.11 \pm   0.01$ & $-28.97 \pm   0.01$& $-28.50 \pm   0.01$& $-28.22 \pm   0.01$& $-27.46 \pm   0.01$ \\
31 & $ 42.8 \pm 1.2$ & $  4.0 \pm 3.1$& $ 6112 \pm   3$ & $-29.76 \pm   0.02$ & $-29.40 \pm   0.02$& $-29.04 \pm   0.02$& $-28.74 \pm   0.02$& $-28.11 \pm   0.02$ \\
32 & $ 30.2 \pm 4.4$ & $  4.0 \pm 3.0$& $ 6141 \pm   3$ & $-29.47 \pm   0.02$ & $-29.10 \pm   0.01$& $-28.75 \pm   0.01$& $-28.45 \pm   0.01$& $-27.87 \pm   0.01$ \\
33 & $117.6 \pm 2.7$ & $  9.1 \pm 1.0$& $ 6113 \pm   2$ & $-29.44 \pm   0.03$ & $-29.21 \pm   0.03$& $-28.72 \pm   0.03$& $-28.42 \pm   0.03$& $-27.68 \pm   0.03$ \\
34 & $104.6 \pm 2.4$ & $ 26.6 \pm 3.2$& $ 6104 \pm   4$ & $-29.16 \pm   0.02$ & $-28.82 \pm   0.02$& $-28.43 \pm   0.02$& $-28.20 \pm   0.02$& $-27.55 \pm   0.02$ \\
35 & $126.7 \pm 1.8$ & $ 21.1 \pm 1.1$& $ 6108 \pm   5$ & $-29.22 \pm   0.01$ & $-29.10 \pm   0.01$& $-28.63 \pm   0.01$& $-28.34 \pm   0.01$& $-27.62 \pm   0.01$ \\
36 & $111.4 \pm 2.3$ & $ 10.6 \pm 1.5$& $ 6130 \pm   3$ & $-29.54 \pm   0.01$ & $-29.23 \pm   0.01$& $-28.80 \pm   0.01$& $-28.46 \pm   0.01$& $-27.82 \pm   0.01$ \\
37 & $ 95.3 \pm 3.5$ & $ 23.3 \pm 1.5$& $ 6125 \pm   5$ & $-29.22 \pm   0.08$ & $-28.96 \pm   0.08$& $-28.50 \pm   0.08$& $-28.26 \pm   0.08$& $-27.56 \pm   0.08$ \\
38 & $ 31.5 \pm 3.7$ & $  4.5 \pm 1.3$& $ 6168 \pm   5$ & $-29.50 \pm   0.45$ & $-29.25 \pm   0.45$& $-28.89 \pm   0.45$& $-28.61 \pm   0.45$& $-27.97 \pm   0.45$ \\
39 & $135.7 \pm 1.1$ & $ 11.2 \pm 1.5$& $ 6088 \pm   2$ & $-29.13 \pm   0.02$ & $-28.99 \pm   0.01$& $-28.59 \pm   0.01$& $-28.31 \pm   0.01$& $-27.55 \pm   0.01$ \\
40 & $ 95.4 \pm 18.8$ & $ 24.1 \pm 1.0$& $ 6122 \pm   4$ & $-29.01 \pm   0.03$ & $-28.78 \pm   0.03$& $-28.35 \pm   0.03$& $-28.17 \pm   0.03$& $-27.54 \pm   0.03$ \\
41 & $115.6 \pm 2.8$ & $ 13.5 \pm 6.9$& $ 6068 \pm   4$ & $-29.33 \pm   0.02$ & $-29.12 \pm   0.02$& $-28.84 \pm   0.02$& $-28.49 \pm   0.02$& $-27.77 \pm   0.02$ \\
42 & $ 93.3 \pm 2.6$ & $  8.3 \pm 2.2$& $ 6102 \pm   3$ & $-28.58 \pm   0.01$ & $-28.47 \pm   0.01$& $-28.29 \pm   0.01$& $-28.05 \pm   0.01$& $-27.50 \pm   0.01$ \\
43 & $101.1 \pm 1.6$ & $  7.4 \pm 1.0$& $ 6122 \pm   3$ & $-28.01 \pm   0.01$ & $-28.00 \pm   0.01$& $-27.85 \pm   0.01$& $-27.77 \pm   0.01$& $-27.31 \pm   0.01$ \\
44 & $166.8 \pm 1.8$ & $ 21.8 \pm 1.2$& $ 6169 \pm   3$ & $-28.67 \pm   0.20$ & $-28.74 \pm   0.20$& $-28.30 \pm   0.20$& $-28.05 \pm   0.20$& $-27.28 \pm   0.20$ \\
45 & $110.6 \pm 1.0$ & $  8.0 \pm 1.3$& $ 6098 \pm   1$ & $-29.88 \pm   0.76$ & $-29.58 \pm   0.76$& $-29.09 \pm   0.76$& $-28.71 \pm   0.76$& $-27.91 \pm   0.76$ \\
46 & $ 94.1 \pm 1.3$ & $  6.7 \pm 0.6$& $ 6150 \pm   2$ & $-29.72 \pm   0.01$ & $-29.39 \pm   0.01$& $-28.97 \pm   0.01$& $-28.64 \pm   0.01$& $-27.96 \pm   0.01$ \\
47 & $105.9 \pm 5.4$ & $ 18.3 \pm 1.8$& $ 6124 \pm   4$ & $-29.23 \pm   0.02$ & $-29.04 \pm   0.01$& $-28.63 \pm   0.01$& $-28.35 \pm   0.01$& $-27.62 \pm   0.01$ \\
48 & $109.6 \pm 1.3$ & $  8.9 \pm 0.7$& $ 6142 \pm   2$ & $-28.54 \pm   0.01$ & $-28.40 \pm   0.01$& $-28.11 \pm   0.01$& $-27.89 \pm   0.01$& $-27.31 \pm   0.01$ \\
49 & $105.0 \pm 3.2$ & $  6.9 \pm 1.1$& $ 6124 \pm   7$ & $-27.89 \pm   0.02$ & $-27.82 \pm   0.02$& $-27.69 \pm   0.02$& $-27.63 \pm   0.02$& $-27.16 \pm   0.04$ \\
50 & $ 88.0 \pm 2.7$ & $ 32.2 \pm 1.9$& $ 6111 \pm   7$ & $-28.37 \pm   0.03$ & $-28.26 \pm   0.03$& $-27.92 \pm   0.03$& $-27.77 \pm   0.03$& $-27.13 \pm   0.03$ \\
51 & $108.9 \pm 3.5$ & $ 17.1 \pm 1.0$& $ 6131 \pm   6$ & $-29.25 \pm   0.02$ & $-29.02 \pm   0.02$& $-28.56 \pm   0.02$& $-28.32 \pm   0.01$& $-27.65 \pm   0.02$ \\
52 & $143.5 \pm 2.2$ & $ 10.2 \pm 0.9$& $ 6142 \pm   3$ & $-29.38 \pm   0.01$ & $-29.17 \pm   0.01$& $-28.69 \pm   0.01$& $-28.40 \pm   0.01$& $-27.62 \pm   0.01$ \\
53 & $150.5 \pm 8.9$ & $ 46.3 \pm 8.6$& $ 6110 \pm   8$ & $-28.70 \pm   0.02$ & $-28.61 \pm   0.02$& $-28.16 \pm   0.02$& $-27.91 \pm   0.02$& $-27.07 \pm   0.04$ \\
54 & $114.7 \pm 2.8$ & $  9.4 \pm 0.9$& $ 6145 \pm   3$ & $-28.53 \pm   0.02$ & $-28.41 \pm   0.01$& $-28.11 \pm   0.01$& $-27.99 \pm   0.01$& $-27.38 \pm   0.01$ \\
55 & $ 85.0 \pm 1.1$ & $  6.0 \pm 0.4$& $ 6140 \pm   3$ & $-29.73 \pm   0.01$ & $-29.43 \pm   0.01$& $-29.04 \pm   0.01$& $-28.72 \pm   0.01$& $-28.07 \pm   0.01$ \\
56 & $134.8 \pm 15.7$ & $ 19.7 \pm 3.7$& $ 6095 \pm   5$ & $-28.56 \pm   0.02$ & $-28.63 \pm   0.01$& $-28.32 \pm   0.01$& $-28.13 \pm   0.01$& $-27.42 \pm   0.01$ \\
57 & $129.6 \pm 5.6$ & $ 31.1 \pm 23.8$& $ 6151 \pm   4$ & $-27.88 \pm   0.02$ & $-28.12 \pm   0.01$& $-28.02 \pm   0.01$& $-27.86 \pm   0.01$& $-27.24 \pm   0.03$ \\
58 & $122.1 \pm 1.7$ & $  8.3 \pm 0.5$& $ 6123 \pm   2$ & $-29.63 \pm   0.01$ & $-29.38 \pm   0.01$& $-28.92 \pm   0.01$& $-28.60 \pm   0.01$& $-27.88 \pm   0.01$ \\
59 & $110.8 \pm 10.7$ & $ 20.4 \pm 1.0$& $ 6135 \pm   4$ & $-29.16 \pm   0.01$ & $-28.99 \pm   0.01$& $-28.59 \pm   0.01$& $-28.29 \pm   0.01$& $-27.61 \pm   0.01$ \\
60 & $ 87.0 \pm 1.4$ & $ 20.7 \pm 0.7$& $ 6132 \pm   2$ & $-28.78 \pm   0.01$ & $-28.66 \pm   0.01$& $-28.30 \pm   0.01$& $-28.08 \pm   0.01$& $-27.41 \pm   0.01$ \\
61 & $120.6 \pm 1.1$ & $  9.3 \pm 0.4$& $ 6132 \pm   4$ & $-29.72 \pm   0.04$ & $-29.43 \pm   0.03$& $-28.99 \pm   0.03$& $-28.64 \pm   0.03$& $-27.90 \pm   0.03$ \\
62 & $102.4 \pm 13.4$ & $ 19.2 \pm 0.6$& $ 6112 \pm   3$ & $-29.46 \pm   0.01$ & $-29.17 \pm   0.01$& $-28.71 \pm   0.01$& $-28.47 \pm   0.01$& $-27.83 \pm   0.01$ \\
63 & $141.1 \pm 3.4$ & $ 11.7 \pm 1.2$& $ 6128 \pm   3$ & $-29.28 \pm   0.15$ & $-29.16 \pm   0.15$& $-28.72 \pm   0.15$& $-28.41 \pm   0.15$& $-27.65 \pm   0.15$ \\
64 & $129.6 \pm 2.7$ & $ 17.3 \pm 1.0$& $ 6100 \pm   2$ & $-28.91 \pm   0.01$ & $-28.84 \pm   0.01$& $-28.46 \pm   0.01$& $-28.18 \pm   0.01$& $-27.43 \pm   0.01$ \\
65 & $146.8 \pm 3.5$ & $  7.5 \pm 2.5$& $ 6108 \pm   4$ & $-28.79 \pm   0.02$ & $-28.65 \pm   0.02$& $-28.35 \pm   0.02$& $-28.12 \pm   0.02$& $-27.46 \pm   0.02$ \\
66 & $117.4 \pm 2.2$ & $  9.0 \pm 1.2$& $ 6135 \pm   3$ & $-29.50 \pm   0.01$ & $-29.23 \pm   0.01$& $-28.80 \pm   0.01$& $-28.47 \pm   0.01$& $-27.74 \pm   0.01$ \\
67 & $159.1 \pm 0.7$ & $ 15.7 \pm 0.8$& $ 6130 \pm   1$ & $-29.37 \pm   0.02$ & $-29.16 \pm   0.02$& $-28.68 \pm   0.02$& $-28.41 \pm   0.02$& $-27.69 \pm   0.02$ \\
68 & $ 91.5 \pm 10.9$ & $ 15.6 \pm 0.5$& $ 6127 \pm   2$ & $-29.43 \pm   0.02$ & $-29.09 \pm   0.02$& $-28.60 \pm   0.01$& $-28.30 \pm   0.01$& $-27.62 \pm   0.04$ \\
69 & $107.8 \pm 14.1$ & $ 20.0 \pm 0.7$& $ 6132 \pm   3$ & $-28.82 \pm   0.02$ & $-28.78 \pm   0.01$& $-28.32 \pm   0.01$& $-28.09 \pm   0.01$& $-27.39 \pm   0.01$ \\
70 & $ 92.1 \pm 16.5$ & $ 21.1 \pm 0.7$& $ 6118 \pm   2$ & $-29.28 \pm   0.02$ & $-28.95 \pm   0.02$& $-28.46 \pm   0.02$& $-28.20 \pm   0.02$& $-27.62 \pm   0.04$ \\
71 & $128.6 \pm 2.8$ & $ 15.6 \pm 1.7$& $ 6150 \pm   4$ & $-29.35 \pm   0.02$ & $-29.11 \pm   0.02$& $-28.61 \pm   0.02$& $-28.32 \pm   0.02$& $-27.67 \pm   0.02$ \\
72 & $102.6 \pm 2.8$ & $ 15.6 \pm 1.1$& $ 6141 \pm   8$ & $-29.16 \pm   0.07$ & $-29.04 \pm   0.06$& $-28.56 \pm   0.06$& $-28.30 \pm   0.06$& $-27.63 \pm   0.06$ \\
73 & $120.0 \pm 1.1$ & $  7.3 \pm 0.6$& $ 6123 \pm   3$ & $-29.64 \pm   0.02$ & $-29.41 \pm   0.02$& $-28.99 \pm   0.02$& $-28.70 \pm   0.02$& $-28.00 \pm   0.02$ \\
74 & $130.0 \pm 1.0$ & $  6.1 \pm 4.2$& $ 6109 \pm   3$ & $-29.42 \pm   0.01$ & $-29.24 \pm   0.01$& $-28.84 \pm   0.01$& $-28.52 \pm   0.01$& $-27.76 \pm   0.01$ \\
75 & $111.9 \pm 1.3$ & $  9.8 \pm 0.7$& $ 6142 \pm   2$ & $-29.63 \pm   0.01$ & $-29.36 \pm   0.01$& $-28.88 \pm   0.01$& $-28.54 \pm   0.01$& $-27.81 \pm   0.01$ \\
76 & $135.2 \pm 1.4$ & $  5.6 \pm 1.6$& $ 6084 \pm   3$ & $-29.44 \pm   0.02$ & $-29.19 \pm   0.02$& $-28.79 \pm   0.02$& $-28.49 \pm   0.02$& $-27.74 \pm   0.02$ \\
77 & $ 65.8 \pm 4.1$ & $ 10.2 \pm 0.9$& $ 6133 \pm   2$ & $-29.53 \pm   0.02$ & $-29.30 \pm   0.02$& $-28.89 \pm   0.02$& $-28.56 \pm   0.02$& $-27.90 \pm   0.02$ \\
78 & $ 87.8 \pm 7.2$ & $ 21.5 \pm 0.9$& $ 6137 \pm   3$ & $-29.00 \pm   0.22$ & $-28.82 \pm   0.22$& $-28.37 \pm   0.22$& $-28.13 \pm   0.22$& $-27.43 \pm   0.22$ \\
79 & $138.1 \pm 2.1$ & $ 11.7 \pm 0.7$& $ 6133 \pm   3$ & $-29.40 \pm   0.02$ & $-29.23 \pm   0.02$& $-28.77 \pm   0.01$& $-28.45 \pm   0.02$& $-27.68 \pm   0.02$ \\
80 & $119.9 \pm 10.4$ & $ 16.4 \pm 1.7$& $ 6132 \pm   6$ & $-29.05 \pm   0.07$ & $-28.92 \pm   0.07$& $-28.50 \pm   0.07$& $-28.22 \pm   0.07$& $-27.55 \pm   0.07$ \\
81 & $101.8 \pm 2.6$ & $ 12.4 \pm 1.0$& $ 6158 \pm   5$ & $-29.46 \pm   0.02$ & $-29.27 \pm   0.02$& $-28.87 \pm   0.02$& $-28.60 \pm   0.02$& $-27.96 \pm   0.02$ \\
82 & $128.5 \pm 1.4$ & $ 10.6 \pm 1.0$& $ 6098 \pm   4$ & $-29.48 \pm   0.02$ & $-29.31 \pm   0.02$& $-28.87 \pm   0.01$& $-28.54 \pm   0.01$& $-27.77 \pm   0.02$ \\
83 & $114.4 \pm 1.2$ & $ 12.6 \pm 0.6$& $ 6136 \pm   3$ & $-29.59 \pm   0.23$ & $-29.34 \pm   0.23$& $-28.91 \pm   0.23$& $-28.58 \pm   0.23$& $-27.87 \pm   0.23$ \\
84 & $ 93.7 \pm 1.0$ & $  7.8 \pm 0.4$& $ 6171 \pm   2$ & $-29.38 \pm   0.02$ & $-29.12 \pm   0.01$& $-28.74 \pm   0.01$& $-28.44 \pm   0.01$& $-27.78 \pm   0.01$ \\
85 & $ 79.4 \pm 3.2$ & $ 12.6 \pm 0.7$& $ 6144 \pm   2$ & $-28.90 \pm   0.02$ & $-28.62 \pm   0.01$& $-28.27 \pm   0.01$& $-28.08 \pm   0.01$& $-27.54 \pm   0.01$ \\
86 & $ 39.0 \pm 22.2$ & $  4.5 \pm 0.2$& $ 6139 \pm   3$ & $-29.46 \pm   0.02$ & $-29.06 \pm   0.02$& $-28.75 \pm   0.02$& $-28.49 \pm   0.02$& $-27.93 \pm   0.02$ \\
87 & $ 75.6 \pm 1.1$ & $  2.2 \pm 0.4$& $ 6142 \pm   3$ & $-29.92 \pm   0.32$ & $-29.49 \pm   0.32$& $-29.02 \pm   0.32$& $-28.63 \pm   0.32$& $-27.90 \pm   0.32$ \\
88 & $ 98.8 \pm 12.1$ & $ 25.9 \pm 0.7$& $ 6114 \pm   7$ & $-29.14 \pm   0.02$ & $-28.91 \pm   0.02$& $-28.44 \pm   0.02$& $-28.21 \pm   0.02$& $-27.49 \pm   0.02$ \\
89 & $ 81.3 \pm 2.3$ & $  6.3 \pm 0.9$& $ 6142 \pm   5$ & $-29.89 \pm   0.02$ & $-29.46 \pm   0.02$& $-28.97 \pm   0.02$& $-28.62 \pm   0.02$& $-27.97 \pm   0.02$ \\
90 & $129.5 \pm 5.6$ & $  6.3 \pm 2.5$& $ 6110 \pm   6$ & $-29.53 \pm   0.02$ & $-29.31 \pm   0.02$& $-28.95 \pm   0.02$& $-28.60 \pm   0.02$& $-27.96 \pm   0.02$ \\
91 & $106.5 \pm 1.8$ & $ 21.4 \pm 1.7$& $ 6162 \pm   4$ & $-28.98 \pm   0.02$ & $-28.92 \pm   0.02$& $-28.66 \pm   0.02$& $-28.42 \pm   0.02$& $-27.77 \pm   0.02$ \\
92 & $109.1 \pm 9.4$ & $ 15.0 \pm 0.6$& $ 6134 \pm   2$ & $-29.36 \pm   0.02$ & $-29.14 \pm   0.01$& $-28.69 \pm   0.01$& $-28.41 \pm   0.01$& $-27.70 \pm   0.01$ \\
93 & $ 97.7 \pm 2.8$ & $ 12.6 \pm 1.4$& $ 6142 \pm   5$ & $-29.50 \pm   0.02$ & $-29.11 \pm   0.02$& $-28.61 \pm   0.02$& $-28.34 \pm   0.02$& $-27.71 \pm   0.02$ \\
94 & $102.7 \pm 1.8$ & $ 11.9 \pm 1.0$& $ 6137 \pm   4$ & $-29.28 \pm   0.02$ & $-29.14 \pm   0.02$& $-28.74 \pm   0.02$& $-28.46 \pm   0.01$& $-27.76 \pm   0.03$ \\
95 & $133.6 \pm 2.1$ & $  6.9 \pm 0.9$& $ 6127 \pm   3$ & $-29.09 \pm   0.01$ & $-28.93 \pm   0.01$& $-28.59 \pm   0.01$& $-28.31 \pm   0.01$& $-27.55 \pm   0.03$ \\
96 & $127.8 \pm 2.5$ & $ 15.8 \pm 1.3$& $ 6117 \pm   4$ & $-28.80 \pm   0.02$ & $-28.68 \pm   0.02$& $-28.37 \pm   0.02$& $-28.18 \pm   0.02$& $-27.55 \pm   0.02$ \\
97 & $134.2 \pm 0.5$ & $  7.4 \pm 0.2$& $ 6097 \pm   1$ & $-29.91 \pm   0.01$ & $-29.70 \pm   0.01$& $-29.31 \pm   0.01$& $-28.94 \pm   0.01$& $-28.10 \pm   0.01$ \\
98 & $100.2 \pm 2.8$ & $ 13.4 \pm 1.1$& $ 6107 \pm   8$ & $-29.33 \pm   0.02$ & $-29.11 \pm   0.02$& $-28.77 \pm   0.02$& $-28.52 \pm   0.02$& $-27.80 \pm   0.02$ \\
99 & $ 88.4 \pm 1.2$ & $  5.4 \pm 0.4$& $ 6168 \pm   2$ & $-29.33 \pm   0.01$ & $-29.05 \pm   0.01$& $-28.68 \pm   0.01$& $-28.40 \pm   0.01$& $-27.79 \pm   0.01$ \\
100 & $125.2 \pm 4.6$ & $ 17.1 \pm 2.0$& $ 6124 \pm  11$ & $-28.58 \pm   0.02$ & $-28.54 \pm   0.02$& $-28.19 \pm   0.02$& $-27.99 \pm   0.02$& $-27.31 \pm   0.02$ \\
101 & $141.7 \pm 2.6$ & $  7.7 \pm 1.0$& $ 6134 \pm   5$ & $-29.38 \pm   0.02$ & $-29.20 \pm   0.02$& $-28.77 \pm   0.02$& $-28.46 \pm   0.02$& $-27.67 \pm   0.02$ \\
102 & $ 22.7 \pm 21.4$ & $  0.9 \pm 0.6$& $ 6197 \pm   7$ & $-29.87 \pm   0.01$ & $-29.43 \pm   0.01$& $-29.04 \pm   0.01$& $-28.74 \pm   0.01$& $-28.11 \pm   0.01$ \\
103 & $127.2 \pm 1.9$ & $ 21.7 \pm 1.2$& $ 6134 \pm   3$ & $-29.14 \pm   0.01$ & $-29.00 \pm   0.01$& $-28.59 \pm   0.01$& $-28.35 \pm   0.01$& $-27.65 \pm   0.01$ \\
104 & $162.3 \pm 9.8$ & $ 19.6 \pm 1.4$& $ 6076 \pm   6$ & $-28.86 \pm   0.02$ & $-28.87 \pm   0.01$& $-28.42 \pm   0.01$& $-28.19 \pm   0.01$& $-27.50 \pm   0.03$ \\
105 & $138.5 \pm 4.0$ & $ 14.0 \pm 2.0$& $ 6111 \pm   2$ & $-27.59 \pm   0.02$ & $-27.78 \pm   0.01$& $-27.73 \pm   0.01$& $-27.71 \pm   0.01$& $-27.19 \pm   0.02$ \\
106 & $ 88.4 \pm 21.6$ & $ 22.3 \pm 1.1$& $ 6141 \pm   3$ & $-29.11 \pm   0.01$ & $-28.89 \pm   0.01$& $-28.45 \pm   0.01$& $-28.23 \pm   0.01$& $-27.60 \pm   0.01$ \\
107 & $109.4 \pm 2.2$ & $ 11.5 \pm 1.7$& $ 6142 \pm   6$ & $-29.35 \pm   0.13$ & $-29.17 \pm   0.13$& $-28.68 \pm   0.13$& $-28.38 \pm   0.13$& $-27.56 \pm   0.13$ \\
108 & $103.7 \pm 17.0$ & $ 23.1 \pm 0.4$& $ 6118 \pm   2$ & $-28.67 \pm   0.01$ & $-28.52 \pm   0.01$& $-28.18 \pm   0.01$& $-28.03 \pm   0.01$& $-27.47 \pm   0.01$ \\
109 & $ 93.8 \pm 12.2$ & $ 23.8 \pm 3.0$& $ 6122 \pm   4$ & $-28.87 \pm   0.02$ & $-28.70 \pm   0.01$& $-28.26 \pm   0.01$& $-28.08 \pm   0.01$& $-27.41 \pm   0.02$ \\
110 & $122.5 \pm 3.8$ & $ 12.7 \pm 1.6$& $ 6131 \pm   4$ & $-29.10 \pm   0.01$ & $-28.96 \pm   0.01$& $-28.51 \pm   0.01$& $-28.25 \pm   0.01$& $-27.57 \pm   0.01$ \\
111 & $101.7 \pm 1.3$ & $  4.8 \pm 0.8$& $ 6126 \pm   2$ & $-29.43 \pm   0.01$ & $-29.14 \pm   0.01$& $-28.74 \pm   0.01$& $-28.43 \pm   0.01$& $-27.76 \pm   0.01$ \\
112 & $ 89.8 \pm 9.9$ & $ 23.4 \pm 1.1$& $ 6109 \pm   5$ & $-29.11 \pm   0.02$ & $-28.91 \pm   0.02$& $-28.50 \pm   0.02$& $-28.27 \pm   0.02$& $-27.57 \pm   0.02$ \\
113 & $ 93.9 \pm 2.4$ & $ 18.5 \pm 1.3$& $ 6108 \pm   4$ & $-29.19 \pm   0.02$ & $-29.03 \pm   0.01$& $-28.59 \pm   0.01$& $-28.30 \pm   0.01$& $-27.55 \pm   0.02$ \\
114 & $ 61.8 \pm 6.5$ & $ 27.1 \pm 1.8$& $ 6154 \pm   8$ & $-28.37 \pm   0.02$ & $-28.27 \pm   0.02$& $-27.97 \pm   0.02$& $-27.80 \pm   0.02$& $-27.24 \pm   0.02$ \\
115 & $ 64.6 \pm 3.8$ & $  3.7 \pm 1.4$& $ 6147 \pm   4$ & $-29.49 \pm   0.02$ & $-29.06 \pm   0.02$& $-28.71 \pm   0.02$& $-28.42 \pm   0.02$& $-27.93 \pm   0.02$ \\
116 & $127.3 \pm 2.4$ & $ 15.5 \pm 1.2$& $ 6127 \pm   2$ & $-28.77 \pm   0.01$ & $-28.66 \pm   0.01$& $-28.37 \pm   0.01$& $-28.15 \pm   0.01$& $-27.52 \pm   0.01$ \\
117 & $ 77.5 \pm 2.5$ & $  5.1 \pm 0.8$& $ 6140 \pm   4$ & $-29.96 \pm   0.06$ & $-29.56 \pm   0.06$& $-29.06 \pm   0.06$& $-28.75 \pm   0.06$& $-28.01 \pm   0.06$ \\
118 & $105.5 \pm 3.0$ & $  6.2 \pm 1.2$& $ 6120 \pm   5$ & $-29.54 \pm   0.01$ & $-29.23 \pm   0.01$& $-28.74 \pm   0.01$& $-28.40 \pm   0.01$& $-27.64 \pm   0.01$ \\
119 & $107.2 \pm 3.5$ & $ 18.6 \pm 1.3$& $ 6132 \pm  14$ & $-28.80 \pm   0.02$ & $-28.69 \pm   0.02$& $-28.34 \pm   0.02$& $-28.13 \pm   0.02$& $-27.52 \pm   0.02$ \\
120 & $ 99.1 \pm 2.7$ & $  5.8 \pm 0.8$& $ 6164 \pm   5$ & $-29.67 \pm   0.01$ & $-29.34 \pm   0.01$& $-28.94 \pm   0.01$& $-28.60 \pm   0.01$& $-27.96 \pm   0.01$ \\
121 & $126.5 \pm 1.2$ & $ 15.4 \pm 1.1$& $ 6155 \pm   2$ & $-28.67 \pm   0.02$ & $-28.60 \pm   0.02$& $-28.31 \pm   0.02$& $-28.13 \pm   0.02$& $-27.58 \pm   0.02$ \\
122 & $ 87.7 \pm 3.6$ & $  7.3 \pm 1.3$& $ 6132 \pm   3$ & $-28.60 \pm   0.02$ & $-28.42 \pm   0.02$& $-28.19 \pm   0.02$& $-28.05 \pm   0.02$& $-27.53 \pm   0.02$ \\
123 & $121.0 \pm 5.3$ & $  9.3 \pm 3.1$& $ 6126 \pm   3$ & $-29.75 \pm   0.01$ & $-29.42 \pm   0.01$& $-28.99 \pm   0.01$& $-28.68 \pm   0.01$& $-27.97 \pm   0.01$ \\
124 & $133.0 \pm 1.5$ & $ 17.6 \pm 0.8$& $ 6112 \pm   4$ & $-29.12 \pm   0.01$ & $-28.98 \pm   0.01$& $-28.54 \pm   0.01$& $-28.28 \pm   0.01$& $-27.51 \pm   0.01$ \\
125 & $106.4 \pm 2.1$ & $ 26.7 \pm 1.3$& $ 6100 \pm   2$ & $-29.00 \pm   0.01$ & $-28.83 \pm   0.01$& $-28.42 \pm   0.01$& $-28.20 \pm   0.01$& $-27.53 \pm   0.04$ \\
126 & $110.3 \pm 1.6$ & $ 14.2 \pm 0.7$& $ 6133 \pm   3$ & $-29.34 \pm   0.01$ & $-29.04 \pm   0.01$& $-28.57 \pm   0.01$& $-28.32 \pm   0.01$& $-27.66 \pm   0.01$ \\
127 & $ 99.5 \pm 1.6$ & $ 30.0 \pm 0.7$& $ 6114 \pm   2$ & $-28.81 \pm   0.01$ & $-28.65 \pm   0.01$& $-28.23 \pm   0.01$& $-28.02 \pm   0.01$& $-27.33 \pm   0.01$ \\
128 & $138.0 \pm 5.1$ & $ 16.2 \pm 1.6$& $ 6111 \pm   3$ & $-27.84 \pm   0.02$ & $-27.92 \pm   0.02$& $-27.69 \pm   0.02$& $-27.60 \pm   0.02$& $-26.99 \pm   0.02$ \\
129 & $ 80.8 \pm 2.4$ & $  4.3 \pm 0.8$& $ 6139 \pm   3$ & $-29.54 \pm   0.02$ & $-29.16 \pm   0.01$& $-28.87 \pm   0.01$& $-28.54 \pm   0.01$& $-28.01 \pm   0.01$ \\
130 & $124.2 \pm 2.4$ & $  6.8 \pm 1.1$& $ 6120 \pm   2$ & $-29.52 \pm   0.01$ & $-29.25 \pm   0.01$& $-28.80 \pm   0.01$& $-28.49 \pm   0.01$& $-27.76 \pm   0.01$ \\
131 & $103.6 \pm 7.2$ & $ 27.2 \pm 1.9$& $ 6120 \pm   5$ & $-28.74 \pm   0.02$ & $-28.46 \pm   0.01$& $-28.09 \pm   0.01$& $-27.87 \pm   0.01$& $-27.26 \pm   0.04$ \\
132 & $ 85.5 \pm 0.6$ & $  5.9 \pm 0.3$& $ 6130 \pm   4$ & $-29.87 \pm   0.02$ & $-29.45 \pm   0.02$& $-28.94 \pm   0.02$& $-28.57 \pm   0.02$& $-27.85 \pm   0.02$ \\
133 & $118.1 \pm 2.1$ & $  4.6 \pm 2.2$& $ 6101 \pm   3$ & $-29.77 \pm   0.02$ & $-29.52 \pm   0.02$& $-29.08 \pm   0.02$& $-28.74 \pm   0.01$& $-27.94 \pm   0.02$ \\
134 & $131.4 \pm 2.2$ & $ 17.4 \pm 1.1$& $ 6116 \pm   4$ & $-28.60 \pm   0.02$ & $-28.55 \pm   0.02$& $-28.25 \pm   0.02$& $-28.03 \pm   0.02$& $-27.34 \pm   0.02$ \\
135 & $ 94.1 \pm 2.0$ & $ 11.2 \pm 0.6$& $ 6131 \pm   3$ & $-29.50 \pm   0.02$ & $-29.11 \pm   0.02$& $-28.67 \pm   0.02$& $-28.37 \pm   0.02$& $-27.71 \pm   0.02$ \\
136 & $126.9 \pm 2.5$ & $ 10.0 \pm 0.7$& $ 6121 \pm   2$ & $-29.38 \pm   0.01$ & $-29.13 \pm   0.01$& $-28.69 \pm   0.01$& $-28.37 \pm   0.01$& $-27.61 \pm   0.01$ \\
137 & $106.9 \pm 0.6$ & $  7.1 \pm 0.7$& $ 6134 \pm   3$ & $-29.29 \pm   0.73$ & $-29.00 \pm   0.73$& $-28.51 \pm   0.73$& $-28.15 \pm   0.73$& $-27.38 \pm   0.73$ \\
138 & $ 82.6 \pm 17.5$ & $ 12.2 \pm 1.4$& $ 6139 \pm   3$ & $-29.51 \pm   0.02$ & $-29.14 \pm   0.01$& $-28.60 \pm   0.01$& $-28.30 \pm   0.01$& $-27.71 \pm   0.02$ \\
139 & $ 78.9 \pm 20.2$ & $ 21.1 \pm 1.4$& $ 6128 \pm   6$ & $-29.37 \pm   0.05$ & $-29.04 \pm   0.05$& $-28.54 \pm   0.05$& $-28.30 \pm   0.05$& $-27.57 \pm   0.05$ \\
140 & $125.5 \pm 3.2$ & $ 10.1 \pm 1.6$& $ 6121 \pm   3$ & $-29.34 \pm   0.01$ & $-29.12 \pm   0.01$& $-28.65 \pm   0.01$& $-28.38 \pm   0.01$& $-27.66 \pm   0.01$ \\
141 & $132.6 \pm 1.4$ & $ 15.2 \pm 1.0$& $ 6113 \pm   2$ & $-29.12 \pm   0.01$ & $-28.95 \pm   0.01$& $-28.53 \pm   0.01$& $-28.27 \pm   0.01$& $-27.55 \pm   0.01$ \\
142 & $ 95.9 \pm 2.3$ & $ 10.6 \pm 1.1$& $ 6143 \pm   6$ & $-29.59 \pm   0.02$ & $-29.22 \pm   0.02$& $-28.68 \pm   0.02$& $-28.42 \pm   0.02$& $-27.77 \pm   0.02$ \\
143 & $167.5 \pm 2.2$ & $ 20.4 \pm 0.6$& $ 6090 \pm   2$ & $-28.75 \pm   0.02$ & $-28.78 \pm   0.02$& $-28.40 \pm   0.02$& $-28.18 \pm   0.02$& $-27.41 \pm   0.02$ \\
144 & $112.1 \pm 2.5$ & $  9.8 \pm 1.0$& $ 6118 \pm   3$ & $-29.34 \pm   0.01$ & $-29.09 \pm   0.01$& $-28.68 \pm   0.01$& $-28.39 \pm   0.01$& $-27.70 \pm   0.01$ \\
145 & $135.8 \pm 1.8$ & $  7.5 \pm 0.9$& $ 6135 \pm   4$ & $-29.22 \pm   0.02$ & $-29.02 \pm   0.02$& $-28.61 \pm   0.02$& $-28.35 \pm   0.01$& $-27.71 \pm   0.02$ \\
146 & $117.6 \pm 2.6$ & $  8.9 \pm 2.0$& $ 6119 \pm   3$ & $-28.84 \pm   0.01$ & $-28.70 \pm   0.01$& $-28.35 \pm   0.01$& $-28.16 \pm   0.01$& $-27.50 \pm   0.01$ \\
147 & $ 78.5 \pm 1.8$ & $  6.3 \pm 0.9$& $ 6140 \pm   3$ & $-29.71 \pm   0.01$ & $-29.34 \pm   0.01$& $-28.93 \pm   0.01$& $-28.61 \pm   0.01$& $-27.92 \pm   0.01$ \\
148 & $137.2 \pm 2.5$ & $  9.3 \pm 1.1$& $ 6112 \pm   7$ & $-29.27 \pm   0.03$ & $-29.09 \pm   0.03$& $-28.73 \pm   0.03$& $-28.38 \pm   0.03$& $-27.68 \pm   0.03$ \\
149 & $ 99.3 \pm 3.6$ & $ 16.8 \pm 0.7$& $ 6129 \pm   3$ & $-28.29 \pm   0.02$ & $-28.27 \pm   0.01$& $-28.05 \pm   0.01$& $-27.95 \pm   0.01$& $-27.40 \pm   0.01$ \\
150 & $122.4 \pm 2.7$ & $ 21.2 \pm 0.8$& $ 6114 \pm   4$ & $-29.07 \pm   0.01$ & $-28.94 \pm   0.01$& $-28.50 \pm   0.01$& $-28.26 \pm   0.01$& $-27.53 \pm   0.03$ \\
151 & $112.2 \pm 2.7$ & $  7.8 \pm 1.0$& $ 6143 \pm   4$ & $-29.42 \pm   0.01$ & $-29.17 \pm   0.01$& $-28.78 \pm   0.01$& $-28.46 \pm   0.01$& $-27.78 \pm   0.01$ \\
152 & $104.5 \pm 5.5$ & $ 24.4 \pm 2.2$& $ 6123 \pm   9$ & $-29.20 \pm   0.05$ & $-29.01 \pm   0.05$& $-28.48 \pm   0.05$& $-28.23 \pm   0.05$& $-27.54 \pm   0.05$ \\
153 & $143.3 \pm 1.6$ & $ 18.9 \pm 1.3$& $ 6103 \pm   4$ & $-28.78 \pm   0.02$ & $-28.79 \pm   0.02$& $-28.44 \pm   0.02$& $-28.18 \pm   0.02$& $-27.45 \pm   0.02$ \\
154 & $ 61.8 \pm 3.5$ & $  3.3 \pm 0.9$& $ 6149 \pm   4$ & $-29.86 \pm   0.01$ & $-29.41 \pm   0.01$& $-29.03 \pm   0.01$& $-28.70 \pm   0.01$& $-28.06 \pm   0.01$ \\
155 & $ 73.9 \pm 2.4$ & $ 12.8 \pm 0.8$& $ 6126 \pm   2$ & $-29.29 \pm   0.02$ & $-28.94 \pm   0.02$& $-28.53 \pm   0.01$& $-28.35 \pm   0.01$& $-27.76 \pm   0.01$ \\
156 & $ 95.4 \pm 41.8$ & $ 35.7 \pm 2.8$& $ 6135 \pm   4$ & $-28.71 \pm   0.01$ & $-28.58 \pm   0.01$& $-28.19 \pm   0.01$& $-27.99 \pm   0.01$& $-27.34 \pm   0.01$ \\
157 & $122.8 \pm 2.3$ & $  9.7 \pm 1.7$& $ 6124 \pm   2$ & $-29.04 \pm   0.02$ & $-28.86 \pm   0.02$& $-28.56 \pm   0.02$& $-28.30 \pm   0.02$& $-27.64 \pm   0.02$ \\
158 & $111.6 \pm 2.6$ & $  6.4 \pm 1.1$& $ 6116 \pm   3$ & $-29.41 \pm   0.01$ & $-29.15 \pm   0.01$& $-28.70 \pm   0.01$& $-28.38 \pm   0.01$& $-27.73 \pm   0.04$ \\
159 & $ 98.1 \pm 1.6$ & $ 10.6 \pm 2.1$& $ 6146 \pm   4$ & $-29.23 \pm   0.01$ & $-29.05 \pm   0.01$& $-28.71 \pm   0.01$& $-28.45 \pm   0.01$& $-27.79 \pm   0.01$ \\
160 & $118.2 \pm 1.5$ & $ 11.3 \pm 1.8$& $ 6157 \pm   5$ & $-29.13 \pm   0.08$ & $-29.01 \pm   0.07$& $-28.62 \pm   0.07$& $-28.32 \pm   0.07$& $-27.68 \pm   0.07$ \\
161 & $118.8 \pm 2.1$ & $ 15.3 \pm 1.3$& $ 6112 \pm   3$ & $-29.35 \pm   0.01$ & $-29.15 \pm   0.01$& $-28.69 \pm   0.01$& $-28.38 \pm   0.01$& $-27.68 \pm   0.01$ \\
162 & $108.8 \pm 2.7$ & $ 13.5 \pm 0.9$& $ 6153 \pm   4$ & $-29.44 \pm   0.02$ & $-29.19 \pm   0.01$& $-28.74 \pm   0.01$& $-28.42 \pm   0.01$& $-27.78 \pm   0.01$ \\\enddata
\end{deluxetable}

\section{Supernova Model}
\label{model:sec}
We hypothesize that at peak brightness
Type Ia supernova broadband magnitudes and colors are correlated with
spectral features; equivalent widths and line velocities, being insensitive to broadband color effects, are considered.
Spectra may not efficiently capture all variations in broadband magnitude, so an additional source
of intrinsic variation is allowed.
The intrinsic spectral and color parameters may not deterministically predict magnitudes, but rather do so with some dispersion.
An extrinsic physical process (i.e.\ dust) subsequently determines the apparent magnitudes.

We assume 
that  intrinsic $UBVRI$ magnitudes are linearly dependent
on the equivalent widths of the Ca and Si spectral features
$EW_{Ca}$, $EW_{Si}$,
the SiII$\lambda6355$ velocity $v_{Si}$:
these spectral features have been previously associated with SN~Ia diversity  
\citep{2006PASP..118..560B, 2009PASP..121..238B, 2009ApJ...699L.139W, 2011ApJ...729...55F,2009A&A...500L..17B}.
Note that $EW_{Si}$ at peak brightness is an effective proxy for the light-curve shape parameter
\citep{2011A&A...529L...4C}. 
A color-variation parameter $D$ describes further magnitude/color variation unaccounted for by the spectral features.
Residual dispersion is described by a Normal distribution with (unknown) covariance $C_c$.  A grey magnitude offset $\Delta$ is included for each supernova
to capture intrinsic dispersion and distance errors introduced when normalizing the supernovae to be at a common distance.
The observed magnitudes are linearly dependent on an
extrinsic color-variation parameter $k$  parameter.  The observables
$U_o, B_o, V_o, R_o, I_o$, $EW_{Ca,o}$, $EW_{Si,o}$, $v_{Si,o}$ have Gaussian measurement uncertainty with covariance $C$; these have
been presented in \S\ref{data:sec}.
The model is written as
\begin{equation}
\begin{pmatrix}
U\\B\\V\\R\\I
\end{pmatrix}
\sim \mathcal{N}
\left(
\Delta +
\begin{pmatrix}
c_0+\alpha_0 EW_{Ca} + \beta_0 EW_{Si} + \eta_0 v_{Si} + \delta_0 D\\
c_1+\alpha_1 EW_{Ca} + \beta_1 EW_{Si} + \eta_1 v_{Si} + \delta_1 D \\
c_2+\alpha_2 EW_{Ca} + \beta_2 EW_{Si} + \eta_2 v_{Si} + \delta_2 D\\
c_3+\alpha_3 EW_{Ca} + \beta_3 EW_{Si} + \eta_3 v_{Si} + \delta_3 D\\
c_4+\alpha_4 EW_{Ca} + \beta_4 EW_{Si}+ \eta_4 v_{Si} + \delta_4 D
\end{pmatrix}
,C_{c}
\right)
\label{ewsiv:eqn}
\end{equation}
\begin{equation}
\begin{pmatrix}
U_o\\B_o\\ V_o\\R_o\\I_o\\EW_{Si, o}\\ EW_{Ca, o} \\ v_{Si, o}
\end{pmatrix}
\sim \mathcal{N}
\left(
\begin{pmatrix}
U +\gamma_0 k \\B +\gamma_1 k \\V+\gamma_2 k\\R+\gamma_3 k\\I+\gamma_4 k\\
EW_{Si}\\ EW_{Ca} \\ v_{Si}
\end{pmatrix}
,C
\right).
\label{dust:eqn}
\end{equation}
The global parameters $c$, $\alpha$, $\beta$, $\eta$, and $\delta$  are the intercepts and slopes of the linear relationships between the spectral
features and intrinsic colors (encoded in $D$),  with intrinsic magnitudes.  The global parameters $\gamma$ are the slopes that connect the extrinsic colors
(encoded in $k$) and observed magnitude.  To constrain the degrees of freedom and degeneracies inherent in the model we impose
\begin{equation}
\langle \Delta \rangle=0, \langle k \rangle=0, \langle D \rangle=0, \gamma_0 > 0, \delta_0 < 0.
\end{equation}
The first two are imposed to constrain the definition of zero color (i.e. $E(B-V)=0$).    The latter two exclude degenerate posterior space
associated
with  simultaneous sign flips of
$\delta$--$D$ and $\gamma$--$k$.
Although not formally a degeneracy in the model, practically the parameter pairs  $\delta$--$D$ and $\gamma$--$k$ are degenerate with
each other,
as confirmed after recalculation of the posterior with exchange of the initial conditions and boundary conditions of $\delta$ and $\gamma$.
The $\delta$ and $\gamma$ values of a single run cannot definitively be attributed as being intrinsic or extrinsic; we break the degeneracy
by assigning the parameter whose values that are more consistent with dust as $\gamma$ and the other parameter as $\delta$.


Having the intrinsic dispersion $C_c$ as fit parameters seemingly introduces degeneracy in the model, as magnitude and color variation
ascribed to $\Delta$, $\delta D$, and $\gamma k$ could also be attributed to intrinsic dispersion.  There are several features of the model
that drives the assignation of variations away from $C_c$:  Maximizing the posterior disfavors the increase of $\det{(C_c)}$;
The distributions of $\Delta$ and $k$ turn out to
be non-Gaussian, and so are not well described by a Normal covariance; Grey offsets would appear as a constant
in all elements of the covariance matrix, which is disfavored for the Bayesian prior selected for $C_c$.

If interpreted as being due to dust, the extrinsic term could be associated with extinction $A_X = \gamma_X k$.  In
this parlance our parameters can be re-expressed as the more familiar color excess
$E(B-V) = (\gamma_1-\gamma_2) k$ and total-to-selective extinction $R_X = \frac{\gamma_X}{\gamma_1-\gamma_2}$.
Similarly we introduce for the intrinsic color term $E_\delta(B-V) = (\delta_1-\delta_2) D$ and $R_{\delta X} = \frac{\delta_X}{\delta_1-\delta_2}$.

In a Bayesian analysis the priors must be described.  A flat prior is used for all parameters except
for the covariance matrix $C_c$, which is constructed from a correlation matrix with  $\nu=4$  LKJ prior
\citep{Lewandowski20091989} and standard
deviations with a  Cauchy distribution prior with
 $\mu=0.1$, $\sigma=0.1$ mag and restricted to non-negative values.

\section{Results}
\label{results:sec}
The posterior of the model parameters is evaluated using Hamiltonian Monte Carlo as implemented in
STAN \citep{stan}. STAN provides output statistics to assess
the convergence of the Markov chains.  The output gives $\hat{R} \sim 1.0$ for all parameters, meaning there is no evidence for non-convergence.  The
output also gives  $N_{eff} \gg 100$ for all parameters, indicating that there is no evidence for sparse sampling of the parameter space.
Empirically, the confidence regions are localized and unimodal as can be seen in  Figure~\ref{global:fig}.  There is no evidence that
the Monte Carlo chains have not converged to the stationary posterior distribution.

\begin{figure}[htbp] %  figure placement: here, top, bottom, or page
   \centering
   \includegraphics[width=2.5in]{output11/coeff0.pdf} 
   \includegraphics[width=2.5in]{output11/coeff1.pdf} 
   \includegraphics[width=2.5in]{output11/coeff2.pdf} 
      \includegraphics[width=2.5in]{output11/coeff3.pdf} 
         \includegraphics[width=2.5in]{output11/coeff4.pdf} 
            \caption{Distributions for $c$, $\alpha$, $\beta$, $\eta$, $\gamma$, $\delta$, and $\sigma$ in each of the 5 bands.   \label{global:fig}}
\end{figure}


For each of the five filters, the 68\% intervals for the global parameters $\alpha$, $\beta$, $\eta$, $R$, $R_\delta$, and $\sigma = \sqrt{C_{c,ii}}$
(and other interesting derived parameters)
are given in Table~\ref{global:tab}.
The two dimensional contours for the parameters, grouped by filter, are shown in Figure~\ref{global:fig}.

\begin{table}
\centering
\begin{tabular}{|c|c|c|c|c|c|}
\hline
& $X=0$ &1&2&3&4\\ \hline
$\alpha_{X}$
&
$0.0042^{0.0009}_{-0.0009}$
&
$0.0014^{0.0007}_{-0.0008}$
&
$0.0014^{0.0006}_{-0.0006}$
&
$0.0014^{0.0005}_{-0.0005}$
&
$0.0025^{0.0005}_{-0.0005}$
\\
$\alpha_X/\alpha_2-1$
&
$   2.0^{   1.3}_{  -0.6}$
&
$   0.0^{   0.1}_{  -0.2}$
&
0
&
$   0.0^{   0.2}_{  -0.1}$
&
$   0.8^{   0.9}_{  -0.4}$
\\
$\beta_{X}$
&
$ 0.033^{ 0.003}_{-0.003}$
&
$ 0.027^{ 0.003}_{-0.003}$
&
$ 0.027^{ 0.002}_{-0.002}$
&
$ 0.021^{ 0.002}_{-0.002}$
&
$ 0.020^{ 0.002}_{-0.002}$
\\
$\beta_X/\beta_2-1$
&
$  0.26^{  0.05}_{ -0.05}$
&
$  0.00^{  0.03}_{ -0.03}$
&
0
&
$ -0.19^{  0.01}_{ -0.01}$
&
$ -0.25^{  0.03}_{ -0.03}$
\\
$\eta_{X}$
&
$0.0010^{0.0012}_{-0.0013}$
&
$0.0006^{0.0010}_{-0.0011}$
&
$0.0008^{0.0008}_{-0.0009}$
&
$0.0008^{0.0007}_{-0.0008}$
&
$0.0002^{0.0006}_{-0.0007}$
\\
$\eta_X/\eta_2-1$
&
$  0.26^{  0.65}_{ -0.65}$
&
$ -0.09^{  0.53}_{ -0.55}$
&
0
&
$ -0.08^{  0.24}_{ -0.21}$
&
$ -0.57^{  0.51}_{ -0.59}$
\\
$\gamma_X/\gamma_2-1$
&
$  0.64^{  0.05}_{ -0.05}$
&
$  0.34^{  0.03}_{ -0.03}$
&
0
&
$ -0.23^{  0.01}_{ -0.01}$
&
$ -0.45^{  0.03}_{ -0.03}$
\\
$R_{X}$
&
$  4.87^{  0.48}_{ -0.43}$
&
$  3.96^{  0.40}_{ -0.35}$
&
$  2.96^{  0.31}_{ -0.28}$
&
$  2.29^{  0.26}_{ -0.23}$
&
$  1.64^{  0.21}_{ -0.19}$
\\
$\delta_X/\delta_2-1$
&
$-0.393^{ 0.182}_{-0.208}$
&
$-0.224^{ 0.101}_{-0.117}$
&
0
&
$-0.056^{ 0.051}_{-0.046}$
&
$-0.147^{ 0.096}_{-0.089}$
\\
$R_{\delta X}$
&
$ -2.70^{  1.32}_{ -2.44}$
&
$ -3.46^{  1.29}_{ -2.65}$
&
$ -4.46^{  1.43}_{ -3.11}$
&
$ -4.20^{  1.32}_{ -2.89}$
&
$ -3.80^{  1.20}_{ -2.61}$
\\
$\sigma_{X}$
&
$ 0.059^{ 0.012}_{-0.011}$
&
$ 0.031^{ 0.007}_{-0.007}$
&
$ 0.020^{ 0.004}_{-0.006}$
&
$ 0.013^{ 0.007}_{-0.008}$
&
$ 0.043^{ 0.005}_{-0.004}$
\\
\hline
\end{tabular}
\caption{68\% Intervals for Global Fit Parameters \label{global:tab}}
\end{table}


We find small but significant non-zero values for $\alpha_0$ and $\alpha_4$, indicating that $EW_{Ca}$ is an indicator of $U$ and $I$
magnitudes.  All bands show larger and significant non-zero values for $\beta$, indicating the importance of
$EW_{Si}$.  This confirms the part of our hypothesis that spectral indicators
are tracers of supernova absolute magnitude.  On the other hand, the values of $\eta$ are consistent with zero within one sigma.
How spectral features affect color can be seen in $\alpha_X/\alpha_V-1$,  $\beta_X/\beta_V-1$, and  $\eta_X/\eta_V-1$.
Both $EW_{Ca}$ and $EW_{Ca}$ are associated with color changes though not in $B-V$ specifically.
We do not detect a significant association between
$v_{Si}$ and color.

The best-fit distributions for $R_X=\frac{\gamma_X}{\gamma_1-\gamma_2}$ are shown in Figure~\ref{rx:fig}.  Figure~\ref{ccm:fig}
shows the expectation of the \citet{1999PASP..111...63F} dust model through its parameter $R^F_V$, plotted together with our measurements of
$R_X$ and $\frac{\gamma_X}{\gamma_2}$ placed at the effective wavelength of the supernova flux that passes
through the synthetic bands.
The $R_X$ are consistent with an  $R^F_V=2.5$.  The uncertainties in the 5 measurements
of $R_X$ are correlated: measurements of $\gamma_X/\gamma_2$ have significantly smaller
uncertainties and are also consistent with $R^F_V=2.5$. 

\begin{figure}[htbp] %  figure placement: here, top, bottom, or page
   \centering
   \includegraphics[width=2.8in]{output11/rx_corner.pdf}
      \includegraphics[width=2.8in]{output11/rxdelta_corner.pdf} 
   \caption{Best-fit distribution for (left)  $R_X=\frac{\gamma_X}{\gamma_1-\gamma_2}$ and (right)  $R_{\delta X}=\frac{\delta_X}{\delta_1-\delta_2}$.
   \label{rx:fig}}
\end{figure}

\begin{figure}[htbp] %  figure placement: here, top, bottom, or page
   \centering
   \includegraphics[width=2.8in]{output11/ccm.pdf}
      \includegraphics[width=2.8in]{output11/ccm2.pdf} 
   \caption{68\% measurements of $R_X=\frac{\gamma_X}{\gamma_1-\gamma_2}$ (left) and $\frac{\gamma_X}{\gamma_2}$ (right). \citet{1999PASP..111...63F} dust
   predictions for these parameters are overlaid for different values of the dust-parameter $R^F_V$.
   \label{ccm:fig}}
\end{figure}

The best-fit distributions for $R_{\delta X}=\frac{\delta_X}{\delta_1-\delta_2}$ are shown in Figure~\ref{rx:fig}.
Although the posteriors of $\delta$ are fairly circular, $\delta_1-\delta_2$ can approach zero, resulting in the
elongated tail in negative  $R_{\delta X}$.
All values are negative and inconsistent with zero: a color excess in $B-V$ produces a brightening of the supernova.
The values of $\delta_X/\delta_V-1$ are not monotonic with color: they increase from the $U$ to $V$, and then decrease
from $V$ to $I$,
a consequence
of which is that
a reddening in $B-V$ comes with a blueing in $V-R$.
These are not the behaviors expected from normal dust or a blackbody in the Raleigh-Jeans tail, and so may be indicative of
more complicated underlying supernova physics.

The distribution of color excess $E(B-V) =(\gamma_1-\gamma_2)k$ and
 $E_\delta(B-V) = (\delta_1-\delta_2)D$ are shown in Figure~\ref{ebv:fig}.
Recall that given our boundary conditions the means of both distributions are equal to zero. 
The standard deviation of $E(B-V)$ is 0.078 mag while that of $E_\delta(B-V)$ is 0.019 mag.
The range of $B-V$ colors is $\sim 4$ times larger for the external contribution:
extreme colors can thus be attributed to the external source. 
Both distributions are non-Gaussian with sharp rises in the blue and extended tails in the red.
\begin{figure}[htbp] %  figure placement: here, top, bottom, or page
   \centering
   \includegraphics[width=3.2in]{output11/ebv.pdf}
   \caption{Histograms of the contributions to color excess from external $E(B-V) =(\gamma_1-\gamma_2)k$ and
   internal $E_\delta(B-V) = (\delta_1-\delta_2)D$ effects for the supernovae in our sample.
   \label{ebv:fig}}
\end{figure}

The two separate color effects are conflated into a combined color excess with a single effective total-to-selective extinction
\begin{equation}
R_{V,eff}  = \frac{\gamma_1 k + \delta_1 D}{(\gamma_1-\gamma_2) k + (\delta_1-\delta_2)D}.
\label{RVeff:eqn}
\end{equation}
The distribution for $R_{V,eff}$ for the supernovae in our sample is shown in Figure~\ref{rveff:fig}
and has mean and standard deviation  $R_{V,eff}  = 1.52 \pm 0.24$.  
Rerunning the posterior calculation with initial conditions for $\gamma$  that correspond to $R_{X,eff}$ 
and $\delta=0$ (technically $\delta_0 = -\epsilon$ due to the $\delta_0<0$ condition)
reproduces the original results, indicating that
the effective and true $R$ values are not separated by local posterior extrema.
\begin{figure}[htbp] %  figure placement: here, top, bottom, or page
   \centering
   \includegraphics[width=3.2in]{output11/rveff.pdf}
   \caption{The effective dust law of the two color terms combined $R_{V,eff}  = \frac{\gamma_1 k + \delta_1 D}{(\gamma_1-\gamma_2) k + (\delta_1-\delta_2)D}$,
   for the supernovae in our sample.
   \label{rveff:fig}}
\end{figure}


The ideogram for the grey offsets $\Delta$ for all supernovae is shown in Figure~\ref{hist:fig}.  The intrinsic dispersion
is given by the standard deviation of $0.096$ mag.  The distribution is non-Gaussian, with a broad tail in the positive (fainter) direction.
\begin{figure}[htbp] %  figure placement: here, top, bottom, or page
   \centering
   \includegraphics[width=3.2in]{output11/Delta_hist.pdf} 
   \caption{Ideogram for the grey offset $\Delta$.
   \label{hist:fig}}
\end{figure}


Non-trivial residual magnitude dispersions are captured in $C_c$.   Figure~\ref{sigma:fig} shows the confidence regions for the
square root of the diagonal elements of $C_c$.  The residual intrinsic dispersion ranges from 0.01 to 0.06 mag, significantly smaller
than the dispersion in $\Delta$.
The off-diagonal elements of $C_c$ are parameterized by the Cholesky factors of a correlation matrix: assembling
a new set of Cholesky factors based on the distributions of each individual factor will not generally satisfy the condition of representing a correlation matrix.  
To characterize a typical posterior draw of $C_c$ we use the matrix that is the mean of all covariance realizations in the
chain, element by element.
For $UBVRI$ the matrix is
\begin{equation}
\begin{pmatrix}
0.0037 & 0.0010 & -0.0001 & -0.0001 & 0.0003 \\
0.0010 & 0.0010 & 0.0002 & -0.0000 & -0.0004 \\
-0.0001 & 0.0002 & 0.0004 & 0.0001 & -0.0001 \\
-0.0001 & -0.0000 & 0.0001 & 0.0002 & 0.0002 \\
0.0003 & -0.0004 & -0.0001 & 0.0002 & 0.0019
 \end{pmatrix} \text{mag}^2.
 \end{equation}
Similarly the typical color covariance, in terms of colors $U-V$, $B-V$, $V-R$, and $V-I$, can
be expressed as having standard deviations and
 correlations
 \begin{equation}
 \begin{pmatrix}
0.066 , 0.033 , 0.022 , 0.050 
  \end{pmatrix} \text{mag}
 \label{color_sd:eqn}
   \end{equation}
 \begin{equation}
\begin{pmatrix}
1.000 & 0.634 & -0.256 & -0.300 \\
0.634 & 1.000 & -0.203 & 0.032 \\
-0.256 & -0.203 & 1.000 & 0.604 \\
-0.300 & 0.032 & 0.604 & 1.000
  \end{pmatrix}.
  \label{color_cor:eqn}
 \end{equation}
 
 \begin{figure}[htbp] %  figure placement: here, top, bottom, or page
   \centering
   \includegraphics[width=3.2in]{output11/sigma_corner.pdf} 
   \caption{Best-fit distribution for the parameters $\sigma$, the square root of the diagonal elements of $C_c$.
   \label{sigma:fig}}
\end{figure}

Each supernova is described by its parameters $EW_{Ca}$, $EW_{Si}$, $v_{Si}$, $E(B-V)$, and $E_\delta(B-V)$, as well as its grey offset
$\Delta$: their distributions for all Monte Carlo chain links for all supernovae, are shown in Figure~\ref{perobject:fig}.
There is a core concentration in the  parameter-space, with around ten objects occupy its outskirts.
Many outliers appear in the red tail of $E(B-V)$, as would be expected in the infrequent selection of supernovae
heavily extinguished by host-galaxy dust; note that $E_\delta(B-V)$ does not have an analogous tail.
Outliers in  $EW_{Ca}$--$v_{Si}$ parameters are also clearly distinguished.

\begin{figure}[htbp] %  figure placement: here, top, bottom, or page
   \centering
   \includegraphics[width=3.2in]{output11/perobject_corner.pdf} 
   \caption{Distributions for the supernova parameters $EW_{Ca}$, $EW_{Si}$, $v_{Si}$, $E(B-V)$, and $E_\delta(B-V)$, as well as its grey offset
$\Delta$.
   \label{perobject:fig}}
\end{figure}

The Pearson correlation coefficients for $\Delta$, $EW_{Ca}$, $EW_{Si}$, $v_{Si}$, $E(B-V)$, and $E_\delta(B-V)$ are given in the matrix
\begin{equation}
\begin{pmatrix}
1.000 & -0.012 & -0.031 & -0.026 & 0.070 & 0.002 \\
-0.012 & 1.000 & 0.128 & -0.331 & -0.064 & -0.004 \\
-0.031 & 0.128 & 1.000 & -0.182 & -0.155 & -0.024 \\
-0.026 & -0.331 & -0.182 & 1.000 & 0.018 & 0.036 \\
0.070 & -0.064 & -0.155 & 0.018 & 1.000 & 0.062 \\
0.002 & -0.004 & -0.024 & 0.036 & 0.062 & 1.000
\end{pmatrix}.
\end{equation}
Again  $EW_{Ca}$--$v_{Si}$ is singled out, this time as having the strongest (anti-)correlation.  The second strongest is in
$EW_{Si}$--$v_{Si}$.
External $E(B-V)$ dust most strongly correlated with $EW_{Si}$, while the internal $E_\delta(B-V)$  has no correlation stronger with internal parameters stronger than
0.04.  Although $E(B-V)$ and $E_\delta(B-V)$ are determined simultaneously for each supernova from colors, they have a low correlation coefficient of 0.062.

\section{SN 2014J}
\label{sn2014j:sec}

SN~2014J \citep{2014ApJ...788L..21A}  is one of several SNe~Ia that exhibits a low $R_V<2.0$.
It is spectroscopically normal, having a spectral evolution similar to SN~2011fe.
As SN~2011fe suffers low Galactic and host reddening, it serves as a good template
for the measurement of color excess, nulling out the effects of Ca and Si equivalent width.
SN~2014J does show
overall higher photospheric velocities than SN~2011fe, but the results in \S\ref{results:sec} show that this is unimportant.
Note that SN~2014J is not part of the data sample described in \S\ref{data:sec}.

Data of SN~2014J are now analyzed using our model and the training with the SNfactory sample.
The data are the color excesses  in $UBRi$  relative to $V$ at peak brightness  taken from \citet{2014ApJ...788L..21A}
following their prescription of averaging measurements within 5-days of peak $B$ magnitude.
Their values 
$E^o(U-V) =   2.23 \pm   0.03$,
$E^o(B-V) =   1.28 \pm   0.04$,
$E^o(R-V) =  -0.47 \pm   0.03$,
$E^o(i-V) =  -0.92 \pm   0.03$
are plotted in Figure~\ref{sn2014j:fig}.
These data are fit to the model
\begin{equation}
E^o(X-V) =  \left(\frac{\gamma_X}{\gamma_2}-1\right)k +  \left(\frac{\delta_X}{\delta_2}-1\right)D,
\end{equation}
where we use the median values of the $\gamma$- and $\delta$-terms from Table~\ref{global:tab}.

\begin{figure}[htbp] %  figure placement: here, top, bottom, or page
   \centering
   \includegraphics[width=3.2in]{output11/sn2014j.pdf} 
   \caption{Measured color excess for SN~2014J and the best-fit predictions from this article and  \citet{2014ApJ...788L..21A}  
   \label{sn2014j:fig}}
\end{figure}

The best-fit parameters for SN~2014J are $k= 2.56$, $ D=-1.57$ with covariance
\begin{equation}
\begin{pmatrix}
0.058 & 0.059 \\
0.059 & 0.204
\end{pmatrix}.
\end{equation}
The predicted values of observed $E^o(B-V)$ from the fit are shown in Figure~\ref{sn2014j:fig}, where they are found to
overlap with the data within the error bars.   For comparison, the best-fit model determined by  \citet{2014ApJ...788L..21A} using
UV through NIR data,
a  \citet{1999PASP..111...63F} dust with $R_V^F=1.4$ and $E(B-V)=1.37$, 
is also plotted.  Our model gives significantly smaller deviations from the data.

In our model, the observed color excess is attributed to contributions from extrinsic
$E(B-V)=  0.86 \pm   0.08$ and intrinsic $E_\delta(B-V)=  0.35 \pm   0.10$ origins.
Among the supernovae in the SNfactory training set, the highest median value for 
$E(B-V)$ is $  0.33^{  +0.04}_{ -0.04}$ and for $E_\delta(B-V)$ is
$  0.07^{  +0.03}_{ -0.03}$ 
(see Figure~\ref{ebv:fig}).
In both parameters SN~2014J lives well outside the range of supernovae used to train the coefficients of the model.

\section{Discussion}
\label{discussion:sec}
We model SNe~Ia broadband optical peak magnitudes allowing for correlations with spectral features at peak and distinct intrinsic and
extrinsic color parameters.  Analyzing SNfactory data using the model, we find significant evidence that the above parameters do
affect supernova magnitudes and colors.  The external color parameter has behavior consistent with $R^F_V=2.5$ \citet{1999PASP..111...63F} dust,
whereas the internal color parameter is inconsistent with dust or a blackbody.

\citet{2014ApJ...789...32B, 2015MNRAS.453.3300A} deduce a wide range of dust behavior $1.5<R^F_V<3$ encountered by the SN~Ia population.
Our model can explain a wide range of effective total-to-selective extinction with a single fixed set of $\gamma$ parameters, consistent
with $R^F_V=2.5$.  The reality is that a range of dust behaviors is expected; while this cannot accommodated in our model, it could be by
introducing hierarchical 
modeling of the $\gamma$ distribution.
For the moment, we assert that variations in extrinsic dust are limited in
size by Eqn.~\ref{color_sd:eqn} and positive correlations possible within \ref{color_cor:eqn}.


\citet{2011ApJ...729...55F, 2014ApJ...789...32B, 2015MNRAS.453.3300A} find that supernovae with large $E^o(B-V)$ 
preferentially exhibit low $R^F_V$.  This is natural in our model, as for a fixed extrinsic dust, $R_{V,eff}$ decreases with increasing $E^o(B-V)$ when
both are positive.


\citet{2009ApJ...699L.139W, 2011ApJ...729...55F} find a connection between $v_{Si}$ and color, whereas
our values of $\eta$ are consistent with zero.  Recall that we find a correlation between $v_{Si}$ and $EW_{Ca}$;
removing from our model the dependence on equivalent widths (eliminating the  $\alpha$ and $\beta$ parameters), we recover
non-zero $\eta$ values at  $\gtrsim 2\sigma$.  We  predict that a reanalysis of the previous data using equivalent width in place
of SiII velocity would also show correlations with color.

Anti-connection between $EW(Ca)$ and $v_{Si}$.  Intermediate mass

We find an intrinsic supernova parameter that does not have a monotonic influence on supernova magnitudes as a function
of wavelength but rather has an inflection in the $V$ band.  In addition, red $B-V$ colors imply brighter magnitudes.  Line opacities
that operate over broad bands are the most obvious candidate for producing this behavior and so warrant further study.

The model and results presented here
carry information on the calibration of absolute magnitude.  The grey parameter $\Delta$ represent Hubble residuals after
application of the model, and its  $0.096$ mag dispersion has contributions from intrinsic dispersion, peculiar velocities, and
measurement uncertainties.  It is also subject to overtraining.
The  analysis of this article was intended to model the color behavior of supernovae in our dataset,
and is not appropriate for inferring the intrinsic dispersion.  A separate analysis is required to determine how
well this model can be used to measure supernova distances.

Low correlation coefficient means that the contribution of $E_\delta$ is not systematic, no high $E(B-V)$ bias.

Could affect distance determinations.  Large $E(B-V)$ does not mean extinction.


Standard candle.
\acknowledgments

%
%To illustrate how the linear model reduces the dispersion in observed absolute magnitude, plot the observed colors as a
%function of the observed  $EW_{Ca, o}$, $EW_{Si,o}$ and $B_o-V_o$ in Fig.~\ref{magresidual:fig}.  Overplotted are lines with the
%best-fit slope.  The data in the color magnitude diagram indicate that the slope at bluer colors is different from that at redder colors
%and that our one-color linear model could be improved upon.
%
%\begin{figure}[htbp] %  figure placement: here, top, bottom, or page
%   \centering
%   \includegraphics[width=2.8in]{output7/speccamag.pdf} 
%      \includegraphics[width=2.8in]{output7/specsimag.pdf} 
%   \includegraphics[width=2.8in]{output7/colormag.pdf} 
%   \caption{Observed $UBVRI$ as a function of observed   $EW_{Ca, o}$ (top left), $EW_{Si,o}$ (top right) and $B_o-V_o$ (bottom).
%   Overplotted are lines with the best-fit slope from the model fit.}
%   \label{magresidual:fig}
%\end{figure}
%
%\subsubsection{Notes}
%Some equivalent widths go positive.  Is this an emission?
%
%
%\subsection{One Dust With Si Velocity}
%\label{onecolor_si:sec}
%Sil$\lambda3655$ velocity is spectral indicator that has been identified to correlated with peak brightness.  The analysis is expanded to include
%this line. We use the weighted mean of all spectral measurements within 3 days of peak brightness.  A handful of supernovae do not have a
%spectrum that close to peak.
%\subsubsection{Setup}
%With the inclusion of SiII$\lambda3655$ velocity  $v_{Si}$ the model looks like
%\begin{equation}
%\left(
%\begin{matrix}
%U\\B\\V\\R\\I
%\end{matrix}
%\right) \sim \mathcal{N}
%\left(
%\Delta +
%\left(
%\begin{matrix}
%c_0+\alpha_0 EW_{Ca} + \beta_0 EW_{Si} + \eta_0 v_{Si}\\
%c_1+\alpha_1 EW_{Ca} + \beta_1 EW_{Si} + \eta_1 v_{Si} \\
%c_2+\alpha_2 EW_{Ca} + \beta_2 EW_{Si} + \eta_2 v_{Si}\\
%c_3+\alpha_3 EW_{Ca} + \beta_3 EW_{Si} + \eta_3 v_{Si}\\
%c_4+\alpha_4 EW_{Ca} + \beta_4 EW_{Si}+ \eta_4 v_{Si}
%\end{matrix}
%\right)
%,C_{c}
%\right)
%\label{ewsiv:eqn}
%\end{equation}
%\begin{equation}
%\left(
%\begin{matrix}
%U_o\\B_o\\ V_o\\R_o\\I_o\\EW_{Si, o}\\ EW_{Ca, o} \\ v_{Si, o}\end{matrix}
%\right) \sim \mathcal{N}
%\left(
%\left(
%\begin{matrix}
%U +\gamma_0 k \\B +\gamma_1 k \\V+\gamma_2 k\\R+\gamma_3 k\\I+\gamma_4 k\\
%EW_{Si}\\ EW_{Ca} \\ v_{Si}
%\end{matrix}
%\right)
%,C
%\right).
%\label{dust:eqn}
%\end{equation}
%The global parameters $\eta$ are the slopes of the linear relationships between Si velocity
%and intrinsic magnitudes. 
%
%\subsubsection{Results}
%The 68\% intervals for the global parameters $\alpha$, $\beta$, $\eta$, $R$, and $\sigma = \sqrt{C_{c,ii}}$  are given in Table~\ref{globalvsi:tab}.
%The two dimensional contours for the parameters of each filter are shown in Figure~\ref{globalvsi:fig}.  
%All bands show large and significant non-zero values for $\eta$ with some non-trivial changes to $\beta$ and a lowering
%if $\sigma$.  The values
%of $\alpha$ are unchanged.
%
%\begin{table}
%\centering
%\begin{tabular}{|c|c|c|c|c|c|}
%\hline
%& $X=0$ &1&2&3&4\\ \hline
%$\alpha_X$
%&
%$0.0031^{0.0009}_{-0.0009}$
%&
%$0.0005^{0.0008}_{-0.0007}$
%&
%$0.0006^{0.0006}_{-0.0006}$
%&
%$0.0008^{0.0005}_{-0.0005}$
%&
%$0.0020^{0.0005}_{-0.0004}$
%\\
%$\beta_X$
%&
%$0.0375^{0.0030}_{-0.0030}$
%&
%$0.0303^{0.0026}_{-0.0025}$
%&
%$0.0298^{0.0023}_{-0.0021}$
%&
%$0.0243^{0.0020}_{-0.0019}$
%&
%$0.0224^{0.0017}_{-0.0017}$
%\\
%$\eta_X$
%&
%$0.0032^{0.0011}_{-0.0011}$
%&
%$0.0024^{0.0009}_{-0.0009}$
%&
%$0.0022^{0.0008}_{-0.0008}$
%&
%$0.0019^{0.0007}_{-0.0007}$
%&
%$0.0011^{0.0006}_{-0.0006}$
%\\
%$R_X$
%&
%$5.1004^{0.2929}_{-0.2517}$
%&
%$4.1677^{0.2656}_{-0.2272}$
%&
%$3.1677^{0.2656}_{-0.2272}$
%&
%$2.4828^{0.2327}_{-0.1964}$
%&
%$1.8320^{0.2050}_{-0.1838}$
%\\
%$\sigma_X$
%&
%$0.0537^{0.0162}_{-0.0185}$
%&
%$0.0405^{0.0107}_{-0.0131}$
%&
%$0.0565^{0.0093}_{-0.0109}$
%&
%$0.0538^{0.0084}_{-0.0084}$
%&
%$0.0636^{0.0076}_{-0.0079}$
%\\
%\hline
%\end{tabular}
%\caption{68\% Intervals for Global Fit Parameters for the One-Dust Model with $v_{Si}$  \label{globalvsi:tab}}
%\end{table}
%
%\begin{figure}[htbp] %  figure placement: here, top, bottom, or page
%   \centering
%   \includegraphics[width=2.5in]{output10/coeff0.pdf} 
%   \includegraphics[width=2.5in]{output10/coeff1.pdf} 
%   \includegraphics[width=2.5in]{output10/coeff2.pdf} 
%      \includegraphics[width=2.5in]{output10/coeff3.pdf} 
%         \includegraphics[width=2.5in]{output10/coeff4.pdf} 
%            \caption{Distributions for $c$, $\alpha$, $\beta$, $\eta$, $\gamma$, and $\sigma$ in each of the 5 bands.}
%   \label{globalvsi:fig}
%\end{figure}
%
%
%
%The best-fit distribution for $R_X=\frac{\gamma_X}{\gamma_1-\gamma_2}$ are shown in Figure~\ref{rxsiv:fig}.  Our measurement of
%$R_X$ and $\frac{\gamma_X}{\gamma_2}$  compared with the predictions of CCM dust:
%our fit values for $\gamma$ are consistent with the  $R_V=3.1$ CCM model.
%
%\begin{figure}[htbp] %  figure placement: here, top, bottom, or page
%   \centering
%   \includegraphics[width=3in]{output10/gamma_corner.pdf} 
%   \caption{Best-fit distribution for the   $R_X=\frac{\gamma_X}{\gamma_1-\gamma_2}$.
%   \label{rxsiv:fig}}
%\end{figure}
%
%\begin{figure}[htbp] %  figure placement: here, top, bottom, or page
%   \centering
%   \includegraphics[width=2.8in]{output10/ccm.pdf}
%      \includegraphics[width=2.8in]{output10/ccm2.pdf} 
%   \caption{68\% measurements of $R_X=\frac{\gamma_X}{\gamma_1-\gamma_2}$ (left) and $\frac{\gamma_X}{\gamma_2}$ (right).  CCM
%   predictions for these parameters are overlaid for different values of $R_V$.}
%   \label{ccmsiv:fig}
%\end{figure}
%
% \begin{figure}[htbp] %  figure placement: here, top, bottom, or page
%   \centering
%   \includegraphics[width=2.4in]{output10/sigma_corner.pdf} 
%   \caption{Best-fit distribution for the parameters $\sigma$, the square root of the diagonal elements of $C_c$.}
%   \label{sigmasiv:fig}
%\end{figure}
%
%The ideogram for the grey offsets $\Delta$ for all supernovae is shown in Figure~\ref{histsiv:fig}.  The intrinsic dispersion
%is given by the standard deviation of $0.09$ mag.  The distribution is non-Gaussian, with a broad tail. 
%\begin{figure}[htbp] %  figure placement: here, top, bottom, or page
%   \centering
%   \includegraphics[width=2.8in]{output10/Delta_hist.pdf} 
%   \includegraphics[width=2.8in]{output10/k_hist.pdf} 
%   \caption{Ideograms for Left: the grey offset $\Delta$; Right: the color term $k$ for all supernovae.}
%   \label{histsiv:fig}
%\end{figure}
%
%\subsection{Two Dust}
%
%Motivated by the consistency of the external extinction with CCV $R_V=3.1$ dust, and the hint of the color-magnitude
%diagrams in Figure~\ref{magresidual:fig}  not following single slopes, we develop a new model in which
%each supernova encounters one of two sources of external extinction.
%
%\subsubsection{Setup}
%The model is identical to that of \S\ref{onecolor:sec} except with the addition of an alternative extrinsic source of extinction.
%The intrinsic magnitudes of the supernova are determined by the spectral properties expressed as Eqn.~\ref{ew:eqn}.
%One extrinsic source of extinction  (Dust 1) is modeled as before with Eqn.~\ref{dust:eqn};
%the second source of extinction (Dust 2)
%is modeled similarly but with the introduction of new parameters $\gamma_{2X}$
%\begin{equation}
%\left(
%\begin{matrix}
%U_o\\B_o\\ V_o\\R_o\\I_o\\EW_{Si, o}\\ EW_{Ca, o}\end{matrix}
%\right) \sim \mathcal{N}
%\left(
%\left(
%\begin{matrix}
%U +\gamma_{20} k \\B +\gamma_{21} k \\V+\gamma_{22} k\\R+\gamma_{23} k\\I+\gamma_{24} k\\
%EW_{Si}\\ EW_{Ca}
%\end{matrix}
%\right)
%,C
%\right).
%\label{dust2:eqn}
%\end{equation}
%The probability that a supernova experiences  Dust 1 is given by the free parameter $p_0$,
%the probability of Dust 2 is $1-p_0$.  
%The same $k$ parameters are used to describe supernovae in both extinction sources.  In order to do this we change
%some of the conditions that constrain the degeneracies of the model:
%\begin{equation*}
%\langle \Delta \rangle=0, \gamma_1 = 1+\gamma_2, \gamma_{21}=1+\gamma_{22}.
%\end{equation*}
%These conditions are useful for interpreting the extinction as being due to dust, as $R_X = \gamma_X$,
%the color excess $E(B-V) \equiv (\gamma_1-\gamma_2)k =k$.
%The $c$ parameters  are fit such that both extinction models
%revert to the same intrinsic magnitudes for $k=0$: the $\langle k \rangle=0$ condition is removed. 
%
%To alleviate the degeneracies inherent to mixture models, we restrict the range of $\alpha$, $\beta$, and $\gamma$ (the parameters for Dust 1)
%to be in the neighborhood of the pdf's of \S\ref{onecolor:sec}.  The one-dust model is embedded in the two-dust model when $p_0=1$.
%We therefore do not claim to find the global optimum of the parameter
%pdf, but we can conclude whether the two-dust model is preferred over the one-dust model.
%The intrinsic covariance $C_c$ is taken to be diagonal.
%
%\subsubsection{Results}
%
%The probability of an object experiencing the first dust is   $p_0=0.3334^{0.0532}_{-0.0721}$,
%with the pdf shown in Figure~\ref{prob0:fig}.
%A single dust with properties centered around the one-dust parameter optimum is not only disfavored, but is most likely the
%minority component.  
%\begin{figure}[htbp] %  figure placement: here, top, bottom, or page
%   \centering
%   \includegraphics[width=3.3in]{output9/prob0.pdf}
%   \caption{pdf of $p_0$, the probability that a supernova experiences Dust 1.  \label{prob0:fig}}
%\end{figure}
%
%
%The 68\% intervals for the global parameters $\alpha$, $\beta$, $R$, $R_2$ and $\sigma = \sqrt{C_{c,ii}}$  are given in Table~\ref{global2:tab}.
%The two dimensional contours for the parameters of each filter are shown in Figure~\ref{global2:fig}.  
%\begin{table}
%\centering
%\begin{tabular}{|c|c|c|c|c|c|}
%\hline
%& $X=0$ &1&2&3&4\\ \hline
%$\alpha_{X}$
%&
%$0.0077^{0.0014}_{-0.0018}$
%&
%$0.0040^{0.0011}_{-0.0013}$
%&
%$0.0025^{0.0008}_{-0.0008}$
%&
%$0.0020^{0.0006}_{-0.0007}$
%&
%$0.0027^{0.0006}_{-0.0006}$
%\\
%$\beta_{X}$
%&
%$0.0276^{0.0048}_{-0.0053}$
%&
%$0.0225^{0.0037}_{-0.0042}$
%&
%$0.0249^{0.0025}_{-0.0026}$
%&
%$0.0213^{0.0019}_{-0.0020}$
%&
%$0.0216^{0.0015}_{-0.0016}$
%\\
%$R_{X}$
%&
%$5.0878^{0.4233}_{-0.4082}$
%&
%$4.3163^{0.3615}_{-0.3327}$
%&
%$3.3163^{0.3615}_{-0.3327}$
%&
%$2.6600^{0.2685}_{-0.2691}$
%&
%$1.9355^{0.2604}_{-0.2474}$
%\\
%$R_{2X}$
%&
%$2.2199^{0.2110}_{-0.2301}$
%&
%$1.5919^{0.1801}_{-0.2246}$
%&
%$0.5919^{0.1801}_{-0.2246}$
%&
%$0.2486^{0.1650}_{-0.1788}$
%&
%$-0.1484^{0.1802}_{-0.1472}$
%\\
%$\sigma_{X}$
%&
%$0.0570^{0.0042}_{-0.0040}$
%&
%$0.0103^{0.0059}_{-0.0048}$
%&
%$0.0229^{0.0021}_{-0.0018}$
%&
%$0.0094^{0.0031}_{-0.0039}$
%&
%$0.0418^{0.0031}_{-0.0030}$
%\\
%\hline
%\end{tabular}
%\caption{68\% Intervals for Global Fit Parameters for the Two-Dust Model \label{global2:tab}}
%\end{table}
%
%\begin{figure}[htbp] %  figure placement: here, top, bottom, or page
%   \centering
%   \includegraphics[width=2.5in]{output9/coeff0.pdf} 
%   \includegraphics[width=2.5in]{output9/coeff1.pdf} 
%   \includegraphics[width=2.5in]{output9/coeff2.pdf} 
%      \includegraphics[width=2.5in]{output9/coeff3.pdf} 
%         \includegraphics[width=2.5in]{output9/coeff4.pdf} 
%            \caption{Distributions for $c$, $\alpha$, $\beta$, $\gamma$, $\gamma_2$, and $\sigma$ in each of the 5 bands.}
%   \label{global2:fig}
%\end{figure}
%
%Our determinations of $R$ and $R_2$ are compared to the expectations of CCM dust in Figure~\ref{ccm2:fig}.
%Within the context of this model there are two distinct types of dust: Dust 1 behaves similarly to $R_V=3.1$ CCM dust.
%In contrast, Dust 2 is more consistent with extremely low values of $R_V$.  Negative values of $R_V$ are unphysical.
%\begin{figure}[htbp] %  figure placement: here, top, bottom, or page
%   \centering
%   \includegraphics[width=2.8in]{output9/ccm.pdf}
%      \includegraphics[width=2.8in]{output9/ccm2.pdf} 
%         \includegraphics[width=2.8in]{output9/ccm12.pdf}
%      \includegraphics[width=2.8in]{output9/ccm22.pdf} 
%   \caption{68\% measurements of $R_X=\frac{\gamma_X}{\gamma_1-\gamma_2}$ (left) and $\frac{\gamma_X}{\gamma_2}$ (right)
%   for Dust 1 (top row) and Dust 2 (bottom row).  CCM
%   predictions for these parameters are overlaid for different values of $R_V$.}
%   \label{ccm2:fig}
%\end{figure}
%
%While the values for $\beta$ are consistent between the one- and two-dust models, the $\alpha$ values change significantly.
%This implies a correlation between $EW_{Ca}$ and the dust environment.
%
\bibliographystyle{apj}
\bibliography{/Users/akim/Documents/alex}


\end{document} 