\documentclass{aastex61}   	% use "amsart" instead of "article" for AMSLaTeX format
\usepackage{geometry}                		% See geometry.pdf to learn the layout options. There are lots.
\geometry{letterpaper}                   		% ... or a4paper or a5paper or ... 
\usepackage{graphicx}				% Use pdf, png, jpg, or eps§ with pdflatex; use eps in DVI mode
\usepackage{amsmath}
\usepackage{amssymb}
\usepackage{natbib}
\usepackage{lineno}
\usepackage{color}
\defcitealias{1999PASP..111...63F}{F99}
\linenumbers

\begin{document}

\title{Evidence for a Color Parameter Unassociated With Dust Within the Type~Ia Supernovae of the Nearby Supernova Factory}
\author[0000-0001-6315-8743]{A.~G.~Kim}
\affiliation{    Physics Division, Lawrence Berkeley National Laboratory, 
    1 Cyclotron Road, Berkeley, CA, 94720}
    

\author{     G.~Aldering}
\affiliation{    Physics Division, Lawrence Berkeley National Laboratory, 
    1 Cyclotron Road, Berkeley, CA, 94720}

\author{     P.~Antilogus}
\affiliation{    Laboratoire de Physique Nucl\'eaire et des Hautes \'Energies,
    Universit\'e Pierre et Marie Curie Paris 6, Universit\'e Paris Diderot Paris 7, CNRS-IN2P3, 
    4 place Jussieu, 75252 Paris Cedex 05, France}
    
\author{     S.~Bailey}
\affiliation{    Physics Division, Lawrence Berkeley National Laboratory, 
    1 Cyclotron Road, Berkeley, CA, 94720}

\author{     C.~Baltay}
\affiliation{    Department of Physics, Yale University, 
    New Haven, CT, 06250-8121}

\author{     K.~Barbary}
\affiliation{
    Department of Physics, University of California Berkeley,
    366 LeConte Hall MC 7300, Berkeley, CA, 94720-7300}

\author{    D.~Baugh}
\affiliation{   Tsinghua Center for Astrophysics, Tsinghua University, Beijing 100084, China }

\author{     K.~Boone}
\affiliation{    Physics Division, Lawrence Berkeley National Laboratory, 
    1 Cyclotron Road, Berkeley, CA, 94720}
\affiliation{
    Department of Physics, University of California Berkeley,
    366 LeConte Hall MC 7300, Berkeley, CA, 94720-7300}

\author{     S.~Bongard}
\affiliation{    Laboratoire de Physique Nucl\'eaire et des Hautes \'Energies,
    Universit\'e Pierre et Marie Curie Paris 6, Universit\'e Paris Diderot Paris 7, CNRS-IN2P3, 
    4 place Jussieu, 75252 Paris Cedex 05, France}

\author{     C.~Buton}
\affiliation{    Universit\'e de Lyon, F-69622, Lyon, France ; Universit\'e de Lyon 1, Villeurbanne ; 
    CNRS/IN2P3, Institut de Physique Nucl\'eaire de Lyon}
    
\author{     J.~Chen}
\affiliation{   Tsinghua Center for Astrophysics, Tsinghua University, Beijing 100084, China }

\author{     N.~Chotard}
\affiliation{    Universit\'e de Lyon, F-69622, Lyon, France ; Universit\'e de Lyon 1, Villeurbanne ; 
    CNRS/IN2P3, Institut de Physique Nucl\'eaire de Lyon}
    
\author{     Y.~Copin}
\affiliation{    Universit\'e de Lyon, F-69622, Lyon, France ; Universit\'e de Lyon 1, Villeurbanne ; 
    CNRS/IN2P3, Institut de Physique Nucl\'eaire de Lyon}

\author{ S.~Dixon}
\affiliation{
    Department of Physics, University of California Berkeley,
    366 LeConte Hall MC 7300, Berkeley, CA, 94720-7300}

\author{     P.~Fagrelius}
\affiliation{    Physics Division, Lawrence Berkeley National Laboratory, 
    1 Cyclotron Road, Berkeley, CA, 94720}
\affiliation{
    Department of Physics, University of California Berkeley,
    366 LeConte Hall MC 7300, Berkeley, CA, 94720-7300}

\author{     H.~K.~Fakhouri}
\affiliation{    Physics Division, Lawrence Berkeley National Laboratory, 
    1 Cyclotron Road, Berkeley, CA, 94720}
  \affiliation{
    Department of Physics, University of California Berkeley,
    366 LeConte Hall MC 7300, Berkeley, CA, 94720-7300}

\author{     U.~Feindt}
\affiliation{The Oskar Klein Centre, Department of Physics, AlbaNova, Stockholm University, SE-106 91 Stockholm, Sweden}

\author{     D.~Fouchez}
\affiliation{    Centre de Physique des Particules de Marseille, 
    Aix-Marseille Universit\'e , CNRS/IN2P3, 
    163 avenue de Luminy - Case 902 - 13288 Marseille Cedex 09, France}
    
\author{     E.~Gangler}  
\affiliation{    Clermont Universit\'e, Universit\'e Blaise Pascal, CNRS/IN2P3, Laboratoire de Physique Corpusculaire,
    BP 10448, F-63000 Clermont-Ferrand, France}
    
\author{     B.~Hayden}
\affiliation{    Physics Division, Lawrence Berkeley National Laboratory, 
    1 Cyclotron Road, Berkeley, CA, 94720}

\author{     W.~Hillebrandt}
\affiliation{    Max-Planck-Institut f\"ur Astrophysik, Karl-Schwarzschild-Str. 1,
D-85748 Garching, Germany}

\author{     M.~Kowalski}
\affiliation{    Institut fur Physik,  Humboldt-Universitat zu Berlin,
    Newtonstr. 15, 12489 Berlin}
\affiliation{ DESY, D-15735 Zeuthen, Germany}

\author{     P.-F.~Leget}
\affiliation{    Clermont Universit\'e, Universit\'e Blaise Pascal, CNRS/IN2P3, Laboratoire de Physique Corpusculaire,
    BP 10448, F-63000 Clermont-Ferrand, France}
    
\author{     S.~Lombardo}
\affiliation{    Institut fur Physik,  Humboldt-Universitat zu Berlin,
    Newtonstr. 15, 12489 Berlin}
    
\author{     J.~Nordin}
\affiliation{    Institut fur Physik,  Humboldt-Universitat zu Berlin,
    Newtonstr. 15, 12489 Berlin}
    
\author{     R.~Pain}
\affiliation{    Laboratoire de Physique Nucl\'eaire et des Hautes \'Energies,
    Universit\'e Pierre et Marie Curie Paris 6, Universit\'e Paris Diderot Paris 7, CNRS-IN2P3, 
    4 place Jussieu, 75252 Paris Cedex 05, France}
     
\author{     E.~Pecontal}
\affiliation{   Centre de Recherche Astronomique de Lyon, Universit\'e Lyon 1,
    9 Avenue Charles Andr\'e, 69561 Saint Genis Laval Cedex, France}
    
\author{    R.~Pereira}
 \affiliation{    Universit\'e de Lyon, F-69622, Lyon, France ; Universit\'e de Lyon 1, Villeurbanne ; 
    CNRS/IN2P3, Institut de Physique Nucl\'eaire de Lyon}
 
 \author{    S.~Perlmutter}
 \affiliation{    Physics Division, Lawrence Berkeley National Laboratory, 
    1 Cyclotron Road, Berkeley, CA, 94720} 
\affiliation{
    Department of Physics, University of California Berkeley,
    366 LeConte Hall MC 7300, Berkeley, CA, 94720-7300}
    
 \author{    D.~Rabinowitz}
 \affiliation{    Department of Physics, Yale University, 
    New Haven, CT, 06250-8121}
    
 \author{    M.~Rigault} 
\affiliation{    Institut fur Physik,  Humboldt-Universitat zu Berlin,
    Newtonstr. 15, 12489 Berlin}
     
 \author{    D.~Rubin}
 \affiliation{    Physics Division, Lawrence Berkeley National Laboratory, 
    1 Cyclotron Road, Berkeley, CA, 94720}
    \affiliation{   Space Telescope Science Institute, 3700 San Martin Drive, Baltimore, MD 21218}
 
 \author{    K.~Runge}
 \affiliation{    Physics Division, Lawrence Berkeley National Laboratory, 
    1 Cyclotron Road, Berkeley, CA, 94720}
 
 \author{    C.~Saunders}
 \affiliation{    Physics Division, Lawrence Berkeley National Laboratory, 
    1 Cyclotron Road, Berkeley, CA, 94720}

\author{    C.~Sofiatti}
\affiliation{    Physics Division, Lawrence Berkeley National Laboratory, 
    1 Cyclotron Road, Berkeley, CA, 94720} 
\affiliation{
    Department of Physics, University of California Berkeley,
    366 LeConte Hall MC 7300, Berkeley, CA, 94720-7300}

\author{    N.~Suzuki}
\affiliation{    Physics Division, Lawrence Berkeley National Laboratory, 
    1 Cyclotron Road, Berkeley, CA, 94720}

\author{     S.~Taubenberger}
\affiliation{    Max-Planck-Institut f\"ur Astrophysik, Karl-Schwarzschild-Str. 1,
D-85748 Garching, Germany}

\author{     C.~Tao}
\affiliation{   Tsinghua Center for Astrophysics, Tsinghua University, Beijing 100084, China }
\affiliation{    Centre de Physique des Particules de Marseille, 
    Aix-Marseille Universit\'e , CNRS/IN2P3, 
    163 avenue de Luminy - Case 902 - 13288 Marseille Cedex 09, France}
   
\author{     R.~C.~Thomas}
\affiliation{    Computational Cosmology Center, Computational Research Division, Lawrence Berkeley National Laboratory, 
    1 Cyclotron Road MS 50B-4206, Berkeley, CA, 94720}
    
\collaboration{(The Nearby Supernova Factory)}


\begin{abstract}
\color{purple}
Through empirical modeling of its observed signal, the peak absolute magnitude of a Type~Ia supernova (SN~Ia) can be accurately determined,
making SNe~Ia  excellent distance indicators.  Improved modeling of SN~Ia absolute magnitudes can account for
further physical
diversity that is expected but not included
in current models, and so consequently can lead to more precise per-object distances with smaller systematic uncertainties.  In this article, we present   
\color{black}
an empirical model for SN~Ia peak magnitudes with two color parameters and dependence on the equivalent widths of CaII, SiII, and SiII velocity
as applied to the supernova sample of the Nearby Supernova Factory.  The peak magnitudes in synthetic
broadband photometry and their colors are found to be 
dependent on the spectral equivalent widths and the color parameters, to better than 
%-----
$(1-5\times10^{-5})$
%------
confidence.
The color changes allowed by the model are similar to those allowed by the extinction model of  \citet{1999PASP..111...63F} with a free
total-to-selective extinction parameter $R^F$.  As an ensemble, the supernova sample has an effective
%-----
$R^F = 2.44$.
%-----
Our model
 is able to reproduce the optical colors of the highly reddened SN~2014J better than the \citet{1999PASP..111...63F} dust model.
Inspired that with no conditioning our  two-parameter model produces behavior similar to that expected from Galactic dust,
we extend  the model to three color parameters.
In this new case we obtain consistent results for
all the parameters in common,
plus a detection of a third parameter at
%--- table.py
$\sim 90\%$
%-----
confidence;
this new parameter represents supernova color diversity
beyond that tracked by the spectral features and two-parameter dust models.
\end{abstract}

\keywords{supernovae: general; supernovae: SN 2014J}

\section{Introduction}
Type~Ia supernovae (SNe~Ia) form a homogenous set of exploding stars and as such were early recognized and utilized as a powerful distance indicator 
and probe of cosmology \citep[e.g.][]{1992ARA&A..30..359B, 1993ApJ...415....1S}.  After further careful consideration of supernova data, it was recognized
that SN~Ia light-curve shapes \citep{1993ApJ...413L.105P} and colors \citep{1996ApJ...473...88R, 1998A&A...331..815T} exhibit subtle signs of heterogeneity
that are correlated with absolute magnitude, and must be considered when inferring distances.  Empirical models parameterizing SNe~Ia by their light-curve shape \citep{1996ApJ...473...88R,
1997ApJ...483..565P,
1999ApJ...517..565P}
and color  \citep{1996ApJ...473...88R}  were developed that enabled absolute magnitude corrections
and accurate distance measurements of cosmological supernovae,
which 
were subsequently used in the discovery of the accelerating expansion of the Universe \citep{1998AJ....116.1009R,1999ApJ...517..565P}.

The two most commonly used supernova-cosmology light-curve fitters today are SALT2 \citep{2007A&A...466...11G} and MLCS2k2
\citep{2007ApJ...659..122J}.\footnote{Light-curve fitters with more flexible degrees of freedom
\citep[e.g.][]{2008ApJ...681..482C, 2011AJ....141...19B, 2011ApJ...731..120M} are available and have for
the most part been used to study SN~Ia heterogeneity.}
They remain two-parameter models, with one parameter characterizing light-curve shape and the other
 color.
In SALT2 the light curve shapes are described by a multiplicative scaling (``stretch'')  of the time evolution,
 whereas MLCS2k2 varies shapes through additive time-dependent magnitude corrections.
The physical cause of the color diversity is interpreted differently by the two sets of authors: 
\citet{2007A&A...466...11G} pragmatically extract color variation empirically from SNe that span a wide range of colors, with no attribution
to either intrinsic or extrinsic origins;
\citet{2007ApJ...659..122J}
attribute changes in color
partially to intrinsic variations linked to light-curve shape, and partially
to the reddening of light from host-galaxy dust.  Differences between these models produce differences in the results of
analyses of both low-redshift \citep{2007ApJ...664L..13C} and high-redshift \citep{2009ApJS..185...32K} supernovae.

There is evidence that supports the expectation that a single parameter beyond light-curve shape  cannot describe the full range
of colors seen in the SN~Ia population.  One approach to look for color diversity is to find correlations between color and spectral features.
\citet{2009ApJ...699L.139W, 2011ApJ...729...55F} find two subpopulations distinguished
by SiII velocity with differing $B_{max}-V_{max}$; this color correlation, in addition to one with $B-R$, is confirmed by
\citet{2014ApJ...797...75M}.
\citet{2015MNRAS.451.1973S}
find that high-velocity SiII$\lambda$6355 is found in objects that have red ultraviolet/optical colors near maximum brightness.
\citet{2011MNRAS.413.3075M} show evidence that supernova asymmetry and viewing angle,
traced by wavelength shifts in nebular emission lines, is an important determinant in controlling supernova color; such correlations are also seen by \citet{2011A&A...534L..15C}.

Another approach to probe color diversity is through multiple colors (at least 3 bands)
of individual supernovae.  Color ratios are sensitive to processes of the responsible physics.   For example,
relative dust absorption varies as a function of wavelength depending on grain size, distribution, composition and shape,
independent (to first order) of the amount of dust along the line of sight.
\citet{2013ApJ...779...23M} find diversity in SNe~Ia UV emission, which they use to distinguish subclasses based on UV-optical colors.
Measurements of color ratios are being advanced with the development of flexible empirical light curve models that accommodate flexibility in multi-band colors
\citep[e.g.][]{2011ApJ...731..120M}.
\citet{2014ApJ...789...32B, 2015MNRAS.453.3300A} find wide
ranges of total-to-selective extinction with average values significantly lower than $R_V = 3.1$,
the canonical value for diffuse Milky Way dust.
They also confirm the \citet{2011ApJ...731..120M, 2011ApJ...729...55F} finding that low $R_V$ is associated with high-extinction supernovae.
In contrast, \citet{2011A&A...529L...4C} argue that after accounting for the diversity of spectral features,
the $R_V=3.1$ measured for the diffuse Milky Way dust is recovered on average and \citet{2017arXiv170101422H}
find $R_V=2.95 \pm 0.08$ for the highly-extincted SN~2012cu.

Hierarchical modeling has recently opened
the study of intrinsic supernova color based on SN~Ia Hubble diagrams. Latent parameters that are not directly tied to observables
but  influence color can be included in such models.
\citet{2016arXiv160904470M} take the approach of modeling the distribution of their parameters, to find that
scatter in the Hubble diagram is better explained by a combination of intrinsic color dispersion and
$R_V=2.7$ dust, rather than by dust with no color dispersion.
They draw these conclusions by using only the SALT2 (v2.4) $x_1$ parameter as the summary statistic that describes color.

The Nearby Supernova Factory \citep[SNfactory;][]{2002SPIE.4836...61A} has systematically observed the
spectrophotometric time series of hundreds of Hubble-flow $0.03<z<0.08$ SNe~Ia.   The $3200$--$10000$~\AA\ spectral coverage
provides measurements of an array of supernova spectral features while also enablng synthetic broadband photometry
spanning near-UV to near-IR SN-frame wavelengths.  SNfactory specifically targeted objects
early in their temporal evolution, so that well over a hundred of these supernovae have  coverage over
peak brightness.  This dataset provides a homogenous sample with which to study SN~Ia colors and spectral features simultaneously.

In this article we use the idea that spectral indicators carry information on intrinsic supernova colors at peak magnitude.
This approach is taken by \citet{2011A&A...529L...4C}, who find that after standardization based on Ca and Si features, remaining residual color
variation is consistent with Milky Way dust models.
We accommodate up to  three independent color parameters: the model is constructed such that 
two are assigned as extrinsic processes attributed to dust, and one is attributed to intrinsic supernova
properties.  (We subsequently find that extrinsic and intrinsic properties cannot be distinguished with our data.)
The data used in this analysis are described in \S\ref{data:sec}.
In \S\ref{model:sec} we present a
first analysis with two extrinsic parameters and omitting the intrinsic parameter.  The properties of the two extrinsic parameters
are consistent with the expectations of parameterized Milky Way extinction models, and do an excellent job of fitting the data of the out-of-sample
supernova SN2014J that has an extremely low total-to-selective extinction.
In \S\ref{model2:sec} we add the intrinsic parameter in the analysis, and find strong evidence for its influence on observed magnitudes.
We summarize our findings, and relate our results to light-curve shape, SiII velocity, and the host-galaxy global stellar ``mass step'' in SN~Ia Hubble
residuals in \S\ref{discussion:sec}.


\section{Data}
\label{data:sec}

Our analysis uses the spectrophotometric data set obtained by
the SNfactory with the SuperNova Integral Field
Spectrograph \citep[SNIFS,][]{2002SPIE.4836...61A, 2004SPIE.5249..146L}.  SNIFS is a fully integrated
instrument optimized for automated observation of point sources on a
structured background over the full ground-based optical window at
moderate spectral resolution ($R \sim 500$).  It consists of a
high-throughput wide-band lenslet integral field spectrograph, a mult-band
imager that covers the field in the vicinity of
the IFS for atmospheric transmission monitoring simultaneous with
spectroscopy, and an acquisition/guiding channel.  The IFS possesses a
fully-filled $6\farcs 4 \times 6\farcs 4$ spectroscopic field of view
subdivided into a grid of $15 \times 15$ spatial elements, a
dual-channel spectrograph covering 3200--5200~\AA\ and 5100--10000~\AA\
simultaneously, and an internal calibration unit (continuum and arc
lamps).  SNIFS is mounted on the south bent Cassegrain port of the
University of Hawaii 2.2~m telescope on Mauna Kea, and is operated
remotely.  Observations are reduced using the SNfactory's dedicated data
reduction pipeline, similar to that presented in \S4 of \citet{2001MNRAS.326...23B}.
A discussion of the software pipeline is presented in
\citet{2006ApJ...650..510A} and is updated in \citet{2010ApJ...713.1073S}. 
The flux calibration is presented in \citet{2013A&A...549A...8B}.
A detailed
description of host-galaxy subtraction is given in \citet{2011MNRAS.418..258B}.

A recent description of the data is presented in \citet{2015ApJ...815...58F}.
We provide a brief summary of the points important for this analysis.
The spectral time-series  are corrected for Milky Way dust
extinction \citep{1989ApJ...345..245C,1998ApJ...500..525S}.  
Each spectral time series is
blue-shifted to rest-frame
based on the systemic redshift of the host \citep[c.f.][]{2013ApJ...770..107C}, and the fluxes are converted to luminosity assuming
distances expected for the supernova redshifts given a flat
$\Lambda$CDM cosmology with $\Omega_M = 0.28$ (with an arbitrarily selected
$H_0$ since the current analysis does not depend on absolute magnitude).

Synthetic supernova-frame photometry is generated for a top-hat filter system
comprised of five 
bands with the following wavelength ranges: $U$ $[3300.00 - 3978.02]$\AA;
$B$ $[3978.02-4795.35]$\AA;
$V$ $[4795.35-5780.60]$\AA;
$R$ $[5780.60-6968.29]$\AA;
$I$ $[6968.29-8400.00]$\AA.
For each supernova, the magnitudes within 5-days of peak brightness are used to regress single-band magnitudes
at $B$-band peak brightness.
The equivalent widths of SiII$\lambda 4141$ and the CaII H\&K features are computed as
in \citet{2008A&A...477..717B} and the 
wavelength of the SiII$\lambda 6355$ feature
as in \citet{chotard:thesis, 2017Chotard}.
Equivalent widths and the
SiII$\lambda 6355$ wavelength are taken from spectra  within $\pm 2.5$ days from $B$-band maximum;
the average is used  in cases where there are multiple spectral measurements within that time window.

Our analysis sample is comprised by the
172
supernovae that have the data coverage to 
give photometric and spectroscopic statistics described above.
%\textcolor{red}{FOR INTERNAL REFERENCE: The CABALLO2 validation and training sets are used with SN2012cu, SNF20061108-001, SNF20080905-005, SNNGC7589 excised
%based on data problems identified by Manu.}
The data are given in \citet{2017Chotard}.
%
%
%Table~\ref{data:tab} \textcolor{red}{[TABLE CAN GO IF THERE IS A NICO PAPER]}.
%The peak magnitude uncertainties do have covariance, which is accounted
%for in the analysis; only the standard deviation is included in the table.
%
%\startlongtable
%\begin{deluxetable}{crrrrrrrr}
%\tabletypesize{\tiny}
%\tablecaption{Supernova Spectral-Feature and Peak-Magnitude Data
%\label{data:tab}}
%\tablehead{
%\colhead{Name} & \colhead{$EW_{Ca}$ (\AA)} & \colhead{$EW_{Si}$ (\AA)} & \colhead{$\lambda_{Si}$ (\AA)} & \colhead{$U$} & \colhead{$B$} & \colhead{$V$} & \colhead{$R$} & \colhead{$I$}
%}
%\startdata
%SN2007bd & $109.7 \pm 5.9$ & $ 17.5 \pm 0.7$& $ 6101 \pm   3$ & $-29.31 \pm   0.01$ & $-29.12 \pm   0.01$& $-28.60 \pm   0.01$& $-28.35 \pm   0.01$& $-27.60 \pm   0.01$ \\
%PTF10zdk & $149.7 \pm 1.2$ & $ 14.3 \pm 0.6$& $ 6150 \pm   3$ & $-28.61 \pm   0.02$ & $-28.69 \pm   0.02$& $-28.32 \pm   0.02$& $-28.08 \pm   0.02$& $-27.40 \pm   0.02$ \\
%SNF20080815-017 & $ 63.8 \pm 21.5$ & $ 27.6 \pm 3.8$& $ 6132 \pm   6$ & $-29.04 \pm   0.07$ & $-28.79 \pm   0.07$& $-28.32 \pm   0.07$& $-28.12 \pm   0.07$& $-27.41 \pm   0.07$ \\
%PTF09dnl & $129.9 \pm 0.9$ & $  9.5 \pm 0.7$& $ 6093 \pm   3$ & $-29.23 \pm   0.01$ & $-29.07 \pm   0.01$& $-28.72 \pm   0.01$& $-28.44 \pm   0.01$& $-27.69 \pm   0.01$ \\
%SN2010ex & $114.4 \pm 0.9$ & $  8.4 \pm 0.4$& $ 6129 \pm   6$ & $-29.26 \pm   0.01$ & $-28.99 \pm   0.01$& $-28.50 \pm   0.01$& $-28.20 \pm   0.01$& $-27.44 \pm   0.01$ \\
%PTF09dnp & $ 64.9 \pm 4.5$ & $ 16.5 \pm 0.7$& $ 6098 \pm   4$ & $-29.55 \pm   0.02$ & $-29.19 \pm   0.02$& $-28.68 \pm   0.02$& $-28.48 \pm   0.02$& $-27.93 \pm   0.02$ \\
%PTF11bnx & $151.4 \pm 3.0$ & $ 13.9 \pm 1.1$& $ 6142 \pm   5$ & $-28.63 \pm   0.02$ & $-28.57 \pm   0.01$& $-28.20 \pm   0.01$& $-27.99 \pm   0.01$& $-27.34 \pm   0.01$ \\
%PTF12jqh & $151.9 \pm 1.5$ & $  7.9 \pm 0.7$& $ 6116 \pm  10$ & $-29.37 \pm   0.01$ & $-29.14 \pm   0.01$& $-28.71 \pm   0.01$& $-28.40 \pm   0.01$& $-27.64 \pm   0.01$ \\
%SNF20080802-006 & $108.2 \pm 6.0$ & $ 20.6 \pm 1.9$& $ 6122 \pm   5$ & $-29.02 \pm   0.06$ & $-28.80 \pm   0.06$& $-28.40 \pm   0.06$& $-28.20 \pm   0.06$& $-27.50 \pm   0.06$ \\
%PTF10xyt & $123.7 \pm 6.6$ & $ 16.4 \pm 4.3$& $ 6101 \pm   4$ & $-28.26 \pm   0.02$ & $-28.20 \pm   0.02$& $-27.93 \pm   0.02$& $-27.74 \pm   0.02$& $-27.22 \pm   0.04$ \\
%PTF11qmo & $101.7 \pm 1.1$ & $  7.7 \pm 0.7$& $ 6150 \pm   8$ & $-29.77 \pm   0.02$ & $-29.43 \pm   0.02$& $-28.97 \pm   0.02$& $-28.64 \pm   0.02$& $-27.93 \pm   0.02$ \\
%SNF20070331-025 & $119.8 \pm 7.4$ & $ 14.2 \pm 2.7$& $ 6120 \pm  10$ & $-28.94 \pm   0.02$ & $-28.75 \pm   0.02$& $-28.32 \pm   0.02$& $-28.07 \pm   0.02$& $-27.31 \pm   0.02$ \\
%SNF20070818-001 & $157.5 \pm 7.5$ & $ 16.7 \pm 1.8$& $ 6115 \pm   5$ & $-28.97 \pm   0.02$ & $-28.96 \pm   0.01$& $-28.61 \pm   0.01$& $-28.37 \pm   0.01$& $-27.62 \pm   0.01$ \\
%SNBOSS38 & $ 57.1 \pm 0.4$ & $ 17.9 \pm 0.3$& $ 6127 \pm   3$ & $-29.20 \pm   0.01$ & $-28.84 \pm   0.01$& $-28.47 \pm   0.01$& $-28.23 \pm   0.01$& $-27.73 \pm   0.04$ \\
%SN2006ob & $ 90.0 \pm 16.5$ & $ 26.5 \pm 1.5$& $ 6112 \pm   5$ & $-29.11 \pm   0.02$ & $-28.82 \pm   0.01$& $-28.42 \pm   0.01$& $-28.19 \pm   0.01$& $-27.54 \pm   0.01$ \\
%PTF12eer & $165.6 \pm 10.7$ & $ 12.7 \pm 2.8$& $ 6150 \pm  10$ & $-28.76 \pm   0.01$ & $-28.76 \pm   0.01$& $-28.40 \pm   0.01$& $-28.17 \pm   0.01$& $-27.45 \pm   0.02$ \\
%PTF10ops & $ 38.7 \pm 9.9$ & $  7.2 \pm 8.7$& $ 6140 \pm   5$ & $-27.93 \pm   0.38$ & $-27.76 \pm   0.38$& $-27.73 \pm   0.38$& $-27.59 \pm   0.38$& $-27.21 \pm   0.38$ \\
%SNF20080514-002 & $ 83.2 \pm 0.7$ & $ 19.4 \pm 0.6$& $ 6131 \pm   3$ & $-29.30 \pm   0.01$ & $-28.95 \pm   0.01$& $-28.44 \pm   0.01$& $-28.17 \pm   0.01$& $-27.49 \pm   0.01$ \\
%PTF12evo & $129.2 \pm 2.8$ & $  9.1 \pm 1.3$& $ 6156 \pm   4$ & $-29.14 \pm   0.02$ & $-28.98 \pm   0.01$& $-28.56 \pm   0.01$& $-28.28 \pm   0.01$& $-27.61 \pm   0.01$ \\
%SNF20080614-010 & $125.4 \pm 5.1$ & $ 26.9 \pm 1.6$& $ 6128 \pm   3$ & $-29.04 \pm   0.04$ & $-28.81 \pm   0.04$& $-28.38 \pm   0.04$& $-28.16 \pm   0.04$& $-27.57 \pm   0.04$ \\
%PTF10icb & $104.8 \pm 0.9$ & $ 12.7 \pm 0.3$& $ 6138 \pm   3$ & $-28.58 \pm   0.02$ & $-28.36 \pm   0.02$& $-27.98 \pm   0.02$& $-27.77 \pm   0.02$& $-27.17 \pm   0.02$ \\
%PTF12efn & $144.9 \pm 3.4$ & $  7.1 \pm 1.8$& $ 6115 \pm   3$ & $-29.40 \pm   0.01$ & $-29.17 \pm   0.01$& $-28.79 \pm   0.01$& $-28.45 \pm   0.01$& $-27.64 \pm   0.01$ \\
%SNNGC4424 & $109.0 \pm 0.3$ & $  8.6 \pm 0.1$& $ 6138 \pm   2$ & $-28.35 \pm   0.01$ & $-28.15 \pm   0.01$& $-27.79 \pm   0.01$& $-27.58 \pm   0.01$& $-26.97 \pm   0.01$ \\
%SNF20080516-022 & $100.1 \pm 2.1$ & $ 13.7 \pm 1.1$& $ 6158 \pm   3$ & $-29.46 \pm   0.01$ & $-29.19 \pm   0.01$& $-28.71 \pm   0.01$& $-28.39 \pm   0.01$& $-27.77 \pm   0.01$ \\
%PTF12hwb & $ 21.1 \pm 78.0$ & $ -1.8 \pm 8.9$& $ 6090 \pm  14$ & $-28.32 \pm   0.02$ & $-28.24 \pm   0.02$& $-28.03 \pm   0.02$& $-27.79 \pm   0.02$& $-27.05 \pm   0.04$ \\
%PTF10qyz & $106.4 \pm 2.1$ & $ 23.0 \pm 1.0$& $ 6120 \pm   5$ & $-29.05 \pm   0.17$ & $-28.92 \pm   0.17$& $-28.41 \pm   0.17$& $-28.14 \pm   0.17$& $-27.30 \pm   0.17$ \\
%SNF20060907-000 & $106.1 \pm 10.4$ & $ 17.0 \pm 0.9$& $ 6149 \pm   4$ & $-29.54 \pm   0.02$ & $-29.28 \pm   0.01$& $-28.76 \pm   0.01$& $-28.42 \pm   0.01$& $-27.74 \pm   0.04$ \\
%LSQ12fxd & $122.9 \pm 1.7$ & $ 11.4 \pm 0.8$& $ 6119 \pm   4$ & $-29.62 \pm   0.07$ & $-29.39 \pm   0.07$& $-28.95 \pm   0.07$& $-28.64 \pm   0.07$& $-27.91 \pm   0.07$ \\
%SNF20080821-000 & $105.1 \pm 2.2$ & $  8.6 \pm 1.3$& $ 6121 \pm   4$ & $-29.34 \pm   0.01$ & $-29.10 \pm   0.01$& $-28.73 \pm   0.01$& $-28.46 \pm   0.01$& $-27.82 \pm   0.01$ \\
%SNF20070802-000 & $158.3 \pm 3.3$ & $ 16.3 \pm 1.7$& $ 6102 \pm   5$ & $-28.90 \pm   0.01$ & $-28.81 \pm   0.01$& $-28.45 \pm   0.01$& $-28.22 \pm   0.01$& $-27.52 \pm   0.01$ \\
%PTF10wnm & $105.8 \pm 2.3$ & $  6.5 \pm 1.0$& $ 6124 \pm   3$ & $-29.38 \pm   0.01$ & $-29.07 \pm   0.01$& $-28.68 \pm   0.01$& $-28.37 \pm   0.01$& $-27.69 \pm   0.01$ \\
%PTF10mwb & $116.5 \pm 1.2$ & $ 19.8 \pm 0.9$& $ 6138 \pm   2$ & $-29.02 \pm   0.07$ & $-28.84 \pm   0.07$& $-28.40 \pm   0.07$& $-28.14 \pm   0.07$& $-27.52 \pm   0.07$ \\
%SN2010dt & $116.2 \pm 14.9$ & $ 15.5 \pm 0.7$& $ 6138 \pm   6$ & $-29.30 \pm   0.01$ & $-29.15 \pm   0.01$& $-28.64 \pm   0.01$& $-28.35 \pm   0.01$& $-27.63 \pm   0.01$ \\
%SNF20080623-001 & $149.1 \pm 1.4$ & $ 14.9 \pm 0.7$& $ 6131 \pm   3$ & $-29.11 \pm   0.01$ & $-28.97 \pm   0.01$& $-28.50 \pm   0.01$& $-28.22 \pm   0.01$& $-27.46 \pm   0.01$ \\
%LSQ12fhe & $ 42.8 \pm 1.2$ & $  4.0 \pm 3.1$& $ 6108 \pm   4$ & $-29.76 \pm   0.02$ & $-29.40 \pm   0.02$& $-29.04 \pm   0.02$& $-28.74 \pm   0.02$& $-28.11 \pm   0.02$ \\
%PTF11bju & $ 30.2 \pm 4.4$ & $  4.0 \pm 3.0$& $ 6139 \pm   5$ & $-29.47 \pm   0.02$ & $-29.10 \pm   0.01$& $-28.75 \pm   0.01$& $-28.45 \pm   0.01$& $-27.87 \pm   0.01$ \\
%PTF09fox & $117.6 \pm 2.7$ & $  9.1 \pm 1.0$& $ 6116 \pm   3$ & $-29.44 \pm   0.03$ & $-29.21 \pm   0.03$& $-28.72 \pm   0.03$& $-28.42 \pm   0.03$& $-27.68 \pm   0.03$ \\
%PTF13ayw & $104.6 \pm 2.4$ & $ 26.6 \pm 3.2$& $ 6115 \pm   6$ & $-29.16 \pm   0.02$ & $-28.82 \pm   0.02$& $-28.43 \pm   0.02$& $-28.20 \pm   0.02$& $-27.55 \pm   0.02$ \\
%SNF20070810-004 & $126.7 \pm 1.8$ & $ 21.1 \pm 1.1$& $ 6118 \pm   7$ & $-29.22 \pm   0.01$ & $-29.10 \pm   0.01$& $-28.63 \pm   0.01$& $-28.34 \pm   0.01$& $-27.62 \pm   0.01$ \\
%PTF11mty & $111.4 \pm 2.3$ & $ 10.6 \pm 1.5$& $ 6138 \pm   5$ & $-29.54 \pm   0.01$ & $-29.23 \pm   0.01$& $-28.80 \pm   0.01$& $-28.46 \pm   0.01$& $-27.82 \pm   0.01$ \\
%SNF20080512-010 & $ 95.3 \pm 3.5$ & $ 23.3 \pm 1.5$& $ 6129 \pm   5$ & $-29.22 \pm   0.08$ & $-28.96 \pm   0.08$& $-28.50 \pm   0.08$& $-28.26 \pm   0.08$& $-27.56 \pm   0.08$ \\
%PTF11mkx & $ 31.5 \pm 3.7$ & $  4.5 \pm 1.3$& $ 6169 \pm   5$ & $-29.50 \pm   0.45$ & $-29.25 \pm   0.45$& $-28.89 \pm   0.45$& $-28.61 \pm   0.45$& $-27.97 \pm   0.45$ \\
%PTF10tce & $135.7 \pm 1.1$ & $ 11.2 \pm 1.5$& $ 6090 \pm   4$ & $-29.13 \pm   0.02$ & $-28.99 \pm   0.01$& $-28.59 \pm   0.01$& $-28.31 \pm   0.01$& $-27.55 \pm   0.01$ \\
%SNF20061020-000 & $ 95.4 \pm 18.8$ & $ 24.1 \pm 1.0$& $ 6120 \pm   5$ & $-29.01 \pm   0.03$ & $-28.78 \pm   0.03$& $-28.35 \pm   0.03$& $-28.17 \pm   0.03$& $-27.54 \pm   0.03$ \\
%SN2005ir & $115.6 \pm 2.8$ & $ 13.5 \pm 6.9$& $ 6069 \pm   5$ & $-29.33 \pm   0.02$ & $-29.12 \pm   0.02$& $-28.84 \pm   0.02$& $-28.49 \pm   0.02$& $-27.77 \pm   0.02$ \\
%SNF20080717-000 & $ 93.3 \pm 2.6$ & $  8.3 \pm 2.2$& $ 6104 \pm   3$ & $-28.58 \pm   0.01$ & $-28.47 \pm   0.01$& $-28.29 \pm   0.01$& $-28.05 \pm   0.01$& $-27.50 \pm   0.01$ \\
%PTF12ena & $101.1 \pm 1.6$ & $  7.4 \pm 1.0$& $ 6129 \pm   4$ & $-28.01 \pm   0.01$ & $-28.00 \pm   0.01$& $-27.85 \pm   0.01$& $-27.77 \pm   0.01$& $-27.31 \pm   0.01$ \\
%PTF13anh & $166.8 \pm 1.8$ & $ 21.8 \pm 1.2$& $ 6175 \pm   4$ & $-28.67 \pm   0.20$ & $-28.74 \pm   0.20$& $-28.30 \pm   0.20$& $-28.05 \pm   0.20$& $-27.28 \pm   0.20$ \\
%CSS110918\_01 & $110.6 \pm 1.0$ & $  8.0 \pm 1.3$& $ 6101 \pm   2$ & $-29.88 \pm   0.76$ & $-29.58 \pm   0.76$& $-29.09 \pm   0.76$& $-28.71 \pm   0.76$& $-27.91 \pm   0.76$ \\
%SNF20061024-000 & $ 86.9 \pm 26.8$ & $ 30.0 \pm 1.5$& $ 6127 \pm   5$ & $-28.88 \pm   0.04$ & $-28.70 \pm   0.04$& $-28.26 \pm   0.04$& $-28.05 \pm   0.04$& $-27.40 \pm   0.04$ \\
%SNF20070506-006 & $ 94.1 \pm 1.3$ & $  6.7 \pm 0.6$& $ 6153 \pm   3$ & $-29.72 \pm   0.01$ & $-29.39 \pm   0.01$& $-28.97 \pm   0.01$& $-28.64 \pm   0.01$& $-27.96 \pm   0.01$ \\
%SNF20070403-001 & $105.9 \pm 5.4$ & $ 18.3 \pm 1.8$& $ 6124 \pm   4$ & $-29.23 \pm   0.02$ & $-29.04 \pm   0.01$& $-28.63 \pm   0.01$& $-28.35 \pm   0.01$& $-27.62 \pm   0.01$ \\
%PTF10hmv & $109.6 \pm 1.3$ & $  8.9 \pm 0.7$& $ 6143 \pm   3$ & $-28.54 \pm   0.01$ & $-28.40 \pm   0.01$& $-28.11 \pm   0.01$& $-27.89 \pm   0.01$& $-27.31 \pm   0.01$ \\
%SNF20071015-000 & $105.0 \pm 3.2$ & $  6.9 \pm 1.1$& $ 6124 \pm   7$ & $-27.89 \pm   0.02$ & $-27.82 \pm   0.02$& $-27.69 \pm   0.02$& $-27.63 \pm   0.02$& $-27.16 \pm   0.04$ \\
%SNhunt89 & $ 88.0 \pm 2.7$ & $ 32.2 \pm 1.9$& $ 6111 \pm   7$ & $-28.37 \pm   0.03$ & $-28.26 \pm   0.03$& $-27.92 \pm   0.03$& $-27.77 \pm   0.03$& $-27.13 \pm   0.03$ \\
%SNF20070902-021 & $108.9 \pm 3.5$ & $ 17.1 \pm 1.0$& $ 6131 \pm   6$ & $-29.25 \pm   0.02$ & $-29.02 \pm   0.02$& $-28.56 \pm   0.02$& $-28.32 \pm   0.01$& $-27.65 \pm   0.02$ \\
%PTF09dlc & $143.5 \pm 2.2$ & $ 10.2 \pm 0.9$& $ 6143 \pm   3$ & $-29.38 \pm   0.01$ & $-29.17 \pm   0.01$& $-28.69 \pm   0.01$& $-28.40 \pm   0.01$& $-27.62 \pm   0.01$ \\
%PTF13ajv & $150.5 \pm 8.9$ & $ 46.3 \pm 8.6$& $ 6110 \pm  21$ & $-28.70 \pm   0.02$ & $-28.61 \pm   0.02$& $-28.16 \pm   0.02$& $-27.91 \pm   0.02$& $-27.07 \pm   0.04$ \\
%SNF20080919-000 & $114.7 \pm 2.8$ & $  9.4 \pm 0.9$& $ 6145 \pm   5$ & $-28.53 \pm   0.02$ & $-28.41 \pm   0.01$& $-28.11 \pm   0.01$& $-27.99 \pm   0.01$& $-27.38 \pm   0.01$ \\
%SNF20080919-001 & $ 85.0 \pm 1.1$ & $  6.0 \pm 0.4$& $ 6150 \pm   5$ & $-29.73 \pm   0.01$ & $-29.43 \pm   0.01$& $-29.04 \pm   0.01$& $-28.72 \pm   0.01$& $-28.07 \pm   0.01$ \\
%SN2010kg & $ 95.1 \pm 28.5$ & $ 21.7 \pm 0.7$& $ 6077 \pm   5$ & $-28.85 \pm   0.01$ & $-28.74 \pm   0.01$& $-28.41 \pm   0.01$& $-28.20 \pm   0.01$& $-27.47 \pm   0.01$ \\
%SNF20080714-008 & $134.8 \pm 15.7$ & $ 19.7 \pm 3.7$& $ 6100 \pm   6$ & $-28.56 \pm   0.02$ & $-28.63 \pm   0.01$& $-28.32 \pm   0.01$& $-28.13 \pm   0.01$& $-27.42 \pm   0.01$ \\
%SNF20070714-007 & $129.6 \pm 5.6$ & $ 31.1 \pm 23.8$& $ 6146 \pm   5$ & $-27.88 \pm   0.02$ & $-28.12 \pm   0.01$& $-28.02 \pm   0.01$& $-27.86 \pm   0.01$& $-27.24 \pm   0.03$ \\
%SNF20080522-011 & $122.1 \pm 1.7$ & $  8.3 \pm 0.5$& $ 6125 \pm   3$ & $-29.63 \pm   0.01$ & $-29.38 \pm   0.01$& $-28.92 \pm   0.01$& $-28.60 \pm   0.01$& $-27.88 \pm   0.01$ \\
%SNF20061111-002 & $110.8 \pm 10.7$ & $ 20.4 \pm 1.0$& $ 6145 \pm   6$ & $-29.16 \pm   0.01$ & $-28.99 \pm   0.01$& $-28.59 \pm   0.01$& $-28.29 \pm   0.01$& $-27.61 \pm   0.01$ \\
%SNNGC6343 & $ 87.0 \pm 1.4$ & $ 20.7 \pm 0.7$& $ 6136 \pm   3$ & $-28.78 \pm   0.01$ & $-28.66 \pm   0.01$& $-28.30 \pm   0.01$& $-28.08 \pm   0.01$& $-27.41 \pm   0.01$ \\
%SNF20061011-005 & $120.6 \pm 1.1$ & $  9.3 \pm 0.4$& $ 6132 \pm   4$ & $-29.72 \pm   0.04$ & $-29.43 \pm   0.03$& $-28.99 \pm   0.03$& $-28.64 \pm   0.03$& $-27.90 \pm   0.03$ \\
%SNF20080825-010 & $102.4 \pm 13.4$ & $ 19.2 \pm 0.6$& $ 6116 \pm   4$ & $-29.46 \pm   0.01$ & $-29.17 \pm   0.01$& $-28.71 \pm   0.01$& $-28.47 \pm   0.01$& $-27.83 \pm   0.01$ \\
%PTF10ufj & $141.1 \pm 3.4$ & $ 11.7 \pm 1.2$& $ 6131 \pm   6$ & $-29.28 \pm   0.15$ & $-29.16 \pm   0.15$& $-28.72 \pm   0.15$& $-28.41 \pm   0.15$& $-27.65 \pm   0.15$ \\
%PTF10wof & $129.6 \pm 2.7$ & $ 17.3 \pm 1.0$& $ 6102 \pm   2$ & $-28.91 \pm   0.01$ & $-28.84 \pm   0.01$& $-28.46 \pm   0.01$& $-28.18 \pm   0.01$& $-27.43 \pm   0.01$ \\
%SNF20080918-000 & $146.8 \pm 3.5$ & $  7.5 \pm 2.5$& $ 6110 \pm   5$ & $-28.79 \pm   0.02$ & $-28.65 \pm   0.02$& $-28.35 \pm   0.02$& $-28.12 \pm   0.02$& $-27.46 \pm   0.02$ \\
%SNF20080516-000 & $117.4 \pm 2.2$ & $  9.0 \pm 1.2$& $ 6135 \pm   3$ & $-29.50 \pm   0.01$ & $-29.23 \pm   0.01$& $-28.80 \pm   0.01$& $-28.47 \pm   0.01$& $-27.74 \pm   0.01$ \\
%SN2005cf & $159.1 \pm 0.7$ & $ 15.7 \pm 0.8$& $ 6141 \pm   3$ & $-29.37 \pm   0.02$ & $-29.16 \pm   0.02$& $-28.68 \pm   0.02$& $-28.41 \pm   0.02$& $-27.69 \pm   0.02$ \\
%CSS130502\_01 & $ 91.5 \pm 10.9$ & $ 15.6 \pm 0.5$& $ 6128 \pm   3$ & $-29.43 \pm   0.02$ & $-29.09 \pm   0.02$& $-28.60 \pm   0.01$& $-28.30 \pm   0.01$& $-27.62 \pm   0.04$ \\
%SNF20080620-000 & $107.8 \pm 14.1$ & $ 20.0 \pm 0.7$& $ 6132 \pm   3$ & $-28.82 \pm   0.02$ & $-28.78 \pm   0.01$& $-28.32 \pm   0.01$& $-28.09 \pm   0.01$& $-27.39 \pm   0.01$ \\
%SNPGC51271 & $ 92.1 \pm 16.5$ & $ 21.1 \pm 0.7$& $ 6121 \pm   2$ & $-29.28 \pm   0.02$ & $-28.95 \pm   0.02$& $-28.46 \pm   0.02$& $-28.20 \pm   0.02$& $-27.62 \pm   0.04$ \\
%PTF11pdk & $128.6 \pm 2.8$ & $ 15.6 \pm 1.7$& $ 6153 \pm   5$ & $-29.35 \pm   0.02$ & $-29.11 \pm   0.02$& $-28.61 \pm   0.02$& $-28.32 \pm   0.02$& $-27.67 \pm   0.02$ \\
%SNF20060511-014 & $102.6 \pm 2.8$ & $ 15.6 \pm 1.1$& $ 6141 \pm   8$ & $-29.16 \pm   0.07$ & $-29.04 \pm   0.06$& $-28.56 \pm   0.06$& $-28.30 \pm   0.06$& $-27.63 \pm   0.06$ \\
%SNF20080612-003 & $120.0 \pm 1.1$ & $  7.3 \pm 0.6$& $ 6123 \pm   3$ & $-29.64 \pm   0.02$ & $-29.41 \pm   0.02$& $-28.99 \pm   0.02$& $-28.70 \pm   0.02$& $-28.00 \pm   0.02$ \\
%SNF20080626-002 & $130.0 \pm 1.0$ & $  6.1 \pm 4.2$& $ 6111 \pm   3$ & $-29.42 \pm   0.01$ & $-29.24 \pm   0.01$& $-28.84 \pm   0.01$& $-28.52 \pm   0.01$& $-27.76 \pm   0.01$ \\
%SNF20060621-015 & $111.9 \pm 1.3$ & $  9.8 \pm 0.7$& $ 6144 \pm   3$ & $-29.63 \pm   0.01$ & $-29.36 \pm   0.01$& $-28.88 \pm   0.01$& $-28.54 \pm   0.01$& $-27.81 \pm   0.01$ \\
%SNF20080920-000 & $135.2 \pm 1.4$ & $  5.6 \pm 1.6$& $ 6085 \pm   3$ & $-29.44 \pm   0.02$ & $-29.19 \pm   0.02$& $-28.79 \pm   0.02$& $-28.49 \pm   0.02$& $-27.74 \pm   0.02$ \\
%SN2007cq & $ 65.8 \pm 4.1$ & $ 10.2 \pm 0.9$& $ 6137 \pm   3$ & $-29.53 \pm   0.02$ & $-29.30 \pm   0.02$& $-28.89 \pm   0.02$& $-28.56 \pm   0.02$& $-27.90 \pm   0.02$ \\
%SNF20080918-004 & $ 87.8 \pm 7.2$ & $ 21.5 \pm 0.9$& $ 6141 \pm   4$ & $-29.00 \pm   0.22$ & $-28.82 \pm   0.22$& $-28.37 \pm   0.22$& $-28.13 \pm   0.22$& $-27.43 \pm   0.22$ \\
%CSS120424\_01 & $138.1 \pm 2.1$ & $ 11.7 \pm 0.7$& $ 6138 \pm   3$ & $-29.40 \pm   0.02$ & $-29.23 \pm   0.02$& $-28.77 \pm   0.01$& $-28.45 \pm   0.02$& $-27.68 \pm   0.02$ \\
%SNF20080610-000 & $119.9 \pm 10.4$ & $ 16.4 \pm 1.7$& $ 6131 \pm   6$ & $-29.05 \pm   0.07$ & $-28.92 \pm   0.07$& $-28.50 \pm   0.07$& $-28.22 \pm   0.07$& $-27.55 \pm   0.07$ \\
%SNF20070701-005 & $101.8 \pm 2.6$ & $ 12.4 \pm 1.0$& $ 6158 \pm   5$ & $-29.46 \pm   0.02$ & $-29.27 \pm   0.02$& $-28.87 \pm   0.02$& $-28.60 \pm   0.02$& $-27.96 \pm   0.02$ \\
%SN2007kk & $128.5 \pm 1.4$ & $ 10.6 \pm 1.0$& $ 6098 \pm   4$ & $-29.48 \pm   0.02$ & $-29.31 \pm   0.02$& $-28.87 \pm   0.01$& $-28.54 \pm   0.01$& $-27.77 \pm   0.02$ \\
%SNF20060908-004 & $114.4 \pm 1.2$ & $ 12.6 \pm 0.6$& $ 6136 \pm   3$ & $-29.59 \pm   0.23$ & $-29.34 \pm   0.23$& $-28.91 \pm   0.23$& $-28.58 \pm   0.23$& $-27.87 \pm   0.23$ \\
%SNF20080909-030 & $ 93.7 \pm 1.0$ & $  7.8 \pm 0.4$& $ 6171 \pm   3$ & $-29.38 \pm   0.02$ & $-29.12 \pm   0.01$& $-28.74 \pm   0.01$& $-28.44 \pm   0.01$& $-27.78 \pm   0.01$ \\
%PTF11bgv & $ 79.4 \pm 3.2$ & $ 12.6 \pm 0.7$& $ 6146 \pm   3$ & $-28.90 \pm   0.02$ & $-28.62 \pm   0.01$& $-28.27 \pm   0.01$& $-28.08 \pm   0.01$& $-27.54 \pm   0.01$ \\
%SNNGC2691 & $ 39.0 \pm 22.2$ & $  4.5 \pm 0.2$& $ 6139 \pm   8$ & $-29.46 \pm   0.02$ & $-29.06 \pm   0.02$& $-28.75 \pm   0.02$& $-28.49 \pm   0.02$& $-27.93 \pm   0.02$ \\
%PTF13asv & $ 75.6 \pm 1.1$ & $  2.2 \pm 0.4$& $ 6148 \pm   4$ & $-29.92 \pm   0.32$ & $-29.49 \pm   0.32$& $-29.02 \pm   0.32$& $-28.63 \pm   0.32$& $-27.90 \pm   0.32$ \\
%SNF20070806-026 & $ 98.8 \pm 12.1$ & $ 25.9 \pm 0.7$& $ 6114 \pm   7$ & $-29.14 \pm   0.02$ & $-28.91 \pm   0.02$& $-28.44 \pm   0.02$& $-28.21 \pm   0.02$& $-27.49 \pm   0.02$ \\
%SNF20070427-001 & $ 81.3 \pm 2.3$ & $  6.3 \pm 0.9$& $ 6142 \pm   5$ & $-29.89 \pm   0.02$ & $-29.46 \pm   0.02$& $-28.97 \pm   0.02$& $-28.62 \pm   0.02$& $-27.97 \pm   0.02$ \\
%SNF20061108-004 & $129.5 \pm 5.6$ & $  6.3 \pm 2.5$& $ 6110 \pm   6$ & $-29.53 \pm   0.02$ & $-29.31 \pm   0.02$& $-28.95 \pm   0.02$& $-28.60 \pm   0.02$& $-27.96 \pm   0.02$ \\
%SNF20060912-000 & $106.5 \pm 1.8$ & $ 21.4 \pm 1.7$& $ 6163 \pm   7$ & $-28.98 \pm   0.02$ & $-28.92 \pm   0.02$& $-28.66 \pm   0.02$& $-28.42 \pm   0.02$& $-27.77 \pm   0.02$ \\
%CSS110918\_02 & $109.1 \pm 9.4$ & $ 15.0 \pm 0.6$& $ 6137 \pm   3$ & $-29.36 \pm   0.02$ & $-29.14 \pm   0.01$& $-28.69 \pm   0.01$& $-28.41 \pm   0.01$& $-27.70 \pm   0.01$ \\
%SNF20080918-002 & $ 97.7 \pm 2.8$ & $ 12.6 \pm 1.4$& $ 6141 \pm   6$ & $-29.50 \pm   0.02$ & $-29.11 \pm   0.02$& $-28.61 \pm   0.02$& $-28.34 \pm   0.02$& $-27.71 \pm   0.02$ \\
%SNIC3573 & $102.7 \pm 1.8$ & $ 11.9 \pm 1.0$& $ 6142 \pm   5$ & $-29.28 \pm   0.02$ & $-29.14 \pm   0.02$& $-28.74 \pm   0.02$& $-28.46 \pm   0.01$& $-27.76 \pm   0.03$ \\
%SNF20080725-004 & $133.6 \pm 2.1$ & $  6.9 \pm 0.9$& $ 6131 \pm   6$ & $-29.09 \pm   0.01$ & $-28.93 \pm   0.01$& $-28.59 \pm   0.01$& $-28.31 \pm   0.01$& $-27.55 \pm   0.03$ \\
%SNF20050728-006 & $127.8 \pm 2.5$ & $ 15.8 \pm 1.3$& $ 6124 \pm   6$ & $-28.80 \pm   0.02$ & $-28.68 \pm   0.02$& $-28.37 \pm   0.02$& $-28.18 \pm   0.02$& $-27.55 \pm   0.02$ \\
%SN2012fr & $134.2 \pm 0.5$ & $  7.4 \pm 0.2$& $ 6102 \pm   1$ & $-29.91 \pm   0.01$ & $-29.70 \pm   0.01$& $-29.31 \pm   0.01$& $-28.94 \pm   0.01$& $-28.10 \pm   0.01$ \\
%SNF20060512-002 & $100.2 \pm 2.8$ & $ 13.4 \pm 1.1$& $ 6107 \pm   8$ & $-29.33 \pm   0.02$ & $-29.11 \pm   0.02$& $-28.77 \pm   0.02$& $-28.52 \pm   0.02$& $-27.80 \pm   0.02$ \\
%SNF20060512-001 & $ 88.4 \pm 1.2$ & $  5.4 \pm 0.4$& $ 6169 \pm   3$ & $-29.33 \pm   0.01$ & $-29.05 \pm   0.01$& $-28.68 \pm   0.01$& $-28.40 \pm   0.01$& $-27.79 \pm   0.01$ \\
%SNF20071003-016 & $125.2 \pm 4.6$ & $ 17.1 \pm 2.0$& $ 6124 \pm  11$ & $-28.58 \pm   0.02$ & $-28.54 \pm   0.02$& $-28.19 \pm   0.02$& $-27.99 \pm   0.02$& $-27.31 \pm   0.02$ \\
%SNF20050821-007 & $141.7 \pm 2.6$ & $  7.7 \pm 1.0$& $ 6140 \pm   9$ & $-29.38 \pm   0.02$ & $-29.20 \pm   0.02$& $-28.77 \pm   0.02$& $-28.46 \pm   0.02$& $-27.67 \pm   0.02$ \\
%SNF20070803-005 & $ 22.7 \pm 21.4$ & $  0.9 \pm 0.6$& $ 6157 \pm  27$ & $-29.87 \pm   0.01$ & $-29.43 \pm   0.01$& $-29.04 \pm   0.01$& $-28.74 \pm   0.01$& $-28.11 \pm   0.01$ \\
%PTF09foz & $127.2 \pm 1.9$ & $ 21.7 \pm 1.2$& $ 6136 \pm   4$ & $-29.14 \pm   0.01$ & $-29.00 \pm   0.01$& $-28.59 \pm   0.01$& $-28.35 \pm   0.01$& $-27.65 \pm   0.01$ \\
%PTF12grk & $162.3 \pm 9.8$ & $ 19.6 \pm 1.4$& $ 6085 \pm   8$ & $-28.86 \pm   0.02$ & $-28.87 \pm   0.01$& $-28.42 \pm   0.01$& $-28.19 \pm   0.01$& $-27.50 \pm   0.03$ \\
%SNF20080720-001 & $138.5 \pm 4.0$ & $ 14.0 \pm 2.0$& $ 6107 \pm   3$ & $-27.59 \pm   0.02$ & $-27.78 \pm   0.01$& $-27.73 \pm   0.01$& $-27.71 \pm   0.01$& $-27.19 \pm   0.02$ \\
%SNF20080810-001 & $ 88.4 \pm 21.6$ & $ 22.3 \pm 1.1$& $ 6145 \pm   5$ & $-29.11 \pm   0.01$ & $-28.89 \pm   0.01$& $-28.45 \pm   0.01$& $-28.23 \pm   0.01$& $-27.60 \pm   0.01$ \\
%SNF20050729-002 & $109.4 \pm 2.2$ & $ 11.5 \pm 1.7$& $ 6142 \pm   6$ & $-29.35 \pm   0.13$ & $-29.17 \pm   0.13$& $-28.68 \pm   0.13$& $-28.38 \pm   0.13$& $-27.56 \pm   0.13$ \\
%SN2008ec & $103.7 \pm 17.0$ & $ 23.1 \pm 0.4$& $ 6125 \pm   3$ & $-28.67 \pm   0.01$ & $-28.52 \pm   0.01$& $-28.18 \pm   0.01$& $-28.03 \pm   0.01$& $-27.47 \pm   0.01$ \\
%SNF20070902-018 & $ 93.8 \pm 12.2$ & $ 23.8 \pm 3.0$& $ 6120 \pm   8$ & $-28.87 \pm   0.02$ & $-28.70 \pm   0.01$& $-28.26 \pm   0.01$& $-28.08 \pm   0.01$& $-27.41 \pm   0.02$ \\
%SNF20070424-003 & $122.5 \pm 3.8$ & $ 12.7 \pm 1.6$& $ 6132 \pm   6$ & $-29.10 \pm   0.01$ & $-28.96 \pm   0.01$& $-28.51 \pm   0.01$& $-28.25 \pm   0.01$& $-27.57 \pm   0.01$ \\
%SN2006cj & $101.7 \pm 1.3$ & $  4.8 \pm 0.8$& $ 6127 \pm   3$ & $-29.43 \pm   0.01$ & $-29.14 \pm   0.01$& $-28.74 \pm   0.01$& $-28.43 \pm   0.01$& $-27.76 \pm   0.01$ \\
%SN2007nq & $ 89.8 \pm 9.9$ & $ 23.4 \pm 1.1$& $ 6109 \pm   5$ & $-29.11 \pm   0.02$ & $-28.91 \pm   0.02$& $-28.50 \pm   0.02$& $-28.27 \pm   0.02$& $-27.57 \pm   0.02$ \\
%SNF20070817-003 & $ 93.9 \pm 2.4$ & $ 18.5 \pm 1.3$& $ 6116 \pm   6$ & $-29.19 \pm   0.02$ & $-29.03 \pm   0.01$& $-28.59 \pm   0.01$& $-28.30 \pm   0.01$& $-27.55 \pm   0.02$ \\
%SNF20070403-000 & $ 61.8 \pm 6.5$ & $ 27.1 \pm 1.8$& $ 6154 \pm   8$ & $-28.37 \pm   0.02$ & $-28.27 \pm   0.02$& $-27.97 \pm   0.02$& $-27.80 \pm   0.02$& $-27.24 \pm   0.02$ \\
%SNF20061022-005 & $ 64.6 \pm 3.8$ & $  3.7 \pm 1.4$& $ 6146 \pm   7$ & $-29.49 \pm   0.02$ & $-29.06 \pm   0.02$& $-28.71 \pm   0.02$& $-28.42 \pm   0.02$& $-27.93 \pm   0.02$ \\
%SNNGC4076 & $127.3 \pm 2.4$ & $ 15.5 \pm 1.2$& $ 6152 \pm   4$ & $-28.77 \pm   0.01$ & $-28.66 \pm   0.01$& $-28.37 \pm   0.01$& $-28.15 \pm   0.01$& $-27.52 \pm   0.01$ \\
%SNF20070727-016 & $ 77.5 \pm 2.5$ & $  5.1 \pm 0.8$& $ 6140 \pm   4$ & $-29.96 \pm   0.06$ & $-29.56 \pm   0.06$& $-29.06 \pm   0.06$& $-28.75 \pm   0.06$& $-28.01 \pm   0.06$ \\
%PTF12fuu & $105.5 \pm 3.0$ & $  6.2 \pm 1.2$& $ 6124 \pm   5$ & $-29.54 \pm   0.01$ & $-29.23 \pm   0.01$& $-28.74 \pm   0.01$& $-28.40 \pm   0.01$& $-27.64 \pm   0.01$ \\
%SNF20070820-000 & $107.2 \pm 3.5$ & $ 18.6 \pm 1.3$& $ 6132 \pm  14$ & $-28.80 \pm   0.02$ & $-28.69 \pm   0.02$& $-28.34 \pm   0.02$& $-28.13 \pm   0.02$& $-27.52 \pm   0.02$ \\
%SNF20070725-001 & $108.4 \pm 2.0$ & $ 11.1 \pm 1.5$& $ 6140 \pm   7$ & $-29.61 \pm   0.02$ & $-29.32 \pm   0.02$& $-28.84 \pm   0.02$& $-28.50 \pm   0.02$& $-27.76 \pm   0.02$ \\
%SNF20071108-021 & $ 99.1 \pm 2.7$ & $  5.8 \pm 0.8$& $ 6164 \pm   5$ & $-29.67 \pm   0.01$ & $-29.34 \pm   0.01$& $-28.94 \pm   0.01$& $-28.60 \pm   0.01$& $-27.96 \pm   0.01$ \\
%SNF20080914-001 & $126.5 \pm 1.2$ & $ 15.4 \pm 1.1$& $ 6159 \pm   3$ & $-28.67 \pm   0.02$ & $-28.60 \pm   0.02$& $-28.31 \pm   0.02$& $-28.13 \pm   0.02$& $-27.58 \pm   0.02$ \\
%SNF20060609-002 & $ 87.7 \pm 3.6$ & $  7.3 \pm 1.3$& $ 6132 \pm   4$ & $-28.60 \pm   0.02$ & $-28.42 \pm   0.02$& $-28.19 \pm   0.02$& $-28.05 \pm   0.02$& $-27.53 \pm   0.02$ \\
%SNF20050624-000 & $121.0 \pm 5.3$ & $  9.3 \pm 3.1$& $ 6126 \pm   6$ & $-29.75 \pm   0.01$ & $-29.42 \pm   0.01$& $-28.99 \pm   0.01$& $-28.68 \pm   0.01$& $-27.97 \pm   0.01$ \\
%SNF20060618-023 & $ 74.9 \pm 4.9$ & $  5.0 \pm 1.8$& $ 6137 \pm  21$ & $-29.61 \pm   0.02$ & $-29.18 \pm   0.02$& $-28.89 \pm   0.02$& $-28.66 \pm   0.02$& $-28.08 \pm   0.02$ \\
%SNF20080531-000 & $133.0 \pm 1.5$ & $ 17.6 \pm 0.8$& $ 6114 \pm   5$ & $-29.12 \pm   0.01$ & $-28.98 \pm   0.01$& $-28.54 \pm   0.01$& $-28.28 \pm   0.01$& $-27.51 \pm   0.01$ \\
%SN2006do & $106.4 \pm 2.1$ & $ 26.7 \pm 1.3$& $ 6101 \pm   2$ & $-29.00 \pm   0.01$ & $-28.83 \pm   0.01$& $-28.42 \pm   0.01$& $-28.20 \pm   0.01$& $-27.53 \pm   0.04$ \\
%PTF12ikt & $110.3 \pm 1.6$ & $ 14.2 \pm 0.7$& $ 6141 \pm   4$ & $-29.34 \pm   0.01$ & $-29.04 \pm   0.01$& $-28.57 \pm   0.01$& $-28.32 \pm   0.01$& $-27.66 \pm   0.01$ \\
%SN2006dm & $ 99.5 \pm 1.6$ & $ 30.0 \pm 0.7$& $ 6118 \pm   3$ & $-28.81 \pm   0.01$ & $-28.65 \pm   0.01$& $-28.23 \pm   0.01$& $-28.02 \pm   0.01$& $-27.33 \pm   0.01$ \\
%PTF13azs & $138.0 \pm 5.1$ & $ 16.2 \pm 1.6$& $ 6125 \pm  10$ & $-27.84 \pm   0.02$ & $-27.92 \pm   0.02$& $-27.69 \pm   0.02$& $-27.60 \pm   0.02$& $-26.99 \pm   0.02$ \\
%SN2005hj & $ 80.8 \pm 2.4$ & $  4.3 \pm 0.8$& $ 6138 \pm   4$ & $-29.54 \pm   0.02$ & $-29.16 \pm   0.01$& $-28.87 \pm   0.01$& $-28.54 \pm   0.01$& $-28.01 \pm   0.01$ \\
%PTF12iiq & $150.4 \pm 2.2$ & $ 22.5 \pm 0.8$& $ 6041 \pm   6$ & $-28.60 \pm   0.01$ & $-28.77 \pm   0.01$& $-28.41 \pm   0.01$& $-28.10 \pm   0.01$& $-27.29 \pm   0.01$ \\
%PTF10ndc & $124.2 \pm 2.4$ & $  6.8 \pm 1.1$& $ 6119 \pm   3$ & $-29.52 \pm   0.01$ & $-29.25 \pm   0.01$& $-28.80 \pm   0.01$& $-28.49 \pm   0.01$& $-27.76 \pm   0.01$ \\
%SNF20080919-002 & $103.6 \pm 7.2$ & $ 27.2 \pm 1.9$& $ 6133 \pm   8$ & $-28.74 \pm   0.02$ & $-28.46 \pm   0.01$& $-28.09 \pm   0.01$& $-27.87 \pm   0.01$& $-27.26 \pm   0.04$ \\
%SNPGC027923 & $ 85.5 \pm 0.6$ & $  5.9 \pm 0.3$& $ 6130 \pm   4$ & $-29.87 \pm   0.02$ & $-29.45 \pm   0.02$& $-28.94 \pm   0.02$& $-28.57 \pm   0.02$& $-27.85 \pm   0.02$ \\
%SNF20070330-024 & $118.1 \pm 2.1$ & $  4.6 \pm 2.2$& $ 6101 \pm   3$ & $-29.77 \pm   0.02$ & $-29.52 \pm   0.02$& $-29.08 \pm   0.02$& $-28.74 \pm   0.01$& $-27.94 \pm   0.02$ \\
%SNF20061030-010 & $131.4 \pm 2.2$ & $ 17.4 \pm 1.1$& $ 6116 \pm   4$ & $-28.60 \pm   0.02$ & $-28.55 \pm   0.02$& $-28.25 \pm   0.02$& $-28.03 \pm   0.02$& $-27.34 \pm   0.02$ \\
%SNhunt46 & $ 94.1 \pm 2.0$ & $ 11.2 \pm 0.6$& $ 6132 \pm   4$ & $-29.50 \pm   0.02$ & $-29.11 \pm   0.02$& $-28.67 \pm   0.02$& $-28.37 \pm   0.02$& $-27.71 \pm   0.02$ \\
%SN2005hc & $126.9 \pm 2.5$ & $ 10.0 \pm 0.7$& $ 6123 \pm   3$ & $-29.38 \pm   0.01$ & $-29.13 \pm   0.01$& $-28.69 \pm   0.01$& $-28.37 \pm   0.01$& $-27.61 \pm   0.01$ \\
%LSQ12dbr & $106.9 \pm 0.6$ & $  7.1 \pm 0.7$& $ 6138 \pm   4$ & $-29.29 \pm   0.73$ & $-29.00 \pm   0.73$& $-28.51 \pm   0.73$& $-28.15 \pm   0.73$& $-27.38 \pm   0.73$ \\
%LSQ12hjm & $ 82.6 \pm 17.5$ & $ 12.2 \pm 1.4$& $ 6144 \pm   5$ & $-29.51 \pm   0.02$ & $-29.14 \pm   0.01$& $-28.60 \pm   0.01$& $-28.30 \pm   0.01$& $-27.71 \pm   0.02$ \\
%SNF20060521-001 & $ 78.9 \pm 20.2$ & $ 21.1 \pm 1.4$& $ 6123 \pm  10$ & $-29.37 \pm   0.05$ & $-29.04 \pm   0.05$& $-28.54 \pm   0.05$& $-28.30 \pm   0.05$& $-27.57 \pm   0.05$ \\
%SNF20070630-006 & $125.5 \pm 3.2$ & $ 10.1 \pm 1.6$& $ 6126 \pm   4$ & $-29.34 \pm   0.01$ & $-29.12 \pm   0.01$& $-28.65 \pm   0.01$& $-28.38 \pm   0.01$& $-27.66 \pm   0.01$ \\
%PTF11drz & $132.6 \pm 1.4$ & $ 15.2 \pm 1.0$& $ 6116 \pm   5$ & $-29.12 \pm   0.01$ & $-28.95 \pm   0.01$& $-28.53 \pm   0.01$& $-28.27 \pm   0.01$& $-27.55 \pm   0.01$ \\
%SNF20080323-009 & $ 95.9 \pm 2.3$ & $ 10.6 \pm 1.1$& $ 6143 \pm   6$ & $-29.59 \pm   0.02$ & $-29.22 \pm   0.02$& $-28.68 \pm   0.02$& $-28.42 \pm   0.02$& $-27.77 \pm   0.02$ \\
%SNF20071021-000 & $167.5 \pm 2.2$ & $ 20.4 \pm 0.6$& $ 6112 \pm   4$ & $-28.75 \pm   0.02$ & $-28.78 \pm   0.02$& $-28.40 \pm   0.02$& $-28.18 \pm   0.02$& $-27.41 \pm   0.02$ \\
%SNNGC0927 & $155.2 \pm 1.4$ & $ 11.0 \pm 0.7$& $ 6109 \pm   4$ & $-28.87 \pm   0.02$ & $-28.81 \pm   0.01$& $-28.46 \pm   0.01$& $-28.22 \pm   0.01$& $-27.48 \pm   0.01$ \\
%SNF20060526-003 & $112.1 \pm 2.5$ & $  9.8 \pm 1.0$& $ 6121 \pm   3$ & $-29.34 \pm   0.01$ & $-29.09 \pm   0.01$& $-28.68 \pm   0.01$& $-28.39 \pm   0.01$& $-27.70 \pm   0.01$ \\
%SNF20080806-002 & $135.8 \pm 1.8$ & $  7.5 \pm 0.9$& $ 6135 \pm   4$ & $-29.22 \pm   0.02$ & $-29.02 \pm   0.02$& $-28.61 \pm   0.02$& $-28.35 \pm   0.01$& $-27.71 \pm   0.02$ \\
%SNF20080803-000 & $117.6 \pm 2.6$ & $  8.9 \pm 2.0$& $ 6125 \pm   4$ & $-28.84 \pm   0.01$ & $-28.70 \pm   0.01$& $-28.35 \pm   0.01$& $-28.16 \pm   0.01$& $-27.50 \pm   0.01$ \\
%SNF20080822-005 & $ 78.5 \pm 1.8$ & $  6.3 \pm 0.9$& $ 6138 \pm   4$ & $-29.71 \pm   0.01$ & $-29.34 \pm   0.01$& $-28.93 \pm   0.01$& $-28.61 \pm   0.01$& $-27.92 \pm   0.01$ \\
%SNF20060618-014 & $137.2 \pm 2.5$ & $  9.3 \pm 1.1$& $ 6112 \pm   7$ & $-29.27 \pm   0.03$ & $-29.09 \pm   0.03$& $-28.73 \pm   0.03$& $-28.38 \pm   0.03$& $-27.68 \pm   0.03$ \\
%PTF12ghy & $ 99.3 \pm 3.6$ & $ 16.8 \pm 0.7$& $ 6134 \pm   3$ & $-28.29 \pm   0.02$ & $-28.27 \pm   0.01$& $-28.05 \pm   0.01$& $-27.95 \pm   0.01$& $-27.40 \pm   0.01$ \\
%SNF20070531-011 & $122.4 \pm 2.7$ & $ 21.2 \pm 0.8$& $ 6114 \pm   4$ & $-29.07 \pm   0.01$ & $-28.94 \pm   0.01$& $-28.50 \pm   0.01$& $-28.26 \pm   0.01$& $-27.53 \pm   0.03$ \\
%SNF20070831-015 & $112.2 \pm 2.7$ & $  7.8 \pm 1.0$& $ 6145 \pm   6$ & $-29.42 \pm   0.01$ & $-29.17 \pm   0.01$& $-28.78 \pm   0.01$& $-28.46 \pm   0.01$& $-27.78 \pm   0.01$ \\
%SNF20070417-002 & $104.5 \pm 5.5$ & $ 24.4 \pm 2.2$& $ 6123 \pm   9$ & $-29.20 \pm   0.05$ & $-29.01 \pm   0.05$& $-28.48 \pm   0.05$& $-28.23 \pm   0.05$& $-27.54 \pm   0.05$ \\
%PTF11cao & $143.3 \pm 1.6$ & $ 18.9 \pm 1.3$& $ 6104 \pm   5$ & $-28.78 \pm   0.02$ & $-28.79 \pm   0.02$& $-28.44 \pm   0.02$& $-28.18 \pm   0.02$& $-27.45 \pm   0.02$ \\
%SNF20080522-000 & $ 61.8 \pm 3.5$ & $  3.3 \pm 0.9$& $ 6131 \pm   7$ & $-29.86 \pm   0.01$ & $-29.41 \pm   0.01$& $-29.03 \pm   0.01$& $-28.70 \pm   0.01$& $-28.06 \pm   0.01$ \\
%PTF10qjq & $ 73.9 \pm 2.4$ & $ 12.8 \pm 0.8$& $ 6133 \pm   3$ & $-29.29 \pm   0.02$ & $-28.94 \pm   0.02$& $-28.53 \pm   0.01$& $-28.35 \pm   0.01$& $-27.76 \pm   0.01$ \\
%PTF12dxm & $ 95.4 \pm 41.8$ & $ 35.7 \pm 2.8$& $ 6136 \pm   4$ & $-28.71 \pm   0.01$ & $-28.58 \pm   0.01$& $-28.19 \pm   0.01$& $-27.99 \pm   0.01$& $-27.34 \pm   0.01$ \\
%SNF20061021-003 & $122.8 \pm 2.3$ & $  9.7 \pm 1.7$& $ 6131 \pm   4$ & $-29.04 \pm   0.02$ & $-28.86 \pm   0.02$& $-28.56 \pm   0.02$& $-28.30 \pm   0.02$& $-27.64 \pm   0.02$ \\
%SNF20080510-005 & $111.6 \pm 2.6$ & $  6.4 \pm 1.1$& $ 6115 \pm   4$ & $-29.41 \pm   0.01$ & $-29.15 \pm   0.01$& $-28.70 \pm   0.01$& $-28.38 \pm   0.01$& $-27.73 \pm   0.04$ \\
%SNF20080507-000 & $ 98.1 \pm 1.6$ & $ 10.6 \pm 2.1$& $ 6143 \pm   5$ & $-29.23 \pm   0.01$ & $-29.05 \pm   0.01$& $-28.71 \pm   0.01$& $-28.45 \pm   0.01$& $-27.79 \pm   0.01$ \\
%SNF20080913-031 & $118.2 \pm 1.5$ & $ 11.3 \pm 1.8$& $ 6158 \pm   5$ & $-29.13 \pm   0.08$ & $-29.01 \pm   0.07$& $-28.62 \pm   0.07$& $-28.32 \pm   0.07$& $-27.68 \pm   0.07$ \\
%SNF20080510-001 & $118.8 \pm 2.1$ & $ 15.3 \pm 1.3$& $ 6115 \pm   4$ & $-29.35 \pm   0.01$ & $-29.15 \pm   0.01$& $-28.69 \pm   0.01$& $-28.38 \pm   0.01$& $-27.68 \pm   0.01$ \\
%SNF20070712-003 & $108.8 \pm 2.7$ & $ 13.5 \pm 0.9$& $ 6155 \pm   6$ & $-29.44 \pm   0.02$ & $-29.19 \pm   0.01$& $-28.74 \pm   0.01$& $-28.42 \pm   0.01$& $-27.78 \pm   0.01$ \\
%\enddata
%\end{deluxetable}
\section{Supernova Model I: Two Extrinsic Parameters}
\label{model:sec}

In what is  referred to as Model~I, we hypothesize that at peak brightness
SN~Ia broadband magnitudes and colors are correlated with
spectral features: equivalent widths and line positions are considered in the model because their localization in wavelength
makes their values insensitive to 
broadband color variations.
These intrinsic spectral  parameters may not deterministically predict magnitudes, but rather do so with some intrinsic dispersion.
The per-supernova intrinsic magnitudes are then
modified by an extrinsic physical process (i.e.\ dust) to produce apparent magnitudes that are subsequently measured by the observer.

\subsection{Model}
We assume 
that  peak intrinsic $UBVRI$ magnitudes of a supernova are linearly dependent
on its
 equivalent widths of the CaII H\&K and SiII$\lambda$4141 spectral features
$EW_{Ca}$ and $EW_{Si}$,
and the wavelength of the minimum of 
the SiII$\lambda6355$ feature $\lambda_{Si}$:
these spectral features are associated with SN~Ia  spectroscopic diversity  
\citep{2006PASP..118..560B, 2008A&A...492..535A, 2009A&A...500L..17B, 2009PASP..121..238B, 2009ApJ...699L.139W, 2011ApJ...729...55F}.
The explicit omission of light-curve shape in our model is compensated by its proxy,
$EW_{Si}$, at peak brightness
\citep{2008A&A...492..535A, 2011A&A...529L...4C}. On top of these feature-dependent magnitudes, intrinsic dispersion
gives each supernova a magnitude contribution drawn from a Normal distribution with a parameterized covariance matrix
$C_c$.  A grey magnitude offset, $\Delta$, is included for each supernova
to capture band-independent intrinsic dispersion, while also absorbing peculiar-velocity errors introduced when converting
fluxes to luminosites.
The apparent magnitudes are the sum of the intrinsic magnitudes plus additional terms that are linearly dependent on the
per-supernova
extrinsic-color parameters $k_0$ and $k_1$.  
Unlike the parameters associated with
the spectral measurements, $EW_{Ca}$, $EW_{Si}$ and $\lambda_{Si}$,  the latent
parameters $k_0$ and $k_1$ are not directly associated
with observables but rather are inferred as part of the analysis.
The observables
$U_o, B_o, V_o, R_o, I_o$, $EW_{Ca,o}$, $EW_{Si,o}$, $\lambda_{Si,o}$
%shown in Table~\ref{data:tab}
have Gaussian measurement uncertainty with covariance $C$.
The
likelihood density for the described model
is
\begin{equation}
\begin{pmatrix}
U\\B\\V\\R\\I
\end{pmatrix}
\sim \mathcal{N}
\left(
\Delta +
\begin{pmatrix}
c_U+\alpha_U EW_{Ca} + \beta_U EW_{Si} + \eta_U \lambda_{Si} \\
c_B+\alpha_B EW_{Ca} + \beta_B EW_{Si} + \eta_B \lambda_{Si}  \\
c_V+\alpha_V EW_{Ca} + \beta_V EW_{Si} + \eta_V \lambda_{Si} \\
c_R+\alpha_R EW_{Ca} + \beta_R EW_{Si} + \eta_R \lambda_{Si} \\
c_I+\alpha_I EW_{Ca} + \beta_I EW_{Si}+ \eta_I \lambda_{Si}
\end{pmatrix}
,C_{c}
\right)
\label{ewsiv:eqn}
\end{equation}
\begin{equation}
\begin{pmatrix}
U_o\\B_o\\ V_o\\R_o\\I_o\\EW_{Si, o}\\ EW_{Ca, o} \\ \lambda_{Si, o}
\end{pmatrix}
\sim \mathcal{N}
\left(
\begin{pmatrix}
U +\gamma^0_{U} k_0 +\gamma^1_{U} k_1 \\B +\gamma^0_{B} k_0 +\gamma^1_{B} k_1 \\
V+\gamma^0_{V} k_0+\gamma^1_{V} k_1\\R+\gamma^0_{R} k_0 + \gamma^1_{R} k_1\\I+\gamma^0_{I} k_0+\gamma^1_{I} k_1\\
EW_{Si}\\ EW_{Ca} \\ \lambda_{Si}
\end{pmatrix}
,C
\right).
\label{dust:eqn}
\end{equation}
The global parameter vectors that describe the SN~Ia population
$\mathbf{c}$ (where $\mathbf{c}=(c_U, c_B, c_V, c_R, c_I) $),
$\pmb{\alpha}$, $\pmb{\beta}$,  and $\pmb{\eta}$,  are the intercept and slopes of the linear relationships that
relate spectral-feature parameters with intrinsic magnitudes.
The global parameter vectors $\pmb{\gamma}^0$, $\pmb{\gamma}^1$  are the slopes that relate 
the latent extrinsic-color parameters with apparent magnitudes.  

\color{purple}
The model as written has degeneracies that need to be constrained in order for fits to converge.
The model does not specify the absolute magnitude nor the intrinsic color of a supernova that has no extrinsic color contribution,
in their place the zeropoints of the per-supernova magnitude and color parameters are set to the sample means
\begin{equation}
\langle \Delta \rangle=0, \langle k_0 \rangle=0, \langle k_1 \rangle=0.
\end{equation}
The model contains the product of latent parameters
$\pmb{\gamma} k$, which leads to the degeneracy $\pmb{\gamma} \rightarrow a\pmb{\gamma}$, $k \rightarrow a^{-1} k$.
To aid in the convergence of $\pmb{\gamma}$ and $k$ we impose a prior on the rms of $k$, though
our physical interpretations are ultimately independent of this scaling.
This prior does not specify the sign of $a$, which leaves a parity degeneracy.  As will be seen, the signal-to-noise in $\pmb{\gamma}$ is sufficiently
high that our finite MCMC chains do not migrate between the degenerate solutions; to simplify the merging of multiple chains
we impose one of the degenerate solutions
\begin{equation}
\gamma^0_U > 0, \gamma^1_U < 0.
\end{equation}
%to exclude degenerate posterior space
%associated
%with the $\pm$ sign.
As  in mixture models, the combinations $\pmb{\gamma}^0 k_0$ and $\pmb{\gamma}^1 k_1$ terms are degenerate under exchange of the 0 and 1 indices. For a cleaner presentation of our results, 
we break that degeneracy  by setting consistent initial conditions that succeed in keeping indices consistent over all analysis chains.  
\color{black}


Another form of degeneracy is illustrated by the  transformations $\Delta \rightarrow \Delta  + \epsilon_\Delta$,
 $\pmb{\gamma}^0 \rightarrow \pmb{\gamma}^0  + \epsilon_{\gamma^0}$, $\pmb{\gamma}^1 \rightarrow \pmb{\gamma}^1 + \epsilon_{\gamma^1}$
 under the condition
$$
\epsilon_\Delta  +  \epsilon_{\gamma^0} k_0+  \epsilon_{\gamma^1} k_1=0,
$$
i.e.\ shifts in the global coefficients $\pmb{\gamma}$ can be compensated by shifts in $\pmb{\Delta}$ for each supernova while maintaining 
$\langle \Delta \rangle=0$.
Another degeneracy is illustrated by $k_0 \rightarrow k_0 + (EW_{Ca}-\langle EW_{Ca}\rangle)\epsilon$,
$\pmb{\alpha} \rightarrow \pmb{\alpha} - \pmb{\gamma} \epsilon$, $\mathbf{c} \rightarrow \mathbf{c} + \langle EW_{Ca}\rangle \epsilon$, i.e. the
linear external terms can leak into the linear spectral feature corrections.
These degeneracies are not broken within the model, but rather are constrained by the priors in which the
per-supernova parameters
$\Delta$, $EW$, $\lambda$, $k$ are uncorrelated. 


The parameterized intrinsic dispersion, $C_c$, seemingly introduces degeneracy in the model, as magnitude and color variation
ascribed to $\Delta$, $\pmb{\gamma}^0 k_0$, and $\pmb{\gamma}^1 k_1$ could also be attributed to intrinsic dispersion.  There are several features of the model
that drive the assignation of variations away from $C_c$:  Maximizing the posterior disfavors the increase of $\det{(C_c)}$;
The distributions of $\Delta$, $k_0$, and $k_1$ turn out to
be non-Gaussian, and so are not well described by the Normal covariance we impose in $C_c$; the grey magnitude offsets, $\Delta$, would appear as a constant
in all elements of the covariance matrix, which is disfavored for the Bayesian prior selected for $C_c$.

In our Bayesian analysis the priors must be described.  A flat prior is used for all parameters except
for the covariance matrix $C_c$, which is constructed from a correlation matrix with  $\nu=4$  LKJ prior\footnote{
Visualization of the LKJ correlation distribution can be found in \url{http://www.psychstatistics.com/2014/12/27/d-lkj-priors/}.}
\citep{Lewandowski20091989} and standard
deviations $\sigma_i = \sqrt{C_{c,ii}}$ with a  Cauchy distribution prior with location
 $0.1$ and scale $0.1$ mag restricted to positive values.
This covariance matrix prior is recommended by STAN \citep{stan}, the Monte Carlo program we use to evaluate the model.
 \color{purple}
 We find that a simpler intrinsic dispersion model with no correlations between bands
 (i.e.\ a  diagonal covariance matrix) produces little change in the posteriors of
 the other parameters.
\color{black}

For $N$ supernovae there are $8N$ observables.  For the top-level model parameters, there are $3N$ spectral parameters, $2(N-2)$
color parameters, $(N-1)$ magnitude offsets,  $5 \times 6$ global coefficients, and $10$ parameters that describe the intrinsic
$C_c$.  For $N=172$ supernovae, there are 1376 observables and 1067 top-level independent parameters.

 
\subsection{Results}
\label{results:sec}
The posterior of the model parameters is evaluated using Hamiltonian Monte Carlo with a No-U-Turn
Sampler as implemented in
STAN \citep{stan}.  We run eight chains, each with 5000 iterations of which
half are used for warmup.
Contours of the posterior surface for parameter pairs grouped by filter are shown in Figures~\ref{global1:fig} -- \ref{global5:fig} .


\begin{figure}[htbp] %  figure placement: here, top, bottom, or page
   \centering
   \includegraphics[width=5.2in]{output11/coeff0.pdf} 
            \caption{Model~I posterior contours for $\mathbf{c}$, $\pmb{\alpha}$, $\pmb{\beta}$, $\pmb{\eta}$, $\pmb{\gamma}^0$, $\pmb{\gamma}^1$, and $\sigma$ in the $U$ band.
            The contours shown here and in future plots represent 1-$\sigma$ in the parameter distribution (i.e.\ they should be
            projected onto the corresponding 1-d parameter axis), not to 68\%, 95\%, etc.\
            enclosed probability.  Lines for zero value for $\alpha_U$, $\beta_U$, $\eta$, $\gamma_U^0$, $\gamma_U^1$, and $\sigma_U$ are shown for reference --
            in most cases they are outside the range of the plot.
            \label{global1:fig}}
\end{figure}

\begin{figure}[htbp] %  figure placement: here, top, bottom, or page
   \centering
   \includegraphics[width=5.2in]{output11/coeff1.pdf} 
            \caption{Model~I posterior contours for $\mathbf{c}$, $\pmb{\alpha}$, $\pmb{\beta}$, $\pmb{\eta}$,  $\pmb{\gamma}^0$, $\pmb{\gamma}^1$, and $\sigma$ in the $B$ band.
 \label{global2:fig}}
\end{figure}

\begin{figure}[htbp] %  figure placement: here, top, bottom, or page
   \centering
   \includegraphics[width=5.2in]{output11/coeff2.pdf} 
            \caption{Model~I posterior contours for $\mathbf{c}$, $\pmb{\alpha}$, $\pmb{\beta}$, $\pmb{\eta}$, $\pmb{\gamma}^0$, $\pmb{\gamma}^1$, and $\sigma$ in the $V$ band.
 \label{global3:fig}}
\end{figure}

\begin{figure}[htbp] %  figure placement: here, top, bottom, or page
   \centering
      \includegraphics[width=5.2in]{output11/coeff3.pdf} 
            \caption{Model~I posterior contours for  $\mathbf{c}$, $\pmb{\alpha}$, $\pmb{\beta}$, $\pmb{\eta}$,  $\pmb{\gamma}^0$, $\pmb{\gamma}^1$, and $\sigma$ in the $R$ band.
 \label{global4:fig}}
\end{figure}

\begin{figure}[htbp] %  figure placement: here, top, bottom, or page
   \centering
         \includegraphics[width=5.2in]{output11/coeff4.pdf} 
            \caption{Model~I posterior contours for  $\mathbf{c}$, $\pmb{\alpha}$, $\pmb{\beta}$, $\pmb{\eta}$, $\pmb{\gamma}^0$, $\pmb{\gamma}^1$, and $\sigma$ in the $I$ band.
 \label{global5:fig}}
\end{figure}

Tests have revealed no evidence for the non-convergence of the MCMC run.
STAN provides output statistics to assess
the convergence of the output Markov chains.
The 
potential scale reduction statistic, $\hat{R}$ \citep{Gelman92}, measures the convergence of the target distribution
in iterative simulations 
by using multiple independent sequences to estimate how much that distribution would sharpen if the simulations were run longer.
$N_{eff}$ is an estimate of the number of independent draws. The output for our data and model gives $\hat{R} \sim 1.0$ for all parameters, meaning there is no evidence for non-convergence.  The
output also gives  $N_{eff} \gg 100$ for all parameters, indicating that are all densely sampled.
Empirically, the confidence regions are localized and unimodal as is seen in  Figures~\ref{global1:fig} -- \ref{global5:fig}.  In these tests there is no evidence that
the Monte Carlo chains have not converged to the stationary posterior distribution.
\color{purple}
For four selected colors, a plot of the
observed versus deduced intrinsic colors and the histogram of their differences
\color{black}
are shown in Figure~\ref{residual:fig}. 
There is no evidence of a catastrophically poor model predictions nor of extreme
outlying data.
We have rurun the analysis with a variety of initial conditions, including one with $\pmb{\gamma}$ equal to zero, except for a small positive 
$\gamma^0_0$ and small negative $\gamma^1_0$; the 68\% credible intervals
remain consistent.

\begin{figure}[htbp] %  figure placement: here, top, bottom, or page
   \centering
   \includegraphics[width=5.2in]{output11/residual.pdf} 
            \caption{
            \color{purple}
            Observed versus deduced intrinsic colors for four selected colors.  Plotted on the right for each color
is the histogram of the difference between the observed and intrinsic colors.
\color{black}
            \label{residual:fig}}
\end{figure}


For each of the five filters, the 68\%  equal-tailed credible intervals for the global parameters $\alpha$, $\beta$, $\eta$, $\pmb{\gamma}^0$, $\pmb{\gamma}^1$, and $\sigma$
are given in Table~\ref{global:tab}.
As the $\pmb{\gamma}$ parameters are free up to a multiplicative constant and have non-zero values of $\gamma^i_V$,
their results are shown in terms of $\gamma^i_X/\gamma^i_V-1$.


\begin{table}
\centering
\begin{tabular}{|c|c|c|c|c|c|}
\hline
Parameter & $X=U$ &$B$&$V$&$R$&$I$\\ \hline
$\alpha_X$
&
$0.0043^{+0.0009}_{-0.0009}$
&
$0.0016^{+0.0008}_{-0.0007}$
&
$0.0015^{+0.0006}_{-0.0006}$
&
$0.0015^{+0.0005}_{-0.0005}$
&
$0.0027^{+0.0005}_{-0.0005}$
\\
${\alpha_X/\alpha_V-1}$
&
$   1.9^{+   1.1}_{  -0.5}$
&
$   0.1^{+   0.1}_{  -0.2}$
&
$\dots$
&
$   0.0^{+   0.1}_{  -0.1}$
&
$   0.8^{+   0.8}_{  -0.3}$
\\
$\beta_X$
&
$ 0.032^{+ 0.003}_{-0.003}$
&
$ 0.026^{+ 0.002}_{-0.002}$
&
$ 0.026^{+ 0.002}_{-0.002}$
&
$ 0.021^{+ 0.002}_{-0.002}$
&
$ 0.020^{+ 0.002}_{-0.002}$
\\
${\beta_X/\beta_V-1}$
&
$  0.25^{+  0.05}_{ -0.05}$
&
$ -0.01^{+  0.03}_{ -0.03}$
&
$\dots$
&
$ -0.19^{+  0.01}_{ -0.01}$
&
$ -0.23^{+  0.03}_{ -0.03}$
\\
$\eta_X$
&
$-0.0002^{+0.0012}_{-0.0011}$
&
$0.0000^{+0.0010}_{-0.0009}$
&
$0.0005^{+0.0008}_{-0.0008}$
&
$0.0006^{+0.0007}_{-0.0007}$
&
$-0.0002^{+0.0006}_{-0.0006}$
\\
${\eta_X/\eta_V-1}$
&
$ -0.39^{+  2.22}_{ -1.82}$
&
$ -0.35^{+  1.58}_{ -1.20}$
&
$\dots$
&
$ -0.09^{+  0.29}_{ -0.28}$
&
$ -0.78^{+  1.48}_{ -1.11}$
\\
$\gamma^0_X$
&
$ 63.46^{+  3.70}_{ -3.66}$
&
$ 51.40^{+  3.09}_{ -3.07}$
&
$ 38.15^{+  2.70}_{ -2.63}$
&
$ 29.26^{+  2.40}_{ -2.35}$
&
$ 20.94^{+  2.24}_{ -2.25}$
\\
${\gamma^0_X/\gamma^0_V-1}$
&
$  0.66^{+  0.06}_{ -0.05}$
&
$  0.35^{+  0.03}_{ -0.03}$
&
$\dots$
&
$ -0.23^{+  0.01}_{ -0.02}$
&
$ -0.45^{+  0.03}_{ -0.03}$
\\
$\gamma^1_X$
&
$-10.34^{+  3.61}_{ -4.29}$
&
$-11.89^{+  2.95}_{ -3.56}$
&
$-14.48^{+  2.47}_{ -2.89}$
&
$-13.71^{+  2.14}_{ -2.45}$
&
$-12.51^{+  2.10}_{ -2.23}$
\\
${\gamma^1_X/\gamma^1_V-1}$
&
$ -0.28^{+  0.17}_{ -0.19}$
&
$ -0.18^{+  0.10}_{ -0.10}$
&
$\dots$
&
$ -0.05^{+  0.05}_{ -0.04}$
&
$ -0.14^{+  0.10}_{ -0.09}$
\\
$\sigma_X$
&
$ 0.060^{+ 0.012}_{-0.012}$
&
$ 0.033^{+ 0.007}_{-0.007}$
&
$ 0.021^{+ 0.004}_{-0.005}$
&
$ 0.011^{+ 0.008}_{-0.007}$
&
$ 0.044^{+ 0.005}_{-0.004}$
\\
\hline
\end{tabular}
\caption{68\% credible intervals for the Global Fit Parameters of the 2-Parameter Extrinsic Model~I in \S\ref{model:sec}.\label{global:tab}}
\end{table}

We find significant non-zero values for $\pmb{\alpha}$ and $\pmb{\beta}$, indicating that $EW_{Ca}$ and $EW_{Si}$ are indicators of broadband
peak magnitudes.
This validates our hypothesis that spectral indicators
are tracers of supernova absolute magnitude.  On the other hand, the values of $\eta$ (the coefficients attached to $\lambda_{Si}$) are consistent with zero
to  within one standard deviation.
The effect of spectral parameters on color (as opposed to magnitude)
is shown in the rows of $\alpha_X/\alpha_V-1$,  $\beta_X/\beta_V-1$, and  $\eta_X/\eta_V-1$
in Table~\ref{global:tab}.
Values of zero signify no color changes associated with magnitude changes.
Both $EW_{Ca}$ and $EW_{Si}$ are associated with color changes, though not in $B-V$ specifically.
We do not detect a significant association between
$\lambda_{Si}$ and color.

The signal in  $\pmb{\alpha}$ and $\pmb{\beta}$ cannot be attributed to the equivalent widths themselves.
The range of SiII$\lambda$4130 equivalent widths is $\pm 20$~\AA\ whereas the width of the $B$-band is 851~\AA, so that its direct affect on magnitude
is
$2.5 \log{(20/850)} \sim 0.03$ mag.  
The implied span in $B$ magnitude based on $\beta_B$ is 0.54~mag.  Therefore $\beta_B$ cannot wholly be attributed to the flux deficit
from the line itself.
Similarly, the CaII H\&K equivalent widths have range $\pm 50$~\AA, while the width of the $U$ band is
701~\AA, so that its direct affect on magnitude
is
$2.5 \log{(50/701)} \sim 0.08$ mags.   The implied span in $U$ magnitude is  0.21~mag, so $\alpha_U$ cannot be completely due to the 
presence of the line itself.

\color{purple}
The fit $\Delta$'s are highly correlated between supernovae.  To aid in visualization, we remove much of this
correlation by subtracting out the $\Delta|_0$ of  an arbitrary supernova.
The histogram of the medians of these relative  grey offsets $\Delta-\Delta|_0$,
is shown in Figure~\ref{hist:fig}.  The distribution
has a standard deviation of
%-----
$0.08$
%-----
mag, and a tail in the positive (fainter) direction. 
Of the three extreme outliers, the supernovae with largest and third largest $\Delta$'s have  heliocentric redshifts of 
 $0.0015$ and $0.0085$ respectively, and so have peculiar velocities that can cause a significant magnitude offset.
\color{black}
\begin{figure}[htbp] %  figure placement: here, top, bottom, or page
   \centering
   \includegraphics[width=5.2in]{output11/deltaDelta_hist.pdf} 
   \caption{
   \color{purple}
   Normalized ideogram and histogram of the medians of the grey offset relative to one arbitrary supernova, $\Delta-\Delta|_0$.  
   \color{black}
   \label{hist:fig}}
\end{figure}

Non-trivial residual magnitude dispersions are captured in $C_c$.  The square
root of the diagonal elements are denoted as $\sigma$, which
range from
$\sim 0.01$ to 0.06 mag, significantly smaller
than the dispersion in $\Delta$.
\color{purple}
Of particular interest within $C_c$ is intrinsic color dispersion.  The intrinsic dispersions in $V$ and colors with respect to $V$ are
\begin{multline}
\sigma(V,U-V,B-V,V-R,V-I)=\\
%\begin{pmatrix}
%\begin{array}{rrrrr}
\left(
0.021^{+0.004}_{-0.005} ,
0.067^{+0.010}_{-0.010},
0.034^{+0.005}_{-0.005},
0.023^{+0.002}_{-0.002},
0.053^{+0.004}_{-0.004}
\right)
%\end{array}
% \end{pmatrix} 
 \text{mag}.
 \label{sig_intrinsic:eqn}
 \end{multline}
The correlations between these parameters are
\begin{multline}
Cor(V,U-V,B-V,V-R,V-I)=\\
\begin{pmatrix}
\begin{array}{rrrrr}
1 & -0.48^{+0.20}_{-0.18} & -0.34^{+0.21}_{-0.22} & 0.88^{+0.10}_{-0.25} & 0.60^{+0.11}_{-0.19} \\
-0.48^{+0.20}_{-0.18} & 1 & 0.60^{+0.10}_{-0.13} & -0.38^{+0.18}_{-0.17} & -0.33^{+0.15}_{-0.14} \\
-0.34^{+0.21}_{-0.22} & 0.60^{+0.10}_{-0.13} & 1 & -0.21^{+0.17}_{-0.17} & 0.08^{+0.14}_{-0.15} \\
0.88^{+0.10}_{-0.25} & -0.38^{+0.18}_{-0.17} & -0.21^{+0.17}_{-0.17} & 1 & 0.65^{+0.06}_{-0.07} \\
0.60^{+0.11}_{-0.19} & -0.33^{+0.15}_{-0.14} & 0.08^{+0.14}_{-0.15} & 0.65^{+0.06}_{-0.07} & 1 \\
\end{array}
 \end{pmatrix}.
 \label{cor_intrinsic:eqn}
 \end{multline}
\color{black}
These color variances are smaller than those derived by \citet{2003A&A...404..901N} or \citet{2007ApJ...659..122J}, for cases where
a direct comparison can be made.
\color{purple}
%model11_rubin.py
Having no intrinsic color dispersion is disfavored, and not allowing for them can have important consequences.  Running an analysis assuming no intrinsic dispersion (replacing the probabilistic intrinsic magnitudes in Eqn.~\ref{ewsiv:eqn}
with a strict equality), we find significant shifts in all slope parameters  and smaller statistical uncertainties.
\color{black}

%Each supernova is described by its parameters $EW_{Ca}$, $EW_{Si}$, $\lambda_{Si}$, $E_{\gamma^0}(B-V)=(\gamma^0_B-\gamma^0_V)k_0$, and
%$E_{\gamma^1}(B-V)=(\gamma^1_B-\gamma^1_V)k_1$, as well as its grey offset
%$\Delta$: their distributions for all Monte Carlo links for all supernovae are shown in Figure~\ref{perobject:fig}.
%There is a core concentration in the  parameter-space, with around ten objects that occupy its outskirts.
%Many outliers appear in the red tail of $E_{\gamma^0}(B-V)$, as would be expected for the (infrequent) selection of supernovae
%heavily extinguished by host-galaxy dust.
%Outliers  are also clearly distinguishable in  $EW_{Ca}$--$\lambda_{Si}$ space.   
%
%\begin{figure}[htbp] %  figure placement: here, top, bottom, or page
%   \centering
%   \includegraphics[width=5.2in]{output11/perobject_corner.pdf} 
%   \caption{Distributions for the supernova parameters $EW_{Ca}$, $EW_{Si}$, $\lambda_{Si}$, $E_{\gamma^0}(B-V)$, and $E_{\gamma^1}(B-V)$, as well as the grey offset
%$\Delta$.  All Monte Carlo links are plotted, so that each supernova contributes a cloud of points.
%   \label{perobject:fig}}
%\end{figure}

Each supernova is described by its parameters $\Delta$, $EW_{Ca}$, $EW_{Si}$, $\lambda_{Si}$, $E_{\gamma^0}(B-V)$, and $E_{\gamma^1}(B-V)$.
Their Pearson correlation coefficients  are given in the matrix
\color{purple}
\begin{multline}
Cor(\Delta, EW_{Ca}, EW_{Si}, \lambda_{Si}, E_{\gamma^0}(B-V), E_{\gamma^1}(B-V)) =\\
\begin{pmatrix}
\begin{array}{rrrrrr}
1 & 0.00^{+0.06}_{-0.06} & -0.07^{+0.07}_{-0.06} & -0.04^{+0.06}_{-0.06} & 0.10^{+0.07}_{-0.07} & -0.00^{+0.08}_{-0.07} \\
0.00^{+0.06}_{-0.06} & 1 & 0.10^{+0.03}_{-0.03} & -0.26^{+0.03}_{-0.03} & -0.07^{+0.06}_{-0.06} & -0.01^{+0.08}_{-0.08} \\
-0.07^{+0.07}_{-0.06} & 0.10^{+0.03}_{-0.03} & 1 & -0.14^{+0.04}_{-0.03} & -0.17^{+0.05}_{-0.05} & -0.04^{+0.08}_{-0.07} \\
-0.04^{+0.06}_{-0.06} & -0.26^{+0.03}_{-0.03} & -0.14^{+0.04}_{-0.03} & 1 & 0.03^{+0.05}_{-0.06} & 0.03^{+0.08}_{-0.08} \\
0.10^{+0.07}_{-0.07} & -0.07^{+0.06}_{-0.06} & -0.17^{+0.05}_{-0.05} & 0.03^{+0.05}_{-0.06} & 1 & 0.06^{+0.08}_{-0.08} \\
-0.00^{+0.08}_{-0.07} & -0.01^{+0.08}_{-0.08} & -0.04^{+0.08}_{-0.07} & 0.03^{+0.08}_{-0.08} & 0.06^{+0.08}_{-0.08} & 1 \\
\end{array}
\end{pmatrix}.
\end{multline}
\color{black}

We find weak correlations between our color-excess parameters and the input features; 
\color{purple}
only the $ -0.17 \pm 0.05$ between $E_{\gamma^0}(B-V)$ and $EW_{Si}$ has significance much larger than $1 \sigma$.
\color{black}
The feature $EW_{Si}$ is correlated with light-curve shape, which is correlated with host-galaxy (including dust) properties 
\citep{2000AJ....120.1479H, 2003MNRAS.340.1057S}.
An extensive discussion of the correlations between the spectral features of the SNfactory data set can be found in \citet{chotard:thesis}
and \citet{leget:thesis}.

\subsection{The Two Extrinsic Per-Supernova Parameters $k_0$ and $k_1$}
All elements of the two $\pmb{\gamma}$ parameter vectors  are significantly non-zero. 
None of the 20000 links of 
our Monte Carlo chains for $\pmb{\gamma}$ extend to 0 (see Figures~\ref{global1:fig}--\ref{global5:fig}), we thus claim detection of the
influence of $k_0$ and $k_1$  on supernova magnitudes
with probability $1-5\times 10^{-5}$.

We compare our model's parameter vectors $\pmb{\gamma}^0$ and $\pmb{\gamma}^1$ with the expectations of  dust models.
The 2-parameter linear model
\begin{equation}
A_X = a(X)  A_V + b(X) E(B-V)
\label{f99:eqn}
\end{equation}
approximately describes dust models \citep[for example, the wavelength dependent model of ][is linear,
though its linearity is lost when integrated over broad-band filters]{1989ApJ...345..245C}.
For the case of
$R^F=2.5$ and $A^F_V=0.1$ dust attenuating light from the SALT2
\citep{2007A&A...466...11G} $s=1$, $x_1=0$ SN~Ia template at $B$-band peak, the \citetalias{1999PASP..111...63F} model
gives
$\mathbf{a} = (0.96,   1.00,   1.00,   0.97,   0.77)$ and $\mathbf{b}=(  1.77,   0.98,   0.12,  -0.50,  -0.53)$.
The $F$ superscript is used to distinguish parameters of the \citetalias{1999PASP..111...63F} model.
Over the ranges of
 $R^F$ and $A^F_V$,
and the wavelengths under consideration,  the values of the elements of $a$ and $b$ vary by $<5$\%;
\color{red}
for $0\le A^F_V\le 1$ and $2 \le R^F \le 3.5$, the largest error in extinction from the linear approximation of
Eqn.~\ref{f99:eqn} has amplitude $<0.008$ mag.
\color{black}

The extrinsic contribution to our model can be re-expressed to have a similar parameterization
\begin{equation}
A_X = \frac{\gamma^0_X}{\gamma^0_B-\gamma^0_V}  E_{\gamma^0}(B-V) +  \frac{\gamma^1_X}{\gamma^1_B-\gamma^1_V}  E_{\gamma^1}(B-V),
\end{equation}
with $E_{\gamma^i}(B-V) \equiv  (\gamma^i_B-\gamma^i_V)k_i$ as defined in \S\ref{results:sec}.
Our median-fit values for the $\pmb{\gamma}$ coefficients are
%---- fitz11.py
$\frac{\pmb{\gamma}^0}{\gamma^0_B-\gamma^0_V}  =(4.80 ,   3.89,   2.89,   2.22,   1.59)$ and
$ \frac{\pmb{\gamma}^1}{\gamma^1_B-\gamma^1_V}=(-3.83 ,  -4.42,  -5.42,  -5.15,  -4.71)$.
%----

Our model $\gamma$ vectors can be written in terms of the 
dust-based $a(X)$ and $b(X)$ vectors plus a residual vector perpendicular to $\mathbf{a}$ and $\mathbf{b}$.
The result is
\begin{equation}
\begin{pmatrix}
 \frac{\pmb{\gamma}^0}{\gamma^0_B-\gamma^0_V} \\
\frac{\pmb{\gamma}^1}{\gamma^1_B-\gamma^1_V} 
\end{pmatrix}=
\begin{pmatrix}
\begin{array}{rr}
2.82 & 1.15  \\
-5.27 & 0.72
\end{array}
\end{pmatrix} 
\begin{pmatrix}
\mathbf{a} \\
\mathbf{b}
\end{pmatrix}+
\begin{pmatrix}
\pmb{\epsilon}_{\gamma^0} \\
\pmb{\epsilon}_{\gamma^1}
\end{pmatrix},
\label{trans_I:eqn}
\end{equation}
where the two residual vectors
\begin{align}
\begin{split}
\pmb{\epsilon}_{\gamma^0} &=(0.06, -0.07, -0.07,  0.07, 0.02), \\
\pmb{\epsilon}_{\gamma^1} & =(-0.06, 0.16, -0.22, 0.33, -0.26)
\end{split}
\label{res_I:eqn}
\end{align}
quadratically contribute  0.03\% and 0.2\% respectively to the total  length of the $\pmb{\gamma}$ vectors.  The most significant
deviation away from the plane is in the $R$-band.
This result indicates that
the allowed variations in $UBVRI$ allowed by the \citetalias{1999PASP..111...63F} model and our best-fit model are confined to almost identical
2-dimensional
planes within
the 5-dimensional magnitude space.
Wiith no prior assumptions of dust extinction behavior or the distribution of $A_V$, the supernova data themselves exhibit
2-dimensional color variations that are closely aligned with the 2-dimensional color variations predicted by the \citetalias{1999PASP..111...63F} dust model.

\color{purple}
The above result is visualized in Figure~\ref{plane:fig}, which  shows  in the $UVI$-subspace two perspectives (left and right columns)
of the unit vectors corresponding to $\pmb{\gamma}^0$ and  $\pmb{\gamma}^1$ of our model (solid lines),
and $\mathbf{a}$, $\mathbf{b}$ of the \citetalias{1999PASP..111...63F} model (dashed lines).  All  vectors are set to intersect the origin.
The two perspectives show that while the four vectors point in different directions for each band combination,
they are almost coplanar in $UVI$ (they are slightly less coplanar in $BVR$).  The $\mathbf{a}$ and $\mathbf{b}$ vectors and the $\pmb{\gamma}^0$ and $\pmb{\gamma}^1$
vectors span near-parallel planes in the 5-dimensional color space, and hence the latter can be almost entirely expressed in terms of
the former.

\begin{figure}[htbp] %  figure placement: here, top, bottom, or page
   \centering
   \includegraphics[width=2.95in]{output11/plane0.pdf}
   \includegraphics[width=2.95in]{output11/plane1.pdf}
%   \includegraphics[width=2.95in]{output11/plane0BVR.pdf}
%   \includegraphics[width=2.95in]{output11/plane1BVR.pdf}
   \caption{
   \color{purple}
   Visualization of how supernova magnitudes can vary in our model and that of \citetalias{1999PASP..111...63F}.  While the models describe
   magnitudes in 5-bands $UBVRI$ this visualization shows only $UVI$.   The left and right plots show the same information from
   two different perspectives.
   Our model is shown in solid-line unit vectors in the directions of $\pmb{\gamma}^0$ and $\pmb{\gamma}^1$. The only possible magnitudes possible
   are a linear combination of these two vectors, and hence are confined to the plane that contains both.
   The  \citetalias{1999PASP..111...63F} model is shown in dashed-line unit vectors
   in the directions of  $\mathbf{a}$, $\mathbf{b}$; the only possible magnitudes possible are confined to the plane that contains both.
   The perspective on the right visually show that all four vectors are nearly coplanar.  
   The combination $\mathbf{a}+\mathbf{b}/2.44$ is shown in the dotted red
   line: it is almost perfectly superimposed on $\pmb{\gamma}^0$.
   \label{plane:fig}}
\end{figure}


Our linear model would be satisfied by any two independent vectors  that span
the plane defined by the $\pmb{\gamma}$ vectors.  We begin discussion on the specific choice of $\pmb{\gamma}$  returned by the fit within the context of the \citetalias{1999PASP..111...63F}  dust model. 
While
Eqn.~\ref{f99:eqn} expresses that model using basis vectors $\mathbf{a}$ and $\mathbf{b}$ and
their corresponding parameters 
$A_V$ and 
$E(B-V)$, the linear dust model could be written in terms of any two other basis vectors  $\mathbf{a}+\kappa_1 \mathbf{b}$
and $\kappa_2 \mathbf{a} + \mathbf{b}$ as parameterized by $\kappa_1$ and $\kappa_2$.
In this basis, the model is then
%To see how this prior would operate on dust models, consider that themodel in Eqn.~\ref{f99:eqn} can
%be expressed in terms of new basis vectors parametrized by $\kappa$:
\begin{equation}
A_X =  (1-\kappa_1 \kappa_2)^{-1} [(A^F_V - \kappa_2 E^F(B-V))\left(a(X)+\kappa_1 b(X) \right) +  (-\kappa_1 A^F_V + E^F(B-V)) (\kappa_2 a(X) + b(X))],
\label{newdust:eqn}
\end{equation}
with the new parameters $A^F_V - \kappa_2 E^F(B-V)$ and $-\kappa_1 A^F_V + E^F(B-V)$  equal to  linear combinations of the original parameters.

Our posterior
includes a prior in which the per-supernova parameters $k_0$ and  $k_1$ associated with $\gamma$ are uncorrelated.
The parameters $\kappa_1$ and $\kappa_2$ are set by this prior.
If the external parameters of our model were indeed described by the \citetalias{1999PASP..111...63F}  dust model, then the prior corresponds
to    $A^F_V - \kappa_2 E^F(B-V)$ and $-\kappa_1 A^F_V + E^F(B-V)$ being uncorrelated in the supernova sample.
A trivial scenario that produces non-correlation is one in which all supernovae have the same parameter value for $-\kappa_1 A^F_V + E^F(B-V)$.
All color changes would then be confined to the $a(X)+\kappa_1 b(X)$ direction. This scenario corresponds to the \citetalias{1999PASP..111...63F}  dust model restricted to a constant $R^F=\kappa_1^{-1}$.   We use this interpretation to assign  $R^F_{eff} = \kappa_1^{-1}$ as
the effective total-to-selective extinction of our sample.
The preferred values for $\kappa$ can be determined from the matrix elements of Eqn.~\ref{trans_I:eqn}: $\kappa^{-1}_1=2.82/1.15 =2.44$.
Figure~\ref{plane:fig} visualizes the near-perfect alignment of $\pmb{\gamma}^0$ and  $\mathbf{a}+\mathbf{b}/2.44$.
From the Monte Carlo chains, we find a 68\% credible interval for  $R^F_{eff}$ of 2.26--2.63.


Given the association of  our parameter $\gamma^0_V k_0$ with changes in $A^F_V$ of an $R^F =\kappa^{-1}_1$ dust model up to an additive offset,
the former's distribution can be compared to the expectation for the latter.
Figure~\ref{k0_med:fig} shows the histogram of
median values and the ideogram of $\gamma^0_V k_0$ 
relative to that of an arbitrary supernova  $\gamma^0_V k_0|_0$.
(Subtracting the value of the one supernova suppresses correlated errors in $k_0$ common to all supernovae.)
The distributions are non-Gaussian, having a sharp rise in the blue and an extended tail in the red.  This is consistent
with the expectation of dust based on the spatial distribution of supernovae within galaxies and the distribution of galaxy orientations with respect to the observer \citep{1998ApJ...502..177H}.  This distribution is given by the data, and so is not dependent of any knowledge or prior
for the distribution of dust extinction the SN population is expected to suffer. 


\begin{figure}[htbp] %  figure placement: here, top, bottom, or page
   \centering
   \includegraphics[width=2.8in]{output11/deltagamma0_med.pdf}
   \includegraphics[width=2.8in]{output11/deltagamma1_med.pdf}
      \caption{
      \color{purple} Normalized ideogram and histogram 
      of
median values of $\gamma^0_V k_0-\gamma^0_V k_0|_0$, which are associated with the $A^F_V$ of an $R^F \sim 1/\kappa_1$ dust model
(left) : $\gamma^1_V k_1-\gamma^1_V k_1|_0$,
which  is associated with extinction corrections due to deviations away from the canonical $R^F_{eff}$ value (right).
      \color{black}
   \label{k0_med:fig}}
\end{figure}

We now consider the direction of $\pmb{\gamma}^1$. 
Different supernovae are expected to encounter dust with different $R^F$;
having  associated $\pmb{\gamma}^0$ with an $R^F_{eff}$, $\pmb{\gamma}^1$ is associated with perturbations
away that effective value.
Let us introduce a new parameter for each supernova $\delta R$ such that its $R^F = R^F_{eff}(1+  \delta R)$.
Variations in the value of $R^F$ (and hence $\delta R$) with the prior of the non-correlation between $k_0$ and $k_1$ lead 
to 
\begin{equation}
\kappa_2 = R^F_{eff} \left(1- \frac{\mbox{Var}(\delta R\, E^F(B-V))}{\mbox{Cov}(\delta R\, E^F(B-V), E^F(B-V))}\right).
\end{equation}
From the matrix elements of Eqn.~\ref{trans_I:eqn},  $\kappa_2=-5.27/0.72=-7.31$, so that $\mbox{Var}(\delta R\, E^F(B-V)) = 4.00\times\mbox{Cov}(\delta R\, E^F(B-V), E^F(B-V))$.  This result indicates that there is a small but significant covariance between  extinction contributions due to
fractional deviations from an effective $R^F$ and
the color-excess parameter $E^F(B-V)$.
Figure~\ref{k0_med:fig} shows the histogram of median values and ideogram of $\gamma^1_V k_1$
relative to that of an arbitrary supernova  $\gamma^1_V k_1|_0$. 

Perhaps a more instructive way to consider $k_1$ is to not view it in isolation, but rather as part of a $\{k_0,k_1\}$ pair.
A matrix such as the one in  Eqn.~\ref{trans_I:eqn} is used to transform this pair into an $\{A_V^F, E^F(B-V)\}$ pair
(modulo additive constants)  for each link in the Monte Carlo chains. A plot showing the expected values and 68\% credible intervals of these parameters
for our supernova
sample is shown in Figure~\ref{kk:fig}. 
For reference, a line that represents $R^V_{eff}=2.44$ is overplotted.  The majority of supernovae lie within a narrow range above the line
meaning they have $R^F > R^V_{eff}$,
while the remaining fraction fall in a broader range below the line for an $R^F < R^V_{eff}$.  
These findings are consistent with previous results:
\citet{2014ApJ...789...32B, 2015MNRAS.453.3300A} deduce a wide range of dust behavior $1.5<R^F<3$ encountered by the SN~Ia population.
A per supernova determination $R^F$ requires the specification of the magnitude and colors of supernovae that experience
no dust, which is unspecified in our model.

\begin{figure}[htbp] %  figure placement: here, top, bottom, or page
   \centering
   \includegraphics[width=5.2in]{output11/avebv_synth.pdf}
      \caption{
      \color{purple}
      Expected values and 68\% credible intervals of effective $E^F(B-V)$ and $A_V^F$ after transformation from our model $k_0$ and $k_1$ parameters, for the supernova in our sample.
      Overplotted is a line with the slope expected for $R^F=2.44$.
      \color{black}
   \label{kk:fig}}
\end{figure}



As a check of our analysis implementation and our
interpretation of $R^F_{eff}$, we generate ten simulated datasets with the same numbers and uncertainties as our data, based on a
SALT2 $x_1=0$, $c=0$ template and  \citetalias{1999PASP..111...63F} dust.  The supernovae
are assigned $E^F(B-V)$ parameters equal to the $B-V$ color excesses determined in our sample
(making an approximative correspondence between our observed top-hat filter colors and the 
model parameter $E^F(B-V)$) , and $R^F$ values drawn from a log-normal
distribution with a mean of $\ln{2.5}$ and 0.05 standard deviation.  The $\alpha$, $\beta$, and $\eta$ parameters are set to zero, so that dust and
measurement uncertainty are the only sources of variation in magnitudes.  We stack the results of the 10 simulations that together yield a median
$R^F_{eff}$ of 2.53, showing that our effective value is close to the  $R^F=2.5$ input into the simulations.

To summarize, in this subsection
we conclude that the combination of
$\{k_0, k_1\}$ describe for the most part the combination of the $\{A^F_V, E^F(B-V)\}$ parameters used
in standard models of Milky Way dust.  
This conclusion is based on the
near coplanarity of the plane defined by our parameters and the plane of  \citetalias{1999PASP..111...63F}, together with the fact that 
the distribution of $k_0$ is consistent with the expected distribution of $A^F_V$.
Nevertheless, there do remain some small inconsistencies between our numbers and expectations from dust models.
\color{black}

\subsection{SN 2014J in the Context of Our Model}
\label{sn2014j:sec}
SN~2014J   is one of several SNe~Ia that exhibit colors that imply a low $R_V<2.0$ \citep{2014ApJ...788L..21A, 2014MNRAS.443.2887F, 
2014arXiv1411.3332J,
2014ApJ...795L...4K, 2015ApJ...805...74B}.
The colors of this supernova can be studied in comparison to SN~2011fe, an object with similar
spectral evolution that 
suffers very low galactic and host reddening
\citep[this technique has been used in][]{2006MNRAS.369.1880E,2007AJ....133...58K,2008MNRAS.384..107E,2010AJ....139..120F, 2014ApJ...788L..21A,
2017arXiv170101422H}.
The overall higher photospheric velocities of
SN~2014J are unimportant  based on the findings of  \S\ref{results:sec}.

Data of SN~2014J are  analyzed using our model and using the parameter values found with the SNfactory sample.
The data for the color excesses  in $UBRi$  relative to $V$ at peak brightness  are taken from \citet{2014ApJ...788L..21A},
following their prescription of averaging measurements within 5-days of peak $B$ magnitude.
Their values 
$E(U_o-V_o) =   2.23 \pm   0.03$,
$E(B_o-V_o) =   1.28 \pm   0.04$,
$E(R_o-V)_o =  -0.47 \pm   0.03$,
$E(i_o-V_o) =  -0.92 \pm   0.03$
are plotted in Figure~\ref{sn2014j:fig}, where the subscript $o$ denotes that this is an observed color excess.
The central wavelengths of the filters are 3660, 4280, 5350, 6310, and 8100 \AA\ respectively, in contrast to the central
wavelengths 
3640, 4390, 5290, 6370, 7680 \AA\ of our synthetic bands.  Despite the imperfect correspondence our synthetic passbands
and those used for the observations,
these data are fit to our model
\begin{equation}
E(X_o-V_o) =  \left(\frac{\gamma^0_X}{\gamma^0_V}-1\right)e_0 +  \left(\frac{\gamma^1_X}{\gamma^1_V}-1\right)e_1,
\end{equation}
where we use the median values of the $\pmb{\gamma}$- and $\pmb{\delta}$-terms from Table~\ref{global:tab} and $e_i $ are the fit
parameters.

\begin{figure}[htbp] %  figure placement: here, top, bottom, or page
   \centering
   \includegraphics[width=5.2in]{output11/sn2014j.pdf} 
   \caption{Top: Measured color excess for SN~2014J and the best-fit predictions from this article.
   \color{purple}
   In addition to the prediction
   of our model, also shown is the prediction from the projection
   of our results onto the \citetalias{1999PASP..111...63F} extinction model.  Bottom: Residual color excess
   between the data and models, with a dotted line at zero shown for reference.
   \color{black}
   \label{sn2014j:fig}}
\end{figure}

The best-fit parameters for SN~2014J are 
%----
$e_0= 2.60$, $ e_1=-1.89$ with covariance
%----
\begin{equation}
\begin{pmatrix}
\begin{array}{rr}
0.050 & 0.058 \\
0.058 & 0.315
\end{array}
\end{pmatrix}.
\end{equation}
The predicted values of observed $E(B_o-V_o)$ from the fit are shown in Figure~\ref{sn2014j:fig}, where they are found to
overlap with the data within the error bars. 



In the fit to our model, the observed color excess is attributed to 
%----
$E_{\gamma^0}(B-V)=  0.90 \pm   0.08$ and  $E_{\gamma^1}(B-V)=  0.34 \pm   0.10$
%-----
contributions.
Among the supernovae in the SNfactory  set used to determine $\pmb{\gamma}^0$ and $\pmb{\gamma}^1$, the
objects with extreme median MCMC-sample values in $E_{\gamma^0}(B-V)$ have 
%---
$-0.07 \pm 0.01$ and  $  0.33 \pm 0.04$, and in $E_{\gamma^1}(B-V)$  have $-0.01 \pm 0.01$  and
$  0.06 \pm 0.03$. 
%---
The deduced parameters for SN~2014J, which are relative to SN~2011fe, live well outside the 
span of supernovae used to train the coefficients of the model, which is not unexpected if
attributable to dust. 

\color{purple}
The  parameters $E_{\gamma^0}(B-V)$ and  $E_{\gamma^1}(B-V)=  0.34$ can be transformed into  the 
\citetalias{1999PASP..111...63F} parameters using the transpose of the matrix in Eqn.~\ref{trans_I:eqn},
assuming that  SN~2011fe has zero extinction.  The colors predicted by this \citetalias{1999PASP..111...63F}--component
of our model are  shown in Figure~\ref{sn2014j:fig}. 
Significant differences between the projection and the data is apparent
in the $B-V$ and  $R-V$ colors.
The data and the full model both have shallower decline in color excess with increasing wavelength.
The difficulty that  the  \citetalias{1999PASP..111...63F} model has in accounting for SN~2014J colors within the optical range is also evident
in Figure 3 of  \citet{2014ApJ...788L..21A}, who fit
UV through NIR data to obtain $R_V^F=1.4$ and $E^F(B-V)=1.37$.
Notwithstanding their bias respect to the data, we present the transformed dust parameters
$A^F_V=0.77$ and $E^F(B-V)=1.28$ and covariance
\begin{equation}
\begin{pmatrix}
\begin{array}{rr}
0.430 & -0.004 \\
-0.004 & 0.007
\end{array}
\end{pmatrix},
\end{equation}
which together give $R^F= 0.59^{+ 0.53}_{-0.48}$.
\color{black}


\section{Supernova Model II: Two Extrinsic Parameters, One Intrinsic Parameter}
\label{model2:sec}
In \S\ref{model:sec} we show that the SNfactory data analyzed with an agnostic model with
two extrinsic parameters yield magnitude variations
similar
to those from  two-parameter dust models established for stars in the Milky Way, LMC, and SMC.
In this Section, our model is extended
to probe the existence of
an additional
color parameter.
The extended model, referred to as Model~II, is written as
\begin{equation}
\begin{pmatrix}
U\\B\\V\\R\\I
\end{pmatrix}
\sim \mathcal{N}
\left(
\Delta +
\begin{pmatrix}
c_U+\alpha_U EW_{Ca} + \beta_U EW_{Si} + \eta_U \lambda_{Si} + \delta_U D\\
c_B+\alpha_B EW_{Ca} + \beta_B EW_{Si} + \eta_B \lambda_{Si} + \delta_B D \\
c_V+\alpha_V EW_{Ca} + \beta_V EW_{Si} + \eta_V \lambda_{Si} + \delta_V D\\
c_R+\alpha_R EW_{Ca} + \beta_R EW_{Si} + \eta_R \lambda_{Si} + \delta_R D\\
c_I+\alpha_I EW_{Ca} + \beta_I EW_{Si}+ \eta_I \lambda_{Si} + \delta_I D
\end{pmatrix}
,C_{c}
\right)
\label{ewsiv2:eqn}
\end{equation}
\begin{equation}
\begin{pmatrix}
U_o\\B_o\\ V_o\\R_o\\I_o\\EW_{Si, o}\\ EW_{Ca, o} \\ \lambda_{Si, o}
\end{pmatrix}
\sim \mathcal{N}
\left(
\begin{pmatrix}
U +\gamma^0_{U} k_0 +\gamma^1_{B} k_1 \\B +\gamma^0_{B} k_0 +\gamma^1_{B} k_1 \\
V+\gamma^0_{V} k_0+\gamma^1_{V} k_1\\R+\gamma^0_{R} k_0 + \gamma^1_{R} k_1\\I+\gamma^0_{I} k_0+\gamma^1_{I} k_1\\
EW_{Si}\\ EW_{Ca} \\ \lambda_{Si}
\end{pmatrix}
,C
\right).
\label{dust2:eqn}
\end{equation}
In all aspects Model~II is identical to Model~I except for the addition of the term $\pmb{\delta} D$: for $\pmb{\delta}=0$ and arbitrary $D$, Model II
and Model~I are the same.
The new term contributes to the mean intrinsic
absolute magnitudes. 
Each supernova has an intrinsic $D$ parameter that is linearly related to
absolute magnitudes through the global coefficients $\pmb{\delta}$.  As with $k_0$ and $k_1$,  $D$ is a latent supernova parameter whose existence is inferred from
its effects on broadband magnitudes.


The $\pmb{\delta} D$ term contributes to the means of the Normal distribution in Eq.~\ref{ewsiv2:eqn}, together with the spectral-feature terms.
However, switching the placement of 
$\pmb{\delta} D$ term from Eq.~\ref{ewsiv2:eqn} to Eq.~\ref{dust2:eqn}, i.e.\ from an intrinsic to extrinsic contribution,
does not significantly affect the results, meaning that data are not constraining enough
to identify $\pmb{\delta} D$ as being physically intrinsic or extrinsic.  Nevertheless, $\pmb{\delta} D$ is referred to as ``intrinsic''  for convenience.

\color{red}
We impose
$\langle D\rangle=0$
to specify the zeropoint color.  We assign a prior on the amplitude of the ensemble of $D$ to break the
degeneracy $\pmb{\delta} \rightarrow a\pmb{\delta}$, $D \rightarrow a^{-1}D$: as described
in \S\ref{model:sec} this prior does not affect physics but aids in the convergence of fitting $D$ and $\pmb{\delta}$ as independent parameters.
A sign degeneracy in $a$ remains.
In contrast to our previous experience with $\pmb{\gamma}$ in Model~I, our MCMC chains migrate between
the sign-degenerate solutions in  $\pmb{\delta}$.  We make
no attempt to resolve this degeneracy in the model.
In the running of the Monte Carlo fit, the initial conditions and ranges for
$\pmb{\gamma}^0$ and $\pmb{\gamma}^1$ were set to be in the neighborhood of the best fit of Model~I
to maintain their correspondence between the two models.
Our finite MCMC chains do not migrate away from these ranges.
The initial condition of $\pmb{\delta}$ is set to zero.
\color{black}

In Model II, for $N$ supernovae there are $8N$ observables.  For the top-level model parameters, there are $3N$ spectral parameters, $3(N-2)$
color parameters, $(N-1)$ magnitude offsets,  $5 \times 7$ global coefficients, and $10$ parameters that describe the intrinsic
$C_c$.  For $N=172$ supernovae, there are 1376 observables and 1242 top-level parameters.

The resulting credible intervals of our parameters are given in Table~\ref{global2:tab}.  Comparison with the
results of the 2-parameter extrinsic model given in Table~\ref{global:tab} shows that  parameters that are common to both
are consistent  to well within 1$\sigma$ uncertainties.  The most significant change is in the reduction of the standard
deviations of the magnitude residuals $\sigma_X$
\color{red}
in the $U$ and $B$ bands.
\color{black}


\begin{table}
\centering
\begin{tabular}{|c|c|c|c|c|c|}
\hline
Parameters& $X=U$ &$B$&$V$&$R$&$I$\\ \hline
$\alpha_X$
&
$0.0039^{+0.0009}_{-0.0009}$
&
$0.0013^{+0.0008}_{-0.0007}$
&
$0.0013^{+0.0006}_{-0.0006}$
&
$0.0013^{+0.0005}_{-0.0005}$
&
$0.0025^{+0.0005}_{-0.0004}$
\\
${\alpha_X/\alpha_V-1}$
&
$   2.1^{+   1.4}_{  -0.6}$
&
$   0.0^{+   0.1}_{  -0.2}$
&
$  \dots $
&
$   0.0^{+   0.2}_{  -0.1}$
&
$   1.0^{+   1.1}_{  -0.5}$
\\
$\beta_X$
&
$ 0.032^{+ 0.003}_{-0.003}$
&
$ 0.025^{+ 0.002}_{-0.002}$
&
$ 0.026^{+ 0.002}_{-0.002}$
&
$ 0.021^{+ 0.002}_{-0.002}$
&
$ 0.020^{+ 0.002}_{-0.002}$
\\
${\beta_X/\beta_V-1}$
&
$  0.24^{+  0.05}_{ -0.05}$
&
$ -0.02^{+  0.03}_{ -0.03}$
&
$   \dots$
&
$ -0.19^{+  0.01}_{ -0.01}$
&
$ -0.23^{+  0.03}_{ -0.03}$
\\
$\eta_X$
&
$-0.0002^{+0.0011}_{-0.0011}$
&
$-0.0000^{+0.0009}_{-0.0009}$
&
$0.0005^{+0.0008}_{-0.0008}$
&
$0.0005^{+0.0007}_{-0.0007}$
&
$-0.0002^{+0.0006}_{-0.0006}$
\\
${\eta_X/\eta_V-1}$
&
$ -0.41^{+  2.27}_{ -1.94}$
&
$ -0.37^{+  1.59}_{ -1.26}$
&
$   \dots$
&
$ -0.10^{+  0.30}_{ -0.29}$
&
$ -0.80^{+  1.54}_{ -1.29}$
\\
$\gamma^0_X$
&
$ 61.66^{+  3.97}_{ -3.86}$
&
$ 50.09^{+  3.25}_{ -3.16}$
&
$ 37.41^{+  2.69}_{ -2.60}$
&
$ 28.81^{+  2.35}_{ -2.31}$
&
$ 20.76^{+  2.17}_{ -2.12}$
\\
${\gamma^0_X/\gamma^0_V-1}$
&
$  0.65^{+  0.06}_{ -0.05}$
&
$  0.34^{+  0.03}_{ -0.03}$
&
$   \dots$
&
$ -0.23^{+  0.01}_{ -0.01}$
&
$ -0.45^{+  0.03}_{ -0.03}$
\\
$\gamma^1_X$
&
$ -9.47^{+  3.70}_{ -4.37}$
&
$-11.09^{+  3.02}_{ -3.57}$
&
$-13.59^{+  2.47}_{ -2.81}$
&
$-12.82^{+  2.14}_{ -2.40}$
&
$-11.66^{+  2.25}_{ -2.32}$
\\
${\gamma^1_X/\gamma^1_V-1}$
&
$ -0.30^{+  0.19}_{ -0.21}$
&
$ -0.18^{+  0.11}_{ -0.12}$
&
$   \dots$
&
$ -0.06^{+  0.06}_{ -0.06}$
&
$ -0.15^{+  0.12}_{ -0.12}$
\\
${{\delta_X/\delta_U-1}}$
&
$   \dots$
&
$ -0.22^{+  0.35}_{ -0.32}$
&
$ -0.73^{+  0.51}_{ -0.61}$
&
$ -0.91^{+  0.43}_{ -0.69}$
&
$ -1.17^{+  0.46}_{ -0.93}$
\\
$\sigma_X$
&
$ 0.053^{+ 0.009}_{-0.011}$
&
$ 0.018^{+ 0.012}_{-0.012}$
&
$ 0.020^{+ 0.004}_{-0.005}$
&
$ 0.010^{+ 0.008}_{-0.007}$
&
$ 0.040^{+ 0.006}_{-0.005}$
\\
\hline
\end{tabular}
\caption{68\% credible intervals for the global fit parameters of the Extrinsic--Intrinsic Model~II in \S\ref{model2:sec}.\label{global2:tab}}
\end{table}

\color{red}

The posteriors of the $\pmb{\delta}$ parameters are shown  in
Figure~\ref{deltacorner:fig}.
In the full 5-dimensional space,
the point $\delta_X=0$, i.e.\ the nested Model~I with no effect on magnitude by a third parameter,  lies on the 85\%-ile contour.
Model~II has a  $\pmb{\delta} \rightarrow -\pmb{\delta}$, $D \rightarrow -D$ degeneracy that results in a bimodal posterior.
To aid in the visualization of the results,
we break this degeneracy by imposing  that the $D$ of the one supernova with the highest signal-to-noise be positive.
The new posterior contours of $\pmb{\delta}$ with this added
condition are shown in Figure~\ref{deltacorner:fig}.
The shapes of the new $\pmb{\delta}$ posteriors are somewhat distorted compared to the original;
despite its relatively strong signal-to-noise, the original posterior for $D$ for the selected supernova does have significance on both sides of zero,
resulting in changes in the shape of the $\pmb{\delta}$ posteriors when  $D$ to forced to be positive.   
The modes of $\delta_X$ are at $\mp 1.12$, $\mp 0.89$, $\mp 0.36$,  $\mp0.16$, $\pm0.21$ in for $X=UBVRI$ respectively.
The values of $\delta_X/\delta_U-1$, which are insensitive to the scaling (including the sign) degeneracy, are shown in Figure~\ref{deltaratio:fig}.

\begin{figure}[htbp] %  figure placement: here, top, bottom, or page
   \centering
   \includegraphics[width=2.8in]{output25/delta_corner.pdf}
     \includegraphics[width=2.8in]{output25/delta_corner_flipped.pdf}
   \caption{Left:  Model~II posterior contours for $\pmb{\delta}$.
   Right: The same posteriors after imposing that $D$ of the one supernova with highest signal-to-noise be positive.
     Lines for $\pmb{\delta}=0$ are overplotted.
   \label{deltacorner:fig}}
\end{figure}


\begin{figure}[htbp] %  figure placement: here, top, bottom, or page
   \centering
      \includegraphics[width=5.2in]{output25/deltaratio.pdf}
   \caption{Model~II median values  and corresponding 68\% intervals for $\delta_X/\delta_U$ in the 5 bands.
   A dotted line at zero is shown for reference.
   \label{deltaratio:fig}}
\end{figure}

\color{black}


%The effective total-to-selective extinction for the supernova sample has a slight shift to $R^F_V=2.23 \pm 0.16$.

The distributions of both sources of color excess $E(B-V)$ and
 $E_\delta(B-V) \equiv   (\delta_B-\delta_V)D$ are shown in the ideograms 
 and histogram in Figure~\ref{ebv:fig}.
 (These values are relative those of an arbitrary supernova in order to null out correlated errors.)
The range of $B-V$ colors is $> 5$ times larger for the external contribution than for the extrinsic dust
component.  
The standard deviations of  the median $E_\gamma(B-V)$ and $E_\delta(B-V)$ are
%-----
0.075
and 0.014
%-----
mag respectively.
The $E_\delta(B-V)$ standard deviation is comparable to its measurement uncertainty per-supernova.
The color excesses of individual supernovae have low signal-to-noise 
and the significance of $\pmb{\delta}$
is drawn from the ensemble of objects.

\begin{figure}[htbp] %  figure placement: here, top, bottom, or page
   \centering
   \includegraphics[width=2.8in]{output25/ebv.pdf}
   \includegraphics[width=2.8in]{output25/ebv_delta.pdf}
      \caption{Left: Ideograms of the external $E(B-V)$ and
   internal $E_\delta(B-V) = (\delta_B-\delta_V)D$  contributions to color excess relative to an arbitrary supernova, for the objects in our sample.
   Right: Ideogram and histogram of the medians of $E_\delta(B-V)$ on an enlarged scale.
   \label{ebv:fig}}
\end{figure}

Each supernova is described by its parameters
 $\Delta$, $EW_{Ca}$, $EW_{Si}$, $\lambda_{Si}$, $E_{\gamma^0}(B-V)$, $E_{\gamma^1}(B-V)$,  $A_{\delta U}=\delta_U D$.
 Their Pearson correlation coefficients are given in the matrix
\color{purple}
\begin{multline}
Cor(\Delta, EW_{Ca}, EW_{Si}, \lambda_{Si}, E_{\gamma^0}(B-V), E_{\gamma^1}(B-V),  A_{\delta U}) =\\
\begin{pmatrix}
\begin{array}{rrrrrrr}
1 & 0.01^{+0.06}_{-0.06} & -0.07^{+0.07}_{-0.06} & -0.04^{+0.06}_{-0.07} & 0.10^{+0.07}_{-0.06} & 0.00^{+0.08}_{-0.07} & -0.01^{+0.08}_{-0.08} \\
0.01^{+0.06}_{-0.06} & 1 & 0.10^{+0.04}_{-0.04} & -0.26^{+0.03}_{-0.03} & -0.04^{+0.06}_{-0.06} & -0.00^{+0.07}_{-0.08} & -0.00^{+0.08}_{-0.08} \\
-0.07^{+0.07}_{-0.06} & 0.10^{+0.04}_{-0.04} & 1 & -0.14^{+0.03}_{-0.03} & -0.16^{+0.05}_{-0.05} & -0.03^{+0.08}_{-0.08} & 0.01^{+0.08}_{-0.08} \\
-0.04^{+0.06}_{-0.07} & -0.26^{+0.03}_{-0.03} & -0.14^{+0.03}_{-0.03} & 1 & 0.03^{+0.05}_{-0.06} & 0.02^{+0.07}_{-0.08} & -0.01^{+0.08}_{-0.08} \\
0.10^{+0.07}_{-0.06} & -0.04^{+0.06}_{-0.06} & -0.16^{+0.05}_{-0.05} & 0.03^{+0.05}_{-0.06} & 1 & 0.05^{+0.08}_{-0.08} & -0.01^{+0.07}_{-0.07} \\
0.00^{+0.08}_{-0.07} & -0.00^{+0.07}_{-0.08} & -0.03^{+0.08}_{-0.08} & 0.02^{+0.07}_{-0.08} & 0.05^{+0.08}_{-0.08} & 1 & -0.00^{+0.07}_{-0.08} \\
-0.01^{+0.08}_{-0.08} & -0.00^{+0.08}_{-0.08} & 0.01^{+0.08}_{-0.08} & -0.01^{+0.08}_{-0.08} & -0.01^{+0.07}_{-0.07} & -0.00^{+0.07}_{-0.08} & 1 \\
\end{array}
\end{pmatrix}.
\end{multline}
\color{black}
Their values for all Monte Carlo links for all supernovae are shown in Figure~\ref{perobject2:fig}.
There is a core concentration in the  parameter-space, with around ten objects that occupy its outskirts.
Many outliers appear in the red tail of $E_{\gamma^0}(B-V)$, as would be expected for the (infrequent) selection of supernovae
heavily extinguished by host-galaxy dust.
Outliers  are also clearly distinguishable in  $EW_{Ca}$--$\lambda_{Si}$ space.   

%The distributions for $\Delta$, $EW_{Ca}$, $EW_{Si}$, $\lambda_{Si}$, $E_{\gamma^0}(B-V)$, 
%$E_{\gamma^1}(B-V)$, and
%$E_{\delta}(B-V)$ for all Monte Carlo links for all supernovae are shown in Figure~\ref{perobject2:fig}.

The correlation coefficients between the ``intrinsic''  ($A_{\delta U}$) and ``extrinsic''
($E_{\gamma^0}(B-V)$, $E_{\gamma^1}(B-V)$)  parameters are
\color{purple}
directed by the prior and are ultimately consistent with zero.
Their independence implies that a single physical parameter is not being artificially attributed to two
model parameters.  One could have expected this to occur since our linear model does not precisely
describe the non-linearity between broad-band magnitudes and dust parameters. A
correlation 
between the parameters despite the prior would of complicated any claims of the detection of a third independent supernova parameter.
\color{black}

\begin{figure}[htbp] %  figure placement: here, top, bottom, or page
   \centering
   \includegraphics[width=5.2in]{output25/perobject_corner.pdf} 
   \caption{
   Distributions for the supernova parameters $\Delta$, $EW_{Ca}$, $EW_{Si}$, $\lambda_{Si}$, $E_{\gamma^0}(B-V)$,  $E_{\gamma^1}(B-V)$,  and $E_{\delta}(B-V)$, as well as the grey offset
$\Delta$.  All Monte Carlo links are plotted, so that each supernova contributes a cloud of points.
   \label{perobject2:fig}}
\end{figure}


\color{red}
The newly fitted $\pmb{\gamma}$ and  $\pmb{\delta}$ vectors (the latter normalized by $\delta_U$) can be expressed 
in terms of the $\mathbf{a}$ and $\mathbf{b}$ vectors that describe the \citetalias{1999PASP..111...63F} dust model
\color{black}
\begin{equation}
\begin{pmatrix}
 \frac{\pmb{\gamma}^0}{\gamma^0_B-\gamma^0_V} \\
\frac{\pmb{\gamma}^1}{\gamma^1_B-\gamma^1_V} \\
\frac{\delta_X}{\delta_U} \\
\end{pmatrix}=
\begin{pmatrix}
\begin{array}{rr}
2.88 & 1.16 \\
-5.00 & 0.73 \\
0.24 & 0.46
\end{array}
\end{pmatrix} 
\begin{pmatrix}
\mathbf{a} \\
\mathbf{b}
\end{pmatrix}+
\begin{pmatrix}
\pmb{\epsilon}_{\gamma^0} \\
\pmb{\epsilon}_{\gamma^1} \\
\pmb{\epsilon}_{\delta}
\end{pmatrix},
\label{model2trans:eqn}
\end{equation}
where the residual vectors
\begin{align}
\begin{split}
\pmb{\epsilon}_{\gamma^0} &=(0.07, -0.07, -0.07,  0.06,  0.02),\\
\pmb{\epsilon}_{\gamma^1} &=( -0.04, 0.14,  -0.24,  0.32, -0.22),\\
\pmb{\epsilon}_{\delta} &=(-0.05, 0.08,  -0.03,  -0.09, -0.12)
\end{split}
\label{model2res:eqn}
\end{align}
are perpendicular to $a$ and $b$.
The $\pmb{\gamma}$ residuals $\pmb{\epsilon}_{\gamma^0}$ and
$\pmb{\epsilon}_{\gamma^1} $
quadratically contribute 0.03\% and 0.2\% to the total  lengths of the vectors, meaning that just as in Model~I most
of the $\pmb{\gamma}$'s project onto the plane predicted by the \citetalias{1999PASP..111...63F} dust model.
\color{red}
 $\pmb{\delta}$ too
has a significant projection onto the dust plane, as the $\pmb{\epsilon}_{\delta}$ residual contributes 2\% of the length of the $\pmb{\delta}$ vector.
The transformation between the new set of $\pmb{\gamma}$'s and dust parameters yields
a 68\%-ile  credible interval for $R^F$ of  $2.31$--$2.69$, shifted to slightly  higher values relative to the results of Model~I.

The significance of the $\pmb{\delta}$ measurement is evaluated using simulated datasets.
For one set of tests, we realize 50 random realizations of data with input  parameters equal to the median values of the 
parameter posteriors found in
\S\ref{model:sec}, and using the same noise as our data.  To represent the null hypothesis, the input $\pmb{\delta}$-vector is set to zero.
Figure~\ref{pvalues_delta2:fig} shows results of fits of these null-realizations next  to the results from the real data.
One statistic shown is the median and 68\% credible interval of the squared amplitude  $\delta^2$.
As $\delta^2$ measures the magnitude response due to the third parameter, it is expected to have a small value
when such a new parameter is not relevant.
Comparing separately the  $-1$, 0, and $1$-sigma
values of  the credible interval of our data relative to those from the simulations, we obtain
three different $p$-values ranging from 0.90--0.98, indicating that the data result is incompatible with the null hypothesis. 
As a point of comparison, recall that in the Bayesian posterior
the null result
$\pmb{\delta}=0$ lies on the 85\%-ile contour.

\begin{figure}[htbp] %  figure placement: here, top, bottom, or page
   \centering
   \includegraphics[width=2.9in]{output25_sim_null/pvalues_delta2.pdf} 
  \includegraphics[width=2.9in]{output25_sim_null/pvalues_relperp.pdf} 
            \caption{ Left: The squared length, $\delta^2$;  and Right: the relative length perpendicular to the plane defined by $\pmb{\gamma}$, $\delta_\perp^2/\delta^2$, for 50 independent simulated data sets with input $\delta=0$.  The fits
            are ordered by the median value, the null realizations are shown in blue and the result from our data is shown in red.
            \label{pvalues_delta2:fig}}
\end{figure}

An interesting feature of $\pmb{\delta}$ is that its direction lies close to the plane defined by the
$\pmb{\gamma}$ vectors.
This leads to the concern that our model and analysis  could spontaneously generate a third
vector almost degenerate with the others, despite the prior for the independence of $k_0$, $k_1$, and $D$.
This hypothesis is tested using the 
relative projection of $\pmb{\delta}$  perpendicular to the plane defined by the $\pmb{\gamma}$'s, $\delta_\perp^2/\delta^2$,
for the same 50 null simulations that were just described.
Figure~\ref{pvalues_delta2:fig} shows  that for none of the simulations does our analysis produce 
 $\delta_\perp^2/\delta^2$ as low as that from our data.  The $p$-values $>0.98$ indicate that the result from our data
 is incompatible with the null hypothesis.


Potential biases in our results are measured through simulation.
We realize 50 random realizations of data with input  parameters equal to the median values found in
this Section, except for $\pmb{\delta}$ which is assigned the, and the same noise as our data.  The posterior for each realization is analyzed individually, and the posteriors of all realizations are stacked.  The stack is sensitive to
biases in the best fit value.
The stack of  $\pmb{\delta}$  is shown in Figure~\ref{simdelta:fig}.  There is a bias between
the input  $(-1.12, -0.89, -0.36,  -0.16, 0.21)$ and the mode of  the stacks $(-1.06, -0.95, -0.41, -0.11, 0.40)$.
We do not consider this bias to be an issue in our detection of a third parameter, as the difference is significantly
smaller than the statistical uncertainty of a single analysis and the length of the vector changes by $< 3\%$.
The posterior stacks for the other global parameters 
are shown in Figures~\ref{simglobal1:fig}--\ref{simglobal5:fig}.  The input parameters are recovered with negligible bias
relative to the statistical uncertainties. 
Note that some bias is expected: the priors for the distributions of $\Delta$, $k$, and $D$ are inconsistent
with the extracted distributions.  Also, 
there is a tendency for the slope parameters associated with the spectral features to be slightly underestimated, which is an expected consequence of 
the uncertainties in the predictor variable that lead to 
regression dilution \citep{spearman04}.

\begin{figure}[htbp] %  figure placement: here, top, bottom, or page
   \centering
   \includegraphics[width=5.2in]{output25_sim/delta_corner.pdf} 
            \caption{Stacked $\pmb{\delta}$ posterior contours for 50 independent simulated data sets.
            The blue crosshairs denote the input $\pmb{\delta}$ used to generate the simulated data.
            \label{simdelta:fig}}
\end{figure}


\begin{figure}[htbp] %  figure placement: here, top, bottom, or page
   \centering
   \includegraphics[width=5.2in]{output25_sim/coeff0.pdf} 
            \caption{Aggregated Model~II posterior contours for $c$, $\alpha$, $\beta$, $\eta$, $\pmb{\gamma}^0$, $\pmb{\gamma}^1$,  $\pmb{\delta}$, and $\sigma$ in the $U$ band of ten independent simulated data sets.  The blue crosshairs denote the input $\pmb{\delta}$ used to generate the simulated data.
            \label{simglobal1:fig}}
\end{figure}

\begin{figure}[htbp] %  figure placement: here, top, bottom, or page
   \centering
   \includegraphics[width=5.2in]{output25_sim/coeff1.pdf} 
            \caption{Aggregated Model~II posterior contours for $c$, $\alpha$, $\beta$, $\eta$, $\pmb{\gamma}^0$, $\pmb{\gamma}^1$,  $\pmb{\delta}$, and $\sigma$ in the $B$ band of ten independent simulated data sets.  The blue crosshairs denote the input $\pmb{\delta}$ used to generate the simulated data.
 \label{simglobal2:fig}}
\end{figure}

\begin{figure}[htbp] %  figure placement: here, top, bottom, or page
   \centering
   \includegraphics[width=5.2in]{output25_sim/coeff2.pdf} 
            \caption{Aggregated Model~II posterior contours for $c$, $\alpha$, $\beta$, $\eta$, $\pmb{\gamma}^0$, $\pmb{\gamma}^1$,  $\pmb{\delta}$, and $\sigma$ in the $V$ band of ten independent simulated data sets.   The blue crosshairs denote the input $\pmb{\delta}$ used to generate the simulated data. \label{simglobal3:fig}}
\end{figure}

\begin{figure}[htbp] %  figure placement: here, top, bottom, or page
   \centering
      \includegraphics[width=5.2in]{output25_sim/coeff3.pdf} 
            \caption{Aggregated Model~II posterior contours for $c$, $\alpha$, $\beta$, $\eta$, $\pmb{\gamma}^0$, $\pmb{\gamma}^1$,  $\pmb{\delta}$, and $\sigma$ in the $R$ band of ten independent simulated data sets.   The blue crosshairs denote the input $\pmb{\delta}$ used to generate the simulated data.
\label{simglobal4:fig}}
\end{figure}

\begin{figure}[htbp] %  figure placement: here, top, bottom, or page
   \centering
         \includegraphics[width=5.2in]{output25_sim/coeff4.pdf} 
            \caption{Aggregated Model~II posterior contours for $c$, $\alpha$, $\beta$, $\eta$, $\pmb{\gamma}^0$, $\pmb{\gamma}^1$,  $\pmb{\delta}$, and $\sigma$ in the $I$ band of ten independent simulated data sets.  The blue crosshairs denote the input $\pmb{\delta}$ used to generate the simulated data.
 \label{simglobal5:fig}}
\end{figure}

The Model~II fit of SN~2014J \citet{2014ApJ...788L..21A}  data 
gives $k_0$ and $k_1$ that are qualitatively similar to  those from the fit to Model~I but with much larger uncertainties.
These parameters are now highly correlated with $D$, which itself is consistent with zero.  The results resemble those of
Model~I shown in Figure~\ref{sn2014j:fig}, except for larger error bars for the color excesses projected onto the dust model.
\color{black}


\section{Discussion}
\label{discussion:sec}

\subsection{Light Curve Shape}
\label{shape:sec}
The motivation of our analyses was to standardize supernovae based on spectral features and colors at peak brightness.
As such, our models do not explicitly include light-curve shape, which is known to track SN~Ia diversity.
We now examine the effect of shape in more depth using the intrinsic--extrinsic Model~II of \S\ref{model2:sec}.
The connection between the spectrum at peak and the light-curve decline rate, seen through
the correlation between $EW_{Si}$ and the SALT2 $x_1$ light-curve shape parameter, is verified
for our sample in \citet{2017Chotard}.
On the other hand, there is no such correlation seen between $A_{\delta I} = \delta_I D$ and $x_1$, as shown in Figure~\ref{x1:fig}. 
Our magnitude residuals, $\sigma_X$, are comparable to the residuals after SALT2 training
\citep{2010A&A...523A...7G}, showing that our model exhibits little loss in its ability to standardize
magnitudes even though $x_1$ is neglected.
There is no apparent correlation
between $x_1$ and the residuals between observed and model-predicted colors, as seen in
Figure~\ref{x1res:fig},
though there is a hint of nonlinearity in the $B-V$ and $R-V$ colors.

A direct consideration of light-curve shape is
performed by explicitly adding to Model~II a new linear term $\zeta x_1$ to the 
intrinsic magnitude. The credible intervals
from this analysis are given in Table~\ref{globalx1:tab}; there are no significant changes with respect to the reference model. 
Light-curve shape is not a strong predictor of supernova magnitude relative to the spectral features: $\zeta_R/\zeta_V$
has the largest significance  at $1\sigma$.
The evidence for the effect of a third parameter persists in the significant  non-zero  $\delta_X/\delta_I-1$ values
in Table~\ref{globalx1:tab}, reinforcing the evidence that the $D$ of Model~II is not primarily related to light-curve shape.

\begin{figure}[htbp] %  figure placement: here, top, bottom, or page
   \centering
%   \includegraphics[width=2.8in]{output25/x1si.pdf}
   \includegraphics[width=5in]{output25/x1D.pdf}
    \caption{The SALT2 $x_1$ versus $A_{\delta U}  = \delta_U D$.
   \label{x1:fig}}
\end{figure}


\begin{figure}[htbp] %  figure placement: here, top, bottom, or page
   \centering
   \includegraphics[width=5.2in]{output25/residualx1.pdf}
    \caption{Differences between observed colors and the colors predicted from the analysis, as a function
            of the SALT2 light-curve shape parameter $x_1$.  Our model does not explicitly use $x_1$.  For reference a dotted line is plotted at zero difference.
   \label{x1res:fig}}
\end{figure}


\begin{table}
\centering
\begin{tabular}{|c|c|c|c|c|c|}
\hline
Parameters & $X=U$ &$B$&$V$&$R$&$I$\\ \hline
$\alpha_X$
&
$0.0037^{+0.0010}_{-0.0010}$
&
$0.0011^{+0.0008}_{-0.0008}$
&
$0.0013^{+0.0007}_{-0.0007}$
&
$0.0013^{+0.0006}_{-0.0006}$
&
$0.0026^{+0.0005}_{-0.0005}$
\\
${\alpha_X/\alpha_V-1}$
&
$   1.7^{+   1.3}_{  -0.5}$
&
$  -0.1^{+   0.2}_{  -0.4}$
&
$  \ldots$
&
$  -0.0^{+   0.2}_{  -0.1}$
&
$   1.0^{+   1.2}_{  -0.5}$
\\
$\beta_X$
&
$ 0.034^{+ 0.006}_{-0.005}$
&
$ 0.026^{+ 0.005}_{-0.005}$
&
$ 0.023^{+ 0.004}_{-0.004}$
&
$ 0.020^{+ 0.003}_{-0.003}$
&
$ 0.017^{+ 0.003}_{-0.003}$
\\
${\beta_X/\beta_V-1}$
&
$  0.50^{+  0.11}_{ -0.10}$
&
$  0.15^{+  0.06}_{ -0.06}$
&
$  \ldots$
&
$ -0.12^{+  0.03}_{ -0.03}$
&
$ -0.26^{+  0.06}_{ -0.06}$
\\
$\eta_X$
&
$0.0002^{+0.0011}_{-0.0011}$
&
$0.0004^{+0.0009}_{-0.0009}$
&
$0.0009^{+0.0008}_{-0.0008}$
&
$0.0008^{+0.0007}_{-0.0007}$
&
$0.0000^{+0.0006}_{-0.0006}$
\\
${\eta_X/\eta_V-1}$
&
$ -0.47^{+  0.69}_{ -1.37}$
&
$ -0.39^{+  0.41}_{ -0.93}$
&
$  \ldots$
&
$ -0.12^{+  0.19}_{ -0.16}$
&
$ -0.78^{+  0.42}_{ -0.88}$
\\
$\zeta_X$
&
$  0.01^{+  0.04}_{ -0.04}$
&
$ -0.00^{+  0.03}_{ -0.03}$
&
$ -0.03^{+  0.03}_{ -0.03}$
&
$ -0.01^{+  0.02}_{ -0.02}$
&
$ -0.03^{+  0.02}_{ -0.02}$
\\
${\zeta_X/\zeta_V-1}$
&
$ -0.78^{+  1.55}_{ -1.96}$
&
$ -0.62^{+  1.21}_{ -1.33}$
&
$  \ldots$
&
$ -0.46^{+  0.45}_{ -0.52}$
&
$ -0.14^{+  0.51}_{ -0.42}$
\\
$\gamma^0_X$
&
$ 59.51^{+  4.09}_{ -4.09}$
&
$ 49.02^{+  3.32}_{ -3.30}$
&
$ 36.96^{+  2.69}_{ -2.70}$
&
$ 28.37^{+  2.36}_{ -2.36}$
&
$ 20.35^{+  2.16}_{ -2.16}$
\\
${\gamma^0_X/\gamma^0_V-1}$
&
$  0.61^{+  0.06}_{ -0.05}$
&
$  0.33^{+  0.03}_{ -0.03}$
&
$  \ldots$
&
$ -0.23^{+  0.01}_{ -0.01}$
&
$ -0.45^{+  0.03}_{ -0.03}$
\\
$\gamma^1_X$
&
$ -7.57^{+  3.75}_{ -4.06}$
&
$ -9.63^{+  2.90}_{ -3.27}$
&
$-13.65^{+  2.36}_{ -2.60}$
&
$-12.98^{+  2.08}_{ -2.27}$
&
$-12.16^{+  2.08}_{ -2.19}$
\\
${\gamma^1_X/\gamma^1_V-1}$
&
$ -0.44^{+  0.19}_{ -0.24}$
&
$ -0.30^{+  0.11}_{ -0.13}$
&
$  \ldots$
&
$ -0.05^{+  0.05}_{ -0.05}$
&
$ -0.11^{+  0.11}_{ -0.10}$
\\
$\delta_X$
&
$  0.14^{+  1.13}_{ -1.41}$
&
$  0.11^{+  0.91}_{ -1.10}$
&
$  0.04^{+  0.56}_{ -0.60}$
&
$  0.01^{+  0.43}_{ -0.43}$
&
$ -0.02^{+  0.43}_{ -0.41}$
\\
${{\delta_X/\delta_U-1}}$
&
$  \ldots$
&
$ -0.22^{+  0.28}_{ -0.21}$
&
$ -0.63^{+  0.33}_{ -0.36}$
&
$ -0.82^{+  0.30}_{ -0.44}$
&
$ -1.05^{+  0.35}_{ -0.63}$
\\
$\sigma_X$
&
$ 0.053^{+ 0.008}_{-0.011}$
&
$ 0.018^{+ 0.011}_{-0.011}$
&
$ 0.017^{+ 0.004}_{-0.006}$
&
$ 0.010^{+ 0.007}_{-0.007}$
&
$ 0.040^{+ 0.006}_{-0.005}$
\\
\hline
\end{tabular}
\caption{68\% credible intervals for the Global Fit Parameters including the light-curve shape parameter, $x_1$ \label{globalx1:tab}}
\end{table}


\subsection{SiII Line Velocity}
\label{velocity:sec}
\citet{2009ApJ...699L.139W, 2011ApJ...729...55F} find a connection between $v_{Si}$ and color, and  
\citet{2015MNRAS.447.1247S} show that $v_{Si}$ is an important spectral classifier within the SNfactory data themselves.
To first order, the $\lambda_{Si}$ treated in this article varies linearly with $v_{Si}$.
The values of the $v_{Si}$ coefficient, $\eta$, in this article are consistent with zero.  $EW_{Ca}$ and $EW_{Si}$ have a significant effect on color,
and in turn $\lambda_{Si}$ is correlated with $EW_{Ca}$.
Removing from our model the dependence on equivalent widths (eliminating the  $\alpha$ and $\beta$ parameters), we recover
non-zero $\eta$ values at  $\gtrsim 2\sigma$.
The Ca and Si responsible for the equivalent widths and wavelength of our selected spectral features
are expected to be produced together in the thermonuclear burning into intermediate mass elements.
We conclude that $v_{Si}$ is correlated with color, 
but this correlation
is absorbed into the equivalent-width corrections.


\color{red}
The results from  \S\ref{shape:sec} and this subsection demonstrate the model handling simultaneous
magnitude corrections 
of  correlated observables.  The model prefers corrections that reduce intrinsic dispersion.
The free parameters that describe the intrinsic dispersion enter the model likelihood through  the term $(\det{C_c})^{-0.5}$, which gives higher likelihood 
for lower dispersions. This leads to magnitude corrections being preferentially
attributed to $EW_{Ca}$ over $v_{Si}$, and  $EW_{Si}$ over SALT2 $x_1$.
\color{black}


\subsection{Host-Galaxy Mass}
A correlation between Hubble residual and host-galaxy mass
was first noted by \citet{2010ApJ...715..743K,2010MNRAS.406..782S}, a signal confirmed to exist in the SNfactory
sample \citep{2013ApJ...770..108C}.
This host-mass bias could be the result of a parameter that was not accounted for in the inference of SN~Ia absolute magnitude.
\citet{2016arXiv160904470M} find that when introducing intrinsic color scatter as a latent parameter, the null mass-step effect falls on the 95\%-level of their posterior.
To explore whether the Hubble residual may be due to unaccounted color,
we plot in Figure~\ref{childress:fig} our intrinsic parameter  $A_{\delta I}$  versus host mass 
\color{purple}
for the subset of supernovae whose host measurements are given in \citet{2013ApJ...770..108C}.
The average $A_{\delta I}$'s of the two subsamples are
%---
$\langle A_{\delta I} \rangle=     -0.0002 \pm    0.0089$ mag,
$\langle A_{\delta I} \rangle= 0.0025 \pm   0.0073 $ mag
for low- and high-mass hosts respectively: there is no significant difference between them. 


\begin{figure}[htbp] %  figure placement: here, top, bottom, or page
   \centering
   \includegraphics[width=2.8in]{output25/childress.pdf}
   \includegraphics[width=2.8in]{output25/childress_pvalue.pdf}
      \caption{Left: Intrinsic parameter $A_{\delta U}=\delta_U D$  versus host galaxy mass. Overplotted are the mean and 1$\sigma$ uncertainty on the mean for supernovae with hosts
      less than and greater than  $\log{(M/M_\sun)}=10$.
Right: Histogram of the two-tailed $p$-value from  the Kolmogorov-Smirnov test applied to all links in the Monte Carlo chains.
   \label{childress:fig}}
\end{figure}


The Kolmogorov-Smirnov test is applied on the distribution of $A_{\delta I}$ of both low- and high-mass samples for
each Monte Carlo link.  The resulting distribution of the two-tailed $p$-value  over all chains spans the possible range from 0 to 1, as shown in Figure~\ref{childress:fig}.  We are unable to perform a constraining test of whether the $A_{\delta I}$ distributions from low- and high-mass samples
are drawn from a common distribution.

A more pertinent test is to see if a host-mass bias persists in Hubble diagram residuals
derived from a recalibration of SN~Ia absolute magnitudes using the parameters of the current article.
The calibration of absolute magnitudes goes beyond the design of our model and analysis (see \S\ref{standard:sec}), so this test
in not presented here.

\subsection{Improving SNe~Ia as Standard Candles}
\label{standard:sec}
The approach of our analysis is to mine for new supernova properties based on colors and spectral features.
Our model and results do carry information on absolute magnitude but are not tailored for its study.
The grey parameter $\Delta$ with its $0.10$ mag dispersion, while containing information on absolute magnitude, 
also conflates
contributions from peculiar velocities and measurement uncertainties.  Studies
focused on improving SNe~Ia as standard candles should modify  our model to
distinguish between these sources of greyness and dispersion.
Our
mining exercise uses all supernovae that pass quality cuts, the approach taken by previous
exploratory work.  In contrast, the calibration of SNe~Ia as standard candles needs procedures such
as cross-validation to avoid overtraining.
An absolute magnitude calibration using the parameters identified in this analysis is left to future work.


\subsection{Physical Interpretation of the Intrinsic Color Parameter}

Based on its direction, a straightforward explanation for the third color term $\pmb{\delta} D$ is that it describes the absorption due to some form of dust
that does not exactly obey the  \citetalias{1999PASP..111...63F} model.
In one scenario,
supernova light can encounter two
dusts, first a 
circumstellar dust that produces relatively low visual-to-selective extinction
\citep{2008ApJ...686L.103G} and then
an interstellar form  that resembles the \citetalias{1999PASP..111...63F} model.
To first order $\pmb{\delta}$ behaves like standard dust;   $>98$\% of its length projects onto the expectation of  \citetalias{1999PASP..111...63F}
with constant $R^F$.
The departure from   \citetalias{1999PASP..111...63F}
can be seen qualitatively; the decrease in magnitude change toward redder wavelengths, seen in Figure~\ref{deltaratio:fig}, matches the 
expectation for the circumstellar form of dust.
However, we
lack the signal-to-noise to show consistency with this explanation.
The best-fit elements $\delta_X$ do not all share the same sign, as would have been expected for dust.
Confining ourselves to the 98\% of   $\pmb{\delta}$ that lies on the dust plane, associating 
$D$ with $A_V$ would imply an unphysical $R^F=0.5$.

An alternative scenario is that supernova light encounters one dust with a degree of freedom beyond that modeled by \citetalias{1999PASP..111...63F}.
The $\pmb{\delta} D$ term represents an intrinsic dust parameter that is independent of $R^F$.
The extra degree-of-freedom could be associated with the skewness in the distribution of dust grains, as an increase of smaller dust grains
can flatten extinction in the red and enhanced steepness in the UV relative to standard dust models  \citep[Figure~3 in][]{2015ApJ...807L..26G,
2017ApJ...836...13H}.  Unfortunately, our signal-to-noise for $\pmb{\delta}$ is insufficient to test these physically-motivated
dust models.



\subsection{Conclusion}
To summarize, we model SNe~Ia broadband optical peak magnitudes allowing for correlations with spectral features at peak and
extrinsic color parameters.  Analyzing SNfactory data with this model, we find significant evidence that the above parameters do
affect supernova magnitudes and colors.  This model  does an excellent job in
describing SN~2014J, a supernova with extreme colors that was not used in the training.  
This work represents the first determination of the dust extinction curve outside the Local Group
derived entirely independently of assumptions about the shape of the extinction curve and/or assumptions about the
distribution of $A_V$.  
\color{purple}
Nesting this model in an extended model produces a result that 
\color{black}
gives the first identification of  a new parameter that affects supernova
colors in a manner that is distinct from the expectations from dust or the spectral features $EW_{Ca}$, $EW_{Si}$, and $\lambda_{Si}$.

\color{purple}
The current data does not show a statistically significant difference in the values of our new parameter for supernovae
in low- and high-mass host-galaxies.
Using the SNfactory sample,
\citet{2013A&A...560A..66R,2017Rigault} find that a step in Hubble residuals is better related to local star formation rate, rather than
global host mass.  The search for a correlation between our parameter and local star formation rate is planned for future work.

\color{black}

An interesting direction for future work is to use narrower bands for generating synthetic photometry
from the SNfactory spectrophotometry.  When the bandwidth becomes small enough
to resolve spectral features, the analysis would produce a spectroscopic model.   Higher resolution would allow us to forego 
the use of a fiducial template necessary to predict broadband fluxes generated by dust models, and also allow direct incorporation
of dust models into our framework.

\acknowledgments
We thank the STAN team for providing the statistical tool without which this analysis would not have been possible,
and Michael Betancourt specifically for his helpful guidance.  We thank Danny Goldstein and
Xiaosheng Huang for useful discussions.
We thank Dan Birchall for observing assistance, the technical and
scientific staffs of the Palomar Observatory, the High Performance
Wireless Radio Network (HPWREN), and the University of Hawaii 2.2~m
telescope.  We recognize the significant cultural role of Mauna Kea
within the indigenous Hawaiian community, and we appreciate the
opportunity to conduct observations from this revered site.  This
work was supported in part by the Director, Office of Science,
Office of High Energy Physics, of the U.S. Department of Energy
under Contract No. DE-AC02- 05CH11231.  Support in France was
provided by CNRS/IN2P3, CNRS/INSU, and PNC; LPNHE acknowledges
support from LABEX ILP, supported by French state funds managed by
the ANR within the Investissements d'Avenir programme under reference
ANR-11-IDEX-0004-02.  NC is grateful to the LABEX Lyon Institute
of Origins (ANR-10-LABX-0066) of the Universit\'e de Lyon for its
financial support within the program ``Investissements d'Avenir''
(ANR-11-IDEX-0007) of the French government operated by the National
Research Agency (ANR).  Support in Germany was provided by the DFG
through TRR33 ``The Dark Universe;'' and in China from Tsinghua
University 985 grant and NSFC grant No~11173017.  Some results were
obtained using resources and support from the National Energy
Research Scientific Computing Center, supported by the Director,
Office of Science, Office of Advanced Scientific Computing Research,
of the U.S. Department of Energy under Contract No. DE-AC02-05CH11231.
HPWREN is funded by National Science Foundation Grant Number
ANI-0087344, and the University of California, San Diego.


\bibliographystyle{apj}
\bibliography{/Users/akim/Documents/alex}


\end{document} 
