\documentclass{aastex61}   	% use "amsart" instead of "article" for AMSLaTeX format
\usepackage{geometry}                		% See geometry.pdf to learn the layout options. There are lots.
\geometry{letterpaper}                   		% ... or a4paper or a5paper or ... 
\usepackage{graphicx}				% Use pdf, png, jpg, or eps§ with pdflatex; use eps in DVI mode
\usepackage{amsmath}
\usepackage{amssymb}
\usepackage{natbib}
\usepackage{lineno}
\usepackage{color}
\defcitealias{1999PASP..111...63F}{F99}
\linenumbers


\begin{document}

\title{Evidence for a Color Parameter Unassociated With Dust Within the Type~Ia Supernovae of the Nearby Supernova Factory}
\author[0000-0001-6315-8743]{A.~G.~Kim}
\affiliation{    Physics Division, Lawrence Berkeley National Laboratory, 
    1 Cyclotron Road, Berkeley, CA, 94720}
    

\author{     G.~Aldering}
\affiliation{    Physics Division, Lawrence Berkeley National Laboratory, ?out???�
    1 Cyclotron Road, Berkeley, CA, 94720}

\author{     P.~Antilogus}
\affiliation{    Laboratoire de Physique Nucl\'eaire et des Hautes \'Energies,
    Universit\'e Pierre et Marie Curie Paris 6, Universit\'e Paris Diderot Paris 7, CNRS-IN2P3, 
    4 place Jussieu, 75252 Paris Cedex 05, France}
    
\author{     S.~Bailey}
\affiliation{    Physics Division, Lawrence Berkeley National Laboratory, 
    1 Cyclotron Road, Berkeley, CA, 94720}

\author{     C.~Baltay}
\affiliation{    Department of Physics, Yale University, 
    New Haven, CT, 06250-8121}

\author{     K.~Barbary}
\affiliation{
    Department of Physics, University of California Berkeley,
    366 LeConte Hall MC 7300, Berkeley, CA, 94720-7300}

\author{    D.~Baugh}
\affiliation{   Tsinghua Center for Astrophysics, Tsinghua University, Beijing 100084, China }

\author{     K.~Boone}
\affiliation{    Physics Division, Lawrence Berkeley National Laboratory, 
    1 Cyclotron Road, Berkeley, CA, 94720}
\affiliation{
    Department of Physics, University of California Berkeley,
    366 LeConte Hall MC 7300, Berkeley, CA, 94720-7300}

\author{     S.~Bongard}
\affiliation{    Laboratoire de Physique Nucl\'eaire et des Hautes \'Energies,
    Universit\'e Pierre et Marie Curie Paris 6, Universit\'e Paris Diderot Paris 7, CNRS-IN2P3, 
    4 place Jussieu, 75252 Paris Cedex 05, France}

\author{     C.~Buton}
\affiliation{    Universit\'e de Lyon, F-69622, Lyon, France ; Universit\'e de Lyon 1, Villeurbanne ; 
    CNRS/IN2P3, Institut de Physique Nucl\'eaire de Lyon}
    
\author{     J.~Chen}
\affiliation{   Tsinghua Center for Astrophysics, Tsinghua University, Beijing 100084, China }

\author{     N.~Chotard}
\affiliation{    Universit\'e de Lyon, F-69622, Lyon, France ; Universit\'e de Lyon 1, Villeurbanne ; 
    CNRS/IN2P3, Institut de Physique Nucl\'eaire de Lyon}
    
\author{     Y.~Copin}
\affiliation{    Universit\'e de Lyon, F-69622, Lyon, France ; Universit\'e de Lyon 1, Villeurbanne ; 
    CNRS/IN2P3, Institut de Physique Nucl\'eaire de Lyon}

\author{ S.~Dixon}
\affiliation{
    Department of Physics, University of California Berkeley,
    366 LeConte Hall MC 7300, Berkeley, CA, 94720-7300}

\author{     P.~Fagrelius}
\affiliation{    Physics Division, Lawrence Berkeley National Laboratory, 
    1 Cyclotron Road, Berkeley, CA, 94720}
\affiliation{
    Department of Physics, University of California Berkeley,
    366 LeConte Hall MC 7300, Berkeley, CA, 94720-7300}

\author{     H.~K.~Fakhouri}
\affiliation{    Physics Division, Lawrence Berkeley National Laboratory, 
    1 Cyclotron Road, Berkeley, CA, 94720}
  \affiliation{
    Department of Physics, University of California Berkeley,
    366 LeConte Hall MC 7300, Berkeley, CA, 94720-7300}

\author{     U.~Feindt}
\affiliation{The Oskar Klein Centre, Department of Physics, AlbaNova, Stockholm University, SE-106 91 Stockholm, Sweden}

\author{     D.~Fouchez}
\affiliation{    Centre de Physique des Particules de Marseille, 
    Aix-Marseille Universit\'e , CNRS/IN2P3, 
    163 avenue de Luminy - Case 902 - 13288 Marseille Cedex 09, France}
    
\author{     E.~Gangler}  
\affiliation{    Clermont Universit\'e, Universit\'e Blaise Pascal, CNRS/IN2P3, Laboratoire de Physique Corpusculaire,
    BP 10448, F-63000 Clermont-Ferrand, France}
    
\author{     B.~Hayden}
\affiliation{    Physics Division, Lawrence Berkeley National Laboratory, 
    1 Cyclotron Road, Berkeley, CA, 94720}

\author{     W.~Hillebrandt}
\affiliation{    Max-Planck-Institut f\"ur Astrophysik, Karl-Schwarzschild-Str. 1,
D-85748 Garching, Germany}

\author{     M.~Kowalski}
\affiliation{    Institut fur Physik,  Humboldt-Universitat zu Berlin,
    Newtonstr. 15, 12489 Berlin}
\affiliation{ DESY, D-15735 Zeuthen, Germany}

\author{     P.-F.~Leget}
\affiliation{    Clermont Universit\'e, Universit\'e Blaise Pascal, CNRS/IN2P3, Laboratoire de Physique Corpusculaire,
    BP 10448, F-63000 Clermont-Ferrand, France}
    
\author{     S.~Lombardo}
\affiliation{    Institut fur Physik,  Humboldt-Universitat zu Berlin,
    Newtonstr. 15, 12489 Berlin}
    
\author{     J.~Nordin}
\affiliation{    Institut fur Physik,  Humboldt-Universitat zu Berlin,
    Newtonstr. 15, 12489 Berlin}
    
\author{     R.~Pain}
\affiliation{    Laboratoire de Physique Nucl\'eaire et des Hautes \'Energies,
    Universit\'e Pierre et Marie Curie Paris 6, Universit\'e Paris Diderot Paris 7, CNRS-IN2P3, 
    4 place Jussieu, 75252 Paris Cedex 05, France}
     
\author{     E.~Pecontal}
\affiliation{   Centre de Recherche Astronomique de Lyon, Universit\'e Lyon 1,
    9 Avenue Charles Andr\'e, 69561 Saint Genis Laval Cedex, France}
    
\author{    R.~Pereira}
 \affiliation{    Universit\'e de Lyon, F-69622, Lyon, France ; Universit\'e de Lyon 1, Villeurbanne ; 
    CNRS/IN2P3, Institut de Physique Nucl\'eaire de Lyon}
 
 \author{    S.~Perlmutter}
 \affiliation{    Physics Division, Lawrence Berkeley National Laboratory, 
    1 Cyclotron Road, Berkeley, CA, 94720} 
\affiliation{
    Department of Physics, University of California Berkeley,
    366 LeConte Hall MC 7300, Berkeley, CA, 94720-7300}
    
 \author{    D.~Rabinowitz}
 \affiliation{    Department of Physics, Yale University, 
    New Haven, CT, 06250-8121}
    
 \author{    M.~Rigault} 
\affiliation{    Institut fur Physik,  Humboldt-Universitat zu Berlin,
    Newtonstr. 15, 12489 Berlin}
     
 \author{    D.~Rubin}
 \affiliation{    Physics Division, Lawrence Berkeley National Laboratory, 
    1 Cyclotron Road, Berkeley, CA, 94720}
    \affiliation{   Space Telescope Science Institute, 3700 San Martin Drive, Baltimore, MD 21218}
 
 \author{    K.~Runge}
 \affiliation{    Physics Division, Lawrence Berkeley National Laboratory, 
    1 Cyclotron Road, Berkeley, CA, 94720}
 
 \author{    C.~Saunders}
 \affiliation{    Physics Division, Lawrence Berkeley National Laboratory, 
    1 Cyclotron Road, Berkeley, CA, 94720}

\author{    C.~Sofiatti}
\affiliation{    Physics Division, Lawrence Berkeley National Laboratory, 
    1 Cyclotron Road, Berkeley, CA, 94720} 
\affiliation{
    Department of Physics, University of California Berkeley,
    366 LeConte Hall MC 7300, Berkeley, CA, 94720-7300}

\author{    N.~Suzuki}
\affiliation{    Physics Division, Lawrence Berkeley National Laboratory, 
    1 Cyclotron Road, Berkeley, CA, 94720}

\author{     S.~Taubenberger}
\affiliation{    Max-Planck-Institut f\"ur Astrophysik, Karl-Schwarzschild-Str. 1,
D-85748 Garching, Germany}

\author{     C.~Tao}
\affiliation{   Tsinghua Center for Astrophysics, Tsinghua University, Beijing 100084, China }
\affiliation{    Centre de Physique des Particules de Marseille, 
    Aix-Marseille Universit\'e , CNRS/IN2P3, 
    163 avenue de Luminy - Case 902 - 13288 Marseille Cedex 09, France}
   
\author{     R.~C.~Thomas}
\affiliation{    Computational Cosmology Center, Computational Research Division, Lawrence Berkeley National Laboratory, 
    1 Cyclotron Road MS 50B-4206, Berkeley, CA, 94720}
    
\collaboration{(The Nearby Supernova Factory)}


\begin{abstract}
An empirical model for SN~Ia peak magnitudes with two color parameters and dependence on the equivalent widths of CaII, SiII, and SiII velocity
is applied to the supernova sample of the Nearby Supernova Factory.  The peak magnitudes in synthetic
broadband photometry and their colors are found to be 
dependent on the spectral equivalent widths and the color parameters, to better than 
%-----
$1-5\times10^{-5}$
%------
confidence.
The derived dust curve is able to reproduce the optical colors of the highly reddened SN~2014J better than the \citet{1999PASP..111...63F} dust model.
Extending the model to three color parameters gives almost identical results for
all the parameters in common,
plus a detection of a third parameter at
%--- table.py
$1-5\times10^{-4}$
%-----
confidence;
this new parameter represents supernova color diversity
beyond that tracked by the spectral features and two-parameter dust models.
\end{abstract}

\keywords{supernovae: general; supernovae: SN 2014J}

\section{Introduction}
Type~Ia supernovae (SNe~Ia) form a homogenous set of exploding stars and as such were early recognized and utilized as a powerful distance indicator 
and probe of cosmology \citep[e.g.][]{1992ARA&A..30..359B, 1993ApJ...415....1S}.  After further careful consideration of supernova data, it was recognized
that SN~Ia light-curve shapes \citep{1993ApJ...413L.105P} and colors \citep{1996ApJ...473...88R, 1998A&A...331..815T} exhibit subtle signs of heterogeneity
that are correlated with absolute magnitude, and must be considered when inferring distances.  Empirical models parameterizing SNe~Ia by their light-curve shape \citep{1996ApJ...473...88R,
1997ApJ...483..565P,
1999ApJ...517..565P}
and color  \citep{1996ApJ...473...88R}  were developed that enabled absolute magnitude corrections
and accurate distance measurements of cosmological supernovae,
which 
were subsequently used in the discovery of the accelerating expansion of the Universe \citep{1998AJ....116.1009R,1999ApJ...517..565P}.

The two most commonly used supernova-cosmology light-curve fitters today are SALT2 \citep{2007A&A...466...11G} and MLCS2k2
\citep{2007ApJ...659..122J}.\footnote{Light-curve fitters with more flexible degrees of freedom
\citep[e.g.][]{2008ApJ...681..482C, 2011AJ....141...19B, 2011ApJ...731..120M} are available and have for
the most part been used to study SN~Ia heterogeneity.}
They remain two-parameter models, with one parameter characterizing light-curve shape and the other
 color.
In SALT2 the light curve shapes are described by a multiplicative scaling (``stretch'')  of the time evolution,
 whereas MLCS2k2 varies shapes through additive time-dependent corrections to magnitudes.
The physical cause of the color diversity is interpreted differently by the two sets of authors: 
\citet{2007A&A...466...11G} pragmatically extract color variation empirically from SNe that span a wide range of colors, with no attribution
to either intrinsic or extrinsic origins;
\citet{2007ApJ...659..122J}
attribute changes in color
partially to intrinsic variations linked to light-curve shape, and partially
to the reddening of light from host-galaxy dust.  Differences between these models produce differences in the results of
analysis of both low-redshift \citep{2007ApJ...664L..13C} and high-redshift \citep{2009ApJS..185...32K} supernovae.

There is evidence that supports the expectation that a single parameter beyond light-curve shape  cannot describe the full range
of colors seen in the SN~Ia population.  One approach to look for color diversity is to find correlations between color and spectral features.
\citet{2009ApJ...699L.139W, 2011ApJ...729...55F} find two subpopulations distinguished
by SiII velocity with differing $B_{max}-V_{max}$; this color correlation, in addition to one with $B-R$, is confirmed by
\citet{2014ApJ...797...75M}.
\citet{2015MNRAS.451.1973S}
find that high-velocity SiII$\lambda$6355 is found in objects that have red ultraviolet/optical colors near maximum brightness.
\citet{2011MNRAS.413.3075M} show evidence that supernova asymmetry and viewing angle,
traced by wavelength shifts in nebular emission lines, is an important determinant in controlling supernova color; such correlations are also seen by \citet{2011A&A...534L..15C}.

Another approach to probe color diversity is through multiple colors (at least 3 bands)
of individual supernovae.  Color ratios are sensitive to processes of the responsible physics.   For example,
relative dust absorption varies as a function of wavelength depending on grain size and shape,
independent (to first order) of the amount of dust along the line of sight.
\citet{2013ApJ...779...23M} find diversity in SNe~Ia UV emission, which they use to distinguish subclasses based on UV-optical colors.
Measurements of color ratios are being advanced with the development of flexible empirical light curve models that accommodate flexibility in multi-band colors
\citep[e.g.][]{2011ApJ...731..120M}.
\citet{2014ApJ...789...32B, 2015MNRAS.453.3300A} find wide
ranges of total-to-selective extinction with average values significantly lower than $R_V = 3.1$,
the canonical value for diffuse Milky Way dust.
They also confirm the \citet{2011ApJ...731..120M, 2011ApJ...729...55F} finding that low $R_V$ is associated with high-extinction supernovae.
In contrast, \citet{2011A&A...529L...4C} argue that after accounting for the diversity of spectral features,
the $R_V=3.1$ measured for the diffuse Milky Way dust is recovered on average and \citet{2017arXiv170101422H}
find $R_V=2.95 \pm 0.08$ for SN~2012cu.

Hierarchical modeling has recently opened
the study of intrinsic supernova color based on SN~Ia Hubble diagrams. Latent parameters that are not directly tied to observables,
such as multiple parameters that simultaneously influence color, can be included in the model.
\citet{2016arXiv160904470M} take the approach of modeling the distribution of their parameters, to find that
scatter in the Hubble diagram is better explained by a combination of intrinsic color dispersion and
$R_B \equiv R_V+1=3.7$ dust, rather than by dust with no color dispersion.
They draw these conclusions by using only the SALT2 (v2.4) $x_1$ parameter as the summary statistic that describes color.

The Nearby Supernova Factory \citep[SNfactory;][]{2002SPIE.4836...61A} has systematically observed the
spectrophotometric time series of hundreds of Hubble-flow $0.03<z<0.08$ SNe~Ia.   The $3200$--$10000$~\AA\ spectral coverage
provides measurements of an array of supernova spectral features while also providing synthetic broadband photometry
spanning near-UV to near-IR SN-frame wavelengths.  SNfactory specifically targeted objects
early in their temporal evolution, so that well over a hundred of these supernovae have  coverage over
peak brightness.  This dataset provides a homogenous sample with which to study SN~Ia colors and spectral features together.

In this article we use the idea that spectral indicators carry information on intrinsic supernova colors at peak magnitude.
\color{red}
This approach is taken by \citet{2011A&A...529L...4C}, who find that after standardization based on Ca and Si features, remaining residual color
variation is consistent with Milky Way dust models.
We accommodate  three independent color parameters: the model is constructed such that 
two are attributed to extrinsic processes attributed to dust, and one is attributed to intrinsic supernova
properties.
\color{black}
The data used in this analysis are described in \S\ref{data:sec}.
In \S\ref{model:sec} we present a
first analysis with two extrinsic parameters and omitting the intrinsic parameter.  The properties of the two extrinsic parameters
are consistent with the expectations of parameterized dust designed for the Milky Way, and do an excellent job of fitting the data of the out-of-sample
supernova SN2014J that has an extreme total-to-selective extinction.
In \S\ref{model2:sec} we add the intrinsic parameter in the analysis, and find strong evidence for its influence on observed magnitudes.
We summarize our findings, and relate our results with light-curve shape, SiII velocity, and the host-galaxy global stellar ``mass step'' in SN~Ia Hubble
residuals in \S\ref{discussion:sec}.


\section{Data}
\label{data:sec}

Our analysis uses the spectrophotometric data set obtained by
the SNfactory with the SuperNova Integral Field
Spectrograph \citep[SNIFS,][]{2002SPIE.4836...61A, 2004SPIE.5249..146L}.  SNIFS is a fully integrated
instrument optimized for automated observation of point sources on a
structured background over the full ground-based optical window at
moderate spectral resolution ($R \sim 500$).  It consists of a
high-throughput wide-band lenslet integral field spectrograph, a mult-band
imager that covers the field in the vicinity of
the IFS for atmospheric transmission monitoring simultaneous with
spectroscopy, and an acquisition/guiding channel.  The IFS possesses a
fully-filled $6\farcs 4 \times 6\farcs 4$ spectroscopic field of view
subdivided into a grid of $15 \times 15$ spatial elements, a
dual-channel spectrograph covering 3200--5200~\AA\ and 5100--10000~\AA\
simultaneously, and an internal calibration unit (continuum and arc
lamps).  SNIFS is mounted on the south bent Cassegrain port of the
University of Hawaii 2.2~m telescope on Mauna Kea, and is operated
remotely.  Observations are reduced using the SNfactory's dedicated data
reduction pipeline, similar to that presented in \S4 of \citet{2001MNRAS.326...23B}.
A discussion of the software pipeline is presented in
\citet{2006ApJ...650..510A} and is updated in \citet{2010ApJ...713.1073S}. 
The flux calibration is presented in \citet{2013A&A...549A...8B}.
A detailed
description of host-galaxy subtraction is given in \citet{2011MNRAS.418..258B}.

A recent description of the data is presented in \citet{2015ApJ...815...58F}.
We provide a brief summary of the points important for this analysis.
The spectral time-series  are corrected for Milky Way dust
extinction \citep{1989ApJ...345..245C,1998ApJ...500..525S}.  
Each spectral time series is
blue-shifted to rest-frame
based on the systemic redshift of the host \citep[c.f.][]{2013ApJ...770..107C}, and the fluxes are converted to luminosity assuming
distances expected for the supernova redshifts given a flat
$\Lambda$CDM cosmology with $\Omega_M = 0.28$ (with an arbitrarily selected
$H_0$ since the current analysis does not depend on the absolute magnitude scale).

Synthetic supernova-frame photometry is generated for a top-hat filter system
comprised of five 
bands with the following wavelength ranges: $U$ $[3300.00 - 3978.02]$\AA;
$B$ $[3978.02-4795.35]$\AA;
$V$ $[4795.35-5780.60]$\AA;
$R$ $[5780.60-6968.29]$\AA;
$I$ $[6968.29-8400.00]$\AA.
For each supernova, the magnitudes within 5-days of peak brightness are used to regress single-band magnitudes
at $B$-band peak brightness.
The equivalent widths of SiII$\lambda 4141$ and the CaII H\&K features are computed as
in \citet{2008A&A...477..717B} and the 
wavelength of the SiII$\lambda 6355$ feature
as in \citet{chotard:thesis, 2017Chotard}.
Equivalent widths and the
SiII$\lambda 6355$ wavelength are taken from spectra  within $\pm 2.5$ days from $B$-band maximum;
the average is used  in cases where there are multiple spectral measurements within that time window.

Our analysis sample is comprised by the
172
supernovae that have the data coverage to 
give photometric and spectroscopic statistics described above.
\textcolor{red}{FOR INTERNAL REFERENCE: The CABALLO2 validation and training sets are used with SN2012cu, SNF20061108-001, SNF20080905-005, SNNGC7589 excised
based on data problems identified by Manu.}
The data are given in Table~\ref{data:tab} \textcolor{red}{[TABLE CAN GO IF THERE IS A NICO PAPER]}.
The peak magnitude uncertainties do have covariance, which is accounted
for in the analysis; only the standard deviation is included in the table.

\startlongtable
\begin{deluxetable}{crrrrrrrr}
\tabletypesize{\tiny}
\tablecaption{Supernova Spectral-Feature and Peak-Magnitude Data
\label{data:tab}}
\tablehead{
\colhead{Name} & \colhead{$EW_{Ca}$ (\AA)} & \colhead{$EW_{Si}$ (\AA)} & \colhead{$\lambda_{Si}$ (\AA)} & \colhead{$U$} & \colhead{$B$} & \colhead{$V$} & \colhead{$R$} & \colhead{$I$}
}
\startdata
SN2007bd & $109.7 \pm 5.9$ & $ 17.5 \pm 0.7$& $ 6101 \pm   3$ & $-29.31 \pm   0.01$ & $-29.12 \pm   0.01$& $-28.60 \pm   0.01$& $-28.35 \pm   0.01$& $-27.60 \pm   0.01$ \\
PTF10zdk & $149.7 \pm 1.2$ & $ 14.3 \pm 0.6$& $ 6150 \pm   3$ & $-28.61 \pm   0.02$ & $-28.69 \pm   0.02$& $-28.32 \pm   0.02$& $-28.08 \pm   0.02$& $-27.40 \pm   0.02$ \\
SNF20080815-017 & $ 63.8 \pm 21.5$ & $ 27.6 \pm 3.8$& $ 6132 \pm   6$ & $-29.04 \pm   0.07$ & $-28.79 \pm   0.07$& $-28.32 \pm   0.07$& $-28.12 \pm   0.07$& $-27.41 \pm   0.07$ \\
PTF09dnl & $129.9 \pm 0.9$ & $  9.5 \pm 0.7$& $ 6093 \pm   3$ & $-29.23 \pm   0.01$ & $-29.07 \pm   0.01$& $-28.72 \pm   0.01$& $-28.44 \pm   0.01$& $-27.69 \pm   0.01$ \\
SN2010ex & $114.4 \pm 0.9$ & $  8.4 \pm 0.4$& $ 6129 \pm   6$ & $-29.26 \pm   0.01$ & $-28.99 \pm   0.01$& $-28.50 \pm   0.01$& $-28.20 \pm   0.01$& $-27.44 \pm   0.01$ \\
PTF09dnp & $ 64.9 \pm 4.5$ & $ 16.5 \pm 0.7$& $ 6098 \pm   4$ & $-29.55 \pm   0.02$ & $-29.19 \pm   0.02$& $-28.68 \pm   0.02$& $-28.48 \pm   0.02$& $-27.93 \pm   0.02$ \\
PTF11bnx & $151.4 \pm 3.0$ & $ 13.9 \pm 1.1$& $ 6142 \pm   5$ & $-28.63 \pm   0.02$ & $-28.57 \pm   0.01$& $-28.20 \pm   0.01$& $-27.99 \pm   0.01$& $-27.34 \pm   0.01$ \\
PTF12jqh & $151.9 \pm 1.5$ & $  7.9 \pm 0.7$& $ 6116 \pm  10$ & $-29.37 \pm   0.01$ & $-29.14 \pm   0.01$& $-28.71 \pm   0.01$& $-28.40 \pm   0.01$& $-27.64 \pm   0.01$ \\
SNF20080802-006 & $108.2 \pm 6.0$ & $ 20.6 \pm 1.9$& $ 6122 \pm   5$ & $-29.02 \pm   0.06$ & $-28.80 \pm   0.06$& $-28.40 \pm   0.06$& $-28.20 \pm   0.06$& $-27.50 \pm   0.06$ \\
PTF10xyt & $123.7 \pm 6.6$ & $ 16.4 \pm 4.3$& $ 6101 \pm   4$ & $-28.26 \pm   0.02$ & $-28.20 \pm   0.02$& $-27.93 \pm   0.02$& $-27.74 \pm   0.02$& $-27.22 \pm   0.04$ \\
PTF11qmo & $101.7 \pm 1.1$ & $  7.7 \pm 0.7$& $ 6150 \pm   8$ & $-29.77 \pm   0.02$ & $-29.43 \pm   0.02$& $-28.97 \pm   0.02$& $-28.64 \pm   0.02$& $-27.93 \pm   0.02$ \\
SNF20070331-025 & $119.8 \pm 7.4$ & $ 14.2 \pm 2.7$& $ 6120 \pm  10$ & $-28.94 \pm   0.02$ & $-28.75 \pm   0.02$& $-28.32 \pm   0.02$& $-28.07 \pm   0.02$& $-27.31 \pm   0.02$ \\
SNF20070818-001 & $157.5 \pm 7.5$ & $ 16.7 \pm 1.8$& $ 6115 \pm   5$ & $-28.97 \pm   0.02$ & $-28.96 \pm   0.01$& $-28.61 \pm   0.01$& $-28.37 \pm   0.01$& $-27.62 \pm   0.01$ \\
SNBOSS38 & $ 57.1 \pm 0.4$ & $ 17.9 \pm 0.3$& $ 6127 \pm   3$ & $-29.20 \pm   0.01$ & $-28.84 \pm   0.01$& $-28.47 \pm   0.01$& $-28.23 \pm   0.01$& $-27.73 \pm   0.04$ \\
SN2006ob & $ 90.0 \pm 16.5$ & $ 26.5 \pm 1.5$& $ 6112 \pm   5$ & $-29.11 \pm   0.02$ & $-28.82 \pm   0.01$& $-28.42 \pm   0.01$& $-28.19 \pm   0.01$& $-27.54 \pm   0.01$ \\
PTF12eer & $165.6 \pm 10.7$ & $ 12.7 \pm 2.8$& $ 6150 \pm  10$ & $-28.76 \pm   0.01$ & $-28.76 \pm   0.01$& $-28.40 \pm   0.01$& $-28.17 \pm   0.01$& $-27.45 \pm   0.02$ \\
PTF10ops & $ 38.7 \pm 9.9$ & $  7.2 \pm 8.7$& $ 6140 \pm   5$ & $-27.93 \pm   0.38$ & $-27.76 \pm   0.38$& $-27.73 \pm   0.38$& $-27.59 \pm   0.38$& $-27.21 \pm   0.38$ \\
SNF20080514-002 & $ 83.2 \pm 0.7$ & $ 19.4 \pm 0.6$& $ 6131 \pm   3$ & $-29.30 \pm   0.01$ & $-28.95 \pm   0.01$& $-28.44 \pm   0.01$& $-28.17 \pm   0.01$& $-27.49 \pm   0.01$ \\
PTF12evo & $129.2 \pm 2.8$ & $  9.1 \pm 1.3$& $ 6156 \pm   4$ & $-29.14 \pm   0.02$ & $-28.98 \pm   0.01$& $-28.56 \pm   0.01$& $-28.28 \pm   0.01$& $-27.61 \pm   0.01$ \\
SNF20080614-010 & $125.4 \pm 5.1$ & $ 26.9 \pm 1.6$& $ 6128 \pm   3$ & $-29.04 \pm   0.04$ & $-28.81 \pm   0.04$& $-28.38 \pm   0.04$& $-28.16 \pm   0.04$& $-27.57 \pm   0.04$ \\
PTF10icb & $104.8 \pm 0.9$ & $ 12.7 \pm 0.3$& $ 6138 \pm   3$ & $-28.58 \pm   0.02$ & $-28.36 \pm   0.02$& $-27.98 \pm   0.02$& $-27.77 \pm   0.02$& $-27.17 \pm   0.02$ \\
PTF12efn & $144.9 \pm 3.4$ & $  7.1 \pm 1.8$& $ 6115 \pm   3$ & $-29.40 \pm   0.01$ & $-29.17 \pm   0.01$& $-28.79 \pm   0.01$& $-28.45 \pm   0.01$& $-27.64 \pm   0.01$ \\
SNNGC4424 & $109.0 \pm 0.3$ & $  8.6 \pm 0.1$& $ 6138 \pm   2$ & $-28.35 \pm   0.01$ & $-28.15 \pm   0.01$& $-27.79 \pm   0.01$& $-27.58 \pm   0.01$& $-26.97 \pm   0.01$ \\
SNF20080516-022 & $100.1 \pm 2.1$ & $ 13.7 \pm 1.1$& $ 6158 \pm   3$ & $-29.46 \pm   0.01$ & $-29.19 \pm   0.01$& $-28.71 \pm   0.01$& $-28.39 \pm   0.01$& $-27.77 \pm   0.01$ \\
PTF12hwb & $ 21.1 \pm 78.0$ & $ -1.8 \pm 8.9$& $ 6090 \pm  14$ & $-28.32 \pm   0.02$ & $-28.24 \pm   0.02$& $-28.03 \pm   0.02$& $-27.79 \pm   0.02$& $-27.05 \pm   0.04$ \\
PTF10qyz & $106.4 \pm 2.1$ & $ 23.0 \pm 1.0$& $ 6120 \pm   5$ & $-29.05 \pm   0.17$ & $-28.92 \pm   0.17$& $-28.41 \pm   0.17$& $-28.14 \pm   0.17$& $-27.30 \pm   0.17$ \\
SNF20060907-000 & $106.1 \pm 10.4$ & $ 17.0 \pm 0.9$& $ 6149 \pm   4$ & $-29.54 \pm   0.02$ & $-29.28 \pm   0.01$& $-28.76 \pm   0.01$& $-28.42 \pm   0.01$& $-27.74 \pm   0.04$ \\
LSQ12fxd & $122.9 \pm 1.7$ & $ 11.4 \pm 0.8$& $ 6119 \pm   4$ & $-29.62 \pm   0.07$ & $-29.39 \pm   0.07$& $-28.95 \pm   0.07$& $-28.64 \pm   0.07$& $-27.91 \pm   0.07$ \\
SNF20080821-000 & $105.1 \pm 2.2$ & $  8.6 \pm 1.3$& $ 6121 \pm   4$ & $-29.34 \pm   0.01$ & $-29.10 \pm   0.01$& $-28.73 \pm   0.01$& $-28.46 \pm   0.01$& $-27.82 \pm   0.01$ \\
SNF20070802-000 & $158.3 \pm 3.3$ & $ 16.3 \pm 1.7$& $ 6102 \pm   5$ & $-28.90 \pm   0.01$ & $-28.81 \pm   0.01$& $-28.45 \pm   0.01$& $-28.22 \pm   0.01$& $-27.52 \pm   0.01$ \\
PTF10wnm & $105.8 \pm 2.3$ & $  6.5 \pm 1.0$& $ 6124 \pm   3$ & $-29.38 \pm   0.01$ & $-29.07 \pm   0.01$& $-28.68 \pm   0.01$& $-28.37 \pm   0.01$& $-27.69 \pm   0.01$ \\
PTF10mwb & $116.5 \pm 1.2$ & $ 19.8 \pm 0.9$& $ 6138 \pm   2$ & $-29.02 \pm   0.07$ & $-28.84 \pm   0.07$& $-28.40 \pm   0.07$& $-28.14 \pm   0.07$& $-27.52 \pm   0.07$ \\
SN2010dt & $116.2 \pm 14.9$ & $ 15.5 \pm 0.7$& $ 6138 \pm   6$ & $-29.30 \pm   0.01$ & $-29.15 \pm   0.01$& $-28.64 \pm   0.01$& $-28.35 \pm   0.01$& $-27.63 \pm   0.01$ \\
SNF20080623-001 & $149.1 \pm 1.4$ & $ 14.9 \pm 0.7$& $ 6131 \pm   3$ & $-29.11 \pm   0.01$ & $-28.97 \pm   0.01$& $-28.50 \pm   0.01$& $-28.22 \pm   0.01$& $-27.46 \pm   0.01$ \\
LSQ12fhe & $ 42.8 \pm 1.2$ & $  4.0 \pm 3.1$& $ 6108 \pm   4$ & $-29.76 \pm   0.02$ & $-29.40 \pm   0.02$& $-29.04 \pm   0.02$& $-28.74 \pm   0.02$& $-28.11 \pm   0.02$ \\
PTF11bju & $ 30.2 \pm 4.4$ & $  4.0 \pm 3.0$& $ 6139 \pm   5$ & $-29.47 \pm   0.02$ & $-29.10 \pm   0.01$& $-28.75 \pm   0.01$& $-28.45 \pm   0.01$& $-27.87 \pm   0.01$ \\
PTF09fox & $117.6 \pm 2.7$ & $  9.1 \pm 1.0$& $ 6116 \pm   3$ & $-29.44 \pm   0.03$ & $-29.21 \pm   0.03$& $-28.72 \pm   0.03$& $-28.42 \pm   0.03$& $-27.68 \pm   0.03$ \\
PTF13ayw & $104.6 \pm 2.4$ & $ 26.6 \pm 3.2$& $ 6115 \pm   6$ & $-29.16 \pm   0.02$ & $-28.82 \pm   0.02$& $-28.43 \pm   0.02$& $-28.20 \pm   0.02$& $-27.55 \pm   0.02$ \\
SNF20070810-004 & $126.7 \pm 1.8$ & $ 21.1 \pm 1.1$& $ 6118 \pm   7$ & $-29.22 \pm   0.01$ & $-29.10 \pm   0.01$& $-28.63 \pm   0.01$& $-28.34 \pm   0.01$& $-27.62 \pm   0.01$ \\
PTF11mty & $111.4 \pm 2.3$ & $ 10.6 \pm 1.5$& $ 6138 \pm   5$ & $-29.54 \pm   0.01$ & $-29.23 \pm   0.01$& $-28.80 \pm   0.01$& $-28.46 \pm   0.01$& $-27.82 \pm   0.01$ \\
SNF20080512-010 & $ 95.3 \pm 3.5$ & $ 23.3 \pm 1.5$& $ 6129 \pm   5$ & $-29.22 \pm   0.08$ & $-28.96 \pm   0.08$& $-28.50 \pm   0.08$& $-28.26 \pm   0.08$& $-27.56 \pm   0.08$ \\
PTF11mkx & $ 31.5 \pm 3.7$ & $  4.5 \pm 1.3$& $ 6169 \pm   5$ & $-29.50 \pm   0.45$ & $-29.25 \pm   0.45$& $-28.89 \pm   0.45$& $-28.61 \pm   0.45$& $-27.97 \pm   0.45$ \\
PTF10tce & $135.7 \pm 1.1$ & $ 11.2 \pm 1.5$& $ 6090 \pm   4$ & $-29.13 \pm   0.02$ & $-28.99 \pm   0.01$& $-28.59 \pm   0.01$& $-28.31 \pm   0.01$& $-27.55 \pm   0.01$ \\
SNF20061020-000 & $ 95.4 \pm 18.8$ & $ 24.1 \pm 1.0$& $ 6120 \pm   5$ & $-29.01 \pm   0.03$ & $-28.78 \pm   0.03$& $-28.35 \pm   0.03$& $-28.17 \pm   0.03$& $-27.54 \pm   0.03$ \\
SN2005ir & $115.6 \pm 2.8$ & $ 13.5 \pm 6.9$& $ 6069 \pm   5$ & $-29.33 \pm   0.02$ & $-29.12 \pm   0.02$& $-28.84 \pm   0.02$& $-28.49 \pm   0.02$& $-27.77 \pm   0.02$ \\
SNF20080717-000 & $ 93.3 \pm 2.6$ & $  8.3 \pm 2.2$& $ 6104 \pm   3$ & $-28.58 \pm   0.01$ & $-28.47 \pm   0.01$& $-28.29 \pm   0.01$& $-28.05 \pm   0.01$& $-27.50 \pm   0.01$ \\
PTF12ena & $101.1 \pm 1.6$ & $  7.4 \pm 1.0$& $ 6129 \pm   4$ & $-28.01 \pm   0.01$ & $-28.00 \pm   0.01$& $-27.85 \pm   0.01$& $-27.77 \pm   0.01$& $-27.31 \pm   0.01$ \\
PTF13anh & $166.8 \pm 1.8$ & $ 21.8 \pm 1.2$& $ 6175 \pm   4$ & $-28.67 \pm   0.20$ & $-28.74 \pm   0.20$& $-28.30 \pm   0.20$& $-28.05 \pm   0.20$& $-27.28 \pm   0.20$ \\
CSS110918\_01 & $110.6 \pm 1.0$ & $  8.0 \pm 1.3$& $ 6101 \pm   2$ & $-29.88 \pm   0.76$ & $-29.58 \pm   0.76$& $-29.09 \pm   0.76$& $-28.71 \pm   0.76$& $-27.91 \pm   0.76$ \\
SNF20061024-000 & $ 86.9 \pm 26.8$ & $ 30.0 \pm 1.5$& $ 6127 \pm   5$ & $-28.88 \pm   0.04$ & $-28.70 \pm   0.04$& $-28.26 \pm   0.04$& $-28.05 \pm   0.04$& $-27.40 \pm   0.04$ \\
SNF20070506-006 & $ 94.1 \pm 1.3$ & $  6.7 \pm 0.6$& $ 6153 \pm   3$ & $-29.72 \pm   0.01$ & $-29.39 \pm   0.01$& $-28.97 \pm   0.01$& $-28.64 \pm   0.01$& $-27.96 \pm   0.01$ \\
SNF20070403-001 & $105.9 \pm 5.4$ & $ 18.3 \pm 1.8$& $ 6124 \pm   4$ & $-29.23 \pm   0.02$ & $-29.04 \pm   0.01$& $-28.63 \pm   0.01$& $-28.35 \pm   0.01$& $-27.62 \pm   0.01$ \\
PTF10hmv & $109.6 \pm 1.3$ & $  8.9 \pm 0.7$& $ 6143 \pm   3$ & $-28.54 \pm   0.01$ & $-28.40 \pm   0.01$& $-28.11 \pm   0.01$& $-27.89 \pm   0.01$& $-27.31 \pm   0.01$ \\
SNF20071015-000 & $105.0 \pm 3.2$ & $  6.9 \pm 1.1$& $ 6124 \pm   7$ & $-27.89 \pm   0.02$ & $-27.82 \pm   0.02$& $-27.69 \pm   0.02$& $-27.63 \pm   0.02$& $-27.16 \pm   0.04$ \\
SNhunt89 & $ 88.0 \pm 2.7$ & $ 32.2 \pm 1.9$& $ 6111 \pm   7$ & $-28.37 \pm   0.03$ & $-28.26 \pm   0.03$& $-27.92 \pm   0.03$& $-27.77 \pm   0.03$& $-27.13 \pm   0.03$ \\
SNF20070902-021 & $108.9 \pm 3.5$ & $ 17.1 \pm 1.0$& $ 6131 \pm   6$ & $-29.25 \pm   0.02$ & $-29.02 \pm   0.02$& $-28.56 \pm   0.02$& $-28.32 \pm   0.01$& $-27.65 \pm   0.02$ \\
PTF09dlc & $143.5 \pm 2.2$ & $ 10.2 \pm 0.9$& $ 6143 \pm   3$ & $-29.38 \pm   0.01$ & $-29.17 \pm   0.01$& $-28.69 \pm   0.01$& $-28.40 \pm   0.01$& $-27.62 \pm   0.01$ \\
PTF13ajv & $150.5 \pm 8.9$ & $ 46.3 \pm 8.6$& $ 6110 \pm  21$ & $-28.70 \pm   0.02$ & $-28.61 \pm   0.02$& $-28.16 \pm   0.02$& $-27.91 \pm   0.02$& $-27.07 \pm   0.04$ \\
SNF20080919-000 & $114.7 \pm 2.8$ & $  9.4 \pm 0.9$& $ 6145 \pm   5$ & $-28.53 \pm   0.02$ & $-28.41 \pm   0.01$& $-28.11 \pm   0.01$& $-27.99 \pm   0.01$& $-27.38 \pm   0.01$ \\
SNF20080919-001 & $ 85.0 \pm 1.1$ & $  6.0 \pm 0.4$& $ 6150 \pm   5$ & $-29.73 \pm   0.01$ & $-29.43 \pm   0.01$& $-29.04 \pm   0.01$& $-28.72 \pm   0.01$& $-28.07 \pm   0.01$ \\
SN2010kg & $ 95.1 \pm 28.5$ & $ 21.7 \pm 0.7$& $ 6077 \pm   5$ & $-28.85 \pm   0.01$ & $-28.74 \pm   0.01$& $-28.41 \pm   0.01$& $-28.20 \pm   0.01$& $-27.47 \pm   0.01$ \\
SNF20080714-008 & $134.8 \pm 15.7$ & $ 19.7 \pm 3.7$& $ 6100 \pm   6$ & $-28.56 \pm   0.02$ & $-28.63 \pm   0.01$& $-28.32 \pm   0.01$& $-28.13 \pm   0.01$& $-27.42 \pm   0.01$ \\
SNF20070714-007 & $129.6 \pm 5.6$ & $ 31.1 \pm 23.8$& $ 6146 \pm   5$ & $-27.88 \pm   0.02$ & $-28.12 \pm   0.01$& $-28.02 \pm   0.01$& $-27.86 \pm   0.01$& $-27.24 \pm   0.03$ \\
SNF20080522-011 & $122.1 \pm 1.7$ & $  8.3 \pm 0.5$& $ 6125 \pm   3$ & $-29.63 \pm   0.01$ & $-29.38 \pm   0.01$& $-28.92 \pm   0.01$& $-28.60 \pm   0.01$& $-27.88 \pm   0.01$ \\
SNF20061111-002 & $110.8 \pm 10.7$ & $ 20.4 \pm 1.0$& $ 6145 \pm   6$ & $-29.16 \pm   0.01$ & $-28.99 \pm   0.01$& $-28.59 \pm   0.01$& $-28.29 \pm   0.01$& $-27.61 \pm   0.01$ \\
SNNGC6343 & $ 87.0 \pm 1.4$ & $ 20.7 \pm 0.7$& $ 6136 \pm   3$ & $-28.78 \pm   0.01$ & $-28.66 \pm   0.01$& $-28.30 \pm   0.01$& $-28.08 \pm   0.01$& $-27.41 \pm   0.01$ \\
SNF20061011-005 & $120.6 \pm 1.1$ & $  9.3 \pm 0.4$& $ 6132 \pm   4$ & $-29.72 \pm   0.04$ & $-29.43 \pm   0.03$& $-28.99 \pm   0.03$& $-28.64 \pm   0.03$& $-27.90 \pm   0.03$ \\
SNF20080825-010 & $102.4 \pm 13.4$ & $ 19.2 \pm 0.6$& $ 6116 \pm   4$ & $-29.46 \pm   0.01$ & $-29.17 \pm   0.01$& $-28.71 \pm   0.01$& $-28.47 \pm   0.01$& $-27.83 \pm   0.01$ \\
PTF10ufj & $141.1 \pm 3.4$ & $ 11.7 \pm 1.2$& $ 6131 \pm   6$ & $-29.28 \pm   0.15$ & $-29.16 \pm   0.15$& $-28.72 \pm   0.15$& $-28.41 \pm   0.15$& $-27.65 \pm   0.15$ \\
PTF10wof & $129.6 \pm 2.7$ & $ 17.3 \pm 1.0$& $ 6102 \pm   2$ & $-28.91 \pm   0.01$ & $-28.84 \pm   0.01$& $-28.46 \pm   0.01$& $-28.18 \pm   0.01$& $-27.43 \pm   0.01$ \\
SNF20080918-000 & $146.8 \pm 3.5$ & $  7.5 \pm 2.5$& $ 6110 \pm   5$ & $-28.79 \pm   0.02$ & $-28.65 \pm   0.02$& $-28.35 \pm   0.02$& $-28.12 \pm   0.02$& $-27.46 \pm   0.02$ \\
SNF20080516-000 & $117.4 \pm 2.2$ & $  9.0 \pm 1.2$& $ 6135 \pm   3$ & $-29.50 \pm   0.01$ & $-29.23 \pm   0.01$& $-28.80 \pm   0.01$& $-28.47 \pm   0.01$& $-27.74 \pm   0.01$ \\
SN2005cf & $159.1 \pm 0.7$ & $ 15.7 \pm 0.8$& $ 6141 \pm   3$ & $-29.37 \pm   0.02$ & $-29.16 \pm   0.02$& $-28.68 \pm   0.02$& $-28.41 \pm   0.02$& $-27.69 \pm   0.02$ \\
CSS130502\_01 & $ 91.5 \pm 10.9$ & $ 15.6 \pm 0.5$& $ 6128 \pm   3$ & $-29.43 \pm   0.02$ & $-29.09 \pm   0.02$& $-28.60 \pm   0.01$& $-28.30 \pm   0.01$& $-27.62 \pm   0.04$ \\
SNF20080620-000 & $107.8 \pm 14.1$ & $ 20.0 \pm 0.7$& $ 6132 \pm   3$ & $-28.82 \pm   0.02$ & $-28.78 \pm   0.01$& $-28.32 \pm   0.01$& $-28.09 \pm   0.01$& $-27.39 \pm   0.01$ \\
SNPGC51271 & $ 92.1 \pm 16.5$ & $ 21.1 \pm 0.7$& $ 6121 \pm   2$ & $-29.28 \pm   0.02$ & $-28.95 \pm   0.02$& $-28.46 \pm   0.02$& $-28.20 \pm   0.02$& $-27.62 \pm   0.04$ \\
PTF11pdk & $128.6 \pm 2.8$ & $ 15.6 \pm 1.7$& $ 6153 \pm   5$ & $-29.35 \pm   0.02$ & $-29.11 \pm   0.02$& $-28.61 \pm   0.02$& $-28.32 \pm   0.02$& $-27.67 \pm   0.02$ \\
SNF20060511-014 & $102.6 \pm 2.8$ & $ 15.6 \pm 1.1$& $ 6141 \pm   8$ & $-29.16 \pm   0.07$ & $-29.04 \pm   0.06$& $-28.56 \pm   0.06$& $-28.30 \pm   0.06$& $-27.63 \pm   0.06$ \\
SNF20080612-003 & $120.0 \pm 1.1$ & $  7.3 \pm 0.6$& $ 6123 \pm   3$ & $-29.64 \pm   0.02$ & $-29.41 \pm   0.02$& $-28.99 \pm   0.02$& $-28.70 \pm   0.02$& $-28.00 \pm   0.02$ \\
SNF20080626-002 & $130.0 \pm 1.0$ & $  6.1 \pm 4.2$& $ 6111 \pm   3$ & $-29.42 \pm   0.01$ & $-29.24 \pm   0.01$& $-28.84 \pm   0.01$& $-28.52 \pm   0.01$& $-27.76 \pm   0.01$ \\
SNF20060621-015 & $111.9 \pm 1.3$ & $  9.8 \pm 0.7$& $ 6144 \pm   3$ & $-29.63 \pm   0.01$ & $-29.36 \pm   0.01$& $-28.88 \pm   0.01$& $-28.54 \pm   0.01$& $-27.81 \pm   0.01$ \\
SNF20080920-000 & $135.2 \pm 1.4$ & $  5.6 \pm 1.6$& $ 6085 \pm   3$ & $-29.44 \pm   0.02$ & $-29.19 \pm   0.02$& $-28.79 \pm   0.02$& $-28.49 \pm   0.02$& $-27.74 \pm   0.02$ \\
SN2007cq & $ 65.8 \pm 4.1$ & $ 10.2 \pm 0.9$& $ 6137 \pm   3$ & $-29.53 \pm   0.02$ & $-29.30 \pm   0.02$& $-28.89 \pm   0.02$& $-28.56 \pm   0.02$& $-27.90 \pm   0.02$ \\
SNF20080918-004 & $ 87.8 \pm 7.2$ & $ 21.5 \pm 0.9$& $ 6141 \pm   4$ & $-29.00 \pm   0.22$ & $-28.82 \pm   0.22$& $-28.37 \pm   0.22$& $-28.13 \pm   0.22$& $-27.43 \pm   0.22$ \\
CSS120424\_01 & $138.1 \pm 2.1$ & $ 11.7 \pm 0.7$& $ 6138 \pm   3$ & $-29.40 \pm   0.02$ & $-29.23 \pm   0.02$& $-28.77 \pm   0.01$& $-28.45 \pm   0.02$& $-27.68 \pm   0.02$ \\
SNF20080610-000 & $119.9 \pm 10.4$ & $ 16.4 \pm 1.7$& $ 6131 \pm   6$ & $-29.05 \pm   0.07$ & $-28.92 \pm   0.07$& $-28.50 \pm   0.07$& $-28.22 \pm   0.07$& $-27.55 \pm   0.07$ \\
SNF20070701-005 & $101.8 \pm 2.6$ & $ 12.4 \pm 1.0$& $ 6158 \pm   5$ & $-29.46 \pm   0.02$ & $-29.27 \pm   0.02$& $-28.87 \pm   0.02$& $-28.60 \pm   0.02$& $-27.96 \pm   0.02$ \\
SN2007kk & $128.5 \pm 1.4$ & $ 10.6 \pm 1.0$& $ 6098 \pm   4$ & $-29.48 \pm   0.02$ & $-29.31 \pm   0.02$& $-28.87 \pm   0.01$& $-28.54 \pm   0.01$& $-27.77 \pm   0.02$ \\
SNF20060908-004 & $114.4 \pm 1.2$ & $ 12.6 \pm 0.6$& $ 6136 \pm   3$ & $-29.59 \pm   0.23$ & $-29.34 \pm   0.23$& $-28.91 \pm   0.23$& $-28.58 \pm   0.23$& $-27.87 \pm   0.23$ \\
SNF20080909-030 & $ 93.7 \pm 1.0$ & $  7.8 \pm 0.4$& $ 6171 \pm   3$ & $-29.38 \pm   0.02$ & $-29.12 \pm   0.01$& $-28.74 \pm   0.01$& $-28.44 \pm   0.01$& $-27.78 \pm   0.01$ \\
PTF11bgv & $ 79.4 \pm 3.2$ & $ 12.6 \pm 0.7$& $ 6146 \pm   3$ & $-28.90 \pm   0.02$ & $-28.62 \pm   0.01$& $-28.27 \pm   0.01$& $-28.08 \pm   0.01$& $-27.54 \pm   0.01$ \\
SNNGC2691 & $ 39.0 \pm 22.2$ & $  4.5 \pm 0.2$& $ 6139 \pm   8$ & $-29.46 \pm   0.02$ & $-29.06 \pm   0.02$& $-28.75 \pm   0.02$& $-28.49 \pm   0.02$& $-27.93 \pm   0.02$ \\
PTF13asv & $ 75.6 \pm 1.1$ & $  2.2 \pm 0.4$& $ 6148 \pm   4$ & $-29.92 \pm   0.32$ & $-29.49 \pm   0.32$& $-29.02 \pm   0.32$& $-28.63 \pm   0.32$& $-27.90 \pm   0.32$ \\
SNF20070806-026 & $ 98.8 \pm 12.1$ & $ 25.9 \pm 0.7$& $ 6114 \pm   7$ & $-29.14 \pm   0.02$ & $-28.91 \pm   0.02$& $-28.44 \pm   0.02$& $-28.21 \pm   0.02$& $-27.49 \pm   0.02$ \\
SNF20070427-001 & $ 81.3 \pm 2.3$ & $  6.3 \pm 0.9$& $ 6142 \pm   5$ & $-29.89 \pm   0.02$ & $-29.46 \pm   0.02$& $-28.97 \pm   0.02$& $-28.62 \pm   0.02$& $-27.97 \pm   0.02$ \\
SNF20061108-004 & $129.5 \pm 5.6$ & $  6.3 \pm 2.5$& $ 6110 \pm   6$ & $-29.53 \pm   0.02$ & $-29.31 \pm   0.02$& $-28.95 \pm   0.02$& $-28.60 \pm   0.02$& $-27.96 \pm   0.02$ \\
SNF20060912-000 & $106.5 \pm 1.8$ & $ 21.4 \pm 1.7$& $ 6163 \pm   7$ & $-28.98 \pm   0.02$ & $-28.92 \pm   0.02$& $-28.66 \pm   0.02$& $-28.42 \pm   0.02$& $-27.77 \pm   0.02$ \\
CSS110918\_02 & $109.1 \pm 9.4$ & $ 15.0 \pm 0.6$& $ 6137 \pm   3$ & $-29.36 \pm   0.02$ & $-29.14 \pm   0.01$& $-28.69 \pm   0.01$& $-28.41 \pm   0.01$& $-27.70 \pm   0.01$ \\
SNF20080918-002 & $ 97.7 \pm 2.8$ & $ 12.6 \pm 1.4$& $ 6141 \pm   6$ & $-29.50 \pm   0.02$ & $-29.11 \pm   0.02$& $-28.61 \pm   0.02$& $-28.34 \pm   0.02$& $-27.71 \pm   0.02$ \\
SNIC3573 & $102.7 \pm 1.8$ & $ 11.9 \pm 1.0$& $ 6142 \pm   5$ & $-29.28 \pm   0.02$ & $-29.14 \pm   0.02$& $-28.74 \pm   0.02$& $-28.46 \pm   0.01$& $-27.76 \pm   0.03$ \\
SNF20080725-004 & $133.6 \pm 2.1$ & $  6.9 \pm 0.9$& $ 6131 \pm   6$ & $-29.09 \pm   0.01$ & $-28.93 \pm   0.01$& $-28.59 \pm   0.01$& $-28.31 \pm   0.01$& $-27.55 \pm   0.03$ \\
SNF20050728-006 & $127.8 \pm 2.5$ & $ 15.8 \pm 1.3$& $ 6124 \pm   6$ & $-28.80 \pm   0.02$ & $-28.68 \pm   0.02$& $-28.37 \pm   0.02$& $-28.18 \pm   0.02$& $-27.55 \pm   0.02$ \\
SN2012fr & $134.2 \pm 0.5$ & $  7.4 \pm 0.2$& $ 6102 \pm   1$ & $-29.91 \pm   0.01$ & $-29.70 \pm   0.01$& $-29.31 \pm   0.01$& $-28.94 \pm   0.01$& $-28.10 \pm   0.01$ \\
SNF20060512-002 & $100.2 \pm 2.8$ & $ 13.4 \pm 1.1$& $ 6107 \pm   8$ & $-29.33 \pm   0.02$ & $-29.11 \pm   0.02$& $-28.77 \pm   0.02$& $-28.52 \pm   0.02$& $-27.80 \pm   0.02$ \\
SNF20060512-001 & $ 88.4 \pm 1.2$ & $  5.4 \pm 0.4$& $ 6169 \pm   3$ & $-29.33 \pm   0.01$ & $-29.05 \pm   0.01$& $-28.68 \pm   0.01$& $-28.40 \pm   0.01$& $-27.79 \pm   0.01$ \\
SNF20071003-016 & $125.2 \pm 4.6$ & $ 17.1 \pm 2.0$& $ 6124 \pm  11$ & $-28.58 \pm   0.02$ & $-28.54 \pm   0.02$& $-28.19 \pm   0.02$& $-27.99 \pm   0.02$& $-27.31 \pm   0.02$ \\
SNF20050821-007 & $141.7 \pm 2.6$ & $  7.7 \pm 1.0$& $ 6140 \pm   9$ & $-29.38 \pm   0.02$ & $-29.20 \pm   0.02$& $-28.77 \pm   0.02$& $-28.46 \pm   0.02$& $-27.67 \pm   0.02$ \\
SNF20070803-005 & $ 22.7 \pm 21.4$ & $  0.9 \pm 0.6$& $ 6157 \pm  27$ & $-29.87 \pm   0.01$ & $-29.43 \pm   0.01$& $-29.04 \pm   0.01$& $-28.74 \pm   0.01$& $-28.11 \pm   0.01$ \\
PTF09foz & $127.2 \pm 1.9$ & $ 21.7 \pm 1.2$& $ 6136 \pm   4$ & $-29.14 \pm   0.01$ & $-29.00 \pm   0.01$& $-28.59 \pm   0.01$& $-28.35 \pm   0.01$& $-27.65 \pm   0.01$ \\
PTF12grk & $162.3 \pm 9.8$ & $ 19.6 \pm 1.4$& $ 6085 \pm   8$ & $-28.86 \pm   0.02$ & $-28.87 \pm   0.01$& $-28.42 \pm   0.01$& $-28.19 \pm   0.01$& $-27.50 \pm   0.03$ \\
SNF20080720-001 & $138.5 \pm 4.0$ & $ 14.0 \pm 2.0$& $ 6107 \pm   3$ & $-27.59 \pm   0.02$ & $-27.78 \pm   0.01$& $-27.73 \pm   0.01$& $-27.71 \pm   0.01$& $-27.19 \pm   0.02$ \\
SNF20080810-001 & $ 88.4 \pm 21.6$ & $ 22.3 \pm 1.1$& $ 6145 \pm   5$ & $-29.11 \pm   0.01$ & $-28.89 \pm   0.01$& $-28.45 \pm   0.01$& $-28.23 \pm   0.01$& $-27.60 \pm   0.01$ \\
SNF20050729-002 & $109.4 \pm 2.2$ & $ 11.5 \pm 1.7$& $ 6142 \pm   6$ & $-29.35 \pm   0.13$ & $-29.17 \pm   0.13$& $-28.68 \pm   0.13$& $-28.38 \pm   0.13$& $-27.56 \pm   0.13$ \\
SN2008ec & $103.7 \pm 17.0$ & $ 23.1 \pm 0.4$& $ 6125 \pm   3$ & $-28.67 \pm   0.01$ & $-28.52 \pm   0.01$& $-28.18 \pm   0.01$& $-28.03 \pm   0.01$& $-27.47 \pm   0.01$ \\
SNF20070902-018 & $ 93.8 \pm 12.2$ & $ 23.8 \pm 3.0$& $ 6120 \pm   8$ & $-28.87 \pm   0.02$ & $-28.70 \pm   0.01$& $-28.26 \pm   0.01$& $-28.08 \pm   0.01$& $-27.41 \pm   0.02$ \\
SNF20070424-003 & $122.5 \pm 3.8$ & $ 12.7 \pm 1.6$& $ 6132 \pm   6$ & $-29.10 \pm   0.01$ & $-28.96 \pm   0.01$& $-28.51 \pm   0.01$& $-28.25 \pm   0.01$& $-27.57 \pm   0.01$ \\
SN2006cj & $101.7 \pm 1.3$ & $  4.8 \pm 0.8$& $ 6127 \pm   3$ & $-29.43 \pm   0.01$ & $-29.14 \pm   0.01$& $-28.74 \pm   0.01$& $-28.43 \pm   0.01$& $-27.76 \pm   0.01$ \\
SN2007nq & $ 89.8 \pm 9.9$ & $ 23.4 \pm 1.1$& $ 6109 \pm   5$ & $-29.11 \pm   0.02$ & $-28.91 \pm   0.02$& $-28.50 \pm   0.02$& $-28.27 \pm   0.02$& $-27.57 \pm   0.02$ \\
SNF20070817-003 & $ 93.9 \pm 2.4$ & $ 18.5 \pm 1.3$& $ 6116 \pm   6$ & $-29.19 \pm   0.02$ & $-29.03 \pm   0.01$& $-28.59 \pm   0.01$& $-28.30 \pm   0.01$& $-27.55 \pm   0.02$ \\
SNF20070403-000 & $ 61.8 \pm 6.5$ & $ 27.1 \pm 1.8$& $ 6154 \pm   8$ & $-28.37 \pm   0.02$ & $-28.27 \pm   0.02$& $-27.97 \pm   0.02$& $-27.80 \pm   0.02$& $-27.24 \pm   0.02$ \\
SNF20061022-005 & $ 64.6 \pm 3.8$ & $  3.7 \pm 1.4$& $ 6146 \pm   7$ & $-29.49 \pm   0.02$ & $-29.06 \pm   0.02$& $-28.71 \pm   0.02$& $-28.42 \pm   0.02$& $-27.93 \pm   0.02$ \\
SNNGC4076 & $127.3 \pm 2.4$ & $ 15.5 \pm 1.2$& $ 6152 \pm   4$ & $-28.77 \pm   0.01$ & $-28.66 \pm   0.01$& $-28.37 \pm   0.01$& $-28.15 \pm   0.01$& $-27.52 \pm   0.01$ \\
SNF20070727-016 & $ 77.5 \pm 2.5$ & $  5.1 \pm 0.8$& $ 6140 \pm   4$ & $-29.96 \pm   0.06$ & $-29.56 \pm   0.06$& $-29.06 \pm   0.06$& $-28.75 \pm   0.06$& $-28.01 \pm   0.06$ \\
PTF12fuu & $105.5 \pm 3.0$ & $  6.2 \pm 1.2$& $ 6124 \pm   5$ & $-29.54 \pm   0.01$ & $-29.23 \pm   0.01$& $-28.74 \pm   0.01$& $-28.40 \pm   0.01$& $-27.64 \pm   0.01$ \\
SNF20070820-000 & $107.2 \pm 3.5$ & $ 18.6 \pm 1.3$& $ 6132 \pm  14$ & $-28.80 \pm   0.02$ & $-28.69 \pm   0.02$& $-28.34 \pm   0.02$& $-28.13 \pm   0.02$& $-27.52 \pm   0.02$ \\
SNF20070725-001 & $108.4 \pm 2.0$ & $ 11.1 \pm 1.5$& $ 6140 \pm   7$ & $-29.61 \pm   0.02$ & $-29.32 \pm   0.02$& $-28.84 \pm   0.02$& $-28.50 \pm   0.02$& $-27.76 \pm   0.02$ \\
SNF20071108-021 & $ 99.1 \pm 2.7$ & $  5.8 \pm 0.8$& $ 6164 \pm   5$ & $-29.67 \pm   0.01$ & $-29.34 \pm   0.01$& $-28.94 \pm   0.01$& $-28.60 \pm   0.01$& $-27.96 \pm   0.01$ \\
SNF20080914-001 & $126.5 \pm 1.2$ & $ 15.4 \pm 1.1$& $ 6159 \pm   3$ & $-28.67 \pm   0.02$ & $-28.60 \pm   0.02$& $-28.31 \pm   0.02$& $-28.13 \pm   0.02$& $-27.58 \pm   0.02$ \\
SNF20060609-002 & $ 87.7 \pm 3.6$ & $  7.3 \pm 1.3$& $ 6132 \pm   4$ & $-28.60 \pm   0.02$ & $-28.42 \pm   0.02$& $-28.19 \pm   0.02$& $-28.05 \pm   0.02$& $-27.53 \pm   0.02$ \\
SNF20050624-000 & $121.0 \pm 5.3$ & $  9.3 \pm 3.1$& $ 6126 \pm   6$ & $-29.75 \pm   0.01$ & $-29.42 \pm   0.01$& $-28.99 \pm   0.01$& $-28.68 \pm   0.01$& $-27.97 \pm   0.01$ \\
SNF20060618-023 & $ 74.9 \pm 4.9$ & $  5.0 \pm 1.8$& $ 6137 \pm  21$ & $-29.61 \pm   0.02$ & $-29.18 \pm   0.02$& $-28.89 \pm   0.02$& $-28.66 \pm   0.02$& $-28.08 \pm   0.02$ \\
SNF20080531-000 & $133.0 \pm 1.5$ & $ 17.6 \pm 0.8$& $ 6114 \pm   5$ & $-29.12 \pm   0.01$ & $-28.98 \pm   0.01$& $-28.54 \pm   0.01$& $-28.28 \pm   0.01$& $-27.51 \pm   0.01$ \\
SN2006do & $106.4 \pm 2.1$ & $ 26.7 \pm 1.3$& $ 6101 \pm   2$ & $-29.00 \pm   0.01$ & $-28.83 \pm   0.01$& $-28.42 \pm   0.01$& $-28.20 \pm   0.01$& $-27.53 \pm   0.04$ \\
PTF12ikt & $110.3 \pm 1.6$ & $ 14.2 \pm 0.7$& $ 6141 \pm   4$ & $-29.34 \pm   0.01$ & $-29.04 \pm   0.01$& $-28.57 \pm   0.01$& $-28.32 \pm   0.01$& $-27.66 \pm   0.01$ \\
SN2006dm & $ 99.5 \pm 1.6$ & $ 30.0 \pm 0.7$& $ 6118 \pm   3$ & $-28.81 \pm   0.01$ & $-28.65 \pm   0.01$& $-28.23 \pm   0.01$& $-28.02 \pm   0.01$& $-27.33 \pm   0.01$ \\
PTF13azs & $138.0 \pm 5.1$ & $ 16.2 \pm 1.6$& $ 6125 \pm  10$ & $-27.84 \pm   0.02$ & $-27.92 \pm   0.02$& $-27.69 \pm   0.02$& $-27.60 \pm   0.02$& $-26.99 \pm   0.02$ \\
SN2005hj & $ 80.8 \pm 2.4$ & $  4.3 \pm 0.8$& $ 6138 \pm   4$ & $-29.54 \pm   0.02$ & $-29.16 \pm   0.01$& $-28.87 \pm   0.01$& $-28.54 \pm   0.01$& $-28.01 \pm   0.01$ \\
PTF12iiq & $150.4 \pm 2.2$ & $ 22.5 \pm 0.8$& $ 6041 \pm   6$ & $-28.60 \pm   0.01$ & $-28.77 \pm   0.01$& $-28.41 \pm   0.01$& $-28.10 \pm   0.01$& $-27.29 \pm   0.01$ \\
PTF10ndc & $124.2 \pm 2.4$ & $  6.8 \pm 1.1$& $ 6119 \pm   3$ & $-29.52 \pm   0.01$ & $-29.25 \pm   0.01$& $-28.80 \pm   0.01$& $-28.49 \pm   0.01$& $-27.76 \pm   0.01$ \\
SNF20080919-002 & $103.6 \pm 7.2$ & $ 27.2 \pm 1.9$& $ 6133 \pm   8$ & $-28.74 \pm   0.02$ & $-28.46 \pm   0.01$& $-28.09 \pm   0.01$& $-27.87 \pm   0.01$& $-27.26 \pm   0.04$ \\
SNPGC027923 & $ 85.5 \pm 0.6$ & $  5.9 \pm 0.3$& $ 6130 \pm   4$ & $-29.87 \pm   0.02$ & $-29.45 \pm   0.02$& $-28.94 \pm   0.02$& $-28.57 \pm   0.02$& $-27.85 \pm   0.02$ \\
SNF20070330-024 & $118.1 \pm 2.1$ & $  4.6 \pm 2.2$& $ 6101 \pm   3$ & $-29.77 \pm   0.02$ & $-29.52 \pm   0.02$& $-29.08 \pm   0.02$& $-28.74 \pm   0.01$& $-27.94 \pm   0.02$ \\
SNF20061030-010 & $131.4 \pm 2.2$ & $ 17.4 \pm 1.1$& $ 6116 \pm   4$ & $-28.60 \pm   0.02$ & $-28.55 \pm   0.02$& $-28.25 \pm   0.02$& $-28.03 \pm   0.02$& $-27.34 \pm   0.02$ \\
SNhunt46 & $ 94.1 \pm 2.0$ & $ 11.2 \pm 0.6$& $ 6132 \pm   4$ & $-29.50 \pm   0.02$ & $-29.11 \pm   0.02$& $-28.67 \pm   0.02$& $-28.37 \pm   0.02$& $-27.71 \pm   0.02$ \\
SN2005hc & $126.9 \pm 2.5$ & $ 10.0 \pm 0.7$& $ 6123 \pm   3$ & $-29.38 \pm   0.01$ & $-29.13 \pm   0.01$& $-28.69 \pm   0.01$& $-28.37 \pm   0.01$& $-27.61 \pm   0.01$ \\
LSQ12dbr & $106.9 \pm 0.6$ & $  7.1 \pm 0.7$& $ 6138 \pm   4$ & $-29.29 \pm   0.73$ & $-29.00 \pm   0.73$& $-28.51 \pm   0.73$& $-28.15 \pm   0.73$& $-27.38 \pm   0.73$ \\
LSQ12hjm & $ 82.6 \pm 17.5$ & $ 12.2 \pm 1.4$& $ 6144 \pm   5$ & $-29.51 \pm   0.02$ & $-29.14 \pm   0.01$& $-28.60 \pm   0.01$& $-28.30 \pm   0.01$& $-27.71 \pm   0.02$ \\
SNF20060521-001 & $ 78.9 \pm 20.2$ & $ 21.1 \pm 1.4$& $ 6123 \pm  10$ & $-29.37 \pm   0.05$ & $-29.04 \pm   0.05$& $-28.54 \pm   0.05$& $-28.30 \pm   0.05$& $-27.57 \pm   0.05$ \\
SNF20070630-006 & $125.5 \pm 3.2$ & $ 10.1 \pm 1.6$& $ 6126 \pm   4$ & $-29.34 \pm   0.01$ & $-29.12 \pm   0.01$& $-28.65 \pm   0.01$& $-28.38 \pm   0.01$& $-27.66 \pm   0.01$ \\
PTF11drz & $132.6 \pm 1.4$ & $ 15.2 \pm 1.0$& $ 6116 \pm   5$ & $-29.12 \pm   0.01$ & $-28.95 \pm   0.01$& $-28.53 \pm   0.01$& $-28.27 \pm   0.01$& $-27.55 \pm   0.01$ \\
SNF20080323-009 & $ 95.9 \pm 2.3$ & $ 10.6 \pm 1.1$& $ 6143 \pm   6$ & $-29.59 \pm   0.02$ & $-29.22 \pm   0.02$& $-28.68 \pm   0.02$& $-28.42 \pm   0.02$& $-27.77 \pm   0.02$ \\
SNF20071021-000 & $167.5 \pm 2.2$ & $ 20.4 \pm 0.6$& $ 6112 \pm   4$ & $-28.75 \pm   0.02$ & $-28.78 \pm   0.02$& $-28.40 \pm   0.02$& $-28.18 \pm   0.02$& $-27.41 \pm   0.02$ \\
SNNGC0927 & $155.2 \pm 1.4$ & $ 11.0 \pm 0.7$& $ 6109 \pm   4$ & $-28.87 \pm   0.02$ & $-28.81 \pm   0.01$& $-28.46 \pm   0.01$& $-28.22 \pm   0.01$& $-27.48 \pm   0.01$ \\
SNF20060526-003 & $112.1 \pm 2.5$ & $  9.8 \pm 1.0$& $ 6121 \pm   3$ & $-29.34 \pm   0.01$ & $-29.09 \pm   0.01$& $-28.68 \pm   0.01$& $-28.39 \pm   0.01$& $-27.70 \pm   0.01$ \\
SNF20080806-002 & $135.8 \pm 1.8$ & $  7.5 \pm 0.9$& $ 6135 \pm   4$ & $-29.22 \pm   0.02$ & $-29.02 \pm   0.02$& $-28.61 \pm   0.02$& $-28.35 \pm   0.01$& $-27.71 \pm   0.02$ \\
SNF20080803-000 & $117.6 \pm 2.6$ & $  8.9 \pm 2.0$& $ 6125 \pm   4$ & $-28.84 \pm   0.01$ & $-28.70 \pm   0.01$& $-28.35 \pm   0.01$& $-28.16 \pm   0.01$& $-27.50 \pm   0.01$ \\
SNF20080822-005 & $ 78.5 \pm 1.8$ & $  6.3 \pm 0.9$& $ 6138 \pm   4$ & $-29.71 \pm   0.01$ & $-29.34 \pm   0.01$& $-28.93 \pm   0.01$& $-28.61 \pm   0.01$& $-27.92 \pm   0.01$ \\
SNF20060618-014 & $137.2 \pm 2.5$ & $  9.3 \pm 1.1$& $ 6112 \pm   7$ & $-29.27 \pm   0.03$ & $-29.09 \pm   0.03$& $-28.73 \pm   0.03$& $-28.38 \pm   0.03$& $-27.68 \pm   0.03$ \\
PTF12ghy & $ 99.3 \pm 3.6$ & $ 16.8 \pm 0.7$& $ 6134 \pm   3$ & $-28.29 \pm   0.02$ & $-28.27 \pm   0.01$& $-28.05 \pm   0.01$& $-27.95 \pm   0.01$& $-27.40 \pm   0.01$ \\
SNF20070531-011 & $122.4 \pm 2.7$ & $ 21.2 \pm 0.8$& $ 6114 \pm   4$ & $-29.07 \pm   0.01$ & $-28.94 \pm   0.01$& $-28.50 \pm   0.01$& $-28.26 \pm   0.01$& $-27.53 \pm   0.03$ \\
SNF20070831-015 & $112.2 \pm 2.7$ & $  7.8 \pm 1.0$& $ 6145 \pm   6$ & $-29.42 \pm   0.01$ & $-29.17 \pm   0.01$& $-28.78 \pm   0.01$& $-28.46 \pm   0.01$& $-27.78 \pm   0.01$ \\
SNF20070417-002 & $104.5 \pm 5.5$ & $ 24.4 \pm 2.2$& $ 6123 \pm   9$ & $-29.20 \pm   0.05$ & $-29.01 \pm   0.05$& $-28.48 \pm   0.05$& $-28.23 \pm   0.05$& $-27.54 \pm   0.05$ \\
PTF11cao & $143.3 \pm 1.6$ & $ 18.9 \pm 1.3$& $ 6104 \pm   5$ & $-28.78 \pm   0.02$ & $-28.79 \pm   0.02$& $-28.44 \pm   0.02$& $-28.18 \pm   0.02$& $-27.45 \pm   0.02$ \\
SNF20080522-000 & $ 61.8 \pm 3.5$ & $  3.3 \pm 0.9$& $ 6131 \pm   7$ & $-29.86 \pm   0.01$ & $-29.41 \pm   0.01$& $-29.03 \pm   0.01$& $-28.70 \pm   0.01$& $-28.06 \pm   0.01$ \\
PTF10qjq & $ 73.9 \pm 2.4$ & $ 12.8 \pm 0.8$& $ 6133 \pm   3$ & $-29.29 \pm   0.02$ & $-28.94 \pm   0.02$& $-28.53 \pm   0.01$& $-28.35 \pm   0.01$& $-27.76 \pm   0.01$ \\
PTF12dxm & $ 95.4 \pm 41.8$ & $ 35.7 \pm 2.8$& $ 6136 \pm   4$ & $-28.71 \pm   0.01$ & $-28.58 \pm   0.01$& $-28.19 \pm   0.01$& $-27.99 \pm   0.01$& $-27.34 \pm   0.01$ \\
SNF20061021-003 & $122.8 \pm 2.3$ & $  9.7 \pm 1.7$& $ 6131 \pm   4$ & $-29.04 \pm   0.02$ & $-28.86 \pm   0.02$& $-28.56 \pm   0.02$& $-28.30 \pm   0.02$& $-27.64 \pm   0.02$ \\
SNF20080510-005 & $111.6 \pm 2.6$ & $  6.4 \pm 1.1$& $ 6115 \pm   4$ & $-29.41 \pm   0.01$ & $-29.15 \pm   0.01$& $-28.70 \pm   0.01$& $-28.38 \pm   0.01$& $-27.73 \pm   0.04$ \\
SNF20080507-000 & $ 98.1 \pm 1.6$ & $ 10.6 \pm 2.1$& $ 6143 \pm   5$ & $-29.23 \pm   0.01$ & $-29.05 \pm   0.01$& $-28.71 \pm   0.01$& $-28.45 \pm   0.01$& $-27.79 \pm   0.01$ \\
SNF20080913-031 & $118.2 \pm 1.5$ & $ 11.3 \pm 1.8$& $ 6158 \pm   5$ & $-29.13 \pm   0.08$ & $-29.01 \pm   0.07$& $-28.62 \pm   0.07$& $-28.32 \pm   0.07$& $-27.68 \pm   0.07$ \\
SNF20080510-001 & $118.8 \pm 2.1$ & $ 15.3 \pm 1.3$& $ 6115 \pm   4$ & $-29.35 \pm   0.01$ & $-29.15 \pm   0.01$& $-28.69 \pm   0.01$& $-28.38 \pm   0.01$& $-27.68 \pm   0.01$ \\
SNF20070712-003 & $108.8 \pm 2.7$ & $ 13.5 \pm 0.9$& $ 6155 \pm   6$ & $-29.44 \pm   0.02$ & $-29.19 \pm   0.01$& $-28.74 \pm   0.01$& $-28.42 \pm   0.01$& $-27.78 \pm   0.01$ \\
\enddata
\end{deluxetable}
\section{Supernova Model I: Two Extrinsic Parameters}
\label{model:sec}

We hypothesize that at peak brightness
SN~Ia broadband magnitudes and colors are correlated with
spectral features: equivalent widths and line positions are considered as statistics localized in wavelength that are insensitive to variations in
broadband colors.
These intrinsic spectral  parameters may not deterministically predict magnitudes, but rather do so with some intrinsic dispersion,
which in this section is not yet associated with a parameter. The intrinsic magnitudes are then
modified by an extrinsic physical process (i.e.\ dust) to produce apparent magnitudes.

\subsection{Model}
We assume 
that  peak intrinsic $UBVRI$ magnitudes are linearly dependent
on the
 equivalent widths of the CaII H\&K and SiII$\lambda$4141 spectral features
$EW_{Ca}$, $EW_{Si}$,
and $\lambda_{Si}$ the wavelength of the minimum of 
the SiII$\lambda6355$ feature:
these spectral features are associated with SN~Ia  spectroscopic diversity  
\citep{2006PASP..118..560B, 2008A&A...492..535A, 2009A&A...500L..17B, 2009PASP..121..238B, 2009ApJ...699L.139W, 2011ApJ...729...55F}.
The explicit omission of light-curve shape in our model is compensated by its proxy,
$EW_{Si}$, at peak brightness
\citep{2008A&A...492..535A, 2011A&A...529L...4C}. 
Residual dispersion is described by a Normal distribution with a parameterized covariance matrix
$C_c$.  A grey magnitude offset, $\Delta$, is included for each supernova
to capture intrinsic dispersion and peculiar-velocity errors introduced when converting fluxes to luminosity.
The observed magnitudes are linearly dependent on the
extrinsic-color parameters $k_0$ and $k_1$.  
The observables
$U_o, B_o, V_o, R_o, I_o$, $EW_{Ca,o}$, $EW_{Si,o}$, $\lambda_{Si,o}$
shown in Table~\ref{data:tab} have Gaussian measurement uncertainty with covariance $C$.
Unlike the parameters associated with
the spectral measurements, $EW_{Ca}$, $EW_{Si}$ and $\lambda_{Si}$,  the latent
parameters $k_0$ and $k_1$ are not directly associated
with observables but are rather drawn from the set of peak magnitudes.
The model 
\color{red}
likelihood density
\color{black}
is written as
\begin{equation}
\begin{pmatrix}
U\\B\\V\\R\\I
\end{pmatrix}
\sim \mathcal{N}
\left(
\Delta +
\begin{pmatrix}
c_U+\alpha_U EW_{Ca} + \beta_U EW_{Si} + \eta_U \lambda_{Si} \\
c_B+\alpha_B EW_{Ca} + \beta_B EW_{Si} + \eta_B \lambda_{Si}  \\
c_V+\alpha_V EW_{Ca} + \beta_V EW_{Si} + \eta_V \lambda_{Si} \\
c_R+\alpha_R EW_{Ca} + \beta_R EW_{Si} + \eta_R \lambda_{Si} \\
c_I+\alpha_I EW_{Ca} + \beta_I EW_{Si}+ \eta_I \lambda_{Si}
\end{pmatrix}
,C_{c}
\right)
\label{ewsiv:eqn}
\end{equation}
\begin{equation}
\begin{pmatrix}
U_o\\B_o\\ V_o\\R_o\\I_o\\EW_{Si, o}\\ EW_{Ca, o} \\ \lambda_{Si, o}
\end{pmatrix}
\sim \mathcal{N}
\left(
\begin{pmatrix}
U +\gamma^0_{U} k_0 +\gamma^1_{U} k_1 \\B +\gamma^0_{B} k_0 +\gamma^1_{B} k_1 \\
V+\gamma^0_{V} k_0+\gamma^1_{V} k_1\\R+\gamma^0_{R} k_0 + \gamma^1_{R} k_1\\I+\gamma^0_{I} k_0+\gamma^1_{I} k_1\\
EW_{Si}\\ EW_{Ca} \\ \lambda_{Si}
\end{pmatrix}
,C
\right).
\label{dust:eqn}
\end{equation}
The global parameters $c_X$, $\alpha_i$, $\beta_i$,  and $\eta_i$,  are the intercepts and slopes of the linear relationships that
relate intrinsic magnitudes with the spectral-feature parameters.
The global parameters $\gamma^0$, $\gamma^1$  are the slopes that connect the extrinsic-color
parameters to observed magnitudes.

To constrain the degrees of freedom and degeneracies inherent in the model we impose that
each of the vectors $k_0+1/N$ and $k_1+1/N$,  where $N$ is the number of objects in our sample, are element-wise $\ge 0$ and sum to one, i.e.\
these vectors are simplexes.
These
conditions break the 
degeneracies  $\gamma ^0\rightarrow a\gamma^0$, $k_0 \rightarrow a^{-1}k_0$ and $\gamma^1 \rightarrow a\gamma^1$, $k_1 \rightarrow a^{-1}k_1$
up to a $\pm$ sign.
Furthermore we impose
\begin{equation}
\langle \Delta \rangle=0, \langle k_0 \rangle=0, \langle k_1 \rangle=0, \gamma^0_U > 0, \gamma^1_U < 0.
\end{equation}
The first two conditions specify the definition of zero color relative to which the color excess is measured:
in a linear model this value is arbitrary.
The latter two exclude degenerate posterior space
associated
with the $\pm$ sign, i.e.\ simultaneous sign flips of
$\gamma^0$--$k_0$ and $\gamma^1$--$k_1$
(when running our fits, $k_0$ and $k_1$ are set to
identical initial conditions, so the difference in the signs of $\gamma^0_U$ and $\gamma^1_U$ distinguishes the two.)
As  $\gamma^0$--$k_0$ and $\gamma^1$--$k_1$ are degenerate with
each other,
the different initial and boundary conditions for $\gamma^0_U$ and $\gamma^1_U$ break the degeneracy of our mixture model;
otherwise identical initial conditions for $\gamma^0$, $\gamma^1$ produce indistinguishable posteriors
with relatively broad credible intervals and lower maximum likelihood.


\color{red}
For $N$ supernovae there are $8N$ observables.  For the top-level model parameters, there are $3N$ spectral parameters, $2(N-2)$
color parameters, $(N-1)$ magnitude offsets,  $5 \times 6$ global coefficients, and $10$ parameters that describe the intrinsic
$C_c$.  For $N=172$ supernovae, there are 1376 observables and 1067 top-level parameters.
\color{black}

Having the intrinsic dispersion, $C_c$, as fit parameters seemingly introduces degeneracy in the model, as magnitude and color variation
ascribed to $\Delta$, $\gamma^0 k_0$, and $\gamma^1 k_1$ could also be attributed to intrinsic dispersion.  There are several features of the model
that drive the assignation of variations away from $C_c$:  Maximizing the posterior disfavors the increase of $\det{(C_c)}$;
The distributions of $\Delta$, $k_0$, and $k_1$ turn out to
be non-Gaussian, and so are not well described by the Normal covariance we impose in $C_c$; the grey magnitude offsets, $\Delta$, would appear as a constant
in all elements of the covariance matrix, which is disfavored for the Bayesian prior selected for $C_c$.

In a Bayesian analysis such as this the priors must be described.  A flat prior is used for all parameters except
for the covariance matrix $C_c$, which is constructed from a correlation matrix with  $\nu=4$  LKJ prior\footnote{
Visualization of the LKJ correlation distribution can be found in \url{http://www.psychstatistics.com/2014/12/27/d-lkj-priors/}.}
\citep{Lewandowski20091989} and standard
deviations $\sigma_i = \sqrt{C_{c,ii}}$ with a  Cauchy distribution prior with location
 $0.1$ and scale $0.1$ mag restricted to positive values.
This covariance matrix prior is recommended by STAN \citep{stan}, the Monte Carlo we use to evaluate the model.
 We find that imposing a stricter assumption of a
 diagonal covariance matrix, with no reference to a correlation matrix with LKJ prior, produces little change in the posteriors of
 the other parameters.

\color{blue}
There remains degeneracies in the model.  One is illustrated by the  transformations $\Delta \rightarrow \Delta  + \epsilon_\Delta$,
 $\gamma^0 \rightarrow \gamma^0  + \epsilon_{\gamma^0}$, $\gamma^1 \rightarrow \gamma^1 + \epsilon_{\gamma^1}$under the condition
$$
\epsilon_\Delta  +  \epsilon_{\gamma^0} k_0+  \epsilon_{\gamma^1} k_1=0,
$$
i.e.\ shifts in the global coefficients $\gamma$ can be compensated by shifts in $\Delta$ for each supernova while maintaining 
$\langle \Delta \rangle=0$.
Another degeneracy is illustrated by $k_0 \rightarrow k_0 + (EW_{Ca}-\langle EW_{Ca}\rangle)\epsilon$,
$\alpha \rightarrow \alpha - \gamma \epsilon$, $c \rightarrow c + \langle EW_{Ca}\rangle \epsilon$, i.e. the
linear external terms can leak into the linear spectral feature corrections.
These degeneracies are not broken within the model, but rather are constrained by the priors, in which the
per-supernova parameters
$\Delta$, $EW$, $\lambda$, $k$ are uncorrelated.  It is important to emphasize that the latent parameters $k$ and $\Delta$
are non-trivially specified by these priors.
\color{black}

\subsection{Results}
\label{results:sec}
The posterior of the model parameters is evaluated using Hamiltonian Monte Carlo as implemented in
STAN \citep{stan}.  We run eight chains, each with 5000 iterations of which
half are used for warmup.
STAN provides output statistics to assess
the convergence of the output Markov chains.
The 
potential scale reduction statistic, $\hat{R}$ \citep{Gelman92}, measures the convergence of the target distribution
in iterative simulations 
by using multiple independent sequences to estimate how much that distribution would sharpen if the simulations were run longer.
$N_{eff}$ is an estimate of the number of independent draws. The output for our data and model gives $\hat{R} \sim 1.0$ for all parameters, meaning there is no evidence for non-convergence.  The
output also gives  $N_{eff} \gg 100$ for all parameters, indicating that are all densely sampled.
Empirically, the confidence regions are localized and unimodal as is seen in  Figures~\ref{global1:fig} -- \ref{global5:fig}.  In these tests there is no evidence that
the Monte Carlo chains have not converged to the stationary posterior distribution.
We have rurun the analysis with a variety of initial conditions, including one with all the $\gamma$'s equal to zero, except for a small positive 
$\gamma^0_0$ and small negative $\gamma^1_0$; the 68\% credible intervals
remain consistent.

\begin{figure}[htbp] %  figure placement: here, top, bottom, or page
   \centering
   \includegraphics[width=5.2in]{output11/coeff0.pdf} 
            \caption{Posterior contours for $c$, $\alpha$, $\beta$, $\eta$, $\gamma^0$, $\gamma^1$, and $\sigma$ in the $U$ band.
            The contours shown here and in future plots represent 1-$\sigma$ in the parameter distribution (i.e.\ they should be
            projected onto the corresponding 1-d parameter axis), not to 68\%, 95\%, etc.\
            enclosed probability.  Lines for zero value for $\alpha$, $\beta$, $\eta$, $\gamma^0$, $\gamma^1$, and $\sigma$ are to guide the eye.
            \label{global1:fig}}
\end{figure}

\begin{figure}[htbp] %  figure placement: here, top, bottom, or page
   \centering
   \includegraphics[width=5.2in]{output11/coeff1.pdf} 
            \caption{Posterior contours for $c$, $\alpha$, $\beta$, $\eta$, $\gamma^0$, $\gamma^1$, and $\sigma$ in the $B$ band.
 \label{global2:fig}}
\end{figure}

\begin{figure}[htbp] %  figure placement: here, top, bottom, or page
   \centering
   \includegraphics[width=5.2in]{output11/coeff2.pdf} 
            \caption{Posterior contours for $c$, $\alpha$, $\beta$, $\eta$, $\gamma^0$, $\gamma^1$, and $\sigma$ in the $V$ band.
 \label{global3:fig}}
\end{figure}

\begin{figure}[htbp] %  figure placement: here, top, bottom, or page
   \centering
      \includegraphics[width=5.2in]{output11/coeff3.pdf} 
            \caption{Posterior contours for $c$, $\alpha$, $\beta$, $\eta$, $\gamma^0$, $\gamma^1$, and $\sigma$ in the $R$ band.
 \label{global4:fig}}
\end{figure}

\begin{figure}[htbp] %  figure placement: here, top, bottom, or page
   \centering
         \includegraphics[width=5.2in]{output11/coeff4.pdf} 
            \caption{Posterior contours for $c$, $\alpha$, $\beta$, $\eta$, $\gamma^0$, $\gamma^1$, and $\sigma$ in the $I$ band.
 \label{global5:fig}}
\end{figure}


Differences between the model-predicted versus observed supernova colors are shown in Figure~\ref{residual:fig}.
There is no evidence of a catastrophic fit nor of extreme
outlying data.

\begin{figure}[htbp] %  figure placement: here, top, bottom, or page
   \centering
   \includegraphics[width=5.2in]{output11/residual.pdf} 
            \caption{Difference between predicted versus observed colors versus the observed color.
            \label{residual:fig}}
\end{figure}


For each of the five filters, the 68\%  equal-tailed credible intervals for the global parameters $\alpha$, $\beta$, $\eta$, $\gamma^0$, $\gamma^1$, and $\sigma$
are given in Table~\ref{global:tab}.
As the $\gamma$ parameters are free up to a multiplicative constant and have non-zero values of $\gamma^i_V$,
their results are shown in terms of $\gamma^i_X/\gamma^i_V-1$.
Contours of the posterior surface for parameter pairs grouped by filter are shown in Figures~\ref{global1:fig} -- \ref{global5:fig} .


\begin{table}
\centering
\begin{tabular}{|c|c|c|c|c|c|}
\hline
Parameter & $X=U$ &$B$&$V$&$R$&$I$\\ \hline
$\alpha_X$
&
$0.0043^{+0.0009}_{-0.0009}$
&
$0.0016^{+0.0008}_{-0.0007}$
&
$0.0015^{+0.0006}_{-0.0006}$
&
$0.0015^{+0.0005}_{-0.0005}$
&
$0.0027^{+0.0005}_{-0.0005}$
\\
${\alpha_X/\alpha_V-1}$
&
$   1.9^{+   1.1}_{  -0.5}$
&
$   0.1^{+   0.1}_{  -0.2}$
&
$\dots$
&
$   0.0^{+   0.1}_{  -0.1}$
&
$   0.8^{+   0.8}_{  -0.3}$
\\
$\beta_X$
&
$ 0.032^{+ 0.003}_{-0.003}$
&
$ 0.026^{+ 0.002}_{-0.002}$
&
$ 0.026^{+ 0.002}_{-0.002}$
&
$ 0.021^{+ 0.002}_{-0.002}$
&
$ 0.020^{+ 0.002}_{-0.002}$
\\
${\beta_X/\beta_V-1}$
&
$  0.25^{+  0.05}_{ -0.05}$
&
$ -0.01^{+  0.03}_{ -0.03}$
&
$\dots$
&
$ -0.19^{+  0.01}_{ -0.01}$
&
$ -0.23^{+  0.03}_{ -0.03}$
\\
$\eta_X$
&
$-0.0002^{+0.0012}_{-0.0011}$
&
$0.0000^{+0.0010}_{-0.0009}$
&
$0.0005^{+0.0008}_{-0.0008}$
&
$0.0006^{+0.0007}_{-0.0007}$
&
$-0.0002^{+0.0006}_{-0.0006}$
\\
${\eta_X/\eta_V-1}$
&
$ -0.39^{+  2.22}_{ -1.82}$
&
$ -0.35^{+  1.58}_{ -1.20}$
&
$\dots$
&
$ -0.09^{+  0.29}_{ -0.28}$
&
$ -0.78^{+  1.48}_{ -1.11}$
\\
$\gamma^0_X$
&
$ 63.46^{+  3.70}_{ -3.66}$
&
$ 51.40^{+  3.09}_{ -3.07}$
&
$ 38.15^{+  2.70}_{ -2.63}$
&
$ 29.26^{+  2.40}_{ -2.35}$
&
$ 20.94^{+  2.24}_{ -2.25}$
\\
${\gamma^0_X/\gamma^0_V-1}$
&
$  0.66^{+  0.06}_{ -0.05}$
&
$  0.35^{+  0.03}_{ -0.03}$
&
$\dots$
&
$ -0.23^{+  0.01}_{ -0.02}$
&
$ -0.45^{+  0.03}_{ -0.03}$
\\
$\gamma^1_X$
&
$-10.34^{+  3.61}_{ -4.29}$
&
$-11.89^{+  2.95}_{ -3.56}$
&
$-14.48^{+  2.47}_{ -2.89}$
&
$-13.71^{+  2.14}_{ -2.45}$
&
$-12.51^{+  2.10}_{ -2.23}$
\\
${\gamma^1_X/\gamma^1_V-1}$
&
$ -0.28^{+  0.17}_{ -0.19}$
&
$ -0.18^{+  0.10}_{ -0.10}$
&
$\dots$
&
$ -0.05^{+  0.05}_{ -0.04}$
&
$ -0.14^{+  0.10}_{ -0.09}$
\\
$\sigma_X$
&
$ 0.060^{+ 0.012}_{-0.012}$
&
$ 0.033^{+ 0.007}_{-0.007}$
&
$ 0.021^{+ 0.004}_{-0.005}$
&
$ 0.011^{+ 0.008}_{-0.007}$
&
$ 0.044^{+ 0.005}_{-0.004}$
\\
\hline
\end{tabular}
\caption{68\% credible intervals for the Global Fit Parameters of the 2-Parameter Extrinsic Model in \S\ref{model:sec}.\label{global:tab}}
\end{table}

We find significant non-zero values for $\alpha$ and $\beta$, indicating that $EW_{Ca}$ and $EW_{Si}$ are indicators of broadband
peak magnitudes.
This validates our hypothesis that spectral indicators
are tracers of supernova absolute magnitude.  On the other hand, the values of $\eta$ (the coefficients attached to $\lambda_{Si}$) are consistent with zero
to  within one standard deviation.
The effect of spectral parameters on color is shown in the rows of $\alpha_X/\alpha_V-1$,  $\beta_X/\beta_V-1$, and  $\eta_X/\eta_V-1$
in Table~\ref{global:tab}.
Values of zero signify no color changes associated with magnitude changes.
Both $EW_{Ca}$ and $EW_{Si}$ are associated with color changes, though not in $B-V$ specifically.
We do not detect a significant association between
$\lambda_{Si}$ and color with this model.


The histogram of the medians of the grey offsets, $\Delta$, for all supernovae is shown in Figure~\ref{hist:fig}.  The distribution is non-Gaussian, 
has a standard deviation of
%-----
$0.10$
%-----
mag, and a broad tail in the positive (fainter) direction.
\begin{figure}[htbp] %  figure placement: here, top, bottom, or page
   \centering
   \includegraphics[width=5.2in]{output11/Delta_med_hist.pdf} 
   \caption{Histogram of the medians of the grey offset $\Delta$. 
   \label{hist:fig}}
\end{figure}


Non-trivial residual magnitude dispersions are captured in $C_c$.  The diagonal elements are captured by the $\sigma$ parameters;
the residual intrinsic dispersion ranges from
$\sim 0.01$ to 0.06 mag, significantly smaller
than the dispersion in $\Delta$.  Given that the 
off-diagonal elements of $C_c$ are parameterized by the Cholesky factors of a correlation matrix rather than the matrix elements themselves,
it is not straightforward to present correlation matrix confidences:
the average over Monte Carlo links of the
Cholesky factors will not  generally yield a correlation matrix.  
To characterize a typical posterior draw of $C_c$ we use the matrix that is the mean of all covariance realizations in the
chain, element by element.
For $UBVRI$ the matrix is
\begin{equation}
C_c(U,B,V,R,I)=
\begin{pmatrix}
\begin{array}{rrrrr}
0.0038 & 0.0010 & -0.0002 & -0.0000 & 0.0003 \\
0.0010 & 0.0011 & 0.0002 & -0.0000 & -0.0006 \\
-0.0002 & 0.0002 & 0.0004 & 0.0000 & -0.0002 \\
-0.0000 & -0.0000 & 0.0000 & 0.0002 & 0.0002 \\
0.0003 & -0.0006 & -0.0002 & 0.0002 & 0.0020
\end{array}
 \end{pmatrix} \text{mag}^2.
 \label{mag_cov:eqn}
 \end{equation}
\color{red}
The average correlation matrix is
\begin{equation}
Cor(U,B,V,R,I)=
\begin{pmatrix}
\begin{array}{rrrrr}
1.00 & 0.42 & -0.18 & -0.05 & 0.12 \\
0.42 & 1.00 & 0.26 & -0.09 & -0.39 \\
-0.18 & 0.26 & 1.00 & 0.09 & -0.22 \\
-0.05 & -0.09 & 0.09 & 1.00 & 0.27 \\
0.12 & -0.39 & -0.22 & 0.27 & 1.00
\end{array}
 \end{pmatrix}.
 \end{equation}
 \color{black}
The  covariance of the colors $U-V$, $B-V$, $V-R$, and $V-I$ is
expressed as the standard deviations and
correlation matrix
 \begin{equation}
 \sigma(U-V, B-V, V-R, V-I)=
 \begin{pmatrix}
0.068 & 0.034 & 0.023 & 0.053
  \end{pmatrix} \text{mag},
 \label{color_sd:eqn}
 \end{equation}
 \begin{equation}
 Cor(U-V, B-V, V-R, V-I)=
\begin{pmatrix}
\begin{array}{rrrr}
1.000 & 0.602 & -0.353 & -0.317 \\
0.602 & 1.000 & -0.197 & 0.082 \\
-0.353 & -0.197 & 1.000 & 0.647 \\
-0.317 & 0.082 & 0.647 & 1.000
\end{array}
\end{pmatrix}.
  \label{color_cor:eqn}
 \end{equation}
These color variances are smaller than those derived by \citet{2003A&A...404..901N} or \citet{2007ApJ...659..122J}, for cases where direct comparison can be made.

Each supernova is described by its parameters $EW_{Ca}$, $EW_{Si}$, $\lambda_{Si}$, $E_{\gamma^0}(B-V)=(\gamma^0_B-\gamma^0_V)k_0$, and
$E_{\gamma^1}(B-V)=(\gamma^1_B-\gamma^1_V)k_1$, as well as its grey offset
$\Delta$: their distributions for all Monte Carlo links for all supernovae are shown in Figure~\ref{perobject:fig}.
There is a core concentration in the  parameter-space, with around ten objects that occupy its outskirts.
Many outliers appear in the red tail of $E_{\gamma^0}(B-V)$, as would be expected for the (infrequent) selection of supernovae
heavily extinguished by host-galaxy dust.
Outliers  are also clearly distinguishable in  $EW_{Ca}$--$\lambda_{Si}$ space.   

\begin{figure}[htbp] %  figure placement: here, top, bottom, or page
   \centering
   \includegraphics[width=5.2in]{output11/perobject_corner.pdf} 
   \caption{Distributions for the supernova parameters $EW_{Ca}$, $EW_{Si}$, $\lambda_{Si}$, $E_{\gamma^0}(B-V)$, and $E_{\gamma^1}(B-V)$, as well as the grey offset
$\Delta$.  All Monte Carlo links are plotted, so that each supernova contributes a cloud of points.
   \label{perobject:fig}}
\end{figure}

The Pearson correlation coefficients for $\Delta$, $EW_{Ca}$, $EW_{Si}$, $\lambda_{Si}$, $E_{\gamma^0}(B-V)$, and $E_{\gamma^0}(B-V)$ are given in the matrix
\begin{multline}
Cor(\Delta, EW_{Ca}, EW_{Si}, \lambda_{Si}, E_{\gamma^0}(B-V), E_{\gamma^1}(B-V)) =\\
\begin{pmatrix}
\begin{array}{rrrrrr}
1.00 & 0.00 & -0.06 & -0.05 & 0.11 & 0.01 \\
0.00 & 1.00 & 0.10 & -0.26 & -0.07 & -0.01 \\
-0.06 & 0.10 & 1.00 & -0.14 & -0.17 & -0.03 \\
-0.05 & -0.26 & -0.14 & 1.00 & 0.03 & 0.03 \\
0.11 & -0.07 & -0.17 & 0.03 & 1.00 & 0.05 \\
0.01 & -0.01 & -0.03 & 0.03 & 0.05 & 1.00
\end{array}
\end{pmatrix}.
\end{multline}

The  strong correlation between $EW_{Ca}$ and $\lambda_{Si}$ is not surprising.   Si~II velocities have been
suggested as being able to distinguish two distinct SN~Ia populations
\citep{2009ApJ...699L.139W, 2013Sci...340..170W}.  CaII~H\&K, and more cleanly CaII~NIR photospheric features
(and 
high-velocity features if present), have been found to trend with light-curve shape  \citep{2012MNRAS.426.2359M}.
\citet{2014MNRAS.437..338C, 2014MNRAS.444.3258M} found a strong correlation between Si~II velocities and the
CaII~NIR photometric velocity feature, and found that the correlation is sensitive to the presence of high-velocity features.

We find weak correlations between our color-excess parameters and the input features, the strongest being
%---
$-0.17$
%---
 between
$E_{\gamma^0}(B-V)$ and $EW_{Si}$.
Recall that $EW_{Si}$ is correlated with light-curve shape, which is correlated with host-galaxy (including dust) properties 
\citep{2000AJ....120.1479H, 2003MNRAS.340.1057S}.
An extensive discussion of the correlations between the spectral features of the SNfactory data set can be found in \citet{chotard:thesis}
and \citet{leget:thesis}.

The range of SiII$\lambda$4130 equivalent widths is $\pm 20$~\AA\ whereas the width of the $B$-band is 851~\AA, so that its affect on magnitude
is
$2.5 \log{(20/850)} \sim 0.03$ mag.  
The implied span in $B$ magnitude based on $\beta_B$ is 0.54~mag.  Therefore $\beta_B$ cannot wholly be attributed to the flux deficit
from the line itself.
Similarly, the CaII H\&K equivalent widths have range $\pm 50$~\AA, while the width of the $U$ band is
701~\AA, so that its affect on magnitude
is
$2.5 \log{(50/701)} \sim 0.08$ mags.   The implied span in $U$ magnitude is  0.21~mag, so $\alpha_U$ cannot be completely due to the line itself.

\subsection{The Two Extrinsic Parameters $k_0$ and $k_1$}
\color{blue}
The two sets of  $\gamma^i$ parameters  are significantly non-zero. 
None of the 20000 links of 
our Monte Carlo chains for $\gamma$ exceed 0 (see Figures~\ref{global1:fig}--\ref{global5:fig}), we thus claim detection of the
influence of $k_0$ and $k_1$  on supernova magnitudes
with probability $(1-5\times 10^{-5})$.

There is a pair of parameter values,  $k_0^0$ and $k_1^0$, that correspond to supernovae experiencing no extrinsic effect.
The true extrinsic
extinction is
$A^T_X(k_0,k_1)= A_X(k_0,k_1) -  A_X(k^0_0,k^0_1)$ where $A_X(k_0,k_1) =  \gamma^0_X k_0 + \gamma^1_X k_1$.
The true extrinsic color excess is $E^T(B-V) = A^T_B-A^T_V$, so that the total-to-selective extinction is $R^T_V=A^T_V/E^T(B-V)$.
Our $\langle k_0 \rangle=0$ and  $\langle k_1 \rangle=0$ constraints
do not determine $k_0^0$ and $k_1^0$, and so the extrinsic extinction and color excess per supernova are not products of our analysis.

%our analysis does not specify their values.
%Recall that there are degeneracies between the intrinsic (Eqn.~\ref{ewsiv:eqn}) and
%extrinsic (Eqn.~\ref{dust:eqn}) parameters of our model.  Part of that degeneracy is
%broken by imposing $\langle k_0 \rangle=0$ and  $\langle k_1 \rangle=0$,
%but this choice does not specify the parameter values.


Nevertheless, we are able to examine aggregate properties of the extinction.  The extrinsic extinction and color relation can be
rearranged as
$$A_X(k_0,k_1) = R^T_V \left( A_B(k_0,k_1) -  A_V(k_0,k_1)\right) -R^T_V \left(A_B(k^0_0,k^0_1) + A_V(k^0_0,k^0_1)\right)+  A_X(k^0_0,k^0_1) .
$$
Making the assumption that it is the same for all supernovae ,
 $R^T_V$  can be determined as the slope of the linear relation between $A_X$ and $E(B-V)$.

Figure~\ref{avebv:fig} shows the relationship between $A_V$ and $E(B-V)$.
The full set of supernovae does not coincide with the expectations of a single linear law; rather the supernova population is peaked
on a ridge line, with a tail distributed below.  That is, for a fixed color excess, the amount of $V$-band extinction is bounded above
but can have values that extend toward lower values.
The assumption of a constant $R^T_V$ is not altogether correct; still we perform a linear fit that gives a slope of
%----- fitz11.py
$R^T_V=2.77 \pm 0.08$,
%----
where the uncertainty is the $68$\% credible interval and does not include goodness
 of fit.  

%\begin{comment}
%\color{blue}
%Our model alone cannot determine $R^T_V = A^T_V/E^T(B-V)$ for individual objects, as $k_0^0$ and $k_1^0$ are unconstrained.
%For a qualitative view, we choose zeros  such that the $V$-band extinction and color-excess of the bluest supernova roughly correspond to $E(B-V)=0$,
%$A_V=0$, and such that the ridge line 
%has close to constant $R^T_V$.
%\color{black}
%With this ansatz, we
%estimate $R^T_V$ per supernova shown in Figure~\ref{avebv:fig}, with the expectations of the
% \citet{1999PASP..111...63F} (hereafter referred to as \citetalias{1999PASP..111...63F})  model for fixed $R^F_V$ overplotted.
%With this choice of zeropoints, the ridge line corresponds to $R_V^F\lesssim 2.5$, and even lower values of $R_V^F$ below the ridge line.
%We emphasize that the shape of the locus in this plot (though not the relative positions of the points)  and the flatness
%of the ridge line is specific to our choice of zeropoints, and that our model does not specify the zeropoints.
%A specification of per-supernova $R_V$ requires a model that can identify supernovae that suffer no extrinsic effects,
%for example by modeling  a constraining distribution for the extrinsic parameters \citep{2016arXiv160904470M}.
%\color{black}
%
\begin{figure}[htbp] %  figure placement: here, top, bottom, or page
   \centering
   \includegraphics[width=5.2in]{output11/avebv.pdf}
   %\includegraphics[width=2.8in]{output11/rv.pdf}
   \caption{
   $A_V$ versus $E(B-V)$, and the slope $R_V$ of the linear fit to the data.
   %Right: Shown for qualitative purposes,
   %$R^T_V=A^T_V/E^T(B-V)$ 
   %versus $E^T(B-V)$ for the supernova sample, given an ansatz for the definition of an intrinsic supernova. 
   %Overplotted is the mean and standard deviation of those points in equally divided bins.
   Overplotted the expected values for the \citetalias{1999PASP..111...63F} model
   for specific choices of $R^F$. Note that these lines are not straight, which is apparent when comparing to the straight black $R_V=2.78$ line.
   \label{avebv:fig}}
\end{figure}
%\end{comment}

We now compare our model and its color-coefficient parameters $\gamma^0$ and $\gamma^1$ with the expectations of  dust models.
A direct analysis of the  \citetalias{1999PASP..111...63F} dust model is presented later
in \S\ref{model0:sec}.



%Until now we have considered $k_0$ and $k_1$ without association with any specific dust model, but they could be interpreted as
%parameters of a single  dust model.
%\color{red}
%Our model for broadband photometry does not specify how spectral features and color parameters affect the supernova spectrum, information
%necessary to predict the magnitude extinction of dust models.
%\color{blue}
%The lack of correspondence between dust models and broadband extinction is quantified when we apply an $R^F_V=2.5$ SN-frame dust to the
%observed spectra used in our sample, to yield a distribution in the quantity $R_V=A_V/(A_B-A_V)$ with mean and standard deviation of $3.11 \pm 0.35$.
%While an accurate study of fundamental dust properties would incorporate the dust law within the model itself, we now proceed nonetheless
%in making a connection between our model outputs and dust.

The 2-parameter linear model
\begin{equation}
A_X = a(X)  A_V + b(X) E(B-V)
\label{f99:eqn}
\end{equation}
approximately describes dust models.
For the case of
$R^F=2.5$ and $A^F_V=0.1$,
and the measurement of the SALT2
\citet{2007A&A...466...11G} $s=1$, $x_1=0$ SN~Ia template at $B$-band peak, the \citetalias{1999PASP..111...63F} model
gives
$a(U,B,V,R,I)=\{0.96,   1.00,   1.00,   0.97,   0.77\}$ and $b(U,B,V,R,I)=\{  1.77,   0.98,   0.12,  -0.50,  -0.53\}$.
The $F$ superscript is added to distinguish \citetalias{1999PASP..111...63F} parameters.
Over the ranges of
 $R^F$ and $A^F_V$,
and the wavelengths under consideration,  the values $a$ and $b$ change but only by $<5$\%.

Our model can be expressed as
\begin{equation}
A_X = \frac{\gamma^0_X}{\gamma^0_B-\gamma^0_V}  E_{\gamma^0}(B-V) +  \frac{\gamma^1_X}{\gamma^1_B-\gamma^1_V}  E_{\gamma^1}(B-V).
\end{equation}
Our median-fit values for the $\gamma$ coefficients,
%---- fitz11.py
$\frac{\gamma^0_X}{\gamma^0_B-\gamma^0_V}  =\{4.80 ,   3.89,   2.89,   2.22,   1.59\}$ and
$ \frac{\gamma^1_X}{\gamma^1_B-\gamma^1_V}=\{-3.83 ,  -4.42,  -5.42,  -5.15,  -4.71\}$,
%----
can be transformed to a basis composed of the $a(X)$ and $b(X)$ vectors plus a residual vector perpendicular to $a$ and $b$.
The result is
\begin{equation}
\begin{pmatrix}
 \frac{\gamma^0_X}{\gamma^0_B-\gamma^0_V} \\
\frac{\gamma^1_X}{\gamma^1_B-\gamma^1_V} 
\end{pmatrix}=
\begin{pmatrix}
\begin{array}{rr}
1.15 & 2.82 \\
0.72 & -5.27
\end{array}
\end{pmatrix} 
\begin{pmatrix}
a(X) \\
b(X)
\end{pmatrix}+
\begin{pmatrix}
\vec{\epsilon}_a \\
\vec{\epsilon}_b
\end{pmatrix},
\end{equation}
where the two residual vectors
\begin{align}
\vec{\epsilon}_a &=\{0.06, -0.07, -0.07,  0.07, 0.02\}, \\
\vec{\epsilon}_b & =\{-0.06, 0.16, -0.22, 0.33, -0.26\}
\end{align}
contribute  2\% and 5\% respectively to the total  length of the $\gamma$ vectors.
This result indicates that
the allowed variation in $UBVRI$ allowed by the \citetalias{1999PASP..111...63F} model and our best-fit model are closely aligned.

The above result is visualized in Figure~\ref{plane:fig}, which  shows two perspectives (left and right columns) of $UVI$- (top row) and $BVR$- (bottom row) spaces
of the vectors corresponding to $\gamma_X/(\gamma_B-\gamma_V)$ of our model (solid lines),
and $a(X)$, $b(X)$ of the \citetalias{1999PASP..111...63F} model (dashed lines).  
The two perspectives show that while the four vectors point in different directions for each band combination,
they are almost coplanar in $UVI$, and close to coplanar in $BVR$.

\begin{figure}[htbp] %  figure placement: here, top, bottom, or page
   \centering
   \includegraphics[width=2.95in]{output11/plane0.pdf}
   \includegraphics[width=2.95in]{output11/plane1.pdf}
   \includegraphics[width=2.95in]{output11/plane0BVR.pdf}
   \includegraphics[width=2.95in]{output11/plane1BVR.pdf}
   \caption{
   Two perspectives (left and right columns) of $UVI$- (top row) and $BVR$- (bottom row) spaces
of the vectors corresponding to $\gamma^i_X/(\gamma^i_B-\gamma^i_V)$ of our model (solid lines),
   and $a(X)$, $b(X)$ of the \citetalias{1999PASP..111...63F} model (dashed lines).  The combinations $a(X)+b(X)/2.4$ and  $a(X)+b(X)/2.6$ are shown in the dotted
   lines: thy are almost perfectly superimposed on $\gamma^0_X/(\gamma^0_B-\gamma^0_V)$ for $UVI$ and $BVR$ respectively.  $b(X)$ shown in the dotted line is
   almost  perfectly superimposed on $\gamma^1_X/(\gamma^1_B-\gamma^1_V)$ in $UVI$ but is clearly distinct in $BVR$.  
   \label{plane:fig}}
\end{figure}

A prior that $k_0$ and  $k_1$ be uncorrelated was enforced to break degeneracies in our linear models, and determines the choice of directions of
the vectors $\gamma^0$ and $\gamma^1$.  Suppose that that the data were described by a dust model.  A \citetalias{1999PASP..111...63F}  dust model 
can be written as 
%To see how this prior would operate on dust models, consider that themodel in Eqn.~\ref{f99:eqn} can
%be expressed in terms of new basis vectors parametrized by $\kappa$:
\begin{equation}
A_X =  (1-\kappa_1 \kappa_2)^{-1} [(A^F_V - \kappa_2 E^F(B-V))\left(a(X)+\kappa_1 b(X) \right) +  (-\kappa_1 A^F_V + E^F(B-V)) (\kappa_2 a(X) + b(X))],
\end{equation}
with the parameters $\kappa_1$ and $\kappa_2$ providing freedom in the directions of the two basis vectors.  In the analysis,
the $\kappa$'s are specified by the  prior of  
zero covariance between $(A^F_V - \kappa_2 E^F(B-V))$ and  $(-\kappa_1 A^F_V + E^F(B-V))$.
The preferred values for $\kappa$ are shown qualitatively in
Figure~\ref{plane:fig}.  For $\kappa_1=1/2.4$, $\kappa_2=-6.8$
the vector $\gamma^0_X$ is  aligned with 
the dust vector $\left(a(X)+\kappa_1 b(X) \right)$  and $\gamma^1_X$
is aligned with $(\kappa_2 a(X) + b(X))$  $UVI$-space.  Slightly shifted values of $\kappa_1=1/2.6$, $\kappa_2=-6.1$ apply in $BVR$-space.

To gain an intuitive understanding of $\kappa_1$, consider the hypothetical case where
all supernovae experience dust with the same $R^F$.   The zero-covariance
prior would be satisfied by  $\kappa_1=1/R^F$ and any arbitrary $\kappa_2$.
The combination  $\gamma^0_V k_0$ would correspond to  $A^F_V$.  Both $k_1=0$  and $\gamma^1$ would be unconstrained.
A slight relaxation away from the constant $R^F$ assumption would result in $\kappa_2$ no longer being arbitrary but determined by the prior, and a
non-zero $k_1$ and a $\gamma^1$.

Given the association of  our parameter $\gamma^0_V k_0$ with the $A^F_V$ of an $R^F \sim 1/\kappa_1$ dust model,
the former's distribution can be compared to the expectation for the latter.
Figure~\ref{k0_med:fig} shows the histogram of
median values of $\gamma^0_V k_0$.
The distribution is non-Gaussian, having a sharp rise in the blue and an extended tail in the red, consistent
with the expectation of dust considering the spatial distribution of supernovae within galaxies and the distribution of galaxy orientations with respect to the observer \citep{1998ApJ...502..177H}.

Figure~\ref{k0_med:fig} also shows the distribution of $\gamma^1_V k_1$, which is associated with 
extinction corrections due to deviations away from the canonical $R^F$ value.   The distribution has a long
negative tail and is peaked at its positive edge, and by construction has a mean of zero.
The implication is that while the bulk of supernovae have positive values associated with $R>1/\kappa_1$, there is a tail
of negative values associated with $R<1/\kappa_1$.  The plot of   $\gamma^0_V k_0$ versus $\gamma^1_V k_1$  shown in Figure~\ref{kk:fig} shows that
the positive edge in $\gamma^1_V k_1$ persists over the full range of  $\gamma^0_V k_0$. On the other hand, lower values of $\gamma^1_V k_1$
occur at larger values of  $\gamma^0_V k_0$.
These findings are consistent with previous results:
\citet{2014ApJ...789...32B, 2015MNRAS.453.3300A} deduce a wide range of dust behavior $1.5<R^F<3$ encountered by the SN~Ia population.
\citet{2011ApJ...729...55F} and the above authors
find that supernovae with large extinction 
preferentially exhibit low $R^F$. 


\begin{figure}[htbp] %  figure placement: here, top, bottom, or page
   \centering
   \includegraphics[width=2.8in]{output11/gamma0_med.pdf}
   \includegraphics[width=2.8in]{output11/gamma1_med.pdf}
      \caption{ Histogram of
median values of Left: $\gamma^0_V k_0$, which is associated with the $A^F_V$ of an $R^F \sim 2.4$ dust model; Right: $\gamma^1_V k_1$,
which  is associated with $\left( E^F(B-V) - \frac{ A_V^F}{2.4}\right)$, i.e.\
extinction corrections due to deviations away from the $R=2.4$ value.
   \label{k0_med:fig}}
\end{figure}

\begin{figure}[htbp] %  figure placement: here, top, bottom, or page
   \centering
   \includegraphics[width=5.2in]{output11/kk.pdf}
      \caption{ $\gamma^0_V k_0$ versus $\gamma^1_V k_1$.
   \label{kk:fig}}
\end{figure}
%
%The model parameters $E_{\gamma^0}(B-V)$ and  $E_{\gamma^1}(B-V)$ for the supernovae in our sample
%are projected onto the  \citetalias{1999PASP..111...63F} values of $E^F(B-V)$ and $A^F$, up to an additive constant:  these are shown in Figure~\ref{avebv:fig}.
%The points are confined to a tight locus around a line, with slope that corresponds to $R^F_V=2.24 \pm 0.16$.  
%The linear fit also produces the zeropoint, allowing us to determine individual $R^F_V$ for all supernovae,
%which are shown in Figure~\ref{avebv_synth:fig}.   The relative values of the projected parameters are similar to
%the native $R_V$, $A_V$, and $E(B-V)$  shown in Figure~\ref{avebv:fig}.
%
%\begin{figure}[htbp] %  figure placement: here, top, bottom, or page
%   \centering
%   \includegraphics[width=2.8in]{output11/avebv_synth.pdf}
%   \includegraphics[width=2.8in]{output11/avrv_synth.pdf}
%    \caption{Left: Projections of our model parameters $E_{\gamma^0}(B-V)$ and  $E_{\gamma^1}(B-V)$ for the supernovae in our sample
%onto the  \citetalias{1999PASP..111...63F} dust-model plane.  Note that the error bars do not capture the strong
%covariance between the parameters. Right: $R^F_V$ for each supernova.
%   \label{avebv_synth:fig}}
%\end{figure}
%


Another comparison between our model predictions with those of dust is presented in
Figure~\ref{avebv:fig}, which shows the expectations for measured $A(V)$, $E(B-V)$ for $R^F=1.1, 1.9, 2.5, 3.1$.
Note that for our broadband measurements, the relationship between $A(V)$ and $E(B-V)$ is subtly
non-linear.
Qualitatively, the $R^F \sim 2.4$  most closely matches our linear model fit.  

The near coplanarity of the plane defined by our parameters and the plane of  \citetalias{1999PASP..111...63F}, together with the fact that our  supernova
parameters translate to  \citetalias{1999PASP..111...63F} parameters with consistent distributions, lead us to conclude that 
$\{k_0, k_1\}$, describe for the most part the $\{ E(B-V), R\}$ parameters used
in standard models of Milky Way dust.  Yet there do remain some small inconsistencies between our numbers and expectations from dust models.
\color{black}

\subsection{SN 2014J in the Context of Our Model}
\label{sn2014j:sec}
SN~2014J   is one of several SNe~Ia that exhibits colors that imply a low $R_V<2.0$ \citep{2014ApJ...788L..21A, 2014MNRAS.443.2887F, 
2014arXiv1411.3332J,
2014ApJ...795L...4K, 2015ApJ...805...74B}.
The colors of this supernova can be studied in comparison to SN~2011fe, an object with similar
spectral evolution that 
suffers very low galactic and host reddening
\citep[this technique has been used in][]{2006MNRAS.369.1880E,2007AJ....133...58K,2008MNRAS.384..107E,2010AJ....139..120F, 2014ApJ...788L..21A}.
The overall higher photospheric velocities of
SN~2014J are unimportant  based on the findings of  \S\ref{results:sec}.

Data of SN~2014J are  analyzed using our model and using the parameter values found with the SNfactory sample.
The data for the color excesses  in $UBRi$  relative to $V$ at peak brightness  are taken from \citet{2014ApJ...788L..21A},
following their prescription of averaging measurements within 5-days of peak $B$ magnitude.
Their values 
$E_o(U-V) =   2.23 \pm   0.03$,
$E_o(B-V) =   1.28 \pm   0.04$,
$E_o(R-V) =  -0.47 \pm   0.03$,
$E_o(i-V) =  -0.92 \pm   0.03$
are plotted in Figure~\ref{sn2014j:fig}.
\color{red}
The central wavelengths of the filters are 3660, 4280, 5350, 6310, and 8100 \AA\ respectively, in contrast to the central
wavelengths 
3640, 4390, 5290, 6370, 7680 \AA\ of our synthetic bands.  Despite the imperfect correspondence our synthetic passbands
and those used for the observations,
\color{black}
these data are fit to our model
\begin{equation}
E_o(X-V) =  \left(\frac{\gamma^0_X}{\gamma^0_V}-1\right)x_0 +  \left(\frac{\gamma^1_X}{\gamma^1_V}-1\right)x_1,
\end{equation}
where we use the median values of the $\gamma$- and $\delta$-terms from Table~\ref{global:tab} and the fit
parameters $x_i $.

\begin{figure}[htbp] %  figure placement: here, top, bottom, or page
   \centering
   \includegraphics[width=5.2in]{output11/sn2014j.pdf} 
   \caption{Measured color excess for SN~2014J and the best-fit predictions from this article. 
   \color{red}
   This can be compared with the best-fit results of
   \citet{2014ApJ...788L..21A} shown in their Figure~3.
   \color{black}
   \label{sn2014j:fig}}
\end{figure}

The best-fit parameters for SN~2014J are 
%----
$x_0= 2.60$, $ x_1=-1.89$ with covariance
%----
\begin{equation}
\begin{pmatrix}
\begin{array}{rr}
0.050 & 0.058 \\
0.058 & 0.315
\end{array}
\end{pmatrix}.
\end{equation}
The predicted values of observed $E_o(B-V)$ from the fit are shown in Figure~\ref{sn2014j:fig}, where they are found to
overlap with the data within the error bars.   For comparison, the best-fit model determined by  \citet{2014ApJ...788L..21A} model using
UV through NIR data,
a  \citet{1999PASP..111...63F} dust with $R_V^F=1.4$ and $E(B-V)=1.37$
\color{red}
is shown in their Figure 3.
\color{black}
Our model predictions provide a significantly better match to the data
\color{red}
despite the only approximate correspondence of the model and observer passbands.
\color{black}

In the fit to our model, the observed color excess is attributed to 
%----
$E_{\gamma^0}(B-V)=  0.90 \pm   0.08$ and  $E_{\gamma^1}(B-V)=  0.34 \pm   0.10$
%-----
contributions.
Among the supernovae in the SNfactory  set used to determine $\gamma^0$ and $\gamma^1$, the
objects with the extreme median values have 
%---
$E_{\gamma^0}(B-V)$ are $-0.07 \pm 0.01$ and  $  0.33 \pm 0.04$,
and for $E_{\gamma^1}(B-V)$ $-0.01 \pm 0.01$  and
$  0.06 \pm 0.03$ 
%---
(see Figure~\ref{ebv:fig}).
The deduced parameters for SN~2014J, which are relative to SN~2011fe, live well outside the 
span of supernovae used to train the coefficients of the model, which is not unexpected if
attributed to dust. 



\section{Supernova Model II: Two Extrinsic Parameters, One Intrinsic Parameter}
\label{model2:sec}
In \S\ref{model:sec} we show that the SNfactory data analyzed with an agnostic model with
two extrinsic parameters produce
magnitudes consistent
with two-parameter dust models that have been established for stars in the Milky Way, LMC, and SMC.
An extension to our model is now added to explore the possibility of
a possible additional
color parameter that could have its origins in supernova astrophysics.
The extended model is written as
\begin{equation}
\begin{pmatrix}
U\\B\\V\\R\\I
\end{pmatrix}
\sim \mathcal{N}
\left(
\Delta +
\begin{pmatrix}
c_U+\alpha_U EW_{Ca} + \beta_U EW_{Si} + \eta_U \lambda_{Si} + \delta_0 D\\
c_B+\alpha_B EW_{Ca} + \beta_B EW_{Si} + \eta_B \lambda_{Si} + \delta_1 D \\
c_V+\alpha_V EW_{Ca} + \beta_V EW_{Si} + \eta_V \lambda_{Si} + \delta_2 D\\
c_R+\alpha_R EW_{Ca} + \beta_R EW_{Si} + \eta_R \lambda_{Si} + \delta_3 D\\
c_I+\alpha_I EW_{Ca} + \beta_I EW_{Si}+ \eta_I \lambda_{Si} + \delta_4 D
\end{pmatrix}
,C_{c}
\right)
\label{ewsiv:eqn}
\end{equation}
\begin{equation}
\begin{pmatrix}
U_o\\B_o\\ V_o\\R_o\\I_o\\EW_{Si, o}\\ EW_{Ca, o} \\ \lambda_{Si, o}
\end{pmatrix}
\sim \mathcal{N}
\left(
\begin{pmatrix}
U +\gamma^0_{U} k_0 +\gamma^1_{B} k_1 \\B +\gamma^0_{B} k_0 +\gamma^1_{B} k_1 \\
V+\gamma^0_{V} k_0+\gamma^1_{V} k_1\\R+\gamma^0_{R} k_0 + \gamma^1_{R} k_1\\I+\gamma^0_{I} k_0+\gamma^1_{I} k_1\\
EW_{Si}\\ EW_{Ca} \\ \lambda_{Si}
\end{pmatrix}
,C
\right).
\label{dust:eqn}
\end{equation}
In all aspects it is identical to the previous model except for the addition of the term $\delta D$,
which affects the mean
absolute magnitudes:
each supernova has an intrinsic $D$ parameter that is linearly related to
absolute magnitudes through global coefficients $\delta$.  As with $k_0$ and $k_1$,  $D$ is a latent supernova parameter whose existence is inferred from
its effects on broadband magnitudes.


\color{red}
For $N$ supernovae there are $8N$ observables.  For the top-level model parameters, there are $3N$ spectral parameters, $3(N-2)$
color parameters, $(N-1)$ magnitude offsets,  $5 \times 7$ global coefficients, and $10$ parameters that describe the intrinsic
$C_c$.  For $N=172$ supernovae, there are 1376 observables and 1242 top-level parameters.
\color{black}


\color{red}
The $\delta D$ terms contribute to the mean of the Normal distribution in Eqn.~\ref{ewsiv:eqn}, together with the spectral-feature terms:
it is for this reason that for convenience these terms are referred to as ``intrinsic''.
However, we emphasize now that we have also run the alternative case  of placing the
$\delta D$ terms in mean of the Normal distribution in Eqn.~\ref{dust:eqn}, i.e.\ as extrinsic contributions.
Analyses of the two different models give practically identical results, meaning that data are not constraining enough
to identify $\delta D$ as being physically intrinsic or extrinsic.
\color{black}

To constrain the new degrees of freedom and degeneracies inherent in the model we impose that
$D+1/N$ be a simplex to break the 
degeneracy  $\delta \rightarrow a\delta$, $D \rightarrow a^{-1}D$.
Furthermore we impose
\begin{equation}
\langle D\rangle=0,  \delta \cdot \gamma^0=0, \delta \cdot \gamma^1=0.
\end{equation}
The first constraint specifies the definition of zero color.   The latter two confine possible magnitude
variations to those that cannot be attributed to 
the $\gamma k$ terms (i.e.\ dust); this condition breaks the degeneracies between $\delta D$  
and $\gamma k$. 
Unlike for the $\gamma$'s, there is no condition on the sign of $\delta_0$,
so that the posterior has degenerate solutions with sign flips of $\delta$ and $D$.
\color{blue}
We apply the condition $\delta_4>0$ to break this degeneracy,
and assume a flat prior for the three linear components of the subspace perpendicular to the $\gamma$ plane.
\color{black}
In the running of the Monte Carlo, the initial conditions for
$\gamma^0$ and $\gamma^1$ were set to be in the neighborhood of the best fit of \S\ref{model:sec}
to maintain their correspondence between the two models. 

The resulting credible intervals of our parameters are given in Table~\ref{global2:tab}.  Comparison with the
results of the 2-parameter extrinsic model given in Table~\ref{global:tab} shows that  parameters that are common to both
are consistent well within 1$\sigma$ uncertainties.  The most significant change is in the slight reduction of the standard
deviations of the magnitude residuals.
The newly fitted $\gamma$ parameters have an almost identical association with the  \citetalias{1999PASP..111...63F} dust model, in that now
\begin{equation}
\begin{pmatrix}
 \frac{\gamma^0_X}{\gamma^0_1-\gamma^0_2} \\
\frac{\gamma^1_X}{\gamma^1_1-\gamma^1_2} 
\end{pmatrix}=
\begin{pmatrix}
\begin{array}{rr}
1.22 & 2.73 \\
0.77 & -4.98
\end{array}
\end{pmatrix} 
\begin{pmatrix}
a(X) \\
b(X)
\end{pmatrix}+
\begin{pmatrix}
\vec{\epsilon}_a \\
\vec{\epsilon}_b
\end{pmatrix},
\end{equation}
where the residual vectors 
\color{red}
\begin{align}
\vec{\epsilon}_a &=\{0.13, -0.15, -0.06,  0.04,  0.06\},\\
\vec{\epsilon}_b &=\{ -0.04, 0.09,  -0.18,  0.28, -0.19\}
\end{align}
\color{black}
are perpendicular to the  \citetalias{1999PASP..111...63F} plane and
contribute similar 3\% and 4\% to the total  length of the vector.
%The effective total-to-selective extinction for the supernova sample has a slight shift to $R^F_V=2.23 \pm 0.16$.

\begin{table}
\centering
\begin{tabular}{|c|c|c|c|c|c|}
\hline
Parameters& $X=U$ &$B$&$V$&$R$&$I$\\ \hline

$\alpha_X$
&
$0.0045^{+0.0009}_{-0.0010}$
&
$0.0017^{+0.0008}_{-0.0008}$
&
$0.0015^{+0.0006}_{-0.0006}$
&
$0.0015^{+0.0005}_{-0.0005}$
&
$0.0027^{+0.0005}_{-0.0005}$
\\
${\alpha_X/\alpha_V-1}$
&
$   1.9^{+   1.1}_{  -0.5}$
&
$   0.1^{+   0.1}_{  -0.2}$
&
$  \ldots$
&
$   0.0^{+   0.1}_{  -0.1}$
&
$   0.8^{+   0.8}_{  -0.3}$
\\
$\beta_X$
&
$ 0.033^{+ 0.003}_{-0.003}$
&
$ 0.026^{+ 0.002}_{-0.002}$
&
$ 0.026^{+ 0.002}_{-0.002}$
&
$ 0.021^{+ 0.002}_{-0.002}$
&
$ 0.020^{+ 0.002}_{-0.002}$
\\
${\beta_X/\beta_V-1}$
&
$  0.25^{+  0.04}_{ -0.05}$
&
$ -0.01^{+  0.02}_{ -0.03}$
&
$  \ldots$
&
$ -0.19^{+  0.01}_{ -0.01}$
&
$ -0.24^{+  0.03}_{ -0.03}$
\\
$\eta_X$
&
$0.0002^{+0.0012}_{-0.0012}$
&
$0.0002^{+0.0010}_{-0.0010}$
&
$0.0007^{+0.0008}_{-0.0008}$
&
$0.0007^{+0.0007}_{-0.0007}$
&
$0.0000^{+0.0006}_{-0.0006}$
\\
${\eta_X/\eta_V-1}$
&
$ -0.25^{+  1.35}_{ -1.38}$
&
$ -0.34^{+  1.21}_{ -1.04}$
&
$  \ldots$
&
$ -0.05^{+  0.34}_{ -0.29}$
&
$ -0.70^{+  0.84}_{ -0.88}$
\\
$\gamma^0_X$
&
$ 64.01^{+  3.79}_{ -3.69}$
&
$ 51.96^{+  3.18}_{ -3.10}$
&
$ 38.49^{+  2.70}_{ -2.63}$
&
$ 29.38^{+  2.40}_{ -2.33}$
&
$ 20.86^{+  2.19}_{ -2.22}$
\\
${\gamma^0_X/\gamma^0_V-1}$
&
$  0.66^{+  0.06}_{ -0.05}$
&
$  0.35^{+  0.03}_{ -0.03}$
&
$  \ldots$
&
$ -0.24^{+  0.01}_{ -0.02}$
&
$ -0.46^{+  0.03}_{ -0.03}$
\\
$\gamma^1_X$
&
$-10.18^{+  3.53}_{ -3.92}$
&
$-11.64^{+  2.95}_{ -3.23}$
&
$-14.42^{+  2.45}_{ -2.72}$
&
$-13.73^{+  2.09}_{ -2.31}$
&
$-12.70^{+  2.03}_{ -2.16}$
\\
${\gamma^1_X/\gamma^1_V-1}$
&
$ -0.29^{+  0.16}_{ -0.18}$
&
$ -0.19^{+  0.09}_{ -0.10}$
&
$  \ldots$
&
$ -0.05^{+  0.05}_{ -0.04}$
&
$ -0.12^{+  0.10}_{ -0.09}$
\\
$\delta_X$
&
$  2.07^{+  0.69}_{ -0.74}$
&
$ -2.25^{+  0.78}_{ -0.66}$
&
$ -1.85^{+  0.63}_{ -0.60}$
&
$  0.41^{+  0.53}_{ -0.57}$
&
$  2.02^{+  0.97}_{ -1.08}$
\\
${{\delta_X/\delta_I-1}}$
&
$  0.04^{+  0.52}_{ -0.27}$
&
$ -2.09^{+  0.27}_{ -0.62}$
&
$ -1.92^{+  0.25}_{ -0.56}$
&
$ -0.80^{+  0.68}_{ -0.26}$
&
$  \ldots$
\\
$\sigma_X$
&
$ 0.053^{+ 0.013}_{-0.012}$
&
$ 0.029^{+ 0.008}_{-0.009}$
&
$ 0.018^{+ 0.004}_{-0.005}$
&
$ 0.009^{+ 0.007}_{-0.006}$
&
$ 0.042^{+ 0.005}_{-0.004}$
\\
\hline
\end{tabular}
\caption{68\% credible intervals for the Global Fit Parameters of the Extrinsic--Intrinsic Model in \S\ref{model2:sec}.\label{global2:tab}}
\end{table}



The distributions of both sources of color excess $E(B-V)$ and
 $E_\delta(B-V) = (\delta_B-\delta_V)D$ are shown in the ideograms in Figure~\ref{ebv:fig}.
The standard deviations of $E_\gamma(B-V)$ and $E_\delta(B-V)$ are
%-----
0.081
and 0.010
%-----
mag respectively.
The range of $B-V$ colors is $\sim 8$ times larger for the external contribution; variation in $B-V$ color due to the intrinsic term is dwarfed by the extrinsic dust
component. 

\begin{figure}[htbp] %  figure placement: here, top, bottom, or page
   \centering
   \includegraphics[width=2.8in]{output23/ebv.pdf}
   \includegraphics[width=2.8in]{output23/ebv_delta.pdf}
      \caption{Left: Ideograms of the external $E(B-V)$ and
   internal $E_\delta(B-V) = (\delta_B-\delta_V)D$  contributions to color excess  for the supernovae in our sample.
   Right: Ideogram of $E_\delta(B-V)$ on an enlarged scale.
   \label{ebv:fig}}
\end{figure}
The significance of the $\delta$ parameters is shown pictorially in
Figure~\ref{deltacorner:fig}, which shows the $\delta$ posteriors.
\color{red}
Recall that fixing $\delta_X=0$ reverts this model into the two extrinsic-parameter model of \S\ref{model:sec}.
\color{black}
Assigning the Monte Carlo links to bins of width 0.5 in each $\delta$ parameter, the maximum number of links in a single bin is 303, the number of bins
with at least one element is 1987, and the bin that contains $\delta=0$ for all bands has zero elements; we thus claim a detection
of a third color-parameter at $> (1-5\times 10^{-4})$.
Figure~\ref{deltaratio:fig} shows  values of $\delta_X/\delta_I-1$ that  have significant non-zero values.
\color{red}
While the signal of the internal parameter is strong within the context of this model, ``model testing'' (comparing the
performance  between this model
and that of \S\ref{model:sec}), is challenging for Bayesian Monte Carlo analysis.  Evidence calculations are planned
for the next generation of STAN and the development of robust algorithms is a subject of theoretical work.
\color{black}

An interesting feature
is that the shift in magnitude is not monotonic in wavelength, but rather decreases from $U$ to $V$, and then increases from $V$
to $I$.
These are not the behaviors expected from normal dust,
circumstellar dust suggested as a source of SN extinction \citep{2005ApJ...635L..33W,2008ApJ...686L.103G,
2015ApJ...807L..26G} and so may be indicative of underlying supernova physics beyond that captured by $EW_{Ca}$, $EW_{Si}$ and
$\lambda_{Si}$.

\begin{figure}[htbp] %  figure placement: here, top, bottom, or page
   \centering
   \includegraphics[width=5.2in]{output23/delta_corner.pdf}
   \caption{Significance of the non-zero values of $\delta$ seen through the posterior contours for $\delta$.
   \label{deltacorner:fig}}
\end{figure}


\begin{figure}[htbp] %  figure placement: here, top, bottom, or page
   \centering
      \includegraphics[width=5.2in]{output23/deltaratio.pdf}
   \caption{Significance of the non-zero values of $\delta$ seen through $\delta_X/\delta_I-1$.
   \label{deltaratio:fig}}
\end{figure}

The Pearson correlation coefficients for $\Delta$, $EW_{Ca}$, $EW_{Si}$, $\lambda_{Si}$, $E_{\gamma^0}(B-V)$, $E_{\gamma^1}(B-V)$,  $A_{\delta I}$ are given in the matrix.
\begin{multline}
Cor(\Delta, EW_{Ca}, EW_{Si}, \lambda_{Si}, E_{\gamma^0}(B-V), E_{\gamma^1}(B-V),  A_{\delta I}) =\\
\begin{pmatrix}
\begin{array}{rrrrrrr}
1.00 & 0.00 & -0.07 & -0.04 & 0.11 & 0.01 & -0.06 \\
0.00 & 1.00 & 0.10 & -0.26 & -0.07 & -0.02 & -0.02 \\
-0.07 & 0.10 & 1.00 & -0.14 & -0.17 & -0.04 & 0.04 \\
-0.04 & -0.26 & -0.14 & 1.00 & 0.03 & 0.03 & -0.07 \\
0.11 & -0.07 & -0.17 & 0.03 & 1.00 & 0.05 & 0.06 \\
0.01 & -0.02 & -0.04 & 0.03 & 0.05 & 1.00 & 0.01 \\
-0.06 & -0.02 & 0.04 & -0.07 & 0.06 & 0.01 & 1.00
\end{array}
\end{pmatrix}.
\end{multline}
The correlation coefficients between the ``intrinsic'' and ``extrinsic'' parameters are $0.06$ and $0.01$.
This has two important implications:
First, while in principle a single physical color effect can affect both intrinsic and extrinsic parameters, from the correlations
there is no evidence
that this occurs.
Second, the intrinsic term does not a second-order extrinsic correction.

The distributions for $\Delta$, $EW_{Ca}$, $EW_{Si}$, $\lambda_{Si}$, $E_{\gamma^0}(B-V)$, 
$E_{\gamma^1}(B-V)$, and
$E_{\delta}(B-V)$ for all Monte Carlo links for all supernovae are shown in Figure~\ref{perobject2:fig}.

\begin{figure}[htbp] %  figure placement: here, top, bottom, or page
   \centering
   \includegraphics[width=5.2in]{output23/perobject_corner.pdf} 
   \caption{
   \color{red}
   Distributions for the supernova parameters $\Delta$, $EW_{Ca}$, $EW_{Si}$, $\lambda_{Si}$, $E_{\gamma^0}(B-V)$,  $E_{\gamma^1}(B-V)$,  and $E_{\delta}(B-V)$, as well as the grey offset
$\Delta$.  All Monte Carlo links are plotted, so that each supernova contributes a cloud of points.
\color{black}
   \label{perobject2:fig}}
\end{figure}

The analysis is validated using simulated datasets with and without third color parameters: the input parameters are recovered within  uncertainties.
\color{black}


\section{Supernova Model 3: Fitzpatrick Dust}
\label{model0:sec}
\color{blue}
Dust models can explain the extrinsic magnitude and color variations found in
the linear models of \S\ref{model:sec} and \S\ref{model2:sec}.  In this section, we replace the $\gamma k$ color terms with extinction
terms predicted by a dust model.  
%
%We analyze our data with a model that incorporates attenuation due to host-galaxy dust: it is identical to that of \S\ref{model:sec}
%except that dust
%is used in place of  the linear two-color-parameter  model.      
The dust model of  \citet{1999PASP..111...63F} is considered:
while commonly used dust
models are qualitatively similar in the optical wavelengths relevant to our data, that of  \citetalias{1999PASP..111...63F} 
does relatively well in reproducing supernova observations \citep{2041-8205-788-2-L21, 2016AAS...22723705H}.


The  \citetalias{1999PASP..111...63F} dust model has two parameters, $R$ and $E(B-V)$.  The first, $R$,
specifies the wavelength-dependent  attenuation  that we notate as $A/E(B-V)$.  The second, $E(B-V)$, specifies the normalization
of that attenuation.  These two parameters can be combined to define a third dependent parameter $A_V \equiv R E(B-V)$.
The dust model is expressed as
\begin{equation}
A(\lambda; E(B-V), R) = E(B-V) \frac{A}{E(B-V)}\left(\lambda; R \right).
\end{equation}

In the broadband photometry of a source with SED $f(\lambda)$ observed in a filter with transmission $X$, the attenuation $A_X$ is
\begin{equation}
A_X\left(A_V, E(B-V)  \right) = 2.5 \log{\left(\frac{\int d\lambda f(\lambda) X(\lambda)}{\int d\lambda 10^{-0.4 A\left(\lambda; E(B-V), R=\frac{A_V}{E(B-V)}\right)} f(\lambda) X(\lambda)}\right)},
\end{equation}
whose first-order approximation
\begin{equation}
A_X\left(A_V, E(B-V)  \right) = \frac{\int d\lambda A\left(\lambda; E(B-V), R=\frac{A_V}{E(B-V)}\right)f(\lambda) X(\lambda)}{\int d\lambda f(\lambda) X(\lambda)}
\end{equation}
is easy to interpret as an effective extinction.


We are  faced with a conundrum: our broad-band model does not specify $f(\lambda)$.  As an ansatz,
the SED of the SALT2, $c=0$, $x_1=0$ supernova at peak $B$ brightness is adopted as the SED of an unextincted
SN~Ia, independent of spectral-feature parameters.  This choice contrasts with the models in the previous sections, which do not specify the
parameter values of an unextincted
supernova.
The SALT2 SED determines the values of $A_X$ and the
intrinsic colors. 

This ansatz is  inconsistent with the results of the linear models evaluated with SNf data presented in previous sections in that the colors of a canonical supernova are not accessible by a SALT2 SED experiencing  \citetalias{1999PASP..111...63F} dust.
Taking the SALT2 template as the unextincted supernova, the canonical supernova with colors given by the best-fit $c_X$'s in \S\ref{model:sec} has the broad-band color excesses $E(X-V)$ shown in
Figure~\ref{model1118:fig}.  Overplotted are representative $E(X-V)$  allowed by dust.
The canonical supernova is in tension with colors induced by dust at extreme wavelengths, toward the red in the $U$ but blue in the $I$,
while simultaneously showing no color excess is $B-R$.
 
\begin{figure}[htbp] %  figure placement: here, top, bottom, or page
   \centering
   \includegraphics[width=2.8in]{output18/model1118ebv.pdf}
   \includegraphics[width=2.8in]{output18/comptolinear.pdf}
    \caption{Left: Broad-band color excesses $E(X-V)$ of the best-fit $c$'s of \S\ref{model:sec} 
    taking the SALT2 template as an unextinced supernova.  Overplotted are possible $E(X-V)$ for a fixed $A_V$ and a range of $R$,
allowed by \citetalias{1999PASP..111...63F} dust.
Right:    $A_X/E(X-V)$ of the best-fit $c$'s 
    relative to SALT2. \label{model1118:fig}}
\end{figure}


%
%We emphasize now that our broadband magnitude model does not ambiguously specify $A_X$, which depend
%on the SED of the source.  
%In our model for broadband magnitudes, the SEDs of supernovae $f(\lambda)$ are not specified.  In its place we use the 
%updated \citet{2007ApJ...663.1187H} template at peak (the $x_1=0$, $c=0$
%SALT2 template yields similar results.)
%The Hsiao template does not perfectly represent the underlying spectrum of a supernova and hence is a source
%of model error.  Consider the SED to be the sum of the template and an offset
%\begin{equation}
%f(\lambda) = f_{Hsiao}(\lambda) + \delta f(\lambda).
%\end{equation}
%To first order, deviations of the true spectrum from the template yield small additive and multiplicative corrections to the template-based
%extinction.  Heterogeneity in the true spectrum amongst supernovae produce scatter on top of those corrections.

%For the purposes of efficiency in our fitter algorithm, we approximate the \citetalias{1999PASP..111...63F} dust model in the range
%$-0.21 \le A_V \le 1.8$, $2.1 \le R \le 6.9$ with a third-order polynomial in $A_V$ and $E(B-V)$.   The limits in $R$ represent the range of
%values that enter the dust-model training.
%The differences between the extinction and the polynomial approximation are small relative to  $\partial A_X/ \partial R$ over the scales of
%$R$ of interest.

%
%
%
%\begin{figure}[htbp] %  figure placement: here, top, bottom, or page
%   \centering
%   \includegraphics[width=5.2in]{output18/dfitz.pdf}
%    \caption{Residuals between the  broadband extinction of the updated \citet{2007ApJ...663.1187H} template at peak due to $A_V=1.8$
%    \citetalias{1999PASP..111...63F} dust,  and the polynomial approximation used in this analysis.
%   \label{dfitz:fig}}
%\end{figure}
%


The model we consider in this section is identical to the linear model described by Eqns.~\ref{ewsiv:eqn}
and
\ref{dust:eqn} in \S\ref{model:sec}, except that the colors of a zero-extinction supernova (parameterized by differences in $c_X$) are fixed to those of the SALT2 template,
and the $\gamma k$ terms are replaced with $A_X(A_V=R E(B-V), E(B-V))$.   Each supernova has an $E(B-V)$ parameter 
and there is one global $R$ parameter.  
%
%
%Three distinct analyses are run with the  \citetalias{1999PASP..111...63F}  dust model.  In the first (denoted
%as $R=const$), $R$ is held the same for all supernovae.  In the second  (denoted as $\ln{R} \sim \mathcal{N}$),
%the $\ln{R}$ for each supernova is taken to be drawn from a Normal distribution with dispersion $\sigma_{\ln{R}}$.
%In the third (denoted as $R$-free), $A_V$ and $R$ are free for
%each supernova, with a uniform prior $-0.21\le A_V \le 1.8$, $2.1 \le R \le 6.9$.

Results from evaluating this model with SNf data
provide further evidence of the inconsistency between the ansatz and the models of the previous section.
The dust-model analysis of this section sets the magnitude of an $A_V=0$ supernova, so now  magnitudes of the canonical
supernova in \S\ref{model:sec} can also be compared (along with color) to the expectations of the dust model.  The magnitude difference, $\Delta c_X$,
is the difference between the $c_X$ parameters of the two analyses.
Then  $\Delta c_X / (\Delta c_B - \Delta c_V)$
corresponds to  $A_X/E(B-V)$ of the canonical supernova relative
to SALT2: these values are plotted in
Figure~\ref{model1118:fig}.
The negative values for $X=BVRI$ and the non-monotonic behavior is inconsistent with the expectations of dust.

The analysis results for the dust parameters are shown in Figure~\ref{AVRV:fig}.  The left plot shows the histogram of the median $A_V$ determined for
each supernova. The
$A_V$ distribution  has a slow rise from zero; recall that the equivalent $\gamma^0_V k_0$
from the linear model of \S\ref{model:sec} has the sharp rise expected from of a large fraction of supernovae
experiencing no dust \citep{1998ApJ...502..177H}. The right plot shows the posterior of $R$, with 68\% credible interval
$R=2.99^{0.18}_{-0.16}$.  We perform the same analysis using the updated template of \citet{2007ApJ...663.1187H} in place of SALT2.
While the $A_V$ distributions using the different models appear similar, there is a significant shift in the $R$ distribution
to $R=3.17^{0.20}_{-0.17}$.


\begin{figure}[htbp] %  figure placement: here, top, bottom, or page
   \centering
   \includegraphics[width=2.8in]{output18/AVs_mode_hist.pdf}
   \includegraphics[width=2.8in]{output18/RVs_hist.pdf}
    \caption{Left: Histogram of the median $A_V$ determined for
each supernova.  The $\gamma^0_V k_0$ medians from our 2-parameter model of \S\ref{model:sec} is also shown.  The means of the measurement
uncertainty over all supernovae are 0.13,   0.16,   0.11, and  0.05 mag respectively, roughly the size of the bin.
Right: Posteriors for $R$.  Results for both the SALT2 supernova model and the updated \citet{2007ApJ...663.1187H} are shown.
   \label{AVRV:fig}}
\end{figure}

The broadband nature of our supernova model does not allow a precise specification for the prediction of dust models,
and our ansatz for incorporating dust is limited systematically in measuring 
the total-to-selective extinction.  We emphasize also that $R^F$ determined in this section are based on  both absolute extinction and color,
whereas the $\gamma$'s (and resulting $\kappa_1$) in \S\ref{model:sec} do not depend on absolute extinction, i.e. the extra model constraint  of the
zero-extinction supernova.


%
%The $R$-free analysis is performed for the dust law of \citet{1989ApJ...345..245C}.  Figure~\ref{CCMF99:fig} shows the difference between per-supernova
%$A_V$ and $R_V$ of  the  \citet{1989ApJ...345..245C} (superscripted with $C$) and \citetalias{1999PASP..111...63F} dust laws.
%The weighted mean of the difference in $A_V$ is $0.06 \pm 0.01$ mag and that of $R_V$ is $-0.26 \pm 0.09$, both significantly
%non-zero.  Insofar that the $A_V$ and $R_V$
%parameters of the two models are designed to represent the same dust observables, we find that the systematic uncertainty in the dust model itself
%is important.
%
%
%\begin{figure}[htbp] %  figure placement: here, top, bottom, or page
%   \centering
%   \includegraphics[width=5.2in]{output18/CCMf99.pdf}
%    \caption{Difference between per-supernova
%$A_V$ and $R_V$ of the  \citet{1989ApJ...345..245C} and \citetalias{1999PASP..111...63F} dust laws.
%   \label{CCMF99:fig}}
%\end{figure}


\color{black}
\section{Summary and Discussion}
\label{discussion:sec}
\subsection{Summary}
To summarize, we model SNe~Ia broadband optical peak magnitudes allowing for correlations with spectral features at peak and
extrinsic color parameters.  Analyzing SNfactory data with the model, we find significant evidence that the above parameters do
affect supernova magnitudes and colors.  This model  does an excellent job in
describing SN~2014J, a supernova with extreme colors that was not used in the training.  
This work represents the first determination of the dust extinction curve outside the Local Group
derived entirely independently of assumptions about the shape of the extinction curve and/or assumptions about the
distribution of $A_V$.  An extended version
of the model retains the features of the first model, and gives the first identification of  a new parameter that affects supernova
colors in a manner that is distinct from  the expectations from dust or the spectral features $EW_{Ca}$, $EW_{Si}$, and $\lambda_{Si}$.

\subsection{Light Curve Shape}
Our models do not explicitly include light-curve shape, which is known to track SN~Ia diversity.
We now examine this in more depth using the intrinsic--extrinsic model of \S\ref{model2:sec}.
The correlation between $EW_{Si}$ and the SALT2 $x_1$ light-curve shape parameter is verified
for our sample.
On the other hand, there is no such correlation seen between $A_{\delta I} = \delta_I D$ and $x_1$, as shown in Figure~\ref{x1:fig}. 
Our magnitude residuals, $\sigma_X$, are comparable to the residuals after SALT2 training
\citep{2010A&A...523A...7G}, showing that our model exhibits little loss in its ability to standardize
magnitudes although $x_1$ is neglected.
There is no apparent correlation
between $x_1$ and the residuals between observed and model-predicted colors, as seen in
Figure~\ref{x1res:fig}. 
We  run an extension of our model to include light-curve shape by adding a new linear term $\zeta x_1$ to the 
intrinsic magnitude; the credible intervals
from this analysis are given in Table~\ref{globalx1:tab}; there are no significant changes with respect to the reference model. 
Light-curve shape is not a strong predictor of supernova magnitude relative to the spectral features: only $\zeta_R/\zeta_V$
has a non-zero value at $>2\sigma$.
The evidence for a non-zero $\delta$ remains, meaning that it is not an artifact of
having neglected light-curve shape.

\begin{figure}[htbp] %  figure placement: here, top, bottom, or page
   \centering
   \includegraphics[width=2.8in]{output23/x1si.pdf}
   \includegraphics[width=2.8in]{output23/x1D.pdf}
    \caption{Left: SALT2 $x_1$ versus $EW_{Si}$.  Right: $x_1$ versus $A_{\delta I}  = \delta_I D$.
   \label{x1:fig}}
\end{figure}


\begin{figure}[htbp] %  figure placement: here, top, bottom, or page
   \centering
   \includegraphics[width=5.2in]{output23/residualx1.pdf}
    \caption{Differences between observed colors and the colors predicted from the analysis, as a function
            of the SALT2 light-curve shape parameter $x_1$.  For reference a dotted line is plotted at zero difference.
   \label{x1res:fig}}
\end{figure}


\begin{table}
\centering
\begin{tabular}{|c|c|c|c|c|c|}
\hline
Parameters & $X=U$ &$B$&$V$&$R$&$I$\\ \hline
$\alpha_X$
&
$0.0043^{+0.0009}_{-0.0010}$
&
$0.0016^{+0.0008}_{-0.0008}$
&
$0.0016^{+0.0007}_{-0.0007}$
&
$0.0015^{+0.0006}_{-0.0005}$
&
$0.0028^{+0.0005}_{-0.0005}$
\\
${\alpha_X/\alpha_V-1}$
&
$   1.7^{+   0.9}_{  -0.4}$
&
$  -0.0^{+   0.1}_{  -0.2}$
&
$ \ldots$
&
$  -0.1^{+   0.1}_{  -0.1}$
&
$   0.7^{+   0.7}_{  -0.3}$
\\
$\beta_X$
&
$ 0.035^{+ 0.005}_{-0.005}$
&
$ 0.026^{+ 0.004}_{-0.004}$
&
$ 0.023^{+ 0.004}_{-0.004}$
&
$ 0.020^{+ 0.003}_{-0.003}$
&
$ 0.017^{+ 0.003}_{-0.003}$
\\
${\beta_X/\beta_V-1}$
&
$  0.54^{+  0.12}_{ -0.10}$
&
$  0.16^{+  0.06}_{ -0.06}$
&
$ \ldots$
&
$ -0.12^{+  0.03}_{ -0.03}$
&
$ -0.26^{+  0.06}_{ -0.06}$
\\
$\eta_X$
&
$0.0007^{+0.0012}_{-0.0012}$
&
$0.0006^{+0.0010}_{-0.0010}$
&
$0.0011^{+0.0008}_{-0.0008}$
&
$0.0010^{+0.0007}_{-0.0007}$
&
$0.0003^{+0.0006}_{-0.0006}$
\\
${\eta_X/\eta_V-1}$
&
$ -0.24^{+  0.47}_{ -0.98}$
&
$ -0.32^{+  0.30}_{ -0.76}$
&
$ \ldots$
&
$ -0.08^{+  0.20}_{ -0.13}$
&
$ -0.66^{+  0.29}_{ -0.58}$
\\
$\zeta_X$
&
$  0.01^{+  0.04}_{ -0.04}$
&
$ -0.01^{+  0.03}_{ -0.03}$
&
$ -0.03^{+  0.03}_{ -0.03}$
&
$ -0.01^{+  0.02}_{ -0.02}$
&
$ -0.03^{+  0.02}_{ -0.02}$
\\
${\zeta_X/\zeta_V-1}$
&
$ -0.91^{+  0.75}_{ -1.78}$
&
$ -0.62^{+  0.46}_{ -1.15}$
&
$ \ldots$
&
$ -0.50^{+  0.21}_{ -0.53}$
&
$ -0.18^{+  0.41}_{ -0.32}$
\\
$\gamma^0_X$
&
$ 62.57^{+  3.71}_{ -3.66}$
&
$ 51.36^{+  3.12}_{ -3.13}$
&
$ 38.26^{+  2.67}_{ -2.69}$
&
$ 29.03^{+  2.36}_{ -2.39}$
&
$ 20.40^{+  2.21}_{ -2.26}$
\\
${\gamma^0_X/\gamma^0_V-1}$
&
$  0.63^{+  0.06}_{ -0.05}$
&
$  0.34^{+  0.03}_{ -0.03}$
&
$ \ldots$
&
$ -0.24^{+  0.01}_{ -0.02}$
&
$ -0.47^{+  0.03}_{ -0.03}$
\\
$\gamma^1_X$
&
$ -9.27^{+  3.50}_{ -3.75}$
&
$-11.08^{+  2.92}_{ -3.18}$
&
$-14.99^{+  2.38}_{ -2.66}$
&
$-14.28^{+  2.05}_{ -2.22}$
&
$-13.27^{+  1.91}_{ -2.09}$
\\
${\gamma^1_X/\gamma^1_V-1}$
&
$ -0.38^{+  0.15}_{ -0.18}$
&
$ -0.26^{+  0.09}_{ -0.11}$
&
$ \ldots$
&
$ -0.05^{+  0.04}_{ -0.04}$
&
$ -0.12^{+  0.09}_{ -0.08}$
\\
$\delta_X$
&
$  1.98^{+  0.52}_{ -0.50}$
&
$ -2.13^{+  0.55}_{ -0.54}$
&
$ -1.84^{+  0.51}_{ -0.52}$
&
$  0.70^{+  0.33}_{ -0.35}$
&
$  1.72^{+  0.75}_{ -0.73}$
\\
${{\delta_X/\delta_I-1}}$
&
$  0.17^{+  0.50}_{ -0.27}$
&
$ -2.23^{+  0.30}_{ -0.64}$
&
$ -2.08^{+  0.25}_{ -0.44}$
&
$ -0.60^{+  0.54}_{ -0.24}$
&
$ \ldots$
\\
$\sigma_X$
&
$ 0.048^{+ 0.012}_{-0.011}$
&
$ 0.027^{+ 0.007}_{-0.007}$
&
$ 0.014^{+ 0.004}_{-0.005}$
&
$ 0.008^{+ 0.006}_{-0.005}$
&
$ 0.041^{+ 0.005}_{-0.004}$
\\
\hline
\end{tabular}
\caption{68\% credible intervals for the Global Fit Parameters including the light-curve shape parameter $x_1$ \label{globalx1:tab}}
\end{table}

\subsection{SiII Line Velocity}

\citet{2009ApJ...699L.139W, 2011ApJ...729...55F} find a connection between $v_{Si}$ and color, and  
\citet{2015MNRAS.447.1247S} show that $v_{Si}$ is an important spectral classifier within the SNfactory data themselves.
The $\lambda_{Si}$ treated in this article varies linearly with $v_{Si}$.
The values of $\eta$ in this article are consistent with zero.  $EW_{Ca}$ and $EW_{Si}$ have a significant effect on color,
and in turn $\lambda_{Si}$ is correlated with $EW_{Ca}$.
Removing from our model the dependence on equivalent widths (eliminating the  $\alpha$ and $\beta$ parameters), we recover
non-zero $\eta$ values at  $\gtrsim 2\sigma$.
\color{red}
The Ca and Si responsible for the equivalent widths and wavelength of our selected spectral features
are expected to be produced together in the thermonuclear burning into intermediate mass elements.
\color{black}
We conclude that $v_{Si}$ is correlated with color, 
but this correlation
is absorbed into the equivalent-width corrections, which give higher likelihood as they give lower residual dispersions $C_c$.

\subsection{Host-Galaxy Mass}
A correlation between Hubble residual and host-galaxy mass
was first noted by \citet{2010ApJ...715..743K,2010MNRAS.406..782S}, a signal confirmed to exist in the SNfactory
sample \citep{2013ApJ...770..108C}.
This host-mass bias could be the result of a parameter that was not accounted for in the inference of SN~Ia absolute magnitude.
\citet{2016arXiv160904470M} find that when introducing intrinsic color scatter as a latent parameter, the null mass-step effect falls on the 95\%-level of their posterior.
To explore whether the Hubble residual may be due to unaccounted color,
we plot in Figure~\ref{childress:fig} our intrinsic parameter  $A_{\delta I}$  versus host mass 
\citep{2016rigault}, and their ideograms for two samples divided by a host mass of  $\log{(M/M_\sun)}=10$
used by  \citet{2013ApJ...770..108C}.
Although by eye it appears that the high-mass hosts tend the have lower $A_{\delta I}$, the difference is
not significant; we calculate  average
%---
$\langle A_{\delta I} \rangle=    -0.0026 \pm    0.0010$,
$\langle A_{\delta I} \rangle=  -0.0037 \pm    0.0010 $ 
for low- and high-mass hosts respectively. 
Using the SNfactory sample,
\citet{2013A&A...560A..66R} find that step in Hubble residuals is better related to local star formation rate, rather than host mass.

\color{black}
\begin{figure}[htbp] %  figure placement: here, top, bottom, or page
   \centering
   \includegraphics[width=2.8in]{output23/rigault3.pdf}
s     \includegraphics[width=2.8in]{output23/rigault2.pdf}
      \caption{Left: Intrinsic parameter $A_{\delta I}=\delta_I D$  versus host mass.  Hosts too faint for a mass measurement
      are assigned a mass of 5.  Overplotted are the mean and 1$\sigma$ values for supernovae with hosts
      less than and greater than  $\log{(M/M_\sun)}=10$.
Right: Ideograms of  $A_{\delta I}$ for supernovae in $\log{(M/M_\sun)}<10$ and $\log{(M/M_\sun)}>10$ hosts. 
   \label{childress:fig}}
\end{figure}


\subsection{Physical Motivation}

We find a third supernova parameter that does not have a monotonic influence on supernova magnitudes as a function
of wavelength but rather has an inflection in the $V$ band.  In addition, red $B-V$ colors imply brighter magnitudes
for this component.  Absorption and re-emission of light
over broad bands are the most obvious candidate for producing this behavior.

\citet{2006ApJ...649..939K} points out that at a temperature of 7000~K, the iron/cobalt gas in the ejecta transition
from doubly and singly ionized states, making the gas phosphorescent and efficient in redistributing energy from the UV/blue to redder
wavelengths.  Qualitatively this would result in the structure for $\gamma_X/\gamma_V-1$ that we measure.
We conjecture that varying $^{56}$Ni mass affects the rise-time to peak $B$ brightness and hence the optical depth
of the reionization front with respect to the core of the ejecta at peak, yielding the wavelength-dependent color behavior
detected in this analysis.


Although the $D$ parameter has ben referred to as ``intrinsic'', do to its placement in the model and the correspondence
of the $k$ parameters with those of extrinsic two-parameter dust models, the analysis cannot quantitatively distinguish
whether its origin is intrinsic or due to a third dust parameter or some other extrinsic effect.

\subsection{Improving SNe~Ia as Standard Candles}
The model and results presented here
carry information on the calibration of absolute magnitude.  The grey parameter $\Delta$ represents Hubble residuals after
application of the model, and its  $0.10$ mag dispersion has contributions from intrinsic dispersion, peculiar velocities, and
measurement uncertainties.
\color{red}
The purpose of this analysis was designed to mine information from the SNfactory data set, and so the results presented here
are specific to it both in terms of its population distribution and potentially noise.  This is the approach taken by previous
exploratory work.
A new analysis that explicitly measures out-of-sample performance is required to robustly determine how
well the addition of third parameter inferred from one data set can be used to measure supernova distances
of an independent set.
\color{black}

\acknowledgments
We thank the STAN team for providing the statistical tool without which this analysis would not have been possible,
and Michael Betancourt specifically for his helpful guidance.  We thank Danny Goldstein for useful discussions.
We thank Dan Birchall for observing assistance, the technical and
scientific staffs of the Palomar Observatory, the High Performance
Wireless Radio Network (HPWREN), and the University of Hawaii 2.2~m
telescope.  We recognize the significant cultural role of Mauna Kea
within the indigenous Hawaiian community, and we appreciate the
opportunity to conduct observations from this revered site.  This
work was supported in part by the Director, Office of Science,
Office of High Energy Physics, of the U.S. Department of Energy
under Contract No. DE-AC02- 05CH11231.  Support in France was
provided by CNRS/IN2P3, CNRS/INSU, and PNC; LPNHE acknowledges
support from LABEX ILP, supported by French state funds managed by
the ANR within the Investissements d'Avenir programme under reference
ANR-11-IDEX-0004-02.  NC is grateful to the LABEX Lyon Institute
of Origins (ANR-10-LABX-0066) of the Universit\'e de Lyon for its
financial support within the program ``Investissements d'Avenir''
(ANR-11-IDEX-0007) of the French government operated by the National
Research Agency (ANR).  Support in Germany was provided by the DFG
through TRR33 ``The Dark Universe;'' and in China from Tsinghua
University 985 grant and NSFC grant No~11173017.  Some results were
obtained using resources and support from the National Energy
Research Scientific Computing Center, supported by the Director,
Office of Science, Office of Advanced Scientific Computing Research,
of the U.S. Department of Energy under Contract No. DE-AC02-05CH11231.
HPWREN is funded by National Science Foundation Grant Number
ANI-0087344, and the University of California, San Diego.


\bibliographystyle{apj}
\bibliography{/Users/akim/Documents/alex}


\end{document} 
